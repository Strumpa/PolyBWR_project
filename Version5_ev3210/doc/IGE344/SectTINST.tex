\subsection{The \moc{TINST:} module}\label{sect:tinst}

\vskip 0.2cm
The \moc{TINST:} module is used to compute the instantaneous burnup for each fuel bundle.
You can also use \moc{TINST:} to refuel your reactor, according to a refueling-scheme. 
The scheme can be either specified with \moc{RESINI:}, or directly in \moc{TINST:}.\\

\noindent
The \moc{TINST:} module specification is:

\begin{DataStructure}{Structure \moc{TINST:}}
$\{$ \dusa{FMAP} \moc{:=} \moc{TINST:} \dusa{FMAP}
$[$ \dusa{POWER} $]~|$ \\~~
\dusa{MICLIB3} \dusa{FMAP} \moc{:=} \moc{TINST:} \dusa{FMAP} \dusa{MICLIB2}
\dusa{MICLIB} $\}$ \\


\moc{::} \dstr{desctinst}
\end{DataStructure}

\noindent where

\begin{ListeDeDescription}{mmmmmmmm}

\item[\dusa{FMAP}] \texttt{character*12} name of a \dds{fmap} object,
that will be updated by the \moc{TINST:} module. The \dusa{FMAP} object
must contain the instantaneous burnups for each fuel bundle and the weight of
each fuel mixture.

\item[\dusa{POWER}] \texttt{character*12} name of a \dds{power} object
containing the channel and bundle powers, previously computed by the
\moc{FLPOW:} module. The channel and bundle powers are used by the
\moc{TINST:} module to compute the new burn-up of each bundle. If bundle-powers are
previously specified with the module \moc{RESINI:}, you can refuel
your core without a \dusa{POWER} object.

\item[\dusa{MICLIB3}] \texttt{character*12} name of a \dds{library} object,
that will be created by the \moc{TINST:} module. This \dds{MICROLIB} contains the fuel properties after refueling when keyword MICRO is used in \dstr{desctinst}.

\item[\dusa{MICLIB2}] \texttt{character*12} name of a \dds{library} object,
that will be read by the \moc{TINST:} module. This must be a fuel-map \dusa{LIBRARY} created either created by the \moc{NCR:} or the \moc{EVO:} module.

\item[\dusa{MICLIB}] \texttt{character*12} name of a \dds{library} object,
that will be read by the \moc{TINST:} module. This \dds{MICROLIB} contains the new fuel properties, that should be used for the refueling.

\item[\dstr{desctinst}] structure describing the input data to the \moc{TINST:} module.

\end{ListeDeDescription}

\clearpage

\vskip 0.2cm
\subsubsection{Input data to the \moc{TINST:} module}\label{sect:strtinst}

\noindent
Note that the input order must be respected. \\

\begin{DataStructure}{Structure \dstr{desctinst}}
$[$ \moc{EDIT} \dusa{iprint} $]$ \\
$[$ \moc{BURN-STEP} \dusa{rburn}  $|$ \moc{TIME} \dusa{rtime} $\{$ \moc{DAY} $|$ \moc{HOUR} $|$ \moc{MINUTE} $|$ \moc{SECOND} $\}$ $]$ \\
$[[$ \moc{REFUEL} $[$ \moc{MICRO} $]$ \moc{CHAN} \dusa{NAMCHA} \dusa{nsh} $]]$ \\
$[[$ \moc{NEWFUEL} $[$ \moc{MICRO} $]$ \moc{CHAN} \dusa{NAMCHA} \dusa{nsh} $\{$ \moc{SOME}
( \dusa{imix}(i), i=1,ABS(\dusa{nsh}) ) $|$ \moc{ALL} \dusa{imix} $\}~]]$ \\
$[[$ \moc{SHUFF} \moc{CHAN} \dusa{NMCHA1} \moc{TO} 
$\{$ \dusa{NMCHA2} $|$ \moc{POOL} $\}$ $]]$ \\
$[$ \moc{PICK}  {\tt >>} \dusa{burnup} {\tt <<} $]$ \\
{\tt ;}
\end{DataStructure}

\noindent where
\begin{ListeDeDescription}{mmmmmmmm}

\item[\moc{EDIT}] keyword used to set \dusa{iprint}.

\item[\dusa{iprint}] integer index used to control the printing on screen:
 = 0 for no print; = 1 for minimum printing (default value); = 2 only the burnup limits
over each channel are printed; = 3 only the axial power-shape values over each channel
are printed; = 4 only the channel refueling rates are printed; for larger values of
\dusa{iprint} everything will be printed.

\item[\moc{BURN-STEP}] keyword used to indicate an increase of core average burn-up.

\item[\dusa{rburn}] keyword used to indicate in MWd/t the average increase
of burn-up in the core.

\item[\moc{TIME}] keyword used to indicate the time of combustion at the power specified
in \dusa{POWER} structure.

\item[\dusa{rtime}] keyword used to set the time combustion value in \moc{DAY} or \moc{HOUR}
or \moc{MINUTE} or \moc{SECOND}.

\item[\moc{DAY}] keyword used to specify that \dusa{rtime} is a number of days.

\item[\moc{HOUR}] keyword used to specify that \dusa{rtime} is a number of hours.

\item[\moc{MINUTE}] keyword used to specify that \dusa{rtime} is a number of minutes.

\item[\moc{SECOND}] keyword used to specify that \dusa{rtime} is a number of seconds.

\item[\moc{REFUEL}] key word to specify a channel refueling.

\item[\moc{MICRO}] keyword used to perform a microscopic refueling. In this case, three libraries have to be provided when \moc{TINST:} is called.

\item[\moc{CHAN}] key word to specify the refueled channel information.

\item[\dusa{NAMCHA}] channel name. In Cartesian geometry, \dusa{NAMCHA} is a character*4 variable defined by \dusa{NXNAME} and
\dusa{NYNAME} and constructed as \\
{\tt WRITE(NAMCHA,'(A1,A3)')} \dusa{NYNAME}(1:1),\dusa{NXNAME}(1:2).

\item[\dusa{nsh}] refueling scheme. The absolute value of \dusa{nsh} is
the number of fuel bundles inserted in the channel \dusa{NAMCHA}.
The sign of \dusa{nsh} define the refueling direction: positive direction
is from the first to the \dusa{nk}-th bundle and negative is from the 
\dusa{nk}-th to the first bundle.

\item[\moc{NEWFUEL}] key word to specify that a channel will be refueled
with a different type of fuel.

\item[\moc{SOME}] key word to specify that the \dusa{nsh} values of
fuel types can be different.

\item[\dusa{imix}(i)] index number of a fuel type with respect to the
values defined in module \moc{NCR:}, \moc{CRE:} or \moc{AFM:}.

\item[\moc{ALL}] key word to specify that the \dusa{nsh} values of
fuel types will be identical to \dusa{imix}.

\item[\moc{SHUFF}] key word to specify that a specified channel will
move into an other one or discharge into the pool.

\item[\moc{CHAN}] key word to specify the moved channel name.

\item[\dusa{NMCHA1}] channel name as defined by \dusa{NXNAME} and
\dusa{NYNAME}. It is constructed as \dusa{NAMCHA}. 

\item[\moc{TO}] key word to specify the bundle destination.

\item[\dusa{NMCHA2}] channel name as defined by \dusa{NXNAME} and
\dusa{NYNAME}. It is constructed as \dusa{NAMCHA}. 

\item[\moc{POOL}] key word to specify that the channel referenced by
\dusa{NMCHA1} is discharged into the pool.

\item[\moc{PICK}]  keyword used to recover the final burnup value (in MW-day/tonne) in a CLE-2000 variable.

\item[\dusa{burnup}] \texttt{character*12} CLE-2000 variable name in which the extracted burnup value will be placed.

\end{ListeDeDescription}
\clearpage
