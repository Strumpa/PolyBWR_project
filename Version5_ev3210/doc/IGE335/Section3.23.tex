\subsection{The {\tt MC:} module}\label{sect:MCData}

This component of the lattice code is dedicated to the Monte-Carlo solution of the transport
equation in multigroup approximation.

\vskip 0.02cm

The calling specifications are:

\begin{DataStructure}{Structure \dstr{MC:}}
\dusa{OUTMC}~$[$~\dusa{TRACK}~$]$~\moc{:=}~\moc{MC:}~$[$~\dusa{OUTMC}~$]$~\dusa{TRACK}~$\{$~\dusa{MICRO}~$|$~\dusa{MACRO}~$\}$~\moc{::}~\dstr{MC\_data} \\
\end{DataStructure}

\noindent where
\begin{ListeDeDescription}{mmmmmmm}

\item[\dusa{OUTMC}] {\tt character*12} name of a {\sc Monte-Carlo} (type {\tt L\_MC}) object open in modification or creation
mode.

\item[\dusa{TRACK}] {\tt character*12} name of a \dusa{NXT:} {\sc tracking} (type {\tt L\_TRACK}) object open in
read-only or modification mode. Object \dusa{TRACK} must be constructed with option \moc{MC} activated (see \Sect{NXTData}). Opening \dusa{TRACK}
in modification mode is useful to add tracking information to be plotted with module \moc{PSP:} (see \Sect{PSPData}).

\item[\dusa{MICRO}] {\tt character*12} name of a {\sc microlib} (type {\tt L\_LIBRARY}) object open in read-only mode. The information on
the embedded macrolib is used.

\item[\dusa{MACRO}] {\tt character*12} name of a {\sc macrolib} (type {\tt L\_MACROLIB}) object open in read-only mode.

\item[\dusa{MC\_data}] input data structure containing specific data (see \Sect{descMC}).

\end{ListeDeDescription}

\subsubsection{Data input for module {\tt MC:}}\label{sect:descMC}

\vskip -0.5cm

\begin{DataStructure}{Structure \dstr{MC\_data}}
$[$~\moc{EDIT} \dusa{iprint}~$]$ \\
\moc{KCODE}~\dusa{nsrck}~\dusa{ikz}~\dusa{kct} \\
$[$~\moc{SEED} \dusa{iseed}~$]~[$~\moc{N2N}~$]$ \\
$[$~\moc{TALLY} \\
\hskip 1.0cm $[$ \moc{MERG} $\{$ \moc{COMP} $|$ \moc{NONE} $|$ \\
\hskip 2.0cm \moc{REGI} (\dusa{iregm}(ii),ii=1,nregio) $|$ \\
\hskip 2.0cm \moc{MIX} $[$ (\dusa{imixm}(ii),ii=1,nbmix) $]~\}$ $]$ \\
\hskip 1.0cm $[$ \moc{COND} $[~\{$  \moc{NONE} $|$ ( \dusa{icond}(ii), ii=1,ngcond) $\}~]~]$\\
\moc{ENDT} $]$ \\
{\tt ;}
\end{DataStructure}

\noindent where
\begin{ListeDeDescription}{mmmmmmmm}

\item[\moc{EDIT}] keyword used to set \dusa{iprint}.

\item[\dusa{iprint}] index used to control the printing in module {\tt MC:}. =0 for no print; =1 for minimum printing (default value);
=100 to add free-path information in object \dusa{TRACK} (must be open in modification mode in that case).

\item[\moc{KCODE}] keyword used to define the power iteration settings.

\item[\dusa{nsrck}] number of neutrons generated per cycle

\item[\dusa{ikz}] number of inactive cycles

\item[\dusa{kct}] number of active cycles

\item[\moc{SEED}] keyword used to set the initial seed integer for the random number generator. By default, the seed integer is set from
the processor clock.

\item[\dusa{iseed}] initial seed integer

\item[\moc{N2N}] keyword used to enable an explicit treatment of $(n,2n)$ reactions. In this case, {\tt N2N} cross sections are
expected to be available in the macrolib. By default, $(n,2n)$ reactions are taken into account implicitly by the correction on scattering
cross sections.

\item[\moc{TALLY}] keyword used to define a tally (macrolib and effective multiplication factor). Using "\moc{TALLY~ENDT}" construct
permits to obtain a virtual collision estimation of the effective multiplication factor {\sl without} estimation of the macrolib
information.

\item[\moc{NONE}] keyword to deactivate the homogeneization or the condensation. 

\item[\moc{MERG}] keyword to specify that the neutron flux is to be
homogenized over specified regions or mixtures. 

\item[\moc{REGI}] keyword to specify that the homogenization of the neutron
flux will take place over the following regions. Here nregio$\le$\dusa{maxreg}
with \dusa{maxreg} the maximum number of regions for which solutions were
obtained.

\item[\dusa{iregm}] array of homogenized region numbers to which are
associated the old regions. In the editing routines a value of \dusa{iregm}=0
allows the corresponding region to be neglected. 

\item[\moc{MIX}] keyword to specify that the homogenization of the neutron
flux will take place over the following mixtures. Here
we must have nbmix$\le$\dusa{maxmix} where \dusa{maxmix} is the maximum number
of mixtures in the macroscopic cross section library.  

\item[\dusa{imixm}] array of homogenized region numbers to which are
associated the material mixtures. In the editing routines a value of
\dusa{imixm}=0 allows the corresponding isotopic mixtures to be neglected. For a mixture in this
library which is not used in the geometry one should insert a value of 0 for the
new region number associated with this mixture. By default, if \moc{MIX} is set and
\dusa{imixm} is not set, \dusa{imixm(ii)}$=$\dusa{ii} is assumed.

\item[\moc{COMP}] keyword to specify that the a complete homogenization is to
take place.

\item[\moc{COND}] keyword to specify that a group condensation of the flux is
to be performed.

\item[\dusa{icond}] array of increasing energy group limits that will be associated with
each of the ngcond condensed groups. The final value of
\dusa{icond} will automatically be set to \dusa{ngroup} while the values of 
\dusa{icond}$>$\dusa{ngroup} will be droped from the condensation. 
We must have ngcond$\le$\dusa{ngroup}. By default, if \moc{COND} is set and \dusa{icond}
is not set, all energy groups are condensed together.

\item[\moc{ENDT}] keyword used to terminate the definition of a tally.

\end{ListeDeDescription}

\eject
