\subsection{The {\tt G2S:} module}\label{sect:G2SData}

The module {\tt G2S:} is used to create the SALOMON--formatted surfacic elements corresponding
to a gigogne geometry. It can also be used to plot a SALOMON--formatted file or as a conversion tool to transform an ALAMOS--formatted file into
a SALOMON--formatted file. The general format of the input data for the {\tt G2S:} module is the following:
\begin{DataStructure}{Structure \dstr{G2S:}}
$[$ \dusa{SURFIL} $]~[$ \dusa{PSFIL} $]$ \moc{:=} \moc{G2S:}~ $\{$ \dusa{SURFIL\_IN} $[$ \dusa{ZAFIL\_IN} $]~|$ \dusa{GEONAM} $\}$ ~\moc{::}~\dstr{G2S\_data} \\
\end{DataStructure}

\noindent where
\begin{ListeDeDescription}{mmmmmm}

\item[\dusa{SURFIL}] \texttt{character*12} name of the output SALOMON--formatted sequential {\sc ascii}
file used to store the surfacic elements of the geometry.

\item[\dusa{PSFIL}] \texttt{character*12} name of the sequential {\sc ascii}
file used to store a postscript representation of the geometry corresponding to \dusa{SURFIL} or \dusa{GEONAM}.

\item[\dusa{GEONAM}] {\tt character*12} name of the {\sl read-only} \dds{geometry} data
structure. This structure may be build using the operator {\tt GEO:} (see \Sect{GEOData}).

\item[\dusa{SURFIL\_IN}] \texttt{character*12} name of the input SALOMON-- or ALAMOS--formatted sequential {\sc ascii}
file used to store the surfacic elements of the geometry.

\item[\dusa{ZAFIL\_IN}] \texttt{character*12} name of the input sequential {\sc ascii} file containing {\sl PropertyMap}
information associated with the ALAMOS geometry. This file generally has a {\tt .za} extension. This information is used to
set node-ordered mixture indices. By default, node-ordered mixture indices are recovered from the {\sl Mailles} record present
in the ALAMOS surfacic file.

\item[\dusa{G2S\_data}] input data structure containing specific data (see \Sect{descG2S}).
\end{ListeDeDescription}

\clearpage

\subsubsection{Data input for module {\tt G2S:}}\label{sect:descG2S}

\vskip -0.5cm

\begin{DataStructure}{Structure \dstr{G2S\_data}}
$[$~\moc{EDIT} \dusa{iprint}~$]$ \\
$[$~\moc{ALAMOS} \dusa{typgeo}~$]$ \\
$[~\{$~\moc{DRAWNOD} $|$ \moc{DRAWMIX} $\}~]~[$ \moc{ZOOMX} \dusa{facx1} \dusa{facx2} $]~[$ \moc{ZOOMY} \dusa{facy1} \dusa{facy2} $]$ \\
\moc{;}
\end{DataStructure}

\vskip -0.3cm

\noindent where
\begin{ListeDeDescription}{mmmmmmmm}

\item[\moc{EDIT}] keyword used to set \dusa{iprint}.

\item[\dusa{iprint}] index used to control the printing in module {\tt G2S:}. =0 for no print; =1 for minimum printing (default value).

\item[\moc{ALAMOS}] keyword to use {\tt G2S:} as a conversion tool to transform an ALAMOS--formatted file into
a SALOMON-formatted file.

\item[\dusa{typgeo}] type of Alamos geometry. A negative value is used for isotropic reflection (white boundary condition) with unfolding in {\tt SALT:} module.
Otherwise, a specular boundary condition is used without unfolding.
\begin{displaymath}
\negthinspace \textsl{typgeo} = \left\{
\begin{array}{ll}
0 & \textrm{Isotropic reflection (white boundary condition) without perimeter}\\
& \textrm{information}\\
\pm 5 & \textrm{Cartesian rectangular geometry with translation}\\
\pm 6 & \textrm{Cartesian rectangular geometry with specular reflection on each}\\
& \textrm{side}\\
\pm 7 & \textrm{Cartesian eight-of-square geometry with specular reflection on each}\\
& \textrm{side}\\
\pm 8 & \textrm{Hexagonal SA60 equilateral triangle geometry with specular reflec-}\\
& \textrm{tion on each side}\\
\pm 9 & \textrm{Full hexagonal geometry with translation}\\
\pm 10 & \textrm{Hexagonal RA60 equilateral triangle geometry with } 60^\circ \textrm{ rotation}\\
& \textrm{and translation}\\
\pm 11 & \textrm{Hexagonal R120 lozenge geometry with } 120^\circ \textrm{ rotation and transla-}\\
& \textrm{tion.}
\end{array} \right.
\end{displaymath}

\item[\moc{DRAWNOD}] keyword used to print the region indices on the LHS postscript plot \dusa{PSFIL}. By default, no indices are printed.

\item[\moc{DRAWMIX}] keyword used to print the material mixture indices on the LHS postscript plot \dusa{PSFIL}. By default, no indices are printed.

\item[\moc{ZOOMX}] keyword used to plot a fraction of the $X$--domain. By default, all the $X$--domain is plotted.

\item[\dusa{facx1}] left factor set in interval $0.0 \le$ \dusa{facx1} $< 1.0$ with 0.0 corresponding to the left boundary and 1.0 corresponding to the right boundary.

\item[\dusa{facx2}] right factor set in interval \dusa{facx1} $<$ \dusa{facx2} $\le 1.0$.

\item[\moc{ZOOMY}] keyword used to plot a fraction of the $Y$--domain. By default, all the $Y$--domain is plotted.

\item[\dusa{facy1}] lower factor set in interval  $0.0 \le$ \dusa{facy1} $< 1.0$ with 0.0 corresponding to the lower boundary and 1.0 corresponding to the upper boundary.

\item[\dusa{facy2}] upper factor set in interval \dusa{facy1} $<$ \dusa{facy2} $\le 1.0$.

\end{ListeDeDescription}


\clearpage
