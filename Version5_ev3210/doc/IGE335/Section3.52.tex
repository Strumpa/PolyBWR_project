\subsection{The {\tt VDG:} module}\label{sect:VDGData}

The {\tt VDG:} module performs a comparison of an approximate self-shielding method with the Autosecol method.
This module is useful to obtain accuracy results for the Van Der Gucht benchmarks.\cite{vdg}
The calling specifications are:

\begin{DataStructure}{Structure \dstr{VDG:}}
\moc{VDG:} \dusa{MICLIB1}  \dusa{MICLIB2} \moc{::} \dstr{descvdg}
\end{DataStructure}

\vskip -0.5cm

\noindent where

\begin{ListeDeDescription}{mmmmmmmm}

\item[\dusa{MICLIB1}] {\tt character*12} name of the self-shielded \dds{microlib} produced by the
Autosecol method (module \moc{AUTO:}).

\item[\dusa{MICLIB2}] {\tt character*12} name of the self-shielded \dds{microlib} produced by an
approximate self-shielding method.

\item[\dstr{descvdg}] structure describing the \moc{VDG:} module options.

\end{ListeDeDescription}

\
\subsubsection{Data input for module {\tt VDG:}}\label{sect:descvdg}

\begin{DataStructure}{Structure \dstr{descvdg}}
$[$ \moc{EDIT} \dusa{iprint} $]$ \\
$[$ \moc{GRMI} \dusa{igrp1} $]~[$ \moc{GRMA} \dusa{igrp2} $]$ \\
$[$ \moc{PICK} $\{$ \moc{MAXV} $|$ \moc{AVER} $|$ \moc{INTG} $\}$ {\tt >>} \dusa{error} {\tt <<} $]$ \\
{\tt ;}
\end{DataStructure}

\vskip -0.5cm

\noindent where

\begin{ListeDeDescription}{mmmmmmmm}

\item[\moc{EDIT}] keyword used to modify the print level \dusa{iprint}.

\item[\dusa{iprint}] index used to control the printing of this module. The
amount of output produced by this module will vary substantially
depending on the print level specified. 

\item[\moc{GRMI}] keyword used to set the index of the first group in the microlib. By default, \moc{GRMI} $=$ 1.

\item[\dusa{igrp1}] index of the first resonant group.

\item[\moc{GRMA}] keyword used to set the index of the last group in the microlib. By default, \moc{GRMA} $=$ 9999999.

\item[\dusa{igrp2}] index of the last resonant group.

\item[\moc{PICK}]  keyword used to recover a relative discrepancy value for the absorption rates in
each DRAGON resonant energy group.

\item[\moc{MAXV}] keyword to select the maximum relative discrepancy.

\item[\moc{AVER}] keyword to select the averaged relative discrepancy.

\item[\moc{INTG}] keyword to select the integrated relative discrepancy.

\item[\dusa{error}] \texttt{character*12} CLE-2000 variable name in which the extracted percent
discrepancy value will be placed.

\end{ListeDeDescription}
\eject
