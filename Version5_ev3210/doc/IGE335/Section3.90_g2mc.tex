\subsection{The {\tt G2MC:} module}\label{sect:G2MCData}

The module {\tt G2MC:} is used to compute the SERPENT--, TRIPOLI4--, or MCNP--formatted surfacic elements corresponding
to a SALOMON--formatted file or corresponding to a gigogne geometry. The general format of the input data for the
{\tt G2MC:} module is the following:
\begin{DataStructure}{Structure \dstr{G2MC:}}
\dusa{MCFIL} $[$ \dusa{PSFIL} $]$ \moc{:=} \moc{G2MC:} $\{$ \dusa{SURFIL} $|$ \dusa{GEONAM} $\}$ ~\moc{::}~\dstr{G2MC\_data} \\
\end{DataStructure}

\noindent where
\begin{ListeDeDescription}{mmmmmm}

\item[\dusa{MCFIL}] \texttt{character*12} name of the SERPENT--, TRIPOLI4-- or MCNP--formatted sequential {\sc ascii}
file used to store the surfacic elements of the geometry. A SERPENT file is
produced if the file name has extension {\tt ".sp"}. A TRIPOLI4 file is
produced if the file name has extension {\tt ".tp"}. Otherwise, a MCNP file is
produced. This file is to be included in the complete dataset of a Monte Carlo code.

\item[\dusa{PSFIL}] \texttt{character*12} name of the sequential {\sc ascii}
file used to store a postscript representation of the geometry corresponding to \dusa{GEONAM}.

\item[\dusa{SURFIL}] \texttt{character*12} name of the {\sl read-only} SALOMON--formatted sequential {\sc ascii}
file used to store the surfacic elements of the geometry.

\item[\dusa{GEONAM}] {\tt character*12} name of the {\sl read-only} \dds{geometry} data
structure. This structure may be build using the operator {\tt GEO:} (see \Sect{GEOData}).

\item[\dusa{G2MC\_data}] input data structure containing specific data (see \Sect{descG2MC}).

\end{ListeDeDescription}

\subsubsection{Data input for module {\tt G2MC:}}\label{sect:descG2MC}

\vskip -0.5cm

\begin{DataStructure}{Structure \dstr{G2MC\_data}}
$[$~\moc{EDIT} \dusa{iprint}~$]$ \\
$[~\{$~\moc{DRAWNOD} $|$ \moc{DRAWMIX} $\}~]~[$ \moc{ZOOMX} \dusa{facx1} \dusa{facx2} $]~[$ \moc{ZOOMY} \dusa{facy1} \dusa{facy2} $]$ \\
\moc{;}
\end{DataStructure}

\noindent where
\begin{ListeDeDescription}{mmmmmmmm}

\item[\moc{EDIT}] keyword used to set \dusa{iprint}.

\item[\dusa{iprint}] index used to control the printing in module {\tt G2MC:}. =0 for no print; =1 for minimum printing (default value).

\item[\moc{DRAWNOD}] keyword used to print the region indices on the LHS postscript plot \dusa{PSFIL}. By default, no indices are printed.

\item[\moc{DRAWMIX}] keyword used to print the material mixture indices on the LHS postscript plot \dusa{PSFIL}. By default, no indices are printed.

\item[\moc{ZOOMX}] keyword used to plot a fraction of the $X$--domain. By default, all the $X$--domain is plotted.

\item[\dusa{facx1}] left factor set in interval $0.0 \le$ \dusa{facx1} $< 1.0$ with 0.0 corresponding to the left boundary and 1.0 corresponding to the right boundary.

\item[\dusa{facx2}] right factor set in interval \dusa{facx1} $<$ \dusa{facx2} $\le 1.0$.

\item[\moc{ZOOMY}] keyword used to plot a fraction of the $Y$--domain. By default, all the $Y$--domain is plotted.

\item[\dusa{facy1}] lower factor set in interval  $0.0 \le$ \dusa{facy1} $< 1.0$ with 0.0 corresponding to the lower boundary and 1.0 corresponding to the upper boundary.

\item[\dusa{facy2}] upper factor set in interval \dusa{facy1} $<$ \dusa{facy2} $\le 1.0$.

\end{ListeDeDescription}

\clearpage
