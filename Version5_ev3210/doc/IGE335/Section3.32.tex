\subsection{The {\tt PSOUR:} module}\label{sect:PSOURData}

This module is used to set transition sources in a multi-particle coupled transport problem or boundary sources in Cartesian geometry. The {\tt PSOUR:} module
is currently limited to a SN discretization.

\vskip 0.08cm

The calling specifications are:

\begin{DataStructure}{Structure \dstr{PSOUR:}}
\dusa{SOURCE}~\moc{:=}~\moc{PSOUR:}~$[$~\dusa{SOURCE}~$]~\{$~\dusa{MICRO}~$|$~\dusa{MACRO}~$\}$~\dusa{TRACK}~$[$~\dusa{GEONAM}~$]~[[$~\dusa{FLUX}~$]]$ \\
\hspace*{0.5cm}  $[[$~\dusa{SOURCE2}~$]]$ \moc{::}~\dstr{PSOUR\_data} \\
\end{DataStructure}

\noindent where
\begin{ListeDeDescription}{mmmmmmm}

\item[\dusa{SOURCE}] {\tt character*12} name of a {\sc fixed source} (type {\tt L\_SOURCE}) object open in creation or modification
mode. This object contains a unique direct or adjoint fixed source taking into account scattering transitions from one or many companion particles and/or boundary sources.

\item[\dusa{MICRO}] {\tt character*12} name of a reference {\sc microlib} (type {\tt L\_LIBRARY}) object open in read-only mode. The information on
the embedded macrolib is used.

\item[\dusa{MACRO}] {\tt character*12} name of a reference {\sc macrolib} (type {\tt L\_MACROLIB}) object open in read-only mode.

\item[\dusa{TRACK}] {\tt character*12} name of a reference {\sc tracking} (type {\tt L\_TRACK}) object, corresponding to {\tt L\_SOURCE} object, open in read-only mode.

\item[\dusa{GEONAM}] {\tt character*12} name of a reference {\sc geometry} (type {\tt L\_GEOM}) object, corresponding to {\tt L\_SOURCE} object, open in read-only mode. This
object is required if and only if one of keywords \moc{ISO} or \moc{MONO} is set.

\item[\dusa{FLUX}] {\tt character*12} name of a {\sc flux} (type {\tt L\_FLUX}) object corresponding to a companion particle open in read-only mode and used to
compute transition sources. The number of {\sc flux} objects on the RHS is equal to the number of companion particles contributing to the fixed source.

\item[\dusa{SOURCE2}] {\tt character*12} name of a {\sc fixed source} (type {\tt L\_SOURCE}) object open in read-only mode. The fixed source in \dusa{SOURCE2} is
added to \dusa{SOURCE}.

\item[\dusa{PSOUR\_data}] input data structure containing specific data (see \Sect{descPSOUR}).

\end{ListeDeDescription}

\subsubsection{Data input for module {\tt PSOUR:}}\label{sect:descPSOUR}

\vskip -0.5cm

\begin{DataStructure}{Structure \dstr{PSOUR\_data}}
$[$~\moc{EDIT} \dusa{iprint}~$]$ \\
$\{$ \\
\hspace*{0.2cm}  $[[$~\moc{PARTICLE} \dusa{htype}~$]]$ \\
$|$ \\
\hspace*{0.2cm}  \moc{ISO} \dusa{nsource}\\
\hspace*{0.2cm}  \moc{INTG} $\{$\dusa{ig}($i$) \dusa{int}($i$), $i$ $|$ \dusa{int}($i$), $i$=1,\dusa{ng}$\}$ \\
\hspace*{0.2cm}  \moc{XLIM} (\dusa{xmin}($i$) \dusa{xmax}($i$), $i$=1,\dusa{nsource}) \\
\hspace*{0.2cm}  $[$\moc{YLIM} (\dusa{ymin}($i$) \dusa{ymax}($i$), $i$=1,\dusa{nsource})$]$ \\
\hspace*{0.2cm}  $[$\moc{ZLIM} (\dusa{zmin}($i$) \dusa{zmax}($i$), $i$=1,\dusa{nsource})$]$ \\
$|$ \\
\hspace*{0.2cm}  \moc{MONO} \dusa{nsource} \\
\hspace*{0.2cm}  \moc{INTG} $\{$\dusa{ig}($i$) \dusa{int}($i$), $i$ $|$ \dusa{int}($i$), $i$=1,\dusa{ng}$\}$ \\
\hspace*{0.2cm}  $\{$ \moc{X-} $|$ \moc{X+} $|$ \moc{Y-} $|$ \moc{Y+} $|$ \moc{Z-} $|$ \moc{Z+} $\}$ \\
\hspace*{0.2cm}  \moc{DIR} $\{$\dusa{mu} \dusa{eta} \dusa{xi} $|$ \dusa{polar} \dusa{azimutal}$\}$ \\
\hspace*{0.2cm}  $[$\moc{XLIM} (\dusa{xmin}($i$) \dusa{xmax}($i$), $i$=1,\dusa{nsource})$]$ \\
\hspace*{0.2cm}  $[$\moc{YLIM} (\dusa{ymin}($i$) \dusa{ymax}($i$), $i$=1,\dusa{nsource})$]$ \\
\hspace*{0.2cm}  $[$\moc{ZLIM} (\dusa{zmin}($i$) \dusa{zmax}($i$), $i$=1,\dusa{nsource})$]$ \\
$\}$\\
{\tt ;}
\end{DataStructure}

\noindent where
\begin{ListeDeDescription}{mmmmmmmm}

\item[\moc{EDIT}] keyword used to set \dusa{iprint}.

\item[\dusa{iprint}] index used to control the printing in module {\tt PSOUR:}. =0 for no print; =1 for minimum printing (default value).

\item[\moc{PARTICLE}] keyword used to specify the transition type recovered from the {\sc macrolib} (primary state of the transition). This keyword is repeated for each type of companion particles, in the same order as the \dusa{FLUX} objects on the RHS.

\item[\dusa{htype}] character*1 name of the companion particle. Usual names are {\tt N}: neutrons, {\tt G}: photons, {\tt B}: electrons,
{\tt C}: positrons and {\tt P}: protons.

\item[\moc{ISO}] keyword used to define an isotropic volumetric source. 

\item[\moc{MONO}] keyword used to define a monodirectional boundary source.

\item[\dusa{nsource}] number of sources of the choosen type.

\item[\moc{INTG}] keyword to specify the energy spectrum of the source.

\item[\dusa{ig}] index of the energy group in which the source is defined.

\item[\dusa{int}] source intensity. With \moc{ISO}, it is defined as the number of particules per volume unit per time unit. With \moc{MONO}, it is defined as the number of particles per surface unit per time unit.

\item[\moc{XLIM}] keywords to specify the \dusa{nsource} source volume (with \moc{ISO}) or surface dimension (with \moc{MONO}) along the $X$-axis.

\item[\moc{YLIM}] keywords to specify the \dusa{nsource} source volume (with \moc{ISO}) or surface dimension (with \moc{MONO}) along the $Y$-axis.

\item[\moc{ZLIM}] keywords to specify the \dusa{nsource} source volume (with \moc{ISO}) or surface dimension (with \moc{MONO}) along the $Z$-axis.  

\item[\dusa{xmin/xmax}] boundaries for each of the \dusa{nsource} volume or surface sources along respectively the $X$-axis.

\item[\dusa{ymin/ymax}] boundaries for each of the \dusa{nsource} volume or surface sources along respectively the $Y$-axis.

\item[\dusa{zmin/zmax}] boundaries for each of the \dusa{nsource} volume or surface sources along respectively the $Z$-axis.

\item[\moc{X-/X+}] keyword to specify that the \moc{MONO}-type source is located on the negative or positive $X$ surface of the Cartesian geometry.

\item[\moc{Y-/Y+}] keyword to specify that the \moc{MONO}-type source is located on the negative or positive $Y$ surface of the Cartesian geometry.

\item[\moc{Z-/Z+}] keyword to specify that the \moc{MONO}-type source is located on the negative or positive $Z$ surface of the Cartesian geometry.  

\item[\moc{DIR}] keyword to specify the orientation of the \moc{MONO}-type source.

\item[\dusa{mu/eta/xi}] direction cosines along respectively the $X$-, $Y$- and $Z$-axis.

\item[\dusa{polar/azimutal}] polar and azimutal angle in degree, with the $X$-axis as the principal axis.

\end{ListeDeDescription}

\eject
