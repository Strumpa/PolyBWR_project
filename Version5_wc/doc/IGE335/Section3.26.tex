\subsection{The {\tt MPO:} module}\label{sect:MPOData}

This component of the lattice code is dedicated to the constitution of the
reactor database in MPO format, similar to the file produced by APOLLO3.\cite{Apollo3}
The MPO file intended to store {\sl all} the nuclear data, produced in
the lattice code, that is useful
in reactor calculations including fuel management and space-time kinetics.
Multigroup lattice calculations are too expensive to be executed dynamically
from the driver of the global reactor calculation. A more feasible
approach is to create a reactor database where a finite number of lattice
calculation results are tabulated against selected {\sl global parameters}
chosen so as to represent expected operating conditions of the reactor. The
\moc{MPO:} operator is used to create and construct a {\sc MPO} file.
The MPO file is written in {\sc hdf5} format, allowing full portability and hierarchical
data organization. It can be edited and modified using the HDFView tool.

\vskip 0.1cm

Each elementary calculation is characterized by a tuple of {\sl global parameters}.
These global parameters are of different types, depending on the nature of the
study under consideration: type of assembly, power, temperature in a mixture,
concentration of an isotope, time, burnup or exposure rate in a depletion calculation,
etc. Each step of a depletion calculation represents an elementary calculation.
The {\sc MPO} file is often presented as a {\sl multi-parameter reactor database}.

\vskip 0.1cm

For each elementary calculation, the results are recovered from the output of the
\moc{EDI:} operator and stored in a set of {\sl homogenized mixture}
directories. The \moc{EDI:} operator is responsible for performing condensation
in energy and homogenization in space of the macroscopic and microscopic cross
sections. All the elementary calculations gathered in a single {\sc mpo} file are
characterized by a single output geometry and a unique output energy-group
structure. The {\sc mpo} file may contain many geometry/energy-group combinations.

\vskip 0.1cm

In each homogenized mixture directory, the \moc{MPO:} operator recover
cross sections for a number of {\sl particularized isotopes} and {\sl macroscopic
residual sets}, a collection of isotopic cross sections weighted by isotopic number densities.
Cross sections for particularized isotopes and macroscopic sets are recovered for
{\sl selected reactions}. Other information is also recovered: multigroup neutron
fluxes, isotopic number densities, fission spectrum and a set
of {\sl local variables}. The local variables are values that characterize each
homogenized mixture: local power, burnup, exposure rate, etc. Some local variables
are arrays of values (eg: SPH equivalence factors). Discontinuity factors and equivalent albedos
are written in group {\tt flux}, as described in Sect.~\ref{sect:adf}. Finally, note that cross section
information written on the {\sc mpo} file is {\sl not} transport corrected and {\sl not}
SPH corrected.

\vskip 0.1cm

A different specification of the \moc{MPO:} function call is used for
creation and construction of the {\sc mpo} file.
\begin{itemize}
\item The first specification is used to initialize the {\sc mpo} data structure
as a function of the \dds{microlib} used in the reference calculation. Optionnally,
the homogenized geometry is also provided. The initialization call is also used to
set the choice of global parameters, local variables, particularized isotopes,
macroscopic sets and selected reactions.
\item A modification call to the \moc{MPO:} function is performed after each
elementary calculation in order to recover output information processed by \moc{EDI:}
(condensed and homogenized cross sections) and \moc{EVO:} (burnup dependant values).
Global parameters and local variables can optionnally be recovered from \dds{microlib}
objects. The \moc{EDI:} calculation is generally performed with option {\tt MICR ALL}.
\end{itemize}

The calling specifications are:

\vskip -0.5cm

\begin{DataStructure}{Structure \dstr{MPO:}}
$\{$~\dusa{MPONAM} \moc{:=} \moc{MPO:} $[$ \dusa{MPONAM} $]~[$~\dusa{HMIC} $]$ \moc{::} \dstr{mpo\_data1} \\
~~$|$~\dusa{MPONAM} \moc{:=} \moc{MPO:} \dusa{MPONAM}~\dusa{EDINAM}~$[$ \dusa{BRNNAM} $]~[$ \dusa{HMIC1}~$[$~\dusa{HMIC2} $]~]$ \moc{::} \dstr{mpo\_data2} \\
~~$|$~\dusa{MPONAM} \moc{:=} \moc{MPO:} \dusa{MPONAM} $[[$ \dusa{MPORHS} $]]$ \moc{::} dstr{mpo\_data3} $\}$ \\
\end{DataStructure}

\noindent where
\begin{ListeDeDescription}{mmmmmmm}

\item[\dusa{MPONAM}] {\tt character*12} name of the {\sc lcm} object containing the
{\sl master} {\sc mpo} data structure.

\item[\dusa{HMIC}] {\tt character*12} name of the reference \dds{microlib} (type {\tt
L\_LIBRARY}) containing the microscopic cross sections.

\item[\dusa{EDINAM}] {\tt character*12} name of the {\sc lcm} object (type {\tt
L\_EDIT}) containing the {\sc edition} data structure corresponding to an elementary
calculation. The {\sc edition} data produced by the last call to the {\tt EDI:} module
is used.

\item[\dusa{BRNNAM}] {\tt character*12} name of the {\sc lcm} object (type {\tt
L\_BURNUP}) containing the {\sc burnup} data structure. This object is compulsory if one
of the following parameters is used: \moc{IRRA}, \moc{FLUB} and/or \moc{TIME}.

\item[\dusa{HMIC1}] {\tt character*12} name of a \dds{microlib} (type {\tt
L\_LIBRARY}) containing global parameter information.

\item[\dusa{HMIC2}] {\tt character*12} name of a \dds{microlib} (type {\tt
L\_LIBRARY}) containing global parameter information.

\item[\dusa{MPORHS}] {\tt character*12} name of the {\sl read-only} {\sc mpo} data structure. This
data structure is concatenated to \dusa{MPONAM} using the \dusa{mpo\_data3} data structure,
as presented in \Sect{descmpo3}. \dusa{MPORHS} must be defined with the same number of energy
groups and the same number of homogeneous regions as \dusa{MPONAM}. Moreover, all the
global and local parameters of \dusa{MPORHS} must be defined in \dusa{MPONAM}. \dusa{MPONAM}
may be defined with {\sl global} parameters not defined in \dusa{MPORHS}.

\item[\dusa{mpo\_data1}] input data structure containing initialization information (see \Sect{descmpo1}).

\item[\dusa{mpo\_data2}] input data structure containing information related to the recovery of an
elementary calculation (see \Sect{descmpo2}).

\item[\dusa{mpo\_data3}] input data structure containing information related to the catenation of one or many
{\sl read-only} {\sc mpo} file(s) (see \Sect{descmpo3}).

\end{ListeDeDescription}

\newpage

\subsubsection{Initialization data input for module {\tt MPO:}}\label{sect:descmpo1}

\begin{DataStructure}{Structure \dstr{mpo\_data1}}
$[$~\moc{EDIT} \dusa{iprint}~$]$ \\
$[$~\moc{COMM}~\dusa{comment}~$]$ \\
$[[$~\moc{PARA}~\dusa{parkey} \\
~~~\{~\moc{TEMP}~\dusa{micnam}~\dusa{imix}~$|$~\moc{CONC}~\dusa{isonam1}~\dusa{micnam}~\dusa{imix}~$|$~\moc{IRRA}~$|$~\moc{FLUB}~$|$ \\
~~~~~~\moc{PUIS}~$|$~\moc{MASL}~$|$~\moc{FLUX}~$|$~\moc{TIME}~$|$~\moc{VALU}~\{~\moc{REAL}~$|$~\moc{CHAR}~$|$~\moc{INTE}~\}~\} \\
$]]$ \\
$[[$~\moc{LOCA}~\dusa{parkey} \\
~~~\{~\moc{TEMP}~$|$~\moc{CONC}~\dusa{isonam2}~$|$~\moc{IRRA}~$|$~\moc{FLUB}~$|$~~\moc{FLUG}~$|$~\moc{PUIS}~$|$~\moc{MASL}~$|$~\moc{FLUX}~$|$~\moc{EQUI}~\} \\
$]]$ \\
$[$~\moc{ISOT}~\{~\moc{TOUT}~$|$ \moc{MILI}~\dusa{imil}~$|~[$~\moc{FISS}~$]~[$~\moc{PF}~$]~[$~(\dusa{HNAISO}(i),~i=1,$N_{\rm iso}$) $]$~\}~$]$ \\
$[$~\moc{REAC}~(\dusa{HNAREA}(i),~i=1,$N_{\rm reac}$) $]$ \\
{\tt ;}
\end{DataStructure}

\goodbreak
\noindent where
\begin{ListeDeDescription}{mmmmmmmm}

\item[\moc{EDIT}] keyword used to set \dusa{iprint}.

\item[\dusa{iprint}] index used to control the printing in module {\tt
MPO:}. =0 for no print; =1 for minimum printing (default value).

\item[\moc{COMM}] keyword used to input a general comment for the {\sc mpo} file.

\item[\dusa{comment}] {\tt character*132} user-defined comment.

\item[\moc{PARA}] keyword used to define a single global parameter.

\item[\moc{LOCA}] keyword used to define a single local variable (a local variable
may be a single value or an array of values).

\item[\dusa{parkey}] {\tt character*24} user-defined keyword associated to a global
parameter or local variable.

\item[\dusa{micnam}] {\tt character*12} name of the \dds{microlib} (type {\tt
L\_LIBRARY}) associated to a global parameter. The corresponding \dds{microlib} will be required on
RHS of the \moc{MPO:} call described in Sect.~\ref{sect:descmpo2}.

\item[\dusa{imix}] index of the mixture associated to a global parameter. This mixture is
located in \dds{microlib} named \dusa{micnam}.

\item[\dusa{isonam1}] {\tt character*8} alias name of the isotope associated to a global
parameter. This isotope is located in \dds{microlib} data structure named \dusa{micnam}.

\item[\dusa{isonam2}] {\tt character*8} alias name of the isotope associated to a local
variable. This isotope is located in the \dds{microlib} directory of the {\sc edition}
data structure named \dusa{EDINAM}.

\item[\moc{TEMP}] keyword used to define a temperature (in Kelvin) as global parameter or
local variable.

\item[\moc{CONC}] keyword used to define a number density as global parameter or
local variable.

\item[\moc{IRRA}] keyword used to define a burnup (in MWday/Tonne) as global
parameter or local variable.

\item[\moc{FLUB}] keyword used to define a {\sl fuel-only} exposure rate (in n/kb) as global
parameter or local variable. The exposure rate is recovered from the \dusa{BRNNAM}
LCM object.

\item[\moc{FLUG}] keyword used to define an exposure rate in global homogenized mixtures (in n/kb) as
local variable. The exposure rate is recovered from the \dusa{BRNNAM}
LCM object.

\item[\moc{PUIS}] keyword used to define the power as global parameter or
local variable.

\item[\moc{MASL}] keyword used to define the mass density of heavy isotopes as
global parameter or local variable.

\item[\moc{FLUX}] keyword used to define the volume-averaged, energy-integrated flux as
global parameter or local variable.

\item[\moc{TIME}] keyword used to define the time (in seconds) as global parameter.

\item[\moc{EQUI}] keyword used to define the SPH equivalence factors as
local variable. A set of SPH factors can be defined as local
variables. Note that the cross sections and fluxes stored in the {\sc mpo} file are
{\sl not} SPH corrected.

\item[\moc{VALU}] keyword used to define a user-defined quantity as global parameter.
This keyword must be followed by the type of parameter.

\item[\moc{REAL}] keyword used to indicate that the user-defined global parameter
is a floating point value.

\item[\moc{CHAR}] keyword used to indicate that the user-defined global parameter
is a {\tt character*12} value.

\item[\moc{INTE}] keyword used to indicate that the user-defined global parameter
is an integer value.

\item[\moc{ISOT}] keyword used to select the set of particularized isotopes. The macroscopic
residual {\tt 'TotalResidual\_mix'} is always included as the last isotope in the list.

\item[\moc{TOUT}] keyword used to select all the available isotopes in the reference
\dds{microlib} named \dusa{HMIC} as particularized isotopes.

\item[\moc{MILI}] keyword used to select the isotopes in the reference
\dds{microlib} named \dusa{HMIC} from a specific mixture as particularized isotopes.

\item[\dusa{imil}] index of the mixture where the particularized isotopes are recovered.

\item[\moc{FISS}] keyword used to select all the available fissile isotopes in the reference
\dds{microlib} named \dusa{HMIC} as particularized isotopes.

\item[\moc{PF}] keyword used to select all the available fission products in the reference
\dds{microlib} named \dusa{HMIC} as particularized isotopes.

\item[\dusa{HNAISO}(i)] {\tt character*12} user-defined isotope name. $N_{\rm iso}$ is the
total number of explicitely--selected particularized isotopes.

\item[\moc{REAC}] keyword used to select the set of nuclear reactions. By fefault, the following reactions are selected:

\begin{tabular}{p{3.5cm} p{12.5cm}|}
\moc{Total} & Total cross sections as $\sigma_g^{\rm absorption}+\sigma_{0,g}^{\rm diffusion}$\\
\moc{Absorption} & Absorption cross sections $\sigma_g^{\rm absorption}$\\
\moc{Diffusion} & Scattering cross section for each available Legendre order \\
& $\sigma_{\ell,g}^{\rm diffusion}$. These cross sections are {\sl not} multiply by the $2\ell+1$ \\
& factor.\\
\moc{Fission} & Fission cross section \\
\moc{FissionSpectrum} & Steady-state fission spectrum \\
\moc{Nexcess} & Excess cross section due to (n,$x$n) reactions \\
\moc{NuFission} & $\nu\Sigma_{\rm f}$ cross sections \\
\moc{Scattering} & Scattering reaction as $\sigma_{\ell,g\rightarrow g'}=\sigma_{\ell,g\rightarrow g'}^{\rm elastic}+
\sigma_{\ell,g\rightarrow g'}^{\rm inelastic}+\sigma_{\ell,g\rightarrow g'}^{({\rm n},x{\rm n})}$\\
\moc{CaptureEnergyCapture} &  Energy production cross section for (n,$\gamma$) reaction only \\
\moc{FissionEnergyFission} & Energy production cross section for (n,f) reaction only \\
\end{tabular}

\item[\dusa{HNAREA}] {\tt character*20} name of a user-selected reaction in addition to default set. \dusa{HNAREA} is
selected among the following values:

\begin{tabular}{p{3.3cm} p{12.7cm}|}
\moc{TotalP1} & Total $P_1$-weighted cross sections \\
\moc{ElasticDiffusion} & Elastic scattering cross section for each available Legendre order \\
\moc{InelasticDiffusion} & Inelastic scattering cross section for each available Legendre order \\
\moc{NxnDiffusion} & (n,$x$n) scattering cross section for each available Legendre order \\
\moc{ElasticScattering} & Elastic scattering reaction $\sigma_{\ell,g\rightarrow g'}^{\rm elastic}$ \\
\moc{InelasticScattering} & Inelastic scattering reaction $\sigma_{\ell,g\rightarrow g'}^{\rm inelastic}$ \\
\moc{NxnScattering} & (n,$x$n) scattering reaction $\sigma_{\ell,g\rightarrow g'}^{({\rm n},x{\rm n})}$ \\
\moc{MT16}   & (n,2n) production cross sections \\
\moc{MT17}   & (n,3n) production cross sections \\
\moc{MT28}   & (n,np) production cross sections \\
\moc{MT37}   & (n,4n) production cross sections \\
\moc{MT103}   & (n,p) production cross sections \\
\moc{MT104}   & (n,d) production cross sections \\
\moc{MT105}   & (n,t) production cross sections \\
\moc{MT107}   & (n,$\alpha$) production cross sections \\
\moc{MT108}   & (n,2$\alpha$) production cross sections \\
\moc{Capture}   & (n,$\gamma$) production cross sections \\
\end{tabular}

\end{ListeDeDescription}

\subsubsection{Modification data input for module {\tt MPO:}}\label{sect:descmpo2}

\begin{DataStructure}{Structure \dstr{mpo\_data2}}
$[$ \moc{EDIT} \dusa{iprint} $]$ \\
$[$ \moc{STEP} \dusa{NAMDIR} $]$ \\
$[[$ \dusa{parkey} \dusa{value} $]]$ \\
$[$ \moc{SET} \dusa{xtr} $\{$ \moc{S} $|$ \moc{DAY} $|$ \moc{YEAR} $\}$ $]$ \\
{\tt ;}
\end{DataStructure}

\goodbreak
\noindent where
\begin{ListeDeDescription}{mmmmmmmm}

\item[\moc{EDIT}] keyword used to set \dusa{iprint}.

\item[\dusa{iprint}] index used to control the printing in module {\tt
MPO:}. =0 for no print; =1 for minimum printing (default value).

\item[\moc{STEP}] keyword used to access the {\sc mpo} database from a group named \dusa{NAMDIR}.
The default value is {\tt 'output\_0'}.

\item[\dusa{NAMDIR}] access the {\sc mpo} database in the group named \dusa{NAMDIR}. This name is
the concatenation of prefix {\tt 'output\_'} with an integer $\ge 0$.

\item[\dusa{parkey}] {\tt character*24} keyword associated to a user-defined global
parameter.

\item[\dusa{value}] floating-point, integer or {\tt character*12} value of a user-defined
global parameter.

\item[\moc{SET}] keyword used to recover the flux normalization factor already
stored on \dusa{BRNNAM} from a sub-directory corresponding to a specific time.

\item[\dusa{xtr}] time associated with the current flux calculation. The
name of the sub-directory where this information is stored will be given by
`{\tt DEPL-DAT}'//{\tt CNN} where {\tt CNN} is a  {\tt character*4} variable
defined by  {\tt WRITE(CNN,'(I4)') INN} where {\tt INN} is an index associated
with the time \dusa{xtr}.

\item[\moc{S}] keyword to specify that the time is given in seconds.

\item[\moc{DAY}] keyword to specify that the time is given in days.

\item[\moc{YEAR}] keyword to specify that the time is given in years.

\end{ListeDeDescription}

\subsubsection{Modification (catenate) data input for module {\tt MPO:}}\label{sect:descmpo3}

\vskip -0.5cm

\begin{DataStructure}{Structure \dstr{mpo\_data3}}
$[$ \moc{EDIT} \dusa{iprint} $]$ \\
$[$ \moc{STEP} \dusa{NAMDIR} $]$ \\
$[[$ \dusa{parkey} \dusa{value} $]]$ \\
$[$ \moc{WARNING-ONLY} $]$ \\
{\tt ;}
\end{DataStructure}

\noindent where
\begin{ListeDeDescription}{mmmmmmmm}

\item[\moc{EDIT}] keyword used to set \dusa{iprint}.

\item[\dusa{iprint}] index used to control the printing in module {\tt
MPO:}. =0 for no print; =1 for minimum printing (default value).

\item[\moc{STEP}] keyword used to access the {\sc mpo} database from a group named \dusa{NAMDIR}.
The default value is {\tt 'output\_0'}.

\item[\dusa{NAMDIR}] access the {\sc mpo} database in the group named \dusa{NAMDIR}. This name is
the concatenation of prefix {\tt 'output\_'} with an integer $\ge 0$.

\item[\dusa{parkey}] {\tt character*24} keyword associated to a
global parameter that is specific to \dusa{MPONAM} (not defined in \dusa{MPORHS}).

\item[\dusa{value}] floating-point, integer or {\tt character*12} value of a user-defined
global parameter.

\item[\moc{WARNING-ONLY}] This option is useful if an elementary calculation in \dusa{MPORHS} 
is already present in \dusa{MPONAM}. If this keyword is set, a warning is send and the \dusa{MPONAM} values
are kept, otherwise the run is aborted (default).

\end{ListeDeDescription}

\subsubsection{Specification of discontinuity factor and equivalent albedo information}\label{sect:adf}

Discontinuity factors and equivalent albedos are written in group {\tt flux} included in each state point group of the MPO file.
Specification of some datasets in group  {\tt flux} are slightly modified to hold this new information:

\begin{DescriptionEnregistrement}{Group /flux/ of the MPO file}{7.5cm}
\label{tabl:tabiso202}
\IntEnr
  {NALBP}{$1$}
  {Number $N_{\rm alb}$ of physical albedo (index $a$) values in each energy group.}
\IntEnr
  {NSURF}{$1$}
  {Number $N_{\rm surf}$ of external surfaces (index $b$) where discontinuity factors are defined.}
\OptRealEnr
  {ALBEDOG}{$N_{\rm alb},N_{\rm grp}$}{$N_{\rm alb}\ge 1$}{1}
  {Multigroup albedos $\beta_{a,g}$ obtained with a nodal equivalence procedure.}
\OptRealEnr
  {ALBEDOGxG}{$N_{\rm alb},N_{\rm grp}^2$}{$N_{\rm alb}\ge 1$}{1}
  {Matrix albedos $\beta_{a,g\to h}$ obtained with the equivalent reflector model (ERM).}
\OptRealEnr
  {SURF}{$N_{\rm surf}$}{$N_{\rm surf}\ge 1$}{cm$^2$}
  {Area $S_b$ of external surfaces.}
\OptRealEnr
  {SURFFLUX}{$N_{\rm surf},N_{\rm mil}\times N_{\rm grp}$}{$N_{\rm surf}\ge 1$}{$\phi \,$cm$^2$}
  {Multigroup fluxes $\phi^{\rm s}_{b,i,g}$ on external surfaces obtained with a nodal equivalence procedure in each output mixture $i$.}
\OptRealEnr
  {SURFFLUXGxG}{$N_{\rm surf},N_{\rm mil}\times N_{\rm grp}^2$}{$N_{\rm surf}\ge 1$}{$\phi \,$cm$^2$}
  {Matrix fluxes $\phi^{\rm s}_{b,i,g\to h}$ on external surfaces obtained with the equivalent reflector model (ERM) in each output mixture $i$.}
\RealEnr
  {TOTALFLUX}{$N_{\rm mil}\times N_{\rm grp}$}{$\phi$}
  {Multigroup fluxes $\phi_{i,g}$ in each output mixture $i$.}
\end{DescriptionEnregistrement}

\noindent where $N_{\rm mil}$ is the number of output mixtures and $N_{\rm grp}$ is the number of energy groups. Discontinuity factors are obtained from equations:
$$
ADF_{b,i,g}={\phi^{\rm s}_{b,i,g} \over S_b \, \phi_{i,g}} \ \  {\rm for \ multigroup \ discontinuity \ factors}
$$
\noindent and
$$
ADF_{b,i,g\to h}={\phi^{\rm s}_{b,i,g\to h} \over S_b \, \phi_{i,g}} \ \  {\rm for \ matrix \ discontinuity \ factors.}
$$

\clearpage
