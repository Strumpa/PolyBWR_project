\subsubsection{The {\tt TRIVAT:} tracking module}\label{sect:TRIVACData}

The {\tt TRIVAT:} module provides an implementation of the diffusion or simplified $P_n$ method. The {\tt TRIVAT:} module is
used to perform a TRIVAC-type ``tracking"  on a 1D/2D/3D regular Cartesian or hexagonal geometry.\cite{BIVAC,TRIVAC} The
geometry is analyzed and a LCM object with signature {\tt L\_TRIVAC} is created with the following information:

\begin{itemize}
\item Diagonal and hexagonal symmetries are unfolded and the mesh-splitting 
operations are performed. Volumes, material mixture and averaged flux recovery
indices are computed on the resulting geometry. \item A finite element
discretization is performed and the corresponding numbering is saved. \item The
unit finite element matrices (mass, stiffness, etc.) are recovered. \item
Indices related to an ADI preconditioning with or without supervectorization
are saved. \end{itemize}

The calling specification for this module is:

\begin{DataStructure}{Structure \dstr{TRIVAT:}}
\dusa{TRKNAM}
\moc{:=} \moc{TRIVAT:} $[$ \dusa{TRKNAM} $]$ 
\dusa{GEONAM} \moc{::}  \dstr{desctrack} \dstr{descTRIVAC}
\end{DataStructure}

\noindent  where
\begin{ListeDeDescription}{mmmmmmm}

\item[\dusa{TRKNAM}] {\tt character*12} name of the \dds{tracking} data
structure that will contain region volume and surface area vectors in
addition to region identification pointers and other tracking information.
If \dusa{TRKNAM} also appears on the RHS, the previous tracking 
parameters will be applied by default on the current geometry.

\item[\dusa{GEONAM}] {\tt character*12} name of the \dds{geometry} data
structure.

\item[\dstr{desctrack}] structure describing the general tracking data (see
\Sect{TRKData})

\item[\dstr{descTRIVAC}] structure describing the transport tracking data
specific to \moc{TRIVAT:}.

\end{ListeDeDescription}

\vskip 0.2cm

The \moc{TRIVAT:} specific tracking data in \dstr{descTRIVAC} is defined as

\begin{DataStructure}{Structure \dstr{descTRIVAC}}
$[~\{$ \moc{PRIM} $[$ \dusa{ielem} $]~|$ \moc{DUAL} $[$ \dusa{ielem} \dusa{icol} $]~|$ \moc{MCFD} $[$ \dusa{ielem} $]~|$ \moc{LUMP} $[$ \dusa{ielem} $]~\}~]$ \\
$[$ \moc{SPN} \dusa{n} $[$ \moc{SCAT} $[$ \moc{DIFF} $]$ \dusa{iscat} $]~[$ \moc{VOID} \dusa{nvd} $]~]$ \\
$[$ \moc{ADI} \dusa{nadi} $]$ \\
$[$ \moc{VECT} $[$ \dusa{iseg} $]~[$ \moc{PRTV} \dusa{impv} $]~]$ \\
{\tt ;}
\end{DataStructure}

\noindent where
\begin{ListeDeDescription}{mmmmmm}

\item[\dstr{desctrack}] structure describing the general tracking data (see
\Sect{TRKData})

\item[\moc{PRIM}] key word to set a discretization based on the variational collocation method.

\item[\moc{DUAL}] key word to set a mixed-dual finite element discretization. If the
geometry is hexagonal, a Thomas-Raviart-Schneider method is used.

\item[\moc{MCFD}] key word to set a discretization based  on the nodal
collocation method. The mesh centered finite difference approximation is the
default option and is generally set using {\tt MCFD~1}. The {\tt MCFD}
approximations are numerically equivalent to the {\tt DUAL} approximations
with \dusa{icol}=2; however, the {\tt MCFD} approximations are less
expensive. 

\item[\moc{LUMP}] key word to set a discretization  based on the nodal
collocation method with serendipity approximation. The serendipity
approximation is different from the \moc{MCFD} option in cases with \dusa{ielem}$\ge$2. This option is not available for hexagonal geometries.

\item[\dusa{ielem}] order of the finite element representation.  The values
permitted are: 1 (linear polynomials), 2 (parabolic polynomials), 3 (cubic
polynomials) or 4 (quartic polynomials). By default \dusa{ielem}=1.

\item[\dusa{icol}] type of quadrature used to  integrate the mass matrices.
The values permitted are: 1 (analytical integration), 2  (Gauss-Lobatto
quadrature) or 3 (Gauss-Legendre quadrature). By default \dusa{icol}=2. The
analytical integration corresponds to classical finite elements; the
Gauss-Lobatto quadrature corresponds to a variational or nodal type
collocation and the Gauss-Legendre quadrature corresponds to superconvergent
finite elements.

\item[\moc{SPN}] keyword to set a simplified spherical harmonics ($SP_n$) expansion
of the flux.\cite{nse2005,ane10a} This option is available with 1D, 2D and 3D Cartesian geometries and with 2D and 3D
hexagonal geometries.

\item[\dusa{n}] order of the $P_n$ or $SP_n$ expansion (odd number). Set to zero for diffusion theory (default value).

\item[\moc{SCAT}] keyword to limit the anisotropy of scattering sources.

\item[\moc{DIFF}] keyword to force using $1/3D^{g}$ as $\Sigma_1^{g}$ cross sections. A $P_1$ or $SP_1$ method
will therefore behave as diffusion theory.

\item[\dusa{iscat}] number of terms in the scattering sources. \dusa{iscat} $=1$ is used for
isotropic scattering in the laboratory system. \dusa{iscat} $=2$ is used for
linearly anisotropic scattering in the laboratory system. The default value is set to $n+1$
in $P_n$ or $SP_n$ case.

\item[\moc{VOID}] key word to set the number of base points in the Gauss-Legendre quadrature used to integrate
void boundary conditions if \dusa{icol} $=3$ and \dusa{n} $\ne 0$.

\item[\dusa{nvd}] type of quadrature. The values
permitted are: 0 (use a (\dusa{n}$+2$)--point quadrature consistent with $P_{\rm n}$ theory),
1 (use a (\dusa{n}$+1$)--point quadrature consistent with $S_{{\rm n}+1}$ theory),
2 (use an analytical integration of the void boundary conditions). By default \dusa{nvd}=0.

\item[\moc{ADI}] keyword to set the number of ADI iterations at the inner
iterative level.

\item[\dusa{nadi}] number of ADI iterations (default: \dusa{nadi} $=2$).

\item[\moc{VECT}] key word to set an ADI preconditionning with
supervectorization. By default, TRIVAC uses an ADI preconditionning without
supervectorization.

\item[\dusa{iseg}] width of a vectorial register. \dusa{iseg} is generally a multiple of 64. By default, \dusa{iseg}=64.

\item[\moc{PRTV}] key word used to set \dusa{impv}.

\item[\dusa{impv}] index used to control the  printing in supervectorization
subroutines. =0 for no print; =1 for minimum printing (default value); Larger
values produce increasing amounts of output.

\end{ListeDeDescription}

Various finite element approximations can be obtained with different values of \dusa{ielem}.

\eject
