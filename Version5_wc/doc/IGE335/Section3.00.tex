\section{THE DRAGON MODULES}\label{sect:DragonModuleInput}

The input to DRAGON is set up in the form of a structure containing commands
which call successively each of the calculation modules required in a given
transport calculation.

\subsection{The \moc{MAC:} module}\label{sect:MACData}

In DRAGON, the macroscopic cross sections associated with each mixture are
stored in a \dds{macrolib} (as an independent data structure or as part of
a \dds{microlib}) which may be generated using one of different ways:
\begin{itemize}
\item First, one can use directly the input stream already used for the remaining
DRAGON data. In this case, a single macroscopic library is involved.
\item The second method is via a GOXS format binary sequential
file.\cite{MATXS} It should be noted that a number of GOXS files may be read
successively by DRAGON and that it is possible to combine data from GOXS files
with data taken from the input stream. One can also transfer the macroscopic cross sections to a
GOXS format binary file if required. In this case, a single macroscopic library is involved.
\item The third input method is through a file which already contains a \dds{macrolib}. In this
case, two macroscopic and microscopic libraries are to be combined
\item The fourth method consists to update an existing \dds{macrolib} using control-variable
data recovered from a {\tt L\_OPTIMIZE} object.
\end{itemize}

The general format of the data for the \moc{MAC:} module is the following:
\begin{DataStructure}{Structure \dstr{MAC:}} 
$\{$ \dusa{MACLIB} \moc{:=} \moc{MAC:} $[$ \dusa{MACLIB} $]$ \moc{::} \dstr{descmacinp} \\
\hspace*{0.2cm} $|$ \dusa{MICLIB} \moc{:=} \moc{MAC:} \dusa{MICLIB} \moc{::} \dstr{descmacinp} \\
\hspace*{0.2cm} $|$ \dusa{MACLIB} \moc{:=} \moc{MAC:} $[$ \dusa{MACLIB} $]~[$ \dusa{OLDLIB} $]$ \moc{::} \dstr{descmacupd} \\
\hspace*{0.2cm} $|$ \dusa{MACLIB} \moc{:=} \moc{MAC:} \dusa{MACLIB} \dusa{OPTIM} \moc{;} \\
\hspace*{0.2cm} $\}$
\end{DataStructure}

\noindent
The meaning of each of the terms above is:

\noindent

\begin{ListeDeDescription}{mmmmmmmm}

\item[\dusa{MACLIB}] {\tt character*12} name of a \dds{macrolib} that will
contain the macroscopic cross sections. If \dusa{MACLIB} appears on both LHS and
RHS, it is updated; otherwise, it is created. If \dusa{MACLIB} is created, all
macroscopic cross sections are first initialized to zero.

\item[\dusa{MICLIB}] {\tt character*12} name of a \dds{microlib}. Only the
\dds{macrolib} data substructure of this \dds{microlib} is then updated. This is
used mainly to associate fixed sources densities with various mixtures. If any
other cross section is modified for a specific mixture, the
microscopic and macroscopic cross sections are no longer compatible. One can
return to a compatible library using the library update module (see
\Sect{LIBData}).

\item[\dusa{OLDLIB}] {\tt character*12} name of a \dds{macrolib} or a \dds{microlib}
which will be used to update or create the \dusa{MACLIB} \dds{macrolib}.

\item[\dusa{OPTIM}] {\tt character*12} name of a {\tt L\_OPTIMIZE} object. The
\dds{macrolib} \dusa{MACLIB} is updated using control-variable data recovered from \dusa{OPTIM}.

\item[\dstr{descmacinp}] macroscopic input data structure for this module (see
\Sect{descmacinp}).

\item[\dstr{descmacupd}] macroscopic update data structure for this module (see
\Sect{descmacupd}).

\end{ListeDeDescription}

\subsubsection{Input structure for module {\tt MAC:}}\label{sect:descmacinp}
 
In the case where there are no \dusa{OLDLIB} specified, the \dstr{descmac} input structure takes
the form:

\begin{DataStructure}{Structure \dstr{descmacinp}}
$[$ \moc{EDIT} \dusa{iprint} $]$ \\
$[$ \moc{NGRO} \dusa{ngroup} $]$ \\
$[$ \moc{NMIX} \dusa{nmixt} $]$ \\
$[$ \moc{NIFI} \dusa{nifiss} $]$ \\
$[$ \moc{DELP} \dusa{ndel} $]$ \\
$[$ \moc{ANIS} \dusa{naniso} $]$ \\
$[$ \moc{NADF} \dusa{nadf} $]$ \\
$[$ \moc{CTRA} $\{$ \moc{NONE} $|$ \moc{APOL} $|$ \moc{WIMS} $|$ \moc{LEAK} $\}$ $]$ \\
$[$ \moc{ALBP} \dusa{nalbp} ((\dusa{albedp}(ig,ia),ig=1,\dusa{ngroup}),ia=1,\dusa{nalbp}) $]$ \\
$[$ \moc{WRIT} \dusa{GOXSWN} $]$ \\
$[$ \moc{ENER} (\dusa{energy}(jg), jg=1,\dusa{ngroup} +1) $]$ \\
$[$ \moc{VOLUME} (\dusa{volume}(ibm), ibm=1,\dusa{nmixt}) $]$ \\
$[$ \moc{ADD} $]$ \\
$[[$ $\{$	\moc{READ} $[$ (\dusa{imat}(i), i=1,\dusa{nmixt}) $]$ \dusa{GOXSRN} $[$ \moc{DELE} $]$
$|$	 \moc{READ}  \moc{INPUT} $[[$ \dstr{descxs} $]]$ $\}$ $]]$ \\
$[[$ \moc{STEP} \dusa{istep} \moc{READ} \moc{INPUT} $[[$ \dstr{descxs} $]]~]]$ \\
$[$ \moc{NORM} $]$ \\
\moc{;}
\end{DataStructure}

\noindent with
\begin{ListeDeDescription}{mmmmmmmm}

\item[\moc{EDIT}] keyword used to modify the print level \dusa{iprint}.

\item[\dusa{iprint}] index used to control the printing in this module.
It must be set to 0 if no printing on the output file is required. The
macroscopic cross sections can written to the output file if the
variable \dusa{iprint} is greater than or equal to 2. The transfer cross
sections will be printed if this parameter is greater than or equal to 3. The
normalization of the transfer cross sections will be checked if \dusa{iprint}
is greater than or equal to 5.

\item[\moc{NGRO}] keyword to specify the number of energy groups for which
the macroscopic cross sections will be provided. This information is required
only if \dusa{MACLIB} is created and the cross sections are taken directly from
the input data stream.

\item[\dusa{ngroup}] the number of energy groups used for the calculations in
DRAGON. The default value is \dusa{ngroup}=1. 

\item[\moc{NMIX}] keyword used to define the number of material mixtures.
This information is required only if \dusa{MACLIB} is created and the cross
sections are taken directly from the input data stream or from a GOXS file.

\item[\dusa{nmixt}] the maximum number of mixtures (a mixture is
characterized by a distinct set of macroscopic cross sections) the 
\dds{macrolib} may contain. The default value is \dusa{nmixt}=1.

\item[\moc{NIFI}] keyword used to specify the maximum number of fissile
spectrum associated with each mixture. Each fission spectrum generally
represents a fissile isotope. This information is required only if \dusa{MACLIB}
is created and the cross sections are taken directly from the input data stream.

\item[\dusa{nifiss}] the maximum number of fissile isotopes per mixture. The
default value is \dusa{nifiss}=1.

\item[\moc{DELP}] keyword used to specify the number of delayed neutron groups.

\item[\dusa{ndel}] the number of delayed neutron groups. The
default value is \dusa{ndel}=0.

\item[\moc{ANIS}] keyword used to specify the maximum level of anisotropy
permitted in the scattering cross sections. This information is required only if
\dusa{MACLIB} is created and the cross sections are taken directly from the
input data stream.

\item[\dusa{naniso}] number of Legendre orders for the representation of the
scattering cross sections. The default value is \dusa{naniso}=1 corresponding to
the use of isotropic scattering cross sections.

\item[\moc{NADF}] keyword used to specify the number of averaged fluxes surrounding the geometry and used
to compute {\sl assenbly discontinuity factors} (ADF).

\item[\dusa{nadf}] number of averaged fluxes surrounding the geometry.

\item[\moc{CTRA}] keyword to specify the type of transport correction that
should be generated and stored on the \dds{macrolib}. The transport correction is to be
substracted from the total and isotropic ($P_0$) within-group scattering cross sections. A leakage correction, equal 
to the difference between current-- and flux--weighted total cross sections ($\Sigma_{1}-\Sigma_{0}$)
is also applied in the \moc{APOL} and \moc{LEAK} cases. All the modules that
will read this \dds{macrolib} will then have access to transport corrected
cross sections. The default is no transport correction when the \dds{macrolib} is created from the
input or GOXS files. 

\item[\moc{NONE}] keyword to specify that no transport correction should be
used in this calculation.

\item[\moc{APOL}] keyword to specify that an APOLLO type transport correction
based on the linearly anisotropic ($P_1$) scattering cross sections is to be set. This correction assumes that
the micro-reversibility principle is valid for all energy groups. $P_1$ scattering
information must exists in the {\sc macrolib}.

\item[\moc{WIMS}] keyword to specify that a WIMS--type transport correction is used.
The transport correction is recovered from a record named \moc{TRANC}. This
record must exists in the {\sc macrolib}.

\item[\moc{LEAK}] A leakage correction is applied to the total and
$P_0$ within-group scattering cross sections. No transport correction is 
applied in this case.

\item[\moc{ALBP}] keyword used for the input of the multigroup physical albedo array.

\item[\dusa{nalbp}] the maximum number of multigroup physical albedos. 

\item[\dusa{albedp}] multigroup physical albedo array. 

\item[\moc{WRIT}] keyword used to write cross section data to a GOXS file. In
the case where \dusa{nifiss}$>$1, this option is invalid. 

\item[\dusa{GOXSWN}] {\tt character*7} name of the GOXS file to be created or
updated.

\item[\moc{ENER}] keyword to specify the energy group limits.

\item[\dusa{energy}] energy (eV) array which define the limits of the groups
(\dusa{ngroup}+1 elements).  Generally \dusa{energy}(1) is the highest energy.

\item[\moc{VOLUME}] keyword to specify the mixture volumes.

\item[\dusa{volume}] volume (cm$^3$) occupied by each mixture.

\item[\moc{ADD}] keyword for adding increments to existing macroscopic cross
sections. In this case, the information provided in \dstr{descxs} represents
incremental rather than standard cross sections. 

\item[\moc{READ}] keyword to specify the input file format. One can use either
the input stream (keyword \moc{INPUT}) or a GOXS format file. 

\item[\dusa{imat}] array of mixture identifiers to be read from a GOXS file.
The maximum number of identifiers permitted is  \dusa{nmixt} and the maximum
value that \dusa{imat} may take is \dusa{nmixt}. When \dusa{imat} is 0, the
corresponding mixture on the GOXS file is not included in the \dds{macrolib}. In the
cases where \dusa{imat} is absent all the mixtures on the GOXS file are
available in a DRAGON execution. They are numbered consecutively starting at 1
or from the last number reached during a previous execution of the \moc{MAC:}
module.

\item[\dusa{GOXSRN}] {\tt character*7} name of the GOXS file to be read.

\item[\moc{DELE}] keyword to specify that the GOXS file is deleted after being read

\item[\moc{INPUT}] keyword to specify that mixture cross sections will be
read on the input stream.

\item[\dstr{descxs}] structure describing the format used for reading the
mixture cross sections from the input stream (see
\Sect{descxs}).

\item[\moc{STEP}] keyword used to create a perturbation directory.

\item[\dusa{istep}] the index of the perturbation directory.

\item[\moc{NORM}] keyword to specify that the macroscopic scattering cross
sections and the fission spectrum have to be normalized. This option is
available even if the mixture cross sections were not read by the \moc{MAC:}
module.

\end{ListeDeDescription}

\goodbreak

\subsubsection{Macroscopic cross section definition}\label{sect:descxs}

\begin{DataStructure}{Structure \dstr{descxs}}
\moc{MIX} $[$ \dusa{matnum} $]$ \\
\hskip 1.0cm $[~\{$ \moc{NTOT0} $|$ \moc{TOTAL} $\}$ (\dusa{xssigt}(jg),    jg=1,\dusa{ngroup}) $]$ \\
\hskip 1.0cm $[$ \moc{NTOT1} (\dusa{xssig1}(jg),    jg=1,\dusa{ngroup}) $]$ \\
\hskip 1.0cm $[$ \moc{TRANC} (\dusa{xsstra}(jg),    jg=1,\dusa{ngroup}) $]$ \\
\hskip 1.0cm $[$ \moc{NUSIGF} ((\dusa{xssigf}(jf,jg), jg=1,\dusa{ngroup}), jf=1,\dusa{nifiss}) $]$ \\
\hskip 1.0cm $[$ \moc{CHI}    ((\dusa{xschi}(jf,jg),    jg=1,\dusa{ngroup}), jf=1,\dusa{nifiss})$]$ \\
\hskip 1.0cm $[$ \moc{FIXE}   (\dusa{xsfixe}(jg),    jg=1,\dusa{ngroup}) $]$ \\
\hskip 1.0cm $[$ \moc{DIFF}   (\dusa{diff}(jg),    jg=1,\dusa{ngroup}) $]$ \\
\hskip 1.0cm $[$ \moc{DIFFX} (\dusa{xdiffx}(jg), jg=1,\dusa{ngroup}) $]$ \\
\hskip 1.0cm $[$ \moc{DIFFY} (\dusa{xdiffy}(jg), jg=1,\dusa{ngroup}) $]$ \\
\hskip 1.0cm $[$ \moc{DIFFZ} (\dusa{xdiffz}(jg), jg=1,\dusa{ngroup}) $]$ \\
\hskip 1.0cm $[$ \moc{NUSIGD} (((\dusa{xssigd}(jf,idel,jg), jg=1,\dusa{ngroup}), idel=1,\dusa{ndel}), jf=1,\dusa{nifiss}) $]$ \\
\hskip 1.0cm $[$ \moc{CHDL}   (((\dusa{xschid}(jf,idel,jg), jg=1,\dusa{ngroup}), idel=1,\dusa{ndel}), jf=1,\dusa{nifiss})$]$ \\
\hskip 1.0cm $[$ \moc{OVERV} (\dusa{overv}(jg), jg=1,\dusa{ngroup}) $]$ \\
\hskip 1.0cm $[$ \moc{NFTOT} (\dusa{nftot}(jg), jg=1,\dusa{ngroup}) $]$ \\
\hskip 1.0cm $[$ \moc{FLUX-INTG} (\dusa{xsint0}(jg), jg=1,\dusa{ngroup}) $]$ \\
\hskip 1.0cm $[$ \moc{FLUX-INTG-P1} (\dusa{xsint1}(jg), jg=1,\dusa{ngroup}) $]$ \\
\hskip 1.0cm $[$ \moc{H-FACTOR}   (\dusa{hfact}(jg),    jg=1,\dusa{ngroup}) $]$ \\
\hskip 1.0cm $[$ \moc{SCAT} (( 
    \dusa{nbscat}(jl,jg), \dusa{ilastg}(jl,jg),(\dusa{xsscat}(jl,jg,ig), \\
\hskip 2.0cm     ig=1,\dusa{nbscat}(jl,jg) ), jg=1,\dusa{ngroup}), jl=1,\dusa{naniso}) $]$ \\
\hskip 1.0cm $[[$ \moc{ADF} \dusa{hadf}  (\dusa{xadf}(jg),    jg=1,\dusa{ngroup}) $]]$
\end{DataStructure}

\begin{ListeDeDescription}{mmmmmmmm}

\item[\moc{MIX}] keyword to specify that the macroscopic cross sections
associated with a new mixture are to be read.

\item[\dusa{matnum}] identifier for the next mixture to be read. The maximum
value permitted for this identifier is \dusa{nmixt}. When \dusa{matnum} is
absent, the mixtures are numbered consecutively starting with 1 or with the last
mixture number read either on the GOXS or the input stream.  

\item[\moc{NTOT0}] keyword to specify that the total macroscopic cross
sections for this mixture follows.

\item[\moc{TOTAL}] alias keyword for \moc{NTOT0}.

\item[\dusa{xssigt}] array representing the multigroup total macroscopic cross
section ($\Sigma^{g}$ in \xsunit) associated with this mixture.

\item[\moc{NTOT1}] keyword to specify that the $P_1$--weighted total macroscopic cross
sections for this mixture follows.

\item[\dusa{xssig1}] array representing the multigroup $P_1$--weighted total macroscopic cross
section ($\Sigma_1^{g}$ in \xsunit) associated with this mixture.

\item[\moc{TRANC}] keyword to specify that the transport correction macroscopic cross
sections for this mixture follows.

\item[\dusa{xsstra}] array representing the multigroup transport correction macroscopic cross
section ($\Sigma_{\rm tc}^{g}$ in \xsunit) associated with this mixture.

\item[\moc{NUSIGF}] keyword to specify that the macroscopic fission cross
section multiplied by the average number of neutrons per fission for this
mixture follows.

\item[\dusa{xssigf}] array representing the multigroup macroscopic fission
cross section multiplied by the average number
of neutrons per fission ($\nu\Sigma_{f}^{g}$ in \xsunit) for all the fissile
isotopes associated with this mixture. 

\item[\moc{CHI}] keyword to specify that the fission spectrum for this mixture
follows.

\item[\dusa{xschi}] array representing the multigroup fission spectrum
($\chi^{g}$) for all the fissile isotopes associated with this mixture.

\item[\moc{FIXE}] keyword to specify that the fixed neutron source density for
this mixture follows.

\item[\dusa{xsfixe}] array representing the multigroup fixed neutron source
density for this mixture ($S^{g}$ in $s^{-1}cm^{-3}$). 

\item[\moc{DIFF}] keyword to specify that the isotropic diffusion coefficient for
this mixture follows.

\item[\dusa{diff}] array representing the multigroup isotropic diffusion coefficient for
this mixture ($D^{g}$ in $cm$). 

\item[\moc{DIFFX}] keyword for input of the $X$--directed diffusion coefficient. 

\item[\dusa{xdiffx}] array representing the multigroup $X$--directed diffusion coefficient ($D^g_x$ in cm) for the mixture 
\dusa{matnum}. 

\item[\moc{DIFFY}] keyword for input of the $Y$--directed diffusion coefficient. 

\item[\dusa{xdiffy}] array representing the multigroup $Y$--directed diffusion coefficient ($D^g_y$ in cm) for the mixture 
\dusa{matnum}. 

\item[\moc{DIFFZ}] keyword for input of the $Z$--directed diffusion coefficient.

\item[\dusa{xdiffz}] array representing the multigroup $Z$--directed diffusion coefficient ($D^g_z$ in cm) for the mixture 
\dusa{matnum}. 

\item[\moc{NUSIGD}] keyword to specify that the delayed macroscopic fission cross
section multiplied by the average number of neutrons per fission for this
mixture follows.

\item[\dusa{xssigd}] array representing the delayed multigroup macroscopic fission
cross section multiplied by the average number
of neutrons per fission ($\nu\Sigma_{f}^{g,idel}$ in \xsunit) for all the fissile
isotopes associated with this mixture. 

\item[\moc{CHDL}] keyword to specify that the delayed fission spectrum for this mixture
follows.

\item[\dusa{xschid}] array representing the delayed multigroup fission spectrum
($\chi^{g,idel}$) for all the fissile isotopes associated with this mixture.

\item[\moc{OVERV}] keyword for input of the multigroup average of the inverse neutron velocity.

\item[\dusa{overv}] array representing the multigroup average of the inverse neutron velocity ($<1/v>_{m}^g$) for the mixture 
\dusa{matnum}. 

\item[\moc{NFTOT}] keyword for input of the multigroup macroscopic fission cross sections.

\item[\dusa{nftot}] array representing the multigroup macroscopic fission cross section ($\Sigma_{f}^g$) for the mixture 
\dusa{matnum}. 

\item[\moc{FLUX-INTG}] keyword for input of the multigroup $P_0$ volume-integrated fluxes.

\item[\dusa{xsint0}] array representing the multigroup $P_0$ volume-integrated fluxes ($V\phi_0^g$) for the mixture 
\dusa{matnum}. 

\item[\moc{FLUX-INTG-P1}] keyword for input of the multigroup $P_1$ volume-integrated fluxes.

\item[\dusa{xsint1}] array representing the multigroup $P_1$ volume-integrated fluxes ($V\phi_1^g$) for the mixture 
\dusa{matnum}. 

\item[\moc{H-FACTOR}] keyword to specify that the power factor for
this mixture follows.

\item[\dusa{hfact}] array representing the multigroup power factor for this
mixture ($H^{g}$ in $MeV~cm^{-1}$). 

\item[\moc{SCAT}] keyword to specify that the macroscopic scattering cross
section matrix for this mixture follows.

\item[\dusa{nbscat}] array representing the number of primary groups ig with
non vanishing macroscopic scattering cross section towards the secondary group jg
considered for each anisotropy level associated with this mixture.

\item[\dusa{ilastg}] array representing the group index of the most thermal
group with non-vanishing macroscopic scattering cross section towards the
secondary group jg considered for each anisotropy level associated with this
mixture.

\item[\dusa{xsscat}] array representing the multigroup macroscopic scattering
cross section ($\Sigma_{sl}^{ig\to jg}$ in \xsunit) from the primary group ig
towards the secondary group jg considered for each anisotropy level associated
with this mixture. The elements are ordered using decreasing primary group
number ig, from \dusa{ilastg} to (\dusa{ilastg}$-$\dusa{nbscat}$+1$), and an
increasing secondary group number jg. Examples of input structures for 
macroscopic scattering cross sections can be
found in \Sect{ExXSData}.

\item[\moc{ADF}] keyword to specify that the boundary flux information for this mixture follows.

\item[\dusa{hadf}] character*8 type of a flux surrounding the geometry. The maximum number of types is equal to \dusa{nadf}.

\item[\dusa{xadf}] array representing a multigroup flux of type \dusa{hadf} surrounding the geometry for this
mixture. 

\end{ListeDeDescription}

\subsubsection{Update structure for operator {\tt MAC:}}\label{sect:descmacupd}
 
In the case where \dusa{OLDLIB} is specified, the \dstr{descmacupd} input structure takes
the form:

\begin{DataStructure}{Structure \dstr{descmacupd}}
$[$ \moc{EDIT} \dusa{iprint} $]$ \\
$[$ \moc{NMIX} \dusa{nmixt} $]$ \\
$[$ \moc{CTRA} \moc{OFF} $]$ \\
$[[$ \moc{MIX} \dusa{numnew} $[$ \dusa{numold} $\{$ \moc{UPDL} $|$ \moc{OLDL} $\}$ $]$ $]]$ \\
\moc{;}
\end{DataStructure}

\noindent with
\begin{ListeDeDescription}{mmmmmm}

\item[\moc{EDIT}] keyword used to modify the print level \dusa{iprint}.

\item[\dusa{iprint}] index used to control the printing in this operator.
It must be set to 0 if no printing on the output file is required. The
macroscopic cross sections can written to the output file if the
variable \dusa{iprint} is greater than or equal to 2. The transfer cross
sections will be printed if this parameter is greater than or equal to 3. The
normalization of the transfer cross sections will be checked if \dusa{iprint}
is greater than or equal to 5.

\item[\moc{NMIX}] keyword used to define the number of material mixtures.
This information is required only if \dusa{MACLIB} contains more mixtures than \dusa{OLDLIB}.

\item[\dusa{nmixt}] the maximum number of mixtures (a mixture is
characterized by a distinct set of macroscopic cross sections) \dusa{MACLIB}
may contain.

\item[\moc{CTRA}] keyword to specify the type of transport correction that
should be generated and stored on the \dds{macrolib}. All the operators that
will read this \dds{macrolib} will then have access to transport corrected
cross sections. In the case where the \dds{macrolib} is updated using other
\dds{macrolib} or \dds{microlib} the default is to use a transport correction whenever one of these
older data structure requires a transport correction.

\item[\moc{OFF}] deactivates the transport correction.

\item[\moc{MIX}] keyword to specify that the macroscopic cross sections
associated with a mixture is to be created or updated.

\item[\dusa{numnew}] mixture number to be updated or created on the output
\dds{macrolib}. 

\item[\dusa{numold}] mixture number on an old \dds{macrolib} or \dds{microlib} which will be used
to update or create \dusa{numnew} on the output macrolib 

\item[\moc{OLDL}] the
macroscopic cross sections associated with mixture \dusa{numold} are taken from \dusa{OLDLIB}. This is the
default option.

\item[\moc{UPDL}] the
macroscopic cross sections associated with mixture \dusa{numold} are taken from \dusa{MACLIB}.

\end{ListeDeDescription}

\eject
 % structure (dragonM)
\subsection{The {\tt LIB:} module}\label{sect:LIBData}

The general format of the input data for the \moc{LIB:} module is the following:

\vspace{-0.2cm}

\begin{DataStructure}{Structure \dstr{LIB:}}
\dusa{MICLIB} \moc{:=} \moc{LIB:} $[$ \dusa{MICLIB} $]~[~\{$ \dusa{MICRHS} $|$ \dusa{MACRHS} $|$ \dusa{EVORHS} $\}~]$ 
\moc{::} \dstr{desclib}
\end{DataStructure}

\vspace{-0.6cm}

\noindent
where

\begin{ListeDeDescription}{mmmmmmmm}

\item[\dusa{MICLIB}] {\tt character*12} name of the \dds{microlib} that will contain the internal
library. If \dusa{MICLIB} appears on both LHS and RHS, it is updated; otherwise, it is created. 

\item[\dusa{MICRHS}] {\tt character*12} name of a read-only \dds{microlib} data structure used by the
\moc{CATL} or \moc{MAXS} option of \Sect{desclib}.

\item[\dusa{MACRHS}] {\tt character*12} name of a read-only \dds{macrolib} data structure to be included
directly in \dusa{MICLIB} before updating it.

\item[\dusa{EVORHS}] {\tt character*12} name of a read-only \dds{burnup} data structure used by the
\moc{BURN} option of \Sect{desclib}. The number densities for the isotopes in file \dusa{MICLIB}
will be replaced selectively by those found in \dusa{EVORHS}.

\item[\dstr{desclib}] input structure for this module (see \Sect{desclib}).

\end{ListeDeDescription}

\subsubsection{Data input for module {\tt LIB:}}\label{sect:desclib}

In the case where \dusa{MICRHS} is absent or represents a \dds{macrolib}, \dstr{desclib} takes the form:

\begin{DataStructure}{Structure \dstr{desclib}}
$[$ \moc{EDIT} \dusa{iprint} $]$ \\
$[$ \moc{NGRO} \dusa{ngroup} $]$ \\
$[$ \moc{MXIS} \dusa{nmisot} $]$ \\
$[$ \moc{NMIX} \dusa{nmixt} $]$ \\
$[$ \moc{CALENDF} \dusa{ipreci} $]$ \\
$[$ \moc{CTRA} $\{$ \moc{NONE} $|$ \moc{APOL} $|$ \moc{WIMS} $|$ \moc{OLDW} $|$ \moc{LEAK} $\}$ $]$
$[$ \moc{ANIS}  \dusa{naniso} $]$ \\
$[$ \moc{STERN} \dusa{nstern} $]$ \\
$[$ \moc{ADJ} $]~[$ \moc{PROM} $]$ \\ 
$[~\{$ \moc{CDEPCHN} $|$ \moc{RDEPCHN} $\}~]$ \\
$[~\{$ \moc{SKIP} $|$ \moc{INTR} $|$ \moc{SUBG} $|$ \moc{PT} $|$ \moc{PTMC} $|$ \moc{PTSL} $|$ \moc{RSE} $[$ \dusa{svdeps} $]~|$ \moc{NEWL} $\}~]$ $[$
\moc{MACR} $]$\\ 
$[$ \moc{ADED} \dusa{nedit}  ( \dusa{HEDIT}(i), i=1,\dusa{nedit} ) $]$  \\
$[$ \moc{DEPL} $\{$ \moc{LIB:} $\{$ \moc{DRAGON} $|$ \moc{WIMSD4} $|$ \moc{WIMSE} $|$ \moc{WIMSAECL} $|$ \moc{NDAS} $|$ \moc{APLIB3} $\}$ \moc{FIL:} \dusa{NAMEFIL} \\
\hskip 0.6cm $|$ \moc{LIB:} $\{$ \moc{APLIB2} $|$ \moc{APXSM} $\}$ \moc{FIL:} \dusa{NAMEFIL} \dstr{descdeplA2} \\
\hskip 0.6cm $|$ \dusa{ndepl} \dstr{descdepl} $\}$ $]$ \\
$[[$ \moc{MIXS} \moc{LIB:} \\
\hskip 0.6cm  $\{$ \moc{DRAGON} $|$ \moc{MATXS} $|$ \moc{MATXS2} $|$
                  \moc{WIMSD4} $|$ \moc{WIMSE} $|$ \moc{WIMSAECL} $|$ \moc{NDAS} $|$    
                  \moc{APLIB1} $|$ \moc{APLIB2} \\
\hskip 0.85cm  $|$ \moc{APXSM} $|$ \moc{APLIB3} $|$ \moc{MICROLIB} $\}$ \\
\hskip 0.6cm  \moc{FIL:} \dusa{NAMEFIL} $[[$ \dstr{descmix1} $]]$ $]]$ \\
{\tt ;}
\end{DataStructure}

\noindent It is possible to reset an existing  \dds{microlib} (i.e., \dusa{MICLIB} is present
in both the LHS and RHS) and to reprocess all the isotopes from the cross section libraries.
In this case, \dstr{desclib} takes the simplified form:

\begin{DataStructure}{Structure \dstr{desclib}}
$[$ \moc{EDIT} \dusa{iprint} $]$ \\
$\{$ \moc{INTR} $|$ \moc{SUBG} $|$ \moc{PT} $|$ \moc{PTMC} $|$ \moc{PTSL} $|$ \moc{RSE} $[$ \dusa{svdeps} $]~|$ \moc{NEWL} $\}~[$ \moc{MACR} $]$ \\ 
\moc{MIXS} \\
{\tt ;}
\end{DataStructure}

\noindent
If keyword \moc{CATL} is given, \dusa{MICLIB} is catenated with the RHS \dusa{LIBRHS} \dds{microlib} .

\begin{DataStructure}{Structure \dstr{desclib}}
$[$ \moc{EDIT} \dusa{iprint} $]$ \\
$[$ \moc{MXIS} \dusa{nmisot} $]$ \\
$[$ \moc{NMIX} \dusa{nmixt} $]$ \\
$[~\{$ \moc{SKIP} $|$ \moc{MACR} $\}~]$
$[~\{$ \moc{CDEPCHN} $|$ \moc{RDEPCHN} $\}~]$ \\
$[$ \moc{DEPL} $\{$ \moc{LIB:} $\{$ \moc{DRAGON} $|$ \moc{WIMSD4} $|$ \moc{WIMSE} $|$ \moc{WIMSAECL} $|$ \moc{NDAS} $|$ \moc{APLIB3} $\}$ \moc{FIL:} \dusa{NAMEFIL} \\
\hskip 0.6cm $|$ \moc{LIB:} $\{$ \moc{APLIB2} $|$ \moc{APXSM} $\}$ \moc{FIL:} \dusa{NAMEFIL} \dstr{descdeplA2} \\
\hskip 0.6cm $|$ \dusa{ndepl} \dstr{descdepl} $\}$ $]$ \\
\moc{CATL} $[[$ \dstr{descmix2} $]]$ \\
{\tt ;}
\end{DataStructure}

\noindent
Alternatively if keyword \moc{BURN} or \moc{MAXS} is given, \dstr{desclib} takes the form:

\begin{DataStructure}{Structure \dstr{desclib}}
$[$ \moc{EDIT} \dusa{iprint} $]$ \\
$\{$ \moc{BURN} $\{$ \dusa{iburn} $|$ \dusa{tburn} $\}~|$ \moc{MAXS} $\}$
$[[$ \dstr{descmix2} $]]$ \\
{\tt ;}
\end{DataStructure}
\noindent where the RHS data structure is a \dds{burnup} (\dusa{EVORHS}) or a \dds{microlib} (\dusa{LIBRHS}) data structure. \dstr{desclib} options are:

\begin{ListeDeDescription}{mmmmmm}

\item[\moc{EDIT}] keyword used to modify the print level \dusa{iprint}.

\item[\dusa{iprint}] index used to control the printing in this operator. It
must be set to 0 if no printing on the output file is required while values
$>$0 will increase in steps the amount of information transferred to the output
file. If \dusa{iprint}$\ge$10, the depletion chain is printed in the format of
structure \dstr{descdepl}. If \dusa{iprint}$\ge$20, the depletion chain is also
printed in the format of structure \dstr{descdeplA2}.

\item[\moc{MXIS}] keyword used to redefine the maximum number of isotopes per
mixture.  

\item[\dusa{nmisot}] the maximum number of isotopes per
mixture. By default up to 300 different isotopes per mixture are permitted.

\item[\moc{NMIX}] keyword used to define the number of material mixtures. This
data is required if \dusa{MICLIB} is created.

\item[\dusa{nmixt}] the maximum number of mixtures (a mixture
is characterized by a distinct set of macroscopic cross sections). 

\item[\moc{CALENDF}] keyword to set the accuracy of the CALENDF probability
tables.

\item[\dusa{ipreci}] integer set to 1, 2, 3 or 4. The highest the value, the
more accurate are the probability tables. The default value is \dusa{ipreci}=4.

\item[\moc{CTRA}] keyword to specify the type of transport correction that
should be generated and stored on the \dds{microlib}. The transport correction is to be
substracted from the total and isotropic ($P_0$) within-group scattering cross sections. A leakage correction, equal 
to the difference between current-- and flux--weighted total cross sections ($\sigma_{1}-\sigma_{0}$)
is also applied in the \moc{APOL}, \moc{OLDW} and \moc{LEAK} cases. All the operators that
will read this \dds{microlib} will then have access to transport corrected
cross sections. The default is no transport correction.

\item[\moc{NONE}] keyword to specify that no transport correction should be
used in this calculation.

\item[\moc{APOL}] keyword to specify that an APOLLO type transport correction
based on the linearly anisotropic ($P_1$) within-group scattering cross sections is to be set. This correction assumes that
the micro-reversibility principle is valid for all energy groups. This type of
correction uses $P_1$ scattering information present on the library.

\item[\moc{WIMS}] This type of correction uses directly a transport-correction
provided on the library.
Such information is available in WIMSD4, WIMSE and WIMS--AECL libraries. This is
the new recommended option with WIMS-type libraries. {\sl This option has no effect on
libraries that does not contain transport correction information.}

\item[\moc{OLDW}] keyword to specify that a WIMS type transport
correction based on the $P_1$ scattering cross sections is to be
set. This correction
assumes that the micro-reversibility principle is valid only for groups energies
less than 4.0 eV. For the remaining groups a $1/E$ current spectrum is considered
in the evaluation of the transport correction. This type of correction uses
$P_1$ scattering information present on the library.

\item[\moc{LEAK}] A leakage correction is applied to the total and
$P_0$ within-group scattering cross sections. No transport correction is 
applied in this case.

\item[\moc{ANIS}] keyword to specify the maximum level of anisotropy for the
scattering cross sections.

\item[\dusa{naniso}] number of Legendre orders for the representation of the
scattering cross sections. Isotropic scattering is represented by
\dusa{naniso}=1 while  \dusa{naniso}=2 represents linearly anisotropic
scattering. Generally the linearly anisotropic ($P_1$) scattering contributions are
taken into account via the transport correction (see \moc{CTRA} keyword) in the
transport calculation. For $B_{1}$ or $P_{1}$ leakage calculations, the linearly
anisotropic scattering cross sections are taken into account explicitly.  The
default value is \dusa{naniso}=2.

\item[\moc{STERN}] keyword to specify the application of the Sternheimer density correction for charged particles.

\item[\dusa{nstern}] index used to control the Sternheimer correction application. Sternheimer correction applied for both restricted total stopping power
and heat deposition cross section ({\tt H-FACTOR}) is represented by \dusa{nstern} $=1$. A complete desactivation of the Sternheimer correction is obtained
by setting \dusa{nstern} $=0$. By default, the Sternheimer density correction is applied for both quantities. Notes: 1) The Sternheimer density correction should be
applied for both quantities except for specific charged particles cross sections perturbations analysis; 2) The Sternheimer density correction should be
applied on macroscopic cross sections. However, the heat deposition cross section contains a microscopic collisional stopping power which has not been
corrected in ELECTR module of NJOY. This is why the charged particle {\tt H-FACTOR} data $-$ recovered from microscopic libraries produced by ELECTR, but not
those produced by CEPXS-BFP $-$ should be corrected in DRAGON5.

\item[\moc{ADJ}] keyword to specify the production of adjoint macroscopic
cross sections. By default, direct cross sections are produced.

\item[\moc{PROM}] keyword to specify that prompt neutrons are to be considered
for the calculation of the fission spectrum. By default, the contribution due to
delayed neutrons is considered. This option is only compatible with a
\moc{MATXS} or \moc{MATXS2} format library.

\item[\moc{CDEPCHN}] keyword to enable the automatic completion of burnup chains.

\item[\moc{DDEPCHN}] keyword to avoid the automatic completion of burnup chains.

\item[\moc{SKIP}] keyword to recover the user--defined microlib data without processing
any library (i.e., without temperature and/or dilution interpolation).

\item[\moc{INTR}] keyword to perform a temperature and dilution interpolation
of the microscopic cross sections present in the libraries. The bin-type
cross-section data is not processed. This is the default option.

\item[\moc{SUBG}] keyword to activate the calculation of the physical probability
tables using the tempera\-tu\-re-interpolated cross-section data as
input.\cite{subg,nse2004} The bin-type cross-section data is not processed.

\item[\moc{PT}] keyword to activate the calculation of the CALENDF-type
mathematical probability tables ({\sl without} slowing-down correlated weight matrices)
using the bin-type cross-section data as input.\cite{pt} This option is
compatible with the Sanchez-Coste self-shielding method and with the subgroup projection method (SPM).\cite{SPM09}

\item[\moc{PTMC}] this option is similar to the \moc{PT} procedure. Here, the base points of the probability tables corresponding
to fission and scattering cross sections and to components of the transfer scattering matrix are also obtained using the CALENDF approach.

\item[\moc{PTSL}] keyword to activate the calculation of the CALENDF-type
mathematical probability tables and slowing-down correlated weight matrices
using the bin-type cross-section data as input.\cite{nse2004}

\item[\moc{RSE}] keyword to activate the generation of information for the resonance spectrum expansion (RSE) method.\cite{rse2021}

\item[\dusa{svdeps}] rank accuracy $\epsilon_{\rm svd}$ of the singular value decomposition. Singular values $w_i \le \epsilon_{\rm svd}\Delta u_{\rm elem}$ are set to zero.
$\Delta u_{\rm elem}$ is the elementary lethargy width of the Autolib. The default value is \dusa{svdeps}=1.0 $\times 10^{-3}$.

\item[\moc{NEWL}] keyword to activate the calculation of a microlib
containing temperature-interpo\-la\-ted cross-section data. The bin-type
cross-section data is also interpolated. Probability tables are not computed.

\item[\moc{MACR}] keyword to force the calculation of the embedded
macrolib. By default, the embedded macrolib is computed, {\sl except if} one of the
key words \moc{SKIP}, \moc{INTR}, \moc{SUBG}, \moc{PT} or \moc{NEWL} is used.

\item[\moc{ADED}] keyword to specify the input of additional cross sections to
be treated by DRAGON. These cross sections are not needed to solve the transport
equation but are recognized by the \moc{EDI:} and utility operators.

\item[\dusa{nedit}] number of types of additional cross sections.

\item[\dusa{HEDIT}] {\tt character*6} name of an additional
cross-section type. This name also corresponds to vectorial reactions in a
\moc{MATXS} and
\moc{MATXS2} format library. For example:

\moc{NWT0}/\moc{NWT1}=$P_0/P_1$ library weight functions.\\
\moc{NTOT0}/\moc{NTOT1}=$P_0/P_1$ neutron total cross sections.\\
\moc{NELAS}=Neutron elastic scattering cross sections (MT=2).\\
\moc{NINEL}=Neutron inelastic scattering cross sections (MT=4).\\
\moc{NG}=Neutron radiative capture cross sections (MT=102).\\
\moc{NFTOT}=Total fission cross sections (MT=18).\\
\moc{NUDEL}=Number of delayed secondary neutrons (Nu-D / MT=455).\\
\moc{NFSLO}=$\nu*$slow fission cross section.\\
\moc{NHEAT}=Heat production cross section.\\
\moc{CHIS}/\moc{CHID}=Slow/delayed fission spectrum.\\
\moc{NF}/\moc{NNF}/\moc{N2NF}/\moc{N3NF}=$\nu*$partial fission cross sections (MT=19, 20, 21 and 38).\\ 
\moc{N2N}/\moc{N3N}/\moc{N4N}=(n,2n), (n,3n), (n,4n) cross sections (MT=16, 17 and 37).\\
\moc{NP}/\moc{NA}=(n,p) and (n,$\alpha$) transmutation cross sections (MT=103 and 107).

By default, DRAGON will always attempt to recover the additional cross sections
\moc{NG}, \moc{NFTOT}, \moc{NHEAT} and \moc{N2N} which are required for the depletion
calculations. 

\item[\moc{DEPL}] keyword to specify that the isotopic depletion (burnup)
chain is to be read. For a given \moc{LIB:} execution only one isotopic
depletion chain can be read. 

\item[\moc{MIXS}] keyword to specify that the mixture description is to be
read. For a given \moc{LIB:} execution more than one cross-section library can
be read. 

\item[\moc{LIB:}] keyword to specify the type of library from which the
isotopic depletion chain or microscopic cross section is to be read. It is
optional when preceded by the keyword \moc{DEPL} in which case the isotopic
depletion chain is read from the standard input file. 

\item[\moc{DRAGON}] keyword to specify that the isotopic depletion chain or
the microscopic cross sections are in the {\sc draglib} format.

\item[\moc{MATXS}] keyword to specify that the microscopic cross sections are
in the MATXS format of NJOY-II and NJOY-89 (no depletion data available for
libraries using this format).

\item[\moc{MATXS2}] keyword to specify that the microscopic cross sections are
in the MATXS format of NJOY-91 (no depletion data available for libraries using
this format). The MATXS file is a binary sequential file by default. If the name
\dusa{NAMEFIL} has a leading ``{\tt \_}'' character, the MATXS file is expected to be
BCD-formatted, as produced by NJOY.

\item[\moc{WIMSD4}] keyword to specify that the isotopic depletion chain and the
microscopic cross sections are in the WIMSD4 format, as produced by module {\tt wimsr} of NJOY with flag
{\tt iverw} $=4$. This format is supported by the WLUP project.\cite{wlup}

\item[\moc{WIMSE}] keyword to specify that the isotopic depletion chain and the
microscopic cross sections are in the WIMSE format, as produced by module {\tt wimsr} of NJOY with flag
{\tt iverw} $=5$.

\item[\moc{WIMSAECL}] keyword to specify that the isotopic depletion chain and the
microscopic cross sections are in the WIMS-AECL format.

\item[\moc{NDAS}] keyword to specify that the isotopic depletion chain and the
microscopic cross sections are in the NDAS format, as used in recent versions of WIMS-AECL.

\item[\moc{APLIB1}] keyword to specify that the microscopic cross sections are
in the APOLLO-1 format. There are no depletion chains available for libraries using this
format.

\item[\moc{APLIB2}] keyword to specify that the microscopic cross sections are
in the APOLLO-2 direct access format. There are no depletion chains available for libraries
using this format. However, fission yields, radioactive decay constants and
energy released per fission or radiative capture are recovered from the file.
Only versions of the APOLIB-2 libraries subsequent or equal to CEA93-V4 can be
processed. The list of isotopes (standard and self-shielded) available in an APOLIB-2
is printed by setting the print flag to a value \dusa{iprint}$\ge$10.

\item[\moc{APXSM}] keyword to specify that the microscopic cross sections are
in the APOLIB-XSM format, the output format of N2A2 utility. There are no depletion chains available for libraries
using this format. However, fission yields, radioactive decay constants and
energy released per fission or radiative capture are recovered from the file.
The list of isotopes (standard and self-shielded) available in an APOLIB-XSM
is printed by setting the print flag to a value \dusa{iprint}$\ge$10.

\item[\moc{APLIB3}] keyword to specify that the microscopic cross sections are
in the APOLIB-3 format, the output format of the Galilee system. An ENDF/B evaluation is
represented by three HDF5 files:
\begin{description}
\item[\dusa{NAME1}:] HDF5 file containing infinite dilution information
\item[\dusa{NAME2}:] HDF5 file containing resonance self-shielding information
\item[\dusa{NAME3}:] HDF5 file containing depletion chains, branching ratio, fission yields and energy deposition information.
\end{description}
After \moc{DEPL}, the \moc{FIL:} keyword is followed by the concatenation of \dusa{NAME1} and \dusa{NAME3} with a colon character ({\tt :}) between
the two names. After \moc{MIXS}, the \moc{FIL:} keyword is followed by the concatenation of \dusa{NAME1} and \dusa{NAME2} with a colon character ({\tt :}) between
the two names. The list of isotopes (standard and self-shielded) available in an APOLIB-3
is printed by setting the print flag to a value \dusa{iprint}$\ge$10.

\item[\moc{MICROLIB}] keyword to specify that the microscopic cross sections are
in a {\sc microlib}-formatted object, as produced by DRAGON. This format is similar to the {\sc draglib}
format where the isotopes are stored in elements of list {\tt ISOTOPESLIST} instead of been stored
as independent sub-directories.

\item[\moc{FIL:}] keyword to specify the name of the file where is stored the
isotopic depletion data. 

\item[\dusa{NAMEFIL}] {\tt character*64} name of the library
where the isotopic depletion chain or the microscopic cross sections are stored.

Library names in {\sc draglib} format are limited to 12 characters.

An \moc{APLIB3} library name is the concatenation of two names with a colon character ({\tt :}) between them:
\begin{verbatim}
  DEPL LIB: APLIB3 FIL: CLA99CEA93:CLA99CEA93_EVO
  MIXS LIB: APLIB3 FIL: CLA99CEA93:CLA99CEA93_SS
\end{verbatim}

A \moc{NDAS} library is made of two or more files. These file names must be concatenated in a single
\dusa{NAMEFIL} name, using colons as separators. The {\sc ascii} index file is always the first,
followed by optional patch files, and terminated by the main direct-access binary file. The
following sample data line corresponds to a {\sc ndas} library without patch:
\begin{verbatim}
  MIXS LIB: NDAS FIL: E65LIB6.idx:E65LIB6.sdb
\end{verbatim}

\item[\dusa{ndepl}] number of isotopes in the depleting chain.

\item[\dstr{descdepl}] input structure describing the
depletion chain (see \Sect{descdepl}).

\item[\dstr{descdeplA2}] simplified input structure describing the
depletion chain in cases where an APOLIB-2 or APOLIB-XSM file is used (see \Sect{descdepl}).

\item[\moc{CATL}] keyword to perform the following operations:
\vspace{-0.15cm}
\begin{itemize}
\item create a new microlib or recover an existing \dds{microlib} in modification mode,
\item catenate with a RHS \dds{microlib} in read-only mode,
\item create the embedded  \dds{macrolib}.
\end{itemize}

\item[\moc{MAXS}] keyword to specify that the mixture density on \dusa{MICLIB}
are to be modified. If \dusa{MICRHS} is present and \dstr{descmix2} is absent, a
direct one to one correspondence between the isotope on both libraries is
assumed. If \dusa{MICRHS} and \dstr{descmix2} are present, only the
mixture on the library file specified by \dstr{descmix2} are updated using
information from the \dusa{MICRHS}. If \dusa{MICRHS} is absent and
\dstr{descmix2} is present, only the mixture on  \dusa{MICLIB} specified by
\dstr{descmix2} are updated. This option is useful for implementing two-level
computational schemes similar to REL-2005.

\item[\moc{BURN}] keyword to specify that the mixture density on \dusa{MICLIB}
are to be updated using information taken from \dusa{EVORHS}. If \dstr{descmix2}
is absent, a direct one to one correspondence between the isotope on
\dusa{EVORHS} and \dusa{MICLIB}  is assumed. If  \dstr{descmix2} is present, only
the mixture specified by \dstr{descmix2} are updated using information from
\dusa{EVORHS}. This option is useful for performing branching calculations.

\item[\dusa{iburn}] burnup step from the burnup file to use. This step must be
already present on the burnup file.

\item[\dusa{tburn}] burnup time in days from the burnup file to use. This time
step must be already present on the burnup file.

\item[\dstr{descmix1}] input structure describing the
isotopic and physical properties of a given mixture (see \Sect{descmix}).

\item[\dstr{descmix2}] input structure describing perturbations to the
isotopic and physical properties of a given mixture (see \Sect{descmix}).


\end{ListeDeDescription}

Note that it is possible to recompute the embedded macrolib in an existing microlib
named {\tt MICRO} by writing
\begin{verbatim}
MICRO := LIB: MICRO :: MACR MIXS ;
\end{verbatim}

\subsubsection{Depletion data structure}\label{sect:descdepl}

The structure \dstr{descdepl} describes the heredity of the radioactive decay
and the neutron activation chain to be used in the isotopic depletion
calculation.
\begin{DataStructure}{Structure \dstr{descdepl}}
\moc{CHAIN} \\
$[[$ \dusa{NAMDPL} $[$ \dusa{izae} $]$ \\
\hskip 1.0cm $[[~\{$ \moc{DECAY} \dusa{dcr} $|$ \\
\hskip 2.0cm \dusa{reaction} $[$ \dusa{energy} $]~\}~]]$ \\
\hskip 1.0cm $[~\{$ \moc{STABLE} $|$ \\
\hskip 2.0cm \moc{FROM} $[[~\{$ \moc{DECAY} $|$ \dusa{reaction} $\}$
$[[$ \dusa{yield} \dusa{NAMPAR} $]]~]]~\}~]~]]$\\
\moc{ENDCHAIN}
\end{DataStructure}

\vspace{-0.15cm}

\noindent
with:

\begin{ListeDeDescription}{mmmmmm}

\item[\moc{CHAIN}] keyword to specify the beginning of the depletion chain.

\item[\dusa{NAMDPL}] {\tt character*12} name of an isotope (or isomer) of the
depletion chain that appears in the cross-section library.

\item[\dusa{izae}] optional six digit integer representing the isotope. The first two
digits represent the atomic number of the isotope; the next three indicate its
mass number and the last digit indicates the  excitation level of the nucleus (0
for a nucleus in its ground state, 1 for an isomer in its first exited state,
etc.). For example, $^{238}$U in its ground state will be represented by
\dusa{izae}=922380.

\item[\moc{DECAY}] indicates that a decay reaction takes place either for
production of this isotope or its depletion.

\item[\dusa{dcr}] radioactive decay constant (in $10^{-8}$ s$^{-1}$) of the
isotope. By default, \dusa{dcr}=0.0.

\item[\dusa{reaction}] {\tt character*6} identification of a neutron-induced
reaction that takes place either for production of this isotope, its depletion,
or for producing energy. Example of reactions are following:

\begin{ListeDeDescription}{mmmmmmmm}
\item[\moc{NG}] indicates that a radiative capture reaction takes place either
for production of this isotope, its depletion or for producing energy.

\item[\moc{N2N}] indicates that the following reaction is taking place:
$$ n +^{A}X_Z \to 2 n + ^{A-1}X_Z$$

\item[\moc{N3N}] indicates that the following reaction is taking place:
$$ n +^{A}X_Z \to 3 n + ^{A-2}X_Z$$

\item[\moc{N4N}] indicates that the following reaction is taking place:
$$ n +^{A}X_Z \to 4 n + ^{A-3}X_Z$$

\item[\moc{NP}] indicates that the following reaction is taking place:
$$ n +^{A}X_Z \to p + ^AY_{Z-1}$$

\item[\moc{NA}] indicates that the following reaction is taking place:
$$ n +^{A}X_Z \to ^4{\rm He}_2 + ^{A-3}X_{Z-2}$$

\item[\moc{NFTOT}] indicates that a fission is taking place.
\end{ListeDeDescription}

\item[\dusa{energy}] energy (in MeV) recoverable per neutron-induced
reaction of type \dusa{reaction}. If the energy associated to radiative capture
is not explicitely given, it should be added to the energy released per fission. By
default, \dusa{energy}=0.0 MeV.

\item[\moc{STABLE}] non depleting isotope. Such an isotope may produces
energy by neutron-induced reactions (such as radiative capture).

\item[\moc{FROM}] indicates that this isotope is produced from decay or
neutron-induced reactions.

\item[\dusa{yield}] branching ratio or production yield expressed in fraction.

\item[\dusa{NAMPAR}] {\tt character*12} name of the a parent isotope
(or isomer) that appears in the cross-section library.

\item[\moc{ENDCHAIN}] keyword to specify the end of the depletion chain.

\end{ListeDeDescription}

\vskip 0.15cm

If the keyword \moc{APLIB2} or \moc{APXSM} was used in structure \dstr{desclib}, part of the
depletion data is recovered from the APOLIB file: the fission yields, the
radioactive decay constants and the energy released per fission or radiative
capture. Moreover, the following simplified structure is used to provide the
remaining depletion data:

\begin{DataStructure}{Structure \dstr{descdeplA2}}
\moc{CHAIN} \\
$[[$ \dusa{NAMDPL} $[$ \moc{FROM} $[[$ $\{$ \moc{DECAY} $|$ \dusa{reaction} $\}$
\dusa{yield} \dusa{NAMPAR} $]]$ $]$ $]]$\\
\moc{ENDCHAIN}
\end{DataStructure}

\vskip 0.15cm

In this case, the following rules apply:
\begin{itemize}
\item We should provide the names \dusa{NAMDPL} of {\sl all} the depleting
isotopes (i.e. isotopes with a time-dependent number density), including the
pseudo fission products (PFP).
\item The fission father reactions (\moc{NFTOT}) are not given.
\item The stable isotopes are automatically recovered from the
APOLIB file. They are not given in structure \dstr{descdeplA2}.
\item An isotope is considered to be stable if it is not present in
structure \dstr{descdeplA2}, has no father and no daughter,
but can release energy by fission or radiative capture.
\item It is possible to truncate the isotope name \dusa{NAMDPL} at the
underscore. For example, {\tt D2O\_3\_P5} can be simply written {\tt D2O}.
\item Only the radioactive decay constants of the isotopes present in
structure \dstr{descdeplA2} are recovered from the APOLIB file. The
radioactive decay constants of the other isotopes are set to zero.
\end{itemize}

\subsubsection{Mixture description structure}\label{sect:descmix}

The structure \dstr{descmix1} is used to describe the isotopic composition and
the physical properties, such as the temperature and density, of a mixture.

\begin{DataStructure}{Structure \dstr{descmix1}}
\moc{MIX} $[$ \dusa{matnum} $]$ $\{$ \\
\hskip 1.0cm $[$\dusa{temp} $[$ \dusa{denmix} $]~]~~[~\{$ \moc{NOEV} $|$ \moc{EVOL} $\}~]~~[~\{$ \moc{NOGAS}
    $|$ \moc{GAS}$\}~]$\\
\hskip 2.0cm $[[~[$ \dusa{NAMALI} \moc{=} $]$ \dusa{NAMISO} \dusa{dens} $[~\{$ \dusa{dil} 
    $|$ \moc{INF} $\}~]$\\
\hskip 2.0cm $[~[$ \moc{CORR} $]$ \dusa{inrs} $]~[$ \moc{DBYE} \dusa{tempd} $]~[$ \moc{SHIB} \dusa{NAMS} $]$ \\
\hskip 2.0cm $[$ \moc{THER} \dusa{ntfg} \dusa{HINC} $[$ \moc{TCOH} \dusa{HCOH} $]~[$ \moc{RESK} $]~]$ \\
\hskip 2.0cm $[$ \moc{IRSET} $\{$ \dusa{gir} $|~\{$ \moc{PT} $|$ \moc{PTMC} $|$ \moc{PTSL}$\}~\}~\{$
\dusa{nir} $|$ \moc{NONE} $\}~]~~[~\{$ \moc{NOEV} $|$ \moc{EVOL} $|$ \moc{SAT} $\}~]~]]$ \\
\hskip 1.0cm $|$ \\
\hskip 1.0cm \moc{COMB} $[[$ \dusa{mati} \dusa{relvol} $]]~\}$
\end{DataStructure}

\vspace{-0.15cm}

\noindent
where:

\begin{ListeDeDescription}{mmmmmm}

\item[\moc{MIX}] keyword to specify the number identifying the next mixture to
be read.

\item[\dusa{matnum}] mixture identifier. The maximum value that \dusa{matnum}
may have is \dusa{nmixt}. When \dusa{matnum} is absent, the mixtures are
numbered successively starting from 1 if no mixture has yet been specified or
from the last mixture number specified + 1.

\item[\dusa{temp}] absolute temperature (in Kelvin) of the isotopic mixture.
It is optional only when this mixture is to be updated, in which case the old
temperature associated with the mixture is used.

\item[\dusa{denmix}] mixture density in $g \ cm^{-3}$. 

\item[\dusa{NAMALI}] {\tt character*8} alias name for an isotope to be used
locally. When the alias name is absent, the isotope name used locally is
identical to the first 8-character isotope name on the library.

\item[\moc{=}] keyword to specify to which isotope in a library is associated
the previous alias name.

\item[\dusa{NAMISO}] {\tt character*12} name of an isotope present in the
library which is included in this mixture.

\item[\dusa{dens}]  isotopic concentration of the isotope \dusa{NAMISO} in the
current mixture in $10^{24}cm^{-3}$.  When the mixture density  \dusa{denmix}
is specified, the relative weight percentage of each of the isotopes in this
mixture is to be provided.

\item[\dusa{dil}] group independent microscopic dilution cross section (in
barns) of the isotope \dusa{NAMISO} in this mixture. It is possible to
recalculate a group dependent dilution for an isotope by the use of the
\moc{SHI:} or \moc{TONE:} operator (see \Sect{SHIData} and \Sect{TONEData}). In this case, the dilution is only used
as a starting point for the self-shielding iterations and has no effect on the
final result. If the dilution is not given or is larger than $10^{10}$ barns,
an infinite dilution is assumed.

\item[\moc{INF}] keyword to specify that a dilution of $10^{10}$ barns is to
be associated with this isotope. This value represents an infinite dilution (the
isotope is present in trace amounts only). It is possible to
recalculate a group dependent dilution for an isotope by the use of the
\moc{SHI:} operator (see \Sect{SHIData}) or \moc{TONE:} operator (see \Sect{TONEData}). In this case, the dilution is only used
as a starting point for the self-shielding iterations and has no effect on the
final result. If the dilution is not given an infinite dilution is assumed.

\item[\moc{CORR}] keyword to specify that the resonances of an isotope are correlated
with those of other isotopes with the same \dusa{inrs} index. This option is only
available with the {\sl Ribon extended} model\cite{nse2004} or wth the {\sl subgroup
projection method} (SPM)\cite{SPM09}  in energy groups where
this model is set. If this option is selected for
an isotope, it must be set for all isotopes with the same \dusa{inrs} index. By default,
the resonances of distinct isotopes are assumed to be uncorrelated.

\item[\dusa{inrs}] index of the resonant region associated with this isotope.
By default \dusa{inrs}=0 and the isotope is not a candidate for self-shielding.
When \dusa{inrs}$\ne$0, the isotope can be self-shielded where it is assumed that a given
isotope distributed with different concentrations in a number of mixtures and
having the same value of \dusa{inrs} will share the same fine flux. 
Should we wish to self-shield both the clad and the fuel it is important
to assign a different \dusa{inrs} number
to each. If a single type of fuel is located in different mixture in
{\sl onion-peel fashion}, it is necessary to attribute a single \dusa{inrs} value
to this fuel.

\item[\moc{DBYE}] keyword to specify that the absolute temperature of the
isotope is different from that of the isotopic mixture. This option is useful to
define Debye-corrected temperature.

\item[\dusa{tempd}] absolute temperature (in Kelvin) of the isotope. By
default \dusa{tempd}=\dusa{temp}.

\item[\moc{SHIB}] keyword to specify that the name of the isotope containing
the information related to the self-shielding is different from the initial name
of the isotope. This option is not required if a MATXS or a {\sc draglib} file is used.

\item[\dusa{NAMS}] {\tt character*12} name of a record in the library
containing the self-shielding data. This name is required if the dilution is
not infinite or a non zero resonant region is associated with this isotope and \dusa{NAMS}
is different from \dusa{NAMISO}. This record must be contained in the same
library file as record \dusa{NAMISO}.

\item[\moc{THER}] keyword to specify that the thermalization and resonant elastic
scattering kernel effects are to be included with the cross sections when using a
\moc{MATXS} or \moc{MATXS2} format library.

\item[\dusa{HINC}] {\tt character*6} name  of the incoherent thermalization
effects which will be taken into account. The incoherent effects are those that
may be described by the $S(\alpha,\beta)$ scattering law. The value \moc{FREE}
is used to simulate the effects of a gas.

\item[\moc{TCOH}]  keyword to specify that coherent thermalization effects
will be taken into account.

\item[\dusa{HCOH}] {\tt character*6} name of the coherent thermalization
effects which will be taken into account. The coherent effects are the
{\sl vectorial reactions} in the \moc{MATXS} or \moc{MATXS2} format library where
the name is terminated by the `\$' suffix. They are generally available for
graphite, beryllium, beryllium oxide, polyethylene and zirconium hydroxide.

\item[\moc{RESK}]  keyword to specify that resonant elastic scattering kernel effects
will be taken into account.

\item[\dusa{ntfg}]  number of energy groups that will be affected by the
thermalization and resonant elastic scattering kernel effects.

\item[\moc{IRSET}] keyword to specify an intermediate resonance (IR)
approximation or the {\sl Ribon extended} model for some energy groups. By default, an
IR approximation with the value of the Goldstein-Cohen parameter found on the library
is used. If no value is found on the library, a statistical (ST) model\cite{st} is set in
all groups by default. The ``{\tt IRSET PT 1}'' option is set by default if keyword \moc{PT}, \moc{PTMC} or
\moc{PTSL} is selected in structure \dstr{desclib}.

\item[\dusa{gir}]  imposed Goldstein-Cohen IR parameter. A Goldstein-Cohen IR parameter
$0 \le \lambda_g\le 1$ is set in energy group $g$. A value of 1.0 stands for
a statistical (ST) approximation. A value of 0.0 stands for an infinite mass
(IM or WR) approximation.

\item[\moc{PT}] keyword to enable the calculation of CALENDF--type probability tables in some energy groups. The
slowing-down correlated weight matrices are {\sl not} computed. This type of probability tables is consistent
with the Sanchez-Coste self-shielding method and with the subgroup projection method (SPM).\cite{SPM09}

\item[\moc{PTMC}] keyword to enable the calculation of CALENDF--type probability tables, similar to the \moc{PT}
procedure. Here, the base points of the probability tables corresponding
to fission and scattering cross sections and to components of the transfer scattering matrix are also obtained using the CALENDF approach.

\item[\moc{PTSL}] keyword to enable the calculation of CALENDF--type probability tables, consistent
with the Ribon extended model, in some energy groups.

\item[\dusa{nir}]  the intermediate resonance (IR) approximation or the Ribon extended
model is imposed for energy groups with an index equal or greater than \dusa{nir}.
A statistical (ST) model is set in other groups.

\item[\moc{NONE}] keyword to specify that a statistical (ST) model is set in
all groups.

\item[\moc{NOEV}] keyword to force a mixture or a nuclide to be non-depleting (even in
cases where it is potentially depleting). Note that the mixture or nuclide keeps its
capability to produce energy. By default, the depleting isotopes are
automatically regognized as depleting.

\item[\moc{EVOL}] keyword to force a mixture or a nuclide to be depleting. By default, only fission products and
fissile isotopes are depleting.

\item[\moc{NOGAS}] keyword to specify that a mixture has a solid or liquid state (used for stopping power correction).
This is the default option.

\item[\moc{GAS}] keyword to specify that a mixture has a gaseous state (used for stopping power correction).

\item[\moc{SAT}] keyword to force a nuclide to be at saturation. By default, the saturation approximation is
automatically set as a function of the half life and capture cross sections of the isotope.

\item[\moc{COMB}]  keyword to specify that this mixture is reset with a
combination of previously defined mixtures.

\item[\dusa{mati}]  number associated with a previously defined mixture. In
order to insert some void in a mixture use \dusa{mati}=0. If the mixture is not
already defined one assumes that it represents a voided mixture.

\item[\dusa{relvol}] relative volume $V_{i}$ occupied by mixture
\dusa{mati}=$i$ in \dusa{matnum}.  Two cases can be considered, namely that
where the density $\rho_{i}$ of each mixture \dusa{mati} is provided along with
the weight percent for each isotope $J$ ($W_{i}^{j}$) and the case where the
explicit concentration $N_{i}^{j}$ of each isotope in a \dusa{mati} was provided
(it is forbidden to combined two mixtures with different isotopic content
description). In the case where the initial mixtures are defined using densities
$\rho_{i}$, the density ($\rho_k$) and volume ($V_{k}$) of the final mixture
will become:
  $$V_{k}=\sum_{i} V_{i} $$
  $$\rho_{k}=\frac{1}{V_{k}} \sum_{i}\rho_{i}V_{i}$$
and the weight percent will be changed in a consistent way, namely
  $$W_{k,J}=\frac{\rho_{i}V_{i}W_{i,J}}{\rho_{k} V_{k} } $$
When the explicit concentration are given we will use:
  $$N_{k,J}=\frac{V_{i}N_{i,J}}{V_{k} } $$

\vskip 0.08cm

There is a very common usage of keyword \moc{COMB}. In the following example, a new mixture with index 42
is defined in such a way to be identical to an existing mixture with index 25. 
\begin{verbatim}
    MIX 42 COMB 25 1.0
\end{verbatim}

\end{ListeDeDescription}

Note that in the structure \dstr{descmix1} one only needs to describe the
isotopes initially present in each mixture. DRAGON will then automatically
associate with each depleting mixture the additional isotopes required by the
available burnup chain. Moreover, the microscopic cross-section library
associated with these new isotopes will be the same as that of their parent
isotope. For example, suppose that mixture 1 contains isotope {\tt U235} which
is to be read on the DRAGON-formatted library associated with file {\tt
DRAGLIB}. Assume also that the depletion chain, which is written on the 
WIMS--AECL format library associated with file {\tt WIMSLIB}, states that isotope
{\tt U236} (initially absent in the mixture) can be generated form {\tt U235} by
neutron capture. Then, one can either specify explicitly from which library file
the microscopic cross sections associated with isotope {\tt U236} (zero
concentration) are to be read, or omit {\tt U236} from the mixture description
in which case DRAGON will assume that the microscopic cross sections associated
with isotope {\tt U236} are to be read from the same library as the cross
section for isotope {\tt U235}. Note that the isotopes added automatically will
remain at infinite dilution.

\vskip 0.15cm

If the \moc{SHI:} or \moc{TONE:} module is used for performing self-shielding calculation,
the self-shielding data for an isotope takes the form
\begin{verbatim}
    U235     = U235  5.105E-5 1
\end{verbatim}
\noindent where the last index indicates the self-shielding region (1 in this case). 

\vskip 0.15cm

If the {\tt USS:} module implementing the subgroup method is used,
additional self-shielding data is required:
\begin{itemize}
\item Physical probability tables are used (keyword {\tt SUBG}). Consider the following data:
\begin{verbatim}
    U235     = U235  5.105E-5 1 IRSET 0.0 81
\end{verbatim}
The data ``{\tt IRSET 0.0 81}'' indicates that a Goldstein-Cohen parameter
$\lambda_g$ equal
to 0.0 is used for all energy groups with an index equal or greater than 81. A value
of $\lambda_g=1.0$ corresponding to a statistical model is used by default.

\item Mathematical probability tables (with slowing-down correlated weight matrices) are used (keyword {\tt PTSL})
{\sl or} mathematical probability tables with the subgroup projection method (SPM)\cite{SPM09} are used (keyword {\tt PT}
or {\tt PTMC}). Consider the following data:
\begin{verbatim}
    U235     = U235  5.105E-5 1 IRSET PT 5
\end{verbatim}
The Goldstein-Cohen approximation is not used with mathematical (CALENDF) probability tables. The data ``{\tt IRSET PT 5}''
indicates that the CALENDF probability tables are used for energy groups with an index equal
or greater than 5, {\sl with the exception of the energy groups where no Autolib data
is available} and a statistical model (with physical probability tables) is used for energy groups with an index smaller
than 5. A statistical model is also imposed in groups where no Autolib data is available.

\vskip 0.15cm

The following data:
\begin{verbatim}
    U235     = U235  5.105E-5 1 IRSET PT NONE
\end{verbatim}
\noindent is useful to impose the statistical model (with physical probability tables) in all energy groups. This is equivalent of selecting
the {\tt SUBG} keyword in structure \dstr{desclib}.

\vskip 0.15cm

Mathematical (CALENDF) probability tables are used in each energy group where Autolib data is available if the following data is set:
\begin{verbatim}
    U235     = U235  5.105E-5 1 IRSET PT 1
\end{verbatim}
\noindent {\sl This latter definition is equivalent to the default behavior obtained using}
\begin{verbatim}
    U235     = U235  5.105E-5 1
\end{verbatim}
\end{itemize}

\vskip 0.25cm
\goodbreak

The structure \dstr{descmix2} is used to describe the modifications in the isotopic composition of a mixture.

\begin{DataStructure}{Structure \dstr{descmix2}}
\moc{MIX}  \dusa{matnum} $[$ \dusa{matold} $]$ $[$ \dusa{relden} $]$
$[$ \dusa{NAMALI} \dusa{dens} $]~[~\{$ \moc{NOEV} $|$ \moc{EVOL} $\}~]$
\end{DataStructure}

\vspace{-0.15cm}

\noindent
where:

\begin{ListeDeDescription}{mmmmmm}

\item[\moc{MIX}] keyword to specify the number identifying the next mixture to
be updated.

\item[\dusa{matnum}] mixture identifier on \dusa{MICLIB}. 

\item[\dusa{matold}] mixture identifier on \dusa{MICRHS}. By default, \dusa{matold}]$=$\dusa{matnum}.

\item[\dusa{relden}] relative density of updated mixture. The  concentration
of each isotope in the mixture is to be multiplied by this factor whether it 
comes from \dusa{MICLIB}, from \dusa{MICRHS} or is
specified explicitly using \dusa{dens}. 

\item[\dusa{NAMALI}] {\tt character*8} alias name for an isotope on
\dusa{MICLIB} to be modified. 

\item[\dusa{dens}] isotopic concentration of the isotope \dusa{NAMISO} in the
current mixture in $10^{24}cm^{-3}$.  When \dusa{relden} is specified, the
isotopic concentration becomes \dusa{dens}$\times$\dusa{relden}.

\item[\moc{NOEV}] keyword to force a mixture to be non-depleting (even in
cases where it is potentially depleting). Note that the mixture keeps its
capability to produce energy.

\item[\moc{EVOL}] keyword to force a mixture to be depleting. By default, only
mixtures containing fission products and/or fissile isotopes are depleting.

\end{ListeDeDescription}

\vskip 0.2cm

\subsubsection{Cross sections in Dragon}\label{sect:xs}
Multigroup cross sections in Draglibs files are of two types:
\begin{itemize}
\item Vectorial cross sections $\sigma_{x,g}$
\item Matrix cross sections $\sigma_{x,g\leftarrow h}.$
\end{itemize}
\begin{enumerate}
\item Total cross sections $\sigma_g$ are provided in ENDF evaluations as {\tt MT} $=1$. They are redundent with other information in the same evaluation. The vectorial total cross section is defined as
\begin{eqnarray}
\nonumber \sigma_g\negthinspace &=&\negthinspace \sigma_{{\rm e},g}+\sigma_{{\rm in},g}+\sigma_{{\rm (n,2n)},g}+\sigma_{{\rm (n,3n)},g}+\sigma_{{\rm (n,4n)},g}+\sigma_{{\rm f},g}+\sigma_{{\rm p},g}+\sigma_{\gamma,g}
+\sigma_{{\rm d},g}+\sigma_{{\rm t},g}+\sigma_{\alpha,g}\\
&+&\negthinspace \sigma_{2\alpha,g}+\sigma_{{\rm (n,np)},g}+\sigma_{{\rm any},g}
\end{eqnarray}
\noindent where $\sigma_{{\rm e},g}$ and $\sigma_{{\rm in},g}$ are the elastic and inelastic scattering cross sections and where the matrix cross sections are transformed into vectorial cross sections using
\begin{equation}
\sigma_{x,g}=\sum_h \sigma_{x,h\leftarrow g} \ , \ \ {\rm except \ for \ (n,}x{\rm n) \ reactions.}
\end{equation}
\item Inelastic scattering cross sections are sum over {\tt MT} 51 to 91 in the ENDF evaluation:
\begin{equation}
\sigma_{{\rm in},g}=\sum_{{\sl mt}=51}^{91} \sigma_{{\sl mt},g}=\sum_{{\sl mt}=51}^{91} \sum_h \sigma_{{\sl mt},h\leftarrow g} .
\end{equation}
\item (n,$x$n) vectorial cross sections are divided by the secondary neutron multiplicity:
\begin{equation}
\sigma_{{\rm (n,2n)},g}={1\over 2}\sum_h \sigma_{{\rm (n,2n)},h\leftarrow g} \ , \ \ \sigma_{{\rm (n,3n)},g}={1\over 3}\sum_h \sigma_{{\rm (n,3n)},h\leftarrow g} \ , \ \ \sigma_{{\rm (n,4n)},g}={1\over 4}\sum_h \sigma_{{\rm (n,4n)},h\leftarrow g} .
\end{equation}
\item {\tt SCAT} matrix reactions in Dragon are defined as
\begin{eqnarray}
\nonumber \sigma_{{\tt scat},h\leftarrow g} \negthinspace\negthinspace &=& \negthinspace\negthinspace \sigma_{{\rm e},h\leftarrow g}+\sigma_{{\rm (n,2n)},h\leftarrow g}+\sigma_{{\rm (n,3n)},h\leftarrow g}+\sigma_{{\rm (n,4n)},h\leftarrow g}
+\sum_{{\sl mt}=51}^{91} \sigma_{{\sl mt},h\leftarrow g} \\
&+& \negthinspace\negthinspace \sigma_{{\rm any},h\leftarrow g} \, .
\end{eqnarray}
\item Vectorial {\sl neutronic scattering} ({\tt SIGS}) in Dragon is defined as
\begin{equation}
\sigma_{{\tt sigs},g}=\sum_h \sigma_{{\tt scat},h\leftarrow g}
\end{equation}
\noindent so that the {\sl neutronic absorption}, used to compute the $K_\infty$ is
\begin{eqnarray}
\nonumber \sigma_g-\sigma_{{\tt sigs},g}\negthinspace &=&\negthinspace \sigma_{{\rm f},g}+\sigma_{{\rm p},g}+\sigma_{\gamma,g}
+\sigma_{{\rm d},g}+\sigma_{{\rm t},g}+\sigma_{\alpha,g}+\sigma_{2\alpha,g}+\sigma_{{\rm (n,np)},g}\\
&-&\negthinspace \sigma_{{\rm (n,2n)},g}-2\sigma_{{\rm (n,3n)},g}-3\sigma_{{\rm (n,4n)},g}
\label{eq:eq1}
\end{eqnarray}
\noindent where all these terms are available in the Dragon microlib under the following names:\\
\vskip 0.1cm
\begin{tabular}{| l | l | l |}
\hline
Dragon name & $\sigma_x$ & type \\
\hline
{\tt NTOT0} & $\sigma_g$ & total \\
{\tt SIGS00} & $\sigma_{{\tt sigs},g}$ & neutronic scattering \\
{\tt NFTOT} &$\sigma_{{\rm f},g}$ & fission \\
{\tt NP} & $\sigma_{{\rm p},g}$ & (n,p) \\
{\tt NG} & $\sigma_{\gamma,g}$ & (n,$\gamma$) \\
{\tt ND} &$\sigma_{{\rm d},g}$ & (n,d) \\
{\tt NT} &$\sigma_{{\rm t},g}$ & (n,t) \\
{\tt NA} &$\sigma_{\alpha,g}$ & (n,$\alpha$) \\
{\tt N2A} &$\sigma_{2\alpha,g}$ & (n,2$\alpha$) \\
{\tt NNP} &$\sigma_{{\rm (n,np)},g}$ & (n,np) \\
{\tt NX} &$\sigma_{{\rm any},g}$ & (n,anything) \\
{\tt N2N} &$\sigma_{{\rm (n,2n)},g}$ & (n,2n) \\
{\tt N3N} &$\sigma_{{\rm (n,3n)},g}$ & (n,3n) \\
{\tt N4N} &$\sigma_{{\rm (n,4n)},g}$ & (n,4n) \\
\hline
\end{tabular}
\item The {\sl infinite multiplication factor} $K_\infty$ in a Dragon mixture is defined as
\begin{equation}
K_\infty={\sum\limits_g \nu\Sigma_{{\rm f},g}\bar\phi_g \over \sum\limits_g \left(\Sigma_g-\Sigma_{{\tt sigs},g}\right)\bar\phi_g}
\end{equation}
\noindent where $\nu\Sigma_{{\rm f},g} $, $\Sigma_g$ and $\Sigma_{{\tt sigs},g}$ are the macroscopic $\nu$-fission, total and
neutronic scattering cross sections, and $\bar\phi_g$ is the neutron flux.

\end{enumerate}

\eject
 % structure (dragonL)
\subsection{The {\tt GEO:} module}\label{sect:GEOData}

The \moc{GEO:} module is used to create or modify a geometry. The geometry
definition module in DRAGON permits all the characteristics (coordinates,
region mixture and boundary conditions) of a simple or complex
geometry to be specified. The method used to specify the geometry is independent
of the discretization module to be used subsequently. Each geometry is stored in
the form of a \dds{geometry} data structure under its given name. It is
always possible to modify an existing geometry or copy it under a new name. 
The calling specifications are:

\begin{DataStructure}{Structure \dstr{GEO:}}
$\{$ \\
\hskip 0.3cm \dusa{GEONAM} \moc{:=} \moc{GEO:} $\{$ \dusa{GEONAM} $|$ \dusa{OLDGEO} $\}$
\moc{::} \dstr{descgcnt}  \\
 $|$ \\
\hskip 0.3cm  \dusa{GEONAM} \moc{:=} \moc{GEO:} \moc{::} \dstr{descgtyp} \dstr{descgcnt}  \\
 $\}$ 
\end{DataStructure}

\noindent

\noindent where
\begin{ListeDeDescription}{mmmmmmmm}

\item[\dusa{GEONAM}] {\tt character*12} name of the \dds{geometry} created or
modified.

\item[\dusa{OLDGEO}] {\tt character*12} name of a read-only \dds{geometry}.
The type and all the characteristics of \dusa{OLDGEO} will be copied onto \dusa{GEONAM}
before this later geometry is modified.

\item[\dstr{descgtyp}] structure describing the geometry type of
\dusa{GEONAM} (see \Sect{descgeo}).

\item[\dstr{descgcnt}] structure describing the characteristics of a geometry
(see \Sect{descgeo}).

\end{ListeDeDescription}

\subsubsection{Data input for module {\tt GEO:}}\label{sect:descgeo}

Structures \dstr{descgtyp} and \dstr{descgcnt} are used to define respectively
the type of geometry that will be define and the contents of this geometry
(dimensions, materials, boundary conditions). The module \moc{GEO:} can be
recursively called from 
\dstr{descgcnt} as an embedded module, in order to define sub-geometries:

\begin{DataStructure}{Structure \dstr{descgtyp}}
$\{$ \moc{VIRTUAL} $|$ \\
\moc{HOMOGE} $|$\\
\moc{SPHERE} \dusa{lr} $|$ \\
\moc{CAR1D} \dusa{lx} $|$ \\
\moc{CAR2D} \dusa{lx} \dusa{ly} $|$\\ 
\moc{CAR3D} \dusa{lx} \dusa{ly} \dusa{lz} $|$  \\
\moc{TUBE} \dusa{lr} $[$ \dusa{lx} \dusa{ly} $]$  $|$\\
\moc{TUBEX} \dusa{lr} $\{$ \dusa{lx} $|$ \dusa{lx} \dusa{ly} \dusa{lz} $\}$ $|$\\ 
\moc{TUBEY} \dusa{lr} $\{$ \dusa{ly} $|$ \dusa{lx} \dusa{ly} \dusa{lz} $\}$ $|$\\
\moc{TUBEZ} \dusa{lr} $\{$ \dusa{lz} $|$ \dusa{lx} \dusa{ly} \dusa{lz} $\}$ $|$ \\
\moc{RTHETA} \dusa{lr} \dusa{lz} $|$ \\
\moc{HEX} \dusa{lh} $|$ \\
\moc{HEXZ} \dusa{lh} \dusa{lz} $|$ \\
\moc{HEXT} \dusa{nhr} $|$ \\
\moc{HEXTZ} \dusa{nhr} \dusa{lz} $|$ \\
\moc{CARCEL} \dusa{lr} $[$ \dusa{lx} \dusa{ly} $]$ $|$\\
\moc{CARCELX} \dusa{lr} $\{$ \dusa{lx} $|$ \dusa{lx} \dusa{ly} \dusa{lz} $\}$ $|$ \\
\moc{CARCELY} \dusa{lr} $\{$ \dusa{ly} $|$ \dusa{lx} \dusa{ly} \dusa{lz} $\}$ $|$ \\ 
\moc{CARCELZ} \dusa{lr} $\{$ \dusa{lz} $|$ \dusa{lx} \dusa{ly} \dusa{lz} $\}$ $|$ \\
\moc{HEXCEL} \dusa{lr} $|$ \\
\moc{HEXCELZ} \dusa{lr} \dusa{lz} $|$ \\
\moc{HEXTCEL} \dusa{lr} \dusa{nhr}$|$ \\
\moc{HEXTCELZ} \dusa{lr} \dusa{nhr} \dusa{lz} $|$ \\
\moc{GROUP} \dusa{lp} $\}$
\end{DataStructure}

\begin{DataStructure}{Structure \dstr{descgcnt}}
$[$ \moc{EDIT} \dusa{iprint} $]$ \\
\dstr{descBC} \\
\dstr{descSP} \\
\dstr{descPP} \\
\dstr{descDH} \\
\dstr{descSIJ} \\
$[[$ \moc{:::} \dusa{SUBGEO} \moc{:=} \moc{GEO:} $\{$ \dstr{descgtyp} $|$
\dusa{SUBGEO} $|$
\dusa{OLDGEO} $\}$ \dstr{descgcnt}$]]$ \\
\moc{;} 
\end{DataStructure}

\noindent
where

\begin{ListeDeDescription}{mmmmmmmm}

\item[\moc{VIRTUAL}] keyword to specify that a virtual geometry description
follows. This type of geometry is used to complete an assembly that has
irregular boundaries.

\item[\moc{HOMOGE}] keyword to specify that a infinite homogeneous geometry
description follows.

\item[\moc{SPHERE}] keyword to specify that a spherical geometry  (concentric
spheres) description follows.

\item[\moc{CAR1D}] keyword to specify that a one dimensional plane geometry
(infinite slab) description follows.

\item[\moc{CAR2D}] keyword to specify that a two-dimensional Cartesian
geometry description follows.

\item[\moc{CAR3D}] keyword to specify that a three-dimensional Cartesian
geometry description follows.

\item[\moc{TUBE}] keyword to specify that a cylindrical geometry (infinite
tubes or cylinders) description follows. This geometry can contain an imbedded $X-Y$ Cartesian mesh.

\item[\moc{TUBEX}] keyword to specify that a polar $R-X$ cylindrical geometry
description follows. This geometry can contain an imbedded $Y-Z$ Cartesian mesh.

\item[\moc{TUBEY}] keyword to specify that a polar $R-Y$ cylindrical geometry
description follows. This geometry can contain an imbedded $Z-X$ Cartesian mesh.

\item[\moc{TUBEZ}] keyword to specify that a polar $R-Z$ cylindrical geometry
description follows. This geometry can contain an imbedded $X-Y$ Cartesian mesh.

\item[\moc{RTHETA}] keyword to specify that a polar geometry ($R-\theta$)
description follows.

\item[\moc{HEX}] keyword to specify that a two-dimensional hexagonal geometry
description follows.

\item[\moc{HEXZ}] keyword to specify that a three-dimensional hexagonal
geometry description follows.

\item[\moc{HEXT}] keyword to specify a single 2-D hexagonal cell geometry having a triangular mesh. This option is only supported by the \moc{NXT:} tracking module (see \Sect{TRKData}).

\item[\moc{HEXTZ}] keyword to specify a single $Z$ directed 3-D hexagonal cell geometry having a triangular mesh (plane $X-Y$). This option is only supported by the \moc{NXT:} tracking module (see \Sect{TRKData}).

\item[\moc{CARCEL}] keyword to specify that a two-dimensional mixed Cartesian
cell (concentric tubes surrounded by a rectangle) description follows. The rectangle can now be
subdivided into a fine mesh when the \moc{EXCELT:} modules is used.

\item[\moc{CARCELX}] keyword to specify that a three-dimensional mixed
Cartesian cell with tubes oriented along the $X-$axis description follows. The three-dimensional 
Cartesian cell can now be subdivided into a fine mesh when the \moc{EXCELT:}
module is used.

\item[\moc{CARCELY}] keyword to specify that a three-dimensional mixed
Cartesian cell with tubes oriented along the $Y-$axis description follows. The three-dimensional 
Cartesian cell can now be subdivided into a fine mesh when the \moc{EXCELT:}
module is used.

\item[\moc{CARCELZ}] keyword to specify that a three-dimensional mixed
Cartesian cell with tubes oriented along the $Z-$axis description follows. The three-dimensional 
Cartesian cell can now be subdivided into a fine mesh when the \moc{EXCELT:}
module is used.

\item[\moc{HEXCEL}] keyword to specify that a two-dimensional mixed hexagonal cell (concentric tubes surrounded by a hexagon) description follows.

\item[\moc{HEXCELZ}] keyword to specify that a three-dimensional mixed hexagonal cell with tubes oriented along the $Z-$axis description follows.

\item[\moc{HEXTCEL}] keyword to specify a single 2-D hexagonal cell geometry having a triangular mesh and containing concentric annular regions.

\item[\moc{HEXTCELZ}] keyword to specify a single $Z$ directed 3-D hexagonal cell geometry a triangular mesh and containing concentric $Z$ directed cylinders. 

\item[\moc{GROUP}] keyword to specify that a {\sl do-it-yourself} type geometry
description follows.

\item[\dusa{lx}] number of subdivisions along the $X-$axis (before
mesh-splitting).

\item[\dusa{ly}] number of subdivisions along the $Y-$axis (before
mesh-splitting).

\item[\dusa{lz}] number of subdivisions along the $Z-$axis (before
mesh-splitting).

\item[\dusa{lr}] number of cylinders or spherical shells (before
mesh-splitting).

\item[\dusa{lh}] number of hexagons in an axial plane (including the virtual
hexagon).

\item[\dusa{nhr}] number of concentric hexagons in a \moc{HEXT}, \moc{HEXTZ}, \moc{HEXTCEL} or \moc{HEXTCELZ}  cell (see \Fig{GeoHEXT4}). This will lead to an hexagon subdivided into $6N^{2}$ identical trangles.

\begin{figure}[h!]  
\begin{center} 
\parbox{9.0cm}{\epsfxsize=9cm \epsffile{GeoHEXT4.eps}}
\parbox{14cm}{\caption{Hexagonal geometry with triangular mesh containing 4 concentric hexagon}\label{fig:GeoHEXT4}}   
\end{center}  
\end{figure}

\item[\dusa{lp}] number of types of cells (number of cells inside which a distinct flux will be calculated) for a \textsl{do-it-yourself} type geometry.

\item[\moc{EDIT}] keyword used to modify the print level \dusa{iprint}.

\item[\dusa{iprint}] index used to control the printing in this module.
It must be set to 0 if no printing on the output file is required, to 1 for
minimum printing (fixed default value) and to 2 for printing the geometry state
vector.

\item[\dstr{descBC}] structure allowing the boundary conditions surrounding
the geometry to be treated (see \Sect{descBC}).

\item[\dstr{descSP}] structure allowing the coordinates of a geometry to be
described (see \Sect{descSP}).

\item[\dstr{descPP}] structure allowing material mixtures to be associated
with a geometry (see \Sect{descPP}).

\item[\dstr{descDH}] structure used to specify double-heterogeneity data (see \Sect{descDH}).

\item[\dstr{descSIJ}] structure used to specify the properties of {\sl do-it-yourself}
geometries (see \Sect{descSIJ}).

\item[\dusa{SUBGEO}] {\tt character*12} name of the directory  that will
contain the sub-geometry.

\item[\dusa{OLDGEO}] {\tt character*12} name of a parallel directory
containing an existing sub-geometry. The type and all the characteristics of
\dusa{OLDGEO} will be copied onto \dusa{SUBGEO}.

\end{ListeDeDescription}

Note that all the geometry described above are called {\sl pure geometry} when
they do not contain sub-geometry. When they do contain sub-geometry they will be
called {\sl composite geometry}. 

\goodbreak
\subsubsection{Boundary conditions}\label{sect:descBC}

The inputs corresponding to the \dstr{descBC} structure are the following:

\begin{DataStructure}{Structure \dstr{descBC}}
$[$ \moc{X-} $\{$ \moc{VOID} $|$ \moc{REFL} $|$ \moc{SSYM} $|$ \moc{DIAG} $|$ \moc{TRAN} $|$
\moc{SYME} $|$ \moc{ALBE} $\{$ \dusa{albedo} $|$ \dusa{icode} $\}$ $|$ \moc{ZERO}
$|$ \moc{PI/2} $|$ \moc{PI} \\
~~~~~~~~ $|$ \moc{CYLI} $|$ \moc{ACYL} $\{$ \dusa{albedo} $|$ \dusa{icode} $\}$ $\}$ $]$ \\
$[$ \moc{X+}   $\{$ \moc{VOID} $|$ \moc{REFL} $|$ \moc{SSYM} $|$ \moc{DIAG} $|$ \moc{TRAN} $|$
\moc{SYME} $|$ \moc{ALBE} $\{$ \dusa{albedo} $|$ \dusa{icode} $\}$ $|$ \moc{ZERO}
$|$ \moc{PI} \\
~~~~~~~~ $|$ \moc{CYLI} $|$ \moc{ACYL} $\{$ \dusa{albedo} $|$ \dusa{icode} $\}$ $\}$ $]$ \\
$[$ \moc{Y-}   $\{$ \moc{VOID} $|$ \moc{REFL} $|$ \moc{SSYM} $|$ \moc{DIAG} $|$ \moc{TRAN} $|$
\moc{SYME} $|$ \moc{ALBE} $\{$ \dusa{albedo} $|$ \dusa{icode} $\}$ $|$ \moc{ZERO}
$|$ \moc{PI/2} $|$ \moc{PI} \\
~~~~~~~~ $|$ \moc{CYLI} $|$ \moc{ACYL} $\{$ \dusa{albedo} $|$ \dusa{icode} $\}$ $\}$ $]$ \\
$[$ \moc{Y+}   $\{$ \moc{VOID} $|$ \moc{REFL} $|$ \moc{SSYM} $|$ \moc{DIAG} $|$ \moc{TRAN} $|$ 
\moc{SYME} $|$ \moc{ALBE} $\{$ \dusa{albedo} $|$ \dusa{icode} $\}$ $|$ \moc{ZERO}
$|$ \moc{PI} \\
~~~~~~~~ $|$ \moc{CYLI} $|$ \moc{ACYL} $\{$ \dusa{albedo} $|$ \dusa{icode} $\}$ $\}$ $]$ \\
$[$ \moc{Z-}   $\{$ \moc{VOID} $|$ \moc{REFL} $|$ \moc{SSYM} $|$ \moc{TRAN} $|$ \moc{SYME} $|$ 
\moc{ALBE} $\{$ \dusa{albedo} $|$ \dusa{icode} $\}$  $|$ \moc{ZERO} $\}$ $]$ \\
$[$ \moc{Z+}   $\{$ \moc{VOID} $|$ \moc{REFL} $|$ \moc{SSYM} $|$ \moc{TRAN} $|$ \moc{SYME} $|$
\moc{ALBE} $\{$ \dusa{albedo} $|$ \dusa{icode} $\}$  $|$ \moc{ZERO} $\}$ $]$ \\
$[$ \moc{R+}   $\{$ \moc{VOID} $|$ \moc{REFL} $|$
\moc{ALBE} $\{$ \dusa{albedo} $|$ \dusa{icode} $\}$  $|$ \moc{ZERO} $\}$ $]$ \\
$[$ \moc{HBC}  $\{$ \moc{S30} $|$ \moc{SA60} $|$ \moc{SB60} $|$ \moc{S90} $|$
\moc{R120} $|$ \moc{R180} $|$ \moc{SA180} $|$ \moc{SB180} $|$ 
\moc{COMPLETE} $\}$ \\ $\{$ \moc{VOID} $|$ \moc{REFL} $|$ \moc{SYME} $|$
\moc{ALBE} $\{$ \dusa{albedo} $|$ \dusa{icode} $\}$ $|$ \moc{ZERO} $\}$ $]$ \\
$[$ \moc{RADS} $[$ \moc{ANG} $]$ \dusa{nrads} (\dusa{xrad}(ir), \dusa{rrad}(ir) $[$, \dusa{ang}(ir) $]$, ir=1,nrads ) $]$
\end{DataStructure}

\noindent
where: 

\begin{ListeDeDescription}{mmmmm}

\item[\moc{X-}/\moc{X+}] keyword to specify the boundary conditions associated with the
negative or positive $X$ surface of a Cartesian geometry.

\item[\moc{Y-}/\moc{Y+}] keyword to specify the boundary conditions associated with the
negative or positive $Y$ surface of a Cartesian geometry.

\item[\moc{Z-}/\moc{Z+}] keyword to specify the boundary conditions associated with the
negative or positive $Z$ surface of a Cartesian geometry.          

\item[\moc{R+}] keyword to specify the boundary conditions associated with the
outer surface of a cylindrical or spherical geometry.

\item[\moc{HBC}] keyword to specify the boundary conditions associated with
the outer surface of an hexagonal geometry.

\item[\moc{VOID}] keyword to specify that the surface under consideration has
zero re-entrant angular flux. This side is an external surface of the domain.

\item[\moc{REFL}] keyword to specify that the surface under consideration has a reflective boundary condition. In 
most DRAGON calculations, this implies white boundary conditions. The main exception to this 
rule is when cyclic tracking in 2-D is considered and mirror like reflections are considered. A geometry is never
unfolded to take into account a \moc{REFL} boundary condition.

\item[\moc{SSYM}] keyword to specify that the surface under consideration has a specular (or mirror) reflective boundary condition. The 
main difference between \moc{REFL} and \moc{SSYM} is that for \moc{SSYM} the cell may be unfolded to take 
into account the reflection at the boundary.

\item[\moc{DIAG}] keyword to specify that the Cartesian surface under
consideration has the same properties as that associated with a diagonal through
the geometry (see \Fig{cartebc}). Note that two and only two \moc{DIAG} surfaces must be specified.
The diagonal symmetry is only permitted for square geometry and in the following
combinations:  

\begin{verbatim}
X+ DIAG Y- DIAG 
\end{verbatim}

\noindent
or

\begin{verbatim}
X- DIAG Y+ DIAG 
\end{verbatim}

\item[\moc{TRAN}] keyword to specify that the surface under consideration is
connected to the opposite surface of a Cartesian domain (see \Fig{cartebcr}).
This option  provides
the facility to treat an infinite geometry with translation symmetry. The only
combinations of translational symmetry permitted are:

\begin{itemize}
\item Translation along the $X-$axis

\begin{verbatim}
X- TRAN X+ TRAN 
\end{verbatim}

\item Translation along the $Y-$axis

\begin{verbatim}
Y- TRAN Y+ TRAN 
\end{verbatim}

\item Translation along the $Z-$axis

\begin{verbatim}
Z- TRAN Z+ TRAN 
\end{verbatim}

\end{itemize}

\item[\moc{SYME}] keyword to specify that the Cartesian surface under
consideration is virtual and that a reflection symmetry is associated with the
adequately directed axis running through the center of the cells closest to this
surface (see \Fig{cartebcr}). Only the hexagonal geometries \moc{S30} and \moc{SA60} can be
surrounded by a \moc{SYME} boundary condition if a specular condition
is to be applied on this boundary.

\item[\moc{ALBE}] keyword to specify that the surface under consideration has
an arbitrary albedo. This side is an external surface of the domain.

\item[\dusa{albedo}] geometric albedo corresponding to the boundary condition
\moc{ALBE} (\dusa{albedo}$>$0.0). 

\item[\dusa{icode}] index of a physical albedo corresponding to the boundary
condition \moc{ALBE}. The numerical values of the physical albedo are supplied
by the operator \moc{MAC:} (see \Sect{MACData}).

\item[\moc{ZERO}] keyword to specify that the surface under consideration has a
zero-flux boundary condition. This side is an external surface of the domain.

\item[\moc{PI/2}] keyword to specify that the surface under consideration has a
$\pi$/2 rotational symmetry (see \Fig{cartebcr}). The only $\pi$/2 symmetry permitted is related to
sides ({\tt X-} and {\tt Y-}). This condition can be combined with a translation
boundary condition:({\tt PI/2 X- TRAN X+}) and/or ({\tt PI/2 Y- TRAN Y+}) (see \Fig{cartebct}).

\item[\moc{PI}] keyword to specify that the surface under consideration has a
$\pi$ rotational symmetry (see \Fig{cartebcr}). This keyword is useful for representing a
Cartesian checkerboard pattern as shown in Fig.~\ref{fig:cartebcdam}.

\item[\moc{CYLI}] the side under consideration has a zero incoming current boundary condition
with a circular correction applied on the Cartesian boundary. This option is only available in
the $X$--$Y$ plane for \moc{CAR2D} and \moc{CAR3D} geometries defined for TRIVAC full--core calculations.

\item[\moc{ACYL}] the side under consideration has an arbitrary albedo with a circular correction
applied on the Cartesian boundary. This option is only available in
the $X$--$Y$ plane for \moc{CAR2D} and \moc{CAR3D} geometries defined for TRIVAC full--core calculations.

\item[\moc{S30}] keyword to specify an hexagonal symmetry of one twelfth of an
assembly (see \Fig{s30}).

\item[\moc{SA60}] keyword to specify an hexagonal symmetry of one sixth of an
assembly of type A (see \Fig{s30}).

\item[\moc{SB60}] keyword to specify an hexagonal symmetry of one sixth of an
assembly of type B (see \Fig{sb60}).

\item[\moc{S90}] keyword to specify an hexagonal symmetry of one quarter of an
assembly (see \Fig{sb60}).

\item[\moc{R120}] keyword to specify a rotation symmetry of one third of an
assembly (see \Fig{r120}).

\item[\moc{R180}] keyword to specify a rotation symmetry of a half assembly
(see \Fig{r120}).

\item[\moc{SA180}] keyword to specify an hexagonal symmetry of half a type A
assembly (see \Fig{sa180}).

\item[\moc{SB180}] keyword to specify an hexagonal symmetry of half a type B
assembly (see \Fig{sb180}).

\item[\moc{COMPLETE}] keyword to specify a complete hexagonal assembly (see
\Fig{compl}).

\item[\moc{RADS}] This key word is used to specify the cylindrical correction applied in the $X-Y$ plane for \moc{CAR2D} and \moc{CAR3D} geometries.\cite{roy}

\item[\moc{ANG}] This key word allows  the angle (see \Fig{corr})
of the cylindrical notch to be set. By default, no notch is present.

\item[\dusa{nrads}] Number of different corrections along the cylinder main axis (i.e. the $Z$ axis).

\item[\dusa{xrad}(ir)] Coordinate of the $Z$ axis from which the correction is applied.

\item[\dusa{rrad}(ir)] Radius of the real cylindrical boundary.

\item[\dusa{ang}(ir)] Angle of the cylindrical notch. This data is given if and only if the key word \moc{ANG} is present. \dusa{ang}(ir) $= {\pi \over 2}$ by default (i.e. the correction is applied at every angle).

\end{ListeDeDescription}
\goodbreak

\begin{figure}[!]  
\begin{center} 
\epsfxsize=13cm
\centerline{ \epsffile{ebc.eps}}
\parbox{14cm}{\caption{Diagonal boundary conditions in Cartesian geometry}\label{fig:cartebc}}   
\end{center}  
\end{figure}

\begin{figure}[!]  
\begin{center} 
\epsfxsize=15cm
\centerline{ \epsffile{ebcr.eps}}
\parbox{14cm}{\caption{Various boundary conditions in Cartesian geometry}\label{fig:cartebcr}}   
\end{center}  
\end{figure}

\begin{figure}[!]  
\begin{center} 
\epsfxsize=10cm
\centerline{ \epsffile{ebct.eps}}
\parbox{14cm}{\caption{Translation/rotation boundary conditions in Cartesian geometry}\label{fig:cartebct}}   
\end{center}  
\end{figure}

\begin{figure}[!]  
\begin{center} 
\epsfxsize=13cm
\centerline{ \epsffile{ebcdam.eps}}
\parbox{14cm}{\caption{Representing a checkerboard in Cartesian geometry}\label{fig:cartebcdam}}   
\end{center}  
\end{figure}

\begin{figure}[!]  
\begin{center}
\epsfxsize=15cm
\centerline{ \epsffile{Gs30.eps}}
\parbox{14cm}{\caption{Hexagonal geometries of type S30 and SA60}\label{fig:s30}}   
\end{center}  
\end{figure}

\begin{figure}[!]  
\begin{center} 
\epsfxsize=15cm
\centerline{ \epsffile{Gsb60.eps}}
\parbox{14cm}{\caption{Hexagonal geometries of type SB60 and S90}\label{fig:sb60}}   
\end{center}  
\end{figure}

\begin{figure}[!]  
\begin{center} 
\epsfxsize=12cm
\centerline{ \epsffile{Gr120.eps}}
\parbox{14cm}{\caption{Hexagonal geometries of type R120 and R180}\label{fig:r120}}   
\end{center}  
\end{figure}

\begin{figure}[!]  
\begin{center} 
\epsfxsize=5cm
\centerline{ \epsffile{Gsa180.eps}}
\parbox{14cm}{\caption{Hexagonal geometry of type SA180}\label{fig:sa180}}   
\end{center}  
\end{figure}

\begin{figure}[!]  
\begin{center} 
\epsfxsize=11cm
\centerline{ \epsffile{Gsb180.eps}}
\parbox{14cm}{\caption{Hexagonal geometry of type SB180}\label{fig:sb180}}   
\end{center}  
\end{figure}

\begin{figure}[!]  
\begin{center} 
\epsfxsize=10cm
\centerline{ \epsffile{Gcomplete.eps}}
\parbox{14cm}{\caption{Hexagonal geometry of type
COMPLETE}\label{fig:compl}}    
\end{center}  
\end{figure}

\begin{figure}[!]
\begin{center} 
\epsfxsize=6cm
\centerline{ \epsffile{Fig6.eps}}
\parbox{14cm}{\caption{Cylindrical correction in Cartesian geometry}
\label{fig:corr}} 
\end{center} 
\end{figure}

\clearpage
\subsubsection{Spatial properties of geometry}\label{sect:descSP}
                                                  
The \dstr{descSP} structure has the following contents:
\begin{DataStructure}{Structure \dstr{descSP}}
$[$ \moc{MESHX} (\dusa{xxx}($i$), $i$=1,\dusa{lx}+1) $]$\\
$[$ \moc{SPLITX} (\dusa{ispltx}($i$), $i$=1,\dusa{lx}) $]$\\
$[$ \moc{MESHY}  (\dusa{yyy}($i$), $i$=1,\dusa{ly}+1) $]$\\
$[$ \moc{SPLITY} (\dusa{isplty}($i$), $i$=1,\dusa{ly}) $]$\\
$[$ \moc{MESHZ}  (\dusa{zzz}($i$), $i$=1,\dusa{lz}+1) $]$\\
$[$ \moc{SPLITZ} (\dusa{ispltz}($i$), $i$=1,\dusa{lz}) $]$\\
$[$ \moc{RADIUS} (\dusa{rrr}($i$), $i$=1,\dusa{lr}+1) $]$\\
$[$ \moc{OFFCENTER} (\dusa{disxyz}($i$), $i$=1,3) $]$\\
$[$ \moc{SPLITR} (\dusa{ispltr}($i$), $i$=1,\dusa{lr}) $]$\\
$[$ \moc{SECT} \dusa{isect} $[$ \dusa{jsect} $]~]$\\
$[$ \moc{SIDE} \dusa{sideh} $[$ \dusa{hexmsh} $]$ $]$\\
$[~\{$ \moc{SPLITH} \dusa{isplth} $|$ \moc{SPLITL} \dusa{ispltl} $\}~]$\\
$[$ $\{$ \moc{NPIN}  \dusa{npins} \\
\hspace{0.75cm} $\{$  $[$ \moc{RPIN} $\{$ \dusa{rpins} $|$ (\dusa{rpins}($i$), $i$=1, \dusa{npins}) $\}$ $]$ \\
\hspace{1.0cm} $[$ \moc{APIN}  $\{$ \dusa{apins} $|$ (\dusa{apins}($i$), $i$=1, \dusa{npins}) $\}$ $]$  $|$ \\
\hspace{1.0cm} $[$ \moc{CPINX} (\dusa{xpins}($i$), $i$=1, \dusa{npins})  $]$ \\
\hspace{1.0cm} $[$ \moc{CPINY} (\dusa{ypins}($i$), $i$=1, \dusa{npins})  $]$   \\
\hspace{1.0cm} $[$ \moc{CPINZ} (\dusa{zpins}($i$), $i$=1, \dusa{npins})  $]$   $\}$\\
\hspace{0.3cm}$|$ \moc{DPIN}  \dusa{dpins} $\}$ $]$
\end{DataStructure}

\begin{ListeDeDescription}{mmmmmmmm}

\item[\moc{MESHX}] keyword to specify the spatial mesh defining the regions along the $X-$axis. 

\item[\dusa{xxx}] array giving the $X$ limits (cm) of the regions making up the geometry. These values
must be given in order, from \moc{X-} to \moc{X+}. If the geometry presents a diagonal symmetry the same
data is also used along the $Y-$axis.

\item[\moc{SPLITX}] keyword to specify that a mesh splitting of the geometry along the $X-$axis is to be
performed.

\item[\dusa{ispltx}] array giving the number of zones that will be considered for each region along the
$X-$axis. If the geometry presents a diagonal symmetry this information is also used for the splitting
along the $Y-$axis. By default,
\dusa{ispltx}=1.

\item[\moc{MESHY}] keyword to specify the spatial mesh defining the regions along the $Y-$axis.

\item[\dusa{yyy}] array giving the $Y$ limits (cm) of the regions making up the geometry. These values
must be given in order, from \moc{Y-} to \moc{Y+}.

\item[\moc{SPLITY}] keyword to specify that a mesh splitting of the geometry along the $Y-$axis is to be
performed.

\item[\dusa{isplty}] array giving the number of zones that will be considered for each region along the
$Y-$axis. By default,
\dusa{isplty}=1 unless a diagonal symmetry is used in which case \dusa{isplty}$=$\dusa{ispltx}.

\item[\moc{MESHZ}] keyword to specify the spatial mesh defining the regions along the $Z-$axis.

\item[\dusa{zzz}] array giving the $Z$ limits (cm) of the regions making up the geometry. These values
must be given in order, from  \moc{Z-} to \moc{Z+}.

\item[\moc{SPLITZ}] keyword to specify that a mesh splitting of the geometry along the $Z-$axis is to be
performed.

\item[\dusa{ispltz}] array giving the number of zones that will be considered for each region along the
$Z-$axis. By default,
\dusa{ispltz}=1.

\item[\moc{RADIUS}] keyword to specify the spatial mesh along the radial direction.

\item[\dusa{rrr}] array giving the radial limits (cm) of the annular
regions (cylindrical or spherical) making up the geometry. It is used for the
following geometries: \moc{TUBE}, \moc{TUBEZ}, \moc{SPHERE}), \moc{CARCEL},
\moc{CARCELX}, \moc{CARCELY}, \moc{CARCELZ}, \moc{HEXCEL} and \moc{HEXCELZ}. It
is important to note that we must have \dusa{rrr}(1)=0.0. The other values
of \dusa{rrr}($i$) in a \moc{CARCEL}-- or \moc{HEXCEL}--type geometry are
defined as shown in \Fig{radius}.

\item[\moc{OFFCENTER}] keyword to specify that the concentric annular regions in a \moc{CARCEL},
\moc{CARCELX}, \moc{CARCELY},
\moc{CARCELZ}, \moc{TUBE}, \moc{TUBEX}, \moc{TUBEY} and \moc{TUBEZ} geometry can now be displaced with
respect to the center of the Cartesian mesh. This option will only be treated when the \moc{EXCELT:},
\moc{NXT:} and \moc{EXCELL:} modules are used.

\item[\dusa{disxyz}] array giving the $x$ (\dusa{disxyz}(1)), $y$ (\dusa{disxyz}(2)) and $z$
(\dusa{disxyz}(3)) displacement (cm) of the concentric annular regions with respect to the center of the
Cartesian mesh. 

\item[\moc{SPLITR}] keyword to specify that a mesh splitting of the geometry along the radial direction is
to be performed.

\item[\dusa{ispltr}] array giving the number of zones that will be considered for each region along the
radial axis.  A negative value results in a splitting of the regions into zones of equal volumes; a
positive value results in a uniform splitting along the radial direction. By default, \dusa{ispltr}=1.

\item[\moc{SECT}] keyword to specify the type of sectorization for a Cartesian
or hexagonal cell. In hexagonal geometry, this keyword is expected to be defined near the
\moc{SIDE} keyword. By default, no sectorization is performed.

\item[\dusa{isect}] sectorization index, defined as
\begin{displaymath}
\negthinspace\negthinspace\negthinspace isect = \left\{
\begin{array}{rl}
-999: & \textrm{non-sectorized cell processed as a sectorized cell} \\
-1: & \textrm{$\times$--type sectorization} \\
 0: & \textrm{non-sectorized cell} \\
 1: & \textrm{$+$--type sectorization} \\
 2: & \textrm{simultaneous $\times$-- and $+$--type sectorization} \\
 3: & \textrm{simultaneous $\times$-- and $+$--type sectorization shifted by 22.5$^\circ$} \\
 4: & \textrm{windmill sectorization.} 
\end{array} \right.
\end{displaymath}

\item[\dusa{jsect}] number of embedded tubes that are {\sl not} sectorized, with 0 $\le$ \dusa{jsect} $\le$ \dusa{lr}. By default, \dusa{jsect} $=0$. Examples of sectorization options are depicted in Figs.~\ref{fig:rect3} and~\ref{fig:hexa3}.

\item[\moc{SIDE}] keyword to specify the length of a side of a hexagon.

\item[\dusa{sideh}] length of one side of a hexagon (cm).

\item[\dusa{hexmsh}] triangular mesh for \moc{HEXT}, \moc{HEXTCEL}, \moc{HEXTZ} and \moc{HEXTCELZ} hexagonal geometries. By default, \dusa{hexmsh}=\dusa{sideh}/\dusa{nhr}. When \dusa{hexmsh} is provided, it is used instead of the default value with the following constraints 
$$
\textit{sideh} \le \textit{nhr}\times \textit{hexmsh}<\textit{sideh}+\textit{hexmsh}
$$
The triangles in the last hexagonal ring are truncated at \dusa{sideh} (see \Fig{GeoHEXT4C}).

\item[\moc{SPLITH}] keyword to specify that a triangular mesh splitting of the hexagonal geometry is to be performed -- for \moc{HEX}, \moc{HEXZ}, \moc{HEXT}, \moc{HEXTCEL}, \moc{HEXTZ} and \moc{HEXTCELZ} type geometries. This is valid only if \dusa{nhr}=1. 

\item[\dusa{isplth}] value of the triangular mesh splitting. Its use is similar to \dusa{nhr} except that each sector of the hexagonal cell will be filled by a unique mixture. The number of triangles per hexagon is given by $6 \times$\dusa{isplth}$^2$.
\dusa{isplth} $=0$ is used for full hexagon discretization.

\item[\moc{SPLITL}] keyword to specify that a lozenge mesh splitting of the hexagonal geometry is to be performed -- for \moc{HEX} and \moc{HEXZ} type geometries.

\item[\dusa{ispltl}] value of the lozenge splitting. The number of lozenges per hexagon is given by $3 \times$\dusa{ispltl}$^2$.

\item[\moc{NPIN}] keyword to specify the number of pins located in a cluster geometry. It can only be used for \moc{SPHERE}, \moc{TUBE}, \moc{TUBEX}, \moc{TUBEY} and \moc{TUBEZ} sub-geometry.

\item[\dusa{npins}] the number of pins associated with this sub-geometry in the primary geometry. 

\item[\moc{DPIN}] keyword to specify the pin density in a geometry that contains clusters. A number $N_{p,r}$ of pins that will be placed randomly in the geometry with
$$
N_{p,r}=\textrm{NINT}\left(\frac{d_{p,r}V_{c}}{V_{p}}\right)
$$
where $d_{p,r}$ is the pin density, $V_{g}$ the volume of the cell containing these pins and$V_{p}$ the volume of this pin type. The function $\textrm{NINT}()$ provides the nearest integer associated with its real argument. It can only be used for \moc{SPHERE}, \moc{TUBE}, \moc{TUBEX}, \moc{TUBEY} and \moc{TUBEZ} sub-geometry.

\item[\dusa{dpins}] the pin density $d_{p,r}$. 

\item[\moc{RPIN}] keyword to specify the radius of an imaginary cylinder where the centers of the pins are to be placed in a cluster geometry.

\item[\dusa{rpins}] the radius (cm) of an imaginary cylinder where the centers of the pins are to be placed. In the case where a single value is provided for \dusa{rpins}, all the pins are located at the same distance from the center of the cell (taking account the offset provided by the keyword \moc{OFFCENTER}). 

\item[\moc{APIN}] keyword to specify the angle of the first pin or each pin centered on an imaginary cylinder in a cluster geometry. 

\item[\dusa{apins}] the angle (radian) of the first pin in the ring (only one value provided for \dusa{apins}, the angular spacing of the pins being $2\pi/$\dusa{npins}) or the angle of each pins in the ring.

\item[\moc{CPINX}] keyword to specify the $x$ position where the centers of the pins are
to be placed in a cluster geometry. 

\item[\dusa{xpins}] the $x$ position (cm) where the centers of the pins are to be
placed.

\item[\moc{CPINY}] keyword to specify the $y$ position where the centers of the pins are
to be placed in a cluster geometry. 

\item[\dusa{ypins}] the $y$ position (cm) where the centers of the pins are to be
placed.

\item[\moc{CPINZ}] keyword to specify the $z$ position where the centers of the pins are
to be placed in a cluster geometry. 

\item[\dusa{zpins}] the $z$ position (cm) where the centers of the pins are to be
placed.

\end{ListeDeDescription}

\begin{figure}[!]  
\begin{center} 
\epsfxsize=6cm
\centerline{ \epsffile{radius.eps}}
\parbox{16cm}{\caption{Definition of the radii in a \moc{CARCEL}-- or
\moc{HEXCEL}--type geometry}\label{fig:radius}}    
\end{center}  
\end{figure}

The user should be warned that the maximum number of zones resulting from the above description of a geometry $L_{\rm{zones}}$ should not exceed the limits imposed by
\dusa{maxreg} and defined in the tracking module \moc{SYBILT:}, \moc{NXT:} or
\moc{EXCELT:} (see \Sect{TRKData}). For pure geometry with splitting we can define the variables $L_x$, $L_y$, $L_z$, $L_r$, $L_h$ and $L_{t}$ as:
  \begin{align*}
  L_x=&\sum_{i=1}^{\textit{lx}} \textit{ispltx}(i) \\ 
  L_y=&\sum_{i=1}^{\textit{ly}} \textit{isplty}(i) \\ 
  L_z=&\sum_{i=1}^{\textit{lz}} \textit{ispltz}(i) \\ 
  L_r=&\sum_{i=1}^{\textit{lr}} |\textit{ispltr}(i)| \\
  L_h=&\textit{lh} \\
  L_t=&\begin{cases}
  6\times\textit{nhr}^{2} &if $\textit{nhr}> 1$\\
  6\times\textit{isplith}^{2} &otherwise  \\ \end{cases}
  \end{align*}
and $L_{\rm{zones}}$ will be given by:

\begin{itemize}

\item \moc{SPHERE} geometry.

$$L_{\rm{zones}}=L_r$$

\item \moc{TUBE} geometry.

$$L_{\rm{zones}}= L_x L_y L_r $$

\item \moc{TUBEX} geometry.

$$L_{\rm{zones}}= L_x L_y L_z L_r$$

\item \moc{TUBEY} geometry.

$$L_{\rm{zones}}= L_x L_y L_z L_r$$

\item \moc{TUBEZ} geometry.

$$L_{\rm{zones}}= L_x L_y L_z L_r$$

\item \moc{CAR1D} geometry.

$$L_{\rm{zones}}=L_x$$

\item \moc{CAR2D} geometry 
\begin{itemize}
\item without diagonal symmetry. 

$$L_{\rm{zones}}=L_x L_y$$

\item with diagonal symmetry. 

$$L_{\rm{zones}}=\frac{L_x (L_y+1)}{2}=\frac{(L_x+1) L_y}{2}$$
\end{itemize}

\item \moc{CARCEL} geometries.

$$L_{\rm{zones}}=L_x L_y (L_r+1) $$

\item \moc{CAR3D} geometry 
\begin{itemize}
\item without diagonal symmetry. 

$$L_{\rm{zones}}=L_x L_y L_z$$

\item with diagonal symmetry. 

$$L_{\rm{zones}}=\frac{L_x (L_y+1) L_z}{2}=\frac{(L_x+1) L_y L_z}{2}$$
\end{itemize}

\item \moc{CARCELX} geometry.

$$L_{\rm{zones}}=L_x L_y L_z (L_r+1) $$

\item \moc{CARCELY} geometry.

$$L_{\rm{zones}}=L_x L_y L_z (L_r+1) $$

\item \moc{CARCELZ} geometries.

$$L_{\rm{zones}}=L_x L_y L_z (L_r+1) $$

\item \moc{HEX} geometry.

\begin{align*}L_{\text{zones}}&=L_h\end{align*}

\item \moc{HEXT} geometry.

\begin{align*}L_{\text{zones}}&=L_{t}\end{align*}

\item \moc{HEXCEL} geometries.

\begin{align*}L_{\text{zones}}&=(L_r+1) \end{align*}

\item \moc{HEXTCEL} geometries.

$$L_{\rm{zones}}=L_{t}$$

\item \moc{HEXZ} geometry.

\begin{align*}L_{\text{zones}}&=L_z L_h\end{align*}

\item \moc{HEXTZ} geometry.

\begin{align*}L_{\text{zones}}&=L_z L_{t}\end{align*}

\item \moc{HEXCELZ} geometries.

\begin{align*}L_{\text{zones}}&=L_z (L_r+1) \end{align*}

\item \moc{HEXTCELZ} geometries.

\begin{align*}L_{\text{zones}}&=L_z L_{t} (L_r+1) \end{align*}

\end{itemize}

For cluster geometries, only one region is associated with each zone in a pin even if this pin is repeated \dusa{npins} times.

\vskip 0.08cm

For mixed geometries, it is important to ensure that $L_{\rm{zones}}$ which represents the
sum over all the sub-geometries of the total number of regions $L^i_t$
associated with each pure sub-geometry $i$ computed using the technique
described above. For cluster geometries, only one region is associated with each
zone in a pin even if this pin is repeated \dusa{npins} times.

\begin{figure}[h!]
\begin{center} 
\epsfxsize=16cm
\centerline{ \epsffile{rect3c.eps}}
\parbox{14cm}{\caption{Numerotation of the sectors in a Cartesian cell}\label{fig:rect3}}   
\end{center}
\end{figure}

\begin{figure}[h!]
\begin{center} 
\epsfxsize=13cm
\centerline{ \epsffile{hexa3c.eps}}
\parbox{14cm}{\caption{Numerotation of the sectors in an hexagonal cell}\label{fig:hexa3}}   
\end{center}
\end{figure}

\begin{figure}[h!]  
\begin{center} 
\parbox{11.0cm}{\epsfxsize=11cm \epsffile{GeoHEXT4C.eps}}
\parbox{14cm}{\caption{Hexagonal geometry with triangular mesh that extends past the hexagonal boundary}\label{fig:GeoHEXT4C}}   
\end{center}  
\end{figure}

\subsubsection{Physical properties of geometry}\label{sect:descPP}

In addition to specifying the mixture associated with each region in the
geometry, the \dstr{descPP} structure is also used to provide information on the
sub-geometry required in this geometry. For example, an optional procedure in
DRAGON groups together regions so as to reduce the number of unknowns
\dusa{maxreg} in the flux calculation. In this way, only the merged regions
contribute to the cost of the calculation. However, the following points must be
considered:

\begin{enumerate}

\item All the cells belonging to the same merged region must have the same
nuclear properties and dimensions. 

\item The grouping procedure is based on the approximation that all the regions
belonging to the same merged region share the same flux. 

\item The merging can also take into account region orientation  (by a rotation
and/or transposition) before they are merged. This procedure facilitates the
merging of regions when a \moc{DIAG} or \moc{SYME} boundary condition is used. 

\end{enumerate}
The \dstr{descPP} structure has the following contents: 

\begin{DataStructure}{Structure \dstr{descPP}}
$[$ \moc{MIX} $\{$  (\dusa{imix}(i),i=1,$n_t$) $[$ \moc{REPEAT} $]~|$\\
$~~~~[[$ \moc{PLANE} \dusa{iplan} $\{$ (\dusa{imix}(i),i=1,\dusa{lp}) $|$ \moc{SAME} \dusa{iplan1}\\
$~~~~|~[[$ \moc{CROWN} $\{$ (\dusa{imix}(i),i=1,\dusa{lc}) $|$ \moc{ALL} \dusa{jmix} $|$ \moc{SAME} \dusa{iplan1} $\}~]]$\\
$~~~~|~[[$ \moc{UPTO} \dusa{ic} \moc{ALL} \dusa{jmix} $|$ \moc{SAME} \dusa{iplan1} $\}~]]~]]~\}$\\
$]$\\
$[$ \moc{HMIX}  (\dusa{ihmix}(i), i=1,$N_t$) $[$ \moc{REPEAT} $]$ $]$\\
$[$ \moc{CELL}  (\dusa{HCELL}(i),i=1,$N_t$) $]$\\
$[$ \moc{MERGE} (\dusa{imerge}(i),i=1,$N_t$) $]$\\
$[$ \moc{TURN}  (\dusa{HTURN}(i),i=1,$N_t$) $]$\\
$[$ \moc{CLUSTER} (\dusa{NAMPIN}(i),i=1,$N_p$) $]$\\
$[$ \moc{MIX-NAMES} (\dusa{NAMMIX}(i),i=1,\dusa{maxmix}) $]$
\end{DataStructure}

\noindent

Here $N_p$ is the number of pin types in the cluster. In addition to the real (physical) mixture \dusa{imix} present in a given region of space and specified by the keyword \moc{MIX}, a virtual mixture \dusa{ihmix} can also be provided using the keyword \moc{HMIX}. This mixture can be used to identify the regions that will be combined in the \moc{EDI:} module to create homogenized region \dusa{ihmix} (see \Sect{EDIData}). Here $N_{t}$
is computed in a way similar to $L_{\rm zones}$ namely
\begin{itemize}

\item \moc{SPHERE} geometry.

$$N_{t}=\textit{lr}$$

The mixtures are then given in the following order
\begin{enumerate}
\item radially outward ($l=1,\textit{lr}$).
\end{enumerate}

\item \moc{TUBE} geometry.

$$N_{t}=\textit{lr}\times\textit{lx}\times \textit{ly}  $$

The mixtures are then given in the following order
\begin{enumerate}
\item radially outward ($l=1,\textit{lr}$) and such that  \dusa{imix} is arbitrary (not used) if radial region $l$ does not intersect Cartesian region $(i,j)$;
\item from surface \moc{X-} to surface \moc{X+} ($i=1,\textit{lx}$ for each $j$);
\item from surface \moc{Y-} to surface \moc{Y+} ($j=1,\textit{ly}$).
\end{enumerate}

\item \moc{TUBEX} geometry.

$$N_{t}=\textit{lr}\times\textit{ly}\times \textit{lz}\times \textit{lx}$$
The mixtures are then given in the following order
\begin{enumerate}
\item radially outward ($l=1,\textit{lr}$) and such that \dusa{imix} is arbitrary (not used) if radial region $l$ does not intersect Cartesian region $(j,k,i)$;
\item from surface \moc{Y-} to surface \moc{Y+} ($j=1,\textit{ly}$ for each $k$ and $i$);
\item from surface \moc{Z-} to surface \moc{Z+} ($k=1,\textit{lz}$ for each $i$);
\item from surface \moc{X-} to surface \moc{X+} ($i=1,\textit{lx}$).
\end{enumerate}

\item \moc{TUBEY} geometry.

$$N_{t}=\textit{lr}\times\textit{lz}\times \textit{lx}\times \textit{ly}$$
The mixtures are then given in the following order
\begin{enumerate}
\item radially outward ($l=1,\textit{lr}$) and such that  \dusa{imix} is arbitrary (not used) if radial region $l$ does not intersect Cartesian region $(k,i,j)$;
\item from surface \moc{Z-} to surface \moc{Z+} ($k=1,\textit{lz}$ for each $i$ and $j$);
\item from surface \moc{X-} to surface \moc{X+} ($i=1,\textit{lx}$ for each $j$);
\item from surface \moc{Y-} to surface \moc{Y+} ($j=1,\textit{ly}$).
\end{enumerate}

\item \moc{TUBEZ} geometry.

$$N_{t}= \textit{lr}\times\textit{lx}\times \textit{ly}\times \textit{lz}$$

The mixtures are then given in the following order
\begin{enumerate}
\item radially outward ($l=1,\textit{lr}$) and such that \dusa{imix} is arbitrary (not used) if radial region $l$ does not intersect Cartesian region $(i,j,k)$;
\item from surface \moc{X-} to surface \moc{X+} ($i=1,\textit{lx}$ for each $j$ and $k$);
\item from surface \moc{Y-} to surface \moc{Y+} ($j=1,\textit{ly}$ for each $k$);
\item from surface \moc{Z-} to surface \moc{Z+} ($k=1,\textit{lz}$).
\end{enumerate}

\item \moc{CAR1D} geometry.

$$N_{t}=\textit{lx}$$

The mixtures are then given in the following order
\begin{enumerate}
\item from surface \moc{X-} to surface \moc{X+} ($i=1,\textit{lx}$).
\end{enumerate}

\item \moc{CAR2D} geometry 
\begin{itemize}
\item without diagonal symmetry. 

$$N_{t}=\textit{lx}\times \textit{ly}$$

The mixtures or cells are then given in the following order
\begin{enumerate}
\item from surface \moc{X-} to surface \moc{X+} ($i=1,\textit{lx}$ for each $j$);
\item from surface \moc{Y-} to surface \moc{Y+} ($j=1,\textit{ly}$).
\end{enumerate}

\item with diagonal symmetry (\moc{X-} and \moc{Y+}). 

$$N_{t}=\frac{\textit{lx}\times (\textit{lx}+1)}{2}$$

The mixtures or cells are then given in the following order
\begin{enumerate}
\item from surface \moc{X-} to surface \moc{X+} ($i=j,\textit{lx}$ for each $j$);
\item from surface \moc{Y-} to surface \moc{Y+} ($j=1,\textit{ly}$).
\end{enumerate}

\item with diagonal symmetry (\moc{X+} and \moc{Y-}). 

$$N_{t}=\frac{\textit{lx}\times (\textit{lx}+1)}{2}$$

The mixtures or cells are then given in the following order
\begin{enumerate}
\item from surface \moc{X-} to surface \moc{X+} ($i=1,j$ for each $j$);
\item from surface \moc{Y-} to surface \moc{Y+} ($j=1,\textit{ly}$).
\end{enumerate}
\end{itemize}

\item \moc{CARCEL} geometries.

$$N_{t}=(\textit{lr}+1)\times\textit{lx}\times \textit{ly} $$

The mixtures are then given in the following order
\begin{enumerate}
\item radially outward ($l=1,\textit{lr}$) and such that \dusa{imix} is arbitrary (not used) if radial region $l$ does not intersect Cartesian region $(i,j)$;
\item $l=\textit{lr+1}$ for the mixture outside the annular regions but inside Cartesian region $(i,j)$;
\item from surface \moc{X-} to surface \moc{X+} ($i=1,\textit{lx}$ for each $j$);
\item from surface \moc{Y-} to surface \moc{Y+} ($j=1,\textit{ly}$).
\end{enumerate}

\item \moc{CAR3D} geometry 
\begin{itemize}
\item without diagonal symmetry. 

$$N_{t}=\textit{lx}\times \textit{ly}\times \textit{lz}$$

The mixtures or cells are then given in the following order
\begin{enumerate}
\item from surface \moc{X-} to surface \moc{X+} ($i=1,\textit{lx}$ for each $j$ and $k$);
\item from surface \moc{Y-} to surface \moc{Y+} ($j=1,\textit{ly}$ for $k$);
\item from surface \moc{Z-} to surface \moc{Z+} ($k=1,\textit{lz}$).
\end{enumerate}

\item with diagonal symmetry (\moc{X-} and \moc{Y+}). 

$$N_{t}=\frac{\textit{lx}\times (\textit{lx}+1)}{2}\times\textit{lz}$$

The mixtures or cells are then given in the following order
\begin{enumerate}
\item from surface \moc{X-} to surface \moc{X+} ($i=j,\textit{lx}$ for each $j$ and $k$);
\item from surface \moc{Y-} to surface \moc{Y+} ($j=1,\textit{ly}$) for each $k$);
\item from surface \moc{Z-} to surface \moc{Z+} ($k=1,\textit{lz}$).
\end{enumerate}
 

\item with diagonal symmetry (\moc{X+} and \moc{Y-}). 

$$N_{t}=\frac{\textit{lx}\times (\textit{lx}+1)}{2}\times\textit{lz}$$

The mixtures or cells are then given in the following order
\begin{enumerate}
\item from surface \moc{X-} to surface \moc{X+} ($i=1,j$ for each $j$ and $k$);
\item from surface \moc{Y-} to surface \moc{Y+} ($j=1,\textit{ly}$ for each $k$);
\item from surface \moc{Z-} to surface \moc{Z+} ($k=1,\textit{lz}$).
\end{enumerate}

\end{itemize}

\item \moc{CARCELX} geometry.

$$N_{t}=(\textit{lr}+1)\times\textit{ly}\times \textit{lz}\times \textit{lx} $$

The mixtures are then given in the following order
\begin{enumerate}
\item radially outward ($l=1,\textit{lr}$) and such that \dusa{imix} is arbitrary (not used) if radial region $l$ does not intersect Cartesian region $(j,k,i)$;
\item $l=\textit{lr+1}$ for the mixture outside the annular regions but inside Cartesian region $(j,k,i)$;
\item from surface \moc{Y-} to surface \moc{Y+} ($j=1,\textit{ly}$ for each $k$ and $i$);
\item from surface \moc{Z-} to surface \moc{Z+} ($k=1,\textit{lz}$ for each $i$);
\item from surface \moc{X-} to surface \moc{X+} ($i=1,\textit{lx}$).
\end{enumerate}

\item \moc{CARCELY} geometry.

$$N_{t}=(\textit{lr}+1)\times\textit{lz}\times \textit{lx}\times \textit{ly}$$

The mixtures are then given in the following order
\begin{enumerate}
\item radially outward ($l=1,\textit{lr}$) and such that \dusa{imix} is arbitrary (not used) if radial region $l$ does not intersect Cartesian region $(k,i,j)$;
\item $l=\textit{lr+1}$ for the mixture outside the annular regions but inside Cartesian region $(k,i,j)$;
\item from surface \moc{Z-} to surface \moc{Z+} ($k=1,\textit{lz}$ for each $i$ and $j$);
\item from surface \moc{X-} to surface \moc{X+} ($i=1,\textit{lx}$ for each $j$);
\item from surface \moc{Y-} to surface \moc{Y+} ($j=1,\textit{ly}$).
\end{enumerate}

\item \moc{CARCELZ} geometries.

$$N_{t}=(\textit{lr}+1)\times\textit{lx}\times \textit{ly}\times \textit{lz}$$

The mixtures are then given in the following order
\begin{enumerate}
\item radially outward ($l=1,\textit{lr}$) and such that \dusa{imix} is arbitrary (not used) if radial region $l$ does not intersect Cartesian region $(i,j,k)$;
\item $l=\textit{lr+1}$ for the mixture outside the annular regions but inside Cartesian region $(i,j,k)$;
\item from surface \moc{X-} to surface \moc{X+} ($i=1,\textit{lx}$ for each $j$ and $k$);
\item from surface \moc{Y-} to surface \moc{Y+} ($j=1,\textit{ly}$ for each $k$).
\item from surface \moc{Z-} to surface \moc{Z+} ($k=1,\textit{lz}$).
\end{enumerate}

\item \moc{HEX} geometry.

$$N_{t}=\textit{lh}$$
The mixtures or cells are then given in the order provided in \Figto{s30}{compl}.

\item \moc{HEXT} geometry.

Three options are possible here:
\begin{itemize}
\item All the triangles in an hexagonal crown have the same mixture. In this case
\begin{align*}N_{t}&=\textit{nhr}\end{align*}
and the real and virtual mixtures are given from each crown starting at the center of the cell.

\item All the triangles in an hexagonal crown in a given sector have the same mixture. In this case
\begin{align*}N_{t}&=6\times \textit{nhr}\end{align*}
and the real and virtual mixtures are given in the following order 
\begin{enumerate}
\item from each crown in sector $j$ starting from the center of the cell;
\item for each sector $j=1,6$.
\end{enumerate}

\item All the triangles contain a different mixture. In this case
\begin{align*}N_{t}&=6\times \textit{nhr}^{2}\end{align*}
and the real and virtual mixtures are given in the following order 
\begin{enumerate}
\item from each triangle $l$ ($l=1,2\times \textit{nhc}-1$) in hexagonal crown $i$ of sector $j$. \Fig{GeoHEXT4} illustrates region and surface ordering in the case where the default value of \dusa{hexmsh} is used and \Fig{GeoHEXT4C} the same information when a different value of \dusa{hexmsh} is provided.
\item from each crown in sector $j$ starting from the center of the cell;
\item for each sector $j=1,6$.
\end{enumerate}
\end{itemize}

\item \moc{HEXCEL} geometries.

$$N_{t}=(\textit{lr}+1)$$

The mixtures are then given in the following order
\begin{enumerate}
\item radially outward ($l=1,\textit{lr}$);
\item $l=\textit{lr+1}$ for the mixture outside the annular regions but inside the hexagonal region.
\end{enumerate}

\item \moc{HEXZ} geometry.

$$N_{t}=\textit{lh}\times \textit{lz}$$

The mixtures or cells are then given in the following order

\begin{enumerate}
\item according to \Figto{s30}{compl} for plane $k$;
\item from surface \moc{Z-} to surface \moc{Z+} ($k=1,\textit{lz}$).
\end{enumerate}

\item \moc{HEXTCEL} geometries.

Three options are possible here:
\begin{itemize}
\item All the triangles in an hexagonal crown have the same mixture. In this case
\begin{align*}N_{t}&=(\textit{lr}+1)\times \textit{nhr}\end{align*}
and the real and virtual mixtures are given in the following order
\begin{enumerate}
\item radially outward ($l=1,\textit{lr}+1$) for each crown ($l=\textit{lr}+1$ is for the part of crown outside the annular regions);
\item from each crown starting from the center of the cell.
\end{enumerate}

\item All the triangles in an hexagonal crown in a given sector have the same mixture. In this case
\begin{align*}N_{t}&=6\times (\textit{lr}+1)\times \textit{nhr}\end{align*}
and the real and virtual mixtures are given in the following order 
\begin{enumerate}
\item radially outward ($l=1,\textit{lr}+1$) for each crown of each sector ($l=\textit{lr}+1$ is for the part of crown outside the annular regions);
\item from each crown in sector $j$ starting from the center of the cell;
\item for each sector $j=1,6$.
\end{enumerate}

\item All the triangles contain a different mixture. In this case
\begin{align*}N_{t}&=6\times (\textit{lr}+1)\times \textit{nhr}^{2}\end{align*}
and the real and virtual mixtures are given in the following order 
\begin{enumerate}
\item radially outward ($l=1,\textit{lr}+1$) for each triangle ($l=\textit{lr}+1$ is for the part of triangle outside the annular regions);
\item from each triangle $l$ ($l=1,2\times \textit{nhc}-1$) in hexagonal crown $i$ of sector $j$. \Fig{GeoHEXT4} illustrates region and surface ordering in the case where the default value of \dusa{hexmsh} is used and \Fig{GeoHEXT4C} the same information when a different value of \dusa{hexmsh} is provided.
\item from each crown in sector $j$ starting from the center of the cell;
\item for each sector $j=1,6$.
\end{enumerate}
\end{itemize}

\item \moc{HEXTZ} geometry.

Three options are again possible here:
\begin{itemize}
\item All the triangles in an hexagonal crown in a plane have the same mixture. In this case
\begin{align*}N_{t}&=\textit{nhr}\times  \textit{lz}\end{align*}
and the real and virtual mixtures are given in the following order
\begin{enumerate}
\item from each crown starting from the center of the cell;
\item from lowest (\moc{Z-}) to highest (\moc{Z+}) plane ($k=1,\textit{lz}$).
\end{enumerate}

\item All the triangles in an hexagonal crown in a given sector in a plane have the same mixture. In this case
\begin{align*}N_{t}&=6\times \textit{nhr}\times \textit{lz}\end{align*}
and the real and virtual mixtures are given in the following order 
\begin{enumerate}
\item from each crown in sector $j$ starting from the center of the cell;
\item for each sector $j=1,6$;
\item from lowest (\moc{Z-}) to highest (\moc{Z+}) plane ($k=1,\textit{lz}$).
\end{enumerate}

\item All the triangles contain a different mixture. In this case
\begin{align*}N_{t}&=6\times \textit{nhr}^{2}\times \textit{lz}\end{align*}
and the real and virtual mixtures are given in the following order 
\begin{enumerate}
\item from each triangle $l$ ($l=1,2\times \textit{nhc}-1$) in hexagonal crown $i$ of sector $j$. \Fig{GeoHEXT4} illustrates region and surface ordering in the case where the default value of \dusa{hexmsh} is used and \Fig{GeoHEXT4C} the same information when a different value of \dusa{hexmsh} is provided.
\item from each crown in sector $j$ starting from the center of the cell;
\item for each sector $j=1,6$;
\item from lowest (\moc{Z-}) to highest (\moc{Z+}) plane ($k=1,\textit{lz}$).
\end{enumerate}
\end{itemize}


\item \moc{HEXCELZ} geometries.

$$N_{t}=(\textit{lr}+1)\times \textit{lz}$$

\item \moc{HEXTCELZ} geometries.

Three options are possible here:
\begin{itemize}
\item All the triangles in an hexagonal crown have the same mixture. In this case
\begin{align*}N_{t}&=(\textit{lr}+1)\times \textit{nhr}\times \textit{lz}\end{align*}
and the real and virtual mixtures are given in the following order
\begin{enumerate}
\item radially outward ($l=1,\textit{lr}+1$) for each crown ($l=\textit{lr}+1$ is for the part of crown outside the annular regions);
\item from each crown starting from the center of the cell;
\item from lowest (\moc{Z-}) to highest (\moc{Z+}) plane ($k=1,\textit{lz}$).
\end{enumerate}

\item All the triangles in an hexagonal crown in a given sector have the same mixture. In this case
\begin{align*}N_{t}&=6\times (\textit{lr}+1)\times \textit{nhr}\times \textit{lz}\end{align*}
and the real and virtual mixtures are given in the following order 
\begin{enumerate}
\item radially outward ($l=1,\textit{lr}+1$) for each crown of each sector ($l=\textit{lr}+1$ is for the part of crown outside the annular regions);
\item from each crown in sector $j$ starting from the center of the cell;
\item for each sector $j=1,6$;
\item from lowest (\moc{Z-}) to highest (\moc{Z+}) plane ($k=1,\textit{lz}$).
\end{enumerate}

\item All the triangles contain a different mixture. In this case
\begin{align*}N_{t}&=6\times (\textit{lr}+1)\times \textit{nhr}^{2}\times \textit{lz}\end{align*}
and the real and virtual mixtures are given in the following order 
\begin{enumerate}
\item radially outward ($l=1,\textit{lr}+1$) for each triangle ($l=\textit{lr}+1$ is for the part of triangle outside the annular regions);
\item from each triangle $l$ ($l=1,2\times \textit{nhc}-1$) in hexagonal crown $i$ of sector $j$. \Fig{GeoHEXT4} illustrates region and surface ordering in the case where the default value of \dusa{hexmsh} is used and \Fig{GeoHEXT4C} the same information when a different value of \dusa{hexmsh} is provided.
\item from each crown in sector $j$ starting from the center of the cell;
\item for each sector $j=1,6$.
\item from lowest (\moc{Z-}) to highest (\moc{Z+}) plane ($k=1,\textit{lz}$).
\end{enumerate}
\end{itemize}

\end{itemize}

The mixtures are then given in the following order
\begin{enumerate}
\item radially outward ($l=1,\textit{lr}$) for plane $k$;
\item $l=\textit{lr+1}$ for the mixture outside the annular regions but inside the hexagonal region on plane $k$;
\item from surface \moc{Z-} to surface \moc{Z+} ($k=1,\textit{lz}$).
\end{enumerate}

\begin{figure}[h!]
\begin{center} 
\epsfxsize=8cm
\centerline{ \epsffile{Goricart.eps}}
\parbox{14cm}{\caption{Description of the various rotations allowed for
Cartesian geometries}\label{fig:oricart}}   
\end{center}  
\end{figure}

\begin{figure}[h!]  
\begin{center} 
\epsfxsize=11cm
\centerline{ \epsffile{Gorihex.eps}}
\parbox{14cm}{\caption{Description of the various rotation allowed for
hexagonal geometries}\label{fig:orihex}}
\end{center}  
\end{figure}

\begin{figure}[h!]  
\begin{center} 
\epsfxsize=7cm
\centerline{ \epsffile{Gcluster.eps}}
\parbox{14cm}{\caption{Typical cluster geometry}\label{fig:cluster}}
\end{center}  
\end{figure}

\clearpage

The inputs associated with this structure have the following meaning:

\begin{ListeDeDescription}{mmmmmm}

\item[\moc{MIX}] keyword to specify the isotopic mixture number or
sub-geometry associated
with each region inside the geometry. When diagonal symmetries are considered,
only the mixture associated with regions inside the symmetrized geometry need to
be specified. When a sub-geometry is located inside symmetrized geometry but
outside the calculation region it must be declared {\sl virtual} (for example,
the corners of a nuclear reactor). 

\item[\dusa{imix}] array of $n_{t}\le N_t$ integers {\sl or} character variables associated
with each region. An integer is a mixture number associated with a region
\dusa{imix}$\le$\dusa{maxmix} (see \Sectand{MACData}{LIBData}). If
\dusa{imix}=0, the corresponding volume is replaced by a void region. If
\dusa{imix} is a character variable, it is replaced by the corresponding
sub-geometry or {\sl generating cell}. These values must be specified in
the following order for most geometries: 

\begin{enumerate}
\item radially from the inside out. 
\item from surface \moc{X-} to surface \moc{X+}
\item from surface \moc{Y-} to surface \moc{Y+}
\item from surface \moc{Z-} to surface \moc{Z+}
\end{enumerate}

In the cases where a \moc{CARCELX} and a \moc{TUBEX} geometry are defined then we will use 

\begin{enumerate}
\item radially from the inside out ($lr+1$ mixtures for \moc{CARCELX} and $lr$ for \moc{TUBEX}). 
\item from surface \moc{Y-} to surface \moc{Y+}
\item from surface \moc{Z-} to surface \moc{Z+}
\item from surface \moc{X-} to surface \moc{X+}
\end{enumerate}

Finally, for a \moc{CARCELY} and \moc{TUBEY} geometry are defined the following order is considered:

\begin{enumerate}
\item radially from the inside out ($lr+1$ mixtures for \moc{CARCELY} and $lr$ for \moc{TUBEY}) 
\item from surface \moc{Z-} to surface \moc{Z+}
\item from surface \moc{X-} to surface \moc{X+}
\item from surface \moc{Y-} to surface \moc{Y+}
\end{enumerate}

In the cases where a sectorized cell geometry is defined, \dusa{imix} must
be defined in each sector, following the order shown in \Figand{rect3}{hexa3}.
Also note that \dusa{imix} is {\sl not affected} by the values of the
mesh-splitting indices \dusa{ispltx}, \dusa{isplty}, \dusa{ispltz}
or \dusa{ispltr}.

\item[\moc{REPEAT}] keyword to specify the previous list of mixtures will be repeated. This is valid only when $N_t/n_t$ 
is an integer. If this keyword is absent and $n_t < N_t$, then the missing mixtures will be replaced 
with void (\dusa{imix}(i) $=0$).

\item[\moc{PLANE}] keyword to attribute mixture numbers to each volume inside a single 2-D plane. This option is 
valid only for 3-D geometries, Cartesian or hexagonal. 

\item[\dusa{iplan}] plane number for which material mixture are input. 

\item[\moc{SAME}] keyword to attribute the same material mixture numbers of the \dusa{iplan1} plane to the \dusa{iplan} plane. In 
hexagonal geometry, it can indicate that the mixture numbers of the current crown of the \dusa{iplan}th 
plane will be identical to those of the same crown of the \dusa{iplan1}th plane. 

\item[\dusa{iplan1}] plane number used as reference to input the current plane or crown(s). 

\item[\dusa{lp}] number of volumes in a plane. In Cartesian geometry, $lp=lx*ly$ and in hexagonal geometry, 
$lp=lh$. 

\item[\moc{CROWN}]  keyword to attribute mixture numbers to each hexagon of a single crown. This option is only 
valid for \moc{COMPLETE} hexagonal geometry definition. Each use of the keyword \moc{CROWN} increases 
the crown number by 1. So it is not required to give its number, but crowns must be defined from 
the center to the peripherical regions of a plane. 

\item[\dusa{lc}] number of hexagons in the current crown. For the \dusa{i}th crown of a compelete hexagonal plane, 
$lc=(i-1)*6$. The first crown is composed of only one hexagon. 

\item[\moc{ALL}] keyword to specify that the \dusa{lc} material mixture number of the current crown have the same value 
\dusa{jmix}. 

\item[\moc{UPTO}] keyword to attribute material mixture numbers of the current crown up to the \dusa{ic} one. 

\item[\dusa{ic}] number of the last crown in \moc{UPTO} option. Its value must be greater than equal to the current 
crown number.

\item[\moc{HMIX}] keyword to specify the virtual isotopic mixture associated with each region inside the geometry. These
virtual mixtures will be produced by homogenization in the {\tt EDI:} module (see \Sect{descedi}).

\item[\moc{CELL}] keyword to specify the location of the sub-geometry called
{\sl generating cells} in a Cartesian or hexagonal geometry.

\item[\dusa{HCELL}] array of sub-geometry {\tt character*12} names which will
be superimposed upon the current Cartesian geometry. The same sub-geometry may
appear in different positions within the global geometry if the material
properties and dimensions are identical. The concept of sub-geometry is useful
for the interface current method in a SYBIL calculation since the collision
probability matrix associated with each sub-geometry is computed independently
of its location in the geometry. In general, the neutron fluxes in identical
sub-geometry located at different locations will be different even if they are
associated with the same collision probability matrix. These sub-geometry names
must be specified in the following order: 

\begin{enumerate}
\item from surface \moc{X-} to surface \moc{X+}
\item from surface \moc{Y-} to surface \moc{Y+}
\item from surface \moc{Z-} to surface \moc{Z+}
\end{enumerate}

\item[\moc{MERGE}] keyword to specify that some sub-geometries or regions must
be merged. 

\item[\dusa{imerge}] array of numbers that associate a global sub-geometry or
region number with each sub-geometry or region. All the sub-geometries  or
regions with the same global number will be attributed the same flux.

\item[\moc{TURN}] keyword to specify that some sub-geometries must be rotated
in space before being located at a specific position.

\item[\dusa{HTURN}] array of  {\tt character*1} keywords to rotate
conveniently each sub-geometry. The letters {\tt A} to {\tt L} are used as
keywords to specify these rotation. For Cartesian geometries, the eight possible
orientations are shown in \Fig{oricart} while for hexagonal geometries
the permitted orientations are shown in \Fig{orihex}. For 3-D cells, the
same letters can be used to describe the rotation in the $X-Y$ plane. However,
an additional $-$ sign can be glued to the 2-D rotation identifier to
indicate reflection of the cell along the $Z$-axis ({\tt -A} to {\tt -L}).

\item[\moc{CLUSTER}] keyword to specify that pin (cylindrical) sub-geometry
will be inserted in the geometry (see \Fig{cluster}). 

\item[\dusa{NAMPIN}] array of cylindrical sub-geometry {\tt character*12} name 
representing a pin. This sub-geometry must be of type \moc{TUBE}, \moc{TUBEX},
\moc{TUBEY} or \moc{TUBEZ}.

\item[\moc{MIX-NAMES}] keyword to specify character names to material mixtures.
By default, the material mixtures are not named.

\item[\dusa{NAMMIX}] array of {\tt character*12} names for the material
mixtures.

\end{ListeDeDescription}

\clearpage

\subsubsection{Double-heterogeneity}\label{sect:descDH}

The structure \dstr{descDH} provides the possibility to define a stochastic mixture of cylindrical or spherical micro-structures that can be distributed inside {\sl composite mixtures} of the current {\sl macro-geometry}. A composite mixture is represented by a {\sl material mixture index} with a value greater than \dusa{maxmix}, the maximum number of real mixtures. Each micro-structure can be composed of many micro-volumes.\cite{BIHET}

\begin{DataStructure}{Structure \dstr{descDH}}
$[$ \moc{BIHET}  $\{$ \moc{TUBE} $|$ \moc{SPHE} $\}$ \dusa{nmistr}
\dusa{nmilg} \\
\hskip 1.0cm (\dusa{ns}(i),i=1,\dusa{nmistr}) \\
\hskip 1.0cm((\dusa{rs}(i,j),j=1,\dusa{ns}(i)+1),i=1,\dusa{nmistr})\\
\hskip 1.0cm(\dusa{milie}(i),i=1,\dusa{nmilg})\\
\hskip 1.0cm(\dusa{mixdil}(i),i=1,\dusa{nmilg})\\
\hskip 1.0cm( (\dusa{fract}(i,j),j=1,\dusa{nmistr}) 
( $[$(\dusa{mixgr}(i,j,k),k=1,\dusa{ns}(j))$]$,j=1,\dusa{nmistr}), i=1,\dusa{nmilg}) $]$
\end{DataStructure}

\noindent where
\begin{ListeDeDescription}{mmmmmmm}

\item[\moc{BIHET}] keyword to specify that the current macro-geometry is containing composite mixtures.

\item[\moc{TUBE}] keyword to specify that the micro-structures are of a
cylindrical geometry;

\item[\moc{SPHE}] keyword to specify that the micro-structures are of a
spherical geometry.

\item[\dusa{nmistr}] maximum number of micro-structure types in the composite mixtures. Each type of
micro-structure is characterized by its dimension and may have distinct
volumetric concentrations in each of the macro-geometry volumes. All the
micro-structures of a given type have the same nuclear properties in a given
macro-volume. The micro-structures of a given type may have different nuclear
properties within different macro-volumes.

\item[\dusa{nmilg}] number of composite mixtures. This is the number of material mixture indices of the macro-geometry with a value $>$\dusa{maxmix}.

\item[\dusa{ns}] array giving the number of sub-regions (tubes or spherical
shells) in the  micro-structures. Each type of micro-structures may contain a
different number of micro-volumes.

\item[\dusa{rs}] array giving the radius of the tubes or spherical shells
making up the micro-structures. For each type of micro structure $i$, we will
have an initial radius of \dusa{rs}$(1,i)=0.0$.

\item[\dusa{milie}] array giving the indices used to defined composite mixtures in the macro-geometry. These composite mixture indices must be $>$\dusa{maxmix}.

\item[\dusa{mixdil}] array giving the mixture indices associated with the diluent in each composite mixtures of the macro-geometry.  These values must be $\le$\dusa{maxmix}.

\item[\dusa{fract}] array of volumetric concentration ($V_{G}/V_{R}$) of
each micro-structures (volume $V_{G}$) in a given region (volume $V_{R}$) of the
macro-geometry.  

\item[\dusa{mixgr}] array giving the mixture index associated with each
region of the micro-structures. Note that \dusa{mixgr} should be specified only
for the regions of the micro-structure which have a concentration
\dusa{fract}$>$0. These values must be $\le$\dusa{maxmix}.

\end{ListeDeDescription}

Examples of geometry definitions can be found in \Sect{ExGEOData}.

\subsubsection{Do-it-yourself geometries}\label{sect:descSIJ}

A {\sl do-it-yourself} geometry is an abstract representation of an assembly of arbitrary unit-cells defined in term of their probability of presence and of their probability to have a particular neighbor. Structure \dstr{descSIJ} is defined as

\begin{DataStructure}{Structure \dstr{descSIJ}}
$[$ \moc{POURCE} (\dusa{pcinl}(i),i=1,\dusa{lp}) $]$\\
$[$ \moc{PROCEL} ((\dusa{pijcel}(i,j),j=1,\dusa{lp}),i=1,\dusa{lp}) $]$
\end{DataStructure}

\noindent where
\begin{ListeDeDescription}{mmmmmmm}

\item[\moc{POURCE}] keyword to specify that a {\sl do-it-yourself} type
geometry is to be defined, that is to say a geometry resembling the multicell
geometry seen in APOLLO-1.\cite{apollo1} This option permits the interactions
between different arbitrarily arranged cells in an infinite lattice to be
treated. The cells are identified by the information
following the keyword \moc{CELL}. The user must ensure that the total number of
regions appearing in all the cells must be less than \dusa{maxreg}.

\item[\dusa{pcinl}] array giving the proportion of each cell type in the
lattice such that:

$$|\sum_{i=1}^{{\it lp}}{\it pcinl}(i)-1.|<10^{-5}$$

\item[\moc{PROCEL}] keyword to specify that in a {\sl do-it-yourself} type
geometry rather than using a statistical arrangement of cells, a pre-calculated
cell distribution is to be considered. If the \moc{POURCE} structure is
given without the \moc{PROCEL} structure, a {\sl statistical} approximation
is used, as defined in Ref.~\citen{apollo1}.

\item[\dusa{pijcel}] array giving the pre-calculated probability for a neutron
leaving a cell of type i to enter a cell of type j without crossing any other
cell. We require: 

$$|S(i){\it pcinl}(i){\it pijcel}(i,j)-S(j){\it pcinl}(j) {\it
pijcel}(j,i)|<10^{-4}$$

\noindent where $S(i)$ and $S(j)$ are the exterior surfaces area of the cells of
type $i$ and $j$ respectively.

\end{ListeDeDescription}

Examples of geometry definitions can be found in \Sect{ExGEOData}.

\eject
 % structure (dragonG)
\subsection{The tracking modules}\label{sect:TRKData}

A tracking module is required to analyze a spatial domain (geometry) assuming
a specific algorithm will be used for the collision probability or method of characteristics
calculations.  It performs zone numbering operations, volume and surface area
calculations and generates the required  integration lines for a geometry that
was previously defined in the \moc{GEO:} module. These operations are carried
out differently depending on the solution algorithm used.

\vskip 0.15cm

Many different operators are available for tracking in DRAGON. The \moc{SYBILT:} module
is used for 1--D geometries (either plane, cylindrical or spherical) and
interface current tracking inside heterogeneous blocks. The \moc{EXCELT:} module
is used to perform full cell collision probability tracking with
isotropic\cite{DragonPIJI,Mtl93a} or specular\cite{DragonPIJS1,Mtl93b}
surface current. The \moc{NXT:} module is an extension of the \moc{EXCELT:}
module to more complex geometry including assemblies of clusters in two and
three dimensions.\cite{ige260}  The \moc{MCCGT:} module is an implementation of the open
characteristics method of I.~R.~Suslov.\cite{mccg,suslov2}. These are the transport
tracking modules which can be used everywhere in the code where tracking
information needs to be generated. The \moc{SNT:} module is an implementation of
the discrete ordinates (or $S_N$) method in 1-D/2-D/3-D geometries.
The module \moc{BIVACT:} is used to perform a finite-element (diffusion or SP$_n$) 1-D/2-D
tracking which may be required for diffusion synthetic acceleration (DSA) or homogenization
purposes.\cite{BIVAC} The final module \moc{TRIVAT:} is used to perform a finite-element
1-D/2-D/3-D tracking which may be required for DSA or homogenization purposes.\cite{TRIVAC}

\vskip 0.15cm

None of these modules can analyze all of the geometry available in the code
DRAGON. In general, the restrictions that apply to a given tracking module
result directly from the approximation associated with this method. Moreover, in
other instances, some geometries which would have had the same tracking file
generated by two different method, such as tube geometry for the \moc{SYBILT:}
and \moc{EXCELT:} module, have been made available only to one of these tracking
module (module \moc{SYBILT:} in this case).

\vskip 0.15cm

The general information resulting from these
tracking is stored in a \dds{tracking} data structure.
For the \moc{EXCELT:} and \moc{NXT:} modules, an additional sequential binary
tracking file may be generated.

\vskip 0.15cm

The global numbering of the zones in a geometry proceeds following an
order of priorities given by:

\begin{itemize}

\item the different rings of a cylindrical or spherical region starting with the
inner most after mesh splitting;

\item for a cluster regions located in a ring, two different numbering schemes are possible. For the \moc{EXCELT:} 
module, one first numbers the region inside the pin in the same way as for cylindrical regions and finishes 
by associating the next region number to the shell of the global geometry which contains this pin. If two 
cluster types are located in a given ring, they are classified according to increasing \dusa{rpin} and \dusa{apin} and then 
numbered in this order. Cluster overlapping annular region are numbered before considering the annular 
regions. For the \moc{NXT:} module, each pin is numbered individually in a Cartesian region according to their 
ordered in the \moc{CLUSTER} keywords and then the Cartesian regions are numbered sequentially. A description 
of the explicit numbering of regions and surfaces can be found in report IGE-260.\cite{ige260}
 
\item the zones in ascending order corresponding to the first axial component
(normally $X$) after mesh splitting;

\item the zones in ascending order corresponding to the second axial component
(normally $Y$) after mesh splitting;

\item the hexagonal zones corresponding to the order described in
\Fig{s30} to \Fig{compl}.

\item  the sub-geometry of type \moc{CARCELX}, \moc{CARCELY} and
\moc{CARCELZ} are numbered assuming that the third component corresponds to 
$X$, $Y$ and $Z$ respectively.

\end{itemize}
 
We should also note that symmetry conditions implicitly force the grouping of
certain calculation zones.

\vskip 0.2cm

All the tracking operators of DRAGON share an identical general tracking data
structure defined as

\begin{DataStructure}{Structure \dstr{desctrack}}
$[$ \moc{EDIT} \dusa{iprint} $]$\\
$[$ \moc{TITL} \dusa{TITLE} $]$ \\
$[$ \moc{MAXR} \dusa{maxreg} $]$\\
$[$ $\{$ \moc{NORE} $|$ \moc{RENO} $|$ \moc{REND} $\}$ $]$
\end{DataStructure}

\noindent with

\begin{ListeDeDescription}{mmmmmmm}

\item[\moc{EDIT}] keyword used to modify the print level \dusa{iprint}.

\item[\dusa{iprint}] index used to control the printing of this operator. The
amount of output produced by this tracking operators will vary substantially
depending on the print level specified. For example,

\begin{itemize}

\item when \dusa{iprint}=0 no output is produced;

\item when \dusa{iprint}=1 a minimum amount of output is produced; the 
main geometry properties are printed (fixed default option);

\item when \dusa{iprint}$\ge$2 In addition to the information printed when
using \dusa{iprint}=1 the zone numbering (zones associated with a flux) is
printed;

\end{itemize}

\item[\moc{TITL}] keyword which allows the run title to be set.

\item[\dusa{TITLE}] the title associated with a DRAGON run. This
title may contain up to 72 characters. The default when \moc{TITL}  is not
specified is no title.

\item[\moc{MAXR}] keyword which permits the maximum number of regions to be
considered during a DRAGON run to be specified.

\item[\dusa{maxreg}] maximum dimensions of the problem to be considered.  The
default value is set to the number of regions previously computed by the
\moc{GEO:} module. However this value is generally insufficient if symmetries or
mesh-splitting are specified.

\item[\moc{NORE}] keyword to specify that the automatic normalization of the integration lines is deactivated.

\item[\moc{RENO}]  keyword to specify the activation of the {\sl direction-independent} normalization procedure of the
integration lines. The normalization factors are {\sl not} function of the subtracks directions. This option is only
valid for modules \moc{NXT:}, \moc{EXCELT:} and \moc{SALT:}. This is the default option for \moc{NXT:} and \moc{SALT:}
modules.

\item[\moc{REND}]  keyword to specify the activation of the {\sl direction-dependent} normalization procedure of the
integration lines. The normalization factors are function of the subtracks directions. This option is only valid for
modules \moc{NXT:}, \moc{EXCELT:} and \moc{SALT:}. This is the default option for \moc{EXCELT:} module.

\end{ListeDeDescription}
\eject

\subsubsection{The {\tt SYBILT:} tracking module}\label{sect:SYBILData}

The {\tt SYBILT:} module provides an implementation of the collision probability (PIJ) method in 1D geometries or of the interface current (IC) method
in 2D geometries. The geometries that can be treated by the module \moc{SYBILT:} are

\begin{enumerate}

\item The homogeneous geometry \moc{HOMOGE}.

\item The one-dimensional geometries \moc{SPHERE}, \moc{TUBE} and
\moc{CAR1D}.\cite{ALCOL}

\item The two-dimensional geometries \moc{CAR2D} and \moc{HEX} including
respectively \moc{CARCEL} and \moc{HEXCEL} sub-geometries as well as 
\moc{VIRTUAL}
sub-geometries. 

\item $S_{ij}$--type two-dimensional non-standard geometries.\cite{Apollo}

\item The double heterogeneity option.\cite{BIHET}

\end{enumerate}

The calling specification for this module is:

\begin{DataStructure}{Structure \dstr{SYBILT:}}
\dusa{TRKNAM}
\moc{:=} \moc{SYBILT:} $[$ \dusa{TRKNAM} $]$
\dusa{GEONAM} \moc{::} \dstr{desctrack} \dstr{descsybil}
\end{DataStructure}

\noindent  where
\begin{ListeDeDescription}{mmmmmmm}

\item[\dusa{TRKNAM}] {\tt character*12} name of the \dds{tracking} data
structure that will contain region volume and surface area vectors in
addition to region identification pointers and other tracking information.
If \dusa{TRKNAM} also appears on the RHS, the previous tracking 
parameters will be applied by default on the current geometry.

\item[\dusa{GEONAM}] {\tt character*12} name of the \dds{geometry} data
structure.

\item[\dstr{desctrack}] structure describing the general tracking data (see
\Sect{TRKData})

\item[\dstr{descsybil}] structure describing the transport tracking data
specific to \moc{SYBILT:}.

\end{ListeDeDescription}

\vskip 0.15cm

The \moc{SYBILT:} specific tracking data in \dstr{descsybil} is defined as

\begin{DataStructure}{Structure \dstr{descsybil}}
$[$ \moc{MAXJ} \dusa{maxcur} $]$  $[$ \moc{MAXZ} \dusa{maxint} $]$ \\
$[$ \moc{HALT} $]$ \\
$[$ \moc{QUA1} \dusa{iqua1} $]$ $[$ \moc{QUA2} \dusa{iqua2}
\dusa{nsegment} $]$ $[$ $\{$ \moc{EQW} $|$ \moc{GAUS} $\}$ $]$ \\
$[$ $\{$ \moc{ROTH} $|$ \moc{ROT+} $|$ \moc{DP00} $|$ \moc{DP01} $\}$ $]$ \\
$[$ $\{$ \moc{WIGN} $|$ \moc{ASKE} $|$ \moc{SANC} $\}$ $]$ $[$ \moc{LIGN} $]$
$[$ \moc{RECT} $]$ \\
$[$ \moc{EPSJ} \dusa{epsj} $]$ \\
$[~[$ \moc{QUAB} \dusa{iquab} $]~[~\{$ \moc{SAPO} $|$ \moc{HEBE} $|$ \moc{SLSI} $[$ \dusa{frtm} $]~\}~]~]$ \\
{\tt ;}
\end{DataStructure}

\noindent where

\begin{ListeDeDescription}{mmmmmmm}

\item[\moc{MAXJ}] keyword to specify the maximum number of interface currents
surrounding the blocks in the calculations. 

\item[\dusa{maxcur}] the maximum number of interface currents surrounding the
blocks. The default value is \dusa{maxcur}=max(18,4$\times$\dusa{maxreg}) for the
\moc{SYBILT:} module.

\item[\moc{MAXZ}] keyword to specify the maximum amount of memory required to
store the integration lines. An insufficiently large value can lead to an
execution failure (core dump).

\item[\dusa{maxint}] the maximum amount of memory required to store the
integration lines. The default value is \dusa{maxint}=10000.

\item[\moc{HALT}] keyword to specify that the program is to be stopped at the
end of the geometry calculations. This option permits the geometry inputs to be
checked, the number of blocks and interface currents to be calculated, and a
conservative estimate of the memory required for storing the tracks to be made
for mixed geometries.

\item[\moc{QUA1}] keyword to specify the one-dimensional integration
parameters.

\item[\dusa{iqua1}] number of basis points for the angular integration of the
blocks in a one-dimensional geometry. This parameter is not used for
\moc{CAR1D} geometries. If a Gauss-Legendre or Gauss-Jacobi quadrature is used,
the values of \dusa{iqua1} allowed are: 1 to 20, 24, 28, 32 or 64. The default
value is \dusa{iqua1}=5. 

\item[\moc{QUA2}] keyword to specify the two-dimensional integration
parameters.

\item[\dusa{iqua2}] number of basis points for the angular integration of the
blocks in a two-dimensional geometry appearing during assembly  
calculations. If a Gauss-Legendre or Gauss-Jacobi formula is used the values
allowed for \dusa{iqua2} are: 1 to 20, 24, 28, 32 or 64. The default value is
\dusa{iqua2}=3 and represents the number of angles in ($0,\pi/4$) for
Cartesian geometries and  ($0,\pi/6$) for hexagonal geometries. 

\item[\dusa{nsegment}] number of basis points for the spatial integration of
the blocks in a two-dimensional geometry appearing during assembly 
calculations. The values of \dusa{nsegment} allowed are: 1 to 10. The default
value is \dusa{nsegment}=3.

\item[\moc{EQW}] keyword to specify the use of equal-weight quadrature.

\item[\moc{GAUS}] keyword to specify the use of the Gauss-Legendre or the
Gauss-Jacobi quadrature. This is the default option.

\item[\moc{ROTH}] keyword to specify that the isotropic ($DP_{0}$) components
of the inter-cell current is used with the incoming current being averaged over
all the faces surrounding a cell. The global collision matrix is calculated in a
annular model. Only used when 2--d assembly of cells are considered.

\item[\moc{ROT+}] keyword to specify that the isotropic ($DP_{0}$) components
of the inter-cell current is used. The global collision matrix is calculated in
a annular model. Only used when 2--d assembly of cells are considered.

\item[\moc{DP00}] keyword to specify that the isotropic ($DP_{0}$) components
of the inter-cell current is used. The global collision matrix are computed
explicitly. Only used when 2--d assembly of cells are considered.

\item[\moc{DP01}] keyword to specify that the linearly anisotropic ($DP_{1}$)
components of the inter-cell current are used. This hypothesis implies 12
currents per cell in a cartesian geometry and 18 currents per cell for an
hexagonal geometry. Linearly anisotropic reflection is used. Only used when 2--d
assembly of cells are considered.

\item[\moc{WIGN}] keyword to specify the use of a {\sl Wigner} cylinderization
which preserves the volume of the external crown. This applies only in cases
where the external surface is annular using the \moc{ROTH} or \moc{ROT+}
options. Only used when 2--d assembly of cells are considered. Note that an
assembly of rectangular cells having unequal volumes cannot use a {\sl Wigner}
cylinderization.  

\item[\moc{ASKE}] keyword to specify the use of an {\sl Askew} cylinderization
which preserves both the external surface of the cells and the material balance
of the external crown (by a modification of its concentration). This applies
only in cases where the external surface is annular using the \moc{ROTH} or
\moc{ROT+} options. Only used when 2--d assembly of cells are considered. Note
that an assembly of rectangular cells having unequal volumes can use an
{\sl Askew} cylinderization.  

\item[\moc{SANC}] keyword to specify the use of a {\sl Sanchez} cylinderization.
This model uses a {\sl Wigner} cylinderization for computing the collision $P_{ij}$
and leakage $P_{iS}$ probabilities. However, the reciprocity and conservation
relations used to compute the incoming $P_{Sj}$ and transmission $P_{SS}$
probabilities are defined in the rectangular cell (with the exact
surface).\cite{SANCHEZ} 
This applies where the external surface is annular using the \moc{ROTH} or
\moc{ROT+} options. Only used when 2--d assembly of cells are considered. Note
that an assembly of rectangular cells having unequal volumes can use a
{\sl Sanchez} cylinderization. This is the default option.

\item[\moc{LIGN}] keyword to specify that all the integration lines are to be
printed. This option should only be used when absolutely necessary because it
generates a rather large amount of output. Only used when 2--d assembly of cells
are considered.

\item[\moc{RECT}] keyword to specify that square cells are to be treated as if
they were rectangular cells, with the inherent loss in performance that this
entails. This option is of purely academic interest.

\item[\moc{EPSJ}] keyword to specify the stopping criterion for the flux-current iterations of the
interface current method in case the {\tt ARM} keyword is set in the {\tt ASM:} module or in
a resonance self-shielding module ({\tt SHI:}, {\tt USS:}, etc.).

\item[\dusa{epsj}] the stopping criterion value. The default value is \dusa{epsj} $= 0.5 \times 10^{-5}$.

\item[\moc{QUAB}] keyword to specify the number of basis point for the
numerical integration of each micro-structure in cases involving double
heterogeneity (Bihet).

\item[\dusa{iquab}] the number of basis point for the numerical integration of
the collision probabilities in the micro-volumes using the  Gauss-Jacobi
formula. The values permitted are: 1 to 20, 24, 28, 32 or  64. The default value
is \dusa{iquab}=5. If \dusa{iquab} is negative, its absolute value will be used in the She-Liu-Shi approach to determine the
split level in the tracking used to compute the probability collisions.

\item[\moc{SAPO}] use the Sanchez-Pomraning double-heterogeneity model.\cite{sapo}

\item[\moc{HEBE}] use the Hebert double-heterogeneity model (default option).\cite{BIHET}

\item[\moc{SLSI}] use the She-Liu-Shi double-heterogeneity model without shadow effect.\cite{She2017}

\item[\dusa{frtm}] the minimum microstructure volume fraction used to compute the size of the equivalent cylinder in She-Liu-Shi approach. The default value is \dusa{frtm} $=0.05$.

\end{ListeDeDescription}
\eject
 % structure (sybilT)
\subsubsection{The {\tt EXCELT:} tracking module}\label{sect:EXCELLData}

The calling specification for this module is:

\begin{DataStructure}{Structure \dstr{EXCELT:}}
\dusa{TRKNAM} $[$ \dusa{TRKFIL} $]$
\moc{:=} \moc{EXCELT:} $[$ \dusa{TRKNAM} $]$ $[$ \dusa{TRKFIL} $]$ 
\dusa{GEONAM} \moc{::}  \dstr{desctrack} \dstr{descexcel}
\end{DataStructure}

\noindent  where
\begin{ListeDeDescription}{mmmmmmm}

\item[\dusa{TRKNAM}] {\tt character*12} name of the \dds{tracking} data
structure that will contain region volume and surface area vectors in
addition to region identification pointers and other tracking information.
If \dusa{TRKNAM} also appears on the RHS, the previous tracking 
parameters will be applied by default on the current geometry.

\item[\dusa{TRKFIL}] {\tt character*12} name of the sequential binary tracking
file  used to store the tracks lengths. If \dusa{TRKFIL} does not appear, the keyword
\moc{XCLL} is set automatically. If the user wants to use a tracking file,
\dusa{TRKFIL} is required for the \moc{EXCELT:} module, either on the LHS, on the RHS or on both sides. In
the case where \dusa{TRKFIL} appears on both LHS and RHS, the existing tracking
file is modified by the module while if \dusa{TRKFIL} appears only on the RHS,
the existing tracking file is read but not modified.

\item[\dusa{GEONAM}] {\tt character*12} name of the \dds{geometry} data
structure.

\item[\dstr{desctrack}] structure describing the general tracking data (see
\Sect{TRKData})

\item[\dstr{descexcel}] structure describing the transport tracking data
specific to \moc{EXCELT:}.

\end{ListeDeDescription}

\vskip 0.15cm

The \moc{EXCELT:} specific tracking data in \dstr{descexcel} is defined as

\begin{DataStructure}{Structure \dstr{descexcel}}
$[$ \moc{ANIS} \dusa{nanis} $]$ \\
$[~\{$ \moc{ONEG} $|$ \moc{ALLG} $[$ \moc{BATCH} \dusa{nbatch} $]~|$ \moc{XCLL} $\}~]$ \\
$[~\{$ \moc{TREG}  $|$ \moc{TMER} $\}~]$ \\
$[$ $\{$ \moc{PISO} $|$ \moc{PSPC} $[$ \moc{CUT} \dusa{pcut} $]$ $\}$ $]$ \\
$[~[$ \moc{QUAB} \dusa{iquab} $]~[~\{$ \moc{SAPO} $|$ \moc{HEBE} $|$ \moc{SLSI} $[$ \dusa{frtm} $]~\}~]~]$ \\
$[$ $\{$ \moc{PRIX} $|$  \moc{PRIY} $|$ \moc{PRIZ} $\}$ \dusa{denspr} $]$ \\
$[$ $\{$ \moc{LCMD} $|$ \moc{OPP1} $|$ \moc{OGAU} $|$ \moc{GAUS} $|$ \moc{CACA} $|$ \moc{CACB} $\}~[$ \dusa{nmu} $]~]$ \\
$[$ \moc{TRAK}  $\{$  \moc{TISO} \dusa{nangl} $[$ \dusa{nangl\_z} $]$ \dusa{dens} $[$ \dusa{dens\_z} $]~[$ \moc{CORN} 
\dusa{pcorn} $]$  $[$ \moc{SYMM} \dusa{isymm} $|$ \moc{NOSY} $]$ $|$ \\
\moc{TSPC} $[$ \moc{MEDI}  $]$ \dusa{nangl} \dusa{dens} $|$ \moc{HALT} $\}$ $]$ \\
{\tt ;}
\end{DataStructure}

\noindent
where

\begin{ListeDeDescription}{mmmmmmmm}

\item[\moc{ANIS}] keyword to specify the order of scattering anisotropy. 

\item[\dusa{nanis}] order of anisotropy in transport calculation.
A default value of 1 represents isotropic (or transport-corrected) scattering while a value of 2
correspond to linearly anisotropic scattering. When anisotropic scattering is considered, user should pay attention to the following points:
\begin{itemize}
\item the usage of \moc{DIAG}, \moc{SYME}, \moc{SSYM} keywords in the definition of the geometry is forbidden. Indeed, in \moc{EXCELT:}/\moc{NXT:} tracking procedures, the geometry is ``unfolded'' according to these symmetries : this is incompatible with the integration of the anisotropic moments of the flux; \\
\item the angular quadratures should be selected paying attention to the restrictions mentioned in this manual in order to ensure the particle conservation.
\end{itemize}

\item[\moc{ONEG}] keyword to specify that the tracking is read before computing each group-dependent collision
probability or algebraic collapsing matrix (default value if \dusa{TRKFIL} is set). The tracking file is
read in each energy group if the method of characteristics (MOC) is used.

\item[\moc{ALLG}] keyword to specify that the tracking is read once and the collision
probability or algebraic collapsing matrices are computed in many energy groups.  The tracking file is
read once if the method of characteristics (MOC) is used.
 
\item[\moc{XCLL}] keyword to specify that the tracking is computed {\sl on-demand} (it is not stored on a file) and the
collision probability matrices are computed in many energy groups. The tracking
file \dusa{TRKFIL} should {\sl not} be provided (default value if \dusa{TRKFIL} is not set).

\item[\moc{BATCH}] keyword to specify the number of tracks processed by each core for each energy group. OpenMP parallelization is processing each energy group on a different core. The default value is \dusa{nbatch} $=1$.

\item[\dusa{nbatch}] the number of tracks processed by each core. Usually, a value \dusa{nbatch} $\ge 100$ is recommended.

\item[\moc{TREG}] keyword to specify that the normalization procedure of the integration lines activated by keywords \moc{RENO}
or \moc{REND} in Sect.~\ref{sect:TRKData} is to be performed with respect of the fine volumes as specified in the {\tt KEYFLX} record
of the tracking object. This is the default option.

\item[\moc{TMER}] keyword to specify that the normalization procedure of the integration lines activated by keywords \moc{RENO}
or \moc{REND} in Sect.~\ref{sect:TRKData} is to be performed with respect of the {\sl merged volumes} as specified in the {\tt KEYMRG} record
of the tracking object.

\item[\moc{PISO}] keyword to specify that a collision probability calculation
with isotropic reflection boundary conditions is required. It is the default
option if a \moc{TISO} type integration is chosen. To obtain accurate
transmission probabilities for the isotropic case it is recommended that the
normalization options in the \moc{ASM:} module be used.

\item[\moc{PSPC}] keyword to specify that  a collision probability calculation
with specular reflection boundary conditions required; this is the default
option if a \moc{TSPC} type integration is chosen. This calculation is only
possible if the file was initially constructed using the \moc{TSPC} option. 

\item[\moc{CUT}] keyword to specify the input of cutting parameters for the
specular integration.

\item[\dusa{pcut}] real value representing the maximum error allowed on the
exponential function used for specular collision probability calculations.
Tracks will be cut at a length such that the error in the probabilities
resulting from this reduced track will be of the order of \dusa{pcut}. By
default, there is no cutting of the tracks and \dusa{pcut}=0.0. If this option
is used in an entirely reflected case, it is preferable to use the \moc{NORM}
command in the \moc{ASM:} module.

\item[\moc{QUAB}] keyword to specify the number of basis point for the
numerical integration of each micro-structure in cases involving double
heterogeneity (Bihet).

\item[\dusa{iquab}] the number of basis point for the numerical integration of
the collision probabilities in the micro-volumes using the  Gauss-Jacobi
formula. The values permitted are: 1 to 20, 24, 28, 32 or  64. The default value
is \dusa{iquab}=5. If \dusa{iquab} is negative, its absolute value will be used in the She-Liu-Shi approach to determine the
split level in the tracking used to compute the probability collisions.

\item[\moc{SAPO}] use the Sanchez-Pomraning double-heterogeneity model.\cite{sapo}

\item[\moc{HEBE}] use the Hebert double-heterogeneity model (default option).\cite{BIHET}

\item[\moc{SLSI}] use the She-Liu-Shi double-heterogeneity model without shadow effect.\cite{She2017}

\item[\dusa{frtm}] the minimum microstructure volume fraction used to compute the size of the equivalent cylinder in She-Liu-Shi approach. The default value is \dusa{frtm} $=0.05$.

\item[\moc{PRIX}] keyword to specify that a prismatic tracking is considered for a 3D geometry invariant along the $x-$ axis. In this case, the 3D geometry is projected in the $y-z$ plane and a 2D tracking on the projected geometry is performed. This capability is limited to the non-cyclic method of characteristics solver for the time being and a subsequent call to \moc{MCCGT:} is mandatory.

\item[\moc{PRIY}] keyword to specify that a prismatic tracking is considered for a 3D geometry invariant along the $y-$ axis. In this case, the 3D geometry is projected in the $z-x$ plane and a 2D tracking on the projected geometry is performed. This capability is limited to the method of characteristics solver for the time being and a subsequent call to \moc{MCCGT:} is mandatory.

\item[\moc{PRIZ}] keyword to specify that a prismatic tracking is considered for a 3D geometry invariant along the $z-$ axis. In this case, the 3D geometry is projected in the $x-y$ plane and a 2D tracking on the projected geometry is performed. This capability is limited to the method of characteristics solver for the time being and a subsequent call to \moc{MCCGT:} is mandatory.

\item[\dusa{denspr}] real value representing the linear track density (in cm$^{-1}$) to be used for the inline contruction of 3D tracks from 2D tracking when a prismatic tracking is considered.

\item[\moc{LCMD}] keyword to specify that optimized (McDaniel--type) polar integration angles are to be
selected for the polar quadrature when a prismatic tracking is considered.\cite{LCMD} This is the default option. The conservation is ensured only for isotropic scattering.

\item[\moc{OPP1}] keyword to specify that $P_1$ constrained optimized (McDaniel--type) polar integration angles are to be selected for the polar quadrature when a prismatic tracking is considered.\cite{LeTellierpa} The conservation is ensured only for isotropic and linearly anisotropic scattering.

\item[\moc{OGAU}] keyword to specify that Optimized Gauss polar integration angles are to be
selected for the method of characteristics.\cite{LCMD,LeTellierpa} The conservation is ensured up to $P_{\dusa{nmu}-1}$ scattering.

\item[\moc{GAUS}] keyword to specify that Gauss-Legendre polar integration angles are to be selected for the polar quadrature when a prismatic tracking is considered. The conservation is ensured up to $P_{\dusa{nmu}-1}$ scattering.

\item[\moc{CACA}] keyword to specify that CACTUS type equal weight polar integration angles are to be
selected for the polar quadrature when a prismatic tracking is considered.\cite{CACTUS} The conservation is ensured only for isotropic scattering.

\item[\moc{CACB}] keyword to specify that CACTUS type uniformly distributed integration polar angles
are to be selected for the polar quadrature when a prismatic tracking is considered.\cite{CACTUS} The conservation is ensured only for isotropic scattering.

\item[\dusa{nmu}] user-defined number of polar angles. By default, a value consistent with \dusa{nangl} is computed by the code. For \moc{LCMD}, \moc{OPP1}, \moc{OGAU} quadratures, \dusa{nmu} is limited to 2, 3 or 4.

\item[\moc{TRAK}] keyword to specify the tracking parameters to be used. 

\item[\moc{TISO}] keyword to specify that isotropic tracking parameters will
be supplied. This is the default tracking option for cluster geometries.


\item[\moc{TSPC}] keyword to specify that specular tracking parameters will be
supplied.

\item[\moc{MEDI}] keyword to specify that instead of selecting the angles
located at the end of each angular interval, the angles located in the middle of
these intervals are selected. This is particularly useful if one wants to avoid
tracking angles that are parallel to the $X-$ or $Y-$axis as its is the case
when the external region of a \moc{CARCEL} geometry is voided.

\item[\dusa{nangl}] angular quadrature parameter. For applications involving
3--D cells, the choices are  \dusa{nangl}=2, 4, 8, 10, 12, 14 or 16; these
angular quadratures  $EQ_{n}$ present a rotational symmetry about the three
cartesian axes. For 2--D isotropic  applications, any value of  \dusa{nangl} $\ge 2$ may
be used; equidistant angles will be selected. For 2--D specular applications the
input value must be of the form $p+1$ where $p$ is a prime number (for example
$p$=7, 11, etc.); the choice of \dusa{nangl} = 8, 12, 14, 18, 20, 24, or 30 are
allowed. For cluster type geometries the default value is \dusa{nangl}=10 for
isotropic cases and \dusa{nangl}=12 for specular cases.

\item[\dusa{nangl\_z}] angular quadrature parameter in the axial $Z$ direction. Used only
with \dusa{HEXZ} and \dusa{HEXCELZ} geometries.

\item[\dusa{dens}] real value representing the density of the integration
lines (in $cm^{-1}$ for 2--D cases and $cm^{-2}$ for 3--D cases). This choice of
density along the plan perpendicular to each angle depends on the geometry of
the cell to be analyzed. If there are zones of very small volume, a high line
density is essential. This value will be readjusted by \moc{EXCELT:}. In the case
of the analysis of a cluster type geometry the default value of this parameter
is $5/r_{m}$ where $r_{m}$ is the minimum radius of the pins or the
minimum thickness of an annular ring in the geometry. If the selected value of \dusa{dens}
is too small, some volumes or surfaces may not be tracked.

\item[\dusa{dens\_z}] real value representing the density of the integration
lines in the axial $Z$ direction. Used only with \dusa{HEXZ} and \dusa{HEXCELZ} geometries.

\item[\moc{CORN}] keyword to specify that the input of the parameters used to
treat the corners for the isotropic integration.

\item[\dusa{pcorn}] maximum distance (cm) between a line and the intersection
of $n\ge 2$ external surfaces where track redistribution will take place. Track
redistribution will take place if a line comes close to the intersection of
$n\ge 2$ external surfaces. In this case the line will be replicated $n$ times,
each of these lines being associated with a different external surface, while
its weight is reduced by a factor of $1/n$. This allows for a better
distribution of tracks which are relatively close to $n$ external surfaces. By
default, there is no treatment of the corners and \dusa{pcorn}=0.0.

\item[\moc{SYMM}] keyword to specify that the geometry has a rotation
symmetry.

\item[\dusa{isymm}] integer value describing the rotation symmetry of the
geometry. The fixed default of this parameter is 1.

\item[\moc{NOSY}] \moc{EXCELT:} automatically try to take into account
geometric symmetries in order to reduce the number of tracks and the CPU
time. The \moc{NOSY} keyword desactivates this automatic capability.

\item[\moc{HALT}] keyword to specify that the program is to be stopped after
the analysis of the geometry, without the explicit tracking being performed.

\end{ListeDeDescription}
\eject
 % structure (excellT)
\subsubsection{The {\tt NXT:} tracking module}\label{sect:NXTData}

The calling specification for this module is:

\begin{DataStructure}{Structure \dstr{NXT:}}
$[$ \dusa{TRKFIL} $]$ \dusa{TRKNAM}
\moc{:=} \moc{NXT:} $[$ \dusa{TRKNAM} $]~[$ \dusa{GEONAM} $]$ \moc{::} \dstr{desctrack} \dstr{descnxt}
\end{DataStructure}

\noindent  where
\begin{ListeDeDescription}{mmmmmmm}

\item[\dusa{TRKNAM}] {\tt character*12} name of the \dds{tracking} data
structure that will contain region volume and surface area vectors in
addition to region identification pointers and other tracking information.
If \dusa{TRKNAM} also appears on the RHS, the previous tracking 
parameters will be applied by default on the current geometry.

\item[\dusa{TRKFIL}] {\tt character*12} name of the sequential binary tracking
file  used to store the tracks lengths. If \dusa{TRKFIL} does not appear, the keyword
\moc{XCLL} is set automatically. If the user wants to use a tracking file,
\dusa{TRKFIL} is required.

\item[\dusa{GEONAM}] {\tt character*12} name of the \dds{geometry} data
structure.

\item[\dstr{desctrack}] structure describing the general tracking data (see
\Sect{TRKData})

\item[\dstr{descnxt}] structure describing the transport tracking data
specific to \moc{NXT:}.

\end{ListeDeDescription}

\vskip 0.15cm

The \moc{NXT:} specific tracking data in \dstr{descnxt} is defined as

\begin{DataStructure}{Structure \dstr{descnxt}}
$[$ \moc{ANIS} \dusa{nanis} $]$ \\
$[~\{$  \moc{ONEG} $|$ \moc{ALLG} $[$ \moc{BATCH} \dusa{nbatch} $]~|$ \moc{XCLL} $\}~]$ \\
$[~[$ \moc{QUAB} \dusa{iquab} $]~[~\{$ \moc{SAPO} $|$ \moc{HEBE} $|$ \moc{SLSI} $[$ \dusa{frtm} $]~\}~]~]$ \\
$[~\{$ \moc{PISO} $|$ \moc{PSPC} $[$ \moc{CUT} \dusa{pcut} $]$ $\}~]$ \\
$[$ $\{$ \moc{SYMM} \dusa{isymm} $|$ \moc{NOSY} $]$ \\
$[$ $\{$ \moc{GAUS}  $|$ \moc{CACA} $|$ \moc{CACB} $|$ \moc{LCMD} $|$ \moc{OPP1} $|$ \moc{OGAU} $\}~[$ \dusa{nmu} $]~]$ \\
$\{$ \moc{TISO} $[~\{$ \moc{EQW} $|$ \moc{GAUS} $|$ \moc{PNTN} $|$ \moc{SMS} $|$ \moc{LSN} $|$ \moc{QRN} $\}~]$ \dusa{nangl} \dusa{dens} $[$ \moc{CORN} 
\dusa{pcorn} $]$ \\
$~~~~~|$ \moc{TSPC} $[~\{$ \moc{EQW} $|$ \moc{MEDI} $|$ \moc{EQW2} $\}~]$ \dusa{nangl} \dusa{dens} $\}$ \\
$[~\{$ \moc{NOTR} $|$ \moc{MC} $\}~]$\\
$[$ \moc{NBSLIN} \dusa{nbslin} $]$ \\
$[$ \moc{MERGMIX} $]$\\
$[$ \moc{LONG} $]$\\
$[$ \moc{PRIZ} \dusa{denspr} $]$ \\
{\tt ;}
\end{DataStructure}

\noindent
where

\begin{ListeDeDescription}{mmmmmmmm}

\item[\moc{ANIS}] keyword to specify the order of scattering anisotropy. 

\item[\dusa{nanis}] order of anisotropy in transport calculation.
A default value of 1 represents isotropic (or transport-corrected) scattering while a value of 2
correspond to linearly anisotropic scattering. When anisotropic scattering is considered, user should pay attention to the following points:
\begin{itemize}
\item the usage of \moc{DIAG}, \moc{SYME}, \moc{SSYM} keywords in the definition of the geometry is forbidden. Indeed, in \moc{EXCELT:}/\moc{NXT:} tracking procedures, the geometry is ``unfolded'' according to these symmetries : this is incompatible with the integration of the anisotropic moments of the flux; \\
\item an angular dependent normalization of the track lengths should be requested in the tracking procedure (\moc{REND} keyword) in order to ensure the particle conservation; \\
\item the angular quadratures should be selected paying attention to the restrictions mentioned in this manual in order to ensure the particle conservation.
\end{itemize}

\item[\moc{ONEG}] keyword to specify that the tracking is read before computing each group-dependent collision
probability or algebraic collapsing matrix (default value if \dusa{TRKFIL} is set). The tracking file is
read in each energy group if the method of characteristics (MOC) is used.

\item[\moc{ALLG}] keyword to specify that the tracking is read once and the collision
probability or algebraic collapsing matrices are computed in many energy groups.  The tracking file is
read once if the method of characteristics (MOC) is used.
 
\item[\moc{XCLL}] keyword to specify that the tracking is computed {\sl on-demand} (it is not stored on a file) and the
collision probability matrices are computed in many energy groups. The tracking
file \dusa{TRKFIL} should {\sl not} be provided (default value if \dusa{TRKFIL} is not set).

\item[\moc{BATCH}] keyword to specify the number of tracks processed by each core for each energy group. OpenMP parallelization is processing each energy group on a different core. The default value is \dusa{nbatch} $=1$.

\item[\dusa{nbatch}] the number of tracks processed by each core. Usually, a value \dusa{nbatch} $\ge 100$ is recommended.

\item[\moc{QUAB}] keyword to specify the number of basis point for the
numerical integration of each micro-structure in cases involving double
heterogeneity (Bihet).

\item[\dusa{iquab}] the number of basis point for the numerical integration of
the collision probabilities in the micro-volumes using the Gauss-Jacobi
formula. The values permitted are: 1 to 20, 24, 28, 32 or 64. The default value
is \dusa{iquab} = 5. If \dusa{iquab} is negative, its absolute value will be used in the She-Liu-Shi approach to determine the
split level in the tracking used to compute the probability collisions.

\item[\moc{SAPO}] use the Sanchez-Pomraning double-heterogeneity model.\cite{sapo}

\item[\moc{HEBE}] use the Hebert double-heterogeneity model (default option).\cite{BIHET}

\item[\moc{SLSI}] use the She-Liu-Shi double-heterogeneity model without shadow effect.\cite{She2017}

\item[\dusa{frtm}] the minimum microstructure volume fraction used to compute the size of the equivalent cylinder in She-Liu-Shi approach. The default value is \dusa{frtm} $=0.05$.

\item[\moc{PISO}] keyword to specify that a collision probability calculation with isotropic reflection boundary 
conditions is required. It is the default option if a \moc{TISO} type integration is chosen. To obtain accurate
transmission probabilities for the isotropic case it is recommended that the normalization 
options in the \moc{ASM:} module be used. 

\item[\moc{PSPC}] keyword to specify that a collision probability calculation with mirror like reflection or periodic 
boundary conditions is required; this is the default option if a \moc{TSPC} type integration is chosen. 
This calculation is only possible if the file was initially constructed using the \moc{TSPC} option. 

\item[\moc{CUT}] keyword to specify the input of cutting parameters for the specular collision probability
of characteristic integration. 

\item[\dusa{pcut}] real value representing the maximum error allowed on the exponential function used
for specular collision probability calculations. Tracks will be cut at a length such that the error in the 
probabilities resulting from this reduced track will be of the order of pcut. By default, the tracks 
are extended to infinity and \dusa{pcut} = 0.0. If this option is used in an entirely reflected case, it is 
recommended to use the \moc{NORM} command in the \moc{ASM:} module. 

\item[\moc{SYMM}] keyword to specify the level to which the tracking will respect the symmetry of the geometry. 

\item[\dusa{isymm}]  level to which the tracking will respect the symmetry of the geometry. For 2-D and 3-D Cartesian geometries it must takes the form \dusa{isymm}=$2 S_{x}+4S_{y}+16 S_{z}$ where
\begin{itemize}
\item $S_{x}=1$ if the $X$ symmetry is to be considered and $S_{x}=0$ otherwise.   
\item $S_{y}=1$ if the $Y$ symmetry is to be considered and $S_{y}=0$ otherwise.   
\item $S_{z}=1$ if the $Z$ symmetry is to be considered and $S_{z}=0$ otherwise.   
\end{itemize}

\item[\moc{NOSY}] keyword to specify the full tracking will take place irrespective of the symmetry of the geometry. This is equivalent to specifying \dusa{isymm}=0.

\item[\moc{GAUS}] keyword to specify that Gauss-Legendre polar integration angles are to be selected for the polar quadrature when a prismatic tracking is considered. The conservation is ensured up to $P_{\dusa{nmu}-1}$ scattering.

\item[\moc{CACA}] keyword to specify that CACTUS type equal weight polar integration angles are to be
selected for the polar quadrature when a prismatic tracking is considered.\cite{CACTUS} The conservation is ensured only for isotropic scattering.

\item[\moc{CACB}] keyword to specify that CACTUS type uniformly distributed integration polar angles
are to be selected for the polar quadrature when a prismatic tracking is considered.\cite{CACTUS} The conservation is ensured only for isotropic scattering.

\item[\moc{LCMD}] keyword to specify that optimized (McDaniel--type) polar integration angles are to be
selected for the polar quadrature when a prismatic tracking is considered.\cite{LCMD} This is the default option. The conservation is ensured only for isotropic scattering.

\item[\moc{OPP1}] keyword to specify that $P_1$ constrained optimized (McDaniel--type) polar integration angles are to be selected for the polar quadrature when a prismatic tracking is considered.\cite{LeTellierpa} The conservation is ensured only for isotropic and linearly anisotropic scattering.

\item[\moc{OGAU}] keyword to specify that Optimized Gauss polar integration angles are to be
selected for the method of characteristics.\cite{LCMD,LeTellierpa} The conservation is ensured up to $P_{\dusa{nmu}-1}$ scattering.

\item[\dusa{nmu}]  user-defined number of polar angles. By default, a value consistent with \dusa{nangl} is computed by the code. For \moc{LCMD}, \moc{OPP1}, \moc{OGAU} quadratures, \dusa{nmu} is limited to 2, 3 or 4.

\item[\moc{TISO}] keyword to specify that isotropic tracking parameters will be supplied. This is the
default tracking option for cluster geometries. 

\item[\moc{TSPC}] keyword to specify that specular tracking parameters will be supplied.

\item[\moc{EQW}] keyword to specify the use of equal weight quadrature.\cite{eqn} The conservation is ensured up to $P_{\dusa{nangl}/2}$ scattering.

\item[\moc{GAUS}] (after \moc{TISO} keyword) keyword to specify the use of the Gauss-Legendre quadrature. This option is valid only if an 
hexagonal geometry is considered.

\item[\moc{PNTN}] keyword to specify that Legendre-Techbychev quadrature quadrature will be selected.\cite{pntn} The conservation is ensured only for isotropic and linearly anisotropic scattering.

\item[\moc{SMS}] keyword to specify that Legendre-trapezoidal quadrature quadrature will be selected.\cite{sms} The conservation is ensured up to $P_{\dusa{nangl}-1}$ scattering.

\item[\moc{LSN}] keyword to specify the use of the $\mu_1$--optimized level-symmetric quadrature. The conservation is ensured up to $P_{\dusa{nangl}/2}$ scattering.

\item[\moc{QRN}] keyword to specify the use of the quadrupole range (QR) quadrature.\cite{quadrupole}

\item[\moc{MEDI}] keyword to specify the use of a median angle quadrature in \moc{TSPC} cases. Instead of
selecting the angles located at the end of each angular interval, the angles located in the middle of
these intervals are selected. This is particularly useful if one wants to avoid
tracking angles that are parallel to the $X-$ or $Y-$axis as its is the case
when the external region of a \moc{CARCEL} geometry is voided.

\item[\moc{EQW2}] keyword to eliminate angles $\phi=0$ and $\phi=\pi/2$ from the \moc{EQW} quadrature in \moc{TSPC} cases.

\item[\dusa{nangl}] angular quadrature parameter. For a 3-D \moc{EQW} option, the choices are \dusa{nangl} = 2, 4, 8, 10, 12, 14 
or 16. For a 3-D \moc{PNTN} or \moc{SMS} option, \dusa{nangl} is an even number smaller than 46.\cite{ige260} For 2-D 
isotropic applications, any value of \dusa{nangl} may be used, equidistant angles will be selected.

For 2-D specular applications the input value must be of the form $p + 1$ where $p$ is a prime number, as proposed
in Ref.~\citen{DragonPIJS3}. In this case, the choice of \dusa{nangl} = 2, 8, 12, 14, 18, 20, 24, or 30 are allowed. For
a rectangular Cartesian domain of size $X \times Y$, the azimuthal angles in $(0,\pi/2)$ interval are obtained from formula
\begin{align*}
\phi_k=\begin{cases}
\arctan\left(\frac{kY}{(p-k)X}\right)  \, , \ \ k=0,\, 1,\, 2,\, \dots, \, p & \text{if {\tt EWQ} (default)}\\
\arctan\left(\frac{kY}{(2p+2-k)X}\right) \, , \ \ k=1,\, 3,\, 5, \, \dots, \, 2p+1 & \text{if {\tt MEDI}} \\
\arctan\left(\frac{kY}{(p+2-k)X}\right) \, , \ \ k=1,\, 2,\, 3, \, \dots, \, p+1 &\text{if {\tt EQW2}.}\\
\end{cases}
\end{align*}

\item[\dusa{dens}] real value representing the density of the integration lines (in cm$^{-1}$ for 2-D Cartesian cases and 
3-D hexagonal cases and cm$^{-2}$ for 3-D cases Cartesian cases). This choice of density along the 
plan perpendicular to each angle depends on the geometry of the cell to be analyzed. If there 
are zones of very small volume, a high line density is essential. This value will be readjusted by 
\moc{NXT:}.

\item[\moc{CORN}] keyword to specify that the input of the parameters used to treat the corners for the isotropic 
integration. 

\item[\dusa{pcorn}] maximum distance (cm) between a line and the intersection of $n\ge 2$ external surfaces where 
track redistributing will take place. Track redistribution will take place if a line comes close to 
the intersection of $n \ge 2$ external surfaces. In this case the line will be replicated $n$ times, each 
of these lines being associated with a different external surface, while its weight is reduced by 
a factor of $1/n$. This allows for a better distribution of tracks which are relatively close to $n$ 
external surfaces. By default, there is no treatment of the corners and \dusa{pcorn} = 0.0.

\item[\moc{NOTR}] keyword to specify that the geometry will not be tracked. This is useful for 2-D geometries 
to generate a tracking data structure that can be used by the \moc{PSP:} module (see \Sect{PSPData}). 
One can then verify visually if the geometry is adequate before the tracking process as such is 
undertaken.

\item[\moc{MC}] keyword to specify that the geometry will not be tracked and that object \dusa{TRKNAM} will be used with the
Monte-Carlo method. This option is similar to \moc{NOTR} with additional information being added into \dusa{TRKNAM}.

\item[\moc{NBSLIN}] keyword to set the maximum number of segments in a single tracking line.

\item[\dusa{nbsl}] integer value representing the maximum number of segments in a single tracking line. The default value is \dusa{nbsl} = 100000.

\item[\moc{MERGMIX}] keyword to specify that all regions belonging to the same mixture will be merged together. This option should only be used as an attempt to reduce CPU costs in resonance self-shielding calculations.

\item[\moc{LONG}] keyword to specify that a ``long'' tracking file will be generated. This option is required if the tracking file is to be used by the \moc{TLM:} module (see \Sect{TLMData}).

\item[\moc{PRIZ}] keyword to specify that a prismatic tracking is considered for a 3D geometry invariant along the $z-$ axis. In this case, the 3D geometry is projected in the $x-y$ plane and a 2D tracking on the projected geometry is performed. This capability is limited to the non-cyclic method of characteristics solver for the time being and a subsequent call to \moc{MCCGT:} is mandatory.

\item[\dusa{denspr}] real value representing the linear track density (in cm$^{-1}$) to be used for the inline contruction of 3D tracks from 2D tracking when a prismatic tracking is considered.

\end{ListeDeDescription}
\clearpage
 % structure (nxtT)
\subsubsection{The {\tt MCCGT:} tracking module}\label{sect:MCCGData}

This module {\sl must} follow a call to module \moc{EXCELT:} or \moc{NXT:}. Its calling
specification is:

\begin{DataStructure}{Structure \dstr{MCCGT:}}
\dusa{TRKNAM} \moc{:=} \moc{MCCGT:} \dusa{TRKNAM} \dusa{TRKFIL} 
$[$ \dusa{GEONAM} $]$ \moc{::} \dstr{descmccg}
\end{DataStructure}

\noindent  where
\begin{ListeDeDescription}{mmmmmmm}

\item[\dusa{TRKNAM}] {\tt character*12} name of the \dds{tracking} data
structure that will contain region volume and surface area vectors in
addition to region identification pointers and other tracking information. It is provided by \moc{EXCELT:} or \moc{NXT:} operator and modified by \moc{MCCGT:} operator.

\item[\dusa{TRKFIL}] {\tt character*12} name of the sequential binary tracking file used to store the tracks lengths. This file is provided by \moc{EXCELT:} or \moc{NXT:} operator and used without modification by \moc{MCCGT:} operator.

\item[\dusa{GEONAM}] {\tt character*12} name of the optional \dds{geometry} data
structure. This structure is only required to recover double-heterogeneity data.

\item[\dstr{descmccg}] structure describing the transport tracking data
specific to \moc{MCCGT:}.

\end{ListeDeDescription}

\vskip 0.15cm

The \moc{MCCGT:} specific tracking data in \dstr{descmccg} is defined as

\begin{DataStructure}{Structure \dstr{descmccg}}
$[$ \moc{EDIT} \dusa{iprint} $]$ \\
$[$ $\{$ \moc{LCMD} $|$ \moc{OPP1} $|$ \moc{OGAU} $|$ \moc{GAUS} $|$ \moc{DGAU} $|$ \moc{CACA} $|$ \moc{CACB} $\}~[$ \dusa{nmu} $]~]$ \\
$\{$ \moc{DIFC} $[~\{$ \moc{NONE} $|$ \moc{DIAG} $|$ \moc{FULL} $|$ \moc{ILU0} $\}~]$ $~[$ \moc{TMT} $]$ $~[$ \moc{LEXA} $]$ \\
$~~~~~|$ \\
$~~~~~[~[$ \moc{AAC} \dusa{iaca} $[~\{$ \moc{NONE} $|$ \moc{DIAG} $|$ \moc{FULL} $|$ \moc{ILU0} $\}~]~[$ \moc{TMT} $]~]~[$ \moc{SCR} \dusa{iscr} $]~[$ \moc{LEXA} $]~]$ \\
$~~~~~[$ \moc{KRYL} \dusa{ikryl} $]$ \\
$~~~~~[$ \moc{MCU} \dusa{imcu} $]$ \\
$~~~~~[$ \moc{HDD} \dusa{xhdd} $]$ \\
$~~~~~[~\{$ \moc{SC} $|$ \moc{LDC} $\}~]$ \\
$~~~~~[$ \moc{LEXF} $]$ \\
$~~~~~[$ \moc{STIS} \dusa{istis} $]$ \\
$\}~$ \\
$~~~~~[$ \moc{MAXI} \dusa{nmaxi} $]$ \\
$~~~~~[$ \moc{EPSI} \dusa{xepsi} $]$ \\
$~~~~~[$ \moc{ADJ} $]$ \\
 {\tt ;}
\end{DataStructure}

\noindent
where

\begin{ListeDeDescription}{mmmmmmmm}

\item[\moc{EDIT}] keyword used to modify the print level iprint.

\item[\dusa{iprint}] index used to control the printing in this operator.

\item[\moc{LCMD}] keyword to specify that optimized (McDaniel--type) polar integration angles are to be
selected for the method of characteristics.\cite{LCMD} The conservation is ensured only for isotropic scattering.

\item[\moc{OPP1}] keyword to specify that $P_1$ constrained optimized (McDaniel--type) polar integration angles are to be selected for the method of characteristics.\cite{LeTellierpa} The conservation is ensured only for isotropic and linearly anisotropic scattering.

\item[\moc{OGAU}] keyword to specify that Optimized Gauss polar integration angles are to be
selected for the method of characteristics.\cite{LCMD,LeTellierpa} The conservation is ensured up to $P_{\dusa{nmu}-1}$ scattering.

\item[\moc{GAUS}] keyword to specify that the polar integration angles are to be selected as a single Gauss-Legendre quadrature for the method of characteristics in interval ($-\pi/2$, $\pi/2$). The conservation is ensured up to $P_{\dusa{nmu}-1}$ scattering. This is the default option.

\item[\moc{DGAU}] keyword to specify that the polar integration angles are to be selected as a double Gauss-Legendre quadrature for the method of characteristics in intervals ($-\pi/2$, $0$) and ($0$, $\pi/2$). The conservation is ensured up to $P_{\dusa{nmu}-1}$ scattering.

\item[\moc{CACA}] keyword to specify that CACTUS type equal weight polar integration angles are to be
selected for the method of characteristics.\cite{CACTUS} The conservation is ensured only for isotropic scattering.

\item[\moc{CACB}] keyword to specify that CACTUS type uniformly distributed integration polar angles
are to be selected for the method of characteristics.\cite{CACTUS} The conservation is ensured only for isotropic scattering.

\item[\dusa{nmu}] user-defined number of polar angles for the integration of the tracks with the method of characteristics for 2D geometries. By default, a value consistent with \dusa{nangl} is computed by the code. For \moc{LCMD}, \moc{OPP1}, \moc{OGAU} quadratures, \dusa{nmu} is limited to 2, 3 or 4.

\item[\moc{DIFC}] keyword used to specify that only an ACA-simplified transport flux calculation is to be performed (not by default). In this case, the maximum
number of ACA iterations is set to \dusa{nmaxi}.

\item[\moc{LEXA}] keyword used to force the usage of exact exponentials in the preconditioner calculation (not by default).

\item[\moc{MAXI}] keyword to specify the maximum number of scattering iterations performed in each energy group. This keyword is also used to set the number of Bi-CGSTAB iterations to solve the ACA-simplified system if \moc{DIFC} is present.

\item[\dusa{nmaxi}] the maximum number of iterations. The default value is \dusa{nmaxi}=20.

\item[\moc{EPSI}] keyword to specify the convergence criterion on inner
iterations (or ACA-simplified flux calculation if \moc{DIFC} is present).

\item[\dusa{xepsi}] convergence criterion. The default value is \dusa{xepsi}=1.0$\times$10$^{-5}$.

\item[\moc{AAC}] keyword to set the ACA preconditioning of inner/multigroup
iterations in case where a transport solution is selected.\cite{cdd,suslov2}

\item[\dusa{iaca}] $0$/$>0$: ACA preconditioning of inner or multigroup iterations off/on. The default value is \dusa{iaca}=1. If \moc{MAXI} is set to 1, ACA is used as a rebalancing technique for multigroup-inner mixed iterations and \dusa{iaca} is the maximum number of iterations allowed to solve the ACA system (e.g. 100).

\item[\moc{NONE}] no preconditioning for the iterative resolution by Bi-CGSTAB of the ACA system.

\item[\moc{DIAG}] diagonal preconditioning for the iterative resolution by Bi-CGSTAB of the ACA system.

\item[\moc{FULL}] full-matrix preconditioning for the iterative resolution by Bi-CGSTAB of the ACA system.

\item[\moc{ILU0}] ILU0 preconditioning for the iterative resolution by Bi-CGSTAB of the ACA system (This is the default option).

\item[\moc{TMT}] two-step collapsing version of ACA which uses a tracking merging technique while building the ACA matrices. 

\item[\moc{SCR}] keyword to set the SCR preconditioning of inner/multigroup
iterations.\cite{gmres}

\item[\dusa{iscr}] $0$/$>0$: SCR preconditioning of inner or multigroup iterations off/on. The default value is \dusa{iscr}=0. If \moc{MAXI} is set to 1, SCR is used as a rebalancing technique for multigroup-inner mixed iterations and \dusa{iscr} is the maximum number of iterations allowed to solve the SCR system. When anisotropic scattering is considered, SCR provides an acceleration of anisotropic flux moments. If both ACA and SCR are selected (\dusa{iscr}$>0$ and \dusa{iaca}$>0$), a two-step acceleration scheme (equivalent to ACA when isotropic scattering is considered) involving both methods is used.

\item[\moc{KRYL}] keyword to enable the Krylov acceleration of scattering iterations performed in each energy group.\cite{gmres}

\item[\dusa{ikryl}] $0$: GMRES/Bi-CGSTAB acceleration not used; $>0$: dimension of the Krylov subspace in GMRES; $<0$: Bi-CGSTAB is used.
The default value is \dusa{ikryl}=10.

\item[\moc{MCU}] keyword used to specify the maximum dimension of the connection matrix for memory allocation.

\item[\dusa{imcu}] The default value is eight (resp. twelve) times the number of volumes and external surfaces for 2D (resp. 3D) geometries.

\item[\moc{HDD}] keyword to select the integration scheme along the tracking lines.

\item[\dusa{xhdd}] selection criterion:

$$
xhdd = \left\{
\begin{array}{rl}
 0.0 & \textrm{step characteristics scheme} \\
>0.0 & \textrm{diamond differencing scheme.}
\end{array} \right.
$$

The default value is \dusa{xhdd}=0.0 so that the step characteristics method is used.

\item[\moc{LEXF}] keyword used to force the usage of exact exponentials in the flux calculation (not by default).

\item[\moc{SC}] keyword used to select the step characteristics (SC) or DD0 diamond differencing approximation. This
option is a flat source approximation (default option).

\item[\moc{LDC}] keyword used to select the linear discontinuous characteristics (LDC) or DD1 diamond differencing approximation. This
option is a linear source approximation.

\item[\moc{STIS}] keyword to select the tracking integration strategy.

\item[\dusa{istis}] $0$: a direct approach with asymptotical treatment is used; $1$: a ``source term isolation'' approach with asymptotical treatment is used (this technique tends to reduce the computational cost and increase the numerical stability but requires the calculation of angular mode-to-mode self-collision probabilities); $-1$:  an "MOCC/MCI"-like approach is used (it tends to reduce further more the computational cost as it doesn't feature any asymptotical treatment for vanishing optical thicknesses). Note that when a zero total cross section is found with \dusa{istis}=-1, it is reset to 1. The default value is \dusa{istis}=1 for $P_{L \le 3}$ anisotropy and 0 otherwise.

\item[\moc{ADJ}] keyword to select an adjoint solution of ACA and characteristics systems. A direct solution is
set by default.

\end{ListeDeDescription}
\eject
 % structure (mccgT)
\subsubsection{The {\tt SNT:} tracking module}\label{sect:SNData}

The {\tt SNT:} module can process one-dimensional, two-dimensional regular geometries and three-dimensional Cartesian geometries
of type \moc{CAR1D}, \moc{TUBE}, \moc{SPHERE}, \moc{CAR2D}, \moc{TUBEZ} and \moc{CAR3D}.

\vskip 0.2cm

The calling specification for this module is:

\begin{DataStructure}{Structure \dstr{SNT:}}
\dusa{TRKNAM}
\moc{:=} \moc{SNT:} $[$ \dusa{TRKNAM} $]$ 
\dusa{GEONAM} \moc{::}  \dstr{desctrack} \dstr{descsn}
\end{DataStructure}

\noindent  where
\begin{ListeDeDescription}{mmmmmmm}

\item[\dusa{TRKNAM}] {\tt character*12} name of the \dds{tracking} data
structure that will contain region volume and surface area vectors in
addition to region identification pointers and other tracking information.
If \dusa{TRKNAM} also appears on the RHS, the previous tracking 
parameters will be applied by default on the current geometry.

\item[\dusa{GEONAM}] {\tt character*12} name of the \dds{geometry} data
structure.

\item[\dstr{desctrack}] structure describing the general tracking data (see
\Sect{TRKData})

\item[\dstr{descsn}] structure describing the transport tracking data
specific to \moc{SNT:}.

\end{ListeDeDescription}

\vskip 0.2cm

The \moc{SNT:} specific tracking data in \dstr{descsn} is defined as

\begin{DataStructure}{Structure \dstr{descsn}}
$[~\{$ \moc{ONEG} $|$ \moc{ALLG} $\}~]~[$ \moc{KBA} \dusa{m} $]$ \\
$[$ \moc{SCHM} \dusa{ischm} $]~[$ \moc{DIAM} \dusa{mm} $]$ \\
\moc{SN} \dusa{n} $~[$ \moc{SCAT} \dusa{iscat} $]~$ \\
$[~\{$ \moc{LIVO} \dusa{icl1} \dusa{icl2} $|$ \moc{NLIVO} $|$ \moc{GMRES} \dusa{nstart} $\}~]$ \\
$[~\{$ \moc{DSA} \dusa{ndsa} \dusa{mdsa} \dusa{sdsa} $|$ \moc{NDSA} $\}~]$ \\
$[$ \moc{NSHT} $]$ \\
$[$ \moc{FOUR} \dusa{nfou} $]$ \\
$[$ \moc{MAXI} \dusa{maxi} $]~[$ \moc{EPSI} \dusa{epsi} $]$ \\
$[$ \moc{QUAD} \dusa{iquad} $]$ \\
$[~\{$ \moc{BTE} $|$ \moc{BFPG} $|$ \moc{BFPL}$ \}~]$ \\
$[$ \moc{ESCHM} \dusa{eschm} $]~[$ \moc{EDIAM} \dusa{emm} $]$ \\
$[~[$ \moc{QUAB} \dusa{iquab} $]~[~\{$ \moc{SAPO} $|$ \moc{HEBE} $|$ \moc{SLSI} $[$ \dusa{frtm} $]~\}~]~]$ \\
{\tt ;}
\end{DataStructure}

\noindent where

\begin{ListeDeDescription}{mmmmmmm}

\item[\dstr{desctrack}] structure describing the general tracking data (see
\Sect{TRKData})

\item[\moc{ONEG}] keyword to specify that the multigroup flux is computed as a sequence of one-group solutions using Gauss-Seidel iterations. This is the default option.

\item[\moc{ALLG}] keyword to specify that the multigroup flux is computed in parallel for a set of energy groups.

\item[\moc{KBA}] keyword to specify that Koch-Baker-Alcouffe (KBA) type nested loops over both angles and macrocells are used for
multithreading in 2D and 3D geometries.\cite{kba,domino}

\item[\dusa{m}] use $m\times m$ or $m \times m \times m$ macrocells in KBA swapping.

%Update to tabular format?
\item[\moc{SCHM}] keyword to specify the spatial discretisation scheme. 

\item[\dusa{ischm}] index to specify the spatial discretisation scheme. \dusa{ischm} $=1$ is used for High-Order Diamond Differencing (HODD) (default value). \dusa{ischm} $=2$ is used for the Discontinuous Galerkin finite element method (DG) currently available only in 1D slab, and 2D/3D Cartesian/hexagonal geometries. \dusa{ischm} $=3$ is used for the Adaptive Weighted Difference method (AWD), only available for Cartesian geometries.

\item[\moc{DIAM}] keyword to fix the spatial approximation order.

\item[\dusa{mm}] order of the Legendre polynomial expansion used in the spatial discretisation method. For HODD, \dusa{mm} $=0$ is the default, while for DG, it is \dusa{mm} $=1$.
For Cartesian geometries, any order \dusa{mm} $\geq0$ is available. For 2D hexagonal geometries, linear and parabolic orders are available. Classical diamond difference (\dusa{mm} $=0$ with \dusa{ischm} $=1$) are available for 1D tube and 1D sphere geometries. Adaptive schemes (\dusa{ischm} $=3$) are only available with constant order.
\begin{displaymath}
\dusa{mm} = \left\{
\begin{array}{rl}
 0 & \textrm{Constant (classical diamond scheme (HODD) or step scheme (DG))} \\
 1 & \textrm{Linear} \\
 2 & \textrm{Parabolic} \\
 >3  & \textrm{Higher-orders}
\end{array} \right.
\end{displaymath}

\item[\moc{SN}] keyword to fix the angular approximation order of the flux.

\item[\dusa{n}] order of the $S_N$ approximation (even number).

\item[\moc{SCAT}] keyword to limit the anisotropy of scattering sources.

\item[\dusa{iscat}] number of terms in the scattering sources. \dusa{iscat} $=1$ is used for
isotropic scattering in the laboratory system. \dusa{iscat} $=2$ is used for
linearly anisotropic scattering in the laboratory system. The default value is set to $n$.

\item[\moc{LIVO}] keyword to enable Livolant acceleration of the scattering iterations (default value).
\item[\dusa{icl1},~\dusa{icl2}] Numbers of respectively free and accelerated iterations in the Livolant method.
\item[\moc{NLIVO}] keyword to disable acceleration method and to perform free scattering iterations

\item[\moc{GMRES}] keyword to set the GMRES(m) acceleration of the scattering iterations. The default value,
equivalent to \dusa{nstart}=0, corresponds to a one-parameter Livolant acceleration.\cite{gmres}

\item[\dusa{nstart}] restarts the GMRES method every \dusa{nstart} iterations.

\item[\moc{DSA}] keyword to enable diffusion synthetic acceleration using BIVAC or TRIVAC.
\item[\dusa{ndsa}] apply the synthetic acceleration every \dusa{ndsa} number of inner flux iterations. Depending on the test case, if the DSA is enabled too soon or enabled at every inner iteration, instabilities and convergence failure can occur. A value of $0$ can be set to start the DSA immediately and have the acceleration applied every inner iteration thereafter. The default is \dusa{nsdsa}~$=1000$, indicating the DSA will not be applied. Benchmarks suggests that the optimal values are $3$ and $5$ for Cartesian and hexagonal geometries respectively.
\item[\dusa{mdsa}] order of the Raviart-Thomas spatial approximation used in the DSA resolution. Sometimes, using the same order as the transport calculation does not provide any benefit to the solution, and ends up being a drain on computational resources. Hence, there is the option of using a different order than the transport approximation.
\begin{displaymath}
\dusa{mdsa} = \left\{
\begin{array}{rl}
 0 & \textrm{Constant} \\
 1 & \textrm{Linear} \\
 2 & \textrm{Parabolic} \\
\end{array} \right.
\end{displaymath}
\item[\dusa{sdsa}] choose the solver to use for the synthetic acceleration. Note that TRIVAC generally works better and is faster with hexagonal geometries for the matrix assemblies. Also, for 3D geometries, TRIVAC \emph{has} to be chosen.
\begin{displaymath}
\dusa{sdsa} = \left\{
\begin{array}{rl}
 1 & \textrm{BIVAC} \\
 2 & \textrm{TRIVAC} \\
\end{array} \right.
\end{displaymath}

\item[\moc{NDSA}] keyword to disable diffusion synthetic acceleration (default).

\item[\moc{NSHT}] keyword to disable the shooting method for 1D cases -- can be useful for debugging purposes.

\item[\moc{FOUR}] keyword to pass the number of frequencies to be investigated in Fourier analysis (only works in 1D Cartesian geometry).
\item[\dusa{nfou}] number of frequencies to be investigated in 1D Fourier analysis along the range $[0, \frac{2\pi}{L})$ where $L$ is the length of the slab.

\item[\moc{MAXI}] keyword to set the maximum number of inner iterations (or GMRES iterations if actived).
\item[\dusa{maxi}] maximum number of inner iterations. Default value: $100$.

\item[\moc{EPSI}] set the convergence criterion on inner iterations (or GMRES iterations if actived).
\item[\dusa{epsi}] convergence criterion on inner iterations. The default value is $1\times 10^{-5}$.
\item[\moc{QUAD}] keyword to set the type of angular quadrature.

\item[\dusa{iquad}] type of quadrature: $=1$: Lathrop-Carlson level-symmetric quadrature;
$=2$: $\mu_1$--optimi\-zed level-symmetric quadrature (default option in 2D and in 3D); $=3$ Snow-code level-symmetric quadrature
(obsolete); $=4$: Legendre-Chebyshev quadrature (variable number of base points
per axial level); $=5$: symmetric Legendre-Chebyshev quadrature; $=6$: quadruple range (QR)
quadrature;\cite{quadrupole} $=10$: product of Gauss-Legendre and Gauss-Chebyshev quadrature (equal
number of base points per axial level).

\item[\moc{BTE}] solution of the Boltzmann transport equation (default option).

\item[\moc{BFPG}] solution of the Boltzmann Fokker-Planck equation with Galerkin energy propagation factors.

\item[\moc{BFPL}] solution of the Boltzmann Fokker-Planck equation with Przybylski and Ligou energy propagation factors.\cite{ligou}

\item[\moc{ESCHM}] keyword to specify the energy discretisation scheme to use for the continuous slowing-down term of the Boltzmann Fokker-Planck equation. 

\item[\dusa{eschm}] index to specify the energy discretisation scheme. \dusa{ischm} $=1$ is used for High-Order Diamond Differencing (HODD) (default value). \dusa{ischm} $=2$ is used for the Discontinuous Galerkin finite element method (DG). \dusa{ischm} $=3$ is used for the Adaptive Weighted Difference method (AWD). All of these schemes are available only for Cartesian geometries.

\item[\moc{EDIAM}] keyword to fix the energy approximation order.

\item[\dusa{emm}] order of the Legendre polynomial expansion used in the energy discretisation method. For HODD, \dusa{mm} $=0$ is the default, while for DG, it is \dusa{mm} $=1$.
For Cartesian geometries, any order \dusa{mm} $\geq0$ is available. Adaptive schemes (\dusa{ischm} $=3$) are only available with constant order.
\begin{displaymath}
\dusa{mm} = \left\{
\begin{array}{rl}
 0 & \textrm{Constant (classical diamond scheme (HODD) or step scheme (DG))} \\
 1 & \textrm{Linear} \\
 2 & \textrm{Parabolic} \\
 >3  & \textrm{Higher-orders}
\end{array} \right.
\end{displaymath}

\item[\moc{QUAB}] keyword to specify the number of basis point for the
numerical integration of each micro-structure in cases involving double
heterogeneity (Bihet).

\item[\dusa{iquab}] the number of basis point for the numerical integration of
the collision probabilities in the micro-volumes using the  Gauss-Jacobi
formula. The values permitted are: 1 to 20, 24, 28, 32 or  64. The default value
is \dusa{iquab}=5. If \dusa{iquab} is negative, its absolute value will be used in the She-Liu-Shi approach to determine the
split level in the tracking used to compute the probability collisions.

\item[\moc{SAPO}] use the Sanchez-Pomraning double-heterogeneity model.\cite{sapo}

\item[\moc{HEBE}] use the Hebert double-heterogeneity model (default option).\cite{BIHET}

\item[\moc{SLSI}] use the She-Liu-Shi double-heterogeneity model without shadow effect.\cite{She2017}

\item[\dusa{frtm}] the minimum microstructure volume fraction used to compute the size of the equivalent cylinder in She-Liu-Shi approach. The default value is \dusa{frtm} $=0.05$.

\end{ListeDeDescription}

\eject
 % structure (snT)
\subsubsection{The {\tt BIVACT:} tracking module}\label{sect:BIVACData}

The {\tt BIVACT:} module provides an implementation of the diffusion or simplified $P_n$ method. The {\tt BIVACT:} module can only process
1D/2D regular geometries of type \moc{CAR1D}, \moc{CAR2D} and \moc{HEX}. The geometry is analyzed and
a LCM object with signature {\tt L\_BIVAC} is created with the tracking information.

\vskip 0.2cm

The calling specification for this module is:

\begin{DataStructure}{Structure \dstr{BIVACT:}}
\dusa{TRKNAM}
\moc{:=} \moc{BIVACT:} $[$ \dusa{TRKNAM} $]$ 
\dusa{GEONAM} \moc{::}  \dstr{desctrack} \dstr{descbivac}
\end{DataStructure}

\noindent  where
\begin{ListeDeDescription}{mmmmmmm}

\item[\dusa{TRKNAM}] {\tt character*12} name of the \dds{tracking} data
structure that will contain region volume and surface area vectors in
addition to region identification pointers and other tracking information.
If \dusa{TRKNAM} also appears on the RHS, the previous tracking 
parameters will be applied by default on the current geometry.

\item[\dusa{GEONAM}] {\tt character*12} name of the \dds{geometry} data
structure.

\item[\dstr{desctrack}] structure describing the general tracking data (see
\Sect{TRKData})

\item[\dstr{descbivac}] structure describing the transport tracking data
specific to \moc{BIVACT:}.

\end{ListeDeDescription}

\vskip 0.2cm

The \moc{BIVACT:} specific tracking data in \dstr{descbivac} is defined as

\begin{DataStructure}{Structure \dstr{descbivac}}
$[$ $\{$ \moc{PRIM} $[$ \dusa{ielem} \dusa{icol} $]$ \\
~~~~$|$ \moc{DUAL} $[$ \dusa{ielem} \dusa{icol} $]$ \\
~~~~$|$ \moc{MCFD} $\}~]$ \\
$[~\{$ \moc{PN} $|$ \moc{SPN} $\}$ $[$ \moc{DIFF} $]$ \dusa{nlf} $[$ \moc{SCAT} \dusa{iscat} $]~[$ \moc{VOID} \dusa{nvd}~$]~]$ \\
{\tt ;}
\end{DataStructure}

\noindent where

\begin{ListeDeDescription}{mmmmmmm}

\item[\dstr{desctrack}] structure describing the general tracking data (see
\Sect{TRKData})

\item[\moc{PRIM}] keyword to set a primal finite element (classical)
discretization.

\item[\moc{DUAL}] keyword to set a mixed-dual finite element discretization. If the
geometry is hexagonal, a Thomas-Raviart-Schneider method is used.

\item[\moc{MCFD}] keyword to set a mesh-centered finite difference discretization
in hexagonal geometry.

\item[\dusa{ielem}] order of the finite element representation.  The values
permitted are: 1 (linear polynomials), 2 (parabolic polynomials), 3 (cubic
polynomials) or 4 (quartic polynomials). By default \dusa{ielem}=1.

\item[\dusa{icol}] type of quadrature used to integrate the mass matrices. The
values permitted are: 1 (analytical integration), 2  (Gauss-Lobatto quadrature)
or 3 (Gauss-Legendre quadrature). By default \dusa{icol}=2. The analytical
integration corresponds to classical finite elements; the Gauss-Lobatto
quadrature corresponds to a variational or nodal type collocation and the
Gauss-Legendre quadrature corresponds to superconvergent finite elements.

\item[\moc{PN}] keyword to set a spherical harmonics ($P_n$) expansion of the flux.\cite{nse2005} This option is currently limited to 1D
and 2D Cartesian geometries.

\item[\moc{SPN}] keyword to set a simplified spherical harmonics ($SP_n$) expansion
of the flux.\cite{nse2005,ane10a} This option is currently available with 1D and 2D Cartesian geometries
and with 2D hexagonal geometries.

\item[\moc{DIFF}] keyword to force using $1/3D^{g}$ as $\Sigma_1^{g}-\Sigma_{{\rm s}1}^{g}$ cross sections. A $P_1$ or $SP_1$ method
will therefore behave as diffusion theory.

\item[\dusa{nlf}] order of the $P_n$ or $SP_n$ expansion (odd number). Set to zero for diffusion theory (default value).

\item[\moc{SCAT}] keyword to limit the anisotropy of scattering sources.

\item[\dusa{iscat}] number of terms in the scattering sources. \dusa{iscat} $=1$ is used for
isotropic scattering in the laboratory system. \dusa{iscat} $=2$ is used for
linearly anisotropic scattering in the laboratory system. The default value is set to $n+1$
in $P_n$ or $SP_n$ case.

\item[\moc{VOID}] key word to set the number of base points in the Gauss-Legendre quadrature used to integrate
void boundary conditions if \dusa{icol} $=3$ and \dusa{n} $\ne 0$.

\item[\dusa{nvd}] type of quadrature. The values
permitted are: 0 (use a (\dusa{n}$+2$)--point quadrature consistent with $P_{{\rm n}}$ theory),
1 (use a (\dusa{n}$+1$)--point quadrature consistent with $S_{{\rm n}+1}$ theory),
2 (use an analytical integration of the void boundary conditions). By default \dusa{nvd}=0.

\end{ListeDeDescription}

Various finite element approximations can be obtained by combining different
values of \dusa{ielem} and \dusa{icol}:

\begin{itemize}

\item {\tt PRIM 1 1~:} Linear finite elements;

\item {\tt PRIM 1 2~:} Mesh corner finite differences;

\item {\tt PRIM 1 3~:} Linear superconvergent finite elements;

\item {\tt PRIM 2 1~:} Quadratic finite elements;

\item {\tt PRIM 2 2~:} Quadratic variational collocation method;

\item {\tt PRIM 2 3~:} Quadratic superconvergent finite elements;

\item {\tt PRIM 3 1~:} Cubic finite elements;

\item {\tt PRIM 3 2~:} Cubic variational collocation method;

\item {\tt PRIM 3 3~:} Cubic superconvergent finite elements;

\item {\tt PRIM 4 2~:} Quartic variational collocation method;

\item {\tt DUAL 1 1~:} Mixed-dual linear finite elements;

\item {\tt DUAL 1 2~:} Mesh centered finite differences;

\item {\tt DUAL 1 3~:} Mixed-dual linear superconvergent finite elements

(numerically equivalent to {\tt PRIM~1~3});

\item {\tt DUAL 2 1~:} Mixed-dual quadratic finite elements;

\item {\tt DUAL 2 2~:} Quadratic nodal collocation method;

\item {\tt DUAL 2 3~:} Mixed-dual quadratic superconvergent finite elements

(numerically equivalent to {\tt PRIM~2~3});

\item {\tt DUAL 3 1~:} Mixed-dual cubic finite elements;

\item {\tt DUAL 3 2~:} Cubic nodal collocation method;

\item {\tt DUAL 3 3~:} Mixed-dual cubic superconvergent finite elements

(numerically equivalent to {\tt PRIM~3~3});

\item {\tt DUAL 4 2~:} Quartic nodal collocation method;

\end{itemize}
\eject
 % structure (bivacT)
\subsubsection{The {\tt TRIVAT:} tracking module}\label{sect:TRIVACData}

The {\tt TRIVAT:} module provides an implementation of the diffusion or simplified $P_n$ method. The {\tt TRIVAT:} module is
used to perform a TRIVAC-type ``tracking"  on a 1D/2D/3D regular Cartesian or hexagonal geometry.\cite{BIVAC,TRIVAC} The
geometry is analyzed and a LCM object with signature {\tt L\_TRIVAC} is created with the following information:

\begin{itemize}
\item Diagonal and hexagonal symmetries are unfolded and the mesh-splitting 
operations are performed. Volumes, material mixture and averaged flux recovery
indices are computed on the resulting geometry. \item A finite element
discretization is performed and the corresponding numbering is saved. \item The
unit finite element matrices (mass, stiffness, etc.) are recovered. \item
Indices related to an ADI preconditioning with or without supervectorization
are saved. \end{itemize}

The calling specification for this module is:

\begin{DataStructure}{Structure \dstr{TRIVAT:}}
\dusa{TRKNAM}
\moc{:=} \moc{TRIVAT:} $[$ \dusa{TRKNAM} $]$ 
\dusa{GEONAM} \moc{::}  \dstr{desctrack} \dstr{descTRIVAC}
\end{DataStructure}

\noindent  where
\begin{ListeDeDescription}{mmmmmmm}

\item[\dusa{TRKNAM}] {\tt character*12} name of the \dds{tracking} data
structure that will contain region volume and surface area vectors in
addition to region identification pointers and other tracking information.
If \dusa{TRKNAM} also appears on the RHS, the previous tracking 
parameters will be applied by default on the current geometry.

\item[\dusa{GEONAM}] {\tt character*12} name of the \dds{geometry} data
structure.

\item[\dstr{desctrack}] structure describing the general tracking data (see
\Sect{TRKData})

\item[\dstr{descTRIVAC}] structure describing the transport tracking data
specific to \moc{TRIVAT:}.

\end{ListeDeDescription}

\vskip 0.2cm

The \moc{TRIVAT:} specific tracking data in \dstr{descTRIVAC} is defined as

\begin{DataStructure}{Structure \dstr{descTRIVAC}}
$[~\{$ \moc{PRIM} $[$ \dusa{ielem} $]~|$ \moc{DUAL} $[$ \dusa{ielem} \dusa{icol} $]~|$ \moc{MCFD} $[$ \dusa{ielem} $]~|$ \moc{LUMP} $[$ \dusa{ielem} $]~\}~]$ \\
$[$ \moc{SPN} \dusa{n} $[$ \moc{SCAT} $[$ \moc{DIFF} $]$ \dusa{iscat} $]~[$ \moc{VOID} \dusa{nvd} $]~]$ \\
$[$ \moc{ADI} \dusa{nadi} $]$ \\
$[$ \moc{VECT} $[$ \dusa{iseg} $]~[$ \moc{PRTV} \dusa{impv} $]~]$ \\
{\tt ;}
\end{DataStructure}

\noindent where
\begin{ListeDeDescription}{mmmmmm}

\item[\dstr{desctrack}] structure describing the general tracking data (see
\Sect{TRKData})

\item[\moc{PRIM}] key word to set a discretization based on the variational collocation method.

\item[\moc{DUAL}] key word to set a mixed-dual finite element discretization. If the
geometry is hexagonal, a Thomas-Raviart-Schneider method is used.

\item[\moc{MCFD}] key word to set a discretization based  on the nodal
collocation method. The mesh centered finite difference approximation is the
default option and is generally set using {\tt MCFD~1}. The {\tt MCFD}
approximations are numerically equivalent to the {\tt DUAL} approximations
with \dusa{icol}=2; however, the {\tt MCFD} approximations are less
expensive. 

\item[\moc{LUMP}] key word to set a discretization  based on the nodal
collocation method with serendipity approximation. The serendipity
approximation is different from the \moc{MCFD} option in cases with \dusa{ielem}$\ge$2. This option is not available for hexagonal geometries.

\item[\dusa{ielem}] order of the finite element representation.  The values
permitted are: 1 (linear polynomials), 2 (parabolic polynomials), 3 (cubic
polynomials) or 4 (quartic polynomials). By default \dusa{ielem}=1.

\item[\dusa{icol}] type of quadrature used to  integrate the mass matrices.
The values permitted are: 1 (analytical integration), 2  (Gauss-Lobatto
quadrature) or 3 (Gauss-Legendre quadrature). By default \dusa{icol}=2. The
analytical integration corresponds to classical finite elements; the
Gauss-Lobatto quadrature corresponds to a variational or nodal type
collocation and the Gauss-Legendre quadrature corresponds to superconvergent
finite elements.

\item[\moc{SPN}] keyword to set a simplified spherical harmonics ($SP_n$) expansion
of the flux.\cite{nse2005,ane10a} This option is available with 1D, 2D and 3D Cartesian geometries and with 2D and 3D
hexagonal geometries.

\item[\dusa{n}] order of the $P_n$ or $SP_n$ expansion (odd number). Set to zero for diffusion theory (default value).

\item[\moc{SCAT}] keyword to limit the anisotropy of scattering sources.

\item[\moc{DIFF}] keyword to force using $1/3D^{g}$ as $\Sigma_1^{g}$ cross sections. A $P_1$ or $SP_1$ method
will therefore behave as diffusion theory.

\item[\dusa{iscat}] number of terms in the scattering sources. \dusa{iscat} $=1$ is used for
isotropic scattering in the laboratory system. \dusa{iscat} $=2$ is used for
linearly anisotropic scattering in the laboratory system. The default value is set to $n+1$
in $P_n$ or $SP_n$ case.

\item[\moc{VOID}] key word to set the number of base points in the Gauss-Legendre quadrature used to integrate
void boundary conditions if \dusa{icol} $=3$ and \dusa{n} $\ne 0$.

\item[\dusa{nvd}] type of quadrature. The values
permitted are: 0 (use a (\dusa{n}$+2$)--point quadrature consistent with $P_{\rm n}$ theory),
1 (use a (\dusa{n}$+1$)--point quadrature consistent with $S_{{\rm n}+1}$ theory),
2 (use an analytical integration of the void boundary conditions). By default \dusa{nvd}=0.

\item[\moc{ADI}] keyword to set the number of ADI iterations at the inner
iterative level.

\item[\dusa{nadi}] number of ADI iterations (default: \dusa{nadi} $=2$).

\item[\moc{VECT}] key word to set an ADI preconditionning with
supervectorization. By default, TRIVAC uses an ADI preconditionning without
supervectorization.

\item[\dusa{iseg}] width of a vectorial register. \dusa{iseg} is generally a multiple of 64. By default, \dusa{iseg}=64.

\item[\moc{PRTV}] key word used to set \dusa{impv}.

\item[\dusa{impv}] index used to control the  printing in supervectorization
subroutines. =0 for no print; =1 for minimum printing (default value); Larger
values produce increasing amounts of output.

\end{ListeDeDescription}

Various finite element approximations can be obtained with different values of \dusa{ielem}.

\eject
 % structure (trivacT)
 % structures (sybilT) ,(excelT), (mccgT), (snT) and (bivacT)
\subsection{The {\tt ASM:} module}\label{sect:ASMData}

We will now describe the assembly modules which can be used to prepare the
group-dependent complete collision probability or the assembly matrices required
by the flux solution module of DRAGON.  The assembly module {\tt ASM:} is
generally called after a tracking module; it recovers tracking lengths and
material numbers from the sequential tracking file and then computes the
collision probability or group--dependent system matrices under various
normalizations. The calling specifications are:

\begin{DataStructure}{Structure \dstr{ASM:}}
\dusa{PIJNAM} \moc{:=} \moc{ASM:} $[$ \dusa{PIJNAM} $]$ \dusa{LIBNAM} 
\dusa{TRKNAM} $[$ \dusa{TRKFIL} $]$ \moc{::} \dstr{descasm}
\end{DataStructure}

\noindent where
\begin{ListeDeDescription}{mmmmmmmm}

\item[\dusa{PIJNAM}] {\tt character*12} name of \dds{asmpij} data
structure containing the system matrices. If \dusa{PIJNAM} appears on the RHS,
the \dstr{descasm} information previously stored in \dusa{PIJNAM} is kept.

\item[\dusa{LIBNAM}] {\tt character*12} name of the \dds{macrolib} or
\dds{microlib} data structure that contains the
macroscopic cross sections (see \Sectand{MACData}{LIBData}).

\item[\dusa{TRKNAM}] {\tt character*12} name of the \dds{tracking} data
structure containing the tracking (see \Sect{TRKData}).

\item[\dusa{TRKFIL}] {\tt character*12} name of the sequential binary tracking
file used to store the tracks lengths. This file is given if and only if it was
required in the previous tracking module call (see \Sect{TRKData}).

\item[\dstr{descasm}] structure containing the input data to this module (see
\Sect{descasm}).

\end{ListeDeDescription}

\subsubsection{Data input for module {\tt ASM:}}\label{sect:descasm}

\begin{DataStructure}{Structure \dstr{descasm}}
$[$ \moc{EDIT} \dusa{iprint} $]$ \\
$[$ $\{$ \moc{ARM} $|$ \\
~~~~$\{$ \moc{PIJ} $|$ \moc{PIJK} $\}$ $[$ \moc{SKIP} $]$ \\
~~~~$[$ $\{$ \moc{NORM} $|$ \moc{ALBS} $\}$ $]$ \\
~~~~$[$ \moc{PNOR} $\{$ \moc{NONE} $|$ \moc{DIAG} $|$ \moc{GELB} $|$ \moc{HELI} $|$ \moc{NONL} $\}$ $]$ \\
$\}$ $]$ \\
$[~\{$ \moc{ECCO} $|$ \moc{HETE} $\}~]$ \\
{\tt ;}
\end{DataStructure}

\noindent
where

\begin{ListeDeDescription}{mmmmmmmm}

\item[\moc{EDIT}] keyword used to modify the print level \dusa{iprint}.

\item[\dusa{iprint}] index used to control the printing of this module. The
amount of output produced by this tracking module will vary substantially
depending on the print level specified. 

\item[\moc{ARM}] keyword to specify that an
assembly calculation is carried out without building the full collision
probability matrices. This option can only be used for a geometry tracked using
the \moc{SYBILT:} (with EURYDICE-2 option), \moc{MCCGT:} or \moc{SNT:} module. By default,
the \moc{PIJ} option is used.

\item[\moc{PIJ}] keyword to specify that the standard scattering-reduced collision
probabilities must be computed. This option cannot be used with the \moc{MCCGT:} and \moc{SNT:}
modules. This is the default option.

\item[\moc{PIJK}] keyword to specify that both the directional and standard
scattering-reduced collision probabilities must be computed. Moreover, the additional directional
collision probability matrix can only be used if \moc{HETE} is activated in
\Sect{FLUData}. Finally, the \moc{PIJK}
option is only available for 2--D geometries analyzed with the operator
\moc{EXCELT:} with collision probability option. By default, the \moc{PIJ}
option is used.

\item[\moc{SKIP}] keyword to specify that only the reduced collision
probability matrix $p^{g}_{ij}$ is to be computed. In general, the scattering
modified collision probability matrix $p^{g}_{s,ij}$ is also computed using:
  $$
p^{g}_{s,ij}=\left[ I-p^{g}_{ij} \Sigma^{g\to g}_{s0} \right] ^{-1}
p^{g}_{ij}
  $$
where $\Sigma^{g\to g}_{s0}$ is the within group isotropic scattering cross
section. When available, $p^{g}_{s,ij}$ is used in the flux solution module in
such a way that for the groups where there is no up-scattering, the thermal
iteration is automatically deactivated. In the case where the \moc{SKIP} option
is activated, the $p^{g}_{ij}$ matrix is used and thermal iterations are
required in every energy group. Consequently, the total number of inner
iterations is greatly increased.

\item[\moc{NORM}] keyword to specify that the scattering-reduced collision probability matrix is
to be normalized in such a way as to eliminate all neutron loss (even if the
region under consideration has external albedo boundary conditions which should
result in neutron loss). When used with a void boundary condition (zero reentrant
current), this option is equivalent to imposing  {\it a posteriori} a uniform
reentrant current.

\item[\moc{ALBS}] keyword to specify that a consistent Selengut normalization
of the scattering-reduced collision probability matrix is to be used both for the flux solution
module (see \Sect{FLUData}) and in the equivalence calculation (see
\Sect{EDIData}). This keyword results in storing the scattering-reduced escape probabilities
$W_{iS}$ in the record named {\tt 'DRAGON-WIS'}. For all the cases where this option is used, it is necessary to
define a geometry with \moc{VOID} external boundary conditions (see
\Sect{GEOData}).

\item[\moc{PNOR}] keyword to specify that the collision, leakage and escape
probability matrices are to be normalized in such a way as to satisfy explicitly
the neutron conservation laws. This option compensates for the errors which will
arise in the numerical evaluation of these probabilities and may result in
non-conservative collision probability matrices. The default option is now \moc{HELI} while it was
formerly \moc{GELB} ({\bf Revision 3.03}).

\item[\moc{NONE}] keyword to specify that the probability matrices are not to
be renormalized.

\item[\moc{DIAG}] keyword to specify that only the diagonal element of the
probability matrices will be modified in order to insure the validity of the
conservation laws.

\item[\moc{GELB}] keyword to specify that the Gelbard algorithm will be used
to normalize the collision probability matrices.\cite{RENOR} 

\item[\moc{HELI}] keyword to specify that the Helios algorithm will be used
to normalize the collision probability matrices.\cite{Helios} 

\item[\moc{NONL}] keyword to specify that a non-linear multiplicative
algorithm will be used to normalize the collision probability
matrices.\cite{RENOR} 

\item[\moc{ECCO}] keyword used to compute the $P_1$--scattering reduced
collision probability or system matrices required by the ECCO isotropic
streaming model. By default, this information is not calculated.

\item[\moc{HETE}] keyword used to compute the information required by a
method of characteristics (MOC) solution of the TIBERE anisotropic
streaming model. By default, this information is not calculated.

\end{ListeDeDescription}
\eject
 % structure (dragonA)
\subsection{The {\tt FLU:} module}\label{sect:FLUData}

The \moc{FLU:} module is used to solve the linear system of multigroup collision
probability or response matrix equations in DRAGON. Different types of solution are
available, such as fixed source problem, fixed source eigenvalue problem (GPT type) or
different types of eigenvalue problems. The calling specifications are:

\begin{DataStructure}{Structure \dstr{FLU:}}
\dusa{FLUNAM} \moc{:=} \moc{FLU:} $[~\{$ \dusa{FLUNAM} $|$ \dusa{FLUDSA} $\}~]$ \dusa{PIJNAM} 
\dusa{LIBNAM}  \dusa{TRKNAM} $[$ \dusa{TRKFIL} $]$ \\
$~~~~[~\{$ \dusa{TRKFLP} \dusa{TRKGPT} $|$ \dusa{SOUNAM} $\}~]$ \moc{::} \dstr{descflu}
\end{DataStructure}

\noindent where
\begin{ListeDeDescription}{mmmmmmmm}

\item[\dusa{FLUNAM}] {\tt character*12} name of the \dds{fluxunk} data structure
containing the solution ({\tt L\_FLUX} signature). If \dusa{FLUNAM} appears on
the RHS, the solution previously stored in \dusa{FLUNAM} (flux and buckling) is used to initialize
the new iterative process; otherwise, a uniform unknown vector and a zero buckling
are used.

\item[\dusa{FLUDSA}] {\tt character*12} name of the \dds{fluxunk} data structure
containing an initial approximation of the solution ({\tt L\_FLUX} signature). This solution
corresponds to a DSA-type simplified
calculation compatible with \dusa{FLUNAM}. This option is only available with a \moc{SNT:} tracking.

\item[\dusa{PIJNAM}] {\tt character*12} name of the \dds{asmpij} data
structure containing the group-dependent system
matrices ({\tt L\_PIJ} signature, see \Sect{ASMData}).

\item[\dusa{LIBNAM}] {\tt character*12} name of the \dds{macrolib} or \dds{microlib} data structure that contains the
macroscopic cross sections ({\tt L\_MACROLIB} or {\tt L\_LIBRARY} signature, see \Sectand{MACData}{LIBData}).
Module {\tt FLU:} is performing a {\sl direct} or {\sl adjoint} calculation, depending if the adjoint flag
is set to {\tt .false.} or {\tt .true.} in the {\tt STATE-VECTOR} record of the \dds{macrolib}.

\item[\dusa{TRKNAM}] {\tt character*12} name of the \dds{tracking} data
structure containing the tracking ({\tt L\_TRACK} signature, see \Sect{TRKData}).

\item[\dusa{TRKFIL}] {\tt character*12} name of the sequential binary tracking
file used to store the tracks lengths. This file is given if and only if it was
required in the previous tracking module call (see \Sect{TRKData}).

\item[\dusa{TRKFLP}] {\tt character*12} name of the \dds{fluxunk} data structure containing the
unperturbed flux used to decontaminate the GPT solution ({\tt L\_FLUX} signature). This object is
mandatory if and only if ``{\tt TYPE P}" is selected.

\item[\dusa{TRKGPT}] {\tt character*12} name of the \dds{source} data structure
containing the GPT fixed sources ({\tt L\_SOURCE} signature). This object is
mandatory if and only if ``{\tt TYPE P}" is selected.

\item[\dusa{SOUNAM}] {\tt character*12} name of the \dds{source} data structure
containing the fixed sources ({\tt L\_SOURCE} signature) used for a ``{\tt TYPE S}" calculation.
By default, piecewise-constant fixed sources available in the \dds{macrolib} (or \dds{microlib}) \dusa{LIBNAM}
are used.

\item[\dstr{descflu}] structure containing the input data to this module (see
\Sect{descflu}).

\end{ListeDeDescription}

\clearpage

\subsubsection{Data input for module {\tt FLU:}}\label{sect:descflu}

\begin{DataStructure}{Structure \dstr{descflu}}
$[$ \moc{EDIT} \dusa{iprint} $]$ \\
$[$ \moc{INIT} $\{$ \moc{OFF} $|$ \moc{ON} $|$ \moc{DSA} $\}~]$ \\
\moc{TYPE} $\{$ \moc{N} $|$ \moc{S} $|$ \moc{F} $|$ \moc{P} $|$ \moc{K} $[$ \dstr{descleak} $]$ $|$
$\{$\moc{B} $|$ \moc{L} $\}$ \dstr{descleak} $\}$ $]$  \\
$[$ \moc{EXTE} $[$ \dusa{maxout} $]~~[$ \dusa{epsout} $]~]$ \\
$[$ \moc{THER} $[$ \dusa{maxthr} $]~~[$ \dusa{epsthr} $]~]~~[$ \moc{REBA} $[$ \moc{OFF} $]~]$ \\
$[$ \moc{UNKT} $[$ \dusa{epsunk} $]~]$ \\
$[$ \moc{ACCE} \dusa{nlibre} \dusa{naccel}  $]$ \\
{\tt ;}
\end{DataStructure}

\goodbreak
\noindent where
\begin{ListeDeDescription}{mmmmmmm}

\item[\moc{EDIT}] keyword used to modify the print level \dusa{iprint}.

\item[\dusa{iprint}] index used to control the printing of this operator. The
amount of output produced by this operator will vary substantially
depending on the print level specified. 

\item[\moc{OFF}] keyword to specify that the neutron flux
is to be initialized with a flat distribution (default option).

\item[\moc{ON}] keyword to specify that the initial neutron flux distribution
is to be recovered from \dusa{FLUNAM} if present in the RHS arguments.

\item[\moc{DSA}] keyword to specify that the initial neutron flux distribution
is to be recovered from the DSA compatible data structure \dusa{FLUDSA} if present in the RHS arguments.
This option is only available with a \moc{SNT:} tracking.

\item[\moc{TYPE}] keyword to specify the type of solution used in the flux
operator.

\item[\moc{N}] keyword to specify that no flux calculation is to be performed.
This option is usually activated when one simply wishes to initialize the
neutron flux distribution and to store this information in \dusa{FLUNAM}.

\item[\moc{S}]  keyword to specify that a fixed source problem is to be
treated. Such problem can also include fission source contributions.

\item[\moc{F}] keyword to specify that a 1D Fourier analysis calculation in $S_n$ is to be treated. This is similar to a fixed source problem, but the calculation stopped early to compute an L2 error norm in the flux. This yields a numerical estimate of the eigenvalue for the scattering source equation.

\item[\moc{P}]  keyword to specify that a fixed source eigenvalue problem (GPT type) is to be
treated. Such problem includes fission source contributions in addition of GPT sources.

\item[\moc{K}] keyword to specify that a fission source eigenvalue problem is
to be treated. The eigenvalue is then the effective multiplication factor $K_{\rm eff}$ with a
fixed buckling $B^2$. In this case, the fixed sources, if any is present on the
\dds{macrolib} or \dds{microlib} data structure, are not used.  

\item[\moc{B}] keyword to specify that a fission source eigenvalue problem is
to be treated. The eigenvalue in this case is the critical buckling $B^2$ with a fixed
effective multiplication factor $K_{\rm eff}$. The buckling eigenvalue has meaning only in the
case of a cell without boundary leakages (see the structure \dstr{descBC} in
\Sect{descBC}). It is also possible to use an open geometry with
\moc{VOID} boundary conditions  provided it is closed by the \moc{ASM:} module
(see \Sect{descasm}) using the keywords \moc{NORM} or \moc{ALSB}. {\sl Note:} \moc{TYPE~B}
cannot be used if no fission occurs in the system.

\item[\moc{L}] keyword to specify that a critical medium eigenvalue problem, with or without
fission sources, is to be treated. The eigenvalue in this case is the critical buckling $B^2$,
with or without a fixed effective multiplication factor $K_{\rm eff}$. The buckling eigenvalue has meaning only
in the case of a cell without boundary leakages (see the structure \dstr{descBC} in
\Sect{descBC}). It is also possible to use an open geometry with
\moc{VOID} boundary conditions  provided it is closed by the \moc{ASM:} module
(see \Sect{descasm}) using the keywords \moc{NORM} or \moc{ALSB}. {\sl Note:} \moc{TYPE~L}
cannot be used if no positive or negative $dB^2$ leakage occurs in the system.

\item[\dstr{descleak}] structure describing the general leakage parameters
options (see \Sect{descleak}). This information is mandatory for producing the
diffusion coefficients.

\item[\moc{EXTE}] keyword to specify that the control parameters for the
external iteration are to be modified. 

\item[\dusa{maxout}] maximum number of external iterations. The fixed default
value for a case with no leakage model is \dusa{maxout}=$2\times n_{f}-1$ where
$n_{f}$ is the number of regions containing fuel. The fixed default value for a
case with a leakage model is \dusa{maxout}=$10\times n_{f}-1$.

\item[\dusa{epsout}] convergence criterion for the external iterations. The
fixed default value is \dusa{epsout}=$5.0\times 10^{-5}$.

\item[\moc{THER}] keyword to specify that the control parameters for the
thermal iterations are to be modified.

\item[\dusa{maxthr}] maximum number of thermal iterations. The fixed default
value is \dusa{maxthr}=2$\times$\dusa{ngroup}-1 (using scattering modified CP)
or \dusa{maxthr}=4$\times$\dusa{ngroup}-1 (using standard CP).

\item[\dusa{epsthr}] convergence criterion for the thermal iterations. The
fixed default value is \dusa{epsthr}=$5.0\times 10^{-5}$.

\item[\moc{UNKT}] keyword to specify the flux error tolerance in
the outer iteration.

\item[\dusa{epsunk}] convergence criterion for flux components in the outer
iteration. The fixed default value is \dusa{epsunk}=\dusa{epsthr}.

\item[\moc{REBA}] keyword used to specify that the flux rebalancing option is
to be turned on or off in the thermal iteration. By default (floating default)
the flux rebalancing option is initially activated. This keyword is required to
toggle between the on and off position of the flux rebalancing option. 

\item[\moc{OFF}] keyword used to deactivate the flux rebalancing option. When
this keyword is absent the flux rebalancing option is reactivated.

\item[\moc{ACCE}] keyword used to modify the variational acceleration
parameters. This option is active by default (floating default) with
\dusa{nlibre}=3 free iterations followed by \dusa{naccel}=3 accelerated
iterations. 

\item[\dusa{nlibre}] number of free iterations per cycle of
\dusa{nlibre}+\dusa{naccel} iterations. 

\item[\dusa{naccel}] number of accelerated iterations per cycle of
\dusa{nlibre}+\dusa{naccel} iterations. Variational acceleration may be
deactivated by using \dusa{naccel}=0.

\end{ListeDeDescription}
\clearpage

\subsubsection{Leakage model specification structure}\label{sect:descleak}

Without leakage model, the multigroup flux $\vec\phi_g$ of the collision
probability method is obtained from equation

\begin{equation}
\vec\phi_g={\bf W}_g \vec Q^\diamond_g
\label{eq:eq3.64}
\end{equation}

\noindent where ${\bf W}_g$ is the scattering reduced collision probability matrix
and $ Q^\diamond_g$ is the fission and out-of-group scattering source. This equation is
modified by the leakage model. The leakage models \moc{PNLR}, \moc{PNL}, \moc{SIGS}
(default model), \moc{HETE} and \moc{ECCO} can be used with any solutions technique of
the Boltzmann transport equation. The leakage model \moc{TIBERE} can be used with the collision
probability method and with the method of characteristics.

\vskip 0.2cm

A leakage model can be set in {\sl fundamental mode condition} if all boundary conditions are
conservative (such as \moc{REFL}, \moc{SYME}, \moc{SSYM}, \moc{DIAG}, \moc{ALBE 1.0}). If a boundary condition is
non-conservative (such as \moc{VOID}), it is nevertheless possible to set a simplified leakage model based on the
Todorova approximation with option \moc{HETE}. The \dstr{descleak} structure allows the following
information to be specified:

\begin{DataStructure}{Leakage structure \dstr{descleak}}
$\{$ \moc{LKRD} $|$ \moc{RHS} $|$ \moc{P0} $|$ \moc{P1} $|$ \moc{P0TR} $|$ \moc{B0} $|$ \moc{B1} $|$ \moc{B0TR} $\}$ \\
$[~\{$ \moc{PNLR} $|$ \moc{PNL} $|$ \moc{SIGS} $|$ \moc{ALBS} $|$ \moc{HETE} $[$ (\dusa{imergl}(ii),ii=1,nbmix) $]~|$ \moc{ECCO} $|$ \moc{TIBERE}
$[$ $\{$ \moc{G} $|$ \moc{R} $|$ \moc{Z} $|$ \moc{X} $|$ \moc{Y} $\}~]~\}~]$ \\
$[$ $\{$ \moc{BUCK} $\{$ \dusa{valb2} $|$ $[$ \moc{G} \dusa{valb2} $]$  
$[$ \moc{R} \dusa{valbr2} $]$ $[$ \moc{Z} \dusa{valbz2} $]$ 
$[$ \moc{X} \dusa{valbx2} $]$
$[$ \moc{Y} \dusa{valby2} $]$ $\}$
$|$ \moc{KEFF} \dusa{valk} $|$ \moc{IDEM} $\}$
$]$  \end{DataStructure}

\begin{ListeDeDescription}{mmmmmmm}

\item[\moc{LKRD}] keyword used to specify that the leakage coefficients are
recovered from data structure named \dusa{LIBNAM}. The \moc{LKRD} option is not
available with the \moc{ECCO} and \moc{TIBERE} leakage models.

\item[\moc{RHS}] keyword used to specify that the leakage coefficients are
recovered from RHS flux data structure named \dusa{FLUNAM}. The \moc{RHS} option is not
available with the \moc{ECCO} and \moc{TIBERE} leakage models. If the flux calculation is
an adjoint calculation, the energy group ordering of the leakage coefficients is permuted.

\item[\moc{P0}] keyword used to specify that the leakage coefficients are
calculated using a $P_0$ model.

\item[\moc{P1}] keyword used to specify that the leakage coefficients are
calculated using a $P_1$ model. 

\item[\moc{P0TR}] keyword used to specify that the leakage coefficients are
calculated using a $P_0$ model with transport correction.

\item[\moc{B0}] keyword used to specify that the leakage coefficients are
calculated using a $B_0$ model. This is the default value when a buckling
calculation is required (\moc{B}).

\item[\moc{B1}] keyword used to specify that the leakage coefficients are
calculated using a $B_1$ model.

\item[\moc{B0TR}] keyword used to specify that the leakage coefficients are
calculated using a $B_0$ model with transport correction.

\item[\moc{PNLR}] keyword used to specify that the elements of the scattering
modified collision probability matrix
are multiplied by the adequate non-leakage homogeneous buckling dependent
factor.\cite{ALSB1}. The non-leakage
factor $P_{{\rm NLR},g}$ is defined as

\begin{equation}
P_{{\rm NLR},g}={\bar\Sigma_g-\bar\Sigma_{{\rm s0},g \gets g}\over{\bar\Sigma_g-\bar\Sigma_{{\rm s0},g \gets g}+d_g(B) \ B^2}}
\end{equation}

\noindent where transport-corrected total
cross sections are used to compute the ${\bf W}_g$ matrix. $\bar\Sigma_{{\rm s0},g \gets g}$ is the average
transport-corrected macroscopic within-group scattering cross section in group $g$,
homogenized over the lattice and transport corrected. \eq(eq3.64) is then replaced by
 
\begin{equation}
\vec\phi_g=P_{{\rm NLR},g} {\bf W}_g \vec Q^\diamond_g \ \ \ .
\label{eq:eq5.32}
\end{equation}

\item[\moc{PNL}] keyword used to specify that the elements of the collision
probability matrix are multiplied by the adequate non-leakage homogeneous buckling
dependent factor.\cite{ALSB1}. The non-leakage factor $P_{{\rm NL},g}$ is defined as

\begin{equation}
P_{{\rm NL},g}={\bar\Sigma_g\over{\bar\Sigma_g+d_g(B) \ B^2}}
\end{equation}

\noindent where $\bar\Sigma_g$ is the average transport-corrected macroscopic total cross section
in group $g$, homogenized over the lattice and transport corrected. \eq(eq3.64) is then replaced by

\begin{equation}
\vec\phi_g={\bf W}_g \left[ P_{{\rm NL},g} \vec Q^\diamond_g -(1-P_{{\rm NL},g}) {\bf \Sigma}_{{\rm s0},g\gets g} \ \vec\phi_g \right]
\label{eq:eq5.33b}
\end{equation}

\noindent where ${\bf \Sigma}_{{\rm s0},g\gets g}={\rm diag} \{ \Sigma_{{\rm s0},i,g \gets g}\> ;\> \forall i \}$
and the total cross sections used to compute the ${\bf W}_g$ matrix are also
transport-corrected.

\vskip 0.02cm

\noindent It is important to note that that the \moc{PNLR} option reduces to the \moc{PNL} option in
cases where no scattering reduction is performed. Scattering reduction can be avoided in module
\moc{ASM:} by setting {\tt PIJ SKIP} (See \Sect{descasm}).

\item[\moc{SIGS}] keyword used to specify that an homogeneous buckling
correction is to be applied on the diffusion cross section ($\Sigma_{s} -
dB^{2}$). \eq(eq3.64) is then replaced by

\begin{equation}
\vec\phi_g={\bf W}_g\left[ \vec Q^\diamond_g-d_g(B) \ B^2 \ \vec\phi_g\right]
\label{eq:eq5.33}
\end{equation}

\noindent where transport-corrected total
cross sections are used to compute the ${\bf W}_g$ matrix. This is the so called
{\sl DIFFON method} used in the APOLLO-family of thermal lattice codes. The \moc{SIGS} option is
the default option when a buckling calculation is required (\moc{TYPE B} or \moc{TYPE L}) or a
fission source eigenvalue problem (\moc{TYPE K}) with imposed buckling is considered.

\item[\moc{ALBS}] keyword used to specify that an homogeneous buckling
contribution is introduced by a group dependent correction of the
albedo.\cite{ALSB2} This leakage model is restricted to the collision probability
method. It is then necessary to define the geometry with an
external boundary condition of type \moc{VOID} (see \Sect{descBC}) and to close
the region in module \moc{ASM:} using the \moc{ALBS} option (see
\Sect{descasm}). \eq(eq3.64) is then replaced by

\begin{equation}
\vec\phi_g={\bf W}_g \ \vec Q^\diamond_g-\left[ {\bf I}+{\bf W}_g{\bf \Sigma}_{{\rm s0},g\gets g}\right] d_g(B) \ B^2 
\ \gamma \ {\bf P}_{{\rm iS},g}
\label{eq:eq5.34}
\end{equation}

\noindent where ${\bf P}_{{\rm iS},g}=\{P_{{\rm iS},g} \ ; \ i=1,I \}$ is the array of escape
probabilities in the open geometry and where

\begin{equation}
\gamma={\sum\limits_j V_j \phi_{j,g} \over \sum\limits_j V_j \phi_{j,g} P_{{\rm jS},g}} \ \ \ .
\label{eq:eq5.35}
\end{equation}

\item[\moc{HETE}] keyword used to perform a simplified heterogeneous leakage calculation, over one or many leakage zones, based
on the Todorova approximation.\cite{todorova} A leakage zone is a set of material mixtures where the leakage coefficient $d_{i,g}$ is
forced to be uniform in each energy group. Such a model is usefull to represent axial leakage in a {\tt TYPE~K} calculation or to
perform colorset calculations with more than one leakage zone. The \moc{HETE} leakage model can be used as an homogeneous model
assuming uniform leakage across the complete domain or as an heterogeneous model with more leakage zones defined using $\dusa{imergl}$
information. If a boundary condition is non-conservative (such as \moc{VOID}), it is nevertheless possible to use the \moc{HETE} option
with a $P_n$ or $B_n$ leakage model.

\item[\dusa{imergl}] array of homogenized leakage zone indices to which are associated the material mixtures. \dusa{nbmix} is the
total number of material mixtures. By default, a unique leakage zone is set. In this case, option $\moc{HETE}$ reduces to option $\moc{SIGS}$.

The simplified heterogeneous leakage model is based on a generalization of \eq(eq5.33), now written as
\begin{equation}
\vec\phi_g={\bf W}_g\left[ \vec Q^\diamond_g-B^2 \ \vec J_g\right]
\label{eq:eq5.36a}
\end{equation}

\noindent where each component of vector $\vec J_g$ is defined in term of heterogeneous leakage coefficients $d_{i,g}$ as
\begin{equation}
J_{i,g}=d_{i,g} \phi_{i,g}.
\label{eq:eq5.36b}
\end{equation}

A leakage zone index $m$ is assigned to each region $i$ using \dusa{imergl} information. In a colorset calculation, leakage zones 1 and 2
are assigned to black and red assemblies, respectively. In the $P_0$ and $B_0$ cases, the heterogeneous leakage coefficients in each leakage zone $m$
are obtained using the {\sl outscatter} approximation as
\begin{equation}
d_{m,g} = {1\over 3\gamma(B,\bar\Sigma_{m,g})}\left[ {\left<\phi_g\right>_m\over \left<\Sigma_g\phi_g\right>_m}\right]={1\over 3\gamma(B,\bar\Sigma_{m,g})\bar\Sigma_{m,g}}
\label{eq:eq5.36c}
\end{equation}
\noindent where $\left<\phi_g\right>_m$ is the integrated flux in leakage zone $m$ and $\left<\Sigma_g\phi_g\right>_m$ is a reaction rate
in zone $m$. The $\gamma(B,\bar\Sigma_{m,g})$ factor is equal to one with $P_n$ leakage models or to a leakage-zone dependent value with $B_n$
leakage models.\cite{PIP2009} Here, $\bar\Sigma_{m,g}$ is the leakage-zone averaged macroscopic total cross section in group $g$ defined as
\begin{equation}
\bar\Sigma_{m,g}={\left<\Sigma_g\phi_g\right>_m \over \left<\phi_g\right>_m}.
\label{eq:eq5.36d}
\end{equation}
\
In the $P_1$ and $B_1$ cases, the leakage coefficients are given as the solution of the following implicit equation, known as the {\sl inscatter} approximation:
\begin{equation}
d_{m,g}\left<\Sigma_g\phi_g\right>_m = {1\over \gamma(B,\bar\Sigma_{m,g})}\left[ {\left<\phi_g\right>_m\over 3}+
\sum_{h=1}^G \, d_{m,h} \left<\Sigma_{{\rm s1},g \leftarrow h}\phi_h\right>_m\right] .
\label{eq:eq5.36e}
\end{equation}

In transport-corrected $P_0$ and $B_0$ cases, we use the micro-reversibility principle, written as
\begin{equation}
\sum_{h=1}^G \Sigma_{{\rm s1},i,g \leftarrow h} J_{i,h} =\sum_{h=1}^G \Sigma_{{\rm s1},i,h \leftarrow g} J_{i,g}=\Sigma_{{\rm s1},i,g} J_{i,g} .
\label{eq:eq5.36f}
\end{equation}

Substitution of \eq(eq5.36f) into \eq(eq5.36e) leads to
\begin{equation}
d_{m,g} = {1\over 3}\left[ {\left<\phi_g\right>_m\over \gamma(B,\bar\Sigma_{m,g})\left<\Sigma_g\phi_g\right>_m-\left<\Sigma_{{\rm s1},g}\phi_g\right>_m}\right]=
{1\over 3\left[\gamma(B,\bar\Sigma_{m,g})\bar\Sigma_{m,g}-\bar\Sigma_{{\rm s1},m,g}\right]} .
\label{eq:eq5.36c}
\end{equation}

\item[\moc{ECCO}] keyword used to perform an ECCO--type leakage
calculation taking into account isotropic streaming effects. This method
introduces an heterogeneous buckling contribution as a group dependent correction
to the source term.\cite{ecco,rimpault} It is then necessary to set the keyword \moc{ECCO}
in module \moc{ASM:} (see \Sect{descasm}). In the $P_1$ non--consistent case,
\eq(eq3.64) is then replaced by

\vskip -0.3cm

\begin{eqnarray}
\vec\varphi_g&=& {\bf W}_g \left(\vec Q^\diamond_g - B^2 \ {i\vec{\cal J}_g\over B}\right)
\label{eq:eq5.37flux} \\
{i\vec{\cal J}_g\over B} &=& {\bf X}_g \left[{1 \over 3}
\ \vec\varphi_g + \sum_{h\not= g} {\bf \Sigma}_{{\rm s1},g \gets h} \
{i\vec{\cal J}_h\over B} \right]
\label{eq:eq5.37cour}
\end{eqnarray}

\noindent where $i\vec{\cal J}_{j,g}/B$ is the multigroup fundamental current, ${\bf \Sigma}_{{\rm s1},g \gets h}={\rm diag}\{ \Sigma_{{\rm s1},i,g \gets h}\> ;\> \forall i \}$ and where

\begin{equation}
{\bf X}_g=[{\bf I}-{\bf p}_g \ {\bf\Sigma}_{{\rm s}1,g\gets g}]^{-1} {\bf p}_g \ \ \ .
\label{eq:eq5.37ter}
\end{equation}

\item[\moc{TIBERE}] keyword used to perform a TIB\`ERE--type leakage
calculation taking into account anisotropic streaming effects. This method
introduces an heterogeneous buckling contribution as a group dependent correction
to the source term.\cite{PIJK0,PIJK} The heterogeneous buckling contribution is
introduced in the $B_n$ model using directional collision probabilities (PIJK method).
It is then necessary to set the keyword
\moc{PIJK} in module \moc{ASM:} (see \Sect{descasm}).

\item[\moc{G}] keyword used to specify that the buckling search will assume
all directional buckling to be identical (floating default option).

\item[\moc{R}] keyword used to specify that a radial buckling search will be
considered assuming an imposed $z$-direction buckling.

\item[\moc{Z}] keyword used to specify that a $z$-direction buckling search
will be considered  assuming an imposed $x$-direction and $y$-direction
buckling.

\item[\moc{X}] keyword used to specify that a $x$-direction buckling search
will be considered  assuming an imposed $y$-direction and $z$-direction
buckling.

\item[\moc{Y}] keyword used to specify that a $y$-direction buckling search
will be considered  assuming an imposed $x$-direction and $z$-direction
buckling.

\item[\moc{BUCK}] keyword used to specify the initial (for a buckling
eigenvalue problem) or fixed (for a effective multiplication factor eigenvalue
problem) buckling. 

\item[\moc{G}] keyword used to specify that the buckling in the $x$-direction,
$y$-direction and $z$-direction are to be initialized to \dusa{valb2}/3
(floating default).

\item[\moc{R}] keyword used to specify that the buckling in the $x$-direction,
and $y$-direction are to be initialized to \dusa{valbr2}/2.

\item[\moc{Z}] keyword used to specify that the buckling in the $z$-direction,
is to be initialized to \dusa{valbz2}.

\item[\moc{X}] keyword used to specify that the buckling in the $x$-direction,
is to be initialized to \dusa{valbx2}.

\item[\moc{Y}] keyword used to specify that the buckling in the $y$-direction,
is to be initialized to \dusa{valby2}.

\item[\dusa{valb2}] value of the fixed or initial total buckling in $cm^{-2}$.
The floating default value is
$${\it valb2}={\it valbx2}+{\it valby2}+{\it valbz2}.$$

\item[\dusa{valbr2}] value of the fixed or initial radial buckling in
$cm^{-2}$. The floating default value is
$${\it valbr2}={\it valbx2}+{\it valby2}.$$

\item[\dusa{valbz2}] value of the fixed or initial $z$-direction buckling in
$cm^{-2}$. The floating default value is \dusa{valbz2}=0.0 $cm^{-2}$. If
\dusa{valb2} is specified then \dusa{valbz2}=\dusa{valb2}/3.

\item[\dusa{valbx2}] value of the fixed or initial $z$-direction buckling in
$cm^{-2}$. The floating default value is \dusa{valbx2}=0.0 $cm^{-2}$. If
\dusa{valb2} is specified then \dusa{valbx2}=\dusa{valb2}/3. If \dusa{valbr2} is
specified then \dusa{valbx2}=\dusa{valbr2}/2.

\item[\dusa{valby2}] value of the fixed or initial $z$-direction buckling in
$cm^{-2}$. The floating default value is \dusa{valby2}=0.0 $cm^{-2}$. If
\dusa{valb2} is specified then \dusa{valby2}=\dusa{valb2}/3. If \dusa{valbr2} is
specified then \dusa{valby2}=\dusa{valbr2}/2.

\item[\moc{KEFF}] keyword used to specify the fixed (for a buckling eigenvalue
problem) effective multiplication factor. 

\item[\dusa{valk}] value of the fixed effective multiplication factor $K_{\rm eff}$. The
fixed default value is \dusa{valk}=1.0.

\item[\moc{IDEM}] keyword used to specify that the initial (for a buckling
eigenvalue problem) or fixed (for a effective multiplication factor eigenvalue
problem) buckling is to be read from the data structure \dusa{LIBNAM}. 

\end{ListeDeDescription}
\eject
 % structure (dragonF)
\subsection{The {\tt EDI:} module}\label{sect:EDIData}

The \moc{EDI:} module supplies the main editing options to DRAGON. It can be
use to compute the reaction rates, average and condensed cross sections to store
this information on a file for further use. The calling specifications are:

\begin{DataStructure}{Structure \dstr{EDI:}}
\dusa{EDINAM} \moc{:=} \moc{EDI:} $[$ \dusa{EDINAM} $]$
\dusa{LIBNAM} $[$ \dusa{TRKNAM} \dusa{FLUNAM} $]$ \\
~~~~~$[$ \dusa{REFGEO} $[$ \dusa{MACROGEO} $]~]~[$ \dusa{REFPIJ} $]~[$ \dusa{SURFIL} $]$ \moc{::} \dstr{descedi}
\end{DataStructure}

\noindent
where
\begin{ListeDeDescription}{mmmmmmmm}

\item[\dusa{EDINAM}] {\tt character*12} name of the \dds{edition} data
structure ({\tt L\_EDIT} signature) where the edition results will be stored.

\item[\dusa{LIBNAM}] {\tt character*12} name of the read-only \dds{macrolib} or
\dds{microlib} data structure ({\tt L\_MACROLIB} or {\tt L\_LIBRARY} signature) that contains the
macroscopic cross sections (see \Sectand{MACData}{LIBData}).

\item[\dusa{TRKNAM}] {\tt character*12} name of the read-only \dds{tracking} data
structure ({\tt L\_TRACK} signature) containing the tracking (see \Sect{TRKData}). {\bf Note:} If data
structures \dusa{TRKNAM} and \dusa{FLUNAM} are not given, a flux is recovered from the \dds{macrolib}
present in \dusa{LIBNAM} and used to perform the editions.

\item[\dusa{FLUNAM}] {\tt character*12} name of the read-only \dds{fluxunk} data
structure ({\tt L\_FLUX} signature) containing a transport solution (see \Sect{FLUData}).

\item[\dusa{REFGEO}] {\tt character*12} optional name of the read-only reference \dds{geometry} data
structure ({\tt L\_GEOM} signature) that was used for the original flux calculation (see \Sect{GEOData}).

\item[\dusa{MACROGEO}] {\tt character*12} optional name of the read-only macro-\dds{geometry} data
structure ({\tt L\_GEOM} signature) that is saved in \dusa{EDINAM} and can be used in the homogenization
process or in the SPH equivalence procedure. In some cases the
module \moc{EDI:} can automatically build a macro-geometry, however it is always
possible to specify explicitly the macro-geometry to be saved in \dusa{EDINAM}.

\item[\dusa{REFPIJ}] {\tt character*12} optional name of the read-only \dds{asmpij} data
structure ({\tt L\_PIJ} signature) that was used for the reference flux calculation (see \Sect{ASMData}).
Compulsory if keyword \moc{ALBS} is used in \Sect{descedi}.

\item[\dusa{SURFIL}] \texttt{character*12} name of the read-only SALOME--formatted sequential {\sc ascii}
file used to store the surfacic elements of the geometry. This file is required if and only if the keyword \moc{G2S}
is set in data structure \dstr{descedi}.

\item[\dstr{descedi}] structure containing the input data to this module
(see \Sect{descedi}).

\end{ListeDeDescription}

\clearpage

\subsubsection{Data input for module {\tt EDI:}}\label{sect:descedi}

\begin{DataStructure}{Structure \dstr{descedi}}
$[$ \moc{EDIT} \dusa{iprint} $]$ \\
$[$ \moc{UPS} $]$ \\
$[$ \moc{MERG} $\{$ \moc{NONE} $|$ \moc{COMP} $|$ \moc{GEO} $|$ \moc{HMIX} $|$ \\
\hskip 0.8cm \moc{G2S} \dusa{nhom} $[[$ \moc{RECT} \dusa{xm} \dusa{xp} \dusa{ym} \dusa{yp} $]]~[[$ \moc{TRIA} \dusa{x1} \dusa{y1} \dusa{x2} \dusa{y2} \dusa{x3} \dusa{y3} $]]~[$ \moc{REMIX} (\dusa{imix2}(ii),ii=1,nhom) $]~|$ \\
\hskip 0.8cm \moc{CELL} $[~\{$ \moc{SYBIL} $|$ \moc{EXCELL} $|$ \moc{NXT} $|$ \moc{DEFAULT} $|$ \moc{UNFOLD} $\}~]~[$ \moc{REMIX} (\dusa{imix2}(ii),ii=1,nbmix2) $]~|$ \\
\hskip 0.8cm \moc{REGI} (\dusa{iregm}(ii),ii=1,nregio) $|$ \\
\hskip 0.8cm \moc{MIX} $[$ (\dusa{imixm}(ii),ii=1,nbmix) $]~\}$ $]$ \\
$[$ \moc{TAKE} $\{$ \\
\hskip 0.8cm \moc{REGI} (\dusa{iregt}(ii),ii=1,nregio) $|$ \\
\hskip 0.8cm \moc{MIX} (\dusa{imixt}(ii),ii=1,nbmix) $\}$ $]$  \\
$[$ $\{$ \moc{P0W} $|$ \moc{P1W\_L} $|$ \moc{P1W\_TO} $|$ \moc{PNW\_SP} $\}$ $]$ \\
$[$ \moc{EDI\_CURR} $]~[$ \moc{GOLVER} $]$ \\
$[$ \moc{COND} $[~\{$  \moc{NONE} $|$ ( \dusa{icond}(ii), ii=1,ngcond) $|$ ( \dusa{energy}(ii), ii=1,ngcond) $\}~]~]$\\
$[$ \moc{MICR} $[$ \moc{ALLX} $]~[$ \moc{ISOTXS} $[$ \moc{ASCII} $]~]$ $\{$ \moc{ALL} $|$ \moc{RES} $|$ 
  \dusa{nis} (\dusa{HISO}(i),i=1,\dusa{nis}) $\}$\\
\hskip 0.8cm $[$ \moc{REAC} \dusa{nreac} (\dusa{HREAC}(i),i=1,\dusa{nreac}) $]~]$\\
$[$ \moc{ACTI} $[$ \moc{ISOTXS} $[$ \moc{ASCII} $]~]$ $\{$ \moc{NONE} $|$ (\dusa{imixa}(ii),ii=1,nbmix) $]$ $\}$\\ 
$[$ \moc{SAVE} $[$ \moc{ON} $\{$ \dusa{DIRN} $|$ \dusa{idirn} $\}$ $]$ $]$ \\
$[$ \moc{PERT} $]$ \\
$[$ \moc{STAT} $\{$ \moc{ALL} $|$ \moc{RATE} $|$ \moc{FLUX} $|$ \moc{DELS} $\}$  
  $[$ \moc{REFE} $\{$ \dusa{DIRO} $|$ \dusa{idiro} $\}$$]$ $]$ \\
$[$ \moc{NOHF} $]~[$ \moc{NBAL} $]$ \\
$[$ \moc{MAXR} \dusa{maxpts} $]$ \\
$[~\{$ \moc{DIRE} $|$ \moc{PROD} $\}~]$ \\
$[$ \moc{MGEO} \dusa{MACGEO} $]$ \\
$[~\{$ \moc{NADF} $|$ \moc{ALBS} $|$ \moc{JOUT} $|$ \moc{ADFM} $|$\\
~~~~~~~$[[$ \moc{ADF} $[$ \moc{*} $]$ \dusa{TYPE} $\{$ \moc{REGI} (\dusa{ireg}(ii),ii=1,iimax) \moc{ENDR} $|$ \moc{MIX} (\dusa{imix}(ii),ii=1,iimax) \moc{ENDM} $\}~]]~\}~]$ \\
$[$~\moc{LEAK}~\dusa{b2}~$]$
\end{DataStructure}

\noindent where

\begin{ListeDeDescription}{mmmmmmmm}

\item[\moc{EDIT}] keyword used to modify the print level \dusa{iprint}.

\item[\dusa{iprint}] index used to control the printing of this module. The
\dusa{iprint} parameter is important for adjusting the amount of data that is
printed by this calculation step:

\begin{itemize}

\item \dusa{iprint}=0 results in no output;

\item \dusa{iprint}=1 results in the average and integrated fluxes being printed
(floating default);

\item \dusa{iprint}=2 results in the reaction rates being printed; 

\item \dusa{iprint}=3 is identical to the previous option, but the condensed
and/or homogenized vectorial cross sections are also printed;

\item \dusa{iprint}=4 is identical to the previous option, but the  condensed
and/or homogenized transfer cross sections are also printed.

\end{itemize}

\item[\moc{UPS}] keyword to specify that the reaction rates and the condensed
and/or homogenized cross sections are corrected so as to eliminate
up-scattering. This option is useful for reactor analysis codes which cannot
take into account such cross sections. Scattering ($\sigma_{{\rm s},h\gets g}$), diffusion ($\sigma_{{\rm s},g}$)
and total ($\sigma_g$) cross sections are corrected as:
\begin{eqnarray*}
\tilde\sigma_{{\rm s},h\gets g}\negthinspace\negthinspace &=&\negthinspace\negthinspace \begin{cases} 0  & {\rm if} \ h < g\\
\sigma_{{\rm s},g\gets g} &{\rm if}  \ h = g\\
\sigma_{{\rm s},h\gets g}-\sigma_{{\rm s},g\gets h}\, {\phi_h\over \phi_g} & {\rm if}  \ h > g
\end{cases} \\
\tilde\sigma_{{\rm s},g}\negthinspace\negthinspace &=&\negthinspace\negthinspace \sum_{h=1}^G \tilde\sigma_{{\rm s},h\gets g}\, = \, \sigma_{{\rm s},g}-\sum_{h=1}^{g-1} \sigma_{{\rm s},h\gets g}-\sum_{h=g+1}^{G} \sigma_{{\rm s},g\gets h}\, {\phi_h\over \phi_g} \\
\tilde\sigma_{g}\negthinspace\negthinspace &=&\negthinspace\negthinspace \sigma_{g}-\sum_{h=1}^{g-1} \sigma_{{\rm s},h\gets g}-\sum_{h=g+1}^{G} \sigma_{{\rm s},g\gets h}\, {\phi_h\over \phi_g} .
\end{eqnarray*}

\item[\moc{NONE}] keyword to deactivate the homogeneization or the condensation. 

\item[\moc{MERG}] keyword to specify that the neutron flux is to be
homogenized over specified regions or mixtures. 

\item[\moc{REGI}] keyword to specify that the homogenization of the neutron
flux will take place over the following regions. Here nregio$\le$\dusa{maxreg}
with \dusa{maxreg} the maximum number of regions for which solutions were
obtained.

\item[\dusa{iregm}] array of homogenized region numbers to which are
associated the old regions. In the editing routines a value of \dusa{iregm}=0
allows the corresponding region to be neglected. 

\item[\moc{MIX}] keyword to specify that the homogenization of the neutron
flux will take place over the following mixtures. Here
we must have nbmix$\le$\dusa{maxmix} where \dusa{maxmix} is the maximum number
of mixtures in the macroscopic cross section library.  

\item[\dusa{imixm}] array of homogenized region numbers to which are
associated the material mixtures. In the editing routines a value of
\dusa{imixm}=0 allows the corresponding isotopic mixtures to be neglected. For a mixture in this
library which is not used in the geometry one should insert a value of 0 for the
new region number associated with this mixture. This option is also useful to homogenize the cross-section data of the second-level mixtures by combining the first-level mixtures in
a two-level computational scheme for a PWR assembly. By default, if \moc{MIX} is set and
\dusa{imixm} is not set, \dusa{imixm(ii)}$=$\dusa{ii} is assumed.

\item[\moc{COMP}] keyword to specify that the a complete homogenization is to
take place. 

\item[\moc{GEO}] keyword to specify that a geometry equivalence procedure (equigeom) is to be used. Merging indices
are automatically computed by comparing the reference geometry \dusa{REFGEO} with the macro-geometry \dusa{MACROGEO}.
This capability is limited to EXCELL--type reference geometries.

\item[\moc{G2S}] keyword to specify that the homogenization will be based on the geometry definition available in the surfacic
file \dusa{SURFIL}.

\item[\dusa{nhom}] number of homogeneous nodes to be defined using \moc{RECT} and/or \moc{TRIA} data structures. Many homogeneous mixtures can be defined by
repeating the \moc{RECT} and/or \moc{TRIA} data structures.

\item[\moc{RECT}] keyword to specify a unique homogeneous mixture based on a rectangular node. 

\item[\dusa{xm}] lower limit of the homogeneous node alonx X--axis.

\item[\dusa{xp}] upper limit of the homogeneous node alonx X--axis.

\item[\dusa{ym}] lower limit of the homogeneous node alonx Y--axis.

\item[\dusa{yp}] upper limit of the homogeneous node alonx Y--axis.

\item[\moc{TRIA}] keyword to specify a unique homogeneous mixture based on a triangular node.

\item[\dusa{x1}] X--coordinate of the first corner.

\item[\dusa{y1}] Y--coordinate of the first corner.

\item[\dusa{x2}] X--coordinate of the second corner.

\item[\dusa{y2}] Y--coordinate of the second corner.

\item[\dusa{x3}] X--coordinate of the third corner.

\item[\dusa{y3}] Y--coordinate of the third corner.

\item[\moc{HMIX}] keyword to specify that the homogenization region will be selected using the information provided by the \moc{HMIX} option in the \moc{GEO:} module (see \Sect{descPP}). In this case, all the regions associated with a virtual homogenization mixture will be homogenized. If the virtual homogenization mixtures were not defined in the geometry, the real mixtures are used instead (see \moc{MIX} keyword in \Sect{descPP}). This option is valid only for \moc{NXT:} based \dds{tracking} data structure (this option uses the information stored on the reference \dds{TRKNAM} data structure).

\item[\moc{CELL}] keyword to specify that the a cell-by-cell homogenization
(with or without SPH equivalence) is to take place. The macro-geometry and the merging indices are automatically
computed and the macro-geometry named {\tt MACRO-GEOM} is created on the root directory of \dusa{EDINAM}. This
capability is limited to reference geometries previously tracked by EURYDICE (see \Sect{SYBILData}) or NXT (see 
\Sect{NXTData}).

\item[\moc{SYBIL}] the macro-geometry produced by \moc{CELL} is tracked by {\tt SYBILT:} module.

\item[\moc{EXCELL}] the macro-geometry produced by \moc{CELL} is tracked by {\tt EXCELT:} module.

\item[\moc{NXT}] the macro-geometry produced by \moc{CELL} is tracked by {\tt NXT:} module.

\item[\moc{DEFAULT}] the macro-geometry produced by \moc{CELL} is tracked by another module (default option).

\item[\moc{UNFOLD}] the macro-geometry produced by \moc{CELL} is unfolded and tracked with the \moc{DEFAULT} option. This option is
useful with fine power reconstruction techniques.

\item[\moc{REMIX}] the homogenization produced by option \moc{MERG} \moc{G2S} or \moc{MERG} \moc{CELL} (cell-by-cell) is further
homogenized according to \dusa{imix2} indices. This option is useful to integrate the assembly gap into the boundary cells. By default, one homogenized region is created
for each region of the macro-geometry.

\item[\dusa{imix2}] array of rehomogenized region numbers to which are associated the regions indices created {\sl after}
the \moc{MERG} \moc{G2S} or \moc{MERG} \moc{CELL} homogenization was performed. In the editing routines a value of \dusa{imix2}=0 allows the corresponding
region to be neglected. Here, nbmix2 is equal to the number of mixtures in the geometry before the \moc{REMIX} operation is performed (equal to the number
of cells in the macro-geometry if \moc{MERG} \moc{CELL} was set).

\item[\moc{TAKE}] keyword to specify that the neutron flux is to be edited
over specified regions or mixtures. 

\item[\moc{REGI}] keyword to specify that the editing of the neutron flux will
take place over the following regions. Here nregio$\le$\dusa{maxreg}
with \dusa{maxreg} the maximum number of regions for which solutions were
obtained.

\item[\dusa{iregt}] regions where the editing will take place. The new region
numbers associated with these editing regions are numbered sequentially.

\item[\moc{MIX}] keyword to specify that the editing of the neutron
flux will take place over the following mixtures. Here
we must have nbmix$\le$\dusa{maxmix} where \dusa{maxmix} is the maximum number
of mixtures in the macroscopic cross section library.  

\item[\dusa{imixt}] mixtures where the editing will take place.
Each mixture set here must exists in the reference geometry.

\item[\moc{P0W}] keyword to specify that the $P_\ell$, $\ell\ge 1$ information is to be
homogenized and condensed using the scalar flux. This is the default option.

\item[\moc{P1W\_L}] keyword to specify that the $P_\ell$, $\ell\ge 1$ information is to be
homogenized and condensed using a current recovered from a consistent $P_1$ or
from a consistent heterogeneous $B_1$ model.

\item[\moc{P1W\_TO}] keyword to specify that the $P_\ell$, $\ell\ge 1$ information is to be
homogenized and condensed using the Todorova flux\cite{todorova}, defined as
$$
\phi_1(\bff(r),E)={\phi(\bff(r),E)\over \Sigma_i(E)-\Sigma_{{\rm s1},i}(E)}
$$
\noindent where $\Sigma_i(E)$ and $\Sigma_{{\rm s1},i}(E)$ are the macroscopic total and $P_1$ scattering
cross sections in the mixture $i$ containing the point $\bff(r)$. This option is not recommended.

\item[\moc{PNW\_SP}] keyword to specify that the $P_\ell$, $\ell\ge 1$ information is to be
homogenized and condensed using a weighting spectra based on the APOLLO3 averaging formula\cite{condPn}, defined as
$$
\phi_\ell(\bff(r),E)={ \displaystyle\sum_{m=-\ell}^\ell \phi_\ell^m(\bff(r),E) \left<\phi_\ell^m\right>_{G,M} \over \displaystyle\sum_{m=-\ell}^\ell \left<\phi_\ell^m\right>_{G,M} }
$$
where $\phi_\ell^m(\bff(r),E)$ are the spherical harmonic $\ell$-th moment of the flux with $E \in G$ and $\bff(r) \in M$. Here, $G$ is the
condensed macrogroup and $M$ is the homogenized mixture.

\item[\moc{EDI\_CURR}] keyword to specify the generation of integrated net currents (homogenized and condensed) in the macrolib along each axis.
This option is only provided with SN and MOC discretizations. By default, only integrated fluxes are generated.

\item[\moc{GOLVER}] keyword to specify the use of the Golfier-Vergain diffusion coefficient formula. This formula is written
$$D_{i,g}={\alpha_g\over 3\Sigma_{{\rm tr},i,g}}$$

\noindent with the Golfier-Vergain factors $\alpha_g$ defined as
$$\alpha_g={\sum_i  \int_{u_{g-1}}^{u_g} du \, {\displaystyle\phi_i(u) \over \displaystyle\left(\Sigma_i(u)-\Sigma_{{\rm s1},i}(u) \right) }
\over \sum_i {\displaystyle\phi_{i,g} \over \displaystyle\Sigma_{{\rm tr},i,g}} }$$

and where the multigroup transport cross sections are defined as
$$\Sigma_{{\rm tr},i,g}={\int_{u_{g-1}}^{u_g} du \left(\Sigma(u) -\Sigma_{{\rm s1},i}(u) \right) \phi_i(u)
\over \int_{u_{g-1}}^{u_g} du \, \phi_i(u) }.$$

By default, the diffusion coefficients are obtained by condensation of the fine-group leakage coefficients $d_i(u)$:
$$D_{i,g}={\int_{u_{g-1}}^{u_g} du \, d_i(u) \, \phi_i(u) \over \int_{u_{g-1}}^{u_g} du \, \phi_i(u) }.$$

\item[\moc{COND}] keyword to specify that a group condensation of the flux is to be performed.

\item[\dusa{icond}] array of increasing energy group limits that will be associated with
each of the ngcond condensed groups. The final value of
\dusa{icond} will automatically be set to \dusa{ngroup} while the values of 
\dusa{icond}$>$\dusa{ngroup} will be droped from the condensation. 
We must have ngcond$\le$\dusa{ngroup}. By default, if \moc{COND} is set and \dusa{icond}
is not set, all energy groups are condensed together.

\item[\dusa{energy}] array of decreasing energy limits (in eV) that will be
associated with each of the ngcond condensed groups. We must have ngcond$\le$\dusa{ngroup+1}. 
Note that if an energy limit is located between two energy groups, the condensation
group will include this associated energy group. In the case where two energy
limits fall within the same energy group the lowest energy will be droped.
Finally the maximum and minimum energy limits can be skipped since they will be
taken automatically from the information available in the library.

\item[\moc{MICR}] keyword to specify that the condensation and homogenization
procedure will be used to associate microscopic cross sections to the isotopes
present in the homogenized regions. The macroscopic cross sections and the
diffusion coefficients are weighted by the multigroup fluxes appearing in the
regions where the isotopes are present. The resulting nuclear properties are
saved on \dusa{EDINAM} when the \moc{SAVE} keyword is present.

\item[\moc{ALLX}] keyword used to register the region number of each isotope before merging, in the 
embedded library. The homogeneized information is therefore registered for each isotope in the merging
region, as depicted by the formulas below. This procedure is useful to produce particular databases, 
in order to perform micro-depletion calculations in diffusion with DONJON.

\item[\moc{ALL}] keyword to specify that all the isotopes present in the
homogenized region are to be kept individual and processed.

\item[\moc{RES}] keyword to specify that all the isotopes present in the
homogenized region will be merged as a single residual isotope.

\item[\dusa{nis}] number of isotopes present in the homogenized
region to be processed.

\item[\dusa{HISO}] array of {\tt character*8} isotopes alias names to be processed.

\item[\moc{REAC}] keyword to specify the reaction names to be included in the output microlib. By default, all available reactions
are included in the output microlib.

\item[\dusa{nreac}] number of reactions to be included in the output microlib.

\item[\dusa{HREAC}] array of {\tt character*8} reaction names to be included in the output microlib.

\item[\moc{ACTI}] keyword to specify that microscopic activation
data will be edited for the isotopes associated with the specified mixture. This
information correspond to the microscopic cross section associated with each
isotope in a given macro-group and macro-region assuming a concentration
for this isotope of 1.0 $\times{\it cm}^{-3}$ in each region. This keyword is
followed by nacti material mixture indices, where
nacti$\le$\dusa{maxmix}.

\item[\moc{NONE}] keyword to specify that no isotope present in the
homogenized region is to be used as activation data.

\item[\dusa{imixa}] array of material mixture indices which contains the
isotopes for which activation data is to be generated.
\dusa{nmix}$\le$\dusa{maxmix}. Even mixture not used in the geometry 
can be considered here.

\item[\moc{ISOTXS}] keyword to specify that the set of microscopic cross
section generated by the \moc{MICR} and \moc{ACTI} command will also
be saved on a microscopic group neutron cross section library in the ISOTXS-IV
format. This will generate a file for each final region specified by the
\moc{TAKE} or \moc{MERG} keyword, numbered consecutively ({\tt IFILE}). The name
of the file ({\tt NISOTXS}) is built using the command 

\begin{verbatim}
WRITE(NISOTXS,'(A6,I6.6)') 'ISOTXS',IFILE
\end{verbatim}

\item[\moc{ASCII}] keyword to specify that the ISOTXS file is created in ascii format.
By  default, it is created in binary format.

\item[\moc{SAVE}] keyword to specify that the fluxes, the macroscopic and
microscopic cross sections and the volumes corresponding to homogenized regions
are to be saved on \dusa{EDINAM}. A \dds{macrolib} is store on a subdirectory
of \dds{edition}.

\item[\moc{ON}] keyword to specify on which directory of \dusa{EDINAM} this
information is to be stored.

\item[\dusa{DIRN}] name of the directory on which the above information is to
be stored.

\item[\dusa{idirn}] number associated with a directory of \dusa{EDINAM} on
which the above information is to be stored. To each number \dusa{idirn} is
associated a directory name \moc{CDIRN}={\tt 'REF-CASE'//CN} where {\tt CN} is a
{\tt character*4} variable defined by {\tt WRITE(CN,'(I4)')} \moc{idirn}.

\item[\moc{PERT}] keyword to specify that first order perturbations for 
the microscopic cross sections are to be saved on \dusa{EDINAM}. 

\item[\moc{STAT}] keyword to specify that a comparison between the current and
a reference set of reaction rates and/or integrated fluxes is to be performed. 

\item[\moc{ALL}] keyword to specify that the relative differences in the
reaction rates and the integrated fluxes are to be printed.

\item[\moc{RATE}] keyword to specify that the relative differences in the
reaction rates are to be printed.

\item[\moc{FLUX}] keyword to specify that the relative differences in the
integrated fluxes are to be printed. 

\item[\moc{DELS}] keyword to specify that the absolute differences in the
macroscopic cross section are to be printed.

\item[\moc{REFE}] keyword to specify the directory of \dusa{EDINAM} where the
reference data requires for the comparison is stored. When this keyword is
absent, the last reaction rates and integrated fluxes saved on \dusa{EDINAM} are
used.

\item[\dusa{DIRO}] name of the directory from which the reference information
is taken.

\item[\dusa{idiro}] number associated with an directory of \dusa{EDINAM} on
which the reference information is  stored. To each number \dusa{idirn} is
associated a the directory  \moc{CDIRN}={\tt 'REF-CASE'//CN} where {\tt CN} is a
{\tt character*4} variable defined by {\tt WRITE(CN,'(I4)')} \moc{idirn}. 

\item[\moc{NOHF}] keyword to suppress the calculation and edition of the H-factors (sum of all
the cross sections producing energy times the energy produced by each reaction).
Note that this calculation may be time-consuming. By default, the H-factors are
computed and edited if keyword \moc{DEPL} and associated data is set in module {\tt LIB:}.

\item[\moc{NBAL}] keyword to specify the editing of the four factors computed
from a group balance. In this case, the user must specify explicitly a three
group condensation.

\item[\moc{MAXR}] keyword to specify the number of components in
region-related dynamically allocated arrays. If the default value is
not sufficient, an error message is issued.

\item[\dusa{maxpts}] user-defined maximum number of components.

\item[\moc{DIRE}] use the direct flux to perform homogenization or/and
condensation (default value).

\item[\moc{PROD}] use the product of the direct and adjoint flux to perform homogenization or/and
condensation. This option is used only in specialized applications such as in the {\sc clio} perturbative
analysis formula.\cite{clio} The homogenization and condensation equations are presented in Sect.~\ref{sect:prod}.
{\bf Note:} The \dusa{FLUNAM} object must contain both an adjoint and a direct flux solution.

\item[\moc{MGEO}] keyword to define the name of the macro-geometry, which must appear among the RHS. The macro-geometry is recovered automatically
by interface modules such as \moc{COMPO:} (see \Sect{COMPOData}) or manually by a CLE-2000 statement such as
\begin{verbatim}
GEONAM := EDINAM :: STEP UP 'MACRO-GEOM' ;
\end{verbatim}
\noindent where {\tt GEONAM} and {\tt EDINAM} are {\tt L\_GEOM} and {\tt L\_EDIT} LCM objects, respectively.

\item[\dusa{MACGEO}] character*12 name of the macro-geometry.

\item[\moc{NADF}] keyword to desactivate boundary editions.

\item[\moc{ALBS}] keyword to specify that the boundary flux is to be obtained from relation
$\phi_{\rm surf}=4J_{\rm out}/S$ where $J_{\rm out}$ is the outgoing interface current. The albedo of
the geometry are to be taken into account in the complete homogenization process. Thus the \moc{MERG}
and \moc{COMP} options must be specified. The boundary fluxes are obtained from a calculation using the collision
probabilities. This option requires a geometry with \moc{VOID} (see \Sect{descBC}) external boundary conditions to
be closed using \moc{ALBS} in module \moc{ASM:} (see \Sect{descasm}).\cite{ALSB2}

\item[\moc{JOUT}] keyword to recover multigroup boundary current information ($J_{\rm out}$ and $J_{\rm in}$). This keyword
is only compatible with \moc{MCCGT:} or \moc{SYBILT:} trackings and if keyword \moc{ARM} is set in module \moc{ASM:}
(see \Sect{descasm}). The outgoing/ingoing interface currents are recovered by direct homogenization and condensation of the
flux unknown components corresponding to external boundary and used with the current iteration method in Eurydice or from a MOC
calculation. The boundary flux required by the SPH method is to be obtained from relation $\phi_{\rm surf}=4J_{\rm out}/S$ where
$J_{\rm out}$ is the outgoing interface current. The net boundary current is to be obtained from relation
$J_{\rm net}=J_{\rm out}-J_{\rm in}$.

\item[\moc{ADFM}] keyword to specify that the ADF information is recovered from macrolib in RHS object \dusa{LIBNAM}. ADF information can
be defined as explained in Sect.~\ref{sect:descxs} of module {\tt MAC:} and recovered in module {\tt EDI:} for further processing.

\item[\moc{ADF}] keyword to specify that boundary editions are required. Averaged fluxes are
computed over boundary regions.

\item[\moc{*}] keyword to specify that boundary fluxes are divided by average assembly fluxes so as to produce {\sl assembly discontinuity factors}
(ADF). By default, boundary fluxes are recovered and saved in the boundary edit without further treatment.

\item[\dusa{TYPE}] {\tt character*8} name of the boundary edit corresponding to
regions \dusa{ireg} or mixtures \dusa{imix}. Any user-defined name can be used, but some
standard names are recognized by module \moc{SPH} (see \Sect{descsph}). Standard names are: $=$ \moc{FD\_C}:
corner flux edition; $=$ \moc{FD\_B}: surface (assembly gap) flux edition; $=$ \moc{FD\_H}:
row flux edition. These are the first row of surrounding cells in the assembly.

\item[\dusa{ireg}] index of a region of the reference geometry belonging to boundary edition.

\item[\dusa{imix}] index of a material mixture of the reference geometry belonging to boundary edition.

\item[\moc{LEAK}] keyword used to introduce leakage in the embedded {\sc macrolib}. This option should only be used for non-regression tests. The {\sc microlib} is not modified.

\item[\dusa{b2}] the imposed buckling corresponding to the leakage.

\end{ListeDeDescription}

\subsubsection{Homogenization and condensation with the flux}

The cross sections are homogenized over macro-volumes $V_{\rm merg}$ and condensed over
macro groups $E_{\rm merg}$. We also use $V_i$ to identify the subset of $V_{\rm merg}$ where
the isotope $i$ is defined. The module {\tt EDI:} produces the following homogenized/condensed information:

\begin{description}
\item[integrated volume:]
$$
\overline V=\int_{V_{\rm merg}} dV
$$

\item[macroscopic cross section of type $\bff(x)$:]
$$
\overline \Sigma_x = {\int_{V_{\rm merg}} dV \int_{E_{\rm merg}} dE \, \Sigma_x(\bff(r),E) \, \phi(\bff(r),E)
\over \int_{V_{\rm merg}} dV \int_{E_{\rm merg}} dE \, \phi(\bff(r),E)}
$$

\item[number density for isotope $\bff(i)$:]
$$
\overline N_i= {1\over \overline V} \int_{V_i} dV N_i(\bff(r))
$$
\noindent where $N_i(\bff(r))$ is the space-dependent number density of isotope $i$.

\item[neutron flux:]
$$
\overline\phi = {1\over \overline V} \, \int_{V_{\rm merg}} dV \int_{E_{\rm merg}} dE \, \phi(\bff(r),E)
$$

\item[microscopic cross section of type $\bff(x)$ for isotope $\bff(i)$:]
\begin{eqnarray*}
\overline \sigma_{x,i} \negthinspace\negthinspace &=& \negthinspace\negthinspace { 1 \over \overline N_i} \, {\int_{V_i} dV \int_{E_{\rm merg}} dE \, N_i(\bff(r)) \, \sigma_{x,i}(\bff(r),E) \, \phi(\bff(r),E)
\over \int_{V_{\rm merg}} dV \int_{E_{\rm merg}} dE \, \phi(\bff(r),E)} \\
&=& \negthinspace\negthinspace { 1 \over \overline N_i\, \overline\phi \, \overline V} \, \int_{V_i} dV \int_{E_{\rm merg}} dE \, N_i(\bff(r)) \, \sigma_{x,i}(\bff(r),E) \, \phi(\bff(r),E) \ \ .
\end{eqnarray*}
\end{description}

\subsubsection{Homogenization and condensation with the flux and adjoint flux}\label{sect:prod}

If the \moc{PROD} keyword is set in data structure \ref{sect:descedi}, the adjoint flux is introduced as a weighting function in the
homogenization and condensation formulas. In this case, the module {\tt EDI:} produces the following homogenized/condensed information:

\begin{description}

\item[adjoint neutron flux:]
$$
\overline\phi^* = {1\over \overline\phi\, \overline V} \, \int_{V_{\rm merg}} dV \int_{E_{\rm merg}} dE \, \phi^*(\bff(r),E)\, \phi(\bff(r),E)
$$

\item[microscopic transfer cross section for isotope $\bff(i)$:]
$$
\overline \sigma_{{\rm s},i} ={ 1 \over \overline N_i\, (\overline\phi^*)' \, \overline\phi \, \overline V} \, \int_{V_i} dV \int_{E'_{\rm merg}} dE' \,\int_{E_{\rm merg}} dE \, N_i(\bff(r)) \, \sigma_{{\rm s},i}(\bff(r),E' \leftarrow E) \, \phi^*(\bff(r),E') \, \phi(\bff(r),E)
$$
\noindent with
$$
(\overline\phi^*)' = {1\over (\overline\phi)' \, \overline V} \, \int_{V_{\rm merg}} dV \int_{E'_{\rm merg}} dE' \, \phi^*(\bff(r),E')\, \phi(\bff(r),E')
$$

\item[microscopic cross section of type $\bff(x)\neq$ f for isotope $\bff(i)$:]
$$
\overline \sigma_{x,i} ={ 1 \over \overline N_i\, \overline\phi^* \, \overline\phi \, \overline V} \, \int_{V_i} dV \int_{E_{\rm merg}} dE \, N_i(\bff(r)) \, \sigma_{x,i}(\bff(r),E) \, \phi^*(\bff(r),E) \, \phi(\bff(r),E)
$$

\item[microscopic $\nu$ times fission cross section for isotope $\bff(i)$:]
$$
\overline\nu\overline\sigma_{{\rm f},i} ={ 1 \over \overline N_i\, \overline\phi \, \overline V} \, \int_{V_i} dV \int_{E_{\rm merg}} dE \, N_i(\bff(r)) \, \nu\sigma_{{\rm f},i}(\bff(r),E) \, \phi(\bff(r),E)
$$

\item[fission spectra for isotope $\bff(i)$:]
$$
\overline\chi_{i} ={ 1 \over \overline{\cal F}_i \overline\phi^* \, \overline V} \, \int_{V_i} dV \int_{E_{\rm merg}} dE \, \chi_{i}(\bff(r),E) \, {\cal F}_i(\bff(r)) \phi^*(\bff(r),E)
$$

\noindent where ${\cal F}_i(\bff(r))$ is the energy-integrated fission rate for isotope $\bff(i)$, defined as
$$
{\cal F}_i(\bff(r))=\int_\infty dE \, N_i(\bff(r)) \, \nu\sigma_{{\rm f},i}(\bff(r),E) \, \phi(\bff(r),E)
$$

\noindent and
$$
\overline{\cal F}_i={1\over \overline V} \int_{V_i} dV \, {\cal F}_i(\bff(r)) \ .
$$
\end{description}

Both the macrolib and microlib information is affected by the adjoint weighting. However, users should be advised that this operation may have some
undesirable effects on the fission spectrum normalization. Its use must therefore be limited to specialized applications where the adjoint weighting
is theoretically required. This is the case, for example, with the {\sc clio} perturbative analysis method.\cite{clio}

\eject
 % structure (dragonE)
\subsection{The {\tt EVO:} module}\label{sect:EVOData}

The \moc{EVO:} module performs the burnup calculations. The depletion equations
for the various isotope of the {\sc microlib} are solved using the burnup chains
also present in the {\sc microlib}. Both in-core and out-of-core calculations
can be considered. For in-core depletion calculations, one assumes linear flux variation
over each irradiation period (time stage). The initial (and possibly final) flux
distributions are recovered from previous \moc{FLU:} calculations. In-core depletion can
be performed at constant flux or constant power (expressed in MW/Tonne of initial heavy
elements) but these values can undergo step variations from one time stage to another.
All the information required for successive burnup calculation is stored on the PyLCM
\dds{burnup} data structure. Thus it is possible at any point in time to return to a previous
time step and restart the calculations.

\vskip 0.2cm

In each burnup mixture of the unit cell, the depletion of $K$ isotopes over a time
stage $(t_0,t_f)$ follows the following equation:

\begin{equation}
{dN_k \over dt} + N_k(t) \ \Lambda_k(t)=S_k(t) \ \ \ ; \ {k=1,K}
\label{eq:depletion}
\end{equation}
 
\noindent with
 
\begin{equation}
\Lambda_k(t)= \lambda_k + \langle \sigma_{{\rm a},k}(t) \phi(t) \rangle \ ,
\end{equation}

\vskip 0.2cm

\begin{equation}
S_k(t)=\sum^L_{l=1} {Y_{kl} \ \langle \sigma_{{\rm f},l}(t) \phi(t) \rangle } \ N_l(t) +
\sum^K_{l=1} m_{kl}(t) \ {N_l(t)} \ ,
\end{equation}

\vskip 0.2cm

\begin{equation}
\langle \sigma_{{\rm x},l}(t) \phi(t) \rangle = \int_0^\infty {\sigma_{{\rm x},l}(u) \phi(t,u) du}
\end{equation}

\noindent and
 
\begin{equation}
\sigma_{{\rm x},k}(t,u)\phi(t,u)= \sigma_{{\rm x},k}(t_0,u)\phi(t_0,u)+
{\sigma_{{\rm x},k}(t_f,u)\phi(t_f,u)-\sigma_{{\rm x},k}(t_0,u)\phi(t_0,u) \over t_f-t_0}(t-t_0)
\end{equation}

\noindent where
\begin{eqnarray}
\nonumber K &=& \hbox{number of depleting isotopes} 
\\
\nonumber L &=& \hbox{number of fissile isotopes producing fission products} 
\\
\nonumber N_k(t) &=& \hbox{time dependant number density for {\sl k}-th isotope} 
\\
\nonumber \lambda_k &=& \hbox{radioactive decay constant for {\sl k}-th isotope} 
\\
\nonumber \sigma_{{\rm x},k}(t,u) &=& \hbox{time and lethargy dependant microscopic cross section for
nuclear reaction x on} 
\\
\nonumber &~& \hbox{{\sl k}-th isotope. x=a, x=f and x=$\gamma$ respectively stands for absorption, fission and} 
\\
\nonumber &~& \hbox{radiative capture cross sections} 
\\
\nonumber \phi(t,u) &=& \hbox{time and lethargy dependant neutron flux} 
\\
\nonumber Y_{kl} &=& \hbox{fission yield for production of fission product {\sl k} by fissile
isotope {\sl l}} 
\\
\nonumber m_{kl}(t) &=& \hbox{radioactive decay constant or $\langle \sigma_{{\rm x},l}(t)
\phi(t) \rangle$ term for production of isotope {\sl k} by}
\\
\nonumber &~& \hbox{isotope {\sl l}.} 
\end{eqnarray}

Depleting isotopes with $\Lambda_k(t_0)\left[t_f-t_0\right]\geq$\dusa{valexp} and
$\Lambda_k(t_f)\left[t_f-t_0\right]\geq$\dusa{valexp} are considered to be at saturation. They are
described by making ${dN_k \over dt}=0$ in \Eq{depletion} to obtain

\begin{equation}
N_k(t)={S_k(t)\over\Lambda_k(t)} \ \ \ ; \ {{\rm if} \ k \ {\rm is \ at \ saturation.}}
\label{eq:sat1}
\end{equation}

If the keyword \moc{SAT} is set, beginning-of-stage and end-of-stage Dirac contributions are
added to the previous equation:

\begin{equation}
N_k(t)={1\over\Lambda_k(t)}\left[a \delta(t-t_0) +S_k(t)+b \delta(t-t_f)\right] \ \ \ ; \ {{\rm
if} \ k \ {\rm is \ at \ saturation}}
\label{eq:sat2}
\end{equation}
 
\noindent where $a$ and $b$ are chosen in order to satisfy the time integral of \Eq{depletion}:

\begin{equation}
N_k(t_f^+)-N_k(t_0^-) + \int_{t_0^-}^{t_f^+}{N_k(t) \ \Lambda_k(t) \ dt} =
\int_{t_0^-}^{t_f^+}{S_k(t) \ dt}
\end{equation}

It is numerically convenient to chose the following values of $a$ and $b$:

\begin{equation}
a=N_k(t_0^-)-{S_k(t_0^+) \over \Lambda_k(t_0^+)}
\end{equation}

\noindent and

\begin{equation}
b={S_k(t_0^+) \over \Lambda_k(t_0^+)}-{S_k(t_f^+) \over \Lambda_k(t_f^+)}
\end{equation}

\vskip 0.2cm

The numerical solution techniques used in the \moc{EVO:} module are the following.
Very short period isotopes are taken at saturation and are solved apart from non-saturating
isotopes. If an isotope is taken at saturation, all its parent isotopes, other than fissiles
isotopes, are also taken at saturation. Isotopes at saturation can procuce daughter isotopes
using decay {\sl and/or} neutron-induced reactions.

\vskip 0.2cm

The lumped depletion matrix system containing the non-saturating isotopes is solved
using either a fifth order Cash-Karp algorithm or a fourth order Kaps-Rentrop
algorithm\cite{recipie}, taking care to perform all matrix operations in sparse matrix algebra.
Matrices $\left[ m_{kl}(t_0) \right]$ and $\left[ m_{kl}(t_f) \right]$ are therefore
represented in diagonal banded storage and kept apart from the yield matrix
$\left[ Y_{kl}\right]$. Every matrix multiplication or linear system solution is obtained
via the LU algorithm.

\vskip 0.2cm

The solution of burnup equations is affected by the flux normalization factors. DRAGON can
perform out-of-core or in-core depletion with a choice between two normalization techniques:

\begin{enumerate}

\item Constant flux depletion. In this case, the lethargy integrated fluxes at
beginning-of-stage and end-of-stage are set to a constant $F$:

\begin{equation}
\int_0^\infty{\phi(t_0,u) du}=\int_0^\infty{\phi(t_f,u) du}=F
\end{equation}

\item Constant power depletion. In this case, the power released per initial heavy element at
beginning-of-stage and end-of-stage are set to a constant $W$.

\vskip -0.5cm

\begin{eqnarray}
\nonumber \sum^K_{k=1} \big[ \kappa_{{\rm f},k} \ \langle \sigma_{{\rm f},k}(t_0) \phi(t_0) \rangle +\kappa_{\gamma,k} \ \langle
\sigma_{\gamma,k}(t_0) \phi(t_0) \rangle \big]  \ N_k(t_0) &=& \\
\sum^K_{k=1} \big[ \kappa_{{\rm f},k} \ \langle \sigma_{{\rm f},k}(t_f) \phi(t_f) \rangle +\kappa_{\gamma,k} \ \langle \sigma_{\gamma,k}
(t_f) \phi(t_f) \rangle \big]\ N_k(t_f) &=& C_0 \ W
\end{eqnarray}

\noindent where
\begin{eqnarray}
\nonumber \kappa_{{\rm f},k} &=& \hbox{energy (MeV) released per fission of the fissile isotope $k$}
\\
\nonumber \kappa_{\gamma,k} &=& \hbox{energy (MeV) released per radiative capture of isotope $k$}
\\
\nonumber C_0 &=& \hbox{conversion factor (MeV/MJ) multiplied by the mass of initial heavy
elements}
\\
\nonumber &~& \hbox{expressed in metric tonnes} 
\end{eqnarray}

The end-of-stage power is function of the number densities $N_k(t_f)$; a few iterations will
therefore be required before the end-of-stage power released can be set equal to the desired
value. Note that there is no warranties that the power released keep its desired value at every time
during the stage; only the beginning-of-stage and end-of-stage are set.

\end{enumerate}

Whatever the normalisation technique used, DRAGON compute the exact burnup of the unit cell
(in MW per tonne of initial heavy element) by adding an additional equation in the depletion
system. This value is the local parameter that should be used to tabulate the output cross
sections.

\vskip 0.2cm

The general format of the data which is used to control
the execution of the \moc{EVO:} module is the following:

\begin{DataStructure}{Structure \dstr{EVO:}}
\dusa{BRNNAM} \dusa{MICNAM} \moc{:=} \moc{EVO:} \\
~~~~~$[$ \dusa{BRNNAM} $]~\{$ \dusa{MICNAM} $|$ \dusa{OLDMIC} $\}~[~\{$ \dusa{FLUNAM} \dusa{TRKNAM} $|$ \dusa{POWNAM} $\}~]$\\
~~~~~\moc{::} \dstr{descevo}
\end{DataStructure}

\noindent where

\begin{ListeDeDescription}{mmmmmmmm}

\item[\dusa{BRNNAM}] {\tt character*12} name of the \dds{burnup} data
structure that will contain the
depletion history as modified by the depletion module. If \dusa{BRNNAM} appears
on both LHS and RHS, it is updated; otherwise, it is created.

\item[\dusa{MICNAM}] {\tt character*12} name of the \dds{microlib} containing
the microscopic cross sections at save point {\sl xts}. \dusa{MICNAM} is modified
to include an embedded \dds{macrolib} containing the updated macroscopic cross
sections at set point {\sl xtr}. If \dusa{MICNAM} appears on both LHS and RHS,
it is updated; otherwise, the internal library \dusa{OLDMIC} is copied in
\dusa{MICNAM} and \dusa{MICNAM} is updated. It is possible to assign different
\dds{microlib} to different save points of the depletion calculation. In this
case, the microscopic reaction rates will be linearly interpolated/extrapolated
between points {\sl xti} and {\sl xtf}.

\item[\dusa{OLDMIC}] {\tt character*12} name of a read-only \dds{microlib}
that is copied in \dusa{MICNAM}.

\item[\dusa{FLUNAM}] {\tt character*12} name of a read-only \dds{fluxunk} at save point
{\sl xts}. This information is used for in-core depletion cases. This information is not required for
out-of-core depletion cases. Otherwise, it is mandatory

\item[\dusa{TRKNAM}] {\tt character*12} name of a read-only \dds{tracking}
constructed for the depleting geometry and consistent with object \dusa{FLUNAM}.

\item[\dusa{POWNAM}] {\tt character*12} name of a read-only \dds{power} object (generated by DONJON) at save point
{\sl xts}. This information is used for micro-depletion cases.

\item[\dstr{descevo}] structure containing the input data to this module
(see \Sect{descevo}).

\end{ListeDeDescription}

For the in-core depletion cases, the tracking \dds{tracking} data structure on which
\dusa{FLUNAM} is based, is automatically recovered in read-only mode from the
generalized driver dependencies.

\subsubsection{Data input for module {\tt EVO:}}\label{sect:descevo}

\begin{DataStructure}{Structure \dstr{descevo}}
$[$ \moc{EDIT} \dusa{iprint} $]$ \\
$[$ $\{$ \moc{SAVE} \dusa{xts} $\{$ \moc{S} $|$ \moc{DAY} $|$ \moc{YEAR} $\}~\{$
\moc{FLUX} \dusa{flux} $|$ \moc{POWR} \dusa{fpower} $|$ \moc{W/CC} \dusa{apower} $\}~|$
\moc{NOSA} $\}$ $]$ \\
$[$ \moc{EPS1} \dusa{valeps1} $]~~[$ \moc{EPS2} \dusa{valeps2} $]~~[~\{$ \moc{EXPM} \dusa{valexp} $|$ \moc{SATOFF} $\}~]$ \\
$[$ \moc{H1} \dusa{valh1} $]~[$ $\{$ \moc{RUNG} $|$ \moc{KAPS} $\}$ $]$ \\
$[~\{$ \moc{TIXS} $|$ \moc{TDXS} $\}~]~[~\{$\moc{NOEX} $|$ \moc{EXTR}$\}~]$~~$[~\{$\moc{NOGL} $|$ \moc{GLOB}$\}~]$~~$[~\{$\moc{NSAT} $|$ \moc{SAT}$\}~]$~~
$[~\{$\moc{NODI} $|$ \moc{DIRA}$\}~]$ \\
$[~\{$\moc{FLUX\_FLUX} $|$ \moc{FLUX\_MAC} $|$ \moc{FLUX\_POW} $\}~]~[~\{$ \moc{CHAIN} $|$ \moc{PIFI} $\}~]$ \\
$[$ \moc{DEPL} $\{$\dusa{xti} \dusa{xtf} $|$ \dusa{dxt} $\}~\{$ \moc{S} $|$ \moc{DAY} $|$ \moc{YEAR} $\}$ $\{$ \moc{COOL} $|$
\moc{FLUX} \dusa{flux} $|$ \moc{POWR} \dusa{fpower} $|$ \moc{W/CC} \dusa{apower} $|$ \moc{KEEP} $\}$ $]$ \\
$[$ \moc{SET} \dusa{xtr} $\{$ \moc{S} $|$ \moc{DAY} $|$ \moc{YEAR} $\}$ $]$ \\
$[$ \moc{MIXB} $[[$ \dusa{mixbrn} $]] ~]~~~[$ \moc{MIXP} $[[$ \dusa{mixpwr} $]] ~]$ \\
$[$ \moc{PICK}  {\tt >>} \dusa{burnup} {\tt <<} $]$ \\
{\tt ;}
\end{DataStructure}

\noindent
where

\begin{ListeDeDescription}{mmmmmmm}

\item[\moc{EDIT}] keyword used to modify the print level \dusa{iprint}.

\item[\dusa{iprint}] index used to control the printing of the module. The
amount of output produced by this tracking module will vary substantially
depending on the print level specified.

\item[\moc{SAVE}] keyword to specify that the current isotopic concentration
and the microscopic reaction rates resulting from the last transport calculation
will be normalized and stored on \dusa{BRNNAM} in a sub-directory corresponding
to a specific time. By default this data is stored at a time corresponding to
\dusa{xti}.

\item[\moc{NOSA}] keyword to specify that the current isotopic concentration
and the results of the last transport calculation will not be stored on
\dusa{BRNNAM}. By default this data is stored at a time corresponding to
\dusa{xti}.

\item[\moc{SET}] keyword used to recover the isotopic concentration already
stored on \dusa{BRNNAM} from a sub-directory corresponding to a specific time. By
default this data is recovered from a time corresponding to \dusa{xtf}.

\item[\moc{DEPL}] keyword to specify that a burnup calculation between an
initial and a final time must be performed. In the case where the \moc{SAVE}
keyword is absent, the initial isotopic concentration will be stored on
\dusa{BRNNAM} on a sub-directory corresponding to the initial time. If the
\moc{SET} keyword is absent, the isotopic concentration corresponding to the
final burnup time will be used to update \moc{MICNAM}.

\item[\dusa{xti}] initial time associated with the burnup calculation. The
name of the sub-directory where this information is stored will be given by
`{\tt DEPL-DAT}'//{\tt CNN} where {\tt CNN} is a  {\tt character*4} variable
defined by  {\tt WRITE(CNN,'(I4.4)') INN} where {\tt INN} is an index associated
with the time \dusa{xti}. The initial values  are recovered from this
sub-directory in \dusa{BRNNAM}.

\item[\dusa{xtf}] end of time for the burnup calculation. The results of the
isotopic depletion calculations are stored in the tables associated with a
sub-directory whose name is constructed in the same manner as the \dusa{xti}
input.

\item[\dusa{dxt}] time interval for the burnup calculation. The initial time \dusa{xti} in
this case is taken as the final time reached at the last depletion step. If this is the first
depletion step, \dusa{xti} $=0$.

\item[\dusa{xts}] time associated with the last transport calculation. The
name  of the sub-directory where this information is to be stored is constructed
in the same manner as the for \dusa{xti} input. By default  (fixed default)
\dusa{xts}=\dusa{xti}.

\item[\dusa{xtr}] time associated with the next flux calculation. The name of
the sub-directory where this information is to be stored is constructed in the
same manner as for the \dusa{xti} input. By default (fixed default)
\dusa{xtr}=\dusa{xtf}.

\item[\moc{S}] keyword to specify that the time is given in seconds.

\item[\moc{DAY}] keyword to specify that the time is given in days.

\item[\moc{YEAR}] keyword to specify that the time is given in years.

\item[\moc{COOL}] keyword to specify that a zero flux burnup calculation is to
be performed. 

\item[\moc{FLUX}] keyword to specify that a constant flux burnup
calculation is to be performed.  

\item[\dusa{flux}] flux expressed in $cm^{-2}s^{-1}$. 

\item[\moc{POWR}] keyword to specify that a constant fuel power depletion
calculation is to be performed. The energy released outside the fuel (e.g., by
(n,$\gamma$) reactions) is {\sl not} taken into account in the flux normalization,
unless the \moc{GLOB} option is set.

\item[\dusa{fpower}] fuel power expressed in $KW\;Kg^{-1}=MW\;{\it tonne}^{-1}$.

\item[\moc{W/CC}] keyword to specify that a constant assembly power depletion
calculation is to be performed. The energy released outside the fuel (e.g., by
(n,$\gamma$) reactions) is always taken into account in the flux normalization.

\item[\dusa{apower}] assembly power density expressed in $W/cm^3$ (Power per
unit assembly volume).

\item[\moc{KEEP}] keyword to specify that the flux is used without been normalized.
This option is useful in cases where the flux was already normalized before the call to
\moc{EVO:} module.

\item[\moc{EPS1}] keyword to specify the tolerance used in the algorithm for
the solution of the depletion equations.

\item[\dusa{valeps1}] the tolerance used in the algorithm for the solution of the
depletion equations. The default value is \dusa{valeps1}=$1.0\times 10^{-5}$.

\item[\moc{EPS2}] keyword to specify the tolerance used in the search
algorithm for a final fixed power (used if the \moc{POWR} or \moc{W/CC} option is activated).

\item[\dusa{valeps2}] the tolerance used in the search algorithm for a final
fixed power. The default value is \dusa{valeps2}=$1.0\times 10^{-4}$.

\item[\moc{EXPM}] keyword to specify the selection criterion for non-fissile
isotopes that are at saturation.

\item[\dusa{valexp}] the isotopes for which $\lambda \times($\dusa{xtf}$-$
\dusa{xti})$ \ge $\dusa{valexp} will be treated by a saturation approximation. Here,
$\lambda$ is the sum of the radioactive decay constant and microscopic neutron
absorption rate. The default value is \dusa{valexp}=80.0. In order to remove the
saturation approximation for all isotopes set \dusa{valexp} to a very large number
such as $1.0\times 10^{5}$. On the other way, the saturation approximation can be set
for a specific isotope by using the keyword \moc{SAT} in Sect.~\ref{sect:descmix}
(module \moc{LIB:}).

\item[\moc{SATOFF}] keyword to remove the saturation approximation for all isotopes
even if \moc{SAT} keyword was set in Sect.~\ref{sect:descmix} (module \moc{LIB:}).

\item[\moc{H1}] keyword to specify an estimate of the relative width of the
time step used in the solution of burnup equations.

\item[\dusa{valh1}] relative width of the time step used in the solution of
burnup equations. An initial time step of 
$\Delta_{t}=$\dusa{valh1}$\times ($\dusa{xtf}$-$\dusa{xti}$)$ 
is used. This value is optimized dynamically by the program. The
default value is \dusa{valh1}=$1.0\times 10^{-4}$.

\item[\moc{RUNG}] keyword to specify that the solution will be obtained using
the $5^{th}$ order Cash-Karp algorithm.

\item[\moc{KAPS}] keyword to specify that the solution will be obtained using
the $4^{th}$ order Kaps-Rentrop algorithm. This is the default value.

\item[\moc{TIXS}] keyword that specified that time independent cross sections will be used.
This is the default option when no time dependent cross sections are provided.

\item[\moc{TDXS}] keyword that specified that time dependent cross sections will be used if available.
This is the default option when time dependent cross sections are provided.

\item[\moc{NOEX}] keyword to supress the linear extrapolation of the
microscopic reaction rates in
the solution of the burnup equations.

\item[\moc{EXTR}] keyword to perform a linear extrapolation of the microscopic reaction rates, using
the available information preceding the initial time \dusa{xti}. This is the
default option.

\item[\moc{NOGL}] keyword to compute the burnup using the energy released in
fuel only. This is the default option.

\item[\moc{GLOB}]  keyword to compute the burnup using the energy released in
the complete geometry. This option has an effect only in cases where some
energy is released outside the fuel (e.g., due to (n,$\gamma$) reactions).
This option affects both the meaning of \dusa{fpower} (given after the
key-word \moc{POWR}) and the value of the burnup, as computed by {\tt EVO:}.

\item[\moc{NSAT}] save the non--saturated initial number densities in the {\sc burnup}
object \dusa{BRNNAM} (default value)

\item[\moc{SAT}]  save the saturated initial number densities in the {\sc burnup}
object \dusa{BRNNAM}

\item[\moc{NODI}]  select \Eq{sat1} to compute the saturated number densities
(default value)

\item[\moc{DIRA}]  select \Eq{sat2} to compute the saturated number densities

\item[\moc{FLUX\_FLUX}]  recover the neutron flux from \dusa{FLUNAM} object (default option)

\item[\moc{FLUX\_MAC}]  recover the neutron flux from embedded macrolib present in \dusa{MICNAM} or \dusa{OLDMIC}
object. This option is useful to deplete in cases where the neutron flux is obtained from a Monte Carlo
calculation.

\item[\moc{FLUX\_POW}]  recover the neutron flux from the \dds{power} object named \dusa{POWNAM} generated in DONJON. This option is useful in
micro-depletion cases. The neutron flux recovered from \dusa{POWNAM} is generally normalized to the power of the full core. It is therefore
recommended to use the \moc{KEEP} option in \moc{DEPL} data structure.

\item[\moc{CHAIN}]  recover the fission yield data from {\tt 'DEPL-CHAIN'} directory of \dusa{MICNAM} or \dusa{OLDMIC}
object (default option). With this option, the fission yield data is the same in all material mixtures.

\item[\moc{PIFI}]  recover the fission yield data from {\tt 'PIFI'} and {\tt 'PYIELD'} records present in isotopic directories
of \dusa{MICNAM} or \dusa{OLDMIC} object. With this option, the fission yield data is mixture-dependent. This option is useful
in micro-depletion cases.

\item[\moc{MIXB}]  keyword to select depleting material mixtures. By default, all mixtures
with depleting isotopes are set as depleting.

\item[\dusa{mixbrn}] indices of depleting material mixtures.

\item[\moc{MIXP}]  keyword to select material mixtures producing power. By default, 
\begin{itemize}
\item if \moc{MIXB} is not set, all mixtures with isotopes producing power are set as producing power
\item if \moc{MIXB} is set, the same mixtures \dusa{mixbrn} are set as producing power.
\end{itemize}

\item[\dusa{mixpwr}] indices of material mixtures producing power.

\item[\moc{PICK}]  keyword used to recover the final burnup value (in MW-day/tonne) in a CLE-2000 variable.

\item[\dusa{burnup}] \texttt{character*12} CLE-2000 variable name in which the extracted burnup value will be placed.

\end{ListeDeDescription}

\subsubsection{Power normalization in {\tt EVO:}}\label{sect:powerevo}

Flux-induced depletion is dependent of the flux or power normalization factor
given after key-words \moc{FLUX}, \moc{POWR} or \moc{W/CC}. The depletion
steps, given after key-words \moc{SAVE}, \moc{DEPL} or \moc{SET}, are set
in time units. Traditionally, the power normalization factor is given in
${\it MW}\;{\it tonne}^{-1}$ and the depletion steps are given in
${\it MWday}\;{\it tonne}^{-1}$. However, a confusion appear in cases where
some energy is released outside the fuel (e.g., due to (n,$\gamma$) reactions).

\vskip 0.2cm

The accepted rule and default option in {\tt EVO:} is to compute the burnup
steps in units of $MWday\;{\it tonne}^{-1}$ by considering only the energy
released in fuel (and only the initial mass of the heavy elements present
in fuel). However, it is also recommended to provide a normalization power
taking into account the {\sl total} energy released in the global geometry.
The \moc{GLOB} option can be use to change this rule and to use
the energy released in the complete geometry to compute the burnup. However,
this is not a
common practice, as it implies a non-usual definition of the burnup.
A more acceptable solution consists in setting the normalization power
in power per unit volume of the complete geometry using the key-word
\moc{W/CC}. The value of \dusa{apower} can be computed from the linear
power $f_{\rm lin}$ (expressed in ${\it Mev}\;{\it s}^{-1}\;{\it cm}^{-1}$)
using:

\begin{equation}
{\it apower}={f_{\rm lin} \ 1.60207 \times 10^{-13} \over V_{\rm assmb}}
\label{eq:eq1}
\end{equation}

\noindent where $V_{\rm assmb}$ is the 2--D lumped volume of the assembly expressed in $cm^2$.

\vskip 0.2cm

The corresponding normalization factor $f_{\rm burnup}$ in
${\it MW}\;{\it tonne}^{-1}$ is given as

\begin{equation}
f_{\rm burnup}={ {\it apower} \over D_{\rm g} \ F_{\rm power}}
\label{eq:eq2}
\end{equation}

\noindent where $D_{\rm g}$ is the mass of heavy elements per unit volume
of the complete geometry ($g\; {\it cm}^{-3}$) and $F_{\rm power}$ is the
ratio of the energy released in the complete geometry over the energy
released in fuel. Numerical values of $D_{\rm g}$ and $f_{\rm power}$ are
computed by {\tt EVO:} when the parameter \dusa{iprint} is greater or
equal to 2. The burnup $B$ corresponding to an elapsed time $\Delta t$ is
therefore given as

\begin{equation}
B=f_{\rm burnup} \ \Delta t
\label{eq:eq3}
\end{equation}

\noindent where $B$ is expressed in ${\it MWday}\;{\it tonne}^{-1}$ and $\Delta t$
is expressed in ${\it day}$.

\vskip 0.2cm

The unit of the reaction rates depends on the normalization applied to the flux. This normalization
takes place after the flux calculation, using the \moc{EVO:} module. Here is an example:

\begin{verbatim}
INTEGER istep := 1 ;
REAL Tend  := 0.0  ;
REAL Fuelpwr := 38.4 ; ! expressed in MW/tonne

BURN MICROLIB := EVO: MICROLIB FLUX TRACKN ::
  EDIT 0
  SAVE <<Tend>> DAY POWR <<Fuelpwr>>
;
\end{verbatim}

\noindent where \moc{BURN} is the burnup object, \moc{MICROLIB} is the Microlib used to compute the flux, \moc{FLUX} is the flux
object and \moc{TRACKN} is the tracking object used to compute the flux. After this call, the record
{\tt 'FLUX-NORM'} in \moc{BURN} contains a unique real number, equal to the flux normalization factor. If \moc{MICROLIB} is
obtained using the \moc{LIB:} module, the \moc{DEPL} keyword with following data must be set (see \Sect{desclib}).
Unfortunately, the normalization factor is kept aside and is not applied to the flux present in object \moc{FLUX}. In
fact, only the advanced post-processing modules \moc{COMPO:} (see \Sect{COMPOData}) and \moc{SAP:} (see \Sect{SAPHYBData})
are making use of this normalization factor.

\eject
 % structure (dragonD)
\subsection{The {\tt SPH:} module}\label{sect:SPHData}

The {\sl superhomog\'en\'eisation} (SPH) equivalence technique is based on the calculation of a set of {\sl equivalence factors}
$\{\mu_{m,k}, m \in C_m \ {\rm and} \ k \in M_k\}$, where $C_m$ and $M_k$ is a macro region and a coarse energy group of a full-core or macro calculation (see Sect. 4.4 of Ref.~\citen{PIP2009}). These equivalence factors are computed in such a way that a macro calculation made over $C_m$ and $M_k$ with a simplified transport operator leads to the same leakage and reaction rates as a reference calculation performed without homogenization and with a fine group discretization.

\vskip 0.08cm

The SPH correction is applied differently, depending on the type of macro-calculation:
\begin{itemize}

\item In the case where the macro-calculation is done with the diffusion theory, neutron balance is satisfied if the SPH correction is applied as
follows:
\begin{equation}
\nabla\cdot\bff(J)_g(\bff(r))+\mu_g\Sigma_g(\bff(r)) {\phi_g(\bff(r))\over \mu_g}={\chi_g\over k_{\rm eff}}\sum_{h=1}^G \mu_h \nu\Sigma_{{\rm f},h}(\bff(r)) {\phi_h(\bff(r))\over \mu_h}
+\sum_{h=1}^G \mu_h \Sigma_{{\rm s0},g\leftarrow h}(\bff(r)){\phi_h(\bff(r))\over \mu_h}
\label{eq:sph1}
\end{equation}


\noindent and
\begin{equation}
\bff(J)_g(\bff(r))=-\mu_g D_g(\bff(r)){\nabla\phi_g(\bff(r))\over \mu_g} .
\label{eq:sph2}
\end{equation}

In conclusion:
\begin{itemize}
\item Diffusion coefficients and all $P_0$ cross sections (including the total cross section {\tt NTOT0}) must be multiplied by $\mu_g$.
\item Scattering matrix terms $\Sigma_{{\rm s0},g\leftarrow h}(\bff(r))$ must be multiplied by $\mu_h$.
\item Fluxes (such as {\tt NWT0} and {\tt FLUX-INTG}) must be divided by $\mu_g$.
\end{itemize}

\item In the case where the macro-calculation is done with the simplified $P_n$ method, the neutron balance is satisfied if the SPH correction is applied on even parity equations as
follows:\cite{sphedf2}
\begin{equation}
\mu_g\Sigma_{0,g}(\bff(r)) {\phi_{0,g}(\bff(r))\over \mu_g}+\nabla\cdot\bff(\phi)_{1,g}(\bff(r))={\chi_g\over k_{\rm eff}}\sum_{h=1}^G \mu_h \nu\Sigma_{{\rm f},h}(\bff(r)) {\phi_{0,h}(\bff(r))\over \mu_h}
+\sum_{h=1}^G \mu_h \Sigma_{{\rm s0},g\leftarrow h}(\bff(r)){\phi_{0,h}(\bff(r))\over \mu_h}
\label{eq:sph3}
\end{equation}
\begin{equation}
{2\ell\over 4\ell+1}\nabla\cdot\bff(\phi)_{2\ell-1,g}(\bff(r))+\mu_g\Sigma_{0,g}(\bff(r)) {\phi_{2\ell,g}(\bff(r))\over \mu_g}+{2\ell+1\over 4\ell+1}\nabla\cdot\bff(\phi)_{2\ell+1,g}(\bff(r))=\sum_{h=1}^G \mu_h \Sigma_{{\rm s2\ell},g\leftarrow h}(\bff(r)){\phi_{2\ell,h}(\bff(r))\over \mu_h}
\label{eq:sph4}
\end{equation}

\noindent and on odd-parity equations as follows:
\begin{equation}
{2\ell+1\over 4\ell+3}\nabla{\phi_{2\ell,g}(\bff(r))\over \mu_g}+{\Sigma_{1,g}(\bff(r))\over \mu_g}\bff(\phi)_{2\ell+1,g}(\bff(r))+{2\ell+2\over 4\ell+3}\nabla{\phi_{2\ell+2,g}(\bff(r))\over \mu_g}=\sum_{h=1}^G {\Sigma_{{\rm s2\ell+1},g\leftarrow h}(\bff(r))\over \mu_g}\phi_{2\ell+1,h}(\bff(r))
\label{eq:sph5}
\end{equation}
\noindent where $\ell\ge 1.$
\vskip 0.08cm

In conclusion:
\begin{itemize}
\item All $P_0$ cross sections (including the total cross section {\tt NTOT0} in the even-parity equations) must be multiplied by $\mu_g$.
\item The total cross section {\tt NTOT1} in the odd-parity equations must be divided by $\mu_g$.
\item Scattering matrix terms $\Sigma_{{\rm s\ell},g\leftarrow h}(\bff(r))$ with $\ell$ even must be multiplied by $\mu_h$.
\item Scattering matrix terms $\Sigma_{{\rm s\ell},g\leftarrow h}(\bff(r))$ with $\ell$ odd must be divided by $\mu_g$.
\item Even parity fluxes (such as {\tt NWT0} and {\tt FLUX-INTG}) must be divided by $\mu_g$.
\item Odd parity fluxes (such as {\tt NWT1} and {\tt FLUX-INTG-P1}) are not modified.
\end{itemize}

\item In the case where the macro-calculation is done in transport theory, but not with a $P_n$--type method, the macroscopic
total cross section {\sl is not modified}, and the even-odd corrections consistent with the simplified $P_n$ method are reported to the
macroscopic within-group scattering cross sections. They are now corrected as\cite{cns2015}
\begin{equation}
\tilde\Sigma_{{\rm s2\ell},g\leftarrow g}(\bff(r))=\mu_g\Sigma_{{\rm s2\ell},g\leftarrow g}(\bff(r))+(1-\mu_g)\, \Sigma_{0,g}(\bff(r))
\label{eq:sph6}
\end{equation}

\noindent and
\begin{equation}
\tilde\Sigma_{{\rm s2\ell+1},g\leftarrow g}(\bff(r))={\Sigma_{{\rm s2\ell+1},g\leftarrow g}(\bff(r))\over\mu_g}+\left(1-{1\over \mu_g}\right)\Sigma_{1,g}(\bff(r))
\label{eq:sph7}
\end{equation}
\noindent where $\ell\ge 0.$

\vskip 0.08cm

Other cross sections and scattering matrix terms are corrected the same way as for the simplified $P_n$ method.

\end{itemize}

\subsubsection{Data input for module {\tt SPH:}}

The \moc{SPH:} module perform a SPH equivalence calculation using
information recovered in a macrolib and apply SPH factors to the corresponding \dds{edition} ({\tt L\_EDIT}),
\dds{microlib} ({\tt L\_LIBRARY}), \dds{macrolib} ({\tt L\_MACROLIB}) or \dds{saphyb} ({\tt L\_SAPHYB}) object. This module is also useful
to extract a corrected or non-corrected \dds{microlib} or \dds{macrolib} from the first RHS object. The calling specification is:

\begin{DataStructure}{Structure \dstr{SPH:}}
$\{$ \dusa{EDINEW} $|$ \dusa{LIBNEW} $|$ \dusa{MACNEW} $|$ \dusa{SAPNEW} $|$ \dusa{CPONEW} $|$ \dusa{EDINAM} $|$ \dusa{LIBNAM} $|$ \dusa{MACNAM} \\
~~~~~$|$ \dusa{SAPNAM} $|$ \dusa{CPONAM} $|$ \dusa{APXNAM} $\}$ \\
~~~~~\moc{:=} \moc{SPH:} $\{$ \dusa{EDINAM} $|$ \dusa{LIBNAM}
$|$ \dusa{MACNAM} $|$ \dusa{SAPNAM} $|$ \dusa{CPONAM} $|$ \dusa{APXNAM} $\}$ \\
~~~~~$\{~[$ \dusa{TRKNAM} $[$ \dusa{TRKFIL} $]~]~|$ \dusa{OPTIM} $\}~[$ \dusa{FLUNAM} $]$ \\
~~~~~\moc{::} \dstr{descsph}
\end{DataStructure}

\noindent
where
\begin{ListeDeDescription}{mmmmmmmm}

\item[\dusa{EDINEW}] {\tt character*12} name of the new \dds{edition} data
structure containing SPH-corrected information (see \Sect{EDIData}). In this
case, an existing \dds{edition} data structure must appear on the RHS.

\item[\dusa{LIBNEW}] {\tt character*12} name of the new \dds{microlib} data
structure containing SPH-corrected information (see \Sect{LIBData}). In this
case, an existing \dds{edition}, \dds{microlib} or \dds{multicompo} data structure
must appear on the RHS.

\item[\dusa{MACNEW}] {\tt character*12} name of the new \dds{macrolib} data
structure containing SPH-corrected information (see \Sect{MACData}).

\item[\dusa{SAPNEW}] {\tt character*12} name of the new \dds{saphyb} data
structure containing SPH information (see \Sect{SAPHYBData}). In this
case, data structure \dusa{SAPNAM} must appear on the RHS.

\item[\dusa{CPONEW}] {\tt character*12} name of the new \dds{multicompo} data
structure containing SPH-corrected information (see \Sect{COMPOData}). In this
case, data structure \dusa{CPONAM} must appear on the RHS.

\item[\dusa{EDINAM}] {\tt character*12} name of the existing \dds{edition} data
structure where the edition information is recovered (see \Sect{EDIData}).

\item[\dusa{LIBNAM}] {\tt character*12} name of the existing \dds{microlib} data
structure where the edition information is recovered (see \Sect{LIBData}).

\item[\dusa{MACNAM}] {\tt character*12} name of the existing \dds{macrolib} data
structure where the edition information is recovered (see \Sect{MACData}).

\item[\dusa{SAPNAM}] {\tt character*12} name of the existing \dds{saphyb} data
structure where the edition information is recovered (see \Sect{SAPHYBData}).

\item[\dusa{CPONAM}] {\tt character*12} name of the existing \dds{multicompo} data
structure where the edition information is recovered (see \Sect{COMPOData}).

\item[\dusa{APXNAM}] {\tt character*12} name of the existing \dds{apex} or \dds{mpo} file where the
edition information is recovered.

\item[\dusa{TRKNAM}] {\tt character*12} name of the existing \dds{tracking} data
structure containing the tracking of the macro-geometry (see \Sect{TRKData}). This object
is compulsory only if a macro-calculation is to be performed by module {\tt SPH:}.

\item[\dusa{TRKFIL}] {\tt character*12} name of the existing sequential binary tracking
file used to store the tracks lengths of the macro-geometry. This file is given if and only if it was
required in the previous tracking module call (see \Sect{TRKData}).

\item[\dusa{OPTIM}] {\tt character*12} name of a {\tt L\_OPTIMIZE} object. The
SPH factors are set equal to the control-variable data recovered from \dusa{OPTIM} if keyword \moc{SPOP} is set.

\item[\dusa{FLUNAM}] {\tt character*12} name of an initialization flux used to start SPH iterations (see \Sect{FLUData}). By
default, a flat estimate of the flux is used.

\item[\dstr{descsph}] structure containing the input data to this module
(see \Sect{descsph}).

\end{ListeDeDescription}

Note: Saphyb files generated by APOLLO2 don't have a signature. If such a Saphyb is given as input
to module {\tt SPH:}, a signature must be included before using it. The following instruction
can do the job:
\begin{verbatim}
Saphyb := UTL: Saphyb :: CREA SIGNATURE 3 = 'L_SA' 'PHYB' ' ' ;
\end{verbatim}

\subsubsection{Specification for the type of equivalence calculation}\label{sect:descsph}

This structure is used to specify the type of equivalence calculation where the
flux and the condensed and/or homogenized cross sections are corrected by SPH
factors, in such a way as to respect a specified transport-transport or
transport-diffusion equivalence criteria.\cite{ALSB1,ALSB2,ALSB3} This
structure is defined as:

\begin{DataStructure}{Structure \dstr{descsph}}
$[$ \moc{EDIT} \dusa{iprint} $]$ \\
$[[$ \moc{STEP} $\{$ \moc{UP} \dusa{NOMDIR} $|$ \moc{AT} \dusa{index} $\}~]]$ \\
$[~\{$ \moc{IDEM} $|$ \moc{MACRO} $|$ \moc{MICRO} $\}~]$ \\
$[~\{$ \moc{OFF} $|$ \moc{SPRD} $[$ \dusa{nmerge} \dusa{ngcond} (\dusa{sph}($i$), $i$=1, \dusa{nmerge}$\times$\dusa{ngcond}) $]~|$ \moc{SPOP} $|$ \moc{HOMO} $|$ \moc{ALBS} $\}~]$  \\
$[~\{$ \moc{PN} $|$ \moc{SN} $[$ \moc{BELL} $]~\}~]$ \\
$[~\{$ \moc{STD} $|$ \moc{SELE\_ALB} $|$ \moc{SELE\_FD} $|$ \moc{SELE\_MWG} $|$ \moc{SELE\_EDF} $|$ \moc{ASYM} \dusa{mixs} $|$ \moc{STD\_FISS} $\}~]~~[$ \moc{ARM} $]$ \\
$[$ \moc{ITER}  $[$ \dusa{maxout} $]$  $[$ \dusa{epsout} $]~]$ \\
$[$ \moc{MAXNB} \dusa{maxnb} $]$ \\
$[$ \moc{GRMIN}  \dusa{lgrmin} $]~[$ \moc{GRMAX}  \dusa{lgrmax} $]$ \\
$[~\{$ \moc{EQUI} \dusa{TEXT4} $[$ \moc{LOCNAM} \dusa{TEXT80} $]~|~$ \moc{EQUI} \dusa{TEXT80} $\}~]$ \\
$[$~\moc{UPS}~$]~[$~\moc{LEAK}~$[$~\dusa{b2}~$]~]$ \\
\end{DataStructure}

\noindent where

\begin{ListeDeDescription}{mmmmmmmm}

\item[\moc{EDIT}] keyword used to modify the print level \dusa{iprint}.

\item[\dusa{iprint}] index used to control the printing of this module. The
\dusa{iprint} parameter is important for adjusting the amount of data that is
printed by this calculation step.

\item[\moc{STEP}] keyword used to set a specific elementary calculation from the first RHS.

\item[\moc{UP}] keyword used to select an elementary calculation located in a subdirectory of \dusa{EDINAM} or \dusa{CPONAM}. By default,
\begin{itemize}
\item the sub-directory name stored in record {\tt 'LAST-EDIT'} is selected if \dusa{EDINAM} is defined at RHS.
\item the sub-directory {\tt 'default'} is selected if \dusa{CPONAM} is defined at RHS.
\end{itemize}

\item[\dusa{NOMDIR}] name of an existing sub-directory of \dusa{EDINAM} or \dusa{CPONAM}. Can also be used to step up in the {\tt output\_n}
group of a \dds{mpo} file.

\item[\moc{AT}] keyword used to select the \dusa{index}--th elementary calculation in \dusa{SAPNAM}, \dusa{CPONAM} or \dusa{APXNAM}.

\item[\dusa{index}] index of the elementary calculation. Can also be used to step at the {\tt statept\_}$($\dusa{index} $-1)$
group of a \dds{mpo} file.

\item[\moc{IDEM}] keyword to force the production of a LHS object of the same type as the RHS (default option).

\item[\moc{MACRO}] keyword to force the production of a macrolib at LHS.

\item[\moc{MICRO}] keyword to force the production of a microlib at LHS.

\item[\moc{OFF}] keyword to specify the SPH factors are all set to 1.0,
meaning no correction. This keyword is useful to get rid of a SPH correction which have been set previously. By
default, the \moc{PN} or \moc{SN} option is activated.

\item[\moc{SPRD}] keyword to specify that the SPH factors are read from input  (if \dusa{nmerge}, \dusa{ngcond} and \dusa{sph}
are set) or recovered from a RHS object (otherwise).

\item[\moc{SPOP}] keyword to specify that the SPH factors are recovered from a RHS object of type \dds{optimize} ({\tt L\_OPTIMIZE}).

\item[\dusa{nmerge}] number of regions.

\item[\dusa{ngcond}] number of energy groups.

\item[\dusa{sph}($i$)] initial value of each SPH factor in each mixture (inner loop) and each group (outer loop).

\item[\moc{HOMO}] keyword to specify that the SPH factors are uniform over the complete
macro-geometry. This option is generally used with a complete homogenization of the
reference geometry, obtained using option \moc{MERG} \moc{COMP}. In this case the
neutron flux  (transport or diffusion) will be
uniform, which allows the SPH factors to be obtained (one per macro-group) using
a non-iterative strategy. For a given macro-group the SPH factor will be equal
to the ratio between the average flux of the region and the surface flux if the
\moc{SELE} option is used otherwise the SPH factor are all set equal to 1.0 (no
correction). The \moc{SELE} option allows an SPH factor equal to the inverse of
the discontinuity factor to be calculated.

\item[\moc{ALBS}] keyword to specify that the albedo of the geometry are to be
taken into account in the complete homogenization process. Thus the \moc{MERG}
and \moc{COMP} options must be specified. The SPH factors are obtained using a
transport-transport equivalence based on a calculation using the collision
probabilities. This option requires a geometry with \moc{VOID} (see
\Sect{descBC}) external boundary conditions to be closed using \moc{ALBS} in
modules \moc{ASM:} (see \Sect{descasm}).\cite{ALSB2} 

\item[\moc{PN}] keyword to activate a calculation of heterogeneous SPH factors based on a converging series of
macro-calculations with the correction strategy of Eqs.~(\ref{eq:sph1}) to~(\ref{eq:sph5}). This is the default option
if the macro-calculation is of diffusion, PN or SPN type. A normalization condition must be set if the macro-geometry
has no boundary leakage ({\sl fundamental mode} condition). If boundary leakage is present, no normalization condition
is used but the SPH iterations are difficult to converge in this case.

\item[\moc{SN}] keyword to activate a calculation of heterogeneous SPH factors based on a converging series of
macro-calculations with the correction strategy of Eqs.~(\ref{eq:sph6}) and~(\ref{eq:sph7}). This is the default option
if the macro-calculation is of PIJ, IC, SN or MOC type. A normalization condition must be set if the macro-geometry
has no boundary leakage ({\sl fundamental mode} condition). If boundary leakage is present, no normalization condition
is used but the SPH iterations are difficult to converge in this case.

\item[\moc{BELL}] keyword to activate the Bell procedure to accelerate the convergence of the SPH factors. This feature is currently
available with macro-calculations of PIJ type.\cite{madrid2}

\item[\moc{STD}] keyword to specify the use of flux-volume normalization for the SPH factors (default option). In each macro-group, the macro-fluxes
in macro regions $i$ are normalized using
$$
\tilde\phi_i=\phi_i\,{\bar\phi_{\rm ref}\over \bar\phi_{\rm mc}}
$$
\noindent where $\bar\phi_{\rm ref}$ is the averaged volumic flux of the reference calculation and $\bar\phi_{\rm mc}$ is the averaged volumic flux of the macro-calculation. Using this definition, the averaged SPH factor is equal to one.

\item[\moc{SELE\_ALB}] keyword to specify the use of Selengut normalization for the SPH factors. It is necessary to know the averaged surfacic flux of the reference calculation. Two possibilities exist:
\begin{itemize}
\item We use collision probabilities. We define the reference geometry with \moc{VOID}
external boundary conditions (see \Sect{descBC}) and to close the region for the collision probability calculations using the \moc{ALBS} option (see \Sect{descasm}).
\item We perform a flux calculation with the current iteration method in Eurydice. This option is only available if a \moc{SYBILT:} tracking is used and if
keyword \moc{ARM} is set in module \moc{ASM:} (see \Sect{descasm}).
\end{itemize}

\item[\moc{SELE\_FD}] keyword to specify the use of Selengut normalization for the SPH factors. It is necessary to know the averaged surfacic flux of the reference calculation. This value can be obtained by defining
a small region near boundary in the reference geometry and by using the \moc{ADF FD\_B} data structure in \Sect{descedi}.

\noindent In each macro-group, the macro-fluxes in macro regions $i$ are normalized using
$$
\tilde\phi_i=\phi_i\,{\phi_{\rm ref}^{\rm gap}\over \bar\phi_{\rm mc}}
$$
\noindent where $\phi_{\rm ref}^{\rm gap}$ is the averaged surfacic flux of the reference calculation. Using this definition, the averaged SPH factor is equal to
$$
\bar\mu={\bar\phi_{\rm ref}\over \phi_{\rm ref}^{\rm gap} } \ \ .
$$

\item[\moc{SELE\_MWG}] keyword to specify the use of Selengut {\sl macro calculation water gap} normalization for the SPH factors.\cite{Chambon2014} It is necessary to know the averaged surfacic flux of the reference and that of the {\sl macro} calculations. This reference value can be obtained by defining
a small region near boundary in the reference geometry and by using the \moc{ADF FD\_B} data structure in \Sect{descedi}.

\noindent In each macro-group, the macro-fluxes in macro regions $i$ are normalized using
$$
\tilde\phi_i=\phi_i\,{\phi_{\rm ref}^{\rm gap}\over \phi^{\rm surf}_{\rm mc}}
$$
\noindent where $\phi_{\rm ref}^{\rm gap}$ is the averaged surfacic flux of the reference calculation and $\phi^{\rm surf}_{\rm mc}$ is the averaged surfacic flux of the macro calculation. Using this definition, the averaged SPH factor is equal to
$$
\bar\mu={\bar\phi_{\rm ref} \,\phi^{\rm surf}_{\rm mc}\over \bar\phi_{\rm mc} \,\phi_{\rm ref}^{\rm gap} } \ \ .
$$

\item[\moc{SELE\_EDF}] keyword to specify the use of generalized Selengut normalization for the SPH factors.\cite{sphedf} It is necessary to know the averaged surfacic flux and the
averaged volumic flux in a row of cells of the reference calculation. The surfacic flux is obtained as with the \moc{SELE} option. The  value of the volumic flux in a row of
cells is computed using index information from the \moc{ADF FD\_H} data structure in \Sect{descedi}.

\noindent In each macro-group, the macro-fluxes in macro regions $i$ are normalized using
$$
\tilde\phi_i=\phi_i\,{\bar\phi_{\rm ref} \, \phi_{\rm ref}^{\rm gap}\over \bar\phi_{\rm mc} \, \phi_{\rm ref}^{\rm row} }
$$
\noindent where $\phi_{\rm ref}^{\rm gap}$ is the averaged surfacic flux of the reference calculation and $\phi_{\rm ref}^{\rm row}$ is the averaged volumic flux in a row of cells of the reference calculation. Using this definition, the averaged SPH factor is equal to
$$
\bar\mu={\phi_{\rm ref}^{\rm row}\over \phi_{\rm ref}^{\rm gap} } \ \ .
$$

\item[\moc{ASYM}] keyword to specify the use of asymptotic normalization of the
SPH factors. The SPH factors in homogenized mixture \dusa{mixs} are set to one
in all macro-energy groups.

\item[\dusa{mixs}] index of the homogenized mixture where asymptotic normalization
is performed.

\item[\moc{STD\_FISS}] keyword to specify the use of flux-volume normalization in all {\sl fissile zones} for the SPH factors. This option is useful for representing assemblies
containing reflector zones.

\item[\moc{ARM}] keyword to activate a solution technique other than the collision probability method. Used with the Eurydice
solution technique within \moc{SYBILT:} to activate the current iteration method.

\item[\moc{ITER}] keyword to specify the main convergence parameters used to control SPH iterations.

\item[\dusa{maxout}] user-defined maximum number of SPH iterations (default value: $200$).

\item[\dusa{epsout}] user-defined convergence criterion (default value: $1.0 \times 10^{-4}$).

\item[\moc{MAXNB}] keyword to specify an auxiliary convergence parameter used to control SPH iterations.

\item[\dusa{maxnb}] acceptable number of SPH iterations with an increase in convergence error before
aborting (default value: $10$).

\item[\moc{GRMIN}] keyword to specify the minimum group index considered
during the equivalence process.

\item[\dusa{lgrmin}] first group number considered during the
equivalence process. By default, \dusa{lgrmin} $=1$.

\item[\moc{GRMAX}]  keyword to specify the maximum group index considered
during the equivalence process.

\item[\dusa{lgrmax}] last group index considered during the equivalence
process. By default, \dusa{lgrmax} is set to the last group
index in the RHS macrolib.

\item[\moc{EQUI}] keyword used to select an existing set of SPH factors in \dusa{SAPNAM} or to store
a new set of SPH factors in \dusa{SAPNEW} or \dusa{SAPNAM}. Also used as \moc{EQUI} \dusa{TEXT80} to select an
existing set of SPH factors in \dusa{APXNAM} or to store a new set of SPH factors in \dusa{APXNAM}.

\item[\dusa{TEXT4}] character*4 user-defined keyword of a set of SPH factors. This keyword is related to variable
\dusa{parkey}, as defined in Sect.~\ref{sect:descsap1} for a local variable.

\item[\moc{LOCNAM}] keyword used to define a character*80 name for the set of SPH factors, if this set is created. By
default, \dusa{TEXT80} is taken equal to \dusa{TEXT4}.

\item[\dusa{TEXT80}] character*80 user-defined name associated to keyword \dusa{TEXT4}. This name is related to
variable \dusa{parnam}, as defined in Sect.~\ref{sect:descsap1} for a local variable. Also used to identify a set of
SPH factors in an \dds{apex} or \dds{mpo} file.

\item[\moc{UPS}] keyword to specify that the macrolib and/or microlib cross sections recovered from a Saphyb, APEX or MPO file are
corrected so as to eliminate up-scattering. This option is useful for reactor analysis codes which cannot
take into account such cross sections.

\item[\moc{LEAK}] keyword used to introduce leakage in the embedded {\sc macrolib}. The buckling is recovered from
a RHS \dds{multicompo}, \dds{saphyb}, \dds{apex} or \dds{mpo} database. Otherwise, the buckling is recovered from optional variable \dusa{b2}.
This option should only be used for non-regression tests.

\item[\dusa{b2}] the imposed buckling corresponding to the leakage.

\end{ListeDeDescription}
\eject
 % structure (dragonSPH)
\subsection{The \moc{CFC:} module}\label{sect:CFCData}

The \moc{CFC:} module is used to generate a Feedback Model database required for a full core
calculation in DONJON.\cite{sissaoui} The general
specifications of this module are:

\begin{DataStructure}{Structure \dstr{CFC:}}
\dusa{CFCNAM} \moc{:=} \moc{CFC:} $[$ \dusa{CFCNAM} $]$ \\
\hspace*{1.0cm} (\dusa{CPONAM}($i$), $i$=1,28) \moc{::} \dstr{desccfc}
\end{DataStructure}

\noindent 
 where

\begin{ListeDeDescription}{mmmmmmmm} 

\item[\dusa{CFCNAM}] \verb|character*12| name of the \dds{fbmxsdb} data structure containing
the Feedback Model reactor database. The reactor database can be updated if \dusa{CFCNAM}
appears on the RHS.

\item[\dusa{CPONAM}] \verb|character*12| name of read only \dds{cpo} data structures. There
are 28 different \dds{cpo} data structures required here each containing respectively

\begin{enumerate}

\item the reactor reference cross section.

\item cell cross section for the first fuel temperature.

\item cell cross section for the second fuel temperature.

\item cell cross section for the first coolant temperature.

\item cell cross section for the second coolant temperature.

\item cell cross section for the first moderator temperature.

\item cell cross section for the second moderator temperature. 

\item cell cross section for the first coolant density. 

\item cell cross section for the second coolant density. 

\item cell cross section for the first moderator density.

\item cell cross section for the second moderator density.

\item cell cross section for a different concentration of boron. 

\item cell cross section for a different moderator purity. 

\item cell cross section for a different concentration of xenon. 

\item cell cross section for a different concentration of samarium.

\item cell cross section for a different concentration of neptunium.

\item cell cross section for the spectral mixed effect fuel/coolant density.

\item cell cross section for the spectral mixed effect coolant density/temperature.

\item cell cross section for low power history. 

\item cell cross section for intermediate power history. 

\item cell cross section for high power history. 

\item reactor reference moderator cross section.

\item moderator cross section for the first moderator temperature.

\item moderator cross section for the second moderator temperature.

\item moderator cross section for the first moderator density. 

\item moderator cross section for the second moderator density.

\item moderator cross section for a different concentration of boron. 

\item moderator cross section for a different moderator purity.

\end{enumerate}

\item[\dstr{desccfc}] structure containing the input data to this module (see
\Sect{desccfc}).

\end{ListeDeDescription}


\subsubsection{Data input for module \moc{CFC:}}\label{sect:desccfc}

\begin{DataStructure}{Structure \dstr{desccfc}}
$[$ \moc{EDIT} \dusa{iprint} $]$ \\
$[$ \moc{INFOR} \dusa{TITLE} $]$ \\
$[$ \moc{DNAME} \dusa{RNANE} $]$ \\
$[$ \moc{PWR} \dusa{powerref} \dusa{powerup} \dusa{powerint} \dusa{powerdown} $]$ \\
$[$ \moc{TCOOL} \dusa{tcoolref} \dusa{tcoolup} \dusa{tcooldown} $]$ \\
$[$ \moc{TMODE} \dusa{tmoderef} \dusa{tmodeup} \dusa{tmodedown} $]$ \\
$[$ \moc{TFUEL} \dusa{tfuelref} \dusa{tfuelup} \dusa{tfueldown} $]$ \\
$[$ \moc{RHOC} \dusa{denscool}$]$ \\
$[$ \moc{RHOM} \dusa{densmode}$]$ \\
$[$ \moc{XIR} \dusa{purityref} \dusa{puritydown} $]$ \\
\end{DataStructure}

\noindent where


\begin{ListeDeDescription}{mmmmmmmm} 

\item[\moc{EDIT}] keyword used to modify the print level \dusa{iprint}.

\item[\dusa{iprint}] index used to control the printing of the module. 

\item[\moc{INFOR}] keyword which allows to set the title.

\item[\dusa{TITLE}] \verb|character*72| title associated to the reactor database generated. 

\item[\moc{DNAME}] keyword to set a specific database name in the data
structure.
 
\item[\dusa{RNAME}] \verb|character*12| name of the feedback database. 

\item[\moc{PWR}] keyword to specify power used for evolution for power history.
 
\item[\dusa{powerref}]  power value for regular power history (\dusa{CPONAM} default). 

\item[\dusa{powerup}]  power value for high power history (\dusa{CPONAM} 21). 

\item[\dusa{powerint}]  power value for intermediate power history (\dusa{CPONAM} 20). 

\item[\dusa{powerdown}]  power value for low power history (\dusa{CPONAM} 19). 

\item[\moc{TCOOL}] keyword to specify coolant temperature used for regular evolution and perturbed cases.
 
\item[\dusa{tcoolref}]  normal coolant temperature (\dusa{CPONAM} default). 

\item[\dusa{tcoolup}]  high coolant temperature (\dusa{CPONAM} 4). 

\item[\dusa{tcooldown}]  low coolant temperature (\dusa{CPONAM} 5). 

\item[\moc{TMODE}] keyword to specify moderator temperature used for regular evolution and perturbed cases.
 
\item[\dusa{tmoderef}]  normal moderator temperature (\dusa{CPONAM} default). 

\item[\dusa{tmodeup}]  high moderator temperature (\dusa{CPONAM} 6 and 23). 

\item[\dusa{tmodedown}]  low moderator temperature (\dusa{CPONAM} 7 and 24). 

\item[\moc{TFUEL}] keyword to specify fuel temperature used for regular evolution and perturbed cases.
 
\item[\dusa{tfuelref}]  normal fuel temperature (\dusa{CPONAM} default). 

\item[\dusa{tfuelup}]  high fuel temperature (\dusa{CPONAM} 2). 

\item[\dusa{tfueldown}]  low fuel temperature (\dusa{CPONAM} 3). 

\item[\moc{RHOC}] keyword to specify coolant density used for regular evolution.
 
\item[\dusa{denscool}]  normal  coolant density (\dusa{CPONAM} default). 

\item[\moc{RHOM}] keyword to specify moderator density used for regular evolution.
 
\item[\dusa{densmode}]  normal moderator density (\dusa{CPONAM} default). 

\item[\moc{XIR}] keyword to specify water purity ($D_2O$ content) used for regular evolution and perturbed cases.
 
\item[\dusa{purityref}]  normal moderator purity (fraction of $D_2O$ in water) (\dusa{CPONAM} default). 

\item[\dusa{puritydown}]  perturbed moderator purity (fraction of $D_2O$ in water) (\dusa{CPONAM} 13 and 28). 

\end{ListeDeDescription}


Note: Other perturbed values are recovered directly from the concentrations and isotope densities stored in the different \dusa{CPONAM}. 
\clearpage
 % structure (dragonCFC)
\subsection{The {\tt INFO:} module}\label{sect:INFOData}

The \moc{INFO:} module is mainly used to compute the number densities for
selected isotopes at specific local conditions. The module can also be used to
compute the density $\rho(T,p,x)$ for a mixture containing a fraction $x$ of heavy and $(1-x)$ of light water according at a temperature $T$ and pressure $p$:
  $$
\rho(T,p,x) = {{\ \rho_{H_2O}(T,p)\ \rho_{D_2O}(T,p)}
\over{ x\ \rho_{H_2O}(T,p) +  (1-x) \ \rho_{D_2O}(T,p)}}\ .
  $$
where $\rho_{H_2O}(T,p)$ and $\rho_{D_2O}(T,p)$ will take different forms depending on the option selected.\cite{Kieffer}

\vskip 0.2cm

The calling specifications are:

\begin{DataStructure}{Structure \dstr{INFO:}}
\moc{INFO:} \moc{::} \dstr{descinfo}
\end{DataStructure}

\goodbreak
\noindent where
\begin{ListeDeDescription}{mmmmmmmm}

\item[\dstr{descinfo}] structure containing the input data to this module
(see \Sect{descinfo}).

\end{ListeDeDescription}

\vskip 0.2cm

\subsubsection{Data input for module {\tt INFO:}}\label{sect:descinfo}

\begin{DataStructure}{Structure \dstr{info}}
$[$ \moc{EDIT} \dusa{iprint} $]$ \\
$[$ \moc{LIB:} $\{$ \moc{DRAGON} $|$ \moc{MATXS}  $|$ \moc{MATXS2} $|$
                    \moc{WIMSD4} $|$ \moc{WIMSAECL} $|$ \moc{NDAS} $|$    
                    \moc{APLIB2} $|$ \moc{APLIB1} $\}$ \\
~~~~~~~\moc{FIL:} \dusa{NAMEFIL} $]$ \\
$[$ \moc{TMP:} \dusa{temp} $\{$ \moc{K} $|$ \moc{C} $\}$ $]$ \\
$[$ \moc{PUR:} \dusa{purity} $\{$ \moc{WGT\%} $|$ \moc{ATM\%} $\}$ $]$ \\
$[$ \moc{PRES:} \dusa{pressure} $\{$ \moc{bar} $|$ \moc{Pa} $|$ \moc{kPa} $|$ \moc{MPa} $\}$ $]$ \\
$[$ \moc{CALC} \moc{DENS} $\{$ \moc{WATER} $>>$\dusa{dens}$<<$ $|$ \moc{PWATER} $>>$\dusa{dens}$<<$ $\}$ $]$ \\
$[$ \moc{ENR:} \dusa{enrichment} $\{$ \moc{WGT\%} $|$ \moc{ATM\%} $\}$ $]$ \\
$[[$ \moc{ISO:} \dusa{nbiso} (\dusa{ISONAM}($i$), $i$=1,nbiso)  \\
$\ \ $  $\{$ \moc{GET}   \moc{MASS} ($>>$\dusa{mass}($i$)$<<$, $i$=1,nbiso) $|$ 
           \moc{CALC}  \moc{WGT\%}  $\{$  \\
\hskip 1.5cm \moc{D2O} $>>$\dusa{nh1}$<<$ $>>$\dusa{hd2}$<<$ $>>$\dusa{no16}$<<$
$|$\\
\hskip 1.5cm \moc{H2O} $>>$\dusa{nh1}$<<$ $>>$\dusa{hd2}$<<$ $>>$\dusa{no16}$<<$
$|$\\
\hskip 1.5cm \moc{UO2} $>>$\dusa{nu5}$<<$ $>>$\dusa{hu8}$<<$ $>>$\dusa{no16}$<<$
$|$\\
\hskip 1.5cm \moc{THO2} $>>$\dusa{nth2}$<<$ $>>$\dusa{nu3}$<<$ $>>$\dusa{no16}$<<$
$\}$ $\}$ 
$]]$
\end{DataStructure}

\noindent
where

\begin{ListeDeDescription}{mmmmmmmm}

\item[\moc{EDIT}] keyword used to modify the print level \dusa{iprint}.

\item[\dusa{iprint}] index used to control the printing of the module. The
amount of output produced by this tracking module will vary substantially
depending on the print level specified.

\item[\moc{LIB:}] keyword to specify the type of library from which the
isotopic mass ratio is to be read. 

\item[\moc{DRAGON}] keyword to specify that the isotopic depletion chain or
the microscopic cross sections are in the DRAGLIB format.

\item[\moc{MATXS}] keyword to specify that the microscopic cross sections are
in the MATXS format of NJOY-II and NJOY-89 (no depletion data available for
libraries using this format).

\item[\moc{MATXS2}] keyword to specify that the microscopic cross sections are
in the MATXS format of NJOY-91  (no depletion data available for libraries using
this format).

\item[\moc{WIMSD4}] keyword to specify that the isotopic depletion chain and the
microscopic cross sections are in the WIMSD4 format.

\item[\moc{WIMSAECL}] keyword to specify that the isotopic depletion chain and the
microscopic cross sections are in the WIMS-AECL format.

\item[\moc{NDAS}] keyword to specify that the isotopic depletion chain and the
microscopic cross sections are in the NDAS format, as used in recent versions of WIMS-AECL.

\item[\moc{APLIB1}] keyword to specify that the microscopic cross sections are
in the APOLLO-1 format.

\item[\moc{APLIB2}] keyword to specify that the microscopic cross sections are
in the APOLLO-2 format.

\item[\moc{FIL:}] keyword to specify the name of the file where is  stored the mass
ratio data. 

\item[\dusa{NAMEFIL}] \verb|character*8| name of the library where the mass ratio
are stored.

\item[\moc{TMP:}] keyword to specify the isotopic temperature.

\item[\dusa{temp}] temperature $T$ in \moc{K} or \moc{C}.

\item[\moc{PUR:}] keyword to specify the water purity, that is fraction of heavy
water in a mix of heavy and light water.

\item[\dusa{purity}] percent weight (\moc{WGT\%}) or atomic (\moc{ATM\%}) fraction of heavy
water in a mix of heavy and light water ($100\times x$).

\item[\moc{PRES:}] keyword to specify the pressure.

\item[\dusa{pressure}] pressure $p$ in \moc{bar}, \moc{Pa}, \moc{kPa} or \moc{MPa}.

\item[\moc{ENR:}] keyword to specify the fuel enrichment.

\item[\dusa{enrichment}] fuel enrichment in weight percent (\moc{WGT\%}) or atomic
percent (\moc{ATM\%}).

\item[\moc{ISO:}] keyword to specify an isotope list. This list will be used either
for getting mass values of isotopes or for computing number  densities.

\item[\dusa{nbiso}] number of isotopic names used for a calculation (limited to
\dusa{nbiso}$\leq 3$).

\item[\dusa{ISONAM}] \verb|character*12| name of an isotope.

\item[\moc{GET MASS}] keyword to recover the mass values as written in the library.
It returns the mass value of each isotope in the output parameter \dusa{mass}. 

\item[\moc{CALC}] keyword to ask the module to compute some parametric values. It
returns one value in the output parameter \dusa{dens}. 

\item[\moc{DENS}] compute density of a mixture of light and heavy water.

\item[\moc{WATER}] keyword to recover the water density as a
function of its temperature and purity (independent of pressure $p$). This option requires the setting of
temperature and purity, and it does not affect any given list of isotope names. This module relies on the water density calculator of WIMS-AECL.\cite{WIMS}

\item[\moc{PWATER}] new keyword to recover the water density as a
function of temperature, pressure and purity developed by C. Kieffer.\cite{Kieffer} This option requires the setting of
temperature, pressure and purity, and it does not affect any given list of isotope names. For light water, it uses the \moc{freesteam} routines.\cite{Freesteam} For heavy water, two options are considered.
\begin{enumerate}
\item For $90\text{ C}<T<350\text{ C}$ and $p<22$ MPa, the heavy water routines written by Ji Zhang at AECL and distributed freely by B. Garland from McMaster University are considered.\cite{McMaster}
\item Otherwise, the density is that of obtained from \moc{freesteam} multiplied by a factor of 1.11 which is approximately the ratio of the molecular mass of D2O to H2O.
\end{enumerate}  

\item[\moc{WGT\%} \moc{D2O}] keywords to recover 3 number densities for a compound
mixture of heavy and light water. The isotope list is assumed to contain $^{1}$H,
$^{2}$D and $^{2}$O. Temperature and purity are supposed to be available. It returns
concentration of these isotopes in the output parameters \dusa{nh1}, \dusa{nd2} and
\dusa{no16}.

\item[\moc{WGT\%} \moc{H2O}] is identical to \moc{WGT\%} \moc{D2O}.

\item[\moc{WGT\%} \moc{UO2}] keywords to recover 3 number densities for a compound
mixture of Uranium oxide. The isotope list is assumed to contain $^{235}$U,
$^{238}$U and $^{16}$O. The $^{235}$U enrichment is supposed to be available. Note
that the number densities will sum to 100. It returns concentration of these
isotopes in the output parameters \dusa{nu5}, \dusa{nu8} and \dusa{no16}.

\item[\moc{WGT\%} \moc{THO2}] keywords to recover 3 number densities for a compound
mixture of Thorium/Uranium oxide. The isotope list is assumed to contain
$^{232}$Th,  $^{233}$U  and $^{16}$O. The $^{233}$U enrichment is supposed to be
available. Note that the number densities will sum to 100. It returns concentration
of these isotopes in the output parameters \dusa{nth2}, \dusa{nu3} and \dusa{no16}.

\end{ListeDeDescription}

The \moc{INFO:} module works the following way. For a given isotope list, the mass is
extracted from the library or a calculation process is expected. Once this
calculation is has been performed, it is possible to list other isotopes and ask for
further calculations. Finally note that the number of output parameters, denoted by
$>>$\dusa{param}$<<$, are recovered as CLE-2000 variables in \dstr{descinfo}. The number
of these parameters must be equal to the number of isotopes
names given, plus the water density when a command \moc{CALC} \moc{DENS} \moc{WATER}
is issued.

\eject
 % structure (dragonI)
\subsection{The \moc{MRG:} module}\label{sect:MRGData}


The \moc{MRG:} module is used to pre-homogenize a geometry after it has been tracked with the \moc{EXCELT:} module. This module can also be used for
the same purpose for \moc{NXT:} tracked geometries.\cite{MRG1,MRG2,Harrisson2011a} In addition, \moc{NXT:} based tracking files can
also be partition using this module. 

The general specifications for this module are presented in \Tabto{MRGDataexcelt}{PARTDatanxtfil}

\begin{DataStructure}{Structure for merging \moc{EXCELT:} tracks}\label{tab:MRGDataexcelt}
\dusa{TRKENEW} \dusa{TFILENEW}
\moc{:=} \moc{MRG:}   \dusa{TRKEOLD} \dusa{TFILEOLD} \moc{::}  \dstr{descmrg} 
\end{DataStructure}

\begin{DataStructure}{Structure for merging \moc{NXT:} tracks}\label{tab:MRGDatanxttrk}
\dusa{TRKNNEW} \moc{:=} \moc{MRG:}   \dusa{TRKNOLD} \moc{::}  \dstr{descmrg} 
\end{DataStructure}

\begin{DataStructure}{Structure for partitioning \moc{NXT:} tracking files}\label{tab:PARTDatanxtfil}
\dusa{TFILEMOD} \dusa{TFILEEXT} \moc{:=} \moc{MRG:}   \dusa{TRKNOLD} \dusa{TFILEOLD} \moc{::}  \dstr{descextr} 
\end{DataStructure}

\begin{ListeDeDescription}{mmmmmmmm}  

\item[\dusa{TRKENEW}] \verb|character*12| name of the new \dds{tracking} data structure that will contain
region volume and surface area vectors in addition to region identification pointers and other tracking
information after the pre-homogenization process.

\item[\dusa{TFILENEW}] \verb|character*12| name of the new \moc{EXCELT:} compatible sequential binary tracking file used to store
the tracks lengths after the pre-homogenization process has take place. 

\item[\dusa{TFILEMOD}] \verb|character*12| name of the new \moc{NXT:} sequential binary tracking file where the
lines not-associated with the regions to extract are stored.

\item[\dusa{TFILEEXT}] \verb|character*12| name of the new \moc{NXT:} compatible sequential binary tracking file where the
lines associated with the regions to extract are stored.

\item[\dusa{TRKEOLD}] \verb|character*12| name of the \dds{tracking} data structure that contains
region volume and surface area vectors in addition to region identification pointers and other tracking
information before the pre-homogenization process.

\item[\dusa{TFILEOLD}] \verb|character*12| name of the old sequential binary tracking file used to store
the tracks lengths before the pre-homogenization process takes place. 

\item[\dstr{descmrg}] structure containing the input data to this module (see \Sect{descmrg}).

\item[\dstr{descextr}] structure containing the input data for track file partitioning process (see \Sect{descextr}).

\end{ListeDeDescription}


\subsubsection{Data input for geometry pre-homogenization}\label{sect:descmrg}

\begin{DataStructure}{Structure \dstr{descmrg}}
$[$ \moc{EDIT} \dusa{iprint} $]$ \\
$[$ \moc{REGI} (\dusa{irmrg}($i$), $i$=1,\dusa{nreg}) $]$ \\
$[$ \moc{SURF} (\dusa{ismrg}($i$), $i$=1,\dusa{nsur}) $]$ 
\end{DataStructure}

\begin{ListeDeDescription}{mmmmmmmm}   

\item[\moc{EDIT}] keyword used to modify the print level \dusa{iprint}.

\item[\dusa{iprint}] index used to control the printing in this module. 

\item[\moc{REGI}] keyword to specify that regions will be pre-homogenized.

\item[\dusa{irmrg}] list of new region numbers associated with old region numbers. Two or more regions can be
combined together only if they contain the same mixture. The number \dusa{nreg} of region is that printed
after the execution of the tracking module.

\item[\moc{SURF}] keyword to specify that surfaces will be pre-homogenized.

\item[\dusa{ismrg}] list of new surface numbers associated with old surface numbers. Two or more surfaces can be
combined together only if they are associated with the same boundary conditions. The number \dusa{nsur} of surfaces is
that printed after the execution of the tracking module.

\end{ListeDeDescription}

\subsubsection{Data input for tracking file partitioning}\label{sect:descextr}

\begin{DataStructure}{Structure \dstr{descextr}}
$[$ \moc{EDIT} \dusa{iprint} $]$ \\
$[$ \moc{EXTR} (\dusa{iext}($i$), $i$=1,\dusa{nreg}) $]$ \\
\end{DataStructure}

\noindent
 where

\begin{ListeDeDescription}{mmmmmmmm}   

\item[\moc{EDIT}] keyword used to modify the print level \dusa{iprint}.

\item[\dusa{iprint}] index used to control the printing in this module.

\item[\moc{EXTR}] keyword to specify that the track associated with a specific set of regions will be extracted from
the reference tracking file.

\item[\dusa{iext}] list of region numbers for track extraction. The number nreg of region is that printed after the execution
of the tracking module.

\end{ListeDeDescription}

\eject
 % structure (dragonMRG)
\subsection{The {\tt COMPO:} module}\label{sect:COMPOData}

This component of the lattice code is dedicated to the constitution of the
reactor database intended to store {\sl all} the nuclear data, produced in
the lattice code, that is useful
in reactor calculations including fuel management and space-time kinetics.
Multigroup lattice calculations are too expensive to be executed dynamically
from the driver of the global reactor calculation. A more feasible
approach is to create a reactor database where a finite number of lattice
calculation results are tabulated against selected {\sl global} and/or {\sl local parameters}
chosen so as to represent expected operating conditions of the reactor.

\vskip 0.1cm

The {\tt COMPO:} module is used to create and construct a {\sc multicompo} object.
This object is generally {\sl persistent} and used to collect information gathered
from many DRAGON {\sl elementary calculations} performed under various conditions.

\vskip 0.1cm

For each elementary calculation, the results are recovered from the output of the
{\tt EDI:} module and stored in a list of {\sl homogenized mixture}
directories. The {\tt EDI:} module is responsible for performing condensation
in energy, homogenization in space of the microscopic cross sections and constitution
of {\sl macroscopic sets} for collecting together many isotopes. All the elementary
calculations gathered in a single {\sc multicompo} object are characterized by the same
number of {\sl homogenized mixtures} and by a specific output energy-group structure.

\vskip 0.1cm

\begin{figure}[h!]
\begin{center} 
\epsfxsize=9cm
\centerline{ \epsffile{compo.eps}}
\parbox{15cm}{\caption{Organization of a {\tt multicompo} 
object.}\label{fig:compo}}   
\end{center}  
\end{figure}

Each elementary calculation is characterized by a tuple of {\sl global} and/or {\sl local parameters}
Global parameters are characteristics of the complete lattice, while local parameters
are characteristics of each homogenized mixture. These parameters are of different types,
depending on the nature of the
study under consideration: type of assembly, power, temperature in a mixture,
concentration of an isotope, time, burnup or exposure rate in a depletion calculation,
etc. Each step of a depletion calculation represents an elementay calculation.
The {\sc multicompo} object is often presented as a {\sl multi-parameter reactor database}.

\vskip 0.1cm

\begin{figure}[h!]  
\begin{center} 
\epsfxsize=9.5cm
\centerline{ \epsffile{tree.eps}}
\parbox{14cm}{\caption{Parameter tree in a {\sc multicompo} object}\label{fig:ctree}}   
\end{center}  
\end{figure}

The {\sc multicompo} object is organized as shown in \Fig{compo}. The root of the object contains
table--of--content information for global and local parameters and two lists of
directories. Each component of the first list ({\tt 'MIXTURES'})
contains the directory {\tt 'TREE'} (the parameter tree) and the list ({\tt 'CALCULATIONS'})
made of {\sc microlib} objects. Each component of the second list ({\tt 'GEOMETRIES'}) contains the homogenized
geometry of an elementary calculation.

\vskip 0.1cm

The localization of an elementary calculation is done using
a tuple of global and/or local parameters. The elementary calculation indices are
stored in a tree with the number of levels equal to the number of global and local parameters.
An example of a tree with three parameters is shown in \Fig{ctree}. Each node
of this tree is associated with the index of the corresponding parameter and with the
reference to the daughter nodes if they exist. The number if leafs is equal to the number
of nodes for the last (third) parameter and is equal to the number of elementary
calculations stored in the {\sc multicompo} object. The index of each elementary calculation is
therefore an attribute of each leaf.

\vskip 0.1cm

In each homogenized mixture component, the {\tt COMPO:} module recover
cross sections for a number of {\sl particularized isotopes} and of a single {\sl macroscopic
set}, a collection of the remaining isotopic cross sections weighted by isotopic number densities.
Other information is also recovered: multigroup neutron
fluxes, isotopic number densities, fission spectrum, delayed neutron data, etc.

\vskip 0.1cm

A different specification of the \moc{COMPO:} function call is used for
creation and construction of the {\sc multicompo} object.
\begin{itemize}
\item The first specification is used to initialize the {\sc multicompo} data structure
and to set the choice of global and local parameters.
\item A modification call to the \moc{COMPO:} function is performed after each
elementary calculation in order to recover output information processed by \moc{EDI:}
(condensed and homogenized cross sections) and \moc{EVO:} (burnup dependant values).
Global and local parameters can optionnally be recovered from \dds{microlib}
objects.
\item Another modification call to the \moc{COMPO:} function is used to
catenate a {\sl read-only} {\sc multicompo} object into a {\sl master} {\sc multicompo} object.
\end{itemize}

The calling specifications are:

\begin{DataStructure}{Structure \dstr{COMPO:}}
$\{$~~\dusa{CPONAM} \moc{:=} \moc{COMPO:} $[$ \dusa{CPONAM} $]$ \moc{::} \dstr{compo\_data1} \\
$|$~~\dusa{CPONAM} \moc{:=} \moc{COMPO:} \dusa{CPONAM}~\dusa{EDINAM}~$[$ \dusa{EDINA2} $]~[$ \dusa{BRNNAM} $]~[$~\dusa{HMIC1}~$[$~\dusa{HMIC2} $]~]$ \\
~~~~~~~~~~~~~~ \moc{::} \dstr{compo\_data2} \\
$|$~~\dusa{CPONAM} \moc{:=} \moc{COMPO:} \dusa{CPONAM} $[[$ \dusa{CPORHS} $]]$ \moc{::} \dstr{compo\_data3} \\
$|$~~\moc{COMPO:} \dusa{CPONAM} \moc{::} \dstr{compo\_data4}~$\}$ \\
\end{DataStructure}

\noindent where
\begin{ListeDeDescription}{mmmmmmm}

\item[\dusa{CPONAM}] {\tt character*12} name of the {\sc lcm} object containing the
{\sl master} {\sc multicompo} data structure.

\item[\dusa{EDINAM}] {\tt character*12} name of the {\sc lcm} object (type {\tt
L\_EDIT}) containing the {\sc edition} data structure corresponding to an elementary
calculation. This {\sc edition} data structure is containing homogenized and
condensed cross-section information. The {\sc edition} data produced by the last call
to the {\tt EDI:} module is used. It is possible to provide a {\sc macrolib} data structure
as replacement for the {\sc edition} data structure. In this case, the \moc{MACRO} keyword
is automatically set.

\item[\dusa{EDINA2}] {\tt character*12} name of an optional {\sc lcm} object (type {\tt
L\_EDIT}) containing the {\sc edition} data structure corresponding to an elementary
calculation. This {\sc edition} data structure is containing {\sl group form factor}
information. The {\sc edition} data produced by the last call to the {\tt EDI:} module
is used.

\item[\dusa{BRNNAM}] {\tt character*12} name of the {\sc lcm} object (type {\tt
L\_BURNUP}) containing the {\sc burnup} data structure.

\item[\dusa{HMIC1}] {\tt character*12} name of a \dds{microlib} (type {\tt
L\_LIBRARY}) containing global parameter information.

\item[\dusa{HMIC2}] {\tt character*12} name of a \dds{microlib} (type {\tt
L\_LIBRARY}) containing global parameter information.

\item[\dusa{CPORHS}] {\tt character*12} name of the {\sl read-only} {\sc multicompo} data structure. This
data structure is concatenated to \dusa{CPONAM} using the \dusa{compo\_data3} data structure,
as presented in \Sect{desccpo3}. \dusa{CPORHS} must be defined with the same number of energy
groups and the same number of homogeneous regions as \dusa{CPONAM}. Moreover, all the
global and local parameters of \dusa{CPORHS} must be defined in \dusa{CPONAM}. \dusa{CPONAM}
may be defined with {\sl global} parameters not defined in \dusa{CPORHS}.

\item[\dusa{compo\_data1}] input data structure containing initialization information (see \Sect{desccpo1}).

\item[\dusa{compo\_data2}] input data structure containing information related to the recovery of an
elementary calculation (see \Sect{desccpo2}).

\item[\dusa{compo\_data3}] input data structure containing information related to the catenation of one or many
{\sl read-only} {\sc multicompo} (see \Sect{desccpo3}).

\item[\dusa{compo\_data4}] input data structure containing information related to the display of a
{\sl read-only} {\sc multicompo} (see \Sect{desccpo4}).

\end{ListeDeDescription}

\subsubsection{Initialization data input for module {\tt COMPO:}}\label{sect:desccpo1}

\vskip -0.5cm

\begin{DataStructure}{Structure \dstr{compo\_data1}}
$[$~\moc{EDIT} \dusa{iprint}~$]$ \\
$[[~[$ \moc{STEP} \moc{UP} \dusa{NAMDIR} $]$ \\
~~~$[$~\moc{MAXCAL} \dusa{maxcal}~$]$ \\
~~~$[$~\moc{COMM}~$[[$~\dusa{HCOM}~$]]$~\moc{ENDC}~$]$ \\
~~~$[[$~\moc{PARA}~\dusa{PARKEY} \\
~~~~~~\{~\moc{TEMP}~\dusa{HMIC}~\dusa{imix}~$|$~\moc{CONC}~\dusa{HISO1}~\dusa{HMIC}~\dusa{imix}~$|$~\moc{IRRA}~$|$~\moc{FLUB}~$|$ \\
~~~~~~~~~\moc{POWR}~$|$~\moc{MASL}~$|$~\moc{FLUX}~$|$~\moc{TIME}~$|$~\moc{VALU}~\{~\moc{REAL}~$|$~\moc{CHAR}~$|$~\moc{INTE}~\}~\} \\
~~~$]]$ \\
~~~$[[$~\moc{LOCA}~\dusa{PARKEY} \\
~~~~~~\{~\moc{TEMP}~$|$~\moc{CONC}~\dusa{HISO2}~$|$~\moc{IRRA}~$|$~\moc{FLUB}~$|$~\moc{FLUG}~$|$~\moc{POWR}~$|$~\moc{MASL}~$|$~\moc{FLUX}~\} \\
~~~$]]$ \\
$[$~\moc{ISOT}~\dusa{nisp} (\dusa{HISOP}(i),i=1,\dusa{nisp})~$]$ \\
$[$ \moc{GFF} $]~[~\{$ \moc{NOALBP} $|$ \moc{ALBP} $\}~]~[~\{$ \moc{NOJSURF} $|$ \moc{JSURF} $\}~]$ \\
\moc{INIT} $]]$ \\
{\tt ;}
\end{DataStructure}

\noindent where
\begin{ListeDeDescription}{mmmmmmmm}

\item[\moc{EDIT}] keyword used to set \dusa{iprint}.

\item[\dusa{iprint}] index used to control the printing in module {\tt
COMPO:}. =0 for no print; =1 for minimum printing (default value).

\item[\moc{STEP}] keyword used to create the database from a sub-directory named \dusa{NAMDIR}. This capability
make possible the creation of a single object with many independent {\sc multicompo} structures in it. By default,
the database is created on directory {\tt 'default'}.

\item[\moc{UP}] keyword used to move up towards a sub-directory of \dusa{CPONAM}.

\item[\dusa{NAMDIR}] create the {\sc multicompo} structure in the sub-directory named \dusa{NAMDIR}.

\item[\moc{MAXCAL}] keyword used to set \dusa{maxcal}.

\item[\dusa{maxcal}] maximum number of elementary calculations to be stored
in the {\sc multicompo}. \dusa{maxcal}$=10$ by default. This maximum size is
automatically increased when the number of elementary calculations exceeds
the current value of \dusa{maxcal}.

\item[\moc{COMM}] keyword used to input a general comment for the {\sc multicompo}.

\item[\dusa{HCOM}] {\tt character*80} user-defined comment.

\item[\moc{ENDC}] end--of--HCOM keyword.

\item[\moc{PARA}] keyword used to define a single global parameter.

\item[\moc{LOCA}] keyword used to define a single local parameter.

\item[\dusa{PARKEY}] {\tt character*12} user-defined keyword associated to a global
or local parameter.

\item[\dusa{HMIC}] {\tt character*12} name of the \dds{microlib} (type {\tt
L\_LIBRARY}) associated to a global parameter. The corresponding \dds{microlib} will be required on
RHS of the \moc{COMPO:} call described in Sect.~\ref{sect:desccpo2}.

\item[\dusa{imix}] index of the mixture associated to a global parameter. This mixture is
located in \dds{microlib} named \dusa{HMIC}.

\item[\dusa{HISO1}] {\tt character*8} alias name of the isotope associated to a global
parameter. This isotope is located in \dds{microlib} data structure named \dusa{HMIC}.

\item[\dusa{HISO2}] {\tt character*8} alias name of the isotope associated to a local
parameter. This isotope is located in the \dds{microlib} directory of the {\sc edition}
data structure named \dusa{EDINAM}.

\item[\moc{TEMP}] keyword used to define a temperature (in Kelvin) as global or
local parameter.

\item[\moc{CONC}] keyword used to define a number density as global or
local parameter.

\item[\moc{IRRA}] keyword used to define a burnup (in MWday/Tonne) as global
or local parameter.

\item[\moc{FLUB}] keyword used to define a {\sl fuel-only} exposure rate (in n/kb) as global
or local parameter. The exposure rate is recovered from the \dusa{BRNNAM}
LCM object.

\item[\moc{FLUG}] keyword used to define an exposure rate in global homogenized mixtures (in n/kb) as
local parameter. The exposure rate is recovered from the \dusa{BRNNAM}
LCM object.

\item[\moc{POWR}] keyword used to define the power as global or
local parameter.

\item[\moc{MASL}] keyword used to define the mass density of heavy isotopes as
global or local parameter.

\item[\moc{FLUX}] keyword used to define the volume-averaged, energy-integrated flux as
global or local parameter.

\item[\moc{TIME}] keyword used to define the time (in seconds) as global parameter.

\item[\moc{VALU}] keyword used to define a user-defined quantity as global parameter.
This keyword must be followed by the type of parameter.

\item[\moc{REAL}] keyword used to indicate that the user-defined global parameter
is a floating point value.

\item[\moc{CHAR}] keyword used to indicate that the user-defined global parameter
is a {\tt character*12} value.

\item[\moc{INTE}] keyword used to indicate that the user-defined global parameter
is an integer value.

\item[\moc{ISOT}] keyword used to select the set of particularized isotopes. By default, all the
isotopes available in the {\sc edition} data structure \dusa{EDINAM} are selected.

\item[\dusa{nisp}] number of user-defined particularized isotopes.

\item[\dusa{HISOP}] {\tt character*8} names of the user-defined particularized isotopes. These names must be present
in the {\sc edition} data structure \dusa{EDINAM}.

\item[\moc{GFF}] keyword used to enable the recovery of group form factor information from {\sc edition} data structure \dusa{EDINA2}.

\item[\moc{NOALBP}] keyword used to avoid the recovery of physical albedo information from {\sc edition} data structure \dusa{EDINAM}.

\item[\moc{ALBP}] keyword used to enable the recovery of physical albedo information from {\sc edition} data structure \dusa{EDINAM} (default option).

\item[\moc{NOJSURF}] keyword used to avoid the recovery of discontinuity factor and boundary multigroup current information from {\sc edition} data structure \dusa{EDINAM}.

\item[\moc{JSURF}] keyword used to enable the recovery of discontinuity factor and boundary multigroup current information from {\sc edition} data structure \dusa{EDINAM} (default option).

\item[\moc{INIT}] keyword used to create the empty structure in the {\sc multicompo}.

\end{ListeDeDescription}

\clearpage

\subsubsection{Modification data input for module {\tt COMPO:}}\label{sect:desccpo2}

\vskip -0.5cm

\begin{DataStructure}{Structure \dstr{compo\_data2}}
$[$ \moc{EDIT} \dusa{iprint} $]$ \\
$[$ \moc{ALLX} $]$ \\
$[$ \moc{STEP} \moc{UP} $\{$ \dusa{NAMDIR} $|$ \moc{*} $\}~]$ \\
$[$ \moc{ORIG} \dusa{orig} $]$ \\
$[[$ \dusa{PARKEY} \dusa{value} $]]$ \\
$[$ \moc{MACRO} $]~[$ \moc{SET} \dusa{xtr} $\{$ \moc{S} $|$ \moc{DAY} $|$ \moc{YEAR} $\}$ $]$ \\
{\tt ;}
\end{DataStructure}

\noindent where
\begin{ListeDeDescription}{mmmmmmmm}

\item[\moc{EDIT}] keyword used to set \dusa{iprint}.

\item[\dusa{iprint}] index used to control the printing in module {\tt
COMPO:}. =0 for no print; =1 for minimum printing (default value).

\item[\moc{ALLX}] keyword used to register the region number of each isotope before merging. This option is useful if the same
keyword has been specified in \moc{EDI:} before. This allows to perform subsequent depletion calculations, in taking into account
different fuel regions in the diffusion calculation.

\item[\moc{STEP}] keyword used to access the database from a sub-directory named \dusa{NAMDIR} instead of
accessing it from the root of \dusa{CPONAM}.

\item[\moc{UP}] keyword used to move up towards a sub-directory of \dusa{CPONAM}.

\item[\dusa{NAMDIR}] access the {\sc multicompo} structure in the sub-directory named \dusa{NAMDIR}.

\item[\moc{*}] use a sub-directory name identical to the directory in \dusa{EDINAM}
where the edition data is coming from.

\item[\moc{ORIG}] keyword used to define the father node in the parameter tree. By
default, the index of the previous elementary calculation is used.

\item[\dusa{orig}] index of the elementary calculation associated to the father node in the
parameter tree.

\item[\dusa{PARKEY}] {\tt character*12} keyword associated to a user-defined
global parameter.

\item[\dusa{value}] floating-point, integer or {\tt character*12} value of a user-defined
global parameter.

\item[\moc{MACRO}] keyword used to recover cross-section information from the
macrolib directory in \dusa{EDINAM}. By default, the cross-section information
is recovered from the microlib in \dusa{EDINAM}.

\item[\moc{SET}] keyword used to recover the flux normalization factor already
stored on \dusa{BRNNAM} from a sub-directory corresponding to a specific time.

\item[\dusa{xtr}] time associated with the current flux calculation. The
name of the sub-directory where this information is stored will be given by
`{\tt DEPL-DAT}'//{\tt CNN} where {\tt CNN} is a  {\tt character*4} variable
defined by  {\tt WRITE(CNN,'(I4.4)') INN} where {\tt INN} is an index associated
with the time \dusa{xtr}.

\item[\moc{S}] keyword to specify that the time is given in seconds.

\item[\moc{DAY}] keyword to specify that the time is given in days.

\item[\moc{YEAR}] keyword to specify that the time is given in years.

\end{ListeDeDescription}

\subsubsection{Modification (catenate) data input for module {\tt COMPO:}}\label{sect:desccpo3}

\vskip -0.5cm

\begin{DataStructure}{Structure \dstr{compo\_data3}}
$[$ \moc{EDIT} \dusa{iprint} $]$ \\
$[$ \moc{STEP} \moc{UP} \dusa{NAMDIR} $]$ \\
$[$ \moc{ORIG} \dusa{orig} $]$ \\
$[[$ \dusa{PARKEY} \dusa{value} $]]$ \\
$[$ \moc{WARNING-ONLY} $]$ \\
{\tt ;}
\end{DataStructure}

\noindent where
\begin{ListeDeDescription}{mmmmmmmm}

\item[\moc{EDIT}] keyword used to set \dusa{iprint}.

\item[\dusa{iprint}] index used to control the printing in module {\tt
COMPO:}. =0 for no print; =1 for minimum printing (default value).

\item[\moc{STEP}] keyword used to access the database from a sub-directory named \dusa{NAMDIR} instead of
accessing it from the root of \dusa{CPONAM}.

\item[\moc{UP}] keyword used to move up towards a sub-directory of \dusa{CPONAM}.

\item[\dusa{NAMDIR}] access the {\sc multicompo} structure in the sub-directory named \dusa{NAMDIR}.

\item[\moc{ORIG}] keyword used to define the father node in the parameter tree. By
default, the index of the previous elementary calculation is used.

\item[\dusa{orig}] index of the elementary calculation associated to the father node in the
parameter tree.

\item[\dusa{PARKEY}] {\tt character*12} keyword associated to a
global parameter that is specific to \dusa{CPONAM} (not defined in \dusa{CPORHS}).

\item[\dusa{value}] floating-point, integer or {\tt character*12} value of a
global parameter that is specific to \dusa{CPONAM}.

\item[\moc{WARNING-ONLY}] This option is useful if an elementary calculation in \dusa{CPORHS} 
is already present in \dusa{CPONAM}. If this keyword is set, a warning is send and the \dusa{CPONAM} values
are kept, otherwise the run is aborted (default).

\end{ListeDeDescription}

\clearpage

\subsubsection{Display data input for module {\tt COMPO:}}\label{sect:desccpo4}

\vskip -0.5cm

\begin{DataStructure}{Structure \dstr{compo\_data4}}
$[$ \moc{EDIT} \dusa{iprint} $]$ \\
$[$ \moc{STEP} \moc{UP} \dusa{NAMDIR} $]$ \\
 \moc{DB-STRUC} \\
{\tt ;}
\end{DataStructure}

\noindent where
\begin{ListeDeDescription}{mmmmmmmm}

\item[\moc{EDIT}] keyword used to set \dusa{iprint}.

\item[\dusa{iprint}] index used to control the printing in module {\tt
COMPO:}. $<$2 for MUPLET display only (default value) and parameters values are presented at the end, $\ge$2 for the parameter value display for each calculation.

\item[\moc{STEP}] keyword used to access the database from a sub-directory named \dusa{NAMDIR} instead of
accessing it from the root of \dusa{CPONAM}.

\item[\moc{UP}] keyword used to move up towards a sub-directory of \dusa{CPONAM}.

\item[\dusa{NAMDIR}] access the {\sc multicompo} structure in the sub-directory named \dusa{NAMDIR}.

\item[\moc{DB-STRUC}] {\tt character*12} keyword used to display the content of the \dusa{CPONAM} object for the \dusa{NAMDIR} directory.

\end{ListeDeDescription}

\clearpage
 % structure (dragonCOMPO)
\subsection{The \moc{TLM:} module}\label{sect:TLMData}

The \moc{TLM:} module has been designed to generate a Matlab \moc{m-file} (in an \moc{ASCII} format) that contains the instructions for
plotting the tracking lines generated by the \moc{NXT:} module or by the \moc{SALT:} module (\moc{LONG} option).\cite{Plamondon2006}
The \moc{TLM:} module is activated using the following list of commands:

\begin{DataStructure}{Structure \dstr{TLM:}}
\dusa{MFILE} \moc{:=} \moc{TLM:} \dusa{MFILE}  \dusa{TRKNAM} \dusa{TRKFIL} 
\moc{::} \dstr{desctlm}
\end{DataStructure}

\noindent
 where

\begin{ListeDeDescription}{mmmmmmmm}

\item[\dusa{MFILE}] \verb|character*12| name of the \moc{ASCII} Matlab \moc{m-file} data structure that
will contain the instructions for plotting the tracking lines.

\item[\dusa{TRKNAM}] \verb|character*12| name of the \dds{tracking} data structure that
will contain region volume and surface area vectors in addition to region
identification pointers and other tracking information.

\item[\dusa{TRKFIL}] \verb|character*12| name of the sequential binary tracking file 
used to store the tracks lengths.\cite{Marleau2001}  

\item[\dstr{desctlm}] structure describing the type of graphics generated (see \Sect{desctlm}).

\end{ListeDeDescription}

\subsubsection{Data input for module \moc{TLM:}}\label{sect:desctlm}

\begin{DataStructure}{Structure \dstr{desctlm}}
$[$ \moc{EDIT} \dusa{iprint} $]$ \\
$[$ \moc{MIXTURE} $]$ \\
$[$ \moc{NTPO} \dusa{nplots} $]$ \\
( $\{$ \\
\hspace{0.4cm}   \moc{POINTS} $[$ \moc{NoPause} $]$ $|$ \\
\hspace{0.4cm}   \moc{DIRECTIONS} $[$ \moc{NoPause} $]$ \moc{DIR} \dusa{idir} $[$ \moc{PLAN} \dusa{iplan} 
$\{$ \moc{U} \dusa{iuv} $|$ \moc{V} \dusa{iuv} $\}$ $]$ $|$ \\
\hspace{0.4cm}   \moc{PLANP} $[$ \moc{NoPause} $]$ \moc{DIR} \dusa{idir} \moc{DIST} \dusa{dist} $[$ \moc{PLAN} \dusa{iplan} $]$ $|$\\
\hspace{0.4cm}   \moc{PLANA} $[$ \moc{NoPause} $]$ \moc{A} \dusa{a} \moc{B} \dusa{b} $[$ \moc{C} \dusa{c} $]$ \moc{D} \dusa{d}\\
\hspace{0.4cm} $\}$ , \dusa{iplot}=$1$, \dusa{nplots} )
\end{DataStructure}

\noindent
 where

\begin{ListeDeDescription}{mmmmmmmm}   

\item[\moc{EDIT}] keyword used to modify the print level \dusa{iprint}.

\item[\dusa{iprint}] index used to control the printing in this module. It must be set to 0 if no printing on the output
file is required. 

\item[\moc{MIXTURE}] keyword to set drawing colors as a function of mixtures. By default, colors are set according to region indices.

\item[\moc{NTPO}] keyword to specify the number of figures to draw.

\item[\dusa{nplots}] integer value for the number of figures to draw.

\item[\moc{POINTS}] keyword to specify that the figure will illustrate the intersection points between the lines and the external faces of the geometry.

\item[\moc{DIRECTIONS}] keyword to specify that the figure will illustrate the lines crossing each region as well as the intersection points between the lines
and the external faces of the geometry.

\item[\moc{PLANP}] keyword to specify that the figure will illustrate the points crossing a plane normal to the line direction.

\item[\moc{PLANA}] keyword to specify that the figure will illustrate the points crossing an arbitrary surface in 3-D or line in 2-D. The equation for the
surface in 3-D is~:
$$
\textit{a} X + \textit{b} Y + \textit{c} Z =\textit{d} 
$$
while the equation for the line in 2-D is~:
$$
\textit{a} X + \textit{b} Y =\textit{d} 
$$

\item[\moc{NoPause}] keyword to specify that all the lines the lines must be drawn without Matlab pause. By default, there is a pause after all the points
associated with an external surface and all the lines associated with a region are drawn.

\item[\moc{DIR}] keyword to specify line direction to draw.

\item[\dusa{idir}] integer value to identify the track direction to draw. In the case where \dusa{idir}=0, all the directions will be drawn. A value of  
\dusa{idir}=0 for 2-D geometry is generally acceptable. However, for 3-D geometry the number of lines generated is such that the figure becomes a mess and it
is generally more convenient to draw the lines direction per direction.

\item[\moc{PLAN}] keyword to specify which of the three planes normal to the specified direction in 3-D will be considered for drawing. This plane is defined
by the axes $U-V$. Used only for 3-D geometries.

\item[\dusa{iplan} ] integer value to identify which of the three planes normal to the specified direction in 3-D will be considered for drawing. the only
values permitted are 0, 1, 2 or 3. When a value of 0 is specified (default) all three planes will be drawn. Used only for 3-D geometries.

\item[\moc{U}] keyword to specify that the all the lines in the $V$ axis associated with a position on the $U$ axis will be drawn. Used only for 3-D geometries.

\item[\moc{V}] keyword to specify that the all the lines in the $U$ axis associated with a position on the $V$ axis will be drawn. Used only for 3-D geometries.

\item[\dusa{iuv}] integer value to identify the position on the $U$ or $V$ axis to be drawn. Used only for 3-D geometries.

\item[\moc{DIST}] keyword to specify the distance between the plane normal to the line direction and the origin.

\item[\dusa{dist} ] real or double precision value for the distance of the plane from the origin.

\item[\moc{A}] keyword to specify the value of \dusa{a} for an arbitrary plane or line.

\item[\dusa{a} ] real or double precision value \dusa{a}.

\item[\moc{B}] keyword to specify the value of \dusa{b} for an arbitrary plane or line.

\item[\dusa{b} ] real or double precision value \dusa{b}.

\item[\moc{C}] keyword to specify the value of \dusa{c} for an arbitrary plane.

\item[\dusa{b} ] real or double precision value \dusa{c}.

\item[\moc{D}] keyword to specify the value of \dusa{d} for an arbitrary plane or line.

\item[\dusa{d} ] real or double precision value \dusa{d}.
\end{ListeDeDescription}

\eject
 % structure (dragonTLM)
\subsection{The {\tt M2T:} module}\label{sect:M2TData}

This component of the lattice code is dedicated to the generation of an {\sc ascii} file
with the Apotrim specification using {\sc macrolib} data. Such a file is useful to transfer multigroup
and macroscopic cross-section data toward a Moret calculation.

\vskip 0.02cm

The calling specifications are:

\begin{DataStructure}{Structure \dstr{M2T:}}
\dusa{APTRIM}~\moc{:=}~\moc{M2T:}~$[$~\dusa{APTRIM}~$]$~\dusa{MLIB}~\moc{::}~\dstr{M2T\_data} \\
\end{DataStructure}

\noindent where
\begin{ListeDeDescription}{mmmmmmm}

\item[\dusa{APTRIM}] {\tt character*12} name of an {\sc ascii} file with the Apotrim specification. If \dusa{APTRIM} appears on the RHS, new information is appended to the existing Apotrim file.

\item[\dusa{MLIB}] {\tt character*12} name of a {\sc macrolib} (type {\tt L\_MACROLIB}) object.

\item[\dusa{M2T\_data}] input data structure containing specific data (see \Sect{descM2T}).

\end{ListeDeDescription}

\subsubsection{Data input for module {\tt M2T:}}\label{sect:descM2T}

\vskip -0.5cm

\begin{DataStructure}{Structure \dstr{M2T\_data}}
$[$~\moc{EDIT} \dusa{iprint}~$]$ \\
$[$~\moc{PN} \dusa{nl}~$]~[$~\moc{TRAN}~$]~[$~\moc{NOMA}~$]$ \\
$[[$~\moc{MIX} \dusa{hmix}~$[$~\moc{FROM}~\dusa{imixold}~$]~[$~\moc{BURN} \dusa{bup}~$]~[$~\moc{TEMP} \dusa{tval}~$]$~\moc{ENDMIX}~$]]$\\
{\tt ;}
\end{DataStructure}

\noindent where
\begin{ListeDeDescription}{mmmmmmmm}

\item[\moc{EDIT}] keyword used to set \dusa{iprint}.

\item[\dusa{iprint}] index used to control the printing in module {\tt
M2T:}. =0 for no print; =1 for minimum printing (default value).

\item[\moc{PN}] keyword used to set the Legendre order of the scattering transfers written on the Apotrim file.

\item[\dusa{nl}] Legendre order. By default, \dusa{nl} $=0$ corresponding to an isotropic collision in LAB.

\item[\moc{TRAN}] keyword used to set a transport correction on cross sections written on the Apotrim file.

\item[\moc{NOMA}] keyword used to avoid writing the energy mesh on the Apotrim file. This
option is useful to catenate additional mixture information on an existing Apotrim file. By
default, the energy mesh is written on the Apotrim file.

\item[\moc{MIX}] keyword used to set \dusa{hmix}.

\item[\dusa{hmix}] {\tt character*20} name of the mixture to be written on the Apotrim file.

\item[\moc{BURN}] keyword used to set the burnup of a mixture.

\item[\dusa{bup}] burnup of a mixture. By default, \dusa{bup} $=0.0$.

\item[\moc{TEMP}] keyword used to set the temperature of a mixture.

\item[\dusa{tval}] temperature of a mixture in Celsius. By default, \dusa{tval} $=0.0 \ ^\circ{\rm C}$.

\item[\moc{FROM}] keyword used to set the index of the mixture in the {\sc macrolib} object.

\item[\dusa{imixold}] index of the mixture that is recovered in the {\sc macrolib} object. By default, \dusa{imixold}$=1$.

\item[\moc{ENDMIX}] end of specification keyword for the material mixture.

\end{ListeDeDescription}

Here is an example of the creation of an Apotrim file named {\tt APOTR} with a Hansen-Roach energy mesh created
from a XMAS 172-group flux calculation. The Apotrim file is created from three
LCM objects {\tt FLUX}, {\tt LIBRARY2} and {\tt TRACK} containing the flux, the XMAS-formatted microlib
and the tracking.

\begin{verbatim}
LINKED_LIST TRACK LIBRARY2 FLUX MAC2 EDIT ;
SEQ_ASCII APOTR ;
...
EDIT := EDI: LIBRARY2 TRACK FLUX :: EDIT 3
*     Hansen-Roach energy mesh follows
      COND 12 17 21 27 33 42 50 60 66 76 84 95 123 140 155 172
      MERGE MIX 1 1 1 1 1 1 2 3 3
      SAVE ON 'EDITCDAT   1' ;
MAC2 := EDIT :: STEP UP 'EDITCDAT   1' STEP UP 'MACROLIB' ;
APOTR := M2T: MAC2 :: EDIT 3 TRAN MIX FUEL FROM 1 ENDMIX
                                  MIX CLAD FROM 2 ENDMIX
                                  MIX COOLANT FROM 3 ENDMIX ;
\end{verbatim}
\eject
 % structure (dragonM2T)
\subsection{The {\tt CHAB:} module}\label{sect:CHABData}

This component of the lattice code is dedicated to the modification of cross section
information in a {\sc microlib}.

\vskip 0.02cm

The calling specifications are:

\begin{DataStructure}{Structure \dstr{CHAB:}}
$\{$~\dusa{MICRO1}~$|$~\dusa{DRAGLIB1}~$\}$~\moc{:=}~\moc{CHAB:}~$\{$~\dusa{MICRO1}~$|$~\dusa{MICRO2}~$|$~\dusa{DRAGLIB2}~$\}$~\moc{::}~\dstr{CHAB\_data} \\
\end{DataStructure}

\noindent where
\begin{ListeDeDescription}{mmmmmmm}

\item[\dusa{MICRO1}] {\tt character*12} name of a {\sc microlib} (type {\tt L\_LIBRARY}) object that is created or modified by {\tt CHAB:}.

\item[\dusa{DRAGLIB1}] {\tt character*12} name of a {\sc draglib} (type {\tt L\_DRAGLIB}) object that is created by {\tt CHAB:}.

\item[\dusa{MICRO2}] {\tt character*12} name of a {\sc microlib} (type {\tt L\_LIBRARY}) object open in read-only mode.

\item[\dusa{DRAGLIB2}] {\tt character*12} name of a {\sc draglib} (type {\tt L\_DRAGLIB}) object open in read-only mode.

\item[\dusa{CHAB\_data}] input data structure containing specific data (see \Sect{descCHAB}).

\end{ListeDeDescription}

\subsubsection{Data input for module {\tt CHAB:}}\label{sect:descCHAB}

\vskip -0.5cm

\begin{DataStructure}{Structure \dstr{CHAB\_data}}
$[$~\moc{EDIT} \dusa{iprint}~$]$ \\
$[[$~\moc{MODI} \dusa{TYPSEC} \dusa{igm} \moc{TO} \dusa{igp} $\{$
\moc{VALE}~$[[$~\dusa{val}~$]]~|$~\moc{CONS}~\dusa{value}~$|$~\moc{PLUS}~\dusa{value}~$|$~\moc{MULT}~\dusa{value} $\}$ \dusa{HISOT}~$]]$
{\tt ;}
\end{DataStructure}

\noindent where
\begin{ListeDeDescription}{mmmmmmmm}

\item[\moc{EDIT}] keyword used to set \dusa{iprint}.

\item[\dusa{iprint}] index used to control the printing in module {\tt CHAB:}. =0 for no print; =1 for minimum printing (default value).

\item[\moc{MODI}] keyword used to define a modification of a nuclear reaction belonging to a given isotope.

\item[\dusa{TYPSEC}] {\tt character*8} name of an existing nuclear reaction chosen among the following values:
\begin{description}
\item[{\tt 'NTOT0'}] Total cross section.
\item[{\tt 'NG'}] Radiative capture cross section. The total ({\tt 'NTOT0'}) cross section is modified accordingly.
\item[{\tt 'NA'}] $(n,\alpha)$ cross section. The total ({\tt 'NTOT0'}) cross section is modified accordingly.
\item[{\tt 'NP'}] $(n,p)$ cross section. The total ({\tt 'NTOT0'}) cross section is modified accordingly.
\item[{\tt 'ND'}] $(n,d)$ cross section. The total ({\tt 'NTOT0'}) cross section is modified accordingly.
\item[{\tt 'NT'}] $(n,t)$ cross section. The total ({\tt 'NTOT0'}) cross section is modified accordingly.
\item[{\tt 'CAPT'}] Capture cross sections. Each present reaction of capture (\textbf{NG}, \textbf{NA}, \textbf{NP}, \textbf{ND}, \textbf{NT}) are taken into account. The total ({\tt 'NTOT0'}) cross section is modified accordingly. Only the keyword \textbf{MULT}, indicating a multiplication of the all cross sections, is available.
\item[{\tt 'NELAS'}] Elastic scattering cross section. The scattering ({\tt 'SIGS00'} and {\tt 'SCAT00'}) and total
({\tt 'NTOT0'}) cross sections are modified accordingly.
\item[{\tt 'NINEL'}] Inelastic scattering cross section. The scattering ({\tt 'SIGS00'} and {\tt 'SCAT00'}) and total
({\tt 'NTOT0'}) cross sections are modified accordingly.
\item[{\tt 'N2N'}] ($n$,$2n$) cross section. The scattering ({\tt 'SIGS00'} and {\tt 'SCAT00'}) and total
({\tt 'NTOT0'}) cross sections are modified accordingly.
\item[{\tt 'N3N'}] ($n$,$3n$) cross section. The scattering ({\tt 'SIGS00'} and {\tt 'SCAT00'}) and total
({\tt 'NTOT0'}) cross sections are modified accordingly.
\item[{\tt 'N4N'}] ($n$,$4n$) cross section. The scattering ({\tt 'SIGS00'} and {\tt 'SCAT00'}) and total
({\tt 'NTOT0'}) cross sections are modified accordingly.
\item[{\tt 'SIGS00'}, {\tt 'SIGS01'}, etc.] Scattering cross section. The total ({\tt 'NTOT0'}) cross section is modified accordingly.
\item[{\tt 'SCAT00'}, {\tt 'SCAT01'}, etc.] Differential scattering cross section. The total ({\tt 'NTOT0'}) cross section is modified accordingly.
\item[{\tt 'NUSIGF'}] $\nu$ times the fission cross section. The fission ({\tt 'NFTOT'}) and total ({\tt 'NTOT0'}) cross sections are modified accordingly.
\item[{\tt 'NFTOT'}] Fission cross section. The $\nu$ times fission ({\tt 'NUSIGF'}) and total ({\tt 'NTOT0'}) cross sections are modified accordingly.
\item[{\tt 'NU'}] Number of neutrons emitted per fission.The $\nu$ times fission ({\tt 'NUSIGF'}) cross section is modified accordingly.
\item[{\tt 'CHI'}] Fission spectrum. The resulting spectrum is normalized.
\end{description}

\item[\dusa{igm}] lower energy group index of the energy domain where the modification is taking place.

\item[\dusa{igp}] upper energy group index of the energy domain where the modification is taking place.

\item[\moc{VALE}] keyword indicating a replacement of all values in the above energy domain by different values.

\item[\dusa{val}] group--dependent real variable used as replacement value. We expect \dusa{igp}$-$\dusa{igm}$+$1 components.

\item[\moc{CONS}] keyword indicating a replacement of all values in the above energy domain by \dusa{value}.

\item[\moc{PLUS}] keyword indicating that \dusa{value} is added to all values in the above energy domain.

\item[\moc{MULT}] keyword indicating a multiplication of all values in the above energy domain by \dusa{value}.

\item[\dusa{value}] real variable used to modify the nuclear reaction.

\item[\dusa{HISOT}] {\tt character*8} or {\tt character*12} name of the isotope to modify. If \dusa{HISOT} is a {\tt character*8} value,
all {\tt character*12} isotope names prefixed by \dusa{HISOT} are modified.

\end{ListeDeDescription}

\eject
 % structure (dragonCHAB)
\subsection{The \moc{CPO:} module}\label{sect:CPOData}

The \moc{CPO:} module is used to generate the reactor cross-section database in Version3 format to be used in a full core
calculation using DONJON. This type of database is only parametrized in burnup
(or irradiation). The calling specifications are:

\begin{DataStructure}{Structure \dstr{CPO:}}
\dusa{CPONAM} \moc{:=} \moc{CPO:}  $[$ \dusa{CPONAM} $]$ \dusa{EDINAM} 
$[$ \dusa{BRNNAM} $]$ \moc{::} \dstr{desccpo}
\end{DataStructure}

\noindent
 where

\begin{ListeDeDescription}{mmmmmmmm}

\item[\dusa{CPONAM}] \verb|character*12| name of the \dds{cpo} data structure containing the reactor
database. Additional contributions can be included in the reactor cross-section database if \dusa{CPONAM}
appears on the RHS.

\item[\dusa{EDINAM}] \verb|character*12| name of the read-only \dds{edition} data structure.

\item[\dusa{BRNNAM}] \verb|character*12| name of the read-only \dds{burnup} data structure containing the
depletion history. This information is given only if the reactor database is to contain burnup dependent data.

\item[\dstr{desccpo}] structure containing the input data to this module (see \Sect{desccpo}).
\end{ListeDeDescription}


\subsubsection{Data input for module \moc{CPO:}}\label{sect:desccpo}

\begin{DataStructure}{Structure \dstr{desccpo}}
$[$ \moc{EDIT} \dusa{iprint} $]$ \\
$[$ \moc{B2} $]~~[$ \moc{NOTR} $]$ \\
$\{$ \moc{STEP} \dusa{NOMDIR} $|$ \moc{BURNUP} \dusa{PREFIX} $\}$ \\
$[$ $[$ \moc{EXTRACT}  $\{$ \moc{ALL} $|$ \dusa{NEWNAME} (\dusa{OLDNAME}($i$), $i$=1,niext) $\}$ $]$ $]$ \\ 
$[$ \moc{NAME} \dusa{NDIR} $]$ \\
$[~\{$ \moc{GLOB} $|$ \moc{LOCA} $\}~]$
\end{DataStructure}

\noindent
 where

\begin{ListeDeDescription}{mmmmmmmm}
\item[\moc{EDIT}] keyword used to modify the print level \dusa{iprint}.

\item[\dusa{iprint}] index used to control the printing of this module. The amount of output produced by this
tracking module will vary substantially depending on the print level specified.

\item[\moc{B2}] keyword to specify that the buckling correction ($dB^{2}$) is to be applied to the cross
section to be stored on the reactor database. By default (fixed default), such a correction is not taken into
account.

\item[\moc{NOTR}] keyword to specify that the cross section to be stored on the reactor database are not to
be transport corrected. By default (fixed default), transport corrected cross section are considered when
the \moc{CTRA} option is activated in \moc{MAC:} or \moc{LIB:} (see \Sectand{MACData}{LIBData}).

\item[\moc{STEP}] keyword to specify that a specific cross section directory stored in \dusa{EDINAM} via the
\moc{SAVE} option in the \moc{EDI:} module is to be transferred to \dusa{CPONAM}.

\item[\dusa{NOMDIR}] \verb|character*12| name of the specific cross section directory to be treated.

\item[\moc{BURNUP}] keyword to specify that a chain of cross section directory stored in \dusa{EDINAM} via
the \moc{SAVE} option in the \moc{EDI:} module will be transferred to \dusa{CPONAM}.

\item[\dusa{PREFIX}] \texttt{character*8} prefix name of the cross section directory to be treated. DRAGON
will transfer into the reactor database all the directories with full name \verb|NAMDIR| created using

\begin{quote}
\verb|WRITE(NAMDIR,'(A8,I4)')| \textit{PREFIX},\verb|nb|
\end{quote}
 where \verb|nb| is an integer greater than 0 indicating the depletion step index. 

\item[\moc{EXTRACT}] keyword to specify that the contribution of some isotopes to the macroscopic cross
sections associated with each homogenized mixture should be extracted before being stored on the reactor
database. The microscopic cross sections and concentrations associated with these isotopes should also be
generated and stored on the reactor database.  

\item[\moc{ALL}] keyword to specify that all the isotopes processed using the \moc{MICR} option of the
\moc{EDI:} module should be extracted from the macroscopic cross sections associated with each homogenized
mixture.

\item[\dusa{NEWNAME}] \verb|character*12| name under which a given set of extracted isotope will be stored
on the reactor database.

\item[\dusa{OLDNAME}] array of \verb|character*8| name of isotopes to be extracted from the macroscopic
cross section associated with each homogenized mixture.

\item[\moc{NAME}] keyword to specify the prefix for the name of the sub-directory where the information
corresponding to a single homogenized region will be stored. The fixed default is
\dusa{NDIR}=\verb*|'COMPO~~~'|.

\item[\dusa{NDIR}] \verb|character*8| prefix for the name of the sub-directory. The complete name is
constructed by the concatenation of \dusa{NDIR} with a four digit integer value. 

\item[\moc{GLOB}] keyword to specify that global parameters are used to index the database (default option). A global parameter is
defined over the complete calculation domain.

\item[\moc{LOCA}] keyword to specify that local parameters are used to index the database. A local parameter is
defined over each homogenization mixture.

\end{ListeDeDescription}

\eject

 % structure (dragonCPO)
\subsection{The {\tt SAP:} module}\label{sect:SAPHYBData}

This component of the lattice code is dedicated to the constitution of the
reactor database intended to store {\sl all} the nuclear data, produced in
the lattice code, that is useful
in reactor calculations including fuel management and space-time kinetics.
Multigroup lattice calculations are too expensive to be executed dynamically
from the driver of the global reactor calculation. A more feasible
approach is to create a reactor database where a finite number of lattice
calculation results are tabulated against selected {\sl global parameters}
chosen so as to represent expected operating conditions of the reactor.

\vskip 0.1cm

The \moc{SAP:} operator is used to create and construct a {\sc saphyb} object.
This object is generally {\sl persistent} and used to collect information gathered
from many DRAGON {\sl elementary calculations} performed under various conditions.
The {\sc saphyb} object is based on a specification of the Saphyr code system.\cite{Apollo2}

\vskip 0.1cm

Each elementary calculation is characterized by a tuple of {\sl global parameters}.
These global parameters are of different types, depending on the nature of the
study under consideration: type of assembly, power, temperature in a mixture,
concentration of an isotope, time, burnup or exposure rate in a depletion calculation,
etc. Each step of a depletion calculation represents an elementary calculation.
The {\sc saphyb} object is often presented as a {\sl multi-parameter reactor database}.

\vskip 0.1cm

For each elementary calculation, the results are recovered from the output of the
\moc{EDI:} operator and stored in a set of {\sl homogenized mixture}
directories. The \moc{EDI:} operator is responsible for performing condensation
in energy and homogenization in space of the macroscopic and microscopic cross
sections. All the elementary calculations gathered in a single {\sc saphyb} object are
characterized by a single output geometry and a unique output energy-group
structure.

\vskip 0.1cm

The {\sc saphyb} object contains table-of-content information apart from a set of specific
{\sl elementary calculation} directories. These directories are themself subdivided
into {\sl homogenized mixture} directories. The localization of an elementary calculation
is done using a tuple of global parameters. The elementary calculation indices are
stored in a tree with the number of levels equal to the number of global parameters.
An example of a tree with three global parameters is shown in \Fig{tree}. Each node
of this tree is associated with the index of the corresponding global parameter and with the
reference to the daughter nodes if they exist. The number of leaves is equal to the number
of nodes for the last (third) parameter and is equal to the number of elementary
calculations stored in the {\sc saphyb} object. The index of each elementary calculation is
therefore an attribute of each leaf.

\begin{figure}[h!]  
\begin{center} 
\epsfxsize=12cm
\centerline{ \epsffile{tree.eps}}
\parbox{14cm}{\caption{Global parameter tree in a {\sc saphyb} object}\label{fig:tree}}   
\end{center}  
\end{figure}

\vskip 0.1cm

In each homogenized mixture directory, the \moc{SAP:} operator recover
cross sections for a number of {\sl particularized isotopes} and {\sl macroscopic
sets}, a collection of isotopic cross sections weighted by isotopic number densities.
Cross sections for particularized isotopes and macroscopic sets are recovered for
{\sl selected reactions}. Other information is also recovered: multigroup neutron
fluxes, isotopic number densities, fission spectrum and a set
of {\sl local variables}. The local variables are values that characterize each
homogenized mixture: local power, burnup, exposure rate, etc. Some local variables
are arrays of values (eg: SPH equivalence factors). Finally, note that cross section
information written on the {\sc saphyb} is {\sl not} transport corrected and {\sl not}
SPH corrected.

\vskip 0.1cm

A different specification of the \moc{SAP:} function call is used for
creation and construction of the {\sc saphyb} object.
\begin{itemize}
\item The first specification is used to initialize the {\sc saphyb} data structure
as a function of the \dds{microlib} used in the reference calculation. Optionnally,
the homogenized geometry is also provided. The initialization call is also used to
set the choice of global parameters, local variables, particularized isotopes,
macroscopic sets and selected reactions.
\item A modification call to the \moc{SAP:} function is performed after each
elementary calculation in order to recover output information processed by \moc{EDI:}
(condensed and homogenized cross sections) and \moc{EVO:} (burnup dependant values).
Global parameters and local variables can optionnally be recovered from \dds{microlib}
objects. The \moc{EDI:} calculation is generally performed with option {\tt MICR ALL}.
\end{itemize}

The calling specifications are:

\vskip -0.5cm

\begin{DataStructure}{Structure \dstr{SAP:}}
$\{$~~\dusa{SAPNAM} \moc{:=} \moc{SAP:} $[$ \dusa{SAPNAM} $]~[$~\dusa{HMIC} $]$ \moc{::} \dstr{saphyb\_data1} \\
~~~$|$~~~\dusa{SAPNAM} \moc{:=} \moc{SAP:} \dusa{SAPNAM}~\dusa{EDINAM}~$[$ \dusa{BRNNAM} $]~[$ \dusa{HMIC1}~$[$~\dusa{HMIC2} $]~]~[$ \dusa{FLUNAM} $]$\\
~~~~~~~~~~ \moc{::} \dstr{saphyb\_data2} \\
~~~$|$~~~\dusa{SAPNAM} \moc{:=} \moc{SAP:} \dusa{SAPNAM} $[[$ \dusa{SAPRHS} $]]$ \moc{::} \dstr{saphyb\_data3} $\}$ \\
\end{DataStructure}

\noindent where
\begin{ListeDeDescription}{mmmmmmm}

\item[\dusa{SAPNAM}] {\tt character*12} name of the {\sc lcm} object containing the
{\sl master} {\sc saphyb} data structure.

\item[\dusa{HMIC}] {\tt character*12} name of the reference \dds{microlib} (type {\tt
L\_LIBRARY}) containing the microscopic cross sections.

\item[\dusa{EDINAM}] {\tt character*12} name of the {\sc lcm} object (type {\tt
L\_EDIT}) containing the {\sc edition} data structure corresponding to an elementary
calculation. The {\sc edition} data produced by the last call to the {\tt EDI:} module
is used.

\item[\dusa{BRNNAM}] {\tt character*12} name of the {\sc lcm} object (type {\tt
L\_BURNUP}) containing the {\sc burnup} data structure. This object is compulsory if one
of the following parameters is used: \moc{IRRA}, \moc{FLUB} and/or \moc{TIME}.

\item[\dusa{HMIC1}] {\tt character*12} name of a \dds{microlib} (type {\tt
L\_LIBRARY}) containing global parameter information.

\item[\dusa{HMIC2}] {\tt character*12} name of a \dds{microlib} (type {\tt
L\_LIBRARY}) containing global parameter information.

\item[\dusa{FLUNAM}] {\tt character*12} name of the reference \dds{flux} (type {\tt
L\_FLUX}). By default, the reference flux is not recovered and not written on the {\sc saphyb}.

\item[\dusa{SAPRHS}] {\tt character*12} name of the {\sl read-only} {\sc saphyb} data structure. This
data structure is concatenated to \dusa{SAPNAM} using the \dusa{saphyb\_data3} data structure,
as presented in \Sect{descsap3}. \dusa{SAPRHS} must be defined with the same number of energy
groups and the same number of homogeneous regions as \dusa{SAPNAM}. Moreover, all the
global and local parameters of \dusa{SAPRHS} must be defined in \dusa{SAPNAM}. \dusa{SAPNAM}
may be defined with {\sl global} parameters not defined in \dusa{SAPRHS}.

\item[\dusa{saphyb\_data1}] input data structure containing initialization information (see \Sect{descsap1}).

\item[\dusa{saphyb\_data2}] input data structure containing information related to the recovery of an
elementary calculation (see \Sect{descsap2}).

\item[\dusa{saphyb\_data3}] input data structure containing information related to the catenation of one or many
{\sl read-only} {\sc saphyb} (see \Sect{descsap3}).

\end{ListeDeDescription}

\newpage

\subsubsection{Initialization data input for module {\tt SAP:}}\label{sect:descsap1}

\begin{DataStructure}{Structure \dstr{saphyb\_data1}}
$[$~\moc{EDIT} \dusa{iprint}~$]$ \\
$[$~\moc{NOML}~\dusa{nomlib}~$]$ \\
$[$~\moc{COMM}~$[[$~\dusa{comment}~$]]$~\moc{ENDC}~$]$ \\
$[[$~\moc{PARA}~\dusa{parnam}~\dusa{parkey} \\
~~~\{~\moc{TEMP}~\dusa{micnam}~\dusa{imix}~$|$~\moc{CONC}~\dusa{isonam1}~\dusa{micnam}~\dusa{imix}~$|$~\moc{IRRA}~$|$~\moc{FLUB}~$|$ \\
~~~~~~\moc{PUIS}~$|$~\moc{MASL}~$|$~\moc{FLUX}~$|$~\moc{TIME}~$|$~\moc{VALE}~\{~\moc{FLOT}~$|$~\moc{CHAI}~$|$~\moc{ENTI}~\}~\} \\
$]]$ \\
$[[$~\moc{LOCA}~\dusa{parnam}~\dusa{parkey} \\
~~~\{~\moc{TEMP}~$|$~\moc{CONC}~\dusa{isonam2}~$|$~\moc{IRRA}~$|$~\moc{FLUB}~$|$~~\moc{FLUG}~$|$~\moc{PUIS}~$|$~\moc{MASL}~$|$~\moc{FLUX}~$|$~\moc{EQUI}~\} \\
$]]$ \\
$[$~\moc{ISOT}~\{~\moc{TOUT}~$|$ \moc{MILI}~\dusa{imil}~$|~[$~\moc{FISS}~$]~[$~\moc{PF}~$]~[$~(\dusa{HNAISO}(i),~i=1,$N_{\rm iso}$) $]$~\}~$]$ \\
$[[$~\moc{MACR}~\dusa{HNAMAC}~\{~\moc{TOUT}~$|$~\moc{REST}~\}~$]]$ \\
$[$~\moc{REAC}~(\dusa{HNAREA}(i),~i=1,$N_{\rm reac}$) $]$ \\
$[$ \moc{NAME} (\dusa{HNAMIX}(i),~i=1,$N_m$) $]$ \\
{\tt ;}
\end{DataStructure}

\goodbreak
\noindent where
\begin{ListeDeDescription}{mmmmmmmm}

\item[\moc{EDIT}] key word used to set \dusa{iprint}.

\item[\dusa{iprint}] index used to control the printing in module {\tt
SAP:}. =0 for no print; =1 for minimum printing (default value).

\item[\moc{NOML}] key word used to input a user--defined name for the {\sc saphyb}. This information is mandatory
if the Saphyb is to be read by the Lisaph module of Cronos.

\item[\dusa{nomlib}] {\tt character*80} user-defined name.

\item[\moc{COMM}] key word used to input a general comment for the {\sc saphyb}.

\item[\dusa{comment}] {\tt character*80} user-defined comment.

\item[\moc{ENDC}] end--of--comment key word.

\item[\moc{PARA}] keyword used to define a single global parameter.

\item[\moc{LOCA}] keyword used to define a single local variable (a local variable
may be a single value or an array of values).

\item[\dusa{parnam}] {\tt character*80} user-defined name of a global parameter or
local variable.

\item[\dusa{parkey}] {\tt character*4} user-defined keyword associated to a global
parameter or local variable.

\item[\dusa{micnam}] {\tt character*12} name of the \dds{microlib} (type {\tt
L\_LIBRARY}) associated to a global parameter. The corresponding \dds{microlib} will be required on
RHS of the \moc{SAP:} call described in Sect.~\ref{sect:descsap2}.

\item[\dusa{imix}] index of the mixture associated to a global parameter. This mixture is
located in \dds{microlib} named \dusa{micnam}.

\item[\dusa{isonam1}] {\tt character*8} alias name of the isotope associated to a global
parameter. This isotope is located in \dds{microlib} data structure named \dusa{micnam}.

\item[\dusa{isonam2}] {\tt character*8} alias name of the isotope associated to a local
variable. This isotope is located in the \dds{microlib} directory of the {\sc edition}
data structure named \dusa{EDINAM}.

\item[\moc{TEMP}] keyword used to define a temperature (in $^{\rm o}$C) as global parameter or
local variable.

\item[\moc{CONC}] keyword used to define a number density as global parameter or
local variable.

\item[\moc{IRRA}] keyword used to define a burnup (in MWday/Tonne) as global
parameter or local variable.

\item[\moc{FLUB}] keyword used to define a {\sl fuel-only} exposure rate (in n/kb) as global
parameter or local variable. The exposure rate is recovered from the \dusa{BRNNAM}
LCM object.

\item[\moc{FLUG}] keyword used to define an exposure rate in global homogenized mixtures (in n/kb) as
local variable. The exposure rate is recovered from the \dusa{BRNNAM}
LCM object.

\item[\moc{PUIS}] keyword used to define the power as global parameter or
local variable.

\item[\moc{MASL}] keyword used to define the mass density of heavy isotopes as
global parameter or local variable.

\item[\moc{FLUX}] keyword used to define the volume-averaged, energy-integrated flux as
global parameter or local variable.

\item[\moc{TIME}] keyword used to define the time (in seconds) as global parameter.

\item[\moc{EQUI}] keyword used to define the SPH equivalence factors as
local variable. A set of SPH factors can be defined as local
variables. Note that the cross sections and fluxes stored in the {\sc saphyb} are
{\sl not} SPH corrected.

\item[\moc{VALE}] keyword used to define a user-defined quantity as global parameter.
This keyword must be followed by the type of parameter.

\item[\moc{FLOT}] keyword used to indicate that the user-defined global parameter
is a floating point value.

\item[\moc{CHAI}] keyword used to indicate that the user-defined global parameter
is a {\tt character*12} value.

\item[\moc{ENTI}] keyword used to indicate that the user-defined global parameter
is an integer value.

\item[\moc{ISOT}] keyword used to select the set of particularized isotopes.

\item[\moc{TOUT}] keyword used to select all the available isotopes in the reference
\dds{microlib} named \dusa{HMIC} as particularized isotopes.

\item[\moc{MILI}] keyword used to select the isotopes in the reference
\dds{microlib} named \dusa{HMIC} from a specific mixture as particularized isotopes.

\item[\dusa{imil}] index of the mixture where the particularized isotopes are recovered.

\item[\moc{FISS}] keyword used to select all the available fissile isotopes in the reference
\dds{microlib} named \dusa{HMIC} as particularized isotopes.

\item[\moc{PF}] keyword used to select all the available fission products in the reference
\dds{microlib} named \dusa{HMIC} as particularized isotopes.

\item[\dusa{HNAISO}(i)] {\tt character*12} user-defined isotope name. $N_{\rm iso}$ is the
total number of explicitely--selected particularized isotopes.

\item[\moc{MACR}] keyword used to select a type of macroscopic set. A maximum of two macroscopic sets is allowed.

\item[\dusa{HNAMAC}] {\tt character*8} user-defined name of the macroscopic set.

\item[\moc{TOUT}] keyword used to select all the available isotopes in the macroscopic set.

\item[\moc{REST}] keyword used to remove all the particularized isotope contributions
from the macroscopic set.

\item[\moc{REAC}] keyword used to select the set of nuclear reactions.

\item[\dusa{HNAREA}(i)] {\tt character*4} name of a user-selected reaction. $N_{\rm reac}$
is the total number of selected reactions. \dusa{HNAREA}(i)
is chosen among the following values:

\begin{tabular}{p{1.0cm} p{16cm}|}
\moc{TOTA} & Total cross sections \\
\moc{TOP1} & Total $P_1$-weighted cross sections \\
\moc{ABSO} & Absorption cross sections \\
\moc{SNNN} & Excess cross section due to (n,$x$n) reactions \\
\moc{FISS} & Fission cross section \\
\moc{CHI}  & Steady-state fission spectrum \\
\moc{NUFI} & $\nu\Sigma_{\rm f}$ cross sections \\
\moc{ENER} & Energy production cross section, taking into account all energy production reactions \\
\moc{EFIS} & Energy production cross section for (n,f) reaction only \\
\moc{EGAM} & Energy production cross section for (n,$\gamma$) reaction only \\
\moc{FUIT} & $B^2$ times the leakage coefficient \\
\moc{SELF} & within-group $P_0$ scattering cross section \\
\moc{DIFF} & scattering cross section for each available Legendre order. These cross sections
are \\
& {\sl not} multiply by the $2\ell+1$ factor.\\
\moc{PROF} & profile of the transfer cross section matrices (i.e. position of the non--zero element in \\
& the transfer cross section matrices) \\
\moc{TRAN} & transfer cross section matrices for each available Legendre order. These cross sections \\
& are multiply by the $2\ell+1$ factor.\\
\moc{CORR} & transport correction. Note that the cross sections stored in the {\sc saphyb} are {\sl not} \\
& transport corrected.\\
\moc{STRD} & STRD cross sections used to compute the diffusion coefficients \\
\moc{NP}   & (n,p) production cross sections \\
\moc{NT}   & (n,t) production cross sections \\
\moc{NA}   & (n,$\alpha$) production cross sections \\
\end{tabular}

\item[\moc{NAME}] key word used to define mixture names. By default, mixtures
names are of the form \dusa{HNAMIX}(i), where
\begin{verbatim}
WRITE(HNAMIX(I),'(3HMIX,I5.5)') I
\end{verbatim}

\item[\dusa{HNAMIX}(i)] Character*20 user-defined mixture name. $N_m$ is the number of mixtures. 

\end{ListeDeDescription}

\subsubsection{Modification data input for module {\tt SAP:}}\label{sect:descsap2}

\begin{DataStructure}{Structure \dstr{saphyb\_data2}}
$[$ \moc{EDIT} \dusa{iprint} $]$ \\
$[$ \moc{CRON} $]$ \\
$[[$ \dusa{parkey} \dusa{value} $]]$ \\
$[$ \moc{ORIG} \dusa{orig} $]$ \\
$[$ \moc{SET} \dusa{xtr} $\{$ \moc{S} $|$ \moc{DAY} $|$ \moc{YEAR} $\}$ $]$ \\
{\tt ;}
\end{DataStructure}

\goodbreak
\noindent where
\begin{ListeDeDescription}{mmmmmmmm}

\item[\moc{EDIT}] key word used to set \dusa{iprint}.

\item[\dusa{iprint}] index used to control the printing in module {\tt
SAP:}. =0 for no print; =1 for minimum printing (default value).

\item[\moc{CRON}] key word used to force the kinetics data to be placed into the {\tt divers} directory. By default,
the kinetics data is placed in the {\tt cinetique} directory of each mixture subdirectory. The \moc{CRON} option can
only be used if the Saphyb contains a unique mixture. This option is mandatory if the Saphyb is to be read by the Lisaph
module of Cronos.

\item[\dusa{parkey}] {\tt character*4} keyword associated to a user-defined global
parameter.

\item[\dusa{value}] floating-point, integer or {\tt character*12} value of a user-defined
global parameter.

\item[\moc{ORIG}] key word used to define the father node in the global parameter tree. By
default, the index of the previous elementary calculation is used.

\item[\dusa{orig}] index of the elementary calculation associated to the father node in the
global parameter tree.

\item[\moc{SET}] keyword used to recover the flux normalization factor already
stored on \dusa{BRNNAM} from a sub-directory corresponding to a specific time.

\item[\dusa{xtr}] time associated with the current flux calculation. The
name of the sub-directory where this information is stored will be given by
`{\tt DEPL-DAT}'//{\tt CNN} where {\tt CNN} is a  {\tt character*4} variable
defined by  {\tt WRITE(CNN,'(I4)') INN} where {\tt INN} is an index associated
with the time \dusa{xtr}.

\item[\moc{S}] keyword to specify that the time is given in seconds.

\item[\moc{DAY}] keyword to specify that the time is given in days.

\item[\moc{YEAR}] keyword to specify that the time is given in years.

\end{ListeDeDescription}

\subsubsection{Modification (catenate) data input for module {\tt SAP:}}\label{sect:descsap3}

\vskip -0.5cm

\begin{DataStructure}{Structure \dstr{saphyb\_data3}}
$[$ \moc{EDIT} \dusa{iprint} $]$ \\
$[$ \moc{ORIG} \dusa{orig} $]$ \\
$[[$ \dusa{parkey} \dusa{value} $]]$ \\
$[$ \moc{WARNING-ONLY} $]$ \\
{\tt ;}
\end{DataStructure}

\noindent where
\begin{ListeDeDescription}{mmmmmmmm}

\item[\moc{EDIT}] keyword used to set \dusa{iprint}.

\item[\dusa{iprint}] index used to control the printing in module {\tt
SAP:}. =0 for no print; =1 for minimum printing (default value).

\item[\dusa{parkey}] {\tt character*4} .keyword associated to a
global parameter that is specific to \dusa{SAPNAM} (not defined in \dusa{SAPRHS}).

\item[\dusa{value}] floating-point, integer or {\tt character*12} value of a user-defined
global parameter.

\item[\moc{ORIG}] keyword used to define the father node in the parameter tree. By
default, the index of the previous elementary calculation is used.

\item[\dusa{orig}] index of the elementary calculation associated to the father node in the
parameter tree.

\item[\moc{WARNING-ONLY}] This option is useful if an elementary calculation in \dusa{SAPRHS} 
is already present in \dusa{SAPNAM}. If this keyword is set, a warning is send and the \dusa{SAPNAM} values
are kept, otherwise the run is aborted (default).

\end{ListeDeDescription}

\clearpage
 % structure (dragonSAP)
\subsection{The {\tt MPO:} module}\label{sect:MPOData}

This component of the lattice code is dedicated to the constitution of the
reactor database in MPO format, similar to the file produced by APOLLO3.\cite{Apollo3}
The MPO file intended to store {\sl all} the nuclear data, produced in
the lattice code, that is useful
in reactor calculations including fuel management and space-time kinetics.
Multigroup lattice calculations are too expensive to be executed dynamically
from the driver of the global reactor calculation. A more feasible
approach is to create a reactor database where a finite number of lattice
calculation results are tabulated against selected {\sl global parameters}
chosen so as to represent expected operating conditions of the reactor. The
\moc{MPO:} operator is used to create and construct a {\sc MPO} file.
The MPO file is written in {\sc hdf5} format, allowing full portability and hierarchical
data organization. It can be edited and modified using the HDFView tool.

\vskip 0.1cm

Each elementary calculation is characterized by a tuple of {\sl global parameters}.
These global parameters are of different types, depending on the nature of the
study under consideration: type of assembly, power, temperature in a mixture,
concentration of an isotope, time, burnup or exposure rate in a depletion calculation,
etc. Each step of a depletion calculation represents an elementary calculation.
The {\sc MPO} file is often presented as a {\sl multi-parameter reactor database}.

\vskip 0.1cm

For each elementary calculation, the results are recovered from the output of the
\moc{EDI:} operator and stored in a set of {\sl homogenized mixture}
directories. The \moc{EDI:} operator is responsible for performing condensation
in energy and homogenization in space of the macroscopic and microscopic cross
sections. All the elementary calculations gathered in a single {\sc mpo} file are
characterized by a single output geometry and a unique output energy-group
structure. The {\sc mpo} file may contain many geometry/energy-group combinations.

\vskip 0.1cm

In each homogenized mixture directory, the \moc{MPO:} operator recover
cross sections for a number of {\sl particularized isotopes} and {\sl macroscopic
residual sets}, a collection of isotopic cross sections weighted by isotopic number densities.
Cross sections for particularized isotopes and macroscopic sets are recovered for
{\sl selected reactions}. Other information is also recovered: multigroup neutron
fluxes, isotopic number densities, fission spectrum and a set
of {\sl local variables}. The local variables are values that characterize each
homogenized mixture: local power, burnup, exposure rate, etc. Some local variables
are arrays of values (eg: SPH equivalence factors). Discontinuity factors and equivalent albedos
are written in group {\tt flux}, as described in Sect.~\ref{sect:adf}. Finally, note that cross section
information written on the {\sc mpo} file is {\sl not} transport corrected and {\sl not}
SPH corrected.

\vskip 0.1cm

A different specification of the \moc{MPO:} function call is used for
creation and construction of the {\sc mpo} file.
\begin{itemize}
\item The first specification is used to initialize the {\sc mpo} data structure
as a function of the \dds{microlib} used in the reference calculation. Optionnally,
the homogenized geometry is also provided. The initialization call is also used to
set the choice of global parameters, local variables, particularized isotopes,
macroscopic sets and selected reactions.
\item A modification call to the \moc{MPO:} function is performed after each
elementary calculation in order to recover output information processed by \moc{EDI:}
(condensed and homogenized cross sections) and \moc{EVO:} (burnup dependant values).
Global parameters and local variables can optionnally be recovered from \dds{microlib}
objects. The \moc{EDI:} calculation is generally performed with option {\tt MICR ALL}.
\end{itemize}

The calling specifications are:

\vskip -0.5cm

\begin{DataStructure}{Structure \dstr{MPO:}}
$\{$~\dusa{MPONAM} \moc{:=} \moc{MPO:} $[$ \dusa{MPONAM} $]~[$~\dusa{HMIC} $]$ \moc{::} \dstr{mpo\_data1} \\
~~$|$~\dusa{MPONAM} \moc{:=} \moc{MPO:} \dusa{MPONAM}~\dusa{EDINAM}~$[$ \dusa{BRNNAM} $]~[$ \dusa{HMIC1}~$[$~\dusa{HMIC2} $]~]$ \moc{::} \dstr{mpo\_data2} \\
~~$|$~\dusa{MPONAM} \moc{:=} \moc{MPO:} \dusa{MPONAM} $[[$ \dusa{MPORHS} $]]$ \moc{::} dstr{mpo\_data3} $\}$ \\
\end{DataStructure}

\noindent where
\begin{ListeDeDescription}{mmmmmmm}

\item[\dusa{MPONAM}] {\tt character*12} name of the {\sc lcm} object containing the
{\sl master} {\sc mpo} data structure.

\item[\dusa{HMIC}] {\tt character*12} name of the reference \dds{microlib} (type {\tt
L\_LIBRARY}) containing the microscopic cross sections.

\item[\dusa{EDINAM}] {\tt character*12} name of the {\sc lcm} object (type {\tt
L\_EDIT}) containing the {\sc edition} data structure corresponding to an elementary
calculation. The {\sc edition} data produced by the last call to the {\tt EDI:} module
is used.

\item[\dusa{BRNNAM}] {\tt character*12} name of the {\sc lcm} object (type {\tt
L\_BURNUP}) containing the {\sc burnup} data structure. This object is compulsory if one
of the following parameters is used: \moc{IRRA}, \moc{FLUB} and/or \moc{TIME}.

\item[\dusa{HMIC1}] {\tt character*12} name of a \dds{microlib} (type {\tt
L\_LIBRARY}) containing global parameter information.

\item[\dusa{HMIC2}] {\tt character*12} name of a \dds{microlib} (type {\tt
L\_LIBRARY}) containing global parameter information.

\item[\dusa{MPORHS}] {\tt character*12} name of the {\sl read-only} {\sc mpo} data structure. This
data structure is concatenated to \dusa{MPONAM} using the \dusa{mpo\_data3} data structure,
as presented in \Sect{descmpo3}. \dusa{MPORHS} must be defined with the same number of energy
groups and the same number of homogeneous regions as \dusa{MPONAM}. Moreover, all the
global and local parameters of \dusa{MPORHS} must be defined in \dusa{MPONAM}. \dusa{MPONAM}
may be defined with {\sl global} parameters not defined in \dusa{MPORHS}.

\item[\dusa{mpo\_data1}] input data structure containing initialization information (see \Sect{descmpo1}).

\item[\dusa{mpo\_data2}] input data structure containing information related to the recovery of an
elementary calculation (see \Sect{descmpo2}).

\item[\dusa{mpo\_data3}] input data structure containing information related to the catenation of one or many
{\sl read-only} {\sc mpo} file(s) (see \Sect{descmpo3}).

\end{ListeDeDescription}

\newpage

\subsubsection{Initialization data input for module {\tt MPO:}}\label{sect:descmpo1}

\begin{DataStructure}{Structure \dstr{mpo\_data1}}
$[$~\moc{EDIT} \dusa{iprint}~$]$ \\
$[$~\moc{COMM}~\dusa{comment}~$]$ \\
$[[$~\moc{PARA}~\dusa{parkey} \\
~~~\{~\moc{TEMP}~\dusa{micnam}~\dusa{imix}~$|$~\moc{CONC}~\dusa{isonam1}~\dusa{micnam}~\dusa{imix}~$|$~\moc{IRRA}~$|$~\moc{FLUB}~$|$ \\
~~~~~~\moc{PUIS}~$|$~\moc{MASL}~$|$~\moc{FLUX}~$|$~\moc{TIME}~$|$~\moc{VALU}~\{~\moc{REAL}~$|$~\moc{CHAR}~$|$~\moc{INTE}~\}~\} \\
$]]$ \\
$[[$~\moc{LOCA}~\dusa{parkey} \\
~~~\{~\moc{TEMP}~$|$~\moc{CONC}~\dusa{isonam2}~$|$~\moc{IRRA}~$|$~\moc{FLUB}~$|$~~\moc{FLUG}~$|$~\moc{PUIS}~$|$~\moc{MASL}~$|$~\moc{FLUX}~$|$~\moc{EQUI}~\} \\
$]]$ \\
$[$~\moc{ISOT}~\{~\moc{TOUT}~$|$ \moc{MILI}~\dusa{imil}~$|~[$~\moc{FISS}~$]~[$~\moc{PF}~$]~[$~(\dusa{HNAISO}(i),~i=1,$N_{\rm iso}$) $]$~\}~$]$ \\
$[$~\moc{REAC}~(\dusa{HNAREA}(i),~i=1,$N_{\rm reac}$) $]$ \\
{\tt ;}
\end{DataStructure}

\goodbreak
\noindent where
\begin{ListeDeDescription}{mmmmmmmm}

\item[\moc{EDIT}] keyword used to set \dusa{iprint}.

\item[\dusa{iprint}] index used to control the printing in module {\tt
MPO:}. =0 for no print; =1 for minimum printing (default value).

\item[\moc{COMM}] keyword used to input a general comment for the {\sc mpo} file.

\item[\dusa{comment}] {\tt character*132} user-defined comment.

\item[\moc{PARA}] keyword used to define a single global parameter.

\item[\moc{LOCA}] keyword used to define a single local variable (a local variable
may be a single value or an array of values).

\item[\dusa{parkey}] {\tt character*24} user-defined keyword associated to a global
parameter or local variable.

\item[\dusa{micnam}] {\tt character*12} name of the \dds{microlib} (type {\tt
L\_LIBRARY}) associated to a global parameter. The corresponding \dds{microlib} will be required on
RHS of the \moc{MPO:} call described in Sect.~\ref{sect:descmpo2}.

\item[\dusa{imix}] index of the mixture associated to a global parameter. This mixture is
located in \dds{microlib} named \dusa{micnam}.

\item[\dusa{isonam1}] {\tt character*8} alias name of the isotope associated to a global
parameter. This isotope is located in \dds{microlib} data structure named \dusa{micnam}.

\item[\dusa{isonam2}] {\tt character*8} alias name of the isotope associated to a local
variable. This isotope is located in the \dds{microlib} directory of the {\sc edition}
data structure named \dusa{EDINAM}.

\item[\moc{TEMP}] keyword used to define a temperature (in Kelvin) as global parameter or
local variable.

\item[\moc{CONC}] keyword used to define a number density as global parameter or
local variable.

\item[\moc{IRRA}] keyword used to define a burnup (in MWday/Tonne) as global
parameter or local variable.

\item[\moc{FLUB}] keyword used to define a {\sl fuel-only} exposure rate (in n/kb) as global
parameter or local variable. The exposure rate is recovered from the \dusa{BRNNAM}
LCM object.

\item[\moc{FLUG}] keyword used to define an exposure rate in global homogenized mixtures (in n/kb) as
local variable. The exposure rate is recovered from the \dusa{BRNNAM}
LCM object.

\item[\moc{PUIS}] keyword used to define the power as global parameter or
local variable.

\item[\moc{MASL}] keyword used to define the mass density of heavy isotopes as
global parameter or local variable.

\item[\moc{FLUX}] keyword used to define the volume-averaged, energy-integrated flux as
global parameter or local variable.

\item[\moc{TIME}] keyword used to define the time (in seconds) as global parameter.

\item[\moc{EQUI}] keyword used to define the SPH equivalence factors as
local variable. A set of SPH factors can be defined as local
variables. Note that the cross sections and fluxes stored in the {\sc mpo} file are
{\sl not} SPH corrected.

\item[\moc{VALU}] keyword used to define a user-defined quantity as global parameter.
This keyword must be followed by the type of parameter.

\item[\moc{REAL}] keyword used to indicate that the user-defined global parameter
is a floating point value.

\item[\moc{CHAR}] keyword used to indicate that the user-defined global parameter
is a {\tt character*12} value.

\item[\moc{INTE}] keyword used to indicate that the user-defined global parameter
is an integer value.

\item[\moc{ISOT}] keyword used to select the set of particularized isotopes. The macroscopic
residual {\tt 'TotalResidual\_mix'} is always included as the last isotope in the list.

\item[\moc{TOUT}] keyword used to select all the available isotopes in the reference
\dds{microlib} named \dusa{HMIC} as particularized isotopes.

\item[\moc{MILI}] keyword used to select the isotopes in the reference
\dds{microlib} named \dusa{HMIC} from a specific mixture as particularized isotopes.

\item[\dusa{imil}] index of the mixture where the particularized isotopes are recovered.

\item[\moc{FISS}] keyword used to select all the available fissile isotopes in the reference
\dds{microlib} named \dusa{HMIC} as particularized isotopes.

\item[\moc{PF}] keyword used to select all the available fission products in the reference
\dds{microlib} named \dusa{HMIC} as particularized isotopes.

\item[\dusa{HNAISO}(i)] {\tt character*12} user-defined isotope name. $N_{\rm iso}$ is the
total number of explicitely--selected particularized isotopes.

\item[\moc{REAC}] keyword used to select the set of nuclear reactions. By fefault, the following reactions are selected:

\begin{tabular}{p{3.5cm} p{12.5cm}|}
\moc{Total} & Total cross sections as $\sigma_g^{\rm absorption}+\sigma_{0,g}^{\rm diffusion}$\\
\moc{Absorption} & Absorption cross sections $\sigma_g^{\rm absorption}$\\
\moc{Diffusion} & Scattering cross section for each available Legendre order \\
& $\sigma_{\ell,g}^{\rm diffusion}$. These cross sections are {\sl not} multiply by the $2\ell+1$ \\
& factor.\\
\moc{Fission} & Fission cross section \\
\moc{FissionSpectrum} & Steady-state fission spectrum \\
\moc{Nexcess} & Excess cross section due to (n,$x$n) reactions \\
\moc{NuFission} & $\nu\Sigma_{\rm f}$ cross sections \\
\moc{Scattering} & Scattering reaction as $\sigma_{\ell,g\rightarrow g'}=\sigma_{\ell,g\rightarrow g'}^{\rm elastic}+
\sigma_{\ell,g\rightarrow g'}^{\rm inelastic}+\sigma_{\ell,g\rightarrow g'}^{({\rm n},x{\rm n})}$\\
\moc{CaptureEnergyCapture} &  Energy production cross section for (n,$\gamma$) reaction only \\
\moc{FissionEnergyFission} & Energy production cross section for (n,f) reaction only \\
\end{tabular}

\item[\dusa{HNAREA}] {\tt character*20} name of a user-selected reaction in addition to default set. \dusa{HNAREA} is
selected among the following values:

\begin{tabular}{p{3.3cm} p{12.7cm}|}
\moc{TotalP1} & Total $P_1$-weighted cross sections \\
\moc{ElasticDiffusion} & Elastic scattering cross section for each available Legendre order \\
\moc{InelasticDiffusion} & Inelastic scattering cross section for each available Legendre order \\
\moc{NxnDiffusion} & (n,$x$n) scattering cross section for each available Legendre order \\
\moc{ElasticScattering} & Elastic scattering reaction $\sigma_{\ell,g\rightarrow g'}^{\rm elastic}$ \\
\moc{InelasticScattering} & Inelastic scattering reaction $\sigma_{\ell,g\rightarrow g'}^{\rm inelastic}$ \\
\moc{NxnScattering} & (n,$x$n) scattering reaction $\sigma_{\ell,g\rightarrow g'}^{({\rm n},x{\rm n})}$ \\
\moc{MT16}   & (n,2n) production cross sections \\
\moc{MT17}   & (n,3n) production cross sections \\
\moc{MT28}   & (n,np) production cross sections \\
\moc{MT37}   & (n,4n) production cross sections \\
\moc{MT103}   & (n,p) production cross sections \\
\moc{MT104}   & (n,d) production cross sections \\
\moc{MT105}   & (n,t) production cross sections \\
\moc{MT107}   & (n,$\alpha$) production cross sections \\
\moc{MT108}   & (n,2$\alpha$) production cross sections \\
\moc{Capture}   & (n,$\gamma$) production cross sections \\
\end{tabular}

\end{ListeDeDescription}

\subsubsection{Modification data input for module {\tt MPO:}}\label{sect:descmpo2}

\begin{DataStructure}{Structure \dstr{mpo\_data2}}
$[$ \moc{EDIT} \dusa{iprint} $]$ \\
$[$ \moc{STEP} \dusa{NAMDIR} $]$ \\
$[[$ \dusa{parkey} \dusa{value} $]]$ \\
$[$ \moc{SET} \dusa{xtr} $\{$ \moc{S} $|$ \moc{DAY} $|$ \moc{YEAR} $\}$ $]$ \\
{\tt ;}
\end{DataStructure}

\goodbreak
\noindent where
\begin{ListeDeDescription}{mmmmmmmm}

\item[\moc{EDIT}] keyword used to set \dusa{iprint}.

\item[\dusa{iprint}] index used to control the printing in module {\tt
MPO:}. =0 for no print; =1 for minimum printing (default value).

\item[\moc{STEP}] keyword used to access the {\sc mpo} database from a group named \dusa{NAMDIR}.
The default value is {\tt 'output\_0'}.

\item[\dusa{NAMDIR}] access the {\sc mpo} database in the group named \dusa{NAMDIR}. This name is
the concatenation of prefix {\tt 'output\_'} with an integer $\ge 0$.

\item[\dusa{parkey}] {\tt character*24} keyword associated to a user-defined global
parameter.

\item[\dusa{value}] floating-point, integer or {\tt character*12} value of a user-defined
global parameter.

\item[\moc{SET}] keyword used to recover the flux normalization factor already
stored on \dusa{BRNNAM} from a sub-directory corresponding to a specific time.

\item[\dusa{xtr}] time associated with the current flux calculation. The
name of the sub-directory where this information is stored will be given by
`{\tt DEPL-DAT}'//{\tt CNN} where {\tt CNN} is a  {\tt character*4} variable
defined by  {\tt WRITE(CNN,'(I4)') INN} where {\tt INN} is an index associated
with the time \dusa{xtr}.

\item[\moc{S}] keyword to specify that the time is given in seconds.

\item[\moc{DAY}] keyword to specify that the time is given in days.

\item[\moc{YEAR}] keyword to specify that the time is given in years.

\end{ListeDeDescription}

\subsubsection{Modification (catenate) data input for module {\tt MPO:}}\label{sect:descmpo3}

\vskip -0.5cm

\begin{DataStructure}{Structure \dstr{mpo\_data3}}
$[$ \moc{EDIT} \dusa{iprint} $]$ \\
$[$ \moc{STEP} \dusa{NAMDIR} $]$ \\
$[[$ \dusa{parkey} \dusa{value} $]]$ \\
$[$ \moc{WARNING-ONLY} $]$ \\
{\tt ;}
\end{DataStructure}

\noindent where
\begin{ListeDeDescription}{mmmmmmmm}

\item[\moc{EDIT}] keyword used to set \dusa{iprint}.

\item[\dusa{iprint}] index used to control the printing in module {\tt
MPO:}. =0 for no print; =1 for minimum printing (default value).

\item[\moc{STEP}] keyword used to access the {\sc mpo} database from a group named \dusa{NAMDIR}.
The default value is {\tt 'output\_0'}.

\item[\dusa{NAMDIR}] access the {\sc mpo} database in the group named \dusa{NAMDIR}. This name is
the concatenation of prefix {\tt 'output\_'} with an integer $\ge 0$.

\item[\dusa{parkey}] {\tt character*24} keyword associated to a
global parameter that is specific to \dusa{MPONAM} (not defined in \dusa{MPORHS}).

\item[\dusa{value}] floating-point, integer or {\tt character*12} value of a user-defined
global parameter.

\item[\moc{WARNING-ONLY}] This option is useful if an elementary calculation in \dusa{MPORHS} 
is already present in \dusa{MPONAM}. If this keyword is set, a warning is send and the \dusa{MPONAM} values
are kept, otherwise the run is aborted (default).

\end{ListeDeDescription}

\subsubsection{Specification of discontinuity factor and equivalent albedo information}\label{sect:adf}

Discontinuity factors and equivalent albedos are written in group {\tt flux} included in each state point group of the MPO file.
Specification of some datasets in group  {\tt flux} are slightly modified to hold this new information:

\begin{DescriptionEnregistrement}{Group /flux/ of the MPO file}{7.5cm}
\label{tabl:tabiso202}
\IntEnr
  {NALBP}{$1$}
  {Number $N_{\rm alb}$ of physical albedo (index $a$) values in each energy group.}
\IntEnr
  {NSURF}{$1$}
  {Number $N_{\rm surf}$ of external surfaces (index $b$) where discontinuity factors are defined.}
\OptRealEnr
  {ALBEDOG}{$N_{\rm alb},N_{\rm grp}$}{$N_{\rm alb}\ge 1$}{1}
  {Multigroup albedos $\beta_{a,g}$ obtained with a nodal equivalence procedure.}
\OptRealEnr
  {ALBEDOGxG}{$N_{\rm alb},N_{\rm grp}^2$}{$N_{\rm alb}\ge 1$}{1}
  {Matrix albedos $\beta_{a,g\to h}$ obtained with the equivalent reflector model (ERM).}
\OptRealEnr
  {SURF}{$N_{\rm surf}$}{$N_{\rm surf}\ge 1$}{cm$^2$}
  {Area $S_b$ of external surfaces.}
\OptRealEnr
  {SURFFLUX}{$N_{\rm surf},N_{\rm mil}\times N_{\rm grp}$}{$N_{\rm surf}\ge 1$}{$\phi \,$cm$^2$}
  {Multigroup fluxes $\phi^{\rm s}_{b,i,g}$ on external surfaces obtained with a nodal equivalence procedure in each output mixture $i$.}
\OptRealEnr
  {SURFFLUXGxG}{$N_{\rm surf},N_{\rm mil}\times N_{\rm grp}^2$}{$N_{\rm surf}\ge 1$}{$\phi \,$cm$^2$}
  {Matrix fluxes $\phi^{\rm s}_{b,i,g\to h}$ on external surfaces obtained with the equivalent reflector model (ERM) in each output mixture $i$.}
\RealEnr
  {TOTALFLUX}{$N_{\rm mil}\times N_{\rm grp}$}{$\phi$}
  {Multigroup fluxes $\phi_{i,g}$ in each output mixture $i$.}
\end{DescriptionEnregistrement}

\noindent where $N_{\rm mil}$ is the number of output mixtures and $N_{\rm grp}$ is the number of energy groups. Discontinuity factors are obtained from equations:
$$
ADF_{b,i,g}={\phi^{\rm s}_{b,i,g} \over S_b \, \phi_{i,g}} \ \  {\rm for \ multigroup \ discontinuity \ factors}
$$
\noindent and
$$
ADF_{b,i,g\to h}={\phi^{\rm s}_{b,i,g\to h} \over S_b \, \phi_{i,g}} \ \  {\rm for \ matrix \ discontinuity \ factors.}
$$

\clearpage
 % structure (dragonMPO)
\subsection{The {\tt MC:} module}\label{sect:MCData}

This component of the lattice code is dedicated to the Monte-Carlo solution of the transport
equation in multigroup approximation.

\vskip 0.02cm

The calling specifications are:

\begin{DataStructure}{Structure \dstr{MC:}}
\dusa{OUTMC}~$[$~\dusa{TRACK}~$]$~\moc{:=}~\moc{MC:}~$[$~\dusa{OUTMC}~$]$~\dusa{TRACK}~$\{$~\dusa{MICRO}~$|$~\dusa{MACRO}~$\}$~\moc{::}~\dstr{MC\_data} \\
\end{DataStructure}

\noindent where
\begin{ListeDeDescription}{mmmmmmm}

\item[\dusa{OUTMC}] {\tt character*12} name of a {\sc Monte-Carlo} (type {\tt L\_MC}) object open in modification or creation
mode.

\item[\dusa{TRACK}] {\tt character*12} name of a \dusa{NXT:} {\sc tracking} (type {\tt L\_TRACK}) object open in
read-only or modification mode. Object \dusa{TRACK} must be constructed with option \moc{MC} activated (see \Sect{NXTData}). Opening \dusa{TRACK}
in modification mode is useful to add tracking information to be plotted with module \moc{PSP:} (see \Sect{PSPData}).

\item[\dusa{MICRO}] {\tt character*12} name of a {\sc microlib} (type {\tt L\_LIBRARY}) object open in read-only mode. The information on
the embedded macrolib is used.

\item[\dusa{MACRO}] {\tt character*12} name of a {\sc macrolib} (type {\tt L\_MACROLIB}) object open in read-only mode.

\item[\dusa{MC\_data}] input data structure containing specific data (see \Sect{descMC}).

\end{ListeDeDescription}

\subsubsection{Data input for module {\tt MC:}}\label{sect:descMC}

\vskip -0.5cm

\begin{DataStructure}{Structure \dstr{MC\_data}}
$[$~\moc{EDIT} \dusa{iprint}~$]$ \\
\moc{KCODE}~\dusa{nsrck}~\dusa{ikz}~\dusa{kct} \\
$[$~\moc{SEED} \dusa{iseed}~$]~[$~\moc{N2N}~$]$ \\
$[$~\moc{TALLY} \\
\hskip 1.0cm $[$ \moc{MERG} $\{$ \moc{COMP} $|$ \moc{NONE} $|$ \\
\hskip 2.0cm \moc{REGI} (\dusa{iregm}(ii),ii=1,nregio) $|$ \\
\hskip 2.0cm \moc{MIX} $[$ (\dusa{imixm}(ii),ii=1,nbmix) $]~\}$ $]$ \\
\hskip 1.0cm $[$ \moc{COND} $[~\{$  \moc{NONE} $|$ ( \dusa{icond}(ii), ii=1,ngcond) $\}~]~]$\\
\moc{ENDT} $]$ \\
{\tt ;}
\end{DataStructure}

\noindent where
\begin{ListeDeDescription}{mmmmmmmm}

\item[\moc{EDIT}] keyword used to set \dusa{iprint}.

\item[\dusa{iprint}] index used to control the printing in module {\tt MC:}. =0 for no print; =1 for minimum printing (default value);
=100 to add free-path information in object \dusa{TRACK} (must be open in modification mode in that case).

\item[\moc{KCODE}] keyword used to define the power iteration settings.

\item[\dusa{nsrck}] number of neutrons generated per cycle

\item[\dusa{ikz}] number of inactive cycles

\item[\dusa{kct}] number of active cycles

\item[\moc{SEED}] keyword used to set the initial seed integer for the random number generator. By default, the seed integer is set from
the processor clock.

\item[\dusa{iseed}] initial seed integer

\item[\moc{N2N}] keyword used to enable an explicit treatment of $(n,2n)$ reactions. In this case, {\tt N2N} cross sections are
expected to be available in the macrolib. By default, $(n,2n)$ reactions are taken into account implicitly by the correction on scattering
cross sections.

\item[\moc{TALLY}] keyword used to define a tally (macrolib and effective multiplication factor). Using "\moc{TALLY~ENDT}" construct
permits to obtain a virtual collision estimation of the effective multiplication factor {\sl without} estimation of the macrolib
information.

\item[\moc{NONE}] keyword to deactivate the homogeneization or the condensation. 

\item[\moc{MERG}] keyword to specify that the neutron flux is to be
homogenized over specified regions or mixtures. 

\item[\moc{REGI}] keyword to specify that the homogenization of the neutron
flux will take place over the following regions. Here nregio$\le$\dusa{maxreg}
with \dusa{maxreg} the maximum number of regions for which solutions were
obtained.

\item[\dusa{iregm}] array of homogenized region numbers to which are
associated the old regions. In the editing routines a value of \dusa{iregm}=0
allows the corresponding region to be neglected. 

\item[\moc{MIX}] keyword to specify that the homogenization of the neutron
flux will take place over the following mixtures. Here
we must have nbmix$\le$\dusa{maxmix} where \dusa{maxmix} is the maximum number
of mixtures in the macroscopic cross section library.  

\item[\dusa{imixm}] array of homogenized region numbers to which are
associated the material mixtures. In the editing routines a value of
\dusa{imixm}=0 allows the corresponding isotopic mixtures to be neglected. For a mixture in this
library which is not used in the geometry one should insert a value of 0 for the
new region number associated with this mixture. By default, if \moc{MIX} is set and
\dusa{imixm} is not set, \dusa{imixm(ii)}$=$\dusa{ii} is assumed.

\item[\moc{COMP}] keyword to specify that the a complete homogenization is to
take place.

\item[\moc{COND}] keyword to specify that a group condensation of the flux is
to be performed.

\item[\dusa{icond}] array of increasing energy group limits that will be associated with
each of the ngcond condensed groups. The final value of
\dusa{icond} will automatically be set to \dusa{ngroup} while the values of 
\dusa{icond}$>$\dusa{ngroup} will be droped from the condensation. 
We must have ngcond$\le$\dusa{ngroup}. By default, if \moc{COND} is set and \dusa{icond}
is not set, all energy groups are condensed together.

\item[\moc{ENDT}] keyword used to terminate the definition of a tally.

\end{ListeDeDescription}

\eject
 % structure (dragonMC)
\subsection{The \moc{T:} module}\label{sect:TData}

A \dds{macrolib}  object can be defined directly using module \moc{MAC:} (see \Sect{MACData})
or as part of a \dds{microlib} object using module \moc{LIB:} (see \Sect{LIBData}). It is possible to
transpose a \dds{macrolib}  using the module \moc{T:}. Transposition consists in
\begin{itemize}
\item renumbering the energy groups from thermal to fast
\item transposing the transfer matrices (\moc{SCAT}) so that the primary and secondary energy group indices are permuted
\item storing \moc{NUSIGF} information in \moc{CHI} and storing \moc{CHI} infomation in \moc{NUSIGF}.
\end{itemize}

A transposed \dds{macrolib}  object permits to make adjoint flux calculations.

\vskip 0.08cm

The general format of the data for the \moc{T:} module is the following:

\begin{DataStructure}{Structure \dstr{T:}} 
\dusa{MACLIB1} \moc{:=} \moc{T:} $\{$ \dusa{MACLIB2} $|$ \dusa{LIBRARY} $\}$ \moc{;}
\end{DataStructure}

\noindent where
\begin{ListeDeDescription}{mmmmmmmm}

\item[\dusa{MACLIB1}] {\tt character*12} name of a the transposed \dds{macrolib}

\item[\dusa{MACLIB2}] {\tt character*12} name of a the original \dds{macrolib}

\item[\dusa{LIBRARY}] {\tt character*12} name of a the original \dds{microlib} containing an embedded \dds{macrolib}.

\end{ListeDeDescription}
\eject
 % structure (dragonT)
\subsection{The {\tt DMAC:} module}\label{sect:DMACData}

This module is used to set fixed sources that can be used in the right hand term of an adjoint
fixed source eigenvalue problem. This type of equation appears in generalized perturbation theory (GPT) applications.
The fixed sources set in {\tt DMAC:} are corresponding to the gradient of a reference
macrolib with respect to homogenization and condensation of the cross-section information. The gradient
of a cross section $\Sigma(\bff(r))={\rm col}\{\Sigma_1(\bff(r)) \, , \ \Sigma_2(\bff(r))\}$ with respect to
homogenization and condensation is defined as
\begin{align*}
\bff(\nabla)P\{\bff(\phi)(\zeta);\bff(r)\}=P\{\bff(\phi)(\bff(r))\}
\left[\begin{matrix}{\Sigma_1(\bff(r))\over \left<\bff(\Sigma),\bff(\phi)\right>}-{1\over
\left<\bff(\phi)\right>} \cr {\Sigma_2(\bff(r))\over \left<\bff(\Sigma),\bff(\phi)\right>}-{1\over
\left<\bff(\phi)\right>}\end{matrix}\right]
\end{align*}

\noindent where the homogenized and condensed cross section is an homogeneous functional of the flux defined as
$$
P\{\bff(\phi)(r)\}={\left<\bff(\Sigma),\bff(\phi)\right>\over \left<\bff(\phi)\right>} \ \ \  .
$$

Each fixed source $\bff(\nabla)P\{\bff(\phi)(\zeta);\bff(r)\}$ is orthogonal to the flux $\bff(\phi)(\bff(r))$.

\vskip 0.02cm

The calling specifications are:

\begin{DataStructure}{Structure \dstr{DMAC:}}
\dusa{SOURCE}~\moc{:=}~\moc{DMAC:}~\dusa{FLUX}~$\{$~\dusa{MICRO}~$|$~\dusa{MACRO}~$\}$~\dusa{TRACK}~\moc{::}~\dstr{DMAC\_data} \\
\end{DataStructure}

\noindent where
\begin{ListeDeDescription}{mmmmmmm}

\item[\dusa{SOURCE}] {\tt character*12} name of a {\sc fixed sources} (type {\tt L\_SOURCE}) object open in creation
mode. This object contains a set of adjoint fixed sources corresponding to different macro-regions, macro-groups and cross-section types
present in the reference macrolib.

\item[\dusa{FLUX}] {\tt character*12} name of a reference {\sc flux} (type {\tt L\_FLUX}) object open in read-only mode.

\item[\dusa{MICRO}] {\tt character*12} name of a reference {\sc microlib} (type {\tt L\_LIBRARY}) object open in read-only mode. The information on
the embedded macrolib is used.

\item[\dusa{MACRO}] {\tt character*12} name of a reference {\sc macrolib} (type {\tt L\_MACROLIB}) object open in read-only mode.

\item[\dusa{TRACK}] {\tt character*12} name of a reference {\sc tracking} (type {\tt L\_TRACK}) object open in read-only mode.

\item[\dusa{DMAC\_data}] input data structure containing specific data (see \Sect{descDMAC}).

\end{ListeDeDescription}

\subsubsection{Data input for module {\tt DMAC:}}\label{sect:descDMAC}

\vskip -0.5cm

\begin{DataStructure}{Structure \dstr{DMAC\_data}}
$[$~\moc{EDIT} \dusa{iprint}~$]$ \\
$[$~\moc{RATE} \\
\hskip 1.0cm $[$ \moc{MERG} $\{$ \moc{COMP} $|$ \moc{NONE} $|$ \\
\hskip 2.0cm \moc{REGI} (\dusa{iregm}(ii),ii=1,nregio) $|$ \\
\hskip 2.0cm \moc{MIX} $[$ (\dusa{imixm}(ii),ii=1,nbmix) $]~\}$ $]$ \\
\hskip 1.0cm $[$ \moc{COND} $[~\{$  \moc{NONE} $|$ ( \dusa{icond}(ii), ii=1,ngcond) $\}~]~]$\\
\moc{ENDR} $]$ \\
{\tt ;}
\end{DataStructure}

\noindent where
\begin{ListeDeDescription}{mmmmmmmm}

\item[\moc{EDIT}] keyword used to set \dusa{iprint}.

\item[\dusa{iprint}] index used to control the printing in module {\tt DMAC:}. =0 for no print; =1 for minimum printing (default value).

\item[\moc{RATE}] keyword used to define the homogenization and condensation limits.

\item[\moc{NONE}] keyword to deactivate the homogeneization or the condensation. 

\item[\moc{MERG}] keyword to specify that the neutron flux is to be
homogenized over specified regions or mixtures. 

\item[\moc{REGI}] keyword to specify that the homogenization of the neutron
flux will take place over the following regions. Here nregio$\le$\dusa{maxreg}
with \dusa{maxreg} the maximum number of regions for which solutions were
obtained.

\item[\dusa{iregm}] array of homogenized region numbers to which are
associated the old regions. In the editing routines a value of \dusa{iregm}=0
allows the corresponding region to be neglected. 

\item[\moc{MIX}] keyword to specify that the homogenization of the neutron
flux will take place over the following mixtures. Here
we must have nbmix$\le$\dusa{maxmix} where \dusa{maxmix} is the maximum number
of mixtures in the macroscopic cross section library.  

\item[\dusa{imixm}] array of homogenized region numbers to which are
associated the material mixtures. In the editing routines a value of
\dusa{imixm}=0 allows the corresponding isotopic mixtures to be neglected. For a mixture in this
library which is not used in the geometry one should insert a value of 0 for the
new region number associated with this mixture. By default, if \moc{MIX} is set and
\dusa{imixm} is not set, \dusa{imixm(ii)}$=$\dusa{ii} is assumed.

\item[\moc{COMP}] keyword to specify that the a complete homogenization is to
take place.

\item[\moc{COND}] keyword to specify that a group condensation of the flux is
to be performed.

\item[\dusa{icond}] array of increasing energy group limits that will be associated with
each of the ngcond condensed groups. The final value of
\dusa{icond} will automatically be set to \dusa{ngroup} while the values of 
\dusa{icond}$>$\dusa{ngroup} will be droped from the condensation. 
We must have ngcond$\le$\dusa{ngroup}. By default, if \moc{COND} is set and \dusa{icond}
is not set, all energy groups are condensed together.

\item[\moc{ENDR}] keyword used to terminate the definition of the homogenization and condensation.

\end{ListeDeDescription}

\eject
 % structure (dragonDMAC)
\subsection{The {\tt SENS:} module}\label{sect:SENSData}

This module is used to perform an explicit sensitivity analysis of keff to nuclear data represented by the cross sections.\cite{Laville}
The calculations are performed using adjoint-based first-order-linear perturbation theory and require the adjoint flux (see \Sect{FLUData}).
The sensitivity coefficients are stored in a \textit{SDF} text file that is compatible with the \moc{JAVAPENO} module of SCALE\cite{SCALE}
(this compatibility is achieved via a slight modification of the \textit{rdragon} execution script).
An example of modification is presented in the file \moc{sens.save} from the \textit{non regression testcase} \moc{sens.x2m}.

\vskip 0.02cm

The calling specifications are:

\begin{DataStructure}{Structure \dstr{SENS:}}
\dusa{SENS.sdf}~\moc{:=}~\moc{SENS:}~\dusa{FLUNAM}~\dusa{ADJ$\_$FLUNAM}~\dusa{TRKNAM}~\dusa{MACRO}~\moc{::}~\dstr{SENS\_data} \\
\end{DataStructure}

\noindent where
\begin{ListeDeDescription}{mmmmmmm}

\item[\dusa{SENS.sdf}] {\tt character*12} name of a {\sc SDF} file object that is created by {\tt SENS:}.

\item[\dusa{FLUNAM}] {\tt character*12} name of the required {\sc flux} (type {\tt L\_FLUX}) object open in read-only mode.

\item[\dusa{ADJ$\_$FLUNAM}] {\tt character*12} name of the required {\sc adjoint flux} (type {\tt L\_FLUX}) object open in read-only mode.

\item[\dusa{TRKNAM}] {\tt character*12} name of the required {\sc tracking} (type {\tt L\_TRACK}) object open in read-only mode.

\item[\dusa{MACRO}] {\tt character*12} name of the required {\sc macrolib} (type {\tt L\_MACROLIB}) object open in read-only mode.

\item[\dusa{SENS\_data}] input data structure containing specific data (see \Sect{descSENS}).

\end{ListeDeDescription}

\subsubsection{Data input for module {\tt SENS:}}\label{sect:descSENS}

\vskip -0.5cm

\begin{DataStructure}{Structure \dstr{SENS\_data}}
$[$~\moc{EDIT} \dusa{iprint}~$]$ \\
$[$~\moc{ANIS} \dusa{nanis}~$]$ \\
\moc{;}
\end{DataStructure}

\noindent where
\begin{ListeDeDescription}{mmmmmmmm}

\item[\moc{EDIT}] keyword used to set \dusa{iprint}.

\item[\dusa{iprint}] index used to control the printing in module {\tt SENS:}. =0 for no print; =1 for minimum printing (default value).

\item[\moc{ANIS}] keyword used to specify the level \dusa{naniso} of anisotropy permitted in the calculation.

\item[\dusa{nanis}] number of Legendre orders for the representation of the scattering cross sections and the anisotropy of the flux. The default value is \dusa{nanis}=1 corresponding to the use of isotropic scattering cross sections and integrated flux. The number of Legendre orders used for the sensitivity calculations is the lowest between \dusa{nanis} and the level of anisotropy available in the \dusa{MACRO} data.

\end{ListeDeDescription}

\eject
 % structure (dragonSENS)
\subsection{The {\tt DUO:} module}\label{sect:DUOData}

This module is used to perform a perturbative analysis of two systems in fundamental mode conditions using the Clio formula and to determine the origins
of Keff discrepancies.

\vskip 0.02cm

The calling specifications are:

\begin{DataStructure}{Structure \dstr{DUO:}}
\moc{DUO:}~\dusa{MICLIB1}~\dusa{MICLIB2}~\moc{::}~\dstr{DUO\_data} \\
\end{DataStructure}

\noindent where
\begin{ListeDeDescription}{mmmmmmm}

\item[\dusa{MICLIB1}] {\tt character*12} name of the first {\sc microlib} (type {\tt L\_LIBRARY}) object open in read-only mode.

\item[\dusa{MICLIB2}] {\tt character*12} name of the second {\sc microlib} (type {\tt L\_LIBRARY}) object open in read-only mode.

\item[\dusa{DUO\_data}] input data structure containing specific data (see \Sect{descDUO}).

\end{ListeDeDescription}

\subsubsection{Data input for module {\tt DUO:}}\label{sect:descDUO}

Note that the input order must be respected.

\vskip -0.5cm

\begin{DataStructure}{Structure \dstr{DUO\_data}}
$[$~\moc{EDIT} \dusa{iprint}~$]$ \\
$[$~\moc{ENERGY} $]~[$ \moc{ISOTOPE} $]~[$ \moc{MIXTURE} $]$ \\
$[$ \moc{REAC} \\
~~~~$[[$ \dusa{reac} $[$ \moc{PICK}  {\tt >>} \dusa{deltaRho} {\tt <<} $]~]] $\\
~~\moc{ENDREAC} $]$ \\
;
\end{DataStructure}

\noindent where
\begin{ListeDeDescription}{mmmmmmmm}

\item[\moc{EDIT}] keyword used to set \dusa{iprint}.

\item[\dusa{iprint}] index used to control the printing in module {\tt DUO:}. =0 for no print; =1 for minimum printing (default value).

\item[\moc{ENERGY}] keyword used to perform a perturbation analysis as a function of the energy group indices.

\item[\moc{ISOTOPE}] keyword used to perform a perturbation analysis as a function of the isotopes present in the geometry.

\item[\moc{MIXTURE}] keyword used to perform a perturbation analysis as a function of the mixtures indices.

\item[\moc{REAC}] keyword used to perform a perturbation analysis for specific nuclear reactions.

\item[ \dusa{reac}] \texttt{character*8} name of a nuclear reaction $\sigma_x$. The reactivity effect is computed using the formula
\begin{equation}
\delta\lambda_x={(\bff(\phi)^*_1)^\top \delta\shadowS_x \, \bff(\phi)_2\over (\bff(\phi)^*_1)^\top \shadowP_2 \bff(\phi)_2} .
\end{equation}
\noindent where $\shadowS_x$ is a matrix containing the the contributions of the reaction $\sigma_x$. The other symbols
are defined in Sect.~\ref{sect:theoryDUO}. Examples of reaction names are:
\begin{description}
\item[{\tt NTOT0}:] total cross section
\item[{\tt NG}:] radiative capture cross section
\item[{\tt N2N}:] (n,2n) cross section
\item[{\tt NFTOT}:] fission cross section
\item[{\tt NELAS}:] elastic scattering cross section
\item[{\tt SCAT00}:] scattering matrix
\item[{\tt NUSIGF}:] dyadic product of the fission spectrum times $\nu$ fission cross section
\item[{\tt LEAK}:] neutron leakage
\end{description}
The balance relation for the global reactivity effect is
\begin{equation}
\delta\lambda=\delta\lambda_{\tt NTOT0}-\delta\lambda_{\tt SCAT00}-{\delta\lambda_{\tt NUSIGF}\over K_{\rm eff}}+\delta\lambda_{\tt LEAK}
\end{equation}
\noindent where $K_{\rm eff}$ is the effective multiplication factor.

\item[\moc{PICK}]  keyword used to recover the delta-rho discrepancy for reaction \dusa{reac} in a CLE-2000 variable.

\item[\dusa{deltaRho}] \texttt{character*12} CLE-2000 variable name in which the extracted  delta-rho discrepancy will be placed.

\item[\moc{ENDREAC}] keyword used to indicate that no more nuclear reactions will be analysed.

\end{ListeDeDescription}

\subsubsection{Theory} \label{sect:theoryDUO}

The module {\tt DUO:} is an implementation of the {\sc clio} perturbative analysis method, as introduced in Ref.~\citen{clio}. This method is useful for comparing two similar systems in fundamental mode conditions. It is based on fundamental mode balance equations that must be satisfied by the direct
and adjoint solutions of each of the two systems. The balance equation of the first system is written
\begin{equation}
\shadowL_1 \bff(\phi)_1-\lambda_1 \, \shadowP_1 \bff(\phi)_1=\bff(0) \ \ \ {\rm and} \ \ \ \shadowL_1^\top \bff(\phi)^*_1-\lambda_1 \, \shadowP^\top_1 \bff(\phi)^*_1=\bff(0)
\label{eq:duo1}
\end{equation}

\noindent where
\begin{description}
\item [$\shadowL_1=$] absorption (total plus leakage minus scattering) reaction rate matrix
\item [$\shadowP_1=$] production (nu times fission) reaction rate matrix
\item [$\lambda_1=$] one over the effective multiplication factor
\item [$\bff(\phi)_1=$] direct multigroup flux in each mixture of the geometry
\item [$\bff(\phi)^*_1=$] adjoint multigroup flux in each mixture of the geometry.
\end{description}

\vskip 0.08cm

Similarly, the balance equation of the second system is written
\begin{equation}
\shadowL_2 \bff(\phi)_2-\lambda_2 \, \shadowP_2 \bff(\phi)_2=\bff(0) .
\label{eq:duo2}
\end{equation}

\vskip 0.08cm

Next, we write
\begin{equation}
\shadowL_2 = \shadowL_1+\delta\shadowL \, \ \ \ \shadowP_2 = \shadowP_1+\delta\shadowP \ , \ \ \  \bff(\phi)_2=\bff(\phi)_1+\delta\bff(\phi) \ \ \ {\rm and} \ \ \ \lambda_2=\lambda_1+\delta\lambda .
\label{eq:duo3}
\end{equation}

\vskip 0.08cm

Substituting \Eq{duo3} into \Eq{duo2}, we write
\begin{equation}
\shadowL_1 \bff(\phi)_1+\shadowL_1 \delta\bff(\phi)+\delta\shadowL \bff(\phi)_2-\left[\lambda_1 \, \shadowP_1 \bff(\phi)_1+\lambda_1 \, \shadowP_1 \delta\bff(\phi)+(\lambda_2 \, \shadowP_2-\lambda_1 \, \shadowP_1) \, \bff(\phi)_2\right]=\bff(0) .
\label{eq:duo4}
\end{equation}

\vskip 0.08cm

Following the guideline from Ref.~\citen{clio}, we subtract \Eq{duo1} from \Eq{duo4} to obtain
\begin{equation}
(\shadowL_1-\lambda_1 \, \shadowP_1) \, \delta\bff(\phi)=(-\delta\shadowL +\lambda_2 \, \shadowP_2-\lambda_1 \, \shadowP_1) \, \bff(\phi)_2
\label{eq:duo5}
\end{equation}

\vskip 0.08cm

Next, we left-multiply this matrix system by a row vector equal to $(\bff(\phi)^*_1)^\top$, in order to make the LHS vanishing. This operation is written
\begin{equation}
(\bff(\phi)^*_1)^\top(\shadowL_1-\lambda_1 \, \shadowP_1) \, \delta\bff(\phi)=(\bff(\phi)^*_1)^\top(-\delta\shadowL +\lambda_2 \, \shadowP_2-\lambda_1 \, \shadowP_1) \, \bff(\phi)_2=0
\label{eq:duo6}
\end{equation}

\noindent because
\begin{equation}
(\bff(\phi)^*_1)^\top(\shadowL_1-\lambda_1 \, \shadowP_1) =\bff(0)^\top
\label{eq:duo7}
\end{equation}

\noindent in term of \Eq{duo1}.

\vskip 0.08cm

Using the relation $\lambda_2 \, \shadowP_2-\lambda_1 \, \shadowP_1=\delta\lambda\, \shadowP_2+\lambda_1 \, \delta\shadowP$, \Eq{duo6}
can be rewritten as
\begin{equation}
(\bff(\phi)^*_1)^\top(-\delta\shadowL +\delta\lambda\, \shadowP_2+\lambda_1 \, \delta\shadowP) \, \bff(\phi)_2=0
\label{eq:duo8}
\end{equation}

\noindent so that
\begin{equation}
\delta\lambda={(\bff(\phi)^*_1)^\top(\delta\shadowL -\lambda_1 \, \delta\shadowP) \, \bff(\phi)_2\over (\bff(\phi)^*_1)^\top \shadowP_2 \bff(\phi)_2} .
\label{eq:duo9}
\end{equation}

\vskip 0.08cm

Equation \ref{eq:duo9} is {\sl not} a first order perturbation approximation of $\delta\lambda$; it is an {\sl exact} expression of it. Its numerator is used
to obtain every component of $\delta\lambda$ in term of energy group, isotope, mixture and/or nuclear reaction.
\eject
 % structure (dragonDUO)
\subsection{The {\tt S2M:} module}\label{sect:S2MData}

This module is used to extract macrocoscopic cross-section data from a Matlab-formatted {\sc ascii} file
generated by the SERPENT Monte Carlo code (see Ref.~\citen{serpent}) and to convert it to the {\sc macrolib} format.

\vskip 0.02cm

The calling specifications are:

\begin{DataStructure}{Structure \dstr{S2M:}}
\dusa{MACRO}~\moc{:=}~\moc{S2M:}~\dusa{matlab.m}~\moc{::}~\dstr{S2M\_data} \\
\end{DataStructure}

\noindent where
\begin{ListeDeDescription}{mmmmmmm}

\item[\dusa{MACRO}] {\tt character*12} name of the required {\sc macrolib} (type {\tt L\_MACROLIB}) object that is created by {\tt S2M:}.

\item[\dusa{matlab.m}] {\tt character*12} name of a {\sc ascii} file, open in read-only mode, containing Matlab-formatted SERPENT information.

\item[\dusa{S2M\_data}] input data structure containing specific data (see \Sect{descS2M}).

\end{ListeDeDescription}

\subsubsection{Data input for module {\tt S2M:}}\label{sect:descS2M}

\vskip -0.5cm

\begin{DataStructure}{Structure \dstr{S2M\_data}}
$[$~\moc{EDIT} \dusa{iprint}~$]$ \\
$[$~\moc{IDX} \dusa{idx}~$]$ \\
$[$~\moc{B1} $]$ \\
\moc{;}
\end{DataStructure}

\noindent where
\begin{ListeDeDescription}{mmmmmmmm}

\item[\moc{EDIT}] keyword used to set \dusa{iprint}.

\item[\dusa{iprint}] index used to control the printing in module {\tt S2M:}. =0 for no print; =1 for minimum printing (default value).

\item[\moc{IDX}] keyword used to specify the occurence index of a flux calculation in the SERPENT output file. This index generally refers to the burnup step.

\item[\dusa{idx}] occurence index.

\item[\moc{B1}] keyword used to specify that diffusion coefficients and other fundamental-mode information are to be recovered from the SERPENT output file.

\end{ListeDeDescription}

\eject
 % structure (dragonS2M)
\subsection{The \moc{FMT:} module}\label{sect:FMTData}

The utility module \moc{FMT:} is used to format various data structure to suit the specific user needs.  Here three formatting options are available.
 \begin{enumerate}
\item The \moc{SUS3D} option where three files are created that respectively contain the integration weights and directions (ASCII), the directional flux (binary or ASCII) and the
directional adjoints (binary or ASCII) in a CP or $S_{N}$ format.\cite{Kodeli2001a,Bidaud2009a} The input specifications for this option are  

\begin{DataStructure}{Structure \dstr{FMT:} for \moc{SUS3D} option}%{SUS3Dcnt}
\dusa{WGTANGL} \dusa{DFLUX} \dusa{DADJOINTS} \moc{:=} \moc{FMT:} \dusa{FLUX}  \dusa{VOLTRK}  \moc{::}\\ 
\hspace{1.0cm} $[$ \moc{EDIT} \dusa{iprint} $]$ \\
\hspace{1.0cm} \moc{SUS3D} $[$ $\{$ \moc{SN} $|$ \moc{CP} $\}$ $]$
\end{DataStructure}

\item The \moc{DIRFLX} option where a single file is created that contain the directional flux, adjoints and generalized adjoints. The input specifications for this option are

\begin{DataStructure}{Structure \dstr{FMT:} for \moc{DIRFLX} option}%{DIRFLXcnt}
\dusa{DAF} \moc{:=} \moc{FMT:} \dusa{FLUX}  \dusa{VOLTRK}  \moc{::} \\ 
\hspace{1.0cm} $[$ \moc{EDIT} \dusa{iprint} $]$ \\
\hspace{1.0cm} \moc{DIRFLX} 
\end{DataStructure}

\item The \moc{BURNUP} option where a Matlab-m or Python file is created that contain the burnup time, the variation of $k_{\text{eff}}$ with time and the time dependent concentration of the different isotopes present in the geometry. The input specifications for this option are 

\begin{DataStructure}{Structure \dstr{FMT:} for \moc{BURNUP} option}%{BURNUPcnt}
\dusa{MFILE} \moc{:=} \moc{FMT:} \dusa{EDITION}  \dusa{BURNUP}  \moc{::} \\ 
\hspace{1.0cm} $[$ \moc{EDIT} \dusa{iprint} $]$ \\
\hspace{1.0cm} \moc{BURNUP} $[~\{$ \moc{MATLAB} $|$ \moc{PYTHON} $\}~]$ \\
\hspace{1.0cm} $[$ \moc{ISOP} $[$ (\moc{NAMISO}($i$),$i=1,I$) $]$ $]$ 
\end{DataStructure}

\end{enumerate}

\begin{ListeDeDescription}{mmmmmmmm}

\item[\dusa{WGTANGL}] \verb|character*12| name of the \moc{ASCII} file that will
contain the angular weights and directions.

\item[\dusa{DFLUX}] \verb|character*12| name of the \moc{ASCII} or \moc{BINARY} file that will
contain the directional flux in a SUS3D compatible format.

\item[\dusa{DADJOINTS}] \verb|character*12| name of the \moc{ASCII} or \moc{BINARY} file that will contain the directional adjoints in a SUS3D compatible format.

\item[\dusa{DAF}] \verb|character*12| name of the \moc{ASCII} file that will contain the spherical harmonic moments of the fluxes, adjoints and generalized
adjoints in a DIRFLX compatible format.

\item[\dusa{FLUX}] \verb|character*12| name of the \dds{fluxunk} data structure to process.

\item[\dusa{VOLTRK}] \verb|character*12| name of the \dds{tracking} data structure to process.

\item[\dusa{MFILE}] \verb|character*12|  name of the \moc{ASCII}  Matlab M-file that will
contain the burnup time, the time dependent $k_{\text{eff}}$ and concentration of the different isotopes present in the geometry.

\item[\dusa{EDITION}] \verb|character*12| name of the \dds{edition} data structure to process.

\item[\dusa{BURNUP}] \verb|character*12| name of the \dds{burnup} data structure to process.

\item[\moc{EDIT}] keyword used to modify the print level \dusa{iprint}.

\item[\dusa{iprint}] index used to control the printing in this module. 

\item[\moc{SUS3D}] keyword to activate the SUS3D processing option.

\item[\moc{SN}] keyword to generate $S_N$ compatible fluxes and adjoints (cell edge values). It is the default value.  

\item[\moc{CP}] keyword to generate CP compatible fluxes and adjoints (cell averaged values).

\item[\moc{DIRFLX}] keyword to activate the DIRFLX processing option.

\item[\moc{BURNUP}] keyword to activate the BURNUP processing option.

\item[\moc{MATLAB}] keyword to create a Matlab-m file. This is the default option.

\item[\moc{PYTHON}] keyword to create a Python-py file.

\item[\moc{ISOP}] keyword to activate the istope processing. If \moc{ISOP} is absent, only the time and time dependent $k_{\text{eff}}$ are saved.

\item[\moc{NAMISO}] name of istopes to precess. If no isitope name is specified, all the isotopes are processed.

\end{ListeDeDescription}

\eject
 % structure (dragonFMT)
\subsection{The {\tt FMAC:} module}\label{sect:FMACData}

This module is used to extract macrocoscopic cross-section data from a FMAC-M {\sc ascii} file.
Transition source information from companion particles are recovered from the FMAC-M file and written in the
output {\sc macrolib}.

\vskip 0.02cm

The calling specifications are:

\begin{DataStructure}{Structure \dstr{FMAC:}}
\dusa{MACRO}~\moc{:=}~\moc{FMAC:}~\dusa{fmac.txt}~\moc{::}~\dstr{FMAC\_data} \\
\end{DataStructure}

\noindent where
\begin{ListeDeDescription}{mmmmmmm}

\item[\dusa{MACRO}] {\tt character*12} name of the output {\sc macrolib} (type {\tt L\_MACROLIB}) object that is created by {\tt FMAC:}.

\item[\dusa{fmac.txt}] {\tt character*12} name of a {\sc ascii} file, open in read-only mode, containing FMAC-M information.

\item[\dusa{FMAC\_data}] input data structure containing specific data (see \Sect{descFMAC}).

\end{ListeDeDescription}

\subsubsection{Data input for module {\tt FMAC:}}\label{sect:descFMAC}

\vskip -0.5cm

\begin{DataStructure}{Structure \dstr{FMAC\_data}}
$[$~\moc{EDIT} \dusa{iprint}~$]$ \\
\moc{PARTICLE} \dusa{htype} \\
\moc{;}
\end{DataStructure}

\noindent where
\begin{ListeDeDescription}{mmmmmmmm}

\item[\moc{EDIT}] keyword used to set \dusa{iprint}.

\item[\dusa{iprint}] index used to control the printing in module {\tt FMAC:}. =0 for no print; =1 for minimum printing (default value).

\item[\moc{PARTICLE}] keyword used to specify the type of particle corresponding to the {\sc macrolib} (secondary state of the transition).

\item[\dusa{htype}] character*1 name of the particle. Usual names are {\tt N}: neutrons, {\tt G}: photons, {\tt B}: electrons,
{\tt C}: positrons and {\tt P}: protons.

\end{ListeDeDescription}

\eject
 % structure (dragonFMAC)
\subsection{The {\tt PSOUR:} module}\label{sect:PSOURData}

This module is used to set transition sources in a multi-particle coupled transport problem or boundary sources in Cartesian geometry. The {\tt PSOUR:} module
is currently limited to a SN discretization.

\vskip 0.08cm

The calling specifications are:

\begin{DataStructure}{Structure \dstr{PSOUR:}}
\dusa{SOURCE}~\moc{:=}~\moc{PSOUR:}~$[$~\dusa{SOURCE}~$]~\{$~\dusa{MICRO}~$|$~\dusa{MACRO}~$\}$~\dusa{TRACK}~$[$~\dusa{TRACK2}~$]$~$[$~\dusa{GEONAM}~$]$ \\
\hspace*{0.5cm}  $[[$~\dusa{FLUX}~$]]$~$[[$~\dusa{SOURCE2}~$]]$ \moc{::}~\dstr{PSOUR\_data} \\
\end{DataStructure}

\noindent where
\begin{ListeDeDescription}{mmmmmmm}

\item[\dusa{SOURCE}] {\tt character*12} name of a {\sc fixed source} (type {\tt L\_SOURCE}) object open in creation or modification
mode. This object contains a unique direct or adjoint fixed source taking into account scattering transitions from one or many companion particles and/or boundary sources.

\item[\dusa{MICRO}] {\tt character*12} name of a reference {\sc microlib} (type {\tt L\_LIBRARY}) object open in read-only mode. The information on
the embedded macrolib is used.

\item[\dusa{MACRO}] {\tt character*12} name of a reference {\sc macrolib} (type {\tt L\_MACROLIB}) object open in read-only mode.

\item[\dusa{TRACK}] {\tt character*12} name of a reference {\sc tracking} (type {\tt L\_TRACK}) object, corresponding to {\tt L\_SOURCE} object, open in read-only mode.

\item[\dusa{TRACK2}] {\tt character*12} name of a reference {\sc tracking} (type {\tt L\_TRACK}) object, corresponding to the companion particle, open in read-only mode. This object is required for sources from companion particles.

\item[\dusa{GEONAM}] {\tt character*12} name of a reference {\sc geometry} (type {\tt L\_GEOM}) object, corresponding to {\tt L\_SOURCE} object, open in read-only mode. This
object is required if and only if one of keywords \moc{ISO} or \moc{MONO} is set.

\item[\dusa{FLUX}] {\tt character*12} name of a {\sc flux} (type {\tt L\_FLUX}) object corresponding to a companion particle open in read-only mode and used to
compute transition sources. The number of {\sc flux} objects on the RHS is equal to the number of companion particles contributing to the fixed source.

\item[\dusa{SOURCE2}] {\tt character*12} name of a {\sc fixed source} (type {\tt L\_SOURCE}) object open in read-only mode. The fixed source in \dusa{SOURCE2} is
added to \dusa{SOURCE}.

\item[\dusa{PSOUR\_data}] input data structure containing specific data (see \Sect{descPSOUR}).

\end{ListeDeDescription}
\eject

\subsubsection{Data input for module {\tt PSOUR:}}\label{sect:descPSOUR}

\vskip -0.5cm

\begin{DataStructure}{Structure \dstr{PSOUR\_data}}
$[$~\moc{EDIT} \dusa{iprint}~$]$ \\
$\{$ \\
\hspace*{0.2cm}  $[[$~\moc{PARTICLE} \dusa{htype}~$]]$ \\
$|$ \\
\hspace*{0.2cm}  \moc{ISO} \dusa{nsource}\\
\hspace*{0.2cm}  \moc{INTG} $\{$\dusa{ig}($i$) \dusa{int}($i$), $i$ $|$ \dusa{int}($i$), $i$=1,\dusa{ng}$\}$ \\
\hspace*{0.2cm}  \moc{XLIM} (\dusa{xmin}($i$) \dusa{xmax}($i$), $i$=1,\dusa{nsource}) \\
\hspace*{0.2cm}  $[$\moc{YLIM} (\dusa{ymin}($i$) \dusa{ymax}($i$), $i$=1,\dusa{nsource})$]$ \\
\hspace*{0.2cm}  $[$\moc{ZLIM} (\dusa{zmin}($i$) \dusa{zmax}($i$), $i$=1,\dusa{nsource})$]$ \\
$|$ \\
\hspace*{0.2cm}  \moc{MONO} \dusa{nsource} \\
\hspace*{0.2cm}  \moc{INTG} $\{$\dusa{ig}($i$) \dusa{int}($i$), $i$ $|$ \dusa{int}($i$), $i$=1,\dusa{ng}$\}$ \\
\hspace*{0.2cm}  $\{$ \moc{X-} $|$ \moc{X+} $|$ \moc{Y-} $|$ \moc{Y+} $|$ \moc{Z-} $|$ \moc{Z+} $\}$ \\
\hspace*{0.2cm}  \moc{DIR} $\{$\dusa{mu} \dusa{eta} \dusa{xi} $|$ \dusa{polar} \dusa{azimutal}$\}$ \\
\hspace*{0.2cm}  $[$\moc{XLIM} (\dusa{xmin}($i$) \dusa{xmax}($i$), $i$=1,\dusa{nsource})$]$ \\
\hspace*{0.2cm}  $[$\moc{YLIM} (\dusa{ymin}($i$) \dusa{ymax}($i$), $i$=1,\dusa{nsource})$]$ \\
\hspace*{0.2cm}  $[$\moc{ZLIM} (\dusa{zmin}($i$) \dusa{zmax}($i$), $i$=1,\dusa{nsource})$]$ \\
$\}$\\
{\tt ;}
\end{DataStructure}

\noindent where
\begin{ListeDeDescription}{mmmmmmmm}

\item[\moc{EDIT}] keyword used to set \dusa{iprint}.

\item[\dusa{iprint}] index used to control the printing in module {\tt PSOUR:}. =0 for no print; =1 for minimum printing (default value).

\item[\moc{PARTICLE}] keyword used to specify the transition type recovered from the {\sc macrolib} (primary state of the transition). This keyword is repeated for each type of companion particles, in the same order as the \dusa{FLUX} objects on the RHS.

\item[\dusa{htype}] character*1 name of the companion particle. Usual names are {\tt N}: neutrons, {\tt G}: photons, {\tt B}: electrons,
{\tt C}: positrons and {\tt P}: protons.

\item[\moc{ISO}] keyword used to define an isotropic volumetric source. 

\item[\moc{MONO}] keyword used to define a monodirectional boundary source.

\item[\dusa{nsource}] number of sources of the choosen type.

\item[\moc{INTG}] keyword to specify the energy spectrum of the source.

\item[\dusa{ig}] index of the energy group in which the source is defined.

\item[\dusa{int}] source intensity. With \moc{ISO}, it is defined as the number of particules per volume unit per time unit. With \moc{MONO}, it is defined as the number of particles per surface unit per time unit.

\item[\moc{XLIM}] keywords to specify the \dusa{nsource} source volume (with \moc{ISO}) or surface dimension (with \moc{MONO}) along the $X$-axis.

\item[\moc{YLIM}] keywords to specify the \dusa{nsource} source volume (with \moc{ISO}) or surface dimension (with \moc{MONO}) along the $Y$-axis.

\item[\moc{ZLIM}] keywords to specify the \dusa{nsource} source volume (with \moc{ISO}) or surface dimension (with \moc{MONO}) along the $Z$-axis.  

\item[\dusa{xmin/xmax}] boundaries for each of the \dusa{nsource} volume or surface sources along respectively the $X$-axis.

\item[\dusa{ymin/ymax}] boundaries for each of the \dusa{nsource} volume or surface sources along respectively the $Y$-axis.

\item[\dusa{zmin/zmax}] boundaries for each of the \dusa{nsource} volume or surface sources along respectively the $Z$-axis.

\item[\moc{X-/X+}] keyword to specify that the \moc{MONO}-type source is located on the negative or positive $X$ surface of the Cartesian geometry.

\item[\moc{Y-/Y+}] keyword to specify that the \moc{MONO}-type source is located on the negative or positive $Y$ surface of the Cartesian geometry.

\item[\moc{Z-/Z+}] keyword to specify that the \moc{MONO}-type source is located on the negative or positive $Z$ surface of the Cartesian geometry.  

\item[\moc{DIR}] keyword to specify the orientation of the \moc{MONO}-type source.

\item[\dusa{mu/eta/xi}] direction cosines along respectively the $X$-, $Y$- and $Z$-axis.

\item[\dusa{polar/azimutal}] polar and azimutal angle in degree, with the $X$-axis as the principal axis.

\end{ListeDeDescription}

\eject
 % structure (dragonPSOUR)
\subsection{The {\tt HEAT:} module}\label{sect:HEATData}

This module is used to compute the energy and charge deposition values from many particles.

\vskip 0.02cm

The calling specifications are:

\begin{DataStructure}{Structure \dstr{HEAT:}}
\dusa{DEPOS}~\moc{:=}~\moc{HEAT:}~$[$~\dusa{DEPOS}~$]~[[$~\dusa{MACRO}~$]]$ \moc{::} \dstr{HEAT\_data} \\
\end{DataStructure}

\noindent where
\begin{ListeDeDescription}{mmmmmmm}

\item[\dusa{DEPOS}] {\tt character*12} name of a {\sc deposition} (type {\tt L\_DEPOSITION}) object containing mixture-ordered energy and charge deposition values, summed over many extended mactolibs. This object can be created by module {\tt HEAT:} or used in modification mode to
accumulate deposition values gathered from successive solutions of the Boltzmann and/or Boltzmann Fokker-Planck transport equations.

\item[\dusa{MACRO}] {\tt character*12} name of an extended {\sc macrolib} (type {\tt L\_MACROLIB}) object containing {\tt FLUX-INTG} and {\tt H-FACTOR} values.
{\tt C-FACTOR} values are also recovered if they are available. There are as many macrolibs on the RHS as particles contributing to the energy and charge deposition.

\item[\dusa{HEAT\_data}] input data structure containing specific data (see \Sect{descHEAT}).

\end{ListeDeDescription}

\subsubsection{Data input for module {\tt HEAT:}}\label{sect:descHEAT}

\vskip -0.5cm

\begin{DataStructure}{Structure \dstr{HEAT\_data}}
$[$~\moc{EDIT} \dusa{iprint}~$]$ \\
$[~\{$ \moc{POWR} \dusa{power} $|$ \moc{SOUR} \dusa{snumb} $|$ \moc{NORM} \dusa{rho}($i$), $i$=1,\dusa{nbmix} $\}~]$ \\
$[~\{$ \moc{BC} $|$ \moc{NBC} $\}~]$ \\
$[~\{$ \moc{PICKE}  {\tt >>} \dusa{esum} {\tt <<} $|$ \moc{PICKC}  {\tt >>} \dusa{csum} {\tt <<} $\}~]$ \\
;
\end{DataStructure}

\noindent where
\begin{ListeDeDescription}{mmmmmm}

\item[\moc{EDIT}] keyword used to set \dusa{iprint}.

\item[\dusa{iprint}] index used to control the printing in module {\tt HEAT:}. =0 for no print; =1 for minimum printing (default value).

\item[\moc{POWR}] keyword used to set \dusa{power}.

\item[\dusa{power}] value of the power in MW used to normalize the flux. By default, the flux is not normalized.

\item[\moc{SOUR}] keyword used to set \dusa{snumb}. Fixed source information (record {\tt FIXE}) must be available in the first extended macrolib \dusa{MACRO}.

\item[\dusa{snumb}] number of source particles used to normalize the flux. By default, the flux is not normalized.

\item[\moc{NORM}] keyword used to obtain the energy deposition per voxels by normalizing the flux by the total number of source particles and the voxel material density. The output units are MeV/g$\times$cm$^{N}$ per particles, where $N$ is the geometry dimension.

\item[\dusa{rho}] densities in g/cm$^{3}$ of the \dusa{nbmix} material defined in calculations.

\item[\moc{BC}]  keyword used to take into account the contribution of the particules falling below the cutoff energy to the charge and energy deposition calculations using method from Morel\cite{morel1996} (by default).

\item[\moc{NBC}] keyword used to disable the contribution of the particules falling below the cutoff energy to the charge and energy deposition calculations.

\item[\moc{PICKE}]  keyword used to recover the total energy deposition value (MeV/cm$^{3}$/s) in a CLE-2000 variable.

\item[\dusa{esum}] \texttt{character*12} CLE-2000 variable name in which the extracted total energy deposition value will be placed.

\item[\moc{PICKC}]  keyword used to recover the total charge deposition value (electrons/cm$^{3}$/s) in a CLE-2000 variable.

\item[\dusa{csum}] \texttt{character*12} CLE-2000 variable name in which the total charge deposition value will be placed.

\end{ListeDeDescription}
\clearpage
 % structure (dragonHEAT)
\subsection{The {\tt BREF:} module}\label{sect:BREFData}

This module compute a {\sc macrolib} for a 1D {\sl equivalent reflector} based on various models. One or many fine-group and
fine-mesh reference calculations (using the $S_n$ method) are first performed so as to produce coarse-group and
coarse-mesh Macrolibs stored within output {\sc edition} objects (\dusa{EDIT\_SN}), compatible with the selected
reflector model. Module {\tt BREF:} recovers the $S_n$ {\sc geometry}, depicted in Fig.~\ref{fig:bref}, from object \dusa{GEOM\_SN}.
The $S_n$ {\sc geometry} must have a reflective ({\tt REFL} or {\tt ALBE 1.0}) boundary condition on its left ({\tt X-}) boundary.
Module {\tt BREF:} recovers the following information from each \dusa{EDIT\_SN} object:
\begin{itemize}
\item Coarse group surfacic fluxes between the nodes using averaged flux values recovered into {\sl gap} volumes, corresponding to
tiny meshes in the reflector zones.
\item Coarse group net currents between the nodes obtained from a balance relation, assuming reflection on the left
boundary.
\item Averaged macroscopic cross sections and diffusion coefficients within the {\sl no-gap} homogenized nodes.
\end{itemize}

\begin{figure}[h!]
\begin{center} 
\epsfxsize=11cm
\centerline{ \epsffile{bref_geom.eps}}
\parbox{14cm}{\caption{Definition of geometries used by the {\tt BREF:} module}
\label{fig:bref}} 
\end{center} 
\end{figure}

At output, a {\sc macrolib} object is produced with equivalent macroscopic cross sections, diffusion coefficients, discontinuity factors and
albedos. A verification calculation performed over the {\tt BREF:} geometry is depicted in Fig.~\ref{fig:brefVerif},

\begin{figure}[h!]
\begin{center} 
\epsfxsize=15cm
\centerline{ \epsffile{ErmBeavrsPwrRefl.eps}}
\parbox{14cm}{\caption{Equivalent ERM-NEM reflector for the BEAVRS benchmark}
\label{fig:brefVerif}} 
\end{center} 
\end{figure}

\vskip 0.02cm

Nodal expansion base functions are used to represent the flux with the {\tt DF-NEM} and {\tt ERM-NEM}
methods. By default, polynomials defined over $(-0.5,0.5)$ are used as base functions:\cite{nestle}
\begin{eqnarray}
\nonumber P_0(u)\negthinspace &=&\negthinspace 1\\
\nonumber P_1(u)\negthinspace &=&\negthinspace u\\
\nonumber P_2(u)\negthinspace &=&\negthinspace 3u^2-{1\over 4}\\
\nonumber P_3(u)\negthinspace &=&\negthinspace \left( u^2-{1\over 4}\right)u\\
P_4(u)\negthinspace &=&\negthinspace \left( u^2-{1\over 4}\right)\left( u^2-{1\over 20}\right)
\end{eqnarray}
There is the option of using hyperbolic functions in some energy groups:
\begin{eqnarray}
\nonumber P_3(u)\negthinspace &=&\negthinspace \sinh(\zeta_g u)\\
P_4(u)\negthinspace &=&\negthinspace \cosh(\zeta_g u)-{2\over \zeta}\, \sinh(\zeta_g/2)
\end{eqnarray}
\noindent where
\begin{equation}
\zeta_g=\Delta x\sqrt{\Sigma_{{\rm r},g} \over D_g}
\end{equation}
\noindent where $\Delta x$, $\Sigma_{{\rm r},g}$ and $D_g$ are the node width (cm), the macroscopic removal cross section (cm$^{-1}$)
and the diffusion coefficient (cm) in group $g$, respectively.

\vskip 0.02cm

Other equivalence techniques are known as {\tt DF-ANM} and {\tt ERM-ANM}. These techniques are similar to {\tt DF-NEM} and
{\tt ERM-NEM} where the nodal expansion method (NEM) is replaced by an analytic solution of the $G-$group diffusion equation.

\goodbreak

The calling specifications are:

\begin{DataStructure}{Structure \dstr{BREF:}}
\dusa{GEOM} \dusa{MACRO}~\moc{:=}~\moc{BREF:}~\dusa{GEOM\_SN} $[[$~\dusa{EDIT\_SN}~$]]$ \moc{::} \dstr{BREF\_data} \\
\end{DataStructure}

\noindent where
\begin{ListeDeDescription}{mmmmmmm}

\item[\dusa{GEOM}] {\tt character*12} name of the nodal {\sc geometry} (type {\tt L\_GEOM}) object open creation mode. This geometry can be used for
performing a verification calculation over the 1D nodal geometry.

\item[\dusa{MACRO}] {\tt character*12} name of the nodal {\sc macrolib} (type {\tt L\_MACROLIB}) object open in creation mode.

\item[\dusa{GEOM\_SN}] {\tt character*12} name of the $S_n$ {\sc geometry} (type {\tt L\_GEOM}) object open read-only mode.

\item[\dusa{EDIT\_SN}] {\tt character*12} name of one or many $S_n$ {\sc edition} (type {\tt L\_EDIT}) object, containing coarse-group and
coarse-mesh {\sc macrolib} for the edition {\sc macro-geometry} with gaps corresponding to one or many reference $S_n$ calculations.

\item[\dusa{BREF\_data}] input data structure containing specific data (see \Sect{descBREF}).

\end{ListeDeDescription}
\clearpage

\subsubsection{Data input for module {\tt BREF:}}\label{sect:descBREF}

\begin{DataStructure}{Structure \dstr{BREF\_data}}
$[$~\moc{EDIT} \dusa{iprint}~$]$ \\
$[$~\moc{HYPE} \dusa{igmax}~$]$ \\
\moc{MIX} $[[$ \dusa{imix} $]]$ \moc{GAP} $[[$ \dusa{igap} $]]$ \\
\moc{MODE} \dusa{hmod} \\
$[~\{$~\moc{SPH}~$|$~\moc{NOSP}~$\}~]~[~\{$~\moc{ALBE}~$|$~\moc{NOAL}~$\}~]$ \\
$[$~\moc{NGET}~$[$~(\dusa{adf}($g$), $g$=1,$N_g$) $]~]$ \\
{\tt ;}
\end{DataStructure}

\noindent where
\begin{ListeDeDescription}{mmmmmmmm}

\item[\moc{EDIT}] keyword used to set \dusa{iprint}.

\item[\dusa{iprint}] index used to control the printing in module {\tt BREF:}. =0 for no print; =1 for minimum printing (default value).

\item[\moc{HYPE}] keyword used to specify the type of nodal expansion base functions with the {\tt DF-NEM} and {\tt ERM-NEM}
methods. By default, polynomial base functions are used in all energy groups.

\item[\dusa{igmax}] hyperbolic base functions are used for coarse energy groups with indices $\ge$ \dusa{igmax}.

\item[\moc{MIX}] keyword used to set the nodal mixture indices \dusa{imix}.

\item[\dusa{imix}] index of a mixture index within object \dusa{EDIT\_SN} corresponding to a node. In Fig.~\ref{fig:bref}, this data
is set as {\tt MIX 1 3} for {\tt DF-NEM}, {\tt DF-ANM}, {\tt ERM-NEM} and {\tt ERM-ANM} geometries and to {\tt MIX 0 3} for {\tt LEFEBVRE-LEB} and {\tt KOEBKE}
geometries.

\item[\moc{GAP}] keyword used to set the gap mixture indices \dusa{igap} where the surfacic fluxes are recovered.

\item[\dusa{igap}] index of a mixture index within object \dusa{EDIT\_SN} corresponding to a gap. In Fig.~\ref{fig:bref}, this data
is set as {\tt GAP 2 4} for {\tt DF-NEM}, {\tt DF-ANM}, {\tt ERM-NEM} and {\tt ERM-ANM} geometries and to {\tt GAP 2 0} for {\tt LEFEBVRE-LEB} and {\tt KOEBKE}
geometries.

\item[\moc{MODE}] keyword used to select a specific reflector equivalence model.

\item[\dusa{hmod}] character*12 name of the reflector equivalence model used. The following options are available:
\begin{description}
\item[{\tt LEFEBVRE-LEB}:] Lefebvre-Lebigot equivalence model. Two \dusa{EDIT\_SN} objects are expected at RHS.\cite{LLB,Frohlicher}
\item[{\tt KOEBKE}:] Koebke equivalence model. Two \dusa{EDIT\_SN} objects are expected at RHS.\cite{Koebke,Frohlicher}
\item[{\tt DF-NEM}:] Pure discontinuity-factor model based on the nodal expansion method (NEM). Only one \dusa{EDIT\_SN} object is expected at RHS.
\item[{\tt DF-ANM}:] Pure discontinuity-factor model based on the analytic nodal method (ANM). Only one \dusa{EDIT\_SN} object is expected at RHS.
\item[{\tt ERM-NEM}:] {\sl Equivalent reflector model} based on {\sl matrix discontinuity factors} and nodal expansion method (NEM).
Two or more \dusa{EDIT\_SN} objects are expected at RHS.
\item[{\tt ERM-ANM}:] {\sl Equivalent reflector model} based on {\sl matrix discontinuity factors} and analytic nodal method (ANM).
Two or more \dusa{EDIT\_SN} objects are expected at RHS.
\end{description}

\item[\moc{SPH}] keyword used to include discontinuity factors within cross sections and diffusion coefficients. This option is not available
with models {\tt ERM-NEM} and {\tt ERM-ANM}.

\item[\moc{NOSP}] keyword used to store discontinuity factors in {\sc macrolib} \dusa{MACRO} (default option).

\item[\moc{ALBE}] keyword used to compute an equivalent albedo in each coarse energy group with {\tt DF-NEM}, {\tt DF-ANM}, {\tt ERM-NEM} and {\tt ERM-ANM} models (default option).

\item[\moc{NOAL}] keyword used to desactivate equivalent albedo calculation.

\item[\moc{NGET}] keyword used to force the value of the fuel assembly discontinuity factor at the fuel-reflector interface, as used
by the NGET normalization. By default, this value is not modified by NGET normalization.

\item[\dusa{adf}] value of the assembly discontinuity factor (ADF) on the fuel-reflector interface in group $g\le N_g$. If keyword \moc{NGET} is set and
\dusa{adf} vakues are not given, the ADF values are recovered from \dusa{EDIT\_SN}.

\end{ListeDeDescription}

\eject
 % structure (dragonBREF)
\subsection{The \tt{PSP:} module}\label{sect:PSPData}

The \moc{PSP:} module is used to generate a graphical file in a PostScript ASCII format for a DRAGON 2-D
geometry which can be analyzed using the \moc{EXCELT:} or \moc{NXT:} tracking module (see Sections.~\ref{sect:EXCELLData}
and~\ref{sect:NXTData}). The module \moc{PSP:} 
is based on the PSPLOT Fortran library from Nova Southeastern University.\cite{PSPLOT} Since only a few
PSPLOT routines were required and because additional PostScript routine not present in the original package were 
needed, the routines have been completely readapted to DRAGON. These routines are no longer machine dependent.
The PostScript files generated by DRAGON can be viewed by any PostScript viewer, such as Ghostview\cite{GHOSTVIEW}
or sent to a printer compatible with this language. In DRAGON the \moc{PSP:} module is activated using the following 
list of commands:

\begin{DataStructure}{Structure \dstr{PSP:}}
\dusa{PSGEO} \moc{:=} \moc{PSP:} \dusa{PSGEO} $\{$ \dusa{GEONAM} $|$ \dusa{TRKNAM} $\}$ $[$ \dusa{FLUNAM} $]$ \moc{::} \dstr{descpsp}
\end{DataStructure}

\noindent  where
\begin{ListeDeDescription}{mmmmmmm}

\item[\dusa{PSGEO}] {\tt character*12} name of the file that will contain the graphical description in a POSTSCRIPT 
format. This file must have a sequential ASCII format. 

\item[\dusa{GEONAM}] {\tt character*12} name of a read-only {\sc geometry} (see Section \Sect{GEOData}). 

\item[\dusa{TRKNAM}] {\tt character*12} name of an NXT or EXCELL type read-only {\sc tracking} (see Sections.~\ref{sect:EXCELLData}
and~\ref{sect:NXTData}).

\item[\dusa{FLUNAM}] {\tt character*12} name of an optional read-only {\sc fluxunk} (see \Sect{FLUData}). It is required only 
if a flux mapping plot is requested. 

\item[\dstr{descpsp}] structure containing the input data to this module (see \Sect{PSPdesc}).

\end{ListeDeDescription}

\subsubsection{Data input for module {\tt PSP:}}\label{sect:PSPdesc}

\begin{DataStructure}{Structure \dstr{descpsp}}
$[$ \moc{EDIT} \dusa{iprint} $]$ \\
$[$ \moc{FILL} $\{$ \moc{NONE} $|$ \moc{GRAY} $|$ \moc{RGB} $|$ \moc{CMYK} $|$ \moc{HSB} $\}~[$ \moc{NOCONTOUR} $]~]$ \\
$[$ \moc{TYPE} $\{$ \moc{REGION} $|$ \moc{MIXTURE} $|$ \moc{FLUX} $|$ \moc{MGFLUX} $\}~]$
\end{DataStructure}

\noindent where

\begin{ListeDeDescription}{mmmmmmmm}

\item[\moc{EDIT}] keyword used to modify the print level \dusa{iprint}. 

\item[\dusa{iprint}] index used to control the printing in this module. It must be set to 0 if no printing on the output 
file is required. 

\item[\moc{FILL}] keyword to specify the drawing options. 

\item[\moc{NONE}] keyword to specify that only region contour are to be drawn. 

\item[\moc{GRAY}] keyword to specify that the regions will be filled with various levels of gray. 

\item[\moc{RGB}] keyword to specify that the regions will be filled with various colors taken using the RGB color 
scheme. 

\item[\moc{CMYK}] keyword to specify that the regions will be filled with various colors taken using the CMYK 
color scheme. 

\item[\moc{HSB}] keyword to specify that the regions will be filled with various colors taken using the HSB color 
scheme. This is the default option. 

\item[\moc{NOCONTOUR}] keyword to specify that the contour lines delimiting each region will not be drawn. 

\item[\moc{TYPE}] keyword to specify the type of graphics generated. 

\item[\moc{REGION}] keyword to specify that different colors or gray levels will be associated with each region. This 
is the default option. 

\item[\moc{MIXTURE}] keyword to specify that different colors or gray levels will be associated with each mixture. 

\item[\moc{FLUX}] keyword to specify that the group integrated flux is to be drawn. 

\item[\moc{MGFLUX}] keyword to specify that the group flux is to be drawn.

\end{ListeDeDescription}

\eject
 % structure (dragonPSP)
\subsection{The {\tt CLM:} module}\label{sect:CLMData}

The {\tt CLM:} module is called to mix and redistribute several liquid mixtures containing identical isotopes contents (different isotopic densities). The module can also change the density of any isotope in the mixture. The calling specifications are:

\begin{DataStructure}{Structure \dstr{CLM:}}
\dusa{MICLIB} \moc{:=} \moc{CLM:} \dusa{MICLIB}   \moc{::} \dstr{descclm}
\end{DataStructure}

\noindent
where

\begin{ListeDeDescription}{mmmmmmmm}

\item[\dusa{MICLIB}] {\tt character*12} name of the \dds{microlib} that will
contain the microscopic and macroscopic cross sections updated by the
module. 

\item[\dstr{descclm}] structure describing the clm options.

\end{ListeDeDescription}

\
\subsubsection{Data input for module {\tt CLM:}}\label{sect:descclm}

\begin{DataStructure}{Structure \dstr{descclm}}
$[$ \moc{EDIT} \dusa{iprint} $]$ \\
\moc{MIXCLM} (idclm(ii),ii=1,nclm) \\
$[$ $\{$ \moc{ADDI} $|$ \moc{SETI} $\}$ $\{$ \moc{ABS} $|$ \moc{REL} $\}$ (isot(ii) dens(ii),ii=1,niso) $]$ \\
{\tt ;}
\end{DataStructure}
\noindent
where

\begin{ListeDeDescription}{mmmmmmmm}

\item[\moc{EDIT}] keyword used to modify the print level \dusa{iprint}.

\item[\dusa{iprint}] index used to control the printing of this module. The
amount of output produced by this module will vary substantially
depending on the print level specified. 

\item[\moc{MIXCLM}] keyword to specify that several liquid mixtures will be combined to replace the original mixtures. This combination will take into account the volume occupied by each mixture.

\item[\moc{idclm}] mixture number to combine.

\item[\moc{ADDI}] to add a isotope concentration to that of this  isotope in the combined mixture, 
\item[\moc{SETI}] to set the concentration of an isotope in the combined mixture to a fix value.

\item[\moc{ABS}] isotopic concentration in $10^{24}$ cm$^{-3}$.

\item[\moc{REL}] relative concentration with respect with the concentration of an isotope in the combined mixture.

\item[\moc{isot}] reference name of isotope to process.

\item[\moc{dens}] absolute or relative density for this isotope.


\end{ListeDeDescription}
\eject
 % structure (dragonCLM)

\section{THE RESONANCE SELF-SHIELDING MODULES}\label{sect:SSModuleInput}

A few DRAGON modules are dedicated to include resonance self-shielding effects in \dds{microlib} and
\dds{macrolib} structures. Module \moc{VDG:} compute self-shielding discrepancy between a reference
calculation based on the Autosecol method in module \moc{AUTO:} and more approximate calculations.\cite{autosecol,vdg}

\subsection{The {\tt SHI:} module}\label{sect:SHIData}

The {\tt SHI:} module perform self-shielding calculations in DRAGON, using
the generalized Stamm'ler method.\cite{SHIBA}  This approach is based on an heterogeneous-homogeneous equivalence principle. In this case, an {\sl equivalent dilution parameter} $\sigma_{{\rm e},g}$ is computed for each resonant isotope, in each resonant region and
each resonant energy group $g$. This dilution parameter is used to interpolate pretabulated effective cross sections for the infinite homogeneous medium, previously obtained with the {\sl flux calculator} of the {\tt GROUPR} module in code NJOY.\cite{njoy2010}
Each resonant isotope, identified as such by the \dusa{inrs}
parameter defined in \Sect{LIBData}, is to be recalculated. The general format of
the data for this module is:

\begin{DataStructure}{Structure \dstr{SHI:}}
\dusa{MICLIB} \moc{:=} \moc{SHI:} $\{$ \dusa{MICLIB} $|$ \dusa{OLDLIB} $\}$ 
\dusa{TRKNAM} $[$ \dusa{TRKFIL} $]$ \moc{::} \dstr{descshi}
\end{DataStructure}

\noindent
where

\begin{ListeDeDescription}{mmmmmmmm}

\item[\dusa{MICLIB}] {\tt character*12} name of the \dds{microlib} that will
contain the microscopic and macroscopic cross sections updated by the
self-shielding module. If
\dusa{MICLIB} appears on both LHS and RHS, it is updated; otherwise, the
internal library \dusa{OLDLIB} is copied into
\dusa{MICLIB} and \dusa{MICLIB} is updated.

\item[\dusa{OLDLIB}] {\tt character*12} name of a read-only \dds{microlib} 
that is copied into \dusa{MICLIB}.

\item[\dusa{TRKNAM}] {\tt character*12} name of the required \dds{tracking}
data structure.

\item[\dusa{TRKFIL}] {\tt character*12} name of the sequential binary tracking
file used to store the tracks lengths. This file is given if and only if it was
required in the previous tracking module call (see \Sect{TRKData}).

\item[\dstr{descshi}] structure describing the self-shielding options.

\end{ListeDeDescription}

Each time the \moc{SHI:} module is called, a sub-directory is updated in the
\dds{microlib} data structure to hold the last values defined in the
\dstr{descshi} structure. The next time this module is called,
these values will be used as floating defaults.

\subsubsection{Data input for module {\tt SHI:}}\label{sect:descshi}

\begin{DataStructure}{Structure \dstr{descshi}}
$[$ \moc{EDIT} \dusa{iprint} $]$ \\
$[$ \moc{GRMIN}  \dusa{lgrmin} $]~~[$ \moc{GRMAX}  \dusa{lgrmax} $]$ \\
$[$ \moc{MXIT} \dusa{imxit} $]~~[$ \moc{EPS}  \dusa{valeps} $]$  \\
$[~\{$ \moc{LJ} $|$ \moc{NOLJ} $\}~]$ $[~\{$ \moc{GC} $|$ \moc{NOGC} $\}$
$[~\{$ \moc{TRAN} $|$ \moc{NOTR} $\}~]$
$[$ \moc{LEVEL} \dusa{ilev} $]$ \\
$[~\{$ \moc{PIJ} $|$ \moc{ARM} $\}~]$ \\
{\tt ;}
\end{DataStructure}

\noindent  where

\begin{ListeDeDescription}{mmmmmmmm}

\item[\moc{EDIT}] keyword used to modify the print level \dusa{iprint}.

\item[\dusa{iprint}] index used to control the printing of this module. The
amount of output produced by this tracking module will vary substantially
depending on the print level specified. 

\item[\moc{GRMIN}] keyword to specify the minimum group number considered
during the self-shielding process.

\item[\dusa{lgrmin}] first group number considered during the
self-shielding process. By default, \dusa{lgrmin} is set to the first group
number containing self-shielding data in the library.

\item[\moc{GRMAX}]  keyword to specify the maximum group number considered
during the self-shielding process.

\item[\dusa{lgrmax}] last group number considered during the self-shielding
process. By default, \dusa{lgrmax} is set is set to the last group
number containing self-shielding data in the library.

\item[\moc{MXIT}]  keyword to specify the maximum number of iterations during
the self-shielding process.

\item[\dusa{imxit}] the maximum number of iterations. The default is
\dusa{imxit}=20.

\item[\moc{EPS}] keyword to specify the convergence criterion for the
self-shielding iteration.

\item[\dusa{valeps}] the convergence criterion for the self-shielding iteration.
By default, \dusa{valeps}=$1.0\times 10^{-4}$.

\item[\moc{LJ}] keyword to activate the Livolant and Jeanpierre
normalization scheme which modifies the self-shielded averaged neutron fluxes in
heterogeneous geometries. By default the Livolant and Jeanpierre
normalization scheme is not activated.

\item[\moc{NOLJ}] keyword to deactivate the Livolant and Jeanpierre
normalization scheme which modifies the self-shielded averaged neutron fluxes in
heterogeneous geometries. This is the default option.

\item[\moc{GC}] keyword to activate the Goldstein-Cohen approximation in
cases where Goldstein-Cohen parameters are stored on the internal library. These
parameters may not be available with some libraries (e.g., {\tt APLIB1}, {\tt
APLIB2} or MATXS-type libraries). The Goldstein-Cohen parameters can always be
imposed using the \moc{IRSET} keyword of the \moc{LIB:} module (see
\Sect{LIBData}). This is the default option.

\item[\moc{NOGC}] keyword to deactivate the Goldstein-Cohen approximation in
cases where Goldstein-Cohen parameters are stored on the internal library.

\item[\moc{TRAN}] keyword to activate the transport correction option for
self-shielding calculations (see \moc{CTRA} in \Sectand{MACData}{LIBData}). This is the default option.

\item[\moc{NOTR}] keyword to deactivate the transport correction option for
self-shielding calculations (see \moc{CTRA} in \Sectand{MACData}{LIBData}).

\item[\moc{LEVEL}] keyword to specify the self-shielding model.

\item[\dusa{ilev}] $=0$: original Stamm'ler model (without distributed effects);
 $=1$: use the Nordheim (PIC) distributed self-shielding model\cite{toronto04};
 $=2$: use both Nordheim (PIC) distributed self-shielding model and Riemann integration
 method\cite{hasan}. By default, \dusa{ilev}$\,=0$.

\item[\moc{PIJ}] keyword to specify the use of complete collision
probabilities in the self-shielding calculations of {\tt SHI:}.
This is the default option for \moc{EXCELT:} and \moc{SYBILT:} trackings.
This option is not available for \moc{MCCGT:} trackings.

\item[\moc{ARM}] keyword to specify the use of iterative flux techniques
in the self-shielding calculations of {\tt SHI:}.
This is the default option for \moc{MCCGT:} trackings.

\end{ListeDeDescription}
\eject
 % structure (dragonS)
\subsection{The \moc{TONE:} module}\label{sect:TONEData}

The \moc{TONE:} module perform self-shielding calculations in DRAGON, using
the Tone's method.\cite{tone}  This approach is based on an heterogeneous-homogeneous equivalence principle. In this case, an {\sl equivalent dilution parameter} $\sigma_{{\rm e},g}$ is computed for each resonant isotope, in each resonant region and
each resonant energy group $g$. This dilution parameter is used to interpolate pretabulated effective cross sections for the infinite homogeneous medium, previously obtained with the {\sl flux calculator} of the {\tt GROUPR} module in code NJOY.\cite{njoy2010}
Each resonant isotope, identified as such by the \dusa{inrs}
parameter defined in \Sect{LIBData}, is to be recalculated. The general format of
the data for this module is:

\begin{DataStructure}{Structure \dstr{TONE:}}
\dusa{MICLIB} \moc{:=} \moc{TONE:} $\{$ \dusa{MICLIB} $|$ \dusa{OLDLIB} $\}$ 
\dusa{TRKNAM} $[$ \dusa{TRKFIL} $]$ \moc{::} \dstr{desctone}
\end{DataStructure}

\noindent where

\begin{ListeDeDescription}{mmmmmmmm}

\item[\dusa{MICLIB}] {\tt character*12} name of the \dds{microlib} that will
contain the microscopic and macroscopic cross sections updated by the
self-shielding module. If
\dusa{MICLIB} appears on both LHS and RHS, it is updated; otherwise, the
internal library \dusa{OLDLIB} is copied into
\dusa{MICLIB} and \dusa{MICLIB} is updated.

\item[\dusa{OLDLIB}] {\tt character*12} name of a read-only \dds{microlib} 
that is copied into \dusa{MICLIB}.

\item[\dusa{TRKNAM}] {\tt character*12} name of the required \dds{tracking}
data structure.

\item[\dusa{TRKFIL}] {\tt character*12} name of the sequential binary tracking
file used to store the tracks lengths. This file is given if and only if it was
required in the previous tracking module call (see \Sect{TRKData}).

\item[\dstr{desctone}] structure describing the self-shielding options.

\end{ListeDeDescription}

Each time the \moc{TONE:} module is called, a sub-directory is updated in the
\dds{microlib} data structure to hold the last values defined in the
\dstr{desctone} structure. The next time this module is called,
these values will be used as floating defaults.

\subsubsection{Data input for module \moc{TONE:}}\label{sect:desctone}

\begin{DataStructure}{Structure \dstr{desctone}}
$[$ \moc{EDIT} \dusa{iprint} $]$ \\
$[$ \moc{GRMIN}  \dusa{lgrmin} $]~~[$ \moc{GRMAX}  \dusa{lgrmax} $]$ \\
$[$ \moc{MXIT} \dusa{imxit} $]~~[$ \moc{EPS}  \dusa{valeps} $]$  \\
$[~\{$ \moc{SPH} $|$ \moc{NOSP} $\}~]$
$[~\{$ \moc{TRAN} $|$ \moc{NOTR} $\}~]$
$[~\{$ \moc{PIJ} $|$ \moc{ARM} $\}~]$ \\
{\tt ;}
\end{DataStructure}

\noindent 
where

\begin{ListeDeDescription}{mmmmmmmm}

\item[\moc{EDIT}] keyword used to modify the print level \dusa{iprint}.

\item[\dusa{iprint}] index used to control the printing of this module. The
amount of output produced by this tracking module will vary substantially
depending on the print level specified. 

\item[\moc{GRMIN}] keyword to specify the minimum group number considered
during the self-shielding process.

\item[\dusa{lgrmin}] first group number considered during the
self-shielding process. By default, \dusa{lgrmin} is set to the first group
number containing self-shielding data in the library.

\item[\moc{GRMAX}]  keyword to specify the maximum group number considered
during the self-shielding process.

\item[\dusa{lgrmax}] last group number considered during the self-shielding
process. By default, \dusa{lgrmax} is set is set to the last group
number containing self-shielding data in the library.

\item[\moc{MXIT}]  keyword to specify the maximum number of iterations during
the self-shielding process.

\item[\dusa{imxit}] the maximum number of iterations. The default is
\dusa{imxit}=20.

\item[\moc{EPS}] keyword to specify the convergence criterion for the
self-shielding iteration.

\item[\dusa{valeps}] the convergence criterion for the self-shielding iteration.
By default, \dusa{valeps}=$1.0\times 10^{-4}$.

\item[\moc{SPH}] keyword to activate the SPH equivalence scheme which
modifies the self-shielded averaged neutron fluxes in
heterogeneous geometries. This is the default option.

\item[\moc{NOSP}] keyword to deactivate the SPH equivalence scheme which
modifies the self-shielded averaged neutron fluxes in
heterogeneous geometries.

\item[\moc{TRAN}] keyword to activate the transport correction option for
self-shielding calculations (see \moc{CTRA} in \Sectand{MACData}{LIBData}). This is the default option.

\item[\moc{NOTR}] keyword to deactivate the transport correction option for
self-shielding calculations (see \moc{CTRA} in \Sectand{MACData}{LIBData}).

\item[\moc{PIJ}] keyword to specify the use of complete collision
probabilities in the self-shielding calculations of \moc{TONE:}.
This is the default option for \moc{EXCELT:} and \moc{SYBILT:} trackings.
This option is not available for \moc{MCCGT:} trackings.

\item[\moc{ARM}] keyword to specify the use of iterative flux techniques
in the self-shielding calculations of \moc{TONE:}.
This is the default option for \moc{MCCGT:} trackings.

\end{ListeDeDescription}
\eject
 % structure (dragonT)
\subsection{The {\tt USS:} module}\label{sect:USSData}

The universal self-shielding module in DRAGON, called {\tt USS:}, allows the
correction of the microscopic cross sections to take into account the
self-shielding effects related to the resonant isotopes. These isotopes are
identified as such by the \dusa{inrs}
parameter, as defined in \Sect{LIBData}. The universal
self-shielding module is based on the following models:

\begin{itemize}
\item The Livolant-Jeanpierre flux factorization and approximations are used to
uncouple the self-shielding treatment from the main flux calculation;
\item The resonant cross sections are represented using probability
tables computed in the \moc{LIB:} module (the keyword \moc{SUBG} or \moc{PTSL} {\sl must} be
used). Two approaches can be used to compute the probability tables:
\begin{enumerate}
\item Physical probability tables can be computed using a RMS approach similar
to the one used in Wims-7 and Helios.\cite{subg} In this case, the slowing-down operator of
each resonant isotope is represented as a pure ST\cite{st}, ST/IR or ST/WR approximation;
\item Mathematical probability tables\cite{pt} and slowing-down correlated weight matrices
can be computed in selected energy groups using the {\sl Ribon extended} approach.\cite{nse2004} In this case,
an elastic slowing-down model is used and a mutual self-shielding model is
available.
\end{enumerate}
\item The resonant fluxes are computed for each band of the probability tables
using a subgroup method if \moc{SUBG}, \moc{PT}, \moc{PTMC} or \moc{PTSL} keyword is set in module \moc{LIB:};
\item The resonance spectrum expansion (RSE) method is used if \moc{RSE} keyword is set in module \moc{LIB:};
\item The flux can be solved using collision probabilities, or using {\sl any}
flux solution technique for which a tracking module is available;
\item The resonant isotopes are computed one-a-time, starting from the isotopes
with the lower values of index \dusa{inrs}, as defined in \Sect{LIBData}; If
many isotopes have the same value of \dusa{inrs}, the isotope with the greatest
number of resonant nuclides is self-shielded first. One or many outer iterations
can be performed;
\item The distributed self-shielded effect is automatically taken into account
if different mixture indices are assigned to different regions inside the
resonant part of the cell. The rim effect can be computed by dividing the fuel
into "onion rings" and by assigning different mixture indices to them. 
\item A SPH (superhomog\'en\'eisation) equivalence is performed to correct the
self-shielded cross sections from the non-linear effects related to the
heterogeneity of the geometry.
\end{itemize}

\vskip 0.2cm

The general format of the data for this module is:

\begin{DataStructure}{Structure \dstr{USS:}}
\dusa{MICLIB} \moc{:=} \moc{USS:} \dusa{MICLIB\_SG} $[$ \dusa{MICLIB} $]$
\dusa{TRKNAM} $[$ \dusa{TRKFIL} $]$ \moc{::} \dstr{descuss}
\end{DataStructure}

\noindent
where

\begin{ListeDeDescription}{mmmmmmmm}

\item[\dusa{MICLIB}] {\tt character*12} name of the \dds{microlib} that will
contain the microscopic and macroscopic cross sections updated by the
self-shielding module. If
\dusa{MICLIB} appears on both LHS and RHS, it is updated; otherwise,
\dusa{MICLIB} is created.

\item[\dusa{MICLIB\_SG}] {\tt character*12} name of the \dds{microlib} builded
by module \moc{LIB:} and containing probability table information (the keyword \moc{SUBG} {\sl must} be
used in module {\tt LIB:}).

\item[\dusa{TRKNAM}] {\tt character*12} name of the required \dds{tracking}
data structure.

\item[\dusa{TRKFIL}] {\tt character*12} name of the sequential binary tracking
file used to store the tracks lengths. This file is given if and only if it was
required in the previous tracking module call (see \Sect{TRKData}).

\item[\dstr{descuss}] structure describing the self-shielding options.

\end{ListeDeDescription}

Each time the \moc{USS:} module is called, a sub-directory is updated in the
\dds{microlib} data structure to hold the last values defined in the
\dstr{descuss} structure. The next time this module is called,
these values will be used as floating defaults.

\subsubsection{Data input for module {\tt USS:}}\label{sect:descuss}

\begin{DataStructure}{Structure \dstr{descuss}}
$[$ \moc{EDIT} \dusa{iprint} $]$ \\
$[$ \moc{GRMIN}  \dusa{lgrmin} $]~~[$ \moc{GRMAX}  \dusa{lgrmax} $]$~~
$[$ \moc{PASS} \dusa{ipass} $]~~[$ \moc{NOCO} $]~~[$ \moc{NOSP} $]$~~$[$ $\{$ \moc{TRAN} $|$
\moc{NOTR} $\}$ $]$ \\ 
$[$ $\{$ \moc{PIJ} $|$ \moc{ARM} $\}$ $]$ \\
$[$ \moc{MAXST} \dusa{imax} $]~[$ \moc{FLAT} $]$ \\
$[$ \moc{CALC} \\
~~~~$[[$ \moc{REGI} \dusa{suffix} $[[$ \dusa{isot} $\{$ \moc{ALL} $|$
(\dusa{imix}(i),i=1,\dusa{nmix}) $\}$ $]]$ \\
~~~~$]]$ \\
\moc{ENDC} $]$ \\
{\tt ;}
\end{DataStructure}

\noindent where

\begin{ListeDeDescription}{mmmmmmmm}

\item[\moc{EDIT}] keyword used to modify the print level \dusa{iprint}.

\item[\dusa{iprint}] index used to control the printing of this module. The
amount of output produced by this tracking module will vary substantially
depending on the print level specified. 

\item[\moc{GRMIN}] keyword to specify the minimum group number considered
during the self-shielding process.

\item[\dusa{lgrmin}] first group number considered during the
self-shielding process. By default, \dusa{lgrmin} is set to the first group
number containing self-shielding data in the library.

\item[\moc{GRMAX}]  keyword to specify the maximum group number considered
during the self-shielding process.

\item[\dusa{lgrmax}] last group number considered during the self-shielding
process. By default, \dusa{lgrmax} is set is set to the last group
number containing self-shielding data in the library.

\item[\moc{PASS}]  keyword to specify the number of outer iterations during
the self-shielding process.

\item[\dusa{ipass}] the number of iterations. The default is
\dusa{ipass} $=2$ if \dusa{MICLIB} is created.

\item[\moc{NOCO}]  keyword to ignore the directives set by {\tt LIB} concerning
the mutual resonance shielding model. This keyword has the effect to replace the
mutual resonance shielding model in the subgroup projection method (SPM) by a full
correlation approximation similar
to the technique used in the ECCO code. This keyword can be used to avoid the message
\begin{verbatim}
USSIST: UNABLE TO FIND CORRELATED ISOTOPE ************.
\end{verbatim}
\noindent that appears with the SPM if the correlated weights matrices are missing in
the microlib.

\item[\moc{NOSP}] keyword to deactivate the SPH equivalence scheme which
modifies the self-shielded averaged neutron fluxes in
heterogeneous geometries. The default option is to perform SPH equivalence.

\item[\moc{TRAN}] keyword to activate the transport correction option for
self-shielding calculations (see \moc{CTRA} in \Sectand{MACData}{LIBData}). This
is the default option.

\item[\moc{NOTR}] keyword to deactivate the transport correction option for
self-shielding calculations (see \moc{CTRA} in \Sectand{MACData}{LIBData}).

\item[\moc{PIJ}] keyword to specify the use of complete collision
probabilities in the subgroup and SPH equivalence calculations of {\tt USS:}.
This is the default option for \moc{EXCELT:} and \moc{SYBILT:} trackings.
This option is not available for \moc{MCCGT:} trackings.

\item[\moc{ARM}] keyword to specify the use of iterative flux techniques
in the subgroup and SPH equivalence calculations of {\tt USS:}.
This is the default option for \moc{MCCGT:} trackings.

\item[\moc{MAXST}] keyword to set the maximum number of fixed point iterations
for the ST scattering source convergence.

\item[\dusa{imax}] the maximum number of ST iterations. The default is
\dusa{imax} $=50$. A non-iterative response matrix approach is available with
the subgroup projection method (SPM) by setting \dusa{imax} $=0$.

\item[\moc{FLAT}] keyword to force the flat-flux initialization of subgroup fluxes if \dusa{MICLIB}
is open in modification mode.

\item[\moc{CALC}] keyword to activate the simplified self-shielding
approximation in which a single self-shielded isotope is shared by many
resonant mixtures.

\item[\moc{REGI}] keyword to specify a set of isotopes and mixtures that
will be self-shielded together. All the self-shielded isotopes in this group
will share the same 4--digit suffix.

\item[\dusa{suffix}] {\tt character*4} suffix for the isotope names in this
group

\item[\dusa{isot}] {\tt character*8} alias name of a self-shielded isotope in this
group

\item[\moc{ALL}] keyword to specify that a unique self-shielded isotope will be
made for the complete domain

\item[\dusa{imix}] list of mixture indices that will share the same self-shielded
isotope

\item[\dusa{nmix}] number of mixtures that will share the same self-shielded
isotope

\item[\moc{ENDC}] end of \moc{CALC} data keyword

\end{ListeDeDescription}

\vskip 0.15cm

Here is an example of the data structure corresponding to a production case where
only $^{238}$U is assumed to show distributed self-shielding effects:

\begin{verbatim}
LIBRARY2 := USS: LIBRARY TRACK ::
     CALC REGI W1 PU239 ALL
          REGI W1 PU241 ALL
          REGI W1 PU240 ALL
          REGI W1 PU242 ALL
          REGI W1 U235 ALL
          REGI W1 U236 ALL
          REGI W1 PU238 ALL
          REGI W1 U234 ALL
          REGI W1 AM241 ALL
          REGI W1 NP237 ALL
          REGI W1 ZRNAT ALL
          REGI W1 U238 <<COMB0101>> <<COMB0201>> <<COMB0301>>
                       <<COMB0401>> <<COMB0501>>
          REGI W2 U238 <<COMB0102>> <<COMB0202>> <<COMB0302>>
                       <<COMB0402>> <<COMB0502>>
          REGI W3 U238 <<COMB0103>> <<COMB0203>> <<COMB0303>>
                       <<COMB0403>> <<COMB0503>>
          REGI W4 U238 <<COMB0104>> <<COMB0204>> <<COMB0304>>
                       <<COMB0404>> <<COMB0504>>
          REGI W5 U238 <<COMB0105>> <<COMB0205>> <<COMB0305>>
                       <<COMB0405>> <<COMB0505>>
          REGI W6 U238 <<COMB0106>> <<COMB0206>> <<COMB0306>>
                       <<COMB0406>> <<COMB0506>>
     ENDC ;
\end{verbatim}

\vskip 0.15cm

In this case, $^{238}$U is self-shielded within six distributed regions (labeled
{\tt W1} to {\tt W6}) and each of these regions are merging volumes belonging
to five different fuel rods. The mixture indices of the 30 resonant volumes belonging
to the fuel are CLE-2000 variables labeled {\tt <<COMB0101>>} to {\tt <<COMB0506>>}.

\eject

 % structure (dragonUSS)
\subsection{The {\tt AUTO:} module}\label{sect:AUTOData}

The Autosecol self-shielding module in DRAGON, called {\tt AUTO:}, allows the
correction of the microscopic cross sections to take into account the
self-shielding effects related to the resonant isotopes.\cite{autosecol}

\vskip 0.08cm

{\sl Autolib data} is a fine-group representation of microscopic cross-section data for the resonant isotopes available in a
{\sl Draglib} or {\sl APOLIB-2} cross-section library. Each fine group in the Autolib has a lethargy width which is an integer multiple of an
{\sl elementary lethargy width}. Elastic slowing-down scattering is assumed for the resonant isotopes.

Integrating the Livolant-Jeanpierre equation over a fine group $g$, the Autosecol equation is written
\begin{equation}
\bff(\Omega)\cdot\bff(\nabla)\varphi_g(\bff(r),\bff(\Omega))\,+\,\Sigma_g(\bff(r))\,\varphi_g(\bff(r),\bff(\Omega))\,=\,{1\over 4\pi} \left[ \Sigma_{{\rm s},g}^+(\bff(r)) \, + \,\sum_h \Sigma_{{\rm s},j,g \leftarrow h}^{*} \, \varphi_h(\bff(r)) {\Delta u_h\over \Delta u_g} \right]
\label{eq:auto1}
\end{equation}

\noindent where the group integrated fine structure function is written
\begin{equation}
\varphi_g(\bff(r))={1\over \Delta u_g}\int_{u_{g-1}}^{u_g} du\, \varphi(\bff(r),u)
\label{eq:auto2}
\end{equation}

\noindent and where the $+$ and $*$ subscripts identify non-resonant and resonant isotopes respectively.

\vskip 0.08cm

The {\sl Autosecol method} consists to solve the Livolant-Jeanpierre equation over the Autolib energy mesh using a solution
technique of the Boltzmann transport equation available in DRAGON.\cite{PIP2009} The Autosecol method
is an accurate self-shielding technique relying on the fine-group solution of an heterogeneous transport equation. This approach may require
substantial CPU resources in actual production cases.

\vskip 0.08cm

Resonant isotopes are identified as such by the \dusa{inrs} parameter, as defined in
\Sect{LIBData}. The Autosecol self-shielding module is based on the following models:

\begin{itemize}
\item The Livolant-Jeanpierre flux factorization and approximations are used to
uncouple the self-shielding treatment from the main flux calculation;
\item The resonant cross sections are represented using {\sl Autolib data} 
recovered by the \moc{LIB:} module.
\item Probability tables are used in the unresolved energy domain to randomly
sample cross-section data into the Autolib fine mesh. The keyword \moc{SUBG} {\sl must} be
set in module {\tt LIB:}.
\item The resonant fine structure values $\varphi_g(\bff(r))$ are obtained as a solution
of the Autosecol Eq.~(\ref{eq:auto1}) over the Autolib fine mesh;
\item The flux can be solved using collision probabilities, or using {\sl any}
flux solution technique for which a tracking module is available;
\item All resonant isotopes with the same \dusa{inrs} index (see Sect.~\ref{sect:descmix})
are computed simultanously;
\item The distributed self-shielded effect is automatically taken into account
if different mixture indices are assigned to different regions inside the
resonant part of the cell. The rim effect can be computed by dividing the fuel
into "onion rings" and by assigning different mixture indices to them. 
\item A SPH (superhomog\'en\'eisation) equivalence is performed to correct the
self-shielded cross sections from the non-linear effects related to the
heterogeneity of the geometry.
\end{itemize}

\vskip 0.2cm

The general format of the data for this module is:

\begin{DataStructure}{Structure \dstr{AUTO:}}
\dusa{MICLIB} \moc{:=} \moc{AUTO:} \dusa{MICLIB\_SG} $[$ \dusa{MICLIB} $]$
\dusa{TRKNAM} $[$ \dusa{TRKFIL} $]$ \moc{::} \dstr{descauto}
\end{DataStructure}

\noindent where

\begin{ListeDeDescription}{mmmmmmmm}

\item[\dusa{MICLIB}] {\tt character*12} name of the \dds{microlib} that will
contain the microscopic and macroscopic cross sections updated by the
self-shielding module. If
\dusa{MICLIB} appears on both LHS and RHS, it is updated; otherwise,
\dusa{MICLIB} is created.

\item[\dusa{MICLIB\_SG}] {\tt character*12} name of the \dds{microlib} builded
by module \moc{LIB:} and containing probability table information for the unresolved
domain.

\item[\dusa{TRKNAM}] {\tt character*12} name of the required \dds{tracking}
data structure.

\item[\dusa{TRKFIL}] {\tt character*12} name of the sequential binary tracking
file used to store the tracks lengths. This file is given if and only if it was
required in the previous tracking module call (see \Sect{TRKData}).

\item[\dstr{descauto}] structure describing the self-shielding options.

\end{ListeDeDescription}

\subsubsection{Data input for module {\tt AUTO:}}\label{sect:descauto}

\begin{DataStructure}{Structure \dstr{descauto}}
$[$ \moc{EDIT} \dusa{iprint} $]$ \\
$[$ \moc{GRMIN}  \dusa{lgrmin} $]~~[$ \moc{GRMAX}  \dusa{lgrmax} $]$~~
$[$ \moc{PASS} \dusa{ipass} $]~~[~\{$ \moc{SPH} $|$ \moc{NOSP} $\}~]$~~$[$ $\{$ \moc{TRAN} $|$ \moc{NOTR} $\}$ $]$ \\ 
$[$ $\{$ \moc{PIJ} $|$ \moc{ARM} $\}$ $]$ \\
$[[$ \moc{DILU} \dusa{isot\_d} \dusa{dilut} $]]$ \\
$[$ \moc{KERN} \dusa{ialter} $]~~[$ \moc{MAXT} \dusa{maxtra} $]$ \\
$[$~\moc{SEED} \dusa{iseed}~$]$ \\
$[$ \moc{CALC} \\
~~~~$[[$ \moc{REGI} \dusa{suffix} $[[$ \dusa{isot} $\{$ \moc{ALL} $|$
(\dusa{imix}(i),i=1,\dusa{nmix}) $\}$ $]]$ \\
~~~~$]]$ \\
\moc{ENDC} $]$ \\
{\tt ;}
\end{DataStructure}

\noindent where

\begin{ListeDeDescription}{mmmmmmmm}

\item[\moc{EDIT}] keyword used to modify the print level \dusa{iprint}.

\item[\dusa{iprint}] index used to control the printing of this module. The
amount of output produced by this tracking module will vary substantially
depending on the print level specified. 

\item[\moc{GRMIN}] keyword to specify the minimum group number considered
during the self-shielding process.

\item[\dusa{lgrmin}] first group number considered during the
self-shielding process. By default, \dusa{lgrmin} is set to the first group
number containing self-shielding data in the library.

\item[\moc{GRMAX}]  keyword to specify the maximum group number considered
during the self-shielding process.

\item[\dusa{lgrmax}] last group number considered during the self-shielding
process. By default, \dusa{lgrmax} is set is set to the last group
number containing self-shielding data in the library.

\item[\moc{PASS}]  keyword to specify the number of outer iterations during
the self-shielding process. If all \dusa{inrs} indices are set to one in module \moc{LIB:},
these iterations are not required.

\item[\dusa{ipass}] the number of iterations. The default is \dusa{ipass} $=1$ if
\dusa{MICLIB} is created.

\item[\moc{SPH}] keyword to activate the SPH equivalence scheme which
modifies the self-shielded averaged neutron fluxes in
heterogeneous geometries (default option).

\item[\moc{NOSP}] keyword to deactivate the SPH equivalence scheme which
modifies the self-shielded averaged neutron fluxes in heterogeneous geometries.

\item[\moc{TRAN}] keyword to activate the transport correction option for
self-shielding calculations (see \moc{CTRA} in \Sectand{MACData}{LIBData}). This
is the default option.

\item[\moc{NOTR}] keyword to deactivate the transport correction option for
self-shielding calculations (see \moc{CTRA} in \Sectand{MACData}{LIBData}).

\item[\moc{PIJ}] keyword to specify the use of complete collision
probabilities in the subgroup and SPH equivalence calculations of {\tt AUTO:}.
This is the default option for \moc{EXCELT:} and \moc{SYBILT:} trackings.
This option is not available for \moc{MCCGT:} trackings.

\item[\moc{ARM}] keyword to specify the use of iterative flux techniques
in the subgroup and SPH equivalence calculations of {\tt AUTO:}.
This is the default option for \moc{MCCGT:} trackings.

\item[\moc{DILU}]  keyword to input an additional microscopic dilution value for a specific isotope. By default, no dilution
source other than $\Sigma_{{\rm s},g}^+(\bff(r))$ is used.

\item[\dusa{isot\_d}] {\tt character*8} alias name of the specific isotope.

\item[\dusa{dilut}] dilution value in barn.

\item[\moc{KERN}]  keyword to input the type of elastic slowing-down kernel.

\item[\dusa{ialter}] integer value indicating the type:
$$
\textsl{ialter} = \left\{
\begin{array}{ll}
0 & \textrm{use exact elastic kernel} \\
1 & \textrm{use an approximate kernel for the resonant isotopes.}
\end{array} \right.
$$

\item[\moc{MAXT}]  keyword to input a maximum storage size for the slowing-down kernel values.

\item[\dusa{maxtra}] integer value indicating the storage size. The default value is \dusa{maxtra} $=$ 10000.

\item[\moc{SEED}] keyword used to set the initial seed integer for the random number generator used in
the unresolved energy domain. By default, the seed integer is set from the processor clock.

\item[\dusa{iseed}] initial seed integer.

\item[\moc{CALC}] keyword to activate the simplified self-shielding
approximation in which a single self-shielded isotope is shared by many
resonant mixtures.

\item[\moc{REGI}] keyword to specify a set of isotopes and mixtures that
will be self-shielded together. All the self-shielded isotopes in this group
will share the same 4--digit suffix.

\item[\dusa{suffix}] {\tt character*4} suffix for the isotope names in this
group

\item[\dusa{isot}] {\tt character*8} alias name of a self-shielded isotope in this
group

\item[\moc{ALL}] keyword to specify that a unique self-shielded isotope will be
made for the complete domain

\item[\dusa{imix}] list of mixture indices that will share the same self-shielded
isotope

\item[\dusa{nmix}] number of mixtures that will share the same self-shielded
isotope

\item[\moc{ENDC}] end of \moc{CALC} data keyword

\end{ListeDeDescription}

\vskip 0.15cm

Here is an example of the data structure corresponding to a production case where
only $^{238}$U is assumed to show distributed self-shielding effects:

\begin{verbatim}
LIBRARY2 := AUTO: LIBRARY TRACK ::
     CALC REGI W1 PU239 ALL
          REGI W1 PU241 ALL
          REGI W1 PU240 ALL
          REGI W1 PU242 ALL
          REGI W1 U235 ALL
          REGI W1 U236 ALL
          REGI W1 PU238 ALL
          REGI W1 U234 ALL
          REGI W1 AM241 ALL
          REGI W1 NP237 ALL
          REGI W1 ZRNAT ALL
          REGI W1 U238 <<COMB0101>> <<COMB0201>> <<COMB0301>>
                       <<COMB0401>> <<COMB0501>>
          REGI W2 U238 <<COMB0102>> <<COMB0202>> <<COMB0302>>
                       <<COMB0402>> <<COMB0502>>
          REGI W3 U238 <<COMB0103>> <<COMB0203>> <<COMB0303>>
                       <<COMB0403>> <<COMB0503>>
          REGI W4 U238 <<COMB0104>> <<COMB0204>> <<COMB0304>>
                       <<COMB0404>> <<COMB0504>>
          REGI W5 U238 <<COMB0105>> <<COMB0205>> <<COMB0305>>
                       <<COMB0405>> <<COMB0505>>
          REGI W6 U238 <<COMB0106>> <<COMB0206>> <<COMB0306>>
                       <<COMB0406>> <<COMB0506>>
     ENDC ;
\end{verbatim}

\vskip 0.15cm

In this case, $^{238}$U is self-shielded within six distributed regions (labeled
{\tt W1} to {\tt W6}) and each of these regions are merging volumes belonging
to five different fuel rods. The mixture indices of the 30 resonant volumes belonging
to the fuel are CLE-2000 variables labeled {\tt <<COMB0101>>} to {\tt <<COMB0506>>}.

\eject
 % structure (dragonAUTO)
\subsection{The {\tt VDG:} module}\label{sect:VDGData}

The {\tt VDG:} module performs a comparison of an approximate self-shielding method with the Autosecol method.
This module is useful to obtain accuracy results for the Van Der Gucht benchmarks.\cite{vdg}
The calling specifications are:

\begin{DataStructure}{Structure \dstr{VDG:}}
\moc{VDG:} \dusa{MICLIB1}  \dusa{MICLIB2} \moc{::} \dstr{descvdg}
\end{DataStructure}

\vskip -0.5cm

\noindent where

\begin{ListeDeDescription}{mmmmmmmm}

\item[\dusa{MICLIB1}] {\tt character*12} name of the self-shielded \dds{microlib} produced by the
Autosecol method (module \moc{AUTO:}).

\item[\dusa{MICLIB2}] {\tt character*12} name of the self-shielded \dds{microlib} produced by an
approximate self-shielding method.

\item[\dstr{descvdg}] structure describing the \moc{VDG:} module options.

\end{ListeDeDescription}

\
\subsubsection{Data input for module {\tt VDG:}}\label{sect:descvdg}

\begin{DataStructure}{Structure \dstr{descvdg}}
$[$ \moc{EDIT} \dusa{iprint} $]$ \\
$[$ \moc{GRMI} \dusa{igrp1} $]~[$ \moc{GRMA} \dusa{igrp2} $]$ \\
$[$ \moc{PICK} $\{$ \moc{MAXV} $|$ \moc{AVER} $|$ \moc{INTG} $\}$ {\tt >>} \dusa{error} {\tt <<} $]$ \\
{\tt ;}
\end{DataStructure}

\vskip -0.5cm

\noindent where

\begin{ListeDeDescription}{mmmmmmmm}

\item[\moc{EDIT}] keyword used to modify the print level \dusa{iprint}.

\item[\dusa{iprint}] index used to control the printing of this module. The
amount of output produced by this module will vary substantially
depending on the print level specified. 

\item[\moc{GRMI}] keyword used to set the index of the first group in the microlib. By default, \moc{GRMI} $=$ 1.

\item[\dusa{igrp1}] index of the first resonant group.

\item[\moc{GRMA}] keyword used to set the index of the last group in the microlib. By default, \moc{GRMA} $=$ 9999999.

\item[\dusa{igrp2}] index of the last resonant group.

\item[\moc{PICK}]  keyword used to recover a relative discrepancy value for the absorption rates in
each DRAGON resonant energy group.

\item[\moc{MAXV}] keyword to select the maximum relative discrepancy.

\item[\moc{AVER}] keyword to select the averaged relative discrepancy.

\item[\moc{INTG}] keyword to select the integrated relative discrepancy.

\item[\dusa{error}] \texttt{character*12} CLE-2000 variable name in which the extracted percent
discrepancy value will be placed.

\end{ListeDeDescription}
\eject
 % structure (dragonVDG)

\section{THE SALOME-RELATED MODULES}\label{sect:SalomeModuleInput}

A few modules have been introduced in DRAGON Version5 in order to facilitate the
processing of geometries originating from the Geometry module of SALOME.\cite{salome}
The methods presented in this section have been initially developed at CEA SERMA and
integrated in the TDT code.\cite{tdt,lyioussi} In the course of year 2001, a subset of
these methods have been integrated into a development version of DRAGON under the terms
of its LGPL license as a prototyping exercise of the DESCARTES operation.\cite{salt}

\vskip 0.08cm

The track generator {\tt SALT:} is a direct descendent of this prototyping exercise.
Later, we have extracted the 5000 lines of Fortran-90 code responsible for the track
calculation and have rewritten them in a way consistent with the {\tt NXT:} tracking
methodology and with the DRAGON architecture.

\vskip 0.08cm

The {\tt SALT:} module can process two types of geometries:
\begin{itemize}
\item {\sl Native geometries} are those defined using the {\tt GEO:} module and transformed into surfacic
geometries using the {\tt G2S:} module. These geometries have many limitations related to their
definition.
\item {\sl Non-native geometries} are surfacic representations based on extensions of the SALOME platform.
A first extension is the SALOMON tool presented in Ref.~\citen{ane15b}. ALAMOS is a more recent tool
available at the Commissariat \`a l'\'Energie Atomique.\cite{alamos} Surfacic geometries produced by
ALAMOS must be converted to the SALOMON format using the {\tt G2S:} module before calling the track
generator {\tt SALT:}.
\end{itemize}

\subsection{The {\tt G2S:} module}\label{sect:G2SData}

The module {\tt G2S:} is used to create the SALOMON--formatted surfacic elements corresponding
to a gigogne geometry. It can also be used to plot a SALOMON--formatted file or as a conversion tool to transform an ALAMOS--formatted file into
a SALOMON--formatted file. The general format of the input data for the {\tt G2S:} module is the following:
\begin{DataStructure}{Structure \dstr{G2S:}}
$[$ \dusa{SURFIL} $]~[$ \dusa{PSFIL} $]$ \moc{:=} \moc{G2S:}~ $\{$ \dusa{SURFIL\_IN} $[$ \dusa{ZAFIL\_IN} $]~|$ \dusa{GEONAM} $\}$ ~\moc{::}~\dstr{G2S\_data} \\
\end{DataStructure}

\noindent where
\begin{ListeDeDescription}{mmmmmm}

\item[\dusa{SURFIL}] \texttt{character*12} name of the output SALOMON--formatted sequential {\sc ascii}
file used to store the surfacic elements of the geometry.

\item[\dusa{PSFIL}] \texttt{character*12} name of the sequential {\sc ascii}
file used to store a postscript representation of the geometry corresponding to \dusa{SURFIL} or \dusa{GEONAM}.

\item[\dusa{GEONAM}] {\tt character*12} name of the {\sl read-only} \dds{geometry} data
structure. This structure may be build using the operator {\tt GEO:} (see \Sect{GEOData}).

Reflective boundary conditions defined in operator {\tt GEO:} can be represented in two different ways:
\vspace{-0.2cm}
\begin{description}
\item[{\tt ALBE 1.0}:] isotropic (or white) boundary condition compatible with \moc{TISO} (non-cyclic) tracking;
\item[{\tt REFL}:] specular boundary condition compatible with \moc{TSPC} (cyclic) tracking.
\end{description}

\item[\dusa{SURFIL\_IN}] \texttt{character*12} name of the input SALOMON-- or ALAMOS--formatted sequential {\sc ascii}
file used to store the surfacic elements of the geometry.

\item[\dusa{ZAFIL\_IN}] \texttt{character*12} name of the input sequential {\sc ascii} file containing {\sl PropertyMap}
information associated with the ALAMOS geometry. This file generally has a {\tt .za} extension. This information is used to
set node-ordered mixture indices. By default, node-ordered mixture indices are recovered from the {\sl Mailles} record present
in the ALAMOS surfacic file.

\item[\dusa{G2S\_data}] input data structure containing specific data (see \Sect{descG2S}).
\end{ListeDeDescription}

\subsubsection{Data input for module {\tt G2S:}}\label{sect:descG2S}

\vskip -0.5cm

\begin{DataStructure}{Structure \dstr{G2S\_data}}
$[$~\moc{EDIT} \dusa{iprint}~$]$ \\
$[$~\moc{ALAMOS} \dusa{typgeo}~$]$ \\
$[~\{$~\moc{DRAWNOD} $|$ \moc{DRAWMIX} $\}~]~[$ \moc{ZOOMX} \dusa{facx1} \dusa{facx2} $]~[$ \moc{ZOOMY} \dusa{facy1} \dusa{facy2} $]$ \\
\moc{;}
\end{DataStructure}

\vskip -0.3cm

\noindent where
\begin{ListeDeDescription}{mmmmmmmm}

\item[\moc{EDIT}] keyword used to set \dusa{iprint}.

\item[\dusa{iprint}] index used to control the printing in module {\tt G2S:}. =0 for no print; =1 for minimum printing (default value).

\item[\moc{ALAMOS}] keyword to use {\tt G2S:} as a conversion tool to transform an ALAMOS--formatted file into
a SALOMON-formatted file.

\item[\dusa{typgeo}] type of Alamos geometry. A negative value is used for isotropic reflection (white boundary condition) with unfolding in {\tt SALT:} module.
Otherwise, a specular boundary condition is used without unfolding.
\begin{displaymath}
\negthinspace \textsl{typgeo} = \left\{
\begin{array}{ll}
0 & \textrm{Isotropic reflection (white boundary condition) without perimeter}\\
& \textrm{information}\\
\pm 5 & \textrm{Cartesian rectangular geometry with translation}\\
\pm 6 & \textrm{Cartesian rectangular geometry with specular reflection on each}\\
& \textrm{side}\\
\pm 7 & \textrm{Cartesian eight-of-square geometry with specular reflection on each}\\
& \textrm{side}\\
\pm 8 & \textrm{Hexagonal SA60 equilateral triangle geometry with specular reflec-}\\
& \textrm{tion on each side}\\
\pm 9 & \textrm{Full hexagonal geometry with translation}\\
\pm 10 & \textrm{Hexagonal RA60 equilateral triangle geometry with } 60^\circ \textrm{ rotation}\\
& \textrm{and translation}\\
\pm 11 & \textrm{Hexagonal R120 lozenge geometry with } 120^\circ \textrm{ rotation and transla-}\\
& \textrm{tion.}
\end{array} \right.
\end{displaymath}

\item[\moc{DRAWNOD}] keyword used to print the region indices on the LHS postscript plot \dusa{PSFIL}. By default, no indices are printed.

\item[\moc{DRAWMIX}] keyword used to print the material mixture indices on the LHS postscript plot \dusa{PSFIL}. By default, no indices are printed.

\item[\moc{ZOOMX}] keyword used to plot a fraction of the $X$--domain. By default, all the $X$--domain is plotted.

\item[\dusa{facx1}] left factor set in interval $0.0 \le$ \dusa{facx1} $< 1.0$ with 0.0 corresponding to the left boundary and 1.0 corresponding to the right boundary.

\item[\dusa{facx2}] right factor set in interval \dusa{facx1} $<$ \dusa{facx2} $\le 1.0$.

\item[\moc{ZOOMY}] keyword used to plot a fraction of the $Y$--domain. By default, all the $Y$--domain is plotted.

\item[\dusa{facy1}] lower factor set in interval  $0.0 \le$ \dusa{facy1} $< 1.0$ with 0.0 corresponding to the lower boundary and 1.0 corresponding to the upper boundary.

\item[\dusa{facy2}] upper factor set in interval \dusa{facy1} $<$ \dusa{facy2} $\le 1.0$.

\end{ListeDeDescription}


\clearpage
 % structure (dragonG2S)
\subsection{The {\tt G2MC:} module}\label{sect:G2MCData}

The module {\tt G2MC:} is used to compute the SERPENT--, TRIPOLI4--, or MCNP--formatted surfacic elements corresponding
to a SALOMON--formatted file or corresponding to a gigogne geometry. The general format of the input data for the
{\tt G2MC:} module is the following:
\begin{DataStructure}{Structure \dstr{G2MC:}}
\dusa{MCFIL} $[$ \dusa{PSFIL} $]$ \moc{:=} \moc{G2MC:} $\{$ \dusa{SURFIL} $|$ \dusa{GEONAM} $\}$ ~\moc{::}~\dstr{G2MC\_data} \\
\end{DataStructure}

\noindent where
\begin{ListeDeDescription}{mmmmmm}

\item[\dusa{MCFIL}] \texttt{character*12} name of the SERPENT--, TRIPOLI4-- or MCNP--formatted sequential {\sc ascii}
file used to store the surfacic elements of the geometry. A SERPENT file is
produced if the file name has extension {\tt ".sp"}. A TRIPOLI4 file is
produced if the file name has extension {\tt ".tp"}. Otherwise, a MCNP file is
produced. This file is to be included in the complete dataset of a Monte Carlo code.

\item[\dusa{PSFIL}] \texttt{character*12} name of the sequential {\sc ascii}
file used to store a postscript representation of the geometry corresponding to \dusa{GEONAM}.

\item[\dusa{SURFIL}] \texttt{character*12} name of the {\sl read-only} SALOMON--formatted sequential {\sc ascii}
file used to store the surfacic elements of the geometry.

\item[\dusa{GEONAM}] {\tt character*12} name of the {\sl read-only} \dds{geometry} data
structure. This structure may be build using the operator {\tt GEO:} (see \Sect{GEOData}).

\item[\dusa{G2MC\_data}] input data structure containing specific data (see \Sect{descG2MC}).

\end{ListeDeDescription}

\subsubsection{Data input for module {\tt G2MC:}}\label{sect:descG2MC}

\vskip -0.5cm

\begin{DataStructure}{Structure \dstr{G2MC\_data}}
$[$~\moc{EDIT} \dusa{iprint}~$]$ \\
$[~\{$~\moc{DRAWNOD} $|$ \moc{DRAWMIX} $\}~]~[$ \moc{ZOOMX} \dusa{facx1} \dusa{facx2} $]~[$ \moc{ZOOMY} \dusa{facy1} \dusa{facy2} $]$ \\
\moc{;}
\end{DataStructure}

\noindent where
\begin{ListeDeDescription}{mmmmmmmm}

\item[\moc{EDIT}] keyword used to set \dusa{iprint}.

\item[\dusa{iprint}] index used to control the printing in module {\tt G2MC:}. =0 for no print; =1 for minimum printing (default value).

\item[\moc{DRAWNOD}] keyword used to print the region indices on the LHS postscript plot \dusa{PSFIL}. By default, no indices are printed.

\item[\moc{DRAWMIX}] keyword used to print the material mixture indices on the LHS postscript plot \dusa{PSFIL}. By default, no indices are printed.

\item[\moc{ZOOMX}] keyword used to plot a fraction of the $X$--domain. By default, all the $X$--domain is plotted.

\item[\dusa{facx1}] left factor set in interval $0.0 \le$ \dusa{facx1} $< 1.0$ with 0.0 corresponding to the left boundary and 1.0 corresponding to the right boundary.

\item[\dusa{facx2}] right factor set in interval \dusa{facx1} $<$ \dusa{facx2} $\le 1.0$.

\item[\moc{ZOOMY}] keyword used to plot a fraction of the $Y$--domain. By default, all the $Y$--domain is plotted.

\item[\dusa{facy1}] lower factor set in interval  $0.0 \le$ \dusa{facy1} $< 1.0$ with 0.0 corresponding to the lower boundary and 1.0 corresponding to the upper boundary.

\item[\dusa{facy2}] upper factor set in interval \dusa{facy1} $<$ \dusa{facy2} $\le 1.0$.

\end{ListeDeDescription}

\clearpage
 % structure (dragonG2MC)
\subsection{The {\tt SALT:} tracking module}\label{sect:SALTData1}

The \moc{{\tt SALT:}} module can process general 2-D geometries defined from
{\sl surfacic elements}. It is used to compute the tracking information requested in
the method of collision probabilities or in the method of characteristics.

\subsubsection{Cyclic tracking}

The {\tt SALT:} module with keyword {\tt TSPC} has the capability to perform {\sl cyclic tracking} over a closed square, rectangular
or equilateral triangular domain. Each track cover a certain surface before going back to the starting point after a distance $L$, as
depicted in Figs.~\ref{fig:cart_tspc} and~\ref{fig:hex_tspc}. Only specific angles, function of integer values $n$ and $m$, make
possible the cycling of trajectories.

\begin{figure}[h!]
\begin{center} 
\epsfxsize=10cm \centerline{ \epsffile{cart_tspc.eps}}
\parbox{14.0cm}{\caption{Cycling tracking over a Cartesian domain.}\label{fig:cart_tspc}}   
\end{center}  
\end{figure}

\begin{figure}[h!]
\begin{center} 
\epsfxsize=15cm \centerline{ \epsffile{hex_tspc.eps}}
\parbox{14.0cm}{\caption{Cycling tracking over an hexagonal domain.}\label{fig:hex_tspc}}   
\end{center}  
\end{figure}

\subsubsection{Calling specifications}

The calling specification for this module is:
\begin{DataStructure}{Structure \dstr{SALT:}}
\dusa{TRKNAM} \dusa{TRKFIL}
\moc{:=} \moc{SALT:}~\dusa{SURFIL} $[$ \dusa{GEONAM} $]$ \moc{::} \dstr{desctrack} \dstr{descsalt}
\end{DataStructure}

\noindent  where
\begin{ListeDeDescription}{mmmmmmm}

\item[\dusa{TRKNAM}] \texttt{character*12} name of the SALT \dds{tracking} data
structure that will contain region volume and surface area vectors in
addition to region identification pointers and other tracking information.

\item[\dusa{TRKFIL}] \texttt{character*12} name of the sequential binary tracking
file used to store the tracks lengths.

\item[\dusa{SURFIL}] \texttt{character*12} name of the SALOMON--formatted sequential {\sc ascii}
file used to store the surfacic elements of the geometry. This file may be build
using the operator {\tt G2S:} (see \Sect{G2SData}) or recovered from SALOME.

\item[\dusa{GEONAM}] {\tt character*12} name of the \dds{geometry} data
structure containing the double heterogeneity (Bihet) data.

\item[\dstr{desctrack}] structure describing the general tracking data (see
\Sect{TRKData})

\item[\dstr{descsalt}] structure describing the transport tracking data
specific to \moc{SALT:}.

\end{ListeDeDescription}

\vskip 0.2cm

All information for the modelization used can be found in \citen{salt}.
The \moc{{\tt SALT:}} specific tracking data in \dstr{descsalt} is defined as :

\begin{DataStructure}{Structure \dstr{descsalt}}
$[$ \moc{ANIS} \dusa{nanis} $]$ \\
$[~\{$  \moc{ONEG} $|$ \moc{ALLG} $\}~]$ \\
$[~[$ \moc{QUAB} \dusa{iquab} $]~[~\{$ \moc{SAPO} $|$ \moc{HEBE} $|$ \moc{SLSI} $[$ \dusa{frtm} $]~\}~]~]$ \\
$[~\{$ \moc{PISO} $|$ \moc{PSPC} $[$ \moc{CUT} \dusa{pcut} $]$ $\}~]$ \\
$[$ $\{$ \moc{GAUS}  $|$ \moc{CACA} $|$ \moc{CACB} $|$ \moc{LCMD} $|$ \moc{OPP1} $|$ \moc{OGAU} $\}~[$ \dusa{nmu} $]~]$ \\
$\{$ \moc{TISO} $[~\{$ \moc{EQW} $|$ \moc{GAUS} $|$ \moc{PNTN} $|$ \moc{SMS} $|$ \moc{LSN} $|$ \moc{QRN} $\}~]$ \dusa{nangl} \dusa{dens} \\
$~~~~~|$ \moc{TSPC} $[~\{$ \moc{MEDI} $|$ \moc{EQW2} $\}~]$ \dusa{nangl} \dusa{dens} $\}$ \\
$[$ \moc{CORN} \dusa{pcorn} $]$ \\
$[$ \moc{NOTR} $]$\\
$[$ \moc{NBSLIN} \dusa{nbslin} $]$ \\
$[$ \moc{MERGMIX} $]$\\
$[$ \moc{LONG} $]$\\
{\tt ;}
\end{DataStructure}

\noindent
where

\begin{ListeDeDescription}{mmmmmmmm}

\item[\moc{ANIS}] keyword to specify the order of scattering anisotropy. 

\item[\dusa{nanis}] order of anisotropy in transport calculation.
A default value of 1 represents isotropic (or transport-corrected) scattering while a value of 2
correspond to linearly anisotropic scattering.

\item[\moc{ONEG}] keyword to specify that the tracking is read before computing each group-dependent collision
probability or algebraic collapsing matrix (default value if \dusa{TRKFIL} is set). The tracking file is
read in each energy group if the method of characteristics (MOC) is used.

\item[\moc{ALLG}] keyword to specify that the tracking is read once and the collision
probability or algebraic collapsing matrices are computed in many energy groups.  The tracking file is
read once if the method of characteristics (MOC) is used.
 
\item[\moc{QUAB}] keyword to specify the number of basis point for the
numerical integration of each micro-structure in cases involving double
heterogeneity (Bihet).

\item[\dusa{iquab}] the number of basis point for the numerical integration of
the collision probabilities in the micro-volumes using the Gauss-Jacobi
formula. The values permitted are: 1 to 20, 24, 28, 32 or 64. The default value
is \dusa{iquab} = 5. If \dusa{iquab} is negative, its absolute value will be used in the She-Liu-Shi approach to determine the
split level in the tracking used to compute the probability collisions.

\item[\moc{SAPO}] use the Sanchez-Pomraning double-heterogeneity model.\cite{sapo}

\item[\moc{HEBE}] use the Hebert double-heterogeneity model (default option).\cite{BIHET}

\item[\moc{SLSI}] use the She-Liu-Shi double-heterogeneity model without shadow effect.\cite{She2017}

\item[\dusa{frtm}] the minimum microstructure volume fraction used to compute the size of the equivalent cylinder in She-Liu-Shi approach. The default value is \dusa{frtm} $=0.05$.

\item[\moc{PISO}] keyword to specify that a collision probability calculation with isotropic reflection boundary 
conditions is required. It is the default option if a \moc{TISO} type integration is chosen. To obtain accurate
transmission probabilities for the isotropic case it is recommended that the normalization 
options in the \moc{ASM:} module be used. 

\item[\moc{PSPC}] keyword to specify that a collision probability calculation with mirror like reflection or periodic 
boundary conditions is required; this is the default option if a \moc{TSPC} type integration is chosen. 
This calculation is only possible if the file was initially constructed using the \moc{TSPC} option. 

\item[\moc{CUT}] keyword to specify the input of cutting parameters for the specular collision probability
of characteristic integration. 

\item[\dusa{pcut}] real value representing the maximum error allowed on the exponential function used
for specular collision probability calculations. Tracks will be cut at a length such that the error in the 
probabilities resulting from this reduced track will be of the order of pcut. By default, the tracks 
are extended to infinity and \dusa{pcut} = 0.0. If this option is used in an entirely reflected case, it is 
recommended to use the \moc{NORM} command in the \moc{ASM:} module. 

\item[\moc{GAUS}] keyword to specify that Gauss-Legendre polar integration angles are to be selected for the polar quadrature when a prismatic tracking is considered. The conservation is ensured up to $P_{\dusa{nmu}-1}$ scattering.

\item[\moc{CACA}] keyword to specify that CACTUS type equal weight polar integration angles are to be
selected for the polar quadrature when a prismatic tracking is considered.\cite{CACTUS} The conservation is ensured only for isotropic scattering.

\item[\moc{CACB}] keyword to specify that CACTUS type uniformly distributed integration polar angles
are to be selected for the polar quadrature when a prismatic tracking is considered.\cite{CACTUS} The conservation is ensured only for isotropic scattering.

\item[\moc{LCMD}] keyword to specify that optimized (McDaniel--type) polar integration angles are to be
selected for the polar quadrature when a prismatic tracking is considered.\cite{LCMD} This is the default option. The conservation is ensured only for isotropic scattering.

\item[\moc{OPP1}] keyword to specify that $P_1$ constrained optimized (McDaniel--type) polar integration angles are to be selected for the polar quadrature when a prismatic tracking is considered.\cite{LeTellierpa} The conservation is ensured only for isotropic and linearly anisotropic scattering.

\item[\moc{OGAU}] keyword to specify that Optimized Gauss polar integration angles are to be
selected for the method of characteristics.\cite{LCMD,LeTellierpa} The conservation is ensured up to $P_{\dusa{nmu}-1}$ scattering.

\item[\dusa{nmu}]  user-defined number of polar angles. By default, a value consistent with \dusa{nangl} is computed by the code. For \moc{LCMD}, \moc{OPP1}, \moc{OGAU} quadratures, \dusa{nmu} is limited to 2, 3 or 4.

\item[\moc{TISO}] keyword to specify that isotropic tracking parameters will be supplied. This is the
default tracking option for cluster geometries. 

\item[\moc{TSPC}] keyword to specify that specular tracking parameters will be
supplied.

\item[\moc{EQW}] keyword to specify the use of equal weight quadrature.\cite{eqn} The conservation is ensured up to $P_{\dusa{nangl}/2}$ scattering.

\item[\moc{GAUS}] (after \moc{TISO} keyword) keyword to specify the use of the Gauss-Legendre quadrature. This option is valid only if an 
hexagonal geometry is considered.

\item[\moc{PNTN}] keyword to specify that Legendre-Techbychev quadrature quadrature will be selected.\cite{pntn} The conservation is ensured only for isotropic and linearly anisotropic scattering.

\item[\moc{SMS}] keyword to specify that Legendre-trapezoidal quadrature quadrature will be selected.\cite{sms} The conservation is ensured up to $P_{\dusa{nangl}-1}$ scattering.

\item[\moc{LSN}] keyword to specify the use of the $\mu_1$--optimized level-symmetric quadrature. The conservation is ensured up to $P_{\dusa{nangl}/2}$ scattering.

\item[\moc{QRN}] keyword to specify the use of the quadrupole range (QR) quadrature.\cite{quadrupole}

\item[\moc{MEDI}] keyword to specify the use of a median angle quadrature in \moc{TSPC} cases. For
a rectangular Cartesian domain of size $X \times Y$, the azimuthal angles in $(0,\pi/2)$ interval are obtained from formula
$$
\phi_k=\tan^{-1}{\displaystyle kY\over\displaystyle (2p+2-k)X} \, , \ \ k=1,\, 3,\, 5, \, \dots, \, 2p+1 .
$$

\item[\moc{EQW2}] keyword to specify the use of a standard cyclic quadrature without angles $\phi=0$ and $\phi=\pi/2$ in \moc{TSPC} cases. For
a rectangular Cartesian domain of size $X \times Y$, the azimuthal angles in $(0,\pi/2)$ interval are obtained from formula
$$
\phi_k=\tan^{-1}{\displaystyle k Y\over\displaystyle (p+2-k)X} \, , \ \ k=1,\, 2,\, 3, \, \dots, \, p+1 .
$$
This is the default option.

\item[\dusa{nangl}] angular quadrature parameter. For a 3-D \moc{EQW} option, the choices are \dusa{nangl} = 2, 4, 8, 10, 12, 14 
or 16. For a 3-D \moc{PNTN} or \moc{SMS} option, \dusa{nangl} is an even number smaller than 46.\cite{ige260} For 2-D 
isotropic applications, any value of \dusa{nangl} may be used, equidistant angles will be selected.

For 2-D specular applications the input value must be of the form $p + 1$ where $p$ is a prime number, as proposed
in Ref.~\citen{DragonPIJS3}. In this case, the choice of \dusa{nangl} = 6, 8, 12, 14, 18, 20, 24, or 30 are allowed. For hexagonal lattices,
including equilateral triangular and lozenge geometry, the choice of \dusa{nangl} = 3, 6, 12 or 18 are allowed.

\item[\dusa{dens}] real value representing the density of the integration lines (in cm$^{-1}$ for 2-D Cartesian cases.
This choice of density along the plan perpendicular to each angle depends on the geometry of the cell to be analyzed. If there 
are zones of very small volume, a high line density is essential. This value will be readjusted by 
\moc{SALT:}.

\item[\moc{CORN}] keyword to specify the meaningful distance (cm) between a tracking line and a surfacic element.

\item[\dusa{pcorn}] meaningful distance (cm) between a tracking line and a surfacic element. By default, \dusa{pcorn} $=1.0 \times 10^{-5}$ cm.

\item[\moc{NOTR}] keyword to specify that the geometry will not be tracked. This is useful for 2-D geometries 
to generate a tracking data structure that can be used by the \moc{PSP:} module (see \Sect{PSPData}). 
One can then verify visually if the geometry is adequate before the tracking process as such is 
undertaken.

\item[\moc{NBSLIN}] keyword to set the maximum number of segments in a single tracking line.

\item[\dusa{nbsl}] integer value representing the maximum number of segments in a single tracking line. The default value is \dusa{nbsl} = 100000.

\item[\moc{MERGMIX}] keyword to specify that all regions belonging to the same mixture will be merged together. This option should only be used as an attempt to reduce CPU costs in resonance self-shielding calculations.

\item[\moc{LONG}] keyword to specify that a ``long'' tracking file will be generated. This option is required if the tracking file is to be used by the \moc{TLM:} module (see \Sect{TLMData}).

\end{ListeDeDescription}
\clearpage
 % structure (dragonSALT)
