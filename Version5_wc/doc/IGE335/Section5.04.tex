\subsection{Macroscopic cross sections examples}\label{sect:ExMACROLIB}

The sample test cases we will consider here use the \moc{MAC:} module to enter
macroscopic cross sections directly into DRAGON. They are numbered successively
from \tst(TCM01) to \tst(TCM08).

\subsubsection{\tst(TCM01) -- Annular region}

\begin{figure}[h!]  
\begin{center} 
\epsfxsize=6cm \centerline{ \epsffile{GTCM01.eps}}
\parbox{16cm}{\caption{Geometry for test case \tst(TCM01) for an annular cell with
macroscopic cross sections.}\label{fig:TCM01}}    
\end{center}  
\end{figure}

This sample input is used to analyze the annular cell presented in \Fig{TCM01}.
It uses two-groups macroscopic cross sections provided directly by the user. One
type of solution is provided here, one with a complete collision probability
calculation (\moc{SYBILT:}). Note that for the second flux calculation the
initial flux distribution is taken from the existing \dds{fluxunk} structure
which already contains the flux distribution from the
\moc{SYBILT:} calculation. 

\listing{TCM01.x2m}

\subsubsection{\tst(TCM02) -- The Stankovski test case.}

\begin{figure}[h!]  
\begin{center} 
\epsfxsize=10cm \centerline{ \epsffile{GTCM02.eps}}
\parbox{14cm}{\caption{Geometry for test case \tst(TCM02).}\label{fig:TCM02}}    
\end{center}    \end{figure}

This test case represents a one group calculation of a $7\times 7$ PWR assembly.
The reaction rates obtained from DRAGON can be compared with those obtained
using the MARSYAS code.\cite{DragonPIJS2,DragonPIJS3,Stankovski} The
corresponding geometry is shown in \Fig{TCM02} where the cell numbers generated
by DRAGON are shown.

\listing{TCM02.x2m}

\subsubsection{\tst(TCM03) -- Watanabe and Maynard problem with a void region.}

\begin{figure}[h!]  
\begin{center} 
\epsfxsize=10cm \centerline{ \epsffile{GTCM03.eps}}
\parbox{14cm}{\caption{Geometry for test case \tst(TCM03).}
\label{fig:TCM03}}     \end{center}    \end{figure}

This test case is a one group problem with a central void region. This benchmark
was first proposed by  Watanabe and Maynard. Akroyd and Riyait used it to
analyze the performance of various codes.\cite{DragonPIJS2,DragonPIJS3,Akroyd}

\listing{TCM03.x2m}

\subsubsection{\tst(TCM04) -- Adjuster rod in a CANDU type supercell.}

\begin{figure}[h!]  
\begin{center} 
\epsfxsize=10cm \centerline{ \epsffile{GTCM04.eps}}
\parbox{14cm}{\caption{Geometry of the CANDU-6 supercell with stainless steel
rods.}\label{fig:TCM04}}   
\end{center}  
\end{figure}

This test case represents a two group calculation of incremental cross sections
resulting from  the insertion of stainless steel adjuster rods in a CANDU-6
supercell.

\listing{TCM04.x2m}

\subsubsection{\tst(TCM05) -- Comparison of leakage models}

This test presents various homogeneous and heterogeneous leakage models on a
simple cell.

\listing{TCM05.x2m}

\subsubsection{\tst(TCM06) -- Buckling search without fission source}

This test is for an homogeneous water cell. A buckling eigenvalue problem is
solved in the abscence of fission source for the neutron flux distribution
inside this cell.

\listing{TCM06.x2m}

\subsubsection{\tst(TCM07) -- Test of boundary conditions}

This test is for a 2--D Cartesian cell with refelctive and void boundary
conditions.

\listing{TCM07.x2m}

\subsubsection{\tst(TCM08) -- Fixed source problem with fission}

This test is for a 2--D Cartesian cell which contains both a fission and a
fixed source.

\listing{TCM08.x2m}
 
\subsubsection{\tst(TCM09) -- Solution of a 2-D fission source problem using \moc{MCCGT:}}\label{sect:ExTCM09}

This test case is for a $3\times 3$ Cartesian assembly in 2-D similar to TCM03. It is
solved using the method of cyclic characteristics.

\listing{TCM09.x2m}

\subsubsection{\tst(TCM10) -- Solution of a 2-D fixed source problem using \moc{MCCGT:}}\label{sect:ExTCM10}

This test case is for a 2--D Cartesian assembly that contains a fixed source. It is solved
using the method of cyclic characteristics.

\listing{TCM10.x2m}

\subsubsection{\tst(TCM11) -- Comparison of CP and MoC solutions}\label{sect:ExTCM11}

This test case is for a $4\times 4$ Cartesian assembly in 2-D. It is solved using the
method of cyclic characteristics and the method of collision probabilities using specular
(mirror like) boundary conditions.

\listing{TCM11.x2m}

\subsubsection{\tst(TCM12) - Solution of a 3-D problem using the \moc{MCU:}
module}\label{sect:ExTCM12}

This test case is for a simplified 3-D Cartesian assembly analyzed using the \moc{EXCELT:}. A
collisions probability solution is generated as well as two solutions using the method of
characteristics.

\listing{TCM12.x2m}

\subsubsection{\tst(TCM13) - Hexagonal assembly with hexagonal cells containing clusters}\label{sect:ExTCM13}

This test represents an example of a 2-D hexagonal assembly filled with triangular/hexagonal cells containing clusters (see \Fig{TCM13}) that can be analyzed with \moc{NXT:}.

\begin{figure}[h!]  
\begin{center} 
\parbox{10.0cm}{\epsfxsize=10cm \epsffile{GTCM13.eps}}
\parbox{14cm}{\caption{Geometry of a 2-D hexagonal assembly filled with triangular/hexagonal cells.}\label{fig:TCM13}}   
\end{center}  
\end{figure}

\listing{TCM13.x2m}

\eject
