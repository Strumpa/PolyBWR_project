\subsection{The tracking modules}\label{sect:TRKData}

A tracking module is required to analyze a spatial domain (geometry) assuming
a specific algorithm will be used for the collision probability or method of characteristics
calculations.  It performs zone numbering operations, volume and surface area
calculations and generates the required  integration lines for a geometry that
was previously defined in the \moc{GEO:} module. These operations are carried
out differently depending on the solution algorithm used.

\vskip 0.15cm

Many different operators are available for tracking in DRAGON. The \moc{SYBILT:} module
is used for 1--D geometries (either plane, cylindrical or spherical) and
interface current tracking inside heterogeneous blocks. The \moc{EXCELT:} module
is used to perform full cell collision probability tracking with
isotropic\cite{DragonPIJI,Mtl93a} or specular\cite{DragonPIJS1,Mtl93b}
surface current. The \moc{NXT:} module is an extension of the \moc{EXCELT:}
module to more complex geometry including assemblies of clusters in two and
three dimensions.\cite{ige260}  The \moc{MCCGT:} module is an implementation of the open
characteristics method of I.~R.~Suslov.\cite{mccg,suslov2}. These are the transport
tracking modules which can be used everywhere in the code where tracking
information needs to be generated. The \moc{SNT:} module is an implementation of
the discrete ordinates (or $S_N$) method in 1-D/2-D/3-D geometries.
The module \moc{BIVACT:} is used to perform a finite-element (diffusion or SP$_n$) 1-D/2-D
tracking which may be required for diffusion synthetic acceleration (DSA) or homogenization
purposes.\cite{BIVAC} The final module \moc{TRIVAT:} is used to perform a finite-element
1-D/2-D/3-D tracking which may be required for DSA or homogenization purposes.\cite{TRIVAC}

\vskip 0.15cm

None of these modules can analyze all of the geometry available in the code
DRAGON. In general, the restrictions that apply to a given tracking module
result directly from the approximation associated with this method. Moreover, in
other instances, some geometries which would have had the same tracking file
generated by two different method, such as tube geometry for the \moc{SYBILT:}
and \moc{EXCELT:} module, have been made available only to one of these tracking
module (module \moc{SYBILT:} in this case).

\vskip 0.15cm

The general information resulting from these
tracking is stored in a \dds{tracking} data structure.
For the \moc{EXCELT:} and \moc{NXT:} modules, an additional sequential binary
tracking file may be generated.

\vskip 0.15cm

The global numbering of the zones in a geometry proceeds following an
order of priorities given by:

\begin{itemize}

\item the different rings of a cylindrical or spherical region starting with the
inner most after mesh splitting;

\item for a cluster regions located in a ring, two different numbering schemes are possible. For the \moc{EXCELT:} 
module, one first numbers the region inside the pin in the same way as for cylindrical regions and finishes 
by associating the next region number to the shell of the global geometry which contains this pin. If two 
cluster types are located in a given ring, they are classified according to increasing \dusa{rpin} and \dusa{apin} and then 
numbered in this order. Cluster overlapping annular region are numbered before considering the annular 
regions. For the \moc{NXT:} module, each pin is numbered individually in a Cartesian region according to their 
ordered in the \moc{CLUSTER} keywords and then the Cartesian regions are numbered sequentially. A description 
of the explicit numbering of regions and surfaces can be found in report IGE-260.\cite{ige260}
 
\item the zones in ascending order corresponding to the first axial component
(normally $X$) after mesh splitting;

\item the zones in ascending order corresponding to the second axial component
(normally $Y$) after mesh splitting;

\item the hexagonal zones corresponding to the order described in
\Fig{s30} to \Fig{compl}.

\item  the sub-geometry of type \moc{CARCELX}, \moc{CARCELY} and
\moc{CARCELZ} are numbered assuming that the third component corresponds to 
$X$, $Y$ and $Z$ respectively.

\end{itemize}
 
We should also note that symmetry conditions implicitly force the grouping of
certain calculation zones.

\vskip 0.2cm

All the tracking operators of DRAGON share an identical general tracking data
structure defined as

\begin{DataStructure}{Structure \dstr{desctrack}}
$[$ \moc{EDIT} \dusa{iprint} $]$\\
$[$ \moc{TITL} \dusa{TITLE} $]$ \\
$[$ \moc{MAXR} \dusa{maxreg} $]$\\
$[$ $\{$ \moc{NORE} $|$ \moc{RENO} $|$ \moc{REND} $\}$ $]$
\end{DataStructure}

\noindent with

\begin{ListeDeDescription}{mmmmmmm}

\item[\moc{EDIT}] keyword used to modify the print level \dusa{iprint}.

\item[\dusa{iprint}] index used to control the printing of this operator. The
amount of output produced by this tracking operators will vary substantially
depending on the print level specified. For example,

\begin{itemize}

\item when \dusa{iprint}=0 no output is produced;

\item when \dusa{iprint}=1 a minimum amount of output is produced; the 
main geometry properties are printed (fixed default option);

\item when \dusa{iprint}$\ge$2 In addition to the information printed when
using \dusa{iprint}=1 the zone numbering (zones associated with a flux) is
printed;

\end{itemize}

\item[\moc{TITL}] keyword which allows the run title to be set.

\item[\dusa{TITLE}] the title associated with a DRAGON run. This
title may contain up to 72 characters. The default when \moc{TITL}  is not
specified is no title.

\item[\moc{MAXR}] keyword which permits the maximum number of regions to be
considered during a DRAGON run to be specified.

\item[\dusa{maxreg}] maximum dimensions of the problem to be considered.  The
default value is set to the number of regions previously computed by the
\moc{GEO:} module. However this value is generally insufficient if symmetries or
mesh-splitting are specified.

\item[\moc{NORE}] keyword to specify that the automatic normalization of the integration lines is deactivated.

\item[\moc{RENO}]  keyword to specify the activation of the {\sl direction-independent} normalization procedure of the
integration lines. The normalization factors are {\sl not} function of the subtracks directions. This option is only
valid for modules \moc{NXT:}, \moc{EXCELT:} and \moc{SALT:}. This is the default option for \moc{NXT:} and \moc{SALT:}
modules.

\item[\moc{REND}]  keyword to specify the activation of the {\sl direction-dependent} normalization procedure of the
integration lines. The normalization factors are function of the subtracks directions. This option is only valid for
modules \moc{NXT:}, \moc{EXCELT:} and \moc{SALT:}. This is the default option for \moc{EXCELT:} module.

\end{ListeDeDescription}
\eject

\subsubsection{The {\tt SYBILT:} tracking module}\label{sect:SYBILData}

The {\tt SYBILT:} module provides an implementation of the collision probability (PIJ) method in 1D geometries or of the interface current (IC) method
in 2D geometries. The geometries that can be treated by the module \moc{SYBILT:} are

\begin{enumerate}

\item The homogeneous geometry \moc{HOMOGE}.

\item The one-dimensional geometries \moc{SPHERE}, \moc{TUBE} and
\moc{CAR1D}.\cite{ALCOL}

\item The two-dimensional geometries \moc{CAR2D} and \moc{HEX} including
respectively \moc{CARCEL} and \moc{HEXCEL} sub-geometries as well as 
\moc{VIRTUAL}
sub-geometries. 

\item $S_{ij}$--type two-dimensional non-standard geometries.\cite{Apollo}

\item The double heterogeneity option.\cite{BIHET}

\end{enumerate}

The calling specification for this module is:

\begin{DataStructure}{Structure \dstr{SYBILT:}}
\dusa{TRKNAM}
\moc{:=} \moc{SYBILT:} $[$ \dusa{TRKNAM} $]$
\dusa{GEONAM} \moc{::} \dstr{desctrack} \dstr{descsybil}
\end{DataStructure}

\noindent  where
\begin{ListeDeDescription}{mmmmmmm}

\item[\dusa{TRKNAM}] {\tt character*12} name of the \dds{tracking} data
structure that will contain region volume and surface area vectors in
addition to region identification pointers and other tracking information.
If \dusa{TRKNAM} also appears on the RHS, the previous tracking 
parameters will be applied by default on the current geometry.

\item[\dusa{GEONAM}] {\tt character*12} name of the \dds{geometry} data
structure.

\item[\dstr{desctrack}] structure describing the general tracking data (see
\Sect{TRKData})

\item[\dstr{descsybil}] structure describing the transport tracking data
specific to \moc{SYBILT:}.

\end{ListeDeDescription}

\vskip 0.15cm

The \moc{SYBILT:} specific tracking data in \dstr{descsybil} is defined as

\begin{DataStructure}{Structure \dstr{descsybil}}
$[$ \moc{MAXJ} \dusa{maxcur} $]$  $[$ \moc{MAXZ} \dusa{maxint} $]$ \\
$[$ \moc{HALT} $]$ \\
$[$ \moc{QUA1} \dusa{iqua1} $]$ $[$ \moc{QUA2} \dusa{iqua2}
\dusa{nsegment} $]$ $[$ $\{$ \moc{EQW} $|$ \moc{GAUS} $\}$ $]$ \\
$[$ $\{$ \moc{ROTH} $|$ \moc{ROT+} $|$ \moc{DP00} $|$ \moc{DP01} $\}$ $]$ \\
$[$ $\{$ \moc{WIGN} $|$ \moc{ASKE} $|$ \moc{SANC} $\}$ $]$ $[$ \moc{LIGN} $]$
$[$ \moc{RECT} $]$ \\
$[$ \moc{EPSJ} \dusa{epsj} $]$ \\
$[~[$ \moc{QUAB} \dusa{iquab} $]~[~\{$ \moc{SAPO} $|$ \moc{HEBE} $|$ \moc{SLSI} $[$ \dusa{frtm} $]~\}~]~]$ \\
{\tt ;}
\end{DataStructure}

\noindent where

\begin{ListeDeDescription}{mmmmmmm}

\item[\moc{MAXJ}] keyword to specify the maximum number of interface currents
surrounding the blocks in the calculations. 

\item[\dusa{maxcur}] the maximum number of interface currents surrounding the
blocks. The default value is \dusa{maxcur}=max(18,4$\times$\dusa{maxreg}) for the
\moc{SYBILT:} module.

\item[\moc{MAXZ}] keyword to specify the maximum amount of memory required to
store the integration lines. An insufficiently large value can lead to an
execution failure (core dump).

\item[\dusa{maxint}] the maximum amount of memory required to store the
integration lines. The default value is \dusa{maxint}=10000.

\item[\moc{HALT}] keyword to specify that the program is to be stopped at the
end of the geometry calculations. This option permits the geometry inputs to be
checked, the number of blocks and interface currents to be calculated, and a
conservative estimate of the memory required for storing the tracks to be made
for mixed geometries.

\item[\moc{QUA1}] keyword to specify the one-dimensional integration
parameters.

\item[\dusa{iqua1}] number of basis points for the angular integration of the
blocks in a one-dimensional geometry. This parameter is not used for
\moc{CAR1D} geometries. If a Gauss-Legendre or Gauss-Jacobi quadrature is used,
the values of \dusa{iqua1} allowed are: 1 to 20, 24, 28, 32 or 64. The default
value is \dusa{iqua1}=5. 

\item[\moc{QUA2}] keyword to specify the two-dimensional integration
parameters.

\item[\dusa{iqua2}] number of basis points for the angular integration of the
blocks in a two-dimensional geometry appearing during assembly  
calculations. If a Gauss-Legendre or Gauss-Jacobi formula is used the values
allowed for \dusa{iqua2} are: 1 to 20, 24, 28, 32 or 64. The default value is
\dusa{iqua2}=3 and represents the number of angles in ($0,\pi/4$) for
Cartesian geometries and  ($0,\pi/6$) for hexagonal geometries. 

\item[\dusa{nsegment}] number of basis points for the spatial integration of
the blocks in a two-dimensional geometry appearing during assembly 
calculations. The values of \dusa{nsegment} allowed are: 1 to 10. The default
value is \dusa{nsegment}=3.

\item[\moc{EQW}] keyword to specify the use of equal-weight quadrature.

\item[\moc{GAUS}] keyword to specify the use of the Gauss-Legendre or the
Gauss-Jacobi quadrature. This is the default option.

\item[\moc{ROTH}] keyword to specify that the isotropic ($DP_{0}$) components
of the inter-cell current is used with the incoming current being averaged over
all the faces surrounding a cell. The global collision matrix is calculated in a
annular model. Only used when 2--d assembly of cells are considered.

\item[\moc{ROT+}] keyword to specify that the isotropic ($DP_{0}$) components
of the inter-cell current is used. The global collision matrix is calculated in
a annular model. Only used when 2--d assembly of cells are considered.

\item[\moc{DP00}] keyword to specify that the isotropic ($DP_{0}$) components
of the inter-cell current is used. The global collision matrix are computed
explicitly. Only used when 2--d assembly of cells are considered.

\item[\moc{DP01}] keyword to specify that the linearly anisotropic ($DP_{1}$)
components of the inter-cell current are used. This hypothesis implies 12
currents per cell in a cartesian geometry and 18 currents per cell for an
hexagonal geometry. Linearly anisotropic reflection is used. Only used when 2--d
assembly of cells are considered.

\item[\moc{WIGN}] keyword to specify the use of a {\sl Wigner} cylinderization
which preserves the volume of the external crown. This applies only in cases
where the external surface is annular using the \moc{ROTH} or \moc{ROT+}
options. Only used when 2--d assembly of cells are considered. Note that an
assembly of rectangular cells having unequal volumes cannot use a {\sl Wigner}
cylinderization.  

\item[\moc{ASKE}] keyword to specify the use of an {\sl Askew} cylinderization
which preserves both the external surface of the cells and the material balance
of the external crown (by a modification of its concentration). This applies
only in cases where the external surface is annular using the \moc{ROTH} or
\moc{ROT+} options. Only used when 2--d assembly of cells are considered. Note
that an assembly of rectangular cells having unequal volumes can use an
{\sl Askew} cylinderization.  

\item[\moc{SANC}] keyword to specify the use of a {\sl Sanchez} cylinderization.
This model uses a {\sl Wigner} cylinderization for computing the collision $P_{ij}$
and leakage $P_{iS}$ probabilities. However, the reciprocity and conservation
relations used to compute the incoming $P_{Sj}$ and transmission $P_{SS}$
probabilities are defined in the rectangular cell (with the exact
surface).\cite{SANCHEZ} 
This applies where the external surface is annular using the \moc{ROTH} or
\moc{ROT+} options. Only used when 2--d assembly of cells are considered. Note
that an assembly of rectangular cells having unequal volumes can use a
{\sl Sanchez} cylinderization. This is the default option.

\item[\moc{LIGN}] keyword to specify that all the integration lines are to be
printed. This option should only be used when absolutely necessary because it
generates a rather large amount of output. Only used when 2--d assembly of cells
are considered.

\item[\moc{RECT}] keyword to specify that square cells are to be treated as if
they were rectangular cells, with the inherent loss in performance that this
entails. This option is of purely academic interest.

\item[\moc{EPSJ}] keyword to specify the stopping criterion for the flux-current iterations of the
interface current method in case the {\tt ARM} keyword is set in the {\tt ASM:} module or in
a resonance self-shielding module ({\tt SHI:}, {\tt USS:}, etc.).

\item[\dusa{epsj}] the stopping criterion value. The default value is \dusa{epsj} $= 0.5 \times 10^{-5}$.

\item[\moc{QUAB}] keyword to specify the number of basis point for the
numerical integration of each micro-structure in cases involving double
heterogeneity (Bihet).

\item[\dusa{iquab}] the number of basis point for the numerical integration of
the collision probabilities in the micro-volumes using the  Gauss-Jacobi
formula. The values permitted are: 1 to 20, 24, 28, 32 or  64. The default value
is \dusa{iquab}=5. If \dusa{iquab} is negative, its absolute value will be used in the She-Liu-Shi approach to determine the
split level in the tracking used to compute the probability collisions.

\item[\moc{SAPO}] use the Sanchez-Pomraning double-heterogeneity model.\cite{sapo}

\item[\moc{HEBE}] use the Hebert double-heterogeneity model (default option).\cite{BIHET}

\item[\moc{SLSI}] use the She-Liu-Shi double-heterogeneity model without shadow effect.\cite{She2017}

\item[\dusa{frtm}] the minimum microstructure volume fraction used to compute the size of the equivalent cylinder in She-Liu-Shi approach. The default value is \dusa{frtm} $=0.05$.

\end{ListeDeDescription}
\eject
 % structure (sybilT)
\subsubsection{The {\tt EXCELT:} tracking module}\label{sect:EXCELLData}

The calling specification for this module is:

\begin{DataStructure}{Structure \dstr{EXCELT:}}
\dusa{TRKNAM} $[$ \dusa{TRKFIL} $]$
\moc{:=} \moc{EXCELT:} $[$ \dusa{TRKNAM} $]$ $[$ \dusa{TRKFIL} $]$ 
\dusa{GEONAM} \moc{::}  \dstr{desctrack} \dstr{descexcel}
\end{DataStructure}

\noindent  where
\begin{ListeDeDescription}{mmmmmmm}

\item[\dusa{TRKNAM}] {\tt character*12} name of the \dds{tracking} data
structure that will contain region volume and surface area vectors in
addition to region identification pointers and other tracking information.
If \dusa{TRKNAM} also appears on the RHS, the previous tracking 
parameters will be applied by default on the current geometry.

\item[\dusa{TRKFIL}] {\tt character*12} name of the sequential binary tracking
file  used to store the tracks lengths. If \dusa{TRKFIL} does not appear, the keyword
\moc{XCLL} is set automatically. If the user wants to use a tracking file,
\dusa{TRKFIL} is required for the \moc{EXCELT:} module, either on the LHS, on the RHS or on both sides. In
the case where \dusa{TRKFIL} appears on both LHS and RHS, the existing tracking
file is modified by the module while if \dusa{TRKFIL} appears only on the RHS,
the existing tracking file is read but not modified.

\item[\dusa{GEONAM}] {\tt character*12} name of the \dds{geometry} data
structure.

\item[\dstr{desctrack}] structure describing the general tracking data (see
\Sect{TRKData})

\item[\dstr{descexcel}] structure describing the transport tracking data
specific to \moc{EXCELT:}.

\end{ListeDeDescription}

\vskip 0.15cm

The \moc{EXCELT:} specific tracking data in \dstr{descexcel} is defined as

\begin{DataStructure}{Structure \dstr{descexcel}}
$[$ \moc{ANIS} \dusa{nanis} $]$ \\
$[~\{$ \moc{ONEG} $|$ \moc{ALLG} $[$ \moc{BATCH} \dusa{nbatch} $]~|$ \moc{XCLL} $\}~]$ \\
$[~\{$ \moc{TREG}  $|$ \moc{TMER} $\}~]$ \\
$[$ $\{$ \moc{PISO} $|$ \moc{PSPC} $[$ \moc{CUT} \dusa{pcut} $]$ $\}$ $]$ \\
$[~[$ \moc{QUAB} \dusa{iquab} $]~[~\{$ \moc{SAPO} $|$ \moc{HEBE} $|$ \moc{SLSI} $[$ \dusa{frtm} $]~\}~]~]$ \\
$[$ $\{$ \moc{PRIX} $|$  \moc{PRIY} $|$ \moc{PRIZ} $\}$ \dusa{denspr} $]$ \\
$[$ $\{$ \moc{LCMD} $|$ \moc{OPP1} $|$ \moc{OGAU} $|$ \moc{GAUS} $|$ \moc{CACA} $|$ \moc{CACB} $\}~[$ \dusa{nmu} $]~]$ \\
$[$ \moc{TRAK}  $\{$  \moc{TISO} \dusa{nangl} $[$ \dusa{nangl\_z} $]$ \dusa{dens} $[$ \dusa{dens\_z} $]~[$ \moc{CORN} 
\dusa{pcorn} $]$  $[$ \moc{SYMM} \dusa{isymm} $|$ \moc{NOSY} $]$ $|$ \\
\moc{TSPC} $[$ \moc{MEDI}  $]$ \dusa{nangl} \dusa{dens} $|$ \moc{HALT} $\}$ $]$ \\
{\tt ;}
\end{DataStructure}

\noindent
where

\begin{ListeDeDescription}{mmmmmmmm}

\item[\moc{ANIS}] keyword to specify the order of scattering anisotropy. 

\item[\dusa{nanis}] order of anisotropy in transport calculation.
A default value of 1 represents isotropic (or transport-corrected) scattering while a value of 2
correspond to linearly anisotropic scattering. When anisotropic scattering is considered, user should pay attention to the following points:
\begin{itemize}
\item the usage of \moc{DIAG}, \moc{SYME}, \moc{SSYM} keywords in the definition of the geometry is forbidden. Indeed, in \moc{EXCELT:}/\moc{NXT:} tracking procedures, the geometry is ``unfolded'' according to these symmetries : this is incompatible with the integration of the anisotropic moments of the flux; \\
\item the angular quadratures should be selected paying attention to the restrictions mentioned in this manual in order to ensure the particle conservation.
\end{itemize}

\item[\moc{ONEG}] keyword to specify that the tracking is read before computing each group-dependent collision
probability or algebraic collapsing matrix (default value if \dusa{TRKFIL} is set). The tracking file is
read in each energy group if the method of characteristics (MOC) is used.

\item[\moc{ALLG}] keyword to specify that the tracking is read once and the collision
probability or algebraic collapsing matrices are computed in many energy groups.  The tracking file is
read once if the method of characteristics (MOC) is used.
 
\item[\moc{XCLL}] keyword to specify that the tracking is computed {\sl on-demand} (it is not stored on a file) and the
collision probability matrices are computed in many energy groups. The tracking
file \dusa{TRKFIL} should {\sl not} be provided (default value if \dusa{TRKFIL} is not set).

\item[\moc{BATCH}] keyword to specify the number of tracks processed by each core for each energy group. OpenMP parallelization is processing each energy group on a different core. The default value is \dusa{nbatch} $=1$.

\item[\dusa{nbatch}] the number of tracks processed by each core. Usually, a value \dusa{nbatch} $\ge 100$ is recommended.

\item[\moc{TREG}] keyword to specify that the normalization procedure of the integration lines activated by keywords \moc{RENO}
or \moc{REND} in Sect.~\ref{sect:TRKData} is to be performed with respect of the fine volumes as specified in the {\tt KEYFLX} record
of the tracking object. This is the default option.

\item[\moc{TMER}] keyword to specify that the normalization procedure of the integration lines activated by keywords \moc{RENO}
or \moc{REND} in Sect.~\ref{sect:TRKData} is to be performed with respect of the {\sl merged volumes} as specified in the {\tt KEYMRG} record
of the tracking object.

\item[\moc{PISO}] keyword to specify that a collision probability calculation
with isotropic reflection boundary conditions is required. It is the default
option if a \moc{TISO} type integration is chosen. To obtain accurate
transmission probabilities for the isotropic case it is recommended that the
normalization options in the \moc{ASM:} module be used.

\item[\moc{PSPC}] keyword to specify that  a collision probability calculation
with specular reflection boundary conditions required; this is the default
option if a \moc{TSPC} type integration is chosen. This calculation is only
possible if the file was initially constructed using the \moc{TSPC} option. 

\item[\moc{CUT}] keyword to specify the input of cutting parameters for the
specular integration.

\item[\dusa{pcut}] real value representing the maximum error allowed on the
exponential function used for specular collision probability calculations.
Tracks will be cut at a length such that the error in the probabilities
resulting from this reduced track will be of the order of \dusa{pcut}. By
default, there is no cutting of the tracks and \dusa{pcut}=0.0. If this option
is used in an entirely reflected case, it is preferable to use the \moc{NORM}
command in the \moc{ASM:} module.

\item[\moc{QUAB}] keyword to specify the number of basis point for the
numerical integration of each micro-structure in cases involving double
heterogeneity (Bihet).

\item[\dusa{iquab}] the number of basis point for the numerical integration of
the collision probabilities in the micro-volumes using the  Gauss-Jacobi
formula. The values permitted are: 1 to 20, 24, 28, 32 or  64. The default value
is \dusa{iquab}=5. If \dusa{iquab} is negative, its absolute value will be used in the She-Liu-Shi approach to determine the
split level in the tracking used to compute the probability collisions.

\item[\moc{SAPO}] use the Sanchez-Pomraning double-heterogeneity model.\cite{sapo}

\item[\moc{HEBE}] use the Hebert double-heterogeneity model (default option).\cite{BIHET}

\item[\moc{SLSI}] use the She-Liu-Shi double-heterogeneity model without shadow effect.\cite{She2017}

\item[\dusa{frtm}] the minimum microstructure volume fraction used to compute the size of the equivalent cylinder in She-Liu-Shi approach. The default value is \dusa{frtm} $=0.05$.

\item[\moc{PRIX}] keyword to specify that a prismatic tracking is considered for a 3D geometry invariant along the $x-$ axis. In this case, the 3D geometry is projected in the $y-z$ plane and a 2D tracking on the projected geometry is performed. This capability is limited to the non-cyclic method of characteristics solver for the time being and a subsequent call to \moc{MCCGT:} is mandatory.

\item[\moc{PRIY}] keyword to specify that a prismatic tracking is considered for a 3D geometry invariant along the $y-$ axis. In this case, the 3D geometry is projected in the $z-x$ plane and a 2D tracking on the projected geometry is performed. This capability is limited to the method of characteristics solver for the time being and a subsequent call to \moc{MCCGT:} is mandatory.

\item[\moc{PRIZ}] keyword to specify that a prismatic tracking is considered for a 3D geometry invariant along the $z-$ axis. In this case, the 3D geometry is projected in the $x-y$ plane and a 2D tracking on the projected geometry is performed. This capability is limited to the method of characteristics solver for the time being and a subsequent call to \moc{MCCGT:} is mandatory.

\item[\dusa{denspr}] real value representing the linear track density (in cm$^{-1}$) to be used for the inline contruction of 3D tracks from 2D tracking when a prismatic tracking is considered.

\item[\moc{LCMD}] keyword to specify that optimized (McDaniel--type) polar integration angles are to be
selected for the polar quadrature when a prismatic tracking is considered.\cite{LCMD} This is the default option. The conservation is ensured only for isotropic scattering.

\item[\moc{OPP1}] keyword to specify that $P_1$ constrained optimized (McDaniel--type) polar integration angles are to be selected for the polar quadrature when a prismatic tracking is considered.\cite{LeTellierpa} The conservation is ensured only for isotropic and linearly anisotropic scattering.

\item[\moc{OGAU}] keyword to specify that Optimized Gauss polar integration angles are to be
selected for the method of characteristics.\cite{LCMD,LeTellierpa} The conservation is ensured up to $P_{\dusa{nmu}-1}$ scattering.

\item[\moc{GAUS}] keyword to specify that Gauss-Legendre polar integration angles are to be selected for the polar quadrature when a prismatic tracking is considered. The conservation is ensured up to $P_{\dusa{nmu}-1}$ scattering.

\item[\moc{CACA}] keyword to specify that CACTUS type equal weight polar integration angles are to be
selected for the polar quadrature when a prismatic tracking is considered.\cite{CACTUS} The conservation is ensured only for isotropic scattering.

\item[\moc{CACB}] keyword to specify that CACTUS type uniformly distributed integration polar angles
are to be selected for the polar quadrature when a prismatic tracking is considered.\cite{CACTUS} The conservation is ensured only for isotropic scattering.

\item[\dusa{nmu}] user-defined number of polar angles. By default, a value consistent with \dusa{nangl} is computed by the code. For \moc{LCMD}, \moc{OPP1}, \moc{OGAU} quadratures, \dusa{nmu} is limited to 2, 3 or 4.

\item[\moc{TRAK}] keyword to specify the tracking parameters to be used. 

\item[\moc{TISO}] keyword to specify that isotropic tracking parameters will
be supplied. This is the default tracking option for cluster geometries.


\item[\moc{TSPC}] keyword to specify that specular tracking parameters will be
supplied.

\item[\moc{MEDI}] keyword to specify that instead of selecting the angles
located at the end of each angular interval, the angles located in the middle of
these intervals are selected. This is particularly useful if one wants to avoid
tracking angles that are parallel to the $X-$ or $Y-$axis as its is the case
when the external region of a \moc{CARCEL} geometry is voided.

\item[\dusa{nangl}] angular quadrature parameter. For applications involving
3--D cells, the choices are  \dusa{nangl}=2, 4, 8, 10, 12, 14 or 16; these
angular quadratures  $EQ_{n}$ present a rotational symmetry about the three
cartesian axes. For 2--D isotropic  applications, any value of  \dusa{nangl} $\ge 2$ may
be used; equidistant angles will be selected. For 2--D specular applications the
input value must be of the form $p+1$ where $p$ is a prime number (for example
$p$=7, 11, etc.); the choice of \dusa{nangl} = 8, 12, 14, 18, 20, 24, or 30 are
allowed. For cluster type geometries the default value is \dusa{nangl}=10 for
isotropic cases and \dusa{nangl}=12 for specular cases.

\item[\dusa{nangl\_z}] angular quadrature parameter in the axial $Z$ direction. Used only
with \dusa{HEXZ} and \dusa{HEXCELZ} geometries.

\item[\dusa{dens}] real value representing the density of the integration
lines (in $cm^{-1}$ for 2--D cases and $cm^{-2}$ for 3--D cases). This choice of
density along the plan perpendicular to each angle depends on the geometry of
the cell to be analyzed. If there are zones of very small volume, a high line
density is essential. This value will be readjusted by \moc{EXCELT:}. In the case
of the analysis of a cluster type geometry the default value of this parameter
is $5/r_{m}$ where $r_{m}$ is the minimum radius of the pins or the
minimum thickness of an annular ring in the geometry. If the selected value of \dusa{dens}
is too small, some volumes or surfaces may not be tracked.

\item[\dusa{dens\_z}] real value representing the density of the integration
lines in the axial $Z$ direction. Used only with \dusa{HEXZ} and \dusa{HEXCELZ} geometries.

\item[\moc{CORN}] keyword to specify that the input of the parameters used to
treat the corners for the isotropic integration.

\item[\dusa{pcorn}] maximum distance (cm) between a line and the intersection
of $n\ge 2$ external surfaces where track redistribution will take place. Track
redistribution will take place if a line comes close to the intersection of
$n\ge 2$ external surfaces. In this case the line will be replicated $n$ times,
each of these lines being associated with a different external surface, while
its weight is reduced by a factor of $1/n$. This allows for a better
distribution of tracks which are relatively close to $n$ external surfaces. By
default, there is no treatment of the corners and \dusa{pcorn}=0.0.

\item[\moc{SYMM}] keyword to specify that the geometry has a rotation
symmetry.

\item[\dusa{isymm}] integer value describing the rotation symmetry of the
geometry. The fixed default of this parameter is 1.

\item[\moc{NOSY}] \moc{EXCELT:} automatically try to take into account
geometric symmetries in order to reduce the number of tracks and the CPU
time. The \moc{NOSY} keyword desactivates this automatic capability.

\item[\moc{HALT}] keyword to specify that the program is to be stopped after
the analysis of the geometry, without the explicit tracking being performed.

\end{ListeDeDescription}
\eject
 % structure (excellT)
\subsubsection{The {\tt NXT:} tracking module}\label{sect:NXTData}

The calling specification for this module is:

\begin{DataStructure}{Structure \dstr{NXT:}}
$[$ \dusa{TRKFIL} $]$ \dusa{TRKNAM}
\moc{:=} \moc{NXT:} $[$ \dusa{TRKNAM} $]~[$ \dusa{GEONAM} $]$ \moc{::} \dstr{desctrack} \dstr{descnxt}
\end{DataStructure}

\noindent  where
\begin{ListeDeDescription}{mmmmmmm}

\item[\dusa{TRKNAM}] {\tt character*12} name of the \dds{tracking} data
structure that will contain region volume and surface area vectors in
addition to region identification pointers and other tracking information.
If \dusa{TRKNAM} also appears on the RHS, the previous tracking 
parameters will be applied by default on the current geometry.

\item[\dusa{TRKFIL}] {\tt character*12} name of the sequential binary tracking
file  used to store the tracks lengths. If \dusa{TRKFIL} does not appear, the keyword
\moc{XCLL} is set automatically. If the user wants to use a tracking file,
\dusa{TRKFIL} is required.

\item[\dusa{GEONAM}] {\tt character*12} name of the \dds{geometry} data
structure.

\item[\dstr{desctrack}] structure describing the general tracking data (see
\Sect{TRKData})

\item[\dstr{descnxt}] structure describing the transport tracking data
specific to \moc{NXT:}.

\end{ListeDeDescription}

\vskip 0.15cm

The \moc{NXT:} specific tracking data in \dstr{descnxt} is defined as

\begin{DataStructure}{Structure \dstr{descnxt}}
$[$ \moc{ANIS} \dusa{nanis} $]$ \\
$[~\{$  \moc{ONEG} $|$ \moc{ALLG} $[$ \moc{BATCH} \dusa{nbatch} $]~|$ \moc{XCLL} $\}~]$ \\
$[~[$ \moc{QUAB} \dusa{iquab} $]~[~\{$ \moc{SAPO} $|$ \moc{HEBE} $|$ \moc{SLSI} $[$ \dusa{frtm} $]~\}~]~]$ \\
$[~\{$ \moc{PISO} $|$ \moc{PSPC} $[$ \moc{CUT} \dusa{pcut} $]$ $\}~]$ \\
$[$ $\{$ \moc{SYMM} \dusa{isymm} $|$ \moc{NOSY} $]$ \\
$[$ $\{$ \moc{GAUS}  $|$ \moc{CACA} $|$ \moc{CACB} $|$ \moc{LCMD} $|$ \moc{OPP1} $|$ \moc{OGAU} $\}~[$ \dusa{nmu} $]~]$ \\
$\{$ \moc{TISO} $[~\{$ \moc{EQW} $|$ \moc{GAUS} $|$ \moc{PNTN} $|$ \moc{SMS} $|$ \moc{LSN} $|$ \moc{QRN} $\}~]$ \dusa{nangl} \dusa{dens} $[$ \moc{CORN} 
\dusa{pcorn} $]$ \\
$~~~~~|$ \moc{TSPC} $[~\{$ \moc{EQW} $|$ \moc{MEDI} $|$ \moc{EQW2} $\}~]$ \dusa{nangl} \dusa{dens} $\}$ \\
$[~\{$ \moc{NOTR} $|$ \moc{MC} $\}~]$\\
$[$ \moc{NBSLIN} \dusa{nbslin} $]$ \\
$[$ \moc{MERGMIX} $]$\\
$[$ \moc{LONG} $]$\\
$[$ \moc{PRIZ} \dusa{denspr} $]$ \\
{\tt ;}
\end{DataStructure}

\noindent
where

\begin{ListeDeDescription}{mmmmmmmm}

\item[\moc{ANIS}] keyword to specify the order of scattering anisotropy. 

\item[\dusa{nanis}] order of anisotropy in transport calculation.
A default value of 1 represents isotropic (or transport-corrected) scattering while a value of 2
correspond to linearly anisotropic scattering. When anisotropic scattering is considered, user should pay attention to the following points:
\begin{itemize}
\item the usage of \moc{DIAG}, \moc{SYME}, \moc{SSYM} keywords in the definition of the geometry is forbidden. Indeed, in \moc{EXCELT:}/\moc{NXT:} tracking procedures, the geometry is ``unfolded'' according to these symmetries : this is incompatible with the integration of the anisotropic moments of the flux; \\
\item an angular dependent normalization of the track lengths should be requested in the tracking procedure (\moc{REND} keyword) in order to ensure the particle conservation; \\
\item the angular quadratures should be selected paying attention to the restrictions mentioned in this manual in order to ensure the particle conservation.
\end{itemize}

\item[\moc{ONEG}] keyword to specify that the tracking is read before computing each group-dependent collision
probability or algebraic collapsing matrix (default value if \dusa{TRKFIL} is set). The tracking file is
read in each energy group if the method of characteristics (MOC) is used.

\item[\moc{ALLG}] keyword to specify that the tracking is read once and the collision
probability or algebraic collapsing matrices are computed in many energy groups.  The tracking file is
read once if the method of characteristics (MOC) is used.
 
\item[\moc{XCLL}] keyword to specify that the tracking is computed {\sl on-demand} (it is not stored on a file) and the
collision probability matrices are computed in many energy groups. The tracking
file \dusa{TRKFIL} should {\sl not} be provided (default value if \dusa{TRKFIL} is not set).

\item[\moc{BATCH}] keyword to specify the number of tracks processed by each core for each energy group. OpenMP parallelization is processing each energy group on a different core. The default value is \dusa{nbatch} $=1$.

\item[\dusa{nbatch}] the number of tracks processed by each core. Usually, a value \dusa{nbatch} $\ge 100$ is recommended.

\item[\moc{QUAB}] keyword to specify the number of basis point for the
numerical integration of each micro-structure in cases involving double
heterogeneity (Bihet).

\item[\dusa{iquab}] the number of basis point for the numerical integration of
the collision probabilities in the micro-volumes using the Gauss-Jacobi
formula. The values permitted are: 1 to 20, 24, 28, 32 or 64. The default value
is \dusa{iquab} = 5. If \dusa{iquab} is negative, its absolute value will be used in the She-Liu-Shi approach to determine the
split level in the tracking used to compute the probability collisions.

\item[\moc{SAPO}] use the Sanchez-Pomraning double-heterogeneity model.\cite{sapo}

\item[\moc{HEBE}] use the Hebert double-heterogeneity model (default option).\cite{BIHET}

\item[\moc{SLSI}] use the She-Liu-Shi double-heterogeneity model without shadow effect.\cite{She2017}

\item[\dusa{frtm}] the minimum microstructure volume fraction used to compute the size of the equivalent cylinder in She-Liu-Shi approach. The default value is \dusa{frtm} $=0.05$.

\item[\moc{PISO}] keyword to specify that a collision probability calculation with isotropic reflection boundary 
conditions is required. It is the default option if a \moc{TISO} type integration is chosen. To obtain accurate
transmission probabilities for the isotropic case it is recommended that the normalization 
options in the \moc{ASM:} module be used. 

\item[\moc{PSPC}] keyword to specify that a collision probability calculation with mirror like reflection or periodic 
boundary conditions is required; this is the default option if a \moc{TSPC} type integration is chosen. 
This calculation is only possible if the file was initially constructed using the \moc{TSPC} option. 

\item[\moc{CUT}] keyword to specify the input of cutting parameters for the specular collision probability
of characteristic integration. 

\item[\dusa{pcut}] real value representing the maximum error allowed on the exponential function used
for specular collision probability calculations. Tracks will be cut at a length such that the error in the 
probabilities resulting from this reduced track will be of the order of pcut. By default, the tracks 
are extended to infinity and \dusa{pcut} = 0.0. If this option is used in an entirely reflected case, it is 
recommended to use the \moc{NORM} command in the \moc{ASM:} module. 

\item[\moc{SYMM}] keyword to specify the level to which the tracking will respect the symmetry of the geometry. 

\item[\dusa{isymm}]  level to which the tracking will respect the symmetry of the geometry. For 2-D and 3-D Cartesian geometries it must takes the form \dusa{isymm}=$2 S_{x}+4S_{y}+16 S_{z}$ where
\begin{itemize}
\item $S_{x}=1$ if the $X$ symmetry is to be considered and $S_{x}=0$ otherwise.   
\item $S_{y}=1$ if the $Y$ symmetry is to be considered and $S_{y}=0$ otherwise.   
\item $S_{z}=1$ if the $Z$ symmetry is to be considered and $S_{z}=0$ otherwise.   
\end{itemize}

\item[\moc{NOSY}] keyword to specify the full tracking will take place irrespective of the symmetry of the geometry. This is equivalent to specifying \dusa{isymm}=0.

\item[\moc{GAUS}] keyword to specify that Gauss-Legendre polar integration angles are to be selected for the polar quadrature when a prismatic tracking is considered. The conservation is ensured up to $P_{\dusa{nmu}-1}$ scattering.

\item[\moc{CACA}] keyword to specify that CACTUS type equal weight polar integration angles are to be
selected for the polar quadrature when a prismatic tracking is considered.\cite{CACTUS} The conservation is ensured only for isotropic scattering.

\item[\moc{CACB}] keyword to specify that CACTUS type uniformly distributed integration polar angles
are to be selected for the polar quadrature when a prismatic tracking is considered.\cite{CACTUS} The conservation is ensured only for isotropic scattering.

\item[\moc{LCMD}] keyword to specify that optimized (McDaniel--type) polar integration angles are to be
selected for the polar quadrature when a prismatic tracking is considered.\cite{LCMD} This is the default option. The conservation is ensured only for isotropic scattering.

\item[\moc{OPP1}] keyword to specify that $P_1$ constrained optimized (McDaniel--type) polar integration angles are to be selected for the polar quadrature when a prismatic tracking is considered.\cite{LeTellierpa} The conservation is ensured only for isotropic and linearly anisotropic scattering.

\item[\moc{OGAU}] keyword to specify that Optimized Gauss polar integration angles are to be
selected for the method of characteristics.\cite{LCMD,LeTellierpa} The conservation is ensured up to $P_{\dusa{nmu}-1}$ scattering.

\item[\dusa{nmu}]  user-defined number of polar angles. By default, a value consistent with \dusa{nangl} is computed by the code. For \moc{LCMD}, \moc{OPP1}, \moc{OGAU} quadratures, \dusa{nmu} is limited to 2, 3 or 4.

\item[\moc{TISO}] keyword to specify that isotropic tracking parameters will be supplied. This is the
default tracking option for cluster geometries. 

\item[\moc{TSPC}] keyword to specify that specular tracking parameters will be supplied.

\item[\moc{EQW}] keyword to specify the use of equal weight quadrature.\cite{eqn} The conservation is ensured up to $P_{\dusa{nangl}/2}$ scattering.

\item[\moc{GAUS}] (after \moc{TISO} keyword) keyword to specify the use of the Gauss-Legendre quadrature. This option is valid only if an 
hexagonal geometry is considered.

\item[\moc{PNTN}] keyword to specify that Legendre-Techbychev quadrature quadrature will be selected.\cite{pntn} The conservation is ensured only for isotropic and linearly anisotropic scattering.

\item[\moc{SMS}] keyword to specify that Legendre-trapezoidal quadrature quadrature will be selected.\cite{sms} The conservation is ensured up to $P_{\dusa{nangl}-1}$ scattering.

\item[\moc{LSN}] keyword to specify the use of the $\mu_1$--optimized level-symmetric quadrature. The conservation is ensured up to $P_{\dusa{nangl}/2}$ scattering.

\item[\moc{QRN}] keyword to specify the use of the quadrupole range (QR) quadrature.\cite{quadrupole}

\item[\moc{MEDI}] keyword to specify the use of a median angle quadrature in \moc{TSPC} cases. Instead of
selecting the angles located at the end of each angular interval, the angles located in the middle of
these intervals are selected. This is particularly useful if one wants to avoid
tracking angles that are parallel to the $X-$ or $Y-$axis as its is the case
when the external region of a \moc{CARCEL} geometry is voided.

\item[\moc{EQW2}] keyword to eliminate angles $\phi=0$ and $\phi=\pi/2$ from the \moc{EQW} quadrature in \moc{TSPC} cases.

\item[\dusa{nangl}] angular quadrature parameter. For a 3-D \moc{EQW} option, the choices are \dusa{nangl} = 2, 4, 8, 10, 12, 14 
or 16. For a 3-D \moc{PNTN} or \moc{SMS} option, \dusa{nangl} is an even number smaller than 46.\cite{ige260} For 2-D 
isotropic applications, any value of \dusa{nangl} may be used, equidistant angles will be selected.

For 2-D specular applications the input value must be of the form $p + 1$ where $p$ is a prime number, as proposed
in Ref.~\citen{DragonPIJS3}. In this case, the choice of \dusa{nangl} = 2, 8, 12, 14, 18, 20, 24, or 30 are allowed. For
a rectangular Cartesian domain of size $X \times Y$, the azimuthal angles in $(0,\pi/2)$ interval are obtained from formula
\begin{align*}
\phi_k=\begin{cases}
\arctan\left(\frac{kY}{(p-k)X}\right)  \, , \ \ k=0,\, 1,\, 2,\, \dots, \, p & \text{if {\tt EWQ} (default)}\\
\arctan\left(\frac{kY}{(2p+2-k)X}\right) \, , \ \ k=1,\, 3,\, 5, \, \dots, \, 2p+1 & \text{if {\tt MEDI}} \\
\arctan\left(\frac{kY}{(p+2-k)X}\right) \, , \ \ k=1,\, 2,\, 3, \, \dots, \, p+1 &\text{if {\tt EQW2}.}\\
\end{cases}
\end{align*}

\item[\dusa{dens}] real value representing the density of the integration lines (in cm$^{-1}$ for 2-D Cartesian cases and 
3-D hexagonal cases and cm$^{-2}$ for 3-D cases Cartesian cases). This choice of density along the 
plan perpendicular to each angle depends on the geometry of the cell to be analyzed. If there 
are zones of very small volume, a high line density is essential. This value will be readjusted by 
\moc{NXT:}.

\item[\moc{CORN}] keyword to specify that the input of the parameters used to treat the corners for the isotropic 
integration. 

\item[\dusa{pcorn}] maximum distance (cm) between a line and the intersection of $n\ge 2$ external surfaces where 
track redistributing will take place. Track redistribution will take place if a line comes close to 
the intersection of $n \ge 2$ external surfaces. In this case the line will be replicated $n$ times, each 
of these lines being associated with a different external surface, while its weight is reduced by 
a factor of $1/n$. This allows for a better distribution of tracks which are relatively close to $n$ 
external surfaces. By default, there is no treatment of the corners and \dusa{pcorn} = 0.0.

\item[\moc{NOTR}] keyword to specify that the geometry will not be tracked. This is useful for 2-D geometries 
to generate a tracking data structure that can be used by the \moc{PSP:} module (see \Sect{PSPData}). 
One can then verify visually if the geometry is adequate before the tracking process as such is 
undertaken.

\item[\moc{MC}] keyword to specify that the geometry will not be tracked and that object \dusa{TRKNAM} will be used with the
Monte-Carlo method. This option is similar to \moc{NOTR} with additional information being added into \dusa{TRKNAM}.

\item[\moc{NBSLIN}] keyword to set the maximum number of segments in a single tracking line.

\item[\dusa{nbsl}] integer value representing the maximum number of segments in a single tracking line. The default value is \dusa{nbsl} = 100000.

\item[\moc{MERGMIX}] keyword to specify that all regions belonging to the same mixture will be merged together. This option should only be used as an attempt to reduce CPU costs in resonance self-shielding calculations.

\item[\moc{LONG}] keyword to specify that a ``long'' tracking file will be generated. This option is required if the tracking file is to be used by the \moc{TLM:} module (see \Sect{TLMData}).

\item[\moc{PRIZ}] keyword to specify that a prismatic tracking is considered for a 3D geometry invariant along the $z-$ axis. In this case, the 3D geometry is projected in the $x-y$ plane and a 2D tracking on the projected geometry is performed. This capability is limited to the non-cyclic method of characteristics solver for the time being and a subsequent call to \moc{MCCGT:} is mandatory.

\item[\dusa{denspr}] real value representing the linear track density (in cm$^{-1}$) to be used for the inline contruction of 3D tracks from 2D tracking when a prismatic tracking is considered.

\end{ListeDeDescription}
\clearpage
 % structure (nxtT)
\subsubsection{The {\tt MCCGT:} tracking module}\label{sect:MCCGData}

This module {\sl must} follow a call to module \moc{EXCELT:} or \moc{NXT:}. Its calling
specification is:

\begin{DataStructure}{Structure \dstr{MCCGT:}}
\dusa{TRKNAM} \moc{:=} \moc{MCCGT:} \dusa{TRKNAM} \dusa{TRKFIL} 
$[$ \dusa{GEONAM} $]$ \moc{::} \dstr{descmccg}
\end{DataStructure}

\noindent  where
\begin{ListeDeDescription}{mmmmmmm}

\item[\dusa{TRKNAM}] {\tt character*12} name of the \dds{tracking} data
structure that will contain region volume and surface area vectors in
addition to region identification pointers and other tracking information. It is provided by \moc{EXCELT:} or \moc{NXT:} operator and modified by \moc{MCCGT:} operator.

\item[\dusa{TRKFIL}] {\tt character*12} name of the sequential binary tracking file used to store the tracks lengths. This file is provided by \moc{EXCELT:} or \moc{NXT:} operator and used without modification by \moc{MCCGT:} operator.

\item[\dusa{GEONAM}] {\tt character*12} name of the optional \dds{geometry} data
structure. This structure is only required to recover double-heterogeneity data.

\item[\dstr{descmccg}] structure describing the transport tracking data
specific to \moc{MCCGT:}.

\end{ListeDeDescription}

\vskip 0.15cm

The \moc{MCCGT:} specific tracking data in \dstr{descmccg} is defined as

\begin{DataStructure}{Structure \dstr{descmccg}}
$[$ \moc{EDIT} \dusa{iprint} $]$ \\
$[$ $\{$ \moc{LCMD} $|$ \moc{OPP1} $|$ \moc{OGAU} $|$ \moc{GAUS} $|$ \moc{DGAU} $|$ \moc{CACA} $|$ \moc{CACB} $\}~[$ \dusa{nmu} $]~]$ \\
$\{$ \moc{DIFC} $[~\{$ \moc{NONE} $|$ \moc{DIAG} $|$ \moc{FULL} $|$ \moc{ILU0} $\}~]$ $~[$ \moc{TMT} $]$ $~[$ \moc{LEXA} $]$ \\
$~~~~~|$ \\
$~~~~~[~[$ \moc{AAC} \dusa{iaca} $[~\{$ \moc{NONE} $|$ \moc{DIAG} $|$ \moc{FULL} $|$ \moc{ILU0} $\}~]~[$ \moc{TMT} $]~]~[$ \moc{SCR} \dusa{iscr} $]~[$ \moc{LEXA} $]~]$ \\
$~~~~~[$ \moc{KRYL} \dusa{ikryl} $]$ \\
$~~~~~[$ \moc{MCU} \dusa{imcu} $]$ \\
$~~~~~[$ \moc{HDD} \dusa{xhdd} $]$ \\
$~~~~~[~\{$ \moc{SC} $|$ \moc{LDC} $\}~]$ \\
$~~~~~[$ \moc{LEXF} $]$ \\
$~~~~~[$ \moc{STIS} \dusa{istis} $]$ \\
$\}~$ \\
$~~~~~[$ \moc{MAXI} \dusa{nmaxi} $]$ \\
$~~~~~[$ \moc{EPSI} \dusa{xepsi} $]$ \\
$~~~~~[$ \moc{ADJ} $]$ \\
 {\tt ;}
\end{DataStructure}

\noindent
where

\begin{ListeDeDescription}{mmmmmmmm}

\item[\moc{EDIT}] keyword used to modify the print level iprint.

\item[\dusa{iprint}] index used to control the printing in this operator.

\item[\moc{LCMD}] keyword to specify that optimized (McDaniel--type) polar integration angles are to be
selected for the method of characteristics.\cite{LCMD} The conservation is ensured only for isotropic scattering.

\item[\moc{OPP1}] keyword to specify that $P_1$ constrained optimized (McDaniel--type) polar integration angles are to be selected for the method of characteristics.\cite{LeTellierpa} The conservation is ensured only for isotropic and linearly anisotropic scattering.

\item[\moc{OGAU}] keyword to specify that Optimized Gauss polar integration angles are to be
selected for the method of characteristics.\cite{LCMD,LeTellierpa} The conservation is ensured up to $P_{\dusa{nmu}-1}$ scattering.

\item[\moc{GAUS}] keyword to specify that the polar integration angles are to be selected as a single Gauss-Legendre quadrature for the method of characteristics in interval ($-\pi/2$, $\pi/2$). The conservation is ensured up to $P_{\dusa{nmu}-1}$ scattering. This is the default option.

\item[\moc{DGAU}] keyword to specify that the polar integration angles are to be selected as a double Gauss-Legendre quadrature for the method of characteristics in intervals ($-\pi/2$, $0$) and ($0$, $\pi/2$). The conservation is ensured up to $P_{\dusa{nmu}-1}$ scattering.

\item[\moc{CACA}] keyword to specify that CACTUS type equal weight polar integration angles are to be
selected for the method of characteristics.\cite{CACTUS} The conservation is ensured only for isotropic scattering.

\item[\moc{CACB}] keyword to specify that CACTUS type uniformly distributed integration polar angles
are to be selected for the method of characteristics.\cite{CACTUS} The conservation is ensured only for isotropic scattering.

\item[\dusa{nmu}] user-defined number of polar angles for the integration of the tracks with the method of characteristics for 2D geometries. By default, a value consistent with \dusa{nangl} is computed by the code. For \moc{LCMD}, \moc{OPP1}, \moc{OGAU} quadratures, \dusa{nmu} is limited to 2, 3 or 4.

\item[\moc{DIFC}] keyword used to specify that only an ACA-simplified transport flux calculation is to be performed (not by default). In this case, the maximum
number of ACA iterations is set to \dusa{nmaxi}.

\item[\moc{LEXA}] keyword used to force the usage of exact exponentials in the preconditioner calculation (not by default).

\item[\moc{MAXI}] keyword to specify the maximum number of scattering iterations performed in each energy group. This keyword is also used to set the number of Bi-CGSTAB iterations to solve the ACA-simplified system if \moc{DIFC} is present.

\item[\dusa{nmaxi}] the maximum number of iterations. The default value is \dusa{nmaxi}=20.

\item[\moc{EPSI}] keyword to specify the convergence criterion on inner
iterations (or ACA-simplified flux calculation if \moc{DIFC} is present).

\item[\dusa{xepsi}] convergence criterion. The default value is \dusa{xepsi}=1.0$\times$10$^{-5}$.

\item[\moc{AAC}] keyword to set the ACA preconditioning of inner/multigroup
iterations in case where a transport solution is selected.\cite{cdd,suslov2}

\item[\dusa{iaca}] $0$/$>0$: ACA preconditioning of inner or multigroup iterations off/on. The default value is \dusa{iaca}=1. If \moc{MAXI} is set to 1, ACA is used as a rebalancing technique for multigroup-inner mixed iterations and \dusa{iaca} is the maximum number of iterations allowed to solve the ACA system (e.g. 100).

\item[\moc{NONE}] no preconditioning for the iterative resolution by Bi-CGSTAB of the ACA system.

\item[\moc{DIAG}] diagonal preconditioning for the iterative resolution by Bi-CGSTAB of the ACA system.

\item[\moc{FULL}] full-matrix preconditioning for the iterative resolution by Bi-CGSTAB of the ACA system.

\item[\moc{ILU0}] ILU0 preconditioning for the iterative resolution by Bi-CGSTAB of the ACA system (This is the default option).

\item[\moc{TMT}] two-step collapsing version of ACA which uses a tracking merging technique while building the ACA matrices. 

\item[\moc{SCR}] keyword to set the SCR preconditioning of inner/multigroup
iterations.\cite{gmres}

\item[\dusa{iscr}] $0$/$>0$: SCR preconditioning of inner or multigroup iterations off/on. The default value is \dusa{iscr}=0. If \moc{MAXI} is set to 1, SCR is used as a rebalancing technique for multigroup-inner mixed iterations and \dusa{iscr} is the maximum number of iterations allowed to solve the SCR system. When anisotropic scattering is considered, SCR provides an acceleration of anisotropic flux moments. If both ACA and SCR are selected (\dusa{iscr}$>0$ and \dusa{iaca}$>0$), a two-step acceleration scheme (equivalent to ACA when isotropic scattering is considered) involving both methods is used.

\item[\moc{KRYL}] keyword to enable the Krylov acceleration of scattering iterations performed in each energy group.\cite{gmres}

\item[\dusa{ikryl}] $0$: GMRES/Bi-CGSTAB acceleration not used; $>0$: dimension of the Krylov subspace in GMRES; $<0$: Bi-CGSTAB is used.
The default value is \dusa{ikryl}=10.

\item[\moc{MCU}] keyword used to specify the maximum dimension of the connection matrix for memory allocation.

\item[\dusa{imcu}] The default value is eight (resp. twelve) times the number of volumes and external surfaces for 2D (resp. 3D) geometries.

\item[\moc{HDD}] keyword to select the integration scheme along the tracking lines.

\item[\dusa{xhdd}] selection criterion:

$$
xhdd = \left\{
\begin{array}{rl}
 0.0 & \textrm{step characteristics scheme} \\
>0.0 & \textrm{diamond differencing scheme.}
\end{array} \right.
$$

The default value is \dusa{xhdd}=0.0 so that the step characteristics method is used.

\item[\moc{LEXF}] keyword used to force the usage of exact exponentials in the flux calculation (not by default).

\item[\moc{SC}] keyword used to select the step characteristics (SC) or DD0 diamond differencing approximation. This
option is a flat source approximation (default option).

\item[\moc{LDC}] keyword used to select the linear discontinuous characteristics (LDC) or DD1 diamond differencing approximation. This
option is a linear source approximation.

\item[\moc{STIS}] keyword to select the tracking integration strategy.

\item[\dusa{istis}] $0$: a direct approach with asymptotical treatment is used; $1$: a ``source term isolation'' approach with asymptotical treatment is used (this technique tends to reduce the computational cost and increase the numerical stability but requires the calculation of angular mode-to-mode self-collision probabilities); $-1$:  an "MOCC/MCI"-like approach is used (it tends to reduce further more the computational cost as it doesn't feature any asymptotical treatment for vanishing optical thicknesses). Note that when a zero total cross section is found with \dusa{istis}=-1, it is reset to 1. The default value is \dusa{istis}=1 for $P_{L \le 3}$ anisotropy and 0 otherwise.

\item[\moc{ADJ}] keyword to select an adjoint solution of ACA and characteristics systems. A direct solution is
set by default.

\end{ListeDeDescription}
\eject
 % structure (mccgT)
\subsubsection{The {\tt SNT:} tracking module}\label{sect:SNData}

The {\tt SNT:} module can process one-dimensional, two-dimensional regular geometries and three-dimensional Cartesian geometries
of type \moc{CAR1D}, \moc{TUBE}, \moc{SPHERE}, \moc{CAR2D}, \moc{TUBEZ} and \moc{CAR3D}.

\vskip 0.2cm

The calling specification for this module is:

\begin{DataStructure}{Structure \dstr{SNT:}}
\dusa{TRKNAM}
\moc{:=} \moc{SNT:} $[$ \dusa{TRKNAM} $]$ 
\dusa{GEONAM} \moc{::}  \dstr{desctrack} \dstr{descsn}
\end{DataStructure}

\noindent  where
\begin{ListeDeDescription}{mmmmmmm}

\item[\dusa{TRKNAM}] {\tt character*12} name of the \dds{tracking} data
structure that will contain region volume and surface area vectors in
addition to region identification pointers and other tracking information.
If \dusa{TRKNAM} also appears on the RHS, the previous tracking 
parameters will be applied by default on the current geometry.

\item[\dusa{GEONAM}] {\tt character*12} name of the \dds{geometry} data
structure.

\item[\dstr{desctrack}] structure describing the general tracking data (see
\Sect{TRKData})

\item[\dstr{descsn}] structure describing the transport tracking data
specific to \moc{SNT:}.

\end{ListeDeDescription}

\vskip 0.2cm

The \moc{SNT:} specific tracking data in \dstr{descsn} is defined as

\begin{DataStructure}{Structure \dstr{descsn}}
$[~\{$ \moc{ONEG} $|$ \moc{ALLG} $\}~]~[$ \moc{KBA} \dusa{m} $]$ \\
$[$ \moc{SCHM} \dusa{ischm} $]~[$ \moc{DIAM} \dusa{mm} $]$ \\
\moc{SN} \dusa{n} $~[$ \moc{SCAT} \dusa{iscat} $]~$ \\
$[~\{$ \moc{LIVO} \dusa{icl1} \dusa{icl2} $|$ \moc{NLIVO} $|$ \moc{GMRES} \dusa{nstart} $\}~]$ \\
$[~\{$ \moc{DSA} \dusa{ndsa} \dusa{mdsa} \dusa{sdsa} $|$ \moc{NDSA} $\}~]$ \\
$[$ \moc{NSHT} $]$ \\
$[$ \moc{FOUR} \dusa{nfou} $]$ \\
$[$ \moc{MAXI} \dusa{maxi} $]~[$ \moc{EPSI} \dusa{epsi} $]$ \\
$[$ \moc{QUAD} \dusa{iquad} $]$ \\
$[~\{$ \moc{BTE} $|$ \moc{BFPG} $|$ \moc{BFPL}$ \}~]$ \\
$[$ \moc{ESCHM} \dusa{eschm} $]~[$ \moc{EDIAM} \dusa{emm} $]$ \\
$[~[$ \moc{QUAB} \dusa{iquab} $]~[~\{$ \moc{SAPO} $|$ \moc{HEBE} $|$ \moc{SLSI} $[$ \dusa{frtm} $]~\}~]~]$ \\
{\tt ;}
\end{DataStructure}

\noindent where

\begin{ListeDeDescription}{mmmmmmm}

\item[\dstr{desctrack}] structure describing the general tracking data (see
\Sect{TRKData})

\item[\moc{ONEG}] keyword to specify that the multigroup flux is computed as a sequence of one-group solutions using Gauss-Seidel iterations. This is the default option.

\item[\moc{ALLG}] keyword to specify that the multigroup flux is computed in parallel for a set of energy groups.

\item[\moc{KBA}] keyword to specify that Koch-Baker-Alcouffe (KBA) type nested loops over both angles and macrocells are used for
multithreading in 2D and 3D geometries.\cite{kba,domino}

\item[\dusa{m}] use $m\times m$ or $m \times m \times m$ macrocells in KBA swapping.

%Update to tabular format?
\item[\moc{SCHM}] keyword to specify the spatial discretisation scheme. 

\item[\dusa{ischm}] index to specify the spatial discretisation scheme. \dusa{ischm} $=1$ is used for High-Order Diamond Differencing (HODD) (default value). \dusa{ischm} $=2$ is used for the Discontinuous Galerkin finite element method (DG) currently available only in 1D slab, and 2D/3D Cartesian/hexagonal geometries. \dusa{ischm} $=3$ is used for the Adaptive Weighted Difference method (AWD), only available for Cartesian geometries.

\item[\moc{DIAM}] keyword to fix the spatial approximation order.

\item[\dusa{mm}] order of the Legendre polynomial expansion used in the spatial discretisation method. For HODD, \dusa{mm} $=0$ is the default, while for DG, it is \dusa{mm} $=1$.
For Cartesian geometries, any order \dusa{mm} $\geq0$ is available. For 2D hexagonal geometries, linear and parabolic orders are available. Classical diamond difference (\dusa{mm} $=0$ with \dusa{ischm} $=1$) are available for 1D tube and 1D sphere geometries. Adaptive schemes (\dusa{ischm} $=3$) are only available with constant order.
\begin{displaymath}
\dusa{mm} = \left\{
\begin{array}{rl}
 0 & \textrm{Constant (classical diamond scheme (HODD) or step scheme (DG))} \\
 1 & \textrm{Linear} \\
 2 & \textrm{Parabolic} \\
 >3  & \textrm{Higher-orders}
\end{array} \right.
\end{displaymath}

\item[\moc{SN}] keyword to fix the angular approximation order of the flux.

\item[\dusa{n}] order of the $S_N$ approximation (even number).

\item[\moc{SCAT}] keyword to limit the anisotropy of scattering sources.

\item[\dusa{iscat}] number of terms in the scattering sources. \dusa{iscat} $=1$ is used for
isotropic scattering in the laboratory system. \dusa{iscat} $=2$ is used for
linearly anisotropic scattering in the laboratory system. The default value is set to $n$.

\item[\moc{LIVO}] keyword to enable Livolant acceleration of the scattering iterations (default value).
\item[\dusa{icl1},~\dusa{icl2}] Numbers of respectively free and accelerated iterations in the Livolant method.
\item[\moc{NLIVO}] keyword to disable acceleration method and to perform free scattering iterations

\item[\moc{GMRES}] keyword to set the GMRES(m) acceleration of the scattering iterations. The default value,
equivalent to \dusa{nstart}=0, corresponds to a one-parameter Livolant acceleration.\cite{gmres}

\item[\dusa{nstart}] restarts the GMRES method every \dusa{nstart} iterations.

\item[\moc{DSA}] keyword to enable diffusion synthetic acceleration using BIVAC or TRIVAC.
\item[\dusa{ndsa}] apply the synthetic acceleration every \dusa{ndsa} number of inner flux iterations. Depending on the test case, if the DSA is enabled too soon or enabled at every inner iteration, instabilities and convergence failure can occur. A value of $0$ can be set to start the DSA immediately and have the acceleration applied every inner iteration thereafter. The default is \dusa{nsdsa}~$=1000$, indicating the DSA will not be applied. Benchmarks suggests that the optimal values are $3$ and $5$ for Cartesian and hexagonal geometries respectively.
\item[\dusa{mdsa}] order of the Raviart-Thomas spatial approximation used in the DSA resolution. Sometimes, using the same order as the transport calculation does not provide any benefit to the solution, and ends up being a drain on computational resources. Hence, there is the option of using a different order than the transport approximation.
\begin{displaymath}
\dusa{mdsa} = \left\{
\begin{array}{rl}
 0 & \textrm{Constant} \\
 1 & \textrm{Linear} \\
 2 & \textrm{Parabolic} \\
\end{array} \right.
\end{displaymath}
\item[\dusa{sdsa}] choose the solver to use for the synthetic acceleration. Note that TRIVAC generally works better and is faster with hexagonal geometries for the matrix assemblies. Also, for 3D geometries, TRIVAC \emph{has} to be chosen.
\begin{displaymath}
\dusa{sdsa} = \left\{
\begin{array}{rl}
 1 & \textrm{BIVAC} \\
 2 & \textrm{TRIVAC} \\
\end{array} \right.
\end{displaymath}

\item[\moc{NDSA}] keyword to disable diffusion synthetic acceleration (default).

\item[\moc{NSHT}] keyword to disable the shooting method for 1D cases -- can be useful for debugging purposes.

\item[\moc{FOUR}] keyword to pass the number of frequencies to be investigated in Fourier analysis (only works in 1D Cartesian geometry).
\item[\dusa{nfou}] number of frequencies to be investigated in 1D Fourier analysis along the range $[0, \frac{2\pi}{L})$ where $L$ is the length of the slab.

\item[\moc{MAXI}] keyword to set the maximum number of inner iterations (or GMRES iterations if actived).
\item[\dusa{maxi}] maximum number of inner iterations. Default value: $100$.

\item[\moc{EPSI}] set the convergence criterion on inner iterations (or GMRES iterations if actived).
\item[\dusa{epsi}] convergence criterion on inner iterations. The default value is $1\times 10^{-5}$.
\item[\moc{QUAD}] keyword to set the type of angular quadrature.

\item[\dusa{iquad}] type of quadrature: $=1$: Lathrop-Carlson level-symmetric quadrature;
$=2$: $\mu_1$--optimi\-zed level-symmetric quadrature (default option in 2D and in 3D); $=3$ Snow-code level-symmetric quadrature
(obsolete); $=4$: Legendre-Chebyshev quadrature (variable number of base points
per axial level); $=5$: symmetric Legendre-Chebyshev quadrature; $=6$: quadruple range (QR)
quadrature;\cite{quadrupole} $=10$: product of Gauss-Legendre and Gauss-Chebyshev quadrature (equal
number of base points per axial level).

\item[\moc{BTE}] solution of the Boltzmann transport equation (default option).

\item[\moc{BFPG}] solution of the Boltzmann Fokker-Planck equation with Galerkin energy propagation factors.

\item[\moc{BFPL}] solution of the Boltzmann Fokker-Planck equation with Przybylski and Ligou energy propagation factors.\cite{ligou}

\item[\moc{ESCHM}] keyword to specify the energy discretisation scheme to use for the continuous slowing-down term of the Boltzmann Fokker-Planck equation. 

\item[\dusa{eschm}] index to specify the energy discretisation scheme. \dusa{ischm} $=1$ is used for High-Order Diamond Differencing (HODD) (default value). \dusa{ischm} $=2$ is used for the Discontinuous Galerkin finite element method (DG). \dusa{ischm} $=3$ is used for the Adaptive Weighted Difference method (AWD). All of these schemes are available only for Cartesian geometries.

\item[\moc{EDIAM}] keyword to fix the energy approximation order.

\item[\dusa{emm}] order of the Legendre polynomial expansion used in the energy discretisation method. For HODD, \dusa{mm} $=0$ is the default, while for DG, it is \dusa{mm} $=1$.
For Cartesian geometries, any order \dusa{mm} $\geq0$ is available. Adaptive schemes (\dusa{ischm} $=3$) are only available with constant order.
\begin{displaymath}
\dusa{mm} = \left\{
\begin{array}{rl}
 0 & \textrm{Constant (classical diamond scheme (HODD) or step scheme (DG))} \\
 1 & \textrm{Linear} \\
 2 & \textrm{Parabolic} \\
 >3  & \textrm{Higher-orders}
\end{array} \right.
\end{displaymath}

\item[\moc{QUAB}] keyword to specify the number of basis point for the
numerical integration of each micro-structure in cases involving double
heterogeneity (Bihet).

\item[\dusa{iquab}] the number of basis point for the numerical integration of
the collision probabilities in the micro-volumes using the  Gauss-Jacobi
formula. The values permitted are: 1 to 20, 24, 28, 32 or  64. The default value
is \dusa{iquab}=5. If \dusa{iquab} is negative, its absolute value will be used in the She-Liu-Shi approach to determine the
split level in the tracking used to compute the probability collisions.

\item[\moc{SAPO}] use the Sanchez-Pomraning double-heterogeneity model.\cite{sapo}

\item[\moc{HEBE}] use the Hebert double-heterogeneity model (default option).\cite{BIHET}

\item[\moc{SLSI}] use the She-Liu-Shi double-heterogeneity model without shadow effect.\cite{She2017}

\item[\dusa{frtm}] the minimum microstructure volume fraction used to compute the size of the equivalent cylinder in She-Liu-Shi approach. The default value is \dusa{frtm} $=0.05$.

\end{ListeDeDescription}

\eject
 % structure (snT)
\subsubsection{The {\tt BIVACT:} tracking module}\label{sect:BIVACData}

The {\tt BIVACT:} module provides an implementation of the diffusion or simplified $P_n$ method. The {\tt BIVACT:} module can only process
1D/2D regular geometries of type \moc{CAR1D}, \moc{CAR2D} and \moc{HEX}. The geometry is analyzed and
a LCM object with signature {\tt L\_BIVAC} is created with the tracking information.

\vskip 0.2cm

The calling specification for this module is:

\begin{DataStructure}{Structure \dstr{BIVACT:}}
\dusa{TRKNAM}
\moc{:=} \moc{BIVACT:} $[$ \dusa{TRKNAM} $]$ 
\dusa{GEONAM} \moc{::}  \dstr{desctrack} \dstr{descbivac}
\end{DataStructure}

\noindent  where
\begin{ListeDeDescription}{mmmmmmm}

\item[\dusa{TRKNAM}] {\tt character*12} name of the \dds{tracking} data
structure that will contain region volume and surface area vectors in
addition to region identification pointers and other tracking information.
If \dusa{TRKNAM} also appears on the RHS, the previous tracking 
parameters will be applied by default on the current geometry.

\item[\dusa{GEONAM}] {\tt character*12} name of the \dds{geometry} data
structure.

\item[\dstr{desctrack}] structure describing the general tracking data (see
\Sect{TRKData})

\item[\dstr{descbivac}] structure describing the transport tracking data
specific to \moc{BIVACT:}.

\end{ListeDeDescription}

\vskip 0.2cm

The \moc{BIVACT:} specific tracking data in \dstr{descbivac} is defined as

\begin{DataStructure}{Structure \dstr{descbivac}}
$[$ $\{$ \moc{PRIM} $[$ \dusa{ielem} \dusa{icol} $]$ \\
~~~~$|$ \moc{DUAL} $[$ \dusa{ielem} \dusa{icol} $]$ \\
~~~~$|$ \moc{MCFD} $\}~]$ \\
$[~\{$ \moc{PN} $|$ \moc{SPN} $\}$ $[$ \moc{DIFF} $]$ \dusa{nlf} $[$ \moc{SCAT} \dusa{iscat} $]~[$ \moc{VOID} \dusa{nvd}~$]~]$ \\
{\tt ;}
\end{DataStructure}

\noindent where

\begin{ListeDeDescription}{mmmmmmm}

\item[\dstr{desctrack}] structure describing the general tracking data (see
\Sect{TRKData})

\item[\moc{PRIM}] keyword to set a primal finite element (classical)
discretization.

\item[\moc{DUAL}] keyword to set a mixed-dual finite element discretization. If the
geometry is hexagonal, a Thomas-Raviart-Schneider method is used.

\item[\moc{MCFD}] keyword to set a mesh-centered finite difference discretization
in hexagonal geometry.

\item[\dusa{ielem}] order of the finite element representation.  The values
permitted are: 1 (linear polynomials), 2 (parabolic polynomials), 3 (cubic
polynomials) or 4 (quartic polynomials). By default \dusa{ielem}=1.

\item[\dusa{icol}] type of quadrature used to integrate the mass matrices. The
values permitted are: 1 (analytical integration), 2  (Gauss-Lobatto quadrature)
or 3 (Gauss-Legendre quadrature). By default \dusa{icol}=2. The analytical
integration corresponds to classical finite elements; the Gauss-Lobatto
quadrature corresponds to a variational or nodal type collocation and the
Gauss-Legendre quadrature corresponds to superconvergent finite elements.

\item[\moc{PN}] keyword to set a spherical harmonics ($P_n$) expansion of the flux.\cite{nse2005} This option is currently limited to 1D
and 2D Cartesian geometries.

\item[\moc{SPN}] keyword to set a simplified spherical harmonics ($SP_n$) expansion
of the flux.\cite{nse2005,ane10a} This option is currently available with 1D and 2D Cartesian geometries
and with 2D hexagonal geometries.

\item[\moc{DIFF}] keyword to force using $1/3D^{g}$ as $\Sigma_1^{g}-\Sigma_{{\rm s}1}^{g}$ cross sections. A $P_1$ or $SP_1$ method
will therefore behave as diffusion theory.

\item[\dusa{nlf}] order of the $P_n$ or $SP_n$ expansion (odd number). Set to zero for diffusion theory (default value).

\item[\moc{SCAT}] keyword to limit the anisotropy of scattering sources.

\item[\dusa{iscat}] number of terms in the scattering sources. \dusa{iscat} $=1$ is used for
isotropic scattering in the laboratory system. \dusa{iscat} $=2$ is used for
linearly anisotropic scattering in the laboratory system. The default value is set to $n+1$
in $P_n$ or $SP_n$ case.

\item[\moc{VOID}] key word to set the number of base points in the Gauss-Legendre quadrature used to integrate
void boundary conditions if \dusa{icol} $=3$ and \dusa{n} $\ne 0$.

\item[\dusa{nvd}] type of quadrature. The values
permitted are: 0 (use a (\dusa{n}$+2$)--point quadrature consistent with $P_{{\rm n}}$ theory),
1 (use a (\dusa{n}$+1$)--point quadrature consistent with $S_{{\rm n}+1}$ theory),
2 (use an analytical integration of the void boundary conditions). By default \dusa{nvd}=0.

\end{ListeDeDescription}

Various finite element approximations can be obtained by combining different
values of \dusa{ielem} and \dusa{icol}:

\begin{itemize}

\item {\tt PRIM 1 1~:} Linear finite elements;

\item {\tt PRIM 1 2~:} Mesh corner finite differences;

\item {\tt PRIM 1 3~:} Linear superconvergent finite elements;

\item {\tt PRIM 2 1~:} Quadratic finite elements;

\item {\tt PRIM 2 2~:} Quadratic variational collocation method;

\item {\tt PRIM 2 3~:} Quadratic superconvergent finite elements;

\item {\tt PRIM 3 1~:} Cubic finite elements;

\item {\tt PRIM 3 2~:} Cubic variational collocation method;

\item {\tt PRIM 3 3~:} Cubic superconvergent finite elements;

\item {\tt PRIM 4 2~:} Quartic variational collocation method;

\item {\tt DUAL 1 1~:} Mixed-dual linear finite elements;

\item {\tt DUAL 1 2~:} Mesh centered finite differences;

\item {\tt DUAL 1 3~:} Mixed-dual linear superconvergent finite elements

(numerically equivalent to {\tt PRIM~1~3});

\item {\tt DUAL 2 1~:} Mixed-dual quadratic finite elements;

\item {\tt DUAL 2 2~:} Quadratic nodal collocation method;

\item {\tt DUAL 2 3~:} Mixed-dual quadratic superconvergent finite elements

(numerically equivalent to {\tt PRIM~2~3});

\item {\tt DUAL 3 1~:} Mixed-dual cubic finite elements;

\item {\tt DUAL 3 2~:} Cubic nodal collocation method;

\item {\tt DUAL 3 3~:} Mixed-dual cubic superconvergent finite elements

(numerically equivalent to {\tt PRIM~3~3});

\item {\tt DUAL 4 2~:} Quartic nodal collocation method;

\end{itemize}
\eject
 % structure (bivacT)
\subsubsection{The {\tt TRIVAT:} tracking module}\label{sect:TRIVACData}

The {\tt TRIVAT:} module provides an implementation of the diffusion or simplified $P_n$ method. The {\tt TRIVAT:} module is
used to perform a TRIVAC-type ``tracking"  on a 1D/2D/3D regular Cartesian or hexagonal geometry.\cite{BIVAC,TRIVAC} The
geometry is analyzed and a LCM object with signature {\tt L\_TRIVAC} is created with the following information:

\begin{itemize}
\item Diagonal and hexagonal symmetries are unfolded and the mesh-splitting 
operations are performed. Volumes, material mixture and averaged flux recovery
indices are computed on the resulting geometry. \item A finite element
discretization is performed and the corresponding numbering is saved. \item The
unit finite element matrices (mass, stiffness, etc.) are recovered. \item
Indices related to an ADI preconditioning with or without supervectorization
are saved. \end{itemize}

The calling specification for this module is:

\begin{DataStructure}{Structure \dstr{TRIVAT:}}
\dusa{TRKNAM}
\moc{:=} \moc{TRIVAT:} $[$ \dusa{TRKNAM} $]$ 
\dusa{GEONAM} \moc{::}  \dstr{desctrack} \dstr{descTRIVAC}
\end{DataStructure}

\noindent  where
\begin{ListeDeDescription}{mmmmmmm}

\item[\dusa{TRKNAM}] {\tt character*12} name of the \dds{tracking} data
structure that will contain region volume and surface area vectors in
addition to region identification pointers and other tracking information.
If \dusa{TRKNAM} also appears on the RHS, the previous tracking 
parameters will be applied by default on the current geometry.

\item[\dusa{GEONAM}] {\tt character*12} name of the \dds{geometry} data
structure.

\item[\dstr{desctrack}] structure describing the general tracking data (see
\Sect{TRKData})

\item[\dstr{descTRIVAC}] structure describing the transport tracking data
specific to \moc{TRIVAT:}.

\end{ListeDeDescription}

\vskip 0.2cm

The \moc{TRIVAT:} specific tracking data in \dstr{descTRIVAC} is defined as

\begin{DataStructure}{Structure \dstr{descTRIVAC}}
$[~\{$ \moc{PRIM} $[$ \dusa{ielem} $]~|$ \moc{DUAL} $[$ \dusa{ielem} \dusa{icol} $]~|$ \moc{MCFD} $[$ \dusa{ielem} $]~|$ \moc{LUMP} $[$ \dusa{ielem} $]~\}~]$ \\
$[$ \moc{SPN} \dusa{n} $[$ \moc{SCAT} $[$ \moc{DIFF} $]$ \dusa{iscat} $]~[$ \moc{VOID} \dusa{nvd} $]~]$ \\
$[$ \moc{ADI} \dusa{nadi} $]$ \\
$[$ \moc{VECT} $[$ \dusa{iseg} $]~[$ \moc{PRTV} \dusa{impv} $]~]$ \\
{\tt ;}
\end{DataStructure}

\noindent where
\begin{ListeDeDescription}{mmmmmm}

\item[\dstr{desctrack}] structure describing the general tracking data (see
\Sect{TRKData})

\item[\moc{PRIM}] key word to set a discretization based on the variational collocation method.

\item[\moc{DUAL}] key word to set a mixed-dual finite element discretization. If the
geometry is hexagonal, a Thomas-Raviart-Schneider method is used.

\item[\moc{MCFD}] key word to set a discretization based  on the nodal
collocation method. The mesh centered finite difference approximation is the
default option and is generally set using {\tt MCFD~1}. The {\tt MCFD}
approximations are numerically equivalent to the {\tt DUAL} approximations
with \dusa{icol}=2; however, the {\tt MCFD} approximations are less
expensive. 

\item[\moc{LUMP}] key word to set a discretization  based on the nodal
collocation method with serendipity approximation. The serendipity
approximation is different from the \moc{MCFD} option in cases with \dusa{ielem}$\ge$2. This option is not available for hexagonal geometries.

\item[\dusa{ielem}] order of the finite element representation.  The values
permitted are: 1 (linear polynomials), 2 (parabolic polynomials), 3 (cubic
polynomials) or 4 (quartic polynomials). By default \dusa{ielem}=1.

\item[\dusa{icol}] type of quadrature used to  integrate the mass matrices.
The values permitted are: 1 (analytical integration), 2  (Gauss-Lobatto
quadrature) or 3 (Gauss-Legendre quadrature). By default \dusa{icol}=2. The
analytical integration corresponds to classical finite elements; the
Gauss-Lobatto quadrature corresponds to a variational or nodal type
collocation and the Gauss-Legendre quadrature corresponds to superconvergent
finite elements.

\item[\moc{SPN}] keyword to set a simplified spherical harmonics ($SP_n$) expansion
of the flux.\cite{nse2005,ane10a} This option is available with 1D, 2D and 3D Cartesian geometries and with 2D and 3D
hexagonal geometries.

\item[\dusa{n}] order of the $P_n$ or $SP_n$ expansion (odd number). Set to zero for diffusion theory (default value).

\item[\moc{SCAT}] keyword to limit the anisotropy of scattering sources.

\item[\moc{DIFF}] keyword to force using $1/3D^{g}$ as $\Sigma_1^{g}$ cross sections. A $P_1$ or $SP_1$ method
will therefore behave as diffusion theory.

\item[\dusa{iscat}] number of terms in the scattering sources. \dusa{iscat} $=1$ is used for
isotropic scattering in the laboratory system. \dusa{iscat} $=2$ is used for
linearly anisotropic scattering in the laboratory system. The default value is set to $n+1$
in $P_n$ or $SP_n$ case.

\item[\moc{VOID}] key word to set the number of base points in the Gauss-Legendre quadrature used to integrate
void boundary conditions if \dusa{icol} $=3$ and \dusa{n} $\ne 0$.

\item[\dusa{nvd}] type of quadrature. The values
permitted are: 0 (use a (\dusa{n}$+2$)--point quadrature consistent with $P_{\rm n}$ theory),
1 (use a (\dusa{n}$+1$)--point quadrature consistent with $S_{{\rm n}+1}$ theory),
2 (use an analytical integration of the void boundary conditions). By default \dusa{nvd}=0.

\item[\moc{ADI}] keyword to set the number of ADI iterations at the inner
iterative level.

\item[\dusa{nadi}] number of ADI iterations (default: \dusa{nadi} $=2$).

\item[\moc{VECT}] key word to set an ADI preconditionning with
supervectorization. By default, TRIVAC uses an ADI preconditionning without
supervectorization.

\item[\dusa{iseg}] width of a vectorial register. \dusa{iseg} is generally a multiple of 64. By default, \dusa{iseg}=64.

\item[\moc{PRTV}] key word used to set \dusa{impv}.

\item[\dusa{impv}] index used to control the  printing in supervectorization
subroutines. =0 for no print; =1 for minimum printing (default value); Larger
values produce increasing amounts of output.

\end{ListeDeDescription}

Various finite element approximations can be obtained with different values of \dusa{ielem}.

\eject
 % structure (trivacT)
\subsection{The {\tt SALT:} tracking module}\label{sect:SALTData1}

The \moc{{\tt SALT:}} module can process general 2-D geometries defined from
{\sl surfacic elements}. It is used to compute the tracking information requested in
the method of collision probabilities or in the method of characteristics.

\subsubsection{Cyclic tracking}

The {\tt SALT:} module with keyword {\tt TSPC} has the capability to perform {\sl cyclic tracking} over a closed square, rectangular
or equilateral triangular domain. Each track cover a certain surface before going back to the starting point after a distance $L$, as
depicted in Figs.~\ref{fig:cart_tspc} and~\ref{fig:hex_tspc}. Only specific angles, function of integer values $n$ and $m$, make
possible the cycling of trajectories.

\begin{figure}[h!]
\begin{center} 
\epsfxsize=10cm \centerline{ \epsffile{cart_tspc.eps}}
\parbox{14.0cm}{\caption{Cycling tracking over a Cartesian domain.}\label{fig:cart_tspc}}   
\end{center}  
\end{figure}

\begin{figure}[h!]
\begin{center} 
\epsfxsize=15cm \centerline{ \epsffile{hex_tspc.eps}}
\parbox{14.0cm}{\caption{Cycling tracking over an hexagonal domain.}\label{fig:hex_tspc}}   
\end{center}  
\end{figure}

\subsubsection{Calling specifications}

The calling specification for this module is:
\begin{DataStructure}{Structure \dstr{SALT:}}
\dusa{TRKNAM} \dusa{TRKFIL}
\moc{:=} \moc{SALT:}~\dusa{SURFIL} $[$ \dusa{GEONAM} $]$ \moc{::} \dstr{desctrack} \dstr{descsalt}
\end{DataStructure}

\noindent  where
\begin{ListeDeDescription}{mmmmmmm}

\item[\dusa{TRKNAM}] \texttt{character*12} name of the SALT \dds{tracking} data
structure that will contain region volume and surface area vectors in
addition to region identification pointers and other tracking information.

\item[\dusa{TRKFIL}] \texttt{character*12} name of the sequential binary tracking
file used to store the tracks lengths.

\item[\dusa{SURFIL}] \texttt{character*12} name of the SALOMON--formatted sequential {\sc ascii}
file used to store the surfacic elements of the geometry. This file may be build
using the operator {\tt G2S:} (see \Sect{G2SData}) or recovered from SALOME.

\item[\dusa{GEONAM}] {\tt character*12} name of the \dds{geometry} data
structure containing the double heterogeneity (Bihet) data.

\item[\dstr{desctrack}] structure describing the general tracking data (see
\Sect{TRKData})

\item[\dstr{descsalt}] structure describing the transport tracking data
specific to \moc{SALT:}.

\end{ListeDeDescription}

\vskip 0.2cm

All information for the modelization used can be found in \citen{salt}.
The \moc{{\tt SALT:}} specific tracking data in \dstr{descsalt} is defined as :

\begin{DataStructure}{Structure \dstr{descsalt}}
$[$ \moc{ANIS} \dusa{nanis} $]$ \\
$[~\{$  \moc{ONEG} $|$ \moc{ALLG} $\}~]$ \\
$[~[$ \moc{QUAB} \dusa{iquab} $]~[~\{$ \moc{SAPO} $|$ \moc{HEBE} $|$ \moc{SLSI} $[$ \dusa{frtm} $]~\}~]~]$ \\
$[~\{$ \moc{PISO} $|$ \moc{PSPC} $[$ \moc{CUT} \dusa{pcut} $]$ $\}~]$ \\
$[$ $\{$ \moc{GAUS}  $|$ \moc{CACA} $|$ \moc{CACB} $|$ \moc{LCMD} $|$ \moc{OPP1} $|$ \moc{OGAU} $\}~[$ \dusa{nmu} $]~]$ \\
$\{$ \moc{TISO} $[~\{$ \moc{EQW} $|$ \moc{GAUS} $|$ \moc{PNTN} $|$ \moc{SMS} $|$ \moc{LSN} $|$ \moc{QRN} $\}~]$ \dusa{nangl} \dusa{dens} \\
$~~~~~|$ \moc{TSPC} $[~\{$ \moc{MEDI} $|$ \moc{EQW2} $\}~]$ \dusa{nangl} \dusa{dens} $\}$ \\
$[$ \moc{CORN} \dusa{pcorn} $]$ \\
$[$ \moc{NOTR} $]$\\
$[$ \moc{NBSLIN} \dusa{nbslin} $]$ \\
$[$ \moc{MERGMIX} $]$\\
$[$ \moc{LONG} $]$\\
{\tt ;}
\end{DataStructure}

\noindent
where

\begin{ListeDeDescription}{mmmmmmmm}

\item[\moc{ANIS}] keyword to specify the order of scattering anisotropy. 

\item[\dusa{nanis}] order of anisotropy in transport calculation.
A default value of 1 represents isotropic (or transport-corrected) scattering while a value of 2
correspond to linearly anisotropic scattering.

\item[\moc{ONEG}] keyword to specify that the tracking is read before computing each group-dependent collision
probability or algebraic collapsing matrix (default value if \dusa{TRKFIL} is set). The tracking file is
read in each energy group if the method of characteristics (MOC) is used.

\item[\moc{ALLG}] keyword to specify that the tracking is read once and the collision
probability or algebraic collapsing matrices are computed in many energy groups.  The tracking file is
read once if the method of characteristics (MOC) is used.
 
\item[\moc{QUAB}] keyword to specify the number of basis point for the
numerical integration of each micro-structure in cases involving double
heterogeneity (Bihet).

\item[\dusa{iquab}] the number of basis point for the numerical integration of
the collision probabilities in the micro-volumes using the Gauss-Jacobi
formula. The values permitted are: 1 to 20, 24, 28, 32 or 64. The default value
is \dusa{iquab} = 5. If \dusa{iquab} is negative, its absolute value will be used in the She-Liu-Shi approach to determine the
split level in the tracking used to compute the probability collisions.

\item[\moc{SAPO}] use the Sanchez-Pomraning double-heterogeneity model.\cite{sapo}

\item[\moc{HEBE}] use the Hebert double-heterogeneity model (default option).\cite{BIHET}

\item[\moc{SLSI}] use the She-Liu-Shi double-heterogeneity model without shadow effect.\cite{She2017}

\item[\dusa{frtm}] the minimum microstructure volume fraction used to compute the size of the equivalent cylinder in She-Liu-Shi approach. The default value is \dusa{frtm} $=0.05$.

\item[\moc{PISO}] keyword to specify that a collision probability calculation with isotropic reflection boundary 
conditions is required. It is the default option if a \moc{TISO} type integration is chosen. To obtain accurate
transmission probabilities for the isotropic case it is recommended that the normalization 
options in the \moc{ASM:} module be used. 

\item[\moc{PSPC}] keyword to specify that a collision probability calculation with mirror like reflection or periodic 
boundary conditions is required; this is the default option if a \moc{TSPC} type integration is chosen. 
This calculation is only possible if the file was initially constructed using the \moc{TSPC} option. 

\item[\moc{CUT}] keyword to specify the input of cutting parameters for the specular collision probability
of characteristic integration. 

\item[\dusa{pcut}] real value representing the maximum error allowed on the exponential function used
for specular collision probability calculations. Tracks will be cut at a length such that the error in the 
probabilities resulting from this reduced track will be of the order of pcut. By default, the tracks 
are extended to infinity and \dusa{pcut} = 0.0. If this option is used in an entirely reflected case, it is 
recommended to use the \moc{NORM} command in the \moc{ASM:} module. 

\item[\moc{GAUS}] keyword to specify that Gauss-Legendre polar integration angles are to be selected for the polar quadrature when a prismatic tracking is considered. The conservation is ensured up to $P_{\dusa{nmu}-1}$ scattering.

\item[\moc{CACA}] keyword to specify that CACTUS type equal weight polar integration angles are to be
selected for the polar quadrature when a prismatic tracking is considered.\cite{CACTUS} The conservation is ensured only for isotropic scattering.

\item[\moc{CACB}] keyword to specify that CACTUS type uniformly distributed integration polar angles
are to be selected for the polar quadrature when a prismatic tracking is considered.\cite{CACTUS} The conservation is ensured only for isotropic scattering.

\item[\moc{LCMD}] keyword to specify that optimized (McDaniel--type) polar integration angles are to be
selected for the polar quadrature when a prismatic tracking is considered.\cite{LCMD} This is the default option. The conservation is ensured only for isotropic scattering.

\item[\moc{OPP1}] keyword to specify that $P_1$ constrained optimized (McDaniel--type) polar integration angles are to be selected for the polar quadrature when a prismatic tracking is considered.\cite{LeTellierpa} The conservation is ensured only for isotropic and linearly anisotropic scattering.

\item[\moc{OGAU}] keyword to specify that Optimized Gauss polar integration angles are to be
selected for the method of characteristics.\cite{LCMD,LeTellierpa} The conservation is ensured up to $P_{\dusa{nmu}-1}$ scattering.

\item[\dusa{nmu}]  user-defined number of polar angles. By default, a value consistent with \dusa{nangl} is computed by the code. For \moc{LCMD}, \moc{OPP1}, \moc{OGAU} quadratures, \dusa{nmu} is limited to 2, 3 or 4.

\item[\moc{TISO}] keyword to specify that isotropic tracking parameters will be supplied. This is the
default tracking option for cluster geometries. 

\item[\moc{TSPC}] keyword to specify that specular tracking parameters will be
supplied.

\item[\moc{EQW}] keyword to specify the use of equal weight quadrature.\cite{eqn} The conservation is ensured up to $P_{\dusa{nangl}/2}$ scattering.

\item[\moc{GAUS}] (after \moc{TISO} keyword) keyword to specify the use of the Gauss-Legendre quadrature. This option is valid only if an 
hexagonal geometry is considered.

\item[\moc{PNTN}] keyword to specify that Legendre-Techbychev quadrature quadrature will be selected.\cite{pntn} The conservation is ensured only for isotropic and linearly anisotropic scattering.

\item[\moc{SMS}] keyword to specify that Legendre-trapezoidal quadrature quadrature will be selected.\cite{sms} The conservation is ensured up to $P_{\dusa{nangl}-1}$ scattering.

\item[\moc{LSN}] keyword to specify the use of the $\mu_1$--optimized level-symmetric quadrature. The conservation is ensured up to $P_{\dusa{nangl}/2}$ scattering.

\item[\moc{QRN}] keyword to specify the use of the quadrupole range (QR) quadrature.\cite{quadrupole}

\item[\moc{MEDI}] keyword to specify the use of a median angle quadrature in \moc{TSPC} cases. For
a rectangular Cartesian domain of size $X \times Y$, the azimuthal angles in $(0,\pi/2)$ interval are obtained from formula
$$
\phi_k=\tan^{-1}{\displaystyle kY\over\displaystyle (2p+2-k)X} \, , \ \ k=1,\, 3,\, 5, \, \dots, \, 2p+1 .
$$

\item[\moc{EQW2}] keyword to specify the use of a standard cyclic quadrature without angles $\phi=0$ and $\phi=\pi/2$ in \moc{TSPC} cases. For
a rectangular Cartesian domain of size $X \times Y$, the azimuthal angles in $(0,\pi/2)$ interval are obtained from formula
$$
\phi_k=\tan^{-1}{\displaystyle k Y\over\displaystyle (p+2-k)X} \, , \ \ k=1,\, 2,\, 3, \, \dots, \, p+1 .
$$
This is the default option.

\item[\dusa{nangl}] angular quadrature parameter. For a 3-D \moc{EQW} option, the choices are \dusa{nangl} = 2, 4, 8, 10, 12, 14 
or 16. For a 3-D \moc{PNTN} or \moc{SMS} option, \dusa{nangl} is an even number smaller than 46.\cite{ige260} For 2-D 
isotropic applications, any value of \dusa{nangl} may be used, equidistant angles will be selected.

For 2-D specular applications the input value must be of the form $p + 1$ where $p$ is a prime number, as proposed
in Ref.~\citen{DragonPIJS3}. In this case, the choice of \dusa{nangl} = 6, 8, 12, 14, 18, 20, 24, or 30 are allowed. For hexagonal lattices,
including equilateral triangular and lozenge geometry, the choice of \dusa{nangl} = 3, 6, 12 or 18 are allowed.

\item[\dusa{dens}] real value representing the density of the integration lines (in cm$^{-1}$ for 2-D Cartesian cases.
This choice of density along the plan perpendicular to each angle depends on the geometry of the cell to be analyzed. If there 
are zones of very small volume, a high line density is essential. This value will be readjusted by 
\moc{SALT:}.

\item[\moc{CORN}] keyword to specify the meaningful distance (cm) between a tracking line and a surfacic element.

\item[\dusa{pcorn}] meaningful distance (cm) between a tracking line and a surfacic element. By default, \dusa{pcorn} $=1.0 \times 10^{-5}$ cm.

\item[\moc{NOTR}] keyword to specify that the geometry will not be tracked. This is useful for 2-D geometries 
to generate a tracking data structure that can be used by the \moc{PSP:} module (see \Sect{PSPData}). 
One can then verify visually if the geometry is adequate before the tracking process as such is 
undertaken.

\item[\moc{NBSLIN}] keyword to set the maximum number of segments in a single tracking line.

\item[\dusa{nbsl}] integer value representing the maximum number of segments in a single tracking line. The default value is \dusa{nbsl} = 100000.

\item[\moc{MERGMIX}] keyword to specify that all regions belonging to the same mixture will be merged together. This option should only be used as an attempt to reduce CPU costs in resonance self-shielding calculations.

\item[\moc{LONG}] keyword to specify that a ``long'' tracking file will be generated. This option is required if the tracking file is to be used by the \moc{TLM:} module (see \Sect{TLMData}).

\end{ListeDeDescription}
\clearpage
 % structure (salT)
