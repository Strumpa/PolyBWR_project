\subsection{The {\tt BREF:} module}\label{sect:BREFData}

This module compute a {\sc macrolib} for a 1D {\sl equivalent reflector} based on various models. One or many fine-group and
fine-mesh reference calculations (using the $S_n$ method) are first performed so as to produce coarse-group and
coarse-mesh Macrolibs stored within output {\sc edition} objects (\dusa{EDIT\_SN}), compatible with the selected
reflector model. Module {\tt BREF:} recovers the $S_n$ {\sc geometry}, depicted in Fig.~\ref{fig:bref}, from object \dusa{GEOM\_SN}.
The $S_n$ {\sc geometry} must have a reflective ({\tt REFL} or {\tt ALBE 1.0}) boundary condition on its left ({\tt X-}) boundary.
Module {\tt BREF:} recovers the following information from each \dusa{EDIT\_SN} object:
\begin{itemize}
\item Coarse group surfacic fluxes between the nodes using averaged flux values recovered into {\sl gap} volumes, corresponding to
tiny meshes in the reflector zones.
\item Coarse group net currents between the nodes obtained from a balance relation, assuming reflection on the left
boundary.
\item Averaged macroscopic cross sections and diffusion coefficients within the {\sl no-gap} homogenized nodes.
\end{itemize}

\begin{figure}[h!]
\begin{center} 
\epsfxsize=11cm
\centerline{ \epsffile{bref_geom.eps}}
\parbox{14cm}{\caption{Definition of geometries used by the {\tt BREF:} module}
\label{fig:bref}} 
\end{center} 
\end{figure}

At output, a {\sc macrolib} object is produced with equivalent macroscopic cross sections, diffusion coefficients, discontinuity factors and
albedos. A verification calculation performed over the {\tt BREF:} geometry is depicted in Fig.~\ref{fig:brefVerif},

\begin{figure}[h!]
\begin{center} 
\epsfxsize=15cm
\centerline{ \epsffile{ErmBeavrsPwrRefl.eps}}
\parbox{14cm}{\caption{Equivalent ERM-NEM reflector for the BEAVRS benchmark}
\label{fig:brefVerif}} 
\end{center} 
\end{figure}

\vskip 0.02cm

Nodal expansion base functions are used to represent the flux with the {\tt DF-NEM} and {\tt ERM-NEM}
methods. By default, polynomials defined over $(-0.5,0.5)$ are used as base functions:\cite{nestle}
\begin{eqnarray}
\nonumber P_0(u)\negthinspace &=&\negthinspace 1\\
\nonumber P_1(u)\negthinspace &=&\negthinspace u\\
\nonumber P_2(u)\negthinspace &=&\negthinspace 3u^2-{1\over 4}\\
\nonumber P_3(u)\negthinspace &=&\negthinspace \left( u^2-{1\over 4}\right)u\\
P_4(u)\negthinspace &=&\negthinspace \left( u^2-{1\over 4}\right)\left( u^2-{1\over 20}\right)
\end{eqnarray}
There is the option of using hyperbolic functions in some energy groups:
\begin{eqnarray}
\nonumber P_3(u)\negthinspace &=&\negthinspace \sinh(\zeta_g u)\\
P_4(u)\negthinspace &=&\negthinspace \cosh(\zeta_g u)-{2\over \zeta}\, \sinh(\zeta_g/2)
\end{eqnarray}
\noindent where
\begin{equation}
\zeta_g=\Delta x\sqrt{\Sigma_{{\rm r},g} \over D_g}
\end{equation}
\noindent where $\Delta x$, $\Sigma_{{\rm r},g}$ and $D_g$ are the node width (cm), the macroscopic removal cross section (cm$^{-1}$)
and the diffusion coefficient (cm) in group $g$, respectively.

\vskip 0.02cm

Other equivalence techniques are known as {\tt DF-ANM} and {\tt ERM-ANM}. These techniques are similar to {\tt DF-NEM} and
{\tt ERM-NEM} where the nodal expansion method (NEM) is replaced by an analytic solution of the $G-$group diffusion equation.

\goodbreak

The calling specifications are:

\begin{DataStructure}{Structure \dstr{BREF:}}
\dusa{GEOM} \dusa{MACRO}~\moc{:=}~\moc{BREF:}~\dusa{GEOM\_SN} $[[$~\dusa{EDIT\_SN}~$]]$ \moc{::} \dstr{BREF\_data} \\
\end{DataStructure}

\noindent where
\begin{ListeDeDescription}{mmmmmmm}

\item[\dusa{GEOM}] {\tt character*12} name of the nodal {\sc geometry} (type {\tt L\_GEOM}) object open creation mode. This geometry can be used for
performing a verification calculation over the 1D nodal geometry.

\item[\dusa{MACRO}] {\tt character*12} name of the nodal {\sc macrolib} (type {\tt L\_MACROLIB}) object open in creation mode.

\item[\dusa{GEOM\_SN}] {\tt character*12} name of the $S_n$ {\sc geometry} (type {\tt L\_GEOM}) object open in read-only mode.

\item[\dusa{EDIT\_SN}] {\tt character*12} name of one or many $S_n$ {\sc edition} (type {\tt L\_EDIT}) object, containing coarse-group and
coarse-mesh {\sc macrolib} for the edition {\sc macro-geometry} with gaps, corresponding to one or many reference $S_n$ calculations.

\item[\dusa{BREF\_data}] input data structure containing specific data (see \Sect{descBREF}).

\end{ListeDeDescription}
\clearpage

\subsubsection{Data input for module {\tt BREF:}}\label{sect:descBREF}

\begin{DataStructure}{Structure \dstr{BREF\_data}}
$[$~\moc{EDIT} \dusa{iprint}~$]$ \\
$[$~\moc{HYPE} \dusa{igmax}~$]$ \\
\moc{MIX} $[[$ \dusa{imix} $]]$ \moc{GAP} $[[$ \dusa{igap} $]]$ \\
\moc{MODE} $\{$~\moc{LEFEBVRE-LEB}~$|$~\moc{KOEBKE}~$|$~\moc{DF-NEM}~$|$~\moc{DF-ANM}~$|$~\moc{DF-RT} \dusa{ielem} \dusa{icol}
$[$~\moc{SPN} $[$ \moc{DIFF} $]$ \dusa{nlf} $]~|$ \\
~~~~\moc{ERM-NEM}~$|$~\moc{ERM-ANM}~$\}$ \\
$[~\{$~\moc{SPH}~$|$~\moc{NOSP}~$\}~]~[~\{$~\moc{ALBE}~$|$~\moc{NOAL}~$\}~]$ \\
$[$~\moc{NGET}~$[$~(\dusa{adf}($g$), $g$=1,$N_g$) $]~]$ \\
{\tt ;}
\end{DataStructure}

\noindent where
\begin{ListeDeDescription}{mmmmmmmm}

\item[\moc{EDIT}] keyword used to set \dusa{iprint}.

\item[\dusa{iprint}] index used to control the printing in module {\tt BREF:}. =0 for no print; =1 for minimum printing (default value).

\item[\moc{HYPE}] keyword used to specify the type of nodal expansion base functions with the {\tt DF-NEM} and {\tt ERM-NEM}
methods. By default, polynomial base functions are used in all energy groups.

\item[\dusa{igmax}] hyperbolic base functions are used for coarse energy groups with indices $\ge$ \dusa{igmax}.

\item[\moc{MIX}] keyword used to set the nodal mixture indices \dusa{imix}.

\item[\dusa{imix}] index of a mixture index within object \dusa{EDIT\_SN} corresponding to a node. In Fig.~\ref{fig:bref}, this data
is set as {\tt MIX 0 3} for {\tt LEFEBVRE-LEB} and {\tt KOEBKE} geometries and set as {\tt MIX 1 3} for other geometries.

\item[\moc{GAP}] keyword used to set the gap mixture indices \dusa{igap} where the surfacic fluxes are recovered.

\item[\dusa{igap}] index of a mixture index within object \dusa{EDIT\_SN} corresponding to a gap. In Fig.~\ref{fig:bref}, this data
is set as {\tt GAP 2 0} for {\tt LEFEBVRE-LEB} and {\tt KOEBKE} geometries and set as {\tt GAP 2 4} for other geometries.

\item[\moc{MODE}] keyword used to select a specific reflector equivalence model. The character*12 name of this model is chosen among the following values:

\item[{\tt LEFEBVRE-LEB}] Lefebvre-Lebigot equivalence model. Two \dusa{EDIT\_SN} objects are expected at RHS.\cite{LLB,Frohlicher}
\item[{\tt KOEBKE}] Koebke equivalence model. Two \dusa{EDIT\_SN} objects are expected at RHS.\cite{Koebke,Frohlicher}
\item[{\tt DF-NEM}] Pure discontinuity-factor model based on the nodal expansion method (NEM). Only one \dusa{EDIT\_SN} object is expected at RHS.
\item[{\tt DF-ANM}] Pure discontinuity-factor model based on the analytic nodal method (ANM). Only one \dusa{EDIT\_SN} object is expected at RHS.
\item[{\tt DF-RT}] Pure discontinuity-factor model based on the Raviart-Thomas finite element method in diffusion or $SP_n$ theory. Only one \dusa{EDIT\_SN} object is expected at RHS.
\item[{\tt ERM-NEM}] {\sl Equivalent reflector model} based on {\sl matrix discontinuity factors} and nodal expansion method (NEM).
Two or more \dusa{EDIT\_SN} objects are expected at RHS.
\item[{\tt ERM-ANM}] {\sl Equivalent reflector model} based on {\sl matrix discontinuity factors} and analytic nodal method (ANM).
Two or more \dusa{EDIT\_SN} objects are expected at RHS.

\item[\dusa{ielem}] order of the Raviart-Thomas finite element method.  The values
permitted are 1 (linear polynomials), 2 (parabolic polynomials), 3 (cubic polynomials) or 4 (quartic polynomials).

\item[\dusa{icol}] type of quadrature used to integrate the mass matrices in the Raviart-Thomas finite element method. The
values permitted are 1 (analytical integration), 2  (Gauss-Lobatto quadrature) or 3 (Gauss-Legendre quadrature).

\item[{\tt SPN}] keyword to define a simplified spherical harmonics ($SP_n$) macro calculation. By default, diffusion theory is used.

\item[\moc{DIFF}] keyword to force using $1/3D^{g}$ as $\Sigma_1^{g}-\Sigma_{{\rm s}1}^{g}$ cross sections with the $SP_n$ approximation. A $SP_1$ method
will therefore behave as diffusion theory.

\item[\dusa{nlf}] order of the $SP_n$ expansion (odd number).

\item[\moc{SPH}] keyword used to include discontinuity factors within cross sections and diffusion coefficients. This option is not available
with models {\tt ERM-NEM} and {\tt ERM-ANM}. This is the default option with model {\tt DF-RT}.

\item[\moc{NOSP}] keyword used to store discontinuity factors in {\sc macrolib} \dusa{MACRO} (default option, except for model {\tt DF-RT}).

\item[\moc{ALBE}] keyword used to compute an equivalent albedo in each coarse energy group with {\tt DF-NEM}, {\tt DF-ANM}, {\tt DF-RT},
{\tt ERM-NEM} and {\tt ERM-ANM} models (default option).

\item[\moc{NOAL}] keyword used to desactivate equivalent albedo calculation.

\item[\moc{NGET}] keyword used to force the value of the fuel assembly discontinuity factor at the fuel-reflector interface, as used
by the NGET normalization. By default, this value is not modified by NGET normalization.

\item[\dusa{adf}] value of the assembly discontinuity factor (ADF) on the fuel-reflector interface in group $g\le N_g$. If keyword \moc{NGET} is set and
\dusa{adf} values are not given, the ADF values are recovered from \dusa{EDIT\_SN}.

\end{ListeDeDescription}

\eject
