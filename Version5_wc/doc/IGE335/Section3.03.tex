\subsection{The {\tt GEO:} module}\label{sect:GEOData}

The \moc{GEO:} module is used to create or modify a geometry. The geometry
definition module in DRAGON permits all the characteristics (coordinates,
region mixture and boundary conditions) of a simple or complex
geometry to be specified. The method used to specify the geometry is independent
of the discretization module to be used subsequently. Each geometry is stored in
the form of a \dds{geometry} data structure under its given name. It is
always possible to modify an existing geometry or copy it under a new name. 
The calling specifications are:

\begin{DataStructure}{Structure \dstr{GEO:}}
$\{$ \\
\hskip 0.3cm \dusa{GEONAM} \moc{:=} \moc{GEO:} $\{$ \dusa{GEONAM} $|$ \dusa{OLDGEO} $\}$
\moc{::} \dstr{descgcnt}  \\
 $|$ \\
\hskip 0.3cm  \dusa{GEONAM} \moc{:=} \moc{GEO:} \moc{::} \dstr{descgtyp} \dstr{descgcnt}  \\
 $\}$ 
\end{DataStructure}

\noindent

\noindent where
\begin{ListeDeDescription}{mmmmmmmm}

\item[\dusa{GEONAM}] {\tt character*12} name of the \dds{geometry} created or
modified.

\item[\dusa{OLDGEO}] {\tt character*12} name of a read-only \dds{geometry}.
The type and all the characteristics of \dusa{OLDGEO} will be copied onto \dusa{GEONAM}
before this later geometry is modified.

\item[\dstr{descgtyp}] structure describing the geometry type of
\dusa{GEONAM} (see \Sect{descgeo}).

\item[\dstr{descgcnt}] structure describing the characteristics of a geometry
(see \Sect{descgeo}).

\end{ListeDeDescription}

\subsubsection{Data input for module {\tt GEO:}}\label{sect:descgeo}

Structures \dstr{descgtyp} and \dstr{descgcnt} are used to define respectively
the type of geometry that will be define and the contents of this geometry
(dimensions, materials, boundary conditions). The module \moc{GEO:} can be
recursively called from 
\dstr{descgcnt} as an embedded module, in order to define sub-geometries:

\begin{DataStructure}{Structure \dstr{descgtyp}}
$\{$ \moc{VIRTUAL} $|$ \\
\moc{HOMOGE} $|$\\
\moc{SPHERE} \dusa{lr} $|$ \\
\moc{CAR1D} \dusa{lx} $|$ \\
\moc{CAR2D} \dusa{lx} \dusa{ly} $|$\\ 
\moc{CAR3D} \dusa{lx} \dusa{ly} \dusa{lz} $|$  \\
\moc{TUBE} \dusa{lr} $[$ \dusa{lx} \dusa{ly} $]$  $|$\\
\moc{TUBEX} \dusa{lr} $\{$ \dusa{lx} $|$ \dusa{lx} \dusa{ly} \dusa{lz} $\}$ $|$\\ 
\moc{TUBEY} \dusa{lr} $\{$ \dusa{ly} $|$ \dusa{lx} \dusa{ly} \dusa{lz} $\}$ $|$\\
\moc{TUBEZ} \dusa{lr} $\{$ \dusa{lz} $|$ \dusa{lx} \dusa{ly} \dusa{lz} $\}$ $|$ \\
\moc{RTHETA} \dusa{lr} \dusa{lz} $|$ \\
\moc{HEX} \dusa{lh} $|$ \\
\moc{HEXZ} \dusa{lh} \dusa{lz} $|$ \\
\moc{HEXT} \dusa{nhr} $|$ \\
\moc{HEXTZ} \dusa{nhr} \dusa{lz} $|$ \\
\moc{CARCEL} \dusa{lr} $[$ \dusa{lx} \dusa{ly} $]$ $|$\\
\moc{CARCELX} \dusa{lr} $\{$ \dusa{lx} $|$ \dusa{lx} \dusa{ly} \dusa{lz} $\}$ $|$ \\
\moc{CARCELY} \dusa{lr} $\{$ \dusa{ly} $|$ \dusa{lx} \dusa{ly} \dusa{lz} $\}$ $|$ \\ 
\moc{CARCELZ} \dusa{lr} $\{$ \dusa{lz} $|$ \dusa{lx} \dusa{ly} \dusa{lz} $\}$ $|$ \\
\moc{HEXCEL} \dusa{lr} $|$ \\
\moc{HEXCELZ} \dusa{lr} \dusa{lz} $|$ \\
\moc{HEXTCEL} \dusa{lr} \dusa{nhr}$|$ \\
\moc{HEXTCELZ} \dusa{lr} \dusa{nhr} \dusa{lz} $|$ \\
\moc{GROUP} \dusa{lp} $\}$
\end{DataStructure}

\begin{DataStructure}{Structure \dstr{descgcnt}}
$[$ \moc{EDIT} \dusa{iprint} $]$ \\
\dstr{descBC} \\
\dstr{descSP} \\
\dstr{descPP} \\
\dstr{descDH} \\
\dstr{descSIJ} \\
$[[$ \moc{:::} \dusa{SUBGEO} \moc{:=} \moc{GEO:} $\{$ \dstr{descgtyp} $|$
\dusa{SUBGEO} $|$
\dusa{OLDGEO} $\}$ \dstr{descgcnt}$]]$ \\
\moc{;} 
\end{DataStructure}

\noindent
where

\begin{ListeDeDescription}{mmmmmmmm}

\item[\moc{VIRTUAL}] keyword to specify that a virtual geometry description
follows. This type of geometry is used to complete an assembly that has
irregular boundaries.

\item[\moc{HOMOGE}] keyword to specify that a infinite homogeneous geometry
description follows.

\item[\moc{SPHERE}] keyword to specify that a spherical geometry  (concentric
spheres) description follows.

\item[\moc{CAR1D}] keyword to specify that a one dimensional plane geometry
(infinite slab) description follows.

\item[\moc{CAR2D}] keyword to specify that a two-dimensional Cartesian
geometry description follows.

\item[\moc{CAR3D}] keyword to specify that a three-dimensional Cartesian
geometry description follows.

\item[\moc{TUBE}] keyword to specify that a cylindrical geometry (infinite
tubes or cylinders) description follows. This geometry can contain an imbedded $X-Y$ Cartesian mesh.

\item[\moc{TUBEX}] keyword to specify that a polar $R-X$ cylindrical geometry
description follows. This geometry can contain an imbedded $Y-Z$ Cartesian mesh.

\item[\moc{TUBEY}] keyword to specify that a polar $R-Y$ cylindrical geometry
description follows. This geometry can contain an imbedded $Z-X$ Cartesian mesh.

\item[\moc{TUBEZ}] keyword to specify that a polar $R-Z$ cylindrical geometry
description follows. This geometry can contain an imbedded $X-Y$ Cartesian mesh.

\item[\moc{RTHETA}] keyword to specify that a polar geometry ($R-\theta$)
description follows.

\item[\moc{HEX}] keyword to specify that a two-dimensional hexagonal geometry
description follows.

\item[\moc{HEXZ}] keyword to specify that a three-dimensional hexagonal
geometry description follows.

\item[\moc{HEXT}] keyword to specify a single 2-D hexagonal cell geometry having a triangular mesh. This option is only supported by the \moc{NXT:} tracking module (see \Sect{TRKData}).

\item[\moc{HEXTZ}] keyword to specify a single $Z$ directed 3-D hexagonal cell geometry having a triangular mesh (plane $X-Y$). This option is only supported by the \moc{NXT:} tracking module (see \Sect{TRKData}).

\item[\moc{CARCEL}] keyword to specify that a two-dimensional mixed Cartesian
cell (concentric tubes surrounded by a rectangle) description follows. The rectangle can now be
subdivided into a fine mesh when the \moc{EXCELT:} modules is used.

\item[\moc{CARCELX}] keyword to specify that a three-dimensional mixed
Cartesian cell with tubes oriented along the $X-$axis description follows. The three-dimensional 
Cartesian cell can now be subdivided into a fine mesh when the \moc{EXCELT:}
module is used.

\item[\moc{CARCELY}] keyword to specify that a three-dimensional mixed
Cartesian cell with tubes oriented along the $Y-$axis description follows. The three-dimensional 
Cartesian cell can now be subdivided into a fine mesh when the \moc{EXCELT:}
module is used.

\item[\moc{CARCELZ}] keyword to specify that a three-dimensional mixed
Cartesian cell with tubes oriented along the $Z-$axis description follows. The three-dimensional 
Cartesian cell can now be subdivided into a fine mesh when the \moc{EXCELT:}
module is used.

\item[\moc{HEXCEL}] keyword to specify that a two-dimensional mixed hexagonal cell (concentric tubes surrounded by a hexagon) description follows.

\item[\moc{HEXCELZ}] keyword to specify that a three-dimensional mixed hexagonal cell with tubes oriented along the $Z-$axis description follows.

\item[\moc{HEXTCEL}] keyword to specify a single 2-D hexagonal cell geometry having a triangular mesh and containing concentric annular regions.

\item[\moc{HEXTCELZ}] keyword to specify a single $Z$ directed 3-D hexagonal cell geometry a triangular mesh and containing concentric $Z$ directed cylinders. 

\item[\moc{GROUP}] keyword to specify that a {\sl do-it-yourself} type geometry
description follows.

\item[\dusa{lx}] number of subdivisions along the $X-$axis (before
mesh-splitting).

\item[\dusa{ly}] number of subdivisions along the $Y-$axis (before
mesh-splitting).

\item[\dusa{lz}] number of subdivisions along the $Z-$axis (before
mesh-splitting).

\item[\dusa{lr}] number of cylinders or spherical shells (before
mesh-splitting).

\item[\dusa{lh}] number of hexagons in an axial plane (including the virtual
hexagon).

\item[\dusa{nhr}] number of concentric hexagons in a \moc{HEXT}, \moc{HEXTZ}, \moc{HEXTCEL} or \moc{HEXTCELZ}  cell (see \Fig{GeoHEXT4}). This will lead to an hexagon subdivided into $6N^{2}$ identical trangles.

\begin{figure}[h!]  
\begin{center} 
\parbox{9.0cm}{\epsfxsize=9cm \epsffile{GeoHEXT4.eps}}
\parbox{14cm}{\caption{Hexagonal geometry with triangular mesh containing 4 concentric hexagon}\label{fig:GeoHEXT4}}   
\end{center}  
\end{figure}

\item[\dusa{lp}] number of types of cells (number of cells inside which a distinct flux will be calculated) for a \textsl{do-it-yourself} type geometry.

\item[\moc{EDIT}] keyword used to modify the print level \dusa{iprint}.

\item[\dusa{iprint}] index used to control the printing in this module.
It must be set to 0 if no printing on the output file is required, to 1 for
minimum printing (fixed default value) and to 2 for printing the geometry state
vector.

\item[\dstr{descBC}] structure allowing the boundary conditions surrounding
the geometry to be treated (see \Sect{descBC}).

\item[\dstr{descSP}] structure allowing the coordinates of a geometry to be
described (see \Sect{descSP}).

\item[\dstr{descPP}] structure allowing material mixtures to be associated
with a geometry (see \Sect{descPP}).

\item[\dstr{descDH}] structure used to specify double-heterogeneity data (see \Sect{descDH}).

\item[\dstr{descSIJ}] structure used to specify the properties of {\sl do-it-yourself}
geometries (see \Sect{descSIJ}).

\item[\dusa{SUBGEO}] {\tt character*12} name of the directory  that will
contain the sub-geometry.

\item[\dusa{OLDGEO}] {\tt character*12} name of a parallel directory
containing an existing sub-geometry. The type and all the characteristics of
\dusa{OLDGEO} will be copied onto \dusa{SUBGEO}.

\end{ListeDeDescription}

Note that all the geometry described above are called {\sl pure geometry} when
they do not contain sub-geometry. When they do contain sub-geometry they will be
called {\sl composite geometry}. 

\goodbreak
\subsubsection{Boundary conditions}\label{sect:descBC}

The inputs corresponding to the \dstr{descBC} structure are the following:

\begin{DataStructure}{Structure \dstr{descBC}}
$[$ \moc{X-} $\{$ \moc{VOID} $|$ \moc{REFL} $|$ \moc{SSYM} $|$ \moc{DIAG} $|$ \moc{TRAN} $|$
\moc{SYME} $|$ \moc{ALBE} $\{$ \dusa{albedo} $|$ \dusa{icode} $\}$ $|$ \moc{ZERO}
$|$ \moc{PI/2} $|$ \moc{PI} \\
~~~~~~~~ $|$ \moc{CYLI} $|$ \moc{ACYL} $\{$ \dusa{albedo} $|$ \dusa{icode} $\}$ $\}$ $]$ \\
$[$ \moc{X+}   $\{$ \moc{VOID} $|$ \moc{REFL} $|$ \moc{SSYM} $|$ \moc{DIAG} $|$ \moc{TRAN} $|$
\moc{SYME} $|$ \moc{ALBE} $\{$ \dusa{albedo} $|$ \dusa{icode} $\}$ $|$ \moc{ZERO}
$|$ \moc{PI} \\
~~~~~~~~ $|$ \moc{CYLI} $|$ \moc{ACYL} $\{$ \dusa{albedo} $|$ \dusa{icode} $\}$ $\}$ $]$ \\
$[$ \moc{Y-}   $\{$ \moc{VOID} $|$ \moc{REFL} $|$ \moc{SSYM} $|$ \moc{DIAG} $|$ \moc{TRAN} $|$
\moc{SYME} $|$ \moc{ALBE} $\{$ \dusa{albedo} $|$ \dusa{icode} $\}$ $|$ \moc{ZERO}
$|$ \moc{PI/2} $|$ \moc{PI} \\
~~~~~~~~ $|$ \moc{CYLI} $|$ \moc{ACYL} $\{$ \dusa{albedo} $|$ \dusa{icode} $\}$ $\}$ $]$ \\
$[$ \moc{Y+}   $\{$ \moc{VOID} $|$ \moc{REFL} $|$ \moc{SSYM} $|$ \moc{DIAG} $|$ \moc{TRAN} $|$ 
\moc{SYME} $|$ \moc{ALBE} $\{$ \dusa{albedo} $|$ \dusa{icode} $\}$ $|$ \moc{ZERO}
$|$ \moc{PI} \\
~~~~~~~~ $|$ \moc{CYLI} $|$ \moc{ACYL} $\{$ \dusa{albedo} $|$ \dusa{icode} $\}$ $\}$ $]$ \\
$[$ \moc{Z-}   $\{$ \moc{VOID} $|$ \moc{REFL} $|$ \moc{SSYM} $|$ \moc{TRAN} $|$ \moc{SYME} $|$ 
\moc{ALBE} $\{$ \dusa{albedo} $|$ \dusa{icode} $\}$  $|$ \moc{ZERO} $\}$ $]$ \\
$[$ \moc{Z+}   $\{$ \moc{VOID} $|$ \moc{REFL} $|$ \moc{SSYM} $|$ \moc{TRAN} $|$ \moc{SYME} $|$
\moc{ALBE} $\{$ \dusa{albedo} $|$ \dusa{icode} $\}$  $|$ \moc{ZERO} $\}$ $]$ \\
$[$ \moc{R+}   $\{$ \moc{VOID} $|$ \moc{REFL} $|$
\moc{ALBE} $\{$ \dusa{albedo} $|$ \dusa{icode} $\}$  $|$ \moc{ZERO} $\}$ $]$ \\
$[$ \moc{HBC}  $\{$ \moc{S30} $|$ \moc{SA60} $|$ \moc{SB60} $|$ \moc{S90} $|$
\moc{R120} $|$ \moc{R180} $|$ \moc{SA180} $|$ \moc{SB180} $|$ 
\moc{COMPLETE} $\}$ \\ $\{$ \moc{VOID} $|$ \moc{REFL} $|$ \moc{SYME} $|$
\moc{ALBE} $\{$ \dusa{albedo} $|$ \dusa{icode} $\}$ $|$ \moc{ZERO} $\}$ $]$ \\
$[$ \moc{RADS} $[$ \moc{ANG} $]$ \dusa{nrads} (\dusa{xrad}(ir), \dusa{rrad}(ir) $[$, \dusa{ang}(ir) $]$, ir=1,nrads ) $]$
\end{DataStructure}

\noindent
where: 

\begin{ListeDeDescription}{mmmmm}

\item[\moc{X-}/\moc{X+}] keyword to specify the boundary conditions associated with the
negative or positive $X$ surface of a Cartesian geometry.

\item[\moc{Y-}/\moc{Y+}] keyword to specify the boundary conditions associated with the
negative or positive $Y$ surface of a Cartesian geometry.

\item[\moc{Z-}/\moc{Z+}] keyword to specify the boundary conditions associated with the
negative or positive $Z$ surface of a Cartesian geometry.          

\item[\moc{R+}] keyword to specify the boundary conditions associated with the
outer surface of a cylindrical or spherical geometry.

\item[\moc{HBC}] keyword to specify the boundary conditions associated with
the outer surface of an hexagonal geometry.

\item[\moc{VOID}] keyword to specify that the surface under consideration has
zero re-entrant angular flux. This side is an external surface of the domain.

\item[\moc{REFL}] keyword to specify that the surface under consideration has a reflective boundary condition. In 
most DRAGON calculations, this implies white boundary conditions. The main exception to this 
rule is when cyclic tracking in 2-D is considered and mirror like reflections are considered. A geometry is never
unfolded to take into account a \moc{REFL} boundary condition.

\item[\moc{SSYM}] keyword to specify that the surface under consideration has a specular (or mirror) reflective boundary condition. The 
main difference between \moc{REFL} and \moc{SSYM} is that for \moc{SSYM} the cell may be unfolded to take 
into account the reflection at the boundary.

\item[\moc{DIAG}] keyword to specify that the Cartesian surface under
consideration has the same properties as that associated with a diagonal through
the geometry (see \Fig{cartebc}). Note that two and only two \moc{DIAG} surfaces must be specified.
The diagonal symmetry is only permitted for square geometry and in the following
combinations:  

\begin{verbatim}
X+ DIAG Y- DIAG 
\end{verbatim}

\noindent
or

\begin{verbatim}
X- DIAG Y+ DIAG 
\end{verbatim}

\item[\moc{TRAN}] keyword to specify that the surface under consideration is
connected to the opposite surface of a Cartesian domain (see \Fig{cartebcr}).
This option  provides
the facility to treat an infinite geometry with translation symmetry. The only
combinations of translational symmetry permitted are:

\begin{itemize}
\item Translation along the $X-$axis

\begin{verbatim}
X- TRAN X+ TRAN 
\end{verbatim}

\item Translation along the $Y-$axis

\begin{verbatim}
Y- TRAN Y+ TRAN 
\end{verbatim}

\item Translation along the $Z-$axis

\begin{verbatim}
Z- TRAN Z+ TRAN 
\end{verbatim}

\end{itemize}

\item[\moc{SYME}] keyword to specify that the Cartesian surface under
consideration is virtual and that a reflection symmetry is associated with the
adequately directed axis running through the center of the cells closest to this
surface (see \Fig{cartebcr}). Only the hexagonal geometries \moc{S30} and \moc{SA60} can be
surrounded by a \moc{SYME} boundary condition if a specular condition
is to be applied on this boundary.

\item[\moc{ALBE}] keyword to specify that the surface under consideration has
an arbitrary albedo. This side is an external surface of the domain.

\item[\dusa{albedo}] geometric albedo corresponding to the boundary condition
\moc{ALBE} (\dusa{albedo}$>$0.0). 

\item[\dusa{icode}] index of a physical albedo corresponding to the boundary
condition \moc{ALBE}. The numerical values of the physical albedo are supplied
by the operator \moc{MAC:} (see \Sect{MACData}).

\item[\moc{ZERO}] keyword to specify that the surface under consideration has a
zero-flux boundary condition. This side is an external surface of the domain.

\item[\moc{PI/2}] keyword to specify that the surface under consideration has a
$\pi$/2 rotational symmetry (see \Fig{cartebcr}). The only $\pi$/2 symmetry permitted is related to
sides ({\tt X-} and {\tt Y-}). This condition can be combined with a translation
boundary condition:({\tt PI/2 X- TRAN X+}) and/or ({\tt PI/2 Y- TRAN Y+}) (see \Fig{cartebct}).

\item[\moc{PI}] keyword to specify that the surface under consideration has a
$\pi$ rotational symmetry (see \Fig{cartebcr}). This keyword is useful for representing a
Cartesian checkerboard pattern as shown in Fig.~\ref{fig:cartebcdam}.

\item[\moc{CYLI}] the side under consideration has a zero incoming current boundary condition
with a circular correction applied on the Cartesian boundary. This option is only available in
the $X$--$Y$ plane for \moc{CAR2D} and \moc{CAR3D} geometries defined for TRIVAC full--core calculations.

\item[\moc{ACYL}] the side under consideration has an arbitrary albedo with a circular correction
applied on the Cartesian boundary. This option is only available in
the $X$--$Y$ plane for \moc{CAR2D} and \moc{CAR3D} geometries defined for TRIVAC full--core calculations.

\item[\moc{S30}] keyword to specify an hexagonal symmetry of one twelfth of an
assembly (see \Fig{s30}).

\item[\moc{SA60}] keyword to specify an hexagonal symmetry of one sixth of an
assembly of type A (see \Fig{s30}).

\item[\moc{SB60}] keyword to specify an hexagonal symmetry of one sixth of an
assembly of type B (see \Fig{sb60}).

\item[\moc{S90}] keyword to specify an hexagonal symmetry of one quarter of an
assembly (see \Fig{sb60}).

\item[\moc{R120}] keyword to specify a rotation symmetry of one third of an
assembly (see \Fig{r120}).

\item[\moc{R180}] keyword to specify a rotation symmetry of a half assembly
(see \Fig{r120}).

\item[\moc{SA180}] keyword to specify an hexagonal symmetry of half a type A
assembly (see \Fig{sa180}).

\item[\moc{SB180}] keyword to specify an hexagonal symmetry of half a type B
assembly (see \Fig{sb180}).

\item[\moc{COMPLETE}] keyword to specify a complete hexagonal assembly (see
\Fig{compl}).

\item[\moc{RADS}] This key word is used to specify the cylindrical correction applied in the $X-Y$ plane for \moc{CAR2D} and \moc{CAR3D} geometries.\cite{roy}

\item[\moc{ANG}] This key word allows  the angle (see \Fig{corr})
of the cylindrical notch to be set. By default, no notch is present.

\item[\dusa{nrads}] Number of different corrections along the cylinder main axis (i.e. the $Z$ axis).

\item[\dusa{xrad}(ir)] Coordinate of the $Z$ axis from which the correction is applied.

\item[\dusa{rrad}(ir)] Radius of the real cylindrical boundary.

\item[\dusa{ang}(ir)] Angle of the cylindrical notch. This data is given if and only if the key word \moc{ANG} is present. \dusa{ang}(ir) $= {\pi \over 2}$ by default (i.e. the correction is applied at every angle).

\end{ListeDeDescription}
\goodbreak

\begin{figure}[!]  
\begin{center} 
\epsfxsize=13cm
\centerline{ \epsffile{ebc.eps}}
\parbox{14cm}{\caption{Diagonal boundary conditions in Cartesian geometry}\label{fig:cartebc}}   
\end{center}  
\end{figure}

\begin{figure}[!]  
\begin{center} 
\epsfxsize=15cm
\centerline{ \epsffile{ebcr.eps}}
\parbox{14cm}{\caption{Various boundary conditions in Cartesian geometry}\label{fig:cartebcr}}   
\end{center}  
\end{figure}

\begin{figure}[!]  
\begin{center} 
\epsfxsize=10cm
\centerline{ \epsffile{ebct.eps}}
\parbox{14cm}{\caption{Translation/rotation boundary conditions in Cartesian geometry}\label{fig:cartebct}}   
\end{center}  
\end{figure}

\begin{figure}[!]  
\begin{center} 
\epsfxsize=13cm
\centerline{ \epsffile{ebcdam.eps}}
\parbox{14cm}{\caption{Representing a checkerboard in Cartesian geometry}\label{fig:cartebcdam}}   
\end{center}  
\end{figure}

\begin{figure}[!]  
\begin{center}
\epsfxsize=15cm
\centerline{ \epsffile{Gs30.eps}}
\parbox{14cm}{\caption{Hexagonal geometries of type S30 and SA60}\label{fig:s30}}   
\end{center}  
\end{figure}

\begin{figure}[!]  
\begin{center} 
\epsfxsize=15cm
\centerline{ \epsffile{Gsb60.eps}}
\parbox{14cm}{\caption{Hexagonal geometries of type SB60 and S90}\label{fig:sb60}}   
\end{center}  
\end{figure}

\begin{figure}[!]  
\begin{center} 
\epsfxsize=12cm
\centerline{ \epsffile{Gr120.eps}}
\parbox{14cm}{\caption{Hexagonal geometries of type R120 and R180}\label{fig:r120}}   
\end{center}  
\end{figure}

\begin{figure}[!]  
\begin{center} 
\epsfxsize=5cm
\centerline{ \epsffile{Gsa180.eps}}
\parbox{14cm}{\caption{Hexagonal geometry of type SA180}\label{fig:sa180}}   
\end{center}  
\end{figure}

\begin{figure}[!]  
\begin{center} 
\epsfxsize=11cm
\centerline{ \epsffile{Gsb180.eps}}
\parbox{14cm}{\caption{Hexagonal geometry of type SB180}\label{fig:sb180}}   
\end{center}  
\end{figure}

\begin{figure}[!]  
\begin{center} 
\epsfxsize=10cm
\centerline{ \epsffile{Gcomplete.eps}}
\parbox{14cm}{\caption{Hexagonal geometry of type
COMPLETE}\label{fig:compl}}    
\end{center}  
\end{figure}

\begin{figure}[!]
\begin{center} 
\epsfxsize=6cm
\centerline{ \epsffile{Fig6.eps}}
\parbox{14cm}{\caption{Cylindrical correction in Cartesian geometry}
\label{fig:corr}} 
\end{center} 
\end{figure}

\clearpage
\subsubsection{Spatial properties of geometry}\label{sect:descSP}
                                                  
The \dstr{descSP} structure has the following contents:
\begin{DataStructure}{Structure \dstr{descSP}}
$[$ \moc{MESHX} (\dusa{xxx}($i$), $i$=1,\dusa{lx}+1) $]$\\
$[$ \moc{SPLITX} (\dusa{ispltx}($i$), $i$=1,\dusa{lx}) $]$\\
$[$ \moc{MESHY}  (\dusa{yyy}($i$), $i$=1,\dusa{ly}+1) $]$\\
$[$ \moc{SPLITY} (\dusa{isplty}($i$), $i$=1,\dusa{ly}) $]$\\
$[$ \moc{MESHZ}  (\dusa{zzz}($i$), $i$=1,\dusa{lz}+1) $]$\\
$[$ \moc{SPLITZ} (\dusa{ispltz}($i$), $i$=1,\dusa{lz}) $]$\\
$[$ \moc{RADIUS} (\dusa{rrr}($i$), $i$=1,\dusa{lr}+1) $]$\\
$[$ \moc{OFFCENTER} (\dusa{disxyz}($i$), $i$=1,3) $]$\\
$[$ \moc{SPLITR} (\dusa{ispltr}($i$), $i$=1,\dusa{lr}) $]$\\
$[$ \moc{SECT} \dusa{isect} $[$ \dusa{jsect} $]~]$\\
$[$ \moc{SIDE} \dusa{sideh} $[$ \dusa{hexmsh} $]$ $]$\\
$[~\{$ \moc{SPLITH} \dusa{isplth} $|$ \moc{SPLITL} \dusa{ispltl} $\}~]$\\
$[$ $\{$ \moc{NPIN}  \dusa{npins} \\
\hspace{0.75cm} $\{$  $[$ \moc{RPIN} $\{$ \dusa{rpins} $|$ (\dusa{rpins}($i$), $i$=1, \dusa{npins}) $\}$ $]$ \\
\hspace{1.0cm} $[$ \moc{APIN}  $\{$ \dusa{apins} $|$ (\dusa{apins}($i$), $i$=1, \dusa{npins}) $\}$ $]$  $|$ \\
\hspace{1.0cm} $[$ \moc{CPINX} (\dusa{xpins}($i$), $i$=1, \dusa{npins})  $]$ \\
\hspace{1.0cm} $[$ \moc{CPINY} (\dusa{ypins}($i$), $i$=1, \dusa{npins})  $]$   \\
\hspace{1.0cm} $[$ \moc{CPINZ} (\dusa{zpins}($i$), $i$=1, \dusa{npins})  $]$   $\}$\\
\hspace{0.3cm}$|$ \moc{DPIN}  \dusa{dpins} $\}$ $]$
\end{DataStructure}

\begin{ListeDeDescription}{mmmmmmmm}

\item[\moc{MESHX}] keyword to specify the spatial mesh defining the regions along the $X-$axis. 

\item[\dusa{xxx}] array giving the $X$ limits (cm) of the regions making up the geometry. These values
must be given in order, from \moc{X-} to \moc{X+}. If the geometry presents a diagonal symmetry the same
data is also used along the $Y-$axis.

\item[\moc{SPLITX}] keyword to specify that a mesh splitting of the geometry along the $X-$axis is to be
performed.

\item[\dusa{ispltx}] array giving the number of zones that will be considered for each region along the
$X-$axis. If the geometry presents a diagonal symmetry this information is also used for the splitting
along the $Y-$axis. By default,
\dusa{ispltx}=1.

\item[\moc{MESHY}] keyword to specify the spatial mesh defining the regions along the $Y-$axis.

\item[\dusa{yyy}] array giving the $Y$ limits (cm) of the regions making up the geometry. These values
must be given in order, from \moc{Y-} to \moc{Y+}.

\item[\moc{SPLITY}] keyword to specify that a mesh splitting of the geometry along the $Y-$axis is to be
performed.

\item[\dusa{isplty}] array giving the number of zones that will be considered for each region along the
$Y-$axis. By default,
\dusa{isplty}=1 unless a diagonal symmetry is used in which case \dusa{isplty}$=$\dusa{ispltx}.

\item[\moc{MESHZ}] keyword to specify the spatial mesh defining the regions along the $Z-$axis.

\item[\dusa{zzz}] array giving the $Z$ limits (cm) of the regions making up the geometry. These values
must be given in order, from  \moc{Z-} to \moc{Z+}.

\item[\moc{SPLITZ}] keyword to specify that a mesh splitting of the geometry along the $Z-$axis is to be
performed.

\item[\dusa{ispltz}] array giving the number of zones that will be considered for each region along the
$Z-$axis. By default,
\dusa{ispltz}=1.

\item[\moc{RADIUS}] keyword to specify the spatial mesh along the radial direction.

\item[\dusa{rrr}] array giving the radial limits (cm) of the annular
regions (cylindrical or spherical) making up the geometry. It is used for the
following geometries: \moc{TUBE}, \moc{TUBEZ}, \moc{SPHERE}), \moc{CARCEL},
\moc{CARCELX}, \moc{CARCELY}, \moc{CARCELZ}, \moc{HEXCEL} and \moc{HEXCELZ}. It
is important to note that we must have \dusa{rrr}(1)=0.0. The other values
of \dusa{rrr}($i$) in a \moc{CARCEL}-- or \moc{HEXCEL}--type geometry are
defined as shown in \Fig{radius}.

\item[\moc{OFFCENTER}] keyword to specify that the concentric annular regions in a \moc{CARCEL},
\moc{CARCELX}, \moc{CARCELY},
\moc{CARCELZ}, \moc{TUBE}, \moc{TUBEX}, \moc{TUBEY} and \moc{TUBEZ} geometry can now be displaced with
respect to the center of the Cartesian mesh. This option will only be treated when the \moc{EXCELT:},
\moc{NXT:} and \moc{EXCELL:} modules are used.

\item[\dusa{disxyz}] array giving the $x$ (\dusa{disxyz}(1)), $y$ (\dusa{disxyz}(2)) and $z$
(\dusa{disxyz}(3)) displacement (cm) of the concentric annular regions with respect to the center of the
Cartesian mesh. 

\item[\moc{SPLITR}] keyword to specify that a mesh splitting of the geometry along the radial direction is
to be performed.

\item[\dusa{ispltr}] array giving the number of zones that will be considered for each region along the
radial axis.  A negative value results in a splitting of the regions into zones of equal volumes; a
positive value results in a uniform splitting along the radial direction. By default, \dusa{ispltr}=1.

\item[\moc{SECT}] keyword to specify the type of sectorization for a Cartesian
or hexagonal cell. In hexagonal geometry, this keyword is expected to be defined near the
\moc{SIDE} keyword. By default, no sectorization is performed.

\item[\dusa{isect}] sectorization index, defined as
\begin{displaymath}
\negthinspace\negthinspace\negthinspace isect = \left\{
\begin{array}{rl}
-999: & \textrm{non-sectorized cell processed as a sectorized cell} \\
-1: & \textrm{$\times$--type sectorization} \\
 0: & \textrm{non-sectorized cell} \\
 1: & \textrm{$+$--type sectorization} \\
 2: & \textrm{simultaneous $\times$-- and $+$--type sectorization} \\
 3: & \textrm{simultaneous $\times$-- and $+$--type sectorization shifted by 22.5$^\circ$} \\
 4: & \textrm{windmill sectorization.} 
\end{array} \right.
\end{displaymath}

\item[\dusa{jsect}] number of embedded tubes that are {\sl not} sectorized, with 0 $\le$ \dusa{jsect} $\le$ \dusa{lr}. By default, \dusa{jsect} $=0$. Examples of sectorization options are depicted in Figs.~\ref{fig:rect3} and~\ref{fig:hexa3}.

\item[\moc{SIDE}] keyword to specify the length of a side of a hexagon.

\item[\dusa{sideh}] length of one side of a hexagon (cm).

\item[\dusa{hexmsh}] triangular mesh for \moc{HEXT}, \moc{HEXTCEL}, \moc{HEXTZ} and \moc{HEXTCELZ} hexagonal geometries. By default, \dusa{hexmsh}=\dusa{sideh}/\dusa{nhr}. When \dusa{hexmsh} is provided, it is used instead of the default value with the following constraints 
$$
\textit{sideh} \le \textit{nhr}\times \textit{hexmsh}<\textit{sideh}+\textit{hexmsh}
$$
The triangles in the last hexagonal ring are truncated at \dusa{sideh} (see \Fig{GeoHEXT4C}).

\item[\moc{SPLITH}] keyword to specify that a triangular mesh splitting of the hexagonal geometry is to be performed -- for \moc{HEX}, \moc{HEXZ}, \moc{HEXT}, \moc{HEXTCEL}, \moc{HEXTZ} and \moc{HEXTCELZ} type geometries. This is valid only if \dusa{nhr}=1. 

\item[\dusa{isplth}] value of the triangular mesh splitting. Its use is similar to \dusa{nhr} except that each sector of the hexagonal cell will be filled by a unique mixture. The number of triangles per hexagon is given by $6 \times$\dusa{isplth}$^2$.
\dusa{isplth} $=0$ is used for full hexagon discretization.

\item[\moc{SPLITL}] keyword to specify that a lozenge mesh splitting of the hexagonal geometry is to be performed -- for \moc{HEX} and \moc{HEXZ} type geometries.

\item[\dusa{ispltl}] value of the lozenge splitting. The number of lozenges per hexagon is given by $3 \times$\dusa{ispltl}$^2$.

\item[\moc{NPIN}] keyword to specify the number of pins located in a cluster geometry. It can only be used for \moc{SPHERE}, \moc{TUBE}, \moc{TUBEX}, \moc{TUBEY} and \moc{TUBEZ} sub-geometry.

\item[\dusa{npins}] the number of pins associated with this sub-geometry in the primary geometry. 

\item[\moc{DPIN}] keyword to specify the pin density in a geometry that contains clusters. A number $N_{p,r}$ of pins that will be placed randomly in the geometry with
$$
N_{p,r}=\textrm{NINT}\left(\frac{d_{p,r}V_{c}}{V_{p}}\right)
$$
where $d_{p,r}$ is the pin density, $V_{g}$ the volume of the cell containing these pins and$V_{p}$ the volume of this pin type. The function $\textrm{NINT}()$ provides the nearest integer associated with its real argument. It can only be used for \moc{SPHERE}, \moc{TUBE}, \moc{TUBEX}, \moc{TUBEY} and \moc{TUBEZ} sub-geometry.

\item[\dusa{dpins}] the pin density $d_{p,r}$. 

\item[\moc{RPIN}] keyword to specify the radius of an imaginary cylinder where the centers of the pins are to be placed in a cluster geometry.

\item[\dusa{rpins}] the radius (cm) of an imaginary cylinder where the centers of the pins are to be placed. In the case where a single value is provided for \dusa{rpins}, all the pins are located at the same distance from the center of the cell (taking account the offset provided by the keyword \moc{OFFCENTER}). 

\item[\moc{APIN}] keyword to specify the angle of the first pin or each pin centered on an imaginary cylinder in a cluster geometry. 

\item[\dusa{apins}] the angle (radian) of the first pin in the ring (only one value provided for \dusa{apins}, the angular spacing of the pins being $2\pi/$\dusa{npins}) or the angle of each pins in the ring.

\item[\moc{CPINX}] keyword to specify the $x$ position where the centers of the pins are
to be placed in a cluster geometry. 

\item[\dusa{xpins}] the $x$ position (cm) where the centers of the pins are to be
placed.

\item[\moc{CPINY}] keyword to specify the $y$ position where the centers of the pins are
to be placed in a cluster geometry. 

\item[\dusa{ypins}] the $y$ position (cm) where the centers of the pins are to be
placed.

\item[\moc{CPINZ}] keyword to specify the $z$ position where the centers of the pins are
to be placed in a cluster geometry. 

\item[\dusa{zpins}] the $z$ position (cm) where the centers of the pins are to be
placed.

\end{ListeDeDescription}

\begin{figure}[!]  
\begin{center} 
\epsfxsize=6cm
\centerline{ \epsffile{radius.eps}}
\parbox{16cm}{\caption{Definition of the radii in a \moc{CARCEL}-- or
\moc{HEXCEL}--type geometry}\label{fig:radius}}    
\end{center}  
\end{figure}

The user should be warned that the maximum number of zones resulting from the above description of a geometry $L_{\rm{zones}}$ should not exceed the limits imposed by
\dusa{maxreg} and defined in the tracking module \moc{SYBILT:}, \moc{NXT:} or
\moc{EXCELT:} (see \Sect{TRKData}). For pure geometry with splitting we can define the variables $L_x$, $L_y$, $L_z$, $L_r$, $L_h$ and $L_{t}$ as:
  \begin{align*}
  L_x=&\sum_{i=1}^{\textit{lx}} \textit{ispltx}(i) \\ 
  L_y=&\sum_{i=1}^{\textit{ly}} \textit{isplty}(i) \\ 
  L_z=&\sum_{i=1}^{\textit{lz}} \textit{ispltz}(i) \\ 
  L_r=&\sum_{i=1}^{\textit{lr}} |\textit{ispltr}(i)| \\
  L_h=&\textit{lh} \\
  L_t=&\begin{cases}
  6\times\textit{nhr}^{2} &if $\textit{nhr}> 1$\\
  6\times\textit{isplith}^{2} &otherwise  \\ \end{cases}
  \end{align*}
and $L_{\rm{zones}}$ will be given by:

\begin{itemize}

\item \moc{SPHERE} geometry.

$$L_{\rm{zones}}=L_r$$

\item \moc{TUBE} geometry.

$$L_{\rm{zones}}= L_x L_y L_r $$

\item \moc{TUBEX} geometry.

$$L_{\rm{zones}}= L_x L_y L_z L_r$$

\item \moc{TUBEY} geometry.

$$L_{\rm{zones}}= L_x L_y L_z L_r$$

\item \moc{TUBEZ} geometry.

$$L_{\rm{zones}}= L_x L_y L_z L_r$$

\item \moc{CAR1D} geometry.

$$L_{\rm{zones}}=L_x$$

\item \moc{CAR2D} geometry 
\begin{itemize}
\item without diagonal symmetry. 

$$L_{\rm{zones}}=L_x L_y$$

\item with diagonal symmetry. 

$$L_{\rm{zones}}=\frac{L_x (L_y+1)}{2}=\frac{(L_x+1) L_y}{2}$$
\end{itemize}

\item \moc{CARCEL} geometries.

$$L_{\rm{zones}}=L_x L_y (L_r+1) $$

\item \moc{CAR3D} geometry 
\begin{itemize}
\item without diagonal symmetry. 

$$L_{\rm{zones}}=L_x L_y L_z$$

\item with diagonal symmetry. 

$$L_{\rm{zones}}=\frac{L_x (L_y+1) L_z}{2}=\frac{(L_x+1) L_y L_z}{2}$$
\end{itemize}

\item \moc{CARCELX} geometry.

$$L_{\rm{zones}}=L_x L_y L_z (L_r+1) $$

\item \moc{CARCELY} geometry.

$$L_{\rm{zones}}=L_x L_y L_z (L_r+1) $$

\item \moc{CARCELZ} geometries.

$$L_{\rm{zones}}=L_x L_y L_z (L_r+1) $$

\item \moc{HEX} geometry.

\begin{align*}L_{\text{zones}}&=L_h\end{align*}

\item \moc{HEXT} geometry.

\begin{align*}L_{\text{zones}}&=L_{t}\end{align*}

\item \moc{HEXCEL} geometries.

\begin{align*}L_{\text{zones}}&=(L_r+1) \end{align*}

\item \moc{HEXTCEL} geometries.

$$L_{\rm{zones}}=L_{t}$$

\item \moc{HEXZ} geometry.

\begin{align*}L_{\text{zones}}&=L_z L_h\end{align*}

\item \moc{HEXTZ} geometry.

\begin{align*}L_{\text{zones}}&=L_z L_{t}\end{align*}

\item \moc{HEXCELZ} geometries.

\begin{align*}L_{\text{zones}}&=L_z (L_r+1) \end{align*}

\item \moc{HEXTCELZ} geometries.

\begin{align*}L_{\text{zones}}&=L_z L_{t} (L_r+1) \end{align*}

\end{itemize}

For cluster geometries, only one region is associated with each zone in a pin even if this pin is repeated \dusa{npins} times.

\vskip 0.08cm

For mixed geometries, it is important to ensure that $L_{\rm{zones}}$ which represents the
sum over all the sub-geometries of the total number of regions $L^i_t$
associated with each pure sub-geometry $i$ computed using the technique
described above. For cluster geometries, only one region is associated with each
zone in a pin even if this pin is repeated \dusa{npins} times.

\begin{figure}[h!]
\begin{center} 
\epsfxsize=16cm
\centerline{ \epsffile{rect3c.eps}}
\parbox{14cm}{\caption{Numerotation of the sectors in a Cartesian cell}\label{fig:rect3}}   
\end{center}
\end{figure}

\begin{figure}[h!]
\begin{center} 
\epsfxsize=13cm
\centerline{ \epsffile{hexa3c.eps}}
\parbox{14cm}{\caption{Numerotation of the sectors in an hexagonal cell}\label{fig:hexa3}}   
\end{center}
\end{figure}

\begin{figure}[h!]  
\begin{center} 
\parbox{11.0cm}{\epsfxsize=11cm \epsffile{GeoHEXT4C.eps}}
\parbox{14cm}{\caption{Hexagonal geometry with triangular mesh that extends past the hexagonal boundary}\label{fig:GeoHEXT4C}}   
\end{center}  
\end{figure}

\subsubsection{Physical properties of geometry}\label{sect:descPP}

In addition to specifying the mixture associated with each region in the
geometry, the \dstr{descPP} structure is also used to provide information on the
sub-geometry required in this geometry. For example, an optional procedure in
DRAGON groups together regions so as to reduce the number of unknowns
\dusa{maxreg} in the flux calculation. In this way, only the merged regions
contribute to the cost of the calculation. However, the following points must be
considered:

\begin{enumerate}

\item All the cells belonging to the same merged region must have the same
nuclear properties and dimensions. 

\item The grouping procedure is based on the approximation that all the regions
belonging to the same merged region share the same flux. 

\item The merging can also take into account region orientation  (by a rotation
and/or transposition) before they are merged. This procedure facilitates the
merging of regions when a \moc{DIAG} or \moc{SYME} boundary condition is used. 

\end{enumerate}
The \dstr{descPP} structure has the following contents: 

\begin{DataStructure}{Structure \dstr{descPP}}
$[$ \moc{MIX} $\{$  (\dusa{imix}(i),i=1,$n_t$) $[$ \moc{REPEAT} $]~|$\\
$~~~~[[$ \moc{PLANE} \dusa{iplan} $\{$ (\dusa{imix}(i),i=1,\dusa{lp}) $|$ \moc{SAME} \dusa{iplan1}\\
$~~~~|~[[$ \moc{CROWN} $\{$ (\dusa{imix}(i),i=1,\dusa{lc}) $|$ \moc{ALL} \dusa{jmix} $|$ \moc{SAME} \dusa{iplan1} $\}~]]$\\
$~~~~|~[[$ \moc{UPTO} \dusa{ic} \moc{ALL} \dusa{jmix} $|$ \moc{SAME} \dusa{iplan1} $\}~]]~]]~\}$\\
$]$\\
$[$ \moc{HMIX}  (\dusa{ihmix}(i), i=1,$N_t$) $[$ \moc{REPEAT} $]$ $]$\\
$[$ \moc{CELL}  (\dusa{HCELL}(i),i=1,$N_t$) $]$\\
$[$ \moc{MERGE} (\dusa{imerge}(i),i=1,$N_t$) $]$\\
$[$ \moc{TURN}  (\dusa{HTURN}(i),i=1,$N_t$) $]$\\
$[$ \moc{CLUSTER} (\dusa{NAMPIN}(i),i=1,$N_p$) $]$\\
$[$ \moc{MIX-NAMES} (\dusa{NAMMIX}(i),i=1,\dusa{maxmix}) $]$
\end{DataStructure}

\noindent

Here $N_p$ is the number of pin types in the cluster. In addition to the real (physical) mixture \dusa{imix} present in a given region of space and specified by the keyword \moc{MIX}, a virtual mixture \dusa{ihmix} can also be provided using the keyword \moc{HMIX}. This mixture can be used to identify the regions that will be combined in the \moc{EDI:} module to create homogenized region \dusa{ihmix} (see \Sect{EDIData}). Here $N_{t}$
is computed in a way similar to $L_{\rm zones}$ namely
\begin{itemize}

\item \moc{SPHERE} geometry.

$$N_{t}=\textit{lr}$$

The mixtures are then given in the following order
\begin{enumerate}
\item radially outward ($l=1,\textit{lr}$).
\end{enumerate}

\item \moc{TUBE} geometry.

$$N_{t}=\textit{lr}\times\textit{lx}\times \textit{ly}  $$

The mixtures are then given in the following order
\begin{enumerate}
\item radially outward ($l=1,\textit{lr}$) and such that  \dusa{imix} is arbitrary (not used) if radial region $l$ does not intersect Cartesian region $(i,j)$;
\item from surface \moc{X-} to surface \moc{X+} ($i=1,\textit{lx}$ for each $j$);
\item from surface \moc{Y-} to surface \moc{Y+} ($j=1,\textit{ly}$).
\end{enumerate}

\item \moc{TUBEX} geometry.

$$N_{t}=\textit{lr}\times\textit{ly}\times \textit{lz}\times \textit{lx}$$
The mixtures are then given in the following order
\begin{enumerate}
\item radially outward ($l=1,\textit{lr}$) and such that \dusa{imix} is arbitrary (not used) if radial region $l$ does not intersect Cartesian region $(j,k,i)$;
\item from surface \moc{Y-} to surface \moc{Y+} ($j=1,\textit{ly}$ for each $k$ and $i$);
\item from surface \moc{Z-} to surface \moc{Z+} ($k=1,\textit{lz}$ for each $i$);
\item from surface \moc{X-} to surface \moc{X+} ($i=1,\textit{lx}$).
\end{enumerate}

\item \moc{TUBEY} geometry.

$$N_{t}=\textit{lr}\times\textit{lz}\times \textit{lx}\times \textit{ly}$$
The mixtures are then given in the following order
\begin{enumerate}
\item radially outward ($l=1,\textit{lr}$) and such that  \dusa{imix} is arbitrary (not used) if radial region $l$ does not intersect Cartesian region $(k,i,j)$;
\item from surface \moc{Z-} to surface \moc{Z+} ($k=1,\textit{lz}$ for each $i$ and $j$);
\item from surface \moc{X-} to surface \moc{X+} ($i=1,\textit{lx}$ for each $j$);
\item from surface \moc{Y-} to surface \moc{Y+} ($j=1,\textit{ly}$).
\end{enumerate}

\item \moc{TUBEZ} geometry.

$$N_{t}= \textit{lr}\times\textit{lx}\times \textit{ly}\times \textit{lz}$$

The mixtures are then given in the following order
\begin{enumerate}
\item radially outward ($l=1,\textit{lr}$) and such that \dusa{imix} is arbitrary (not used) if radial region $l$ does not intersect Cartesian region $(i,j,k)$;
\item from surface \moc{X-} to surface \moc{X+} ($i=1,\textit{lx}$ for each $j$ and $k$);
\item from surface \moc{Y-} to surface \moc{Y+} ($j=1,\textit{ly}$ for each $k$);
\item from surface \moc{Z-} to surface \moc{Z+} ($k=1,\textit{lz}$).
\end{enumerate}

\item \moc{CAR1D} geometry.

$$N_{t}=\textit{lx}$$

The mixtures are then given in the following order
\begin{enumerate}
\item from surface \moc{X-} to surface \moc{X+} ($i=1,\textit{lx}$).
\end{enumerate}

\item \moc{CAR2D} geometry 
\begin{itemize}
\item without diagonal symmetry. 

$$N_{t}=\textit{lx}\times \textit{ly}$$

The mixtures or cells are then given in the following order
\begin{enumerate}
\item from surface \moc{X-} to surface \moc{X+} ($i=1,\textit{lx}$ for each $j$);
\item from surface \moc{Y-} to surface \moc{Y+} ($j=1,\textit{ly}$).
\end{enumerate}

\item with diagonal symmetry (\moc{X-} and \moc{Y+}). 

$$N_{t}=\frac{\textit{lx}\times (\textit{lx}+1)}{2}$$

The mixtures or cells are then given in the following order
\begin{enumerate}
\item from surface \moc{X-} to surface \moc{X+} ($i=j,\textit{lx}$ for each $j$);
\item from surface \moc{Y-} to surface \moc{Y+} ($j=1,\textit{ly}$).
\end{enumerate}

\item with diagonal symmetry (\moc{X+} and \moc{Y-}). 

$$N_{t}=\frac{\textit{lx}\times (\textit{lx}+1)}{2}$$

The mixtures or cells are then given in the following order
\begin{enumerate}
\item from surface \moc{X-} to surface \moc{X+} ($i=1,j$ for each $j$);
\item from surface \moc{Y-} to surface \moc{Y+} ($j=1,\textit{ly}$).
\end{enumerate}
\end{itemize}

\item \moc{CARCEL} geometries.

$$N_{t}=(\textit{lr}+1)\times\textit{lx}\times \textit{ly} $$

The mixtures are then given in the following order
\begin{enumerate}
\item radially outward ($l=1,\textit{lr}$) and such that \dusa{imix} is arbitrary (not used) if radial region $l$ does not intersect Cartesian region $(i,j)$;
\item $l=\textit{lr+1}$ for the mixture outside the annular regions but inside Cartesian region $(i,j)$;
\item from surface \moc{X-} to surface \moc{X+} ($i=1,\textit{lx}$ for each $j$);
\item from surface \moc{Y-} to surface \moc{Y+} ($j=1,\textit{ly}$).
\end{enumerate}

\item \moc{CAR3D} geometry 
\begin{itemize}
\item without diagonal symmetry. 

$$N_{t}=\textit{lx}\times \textit{ly}\times \textit{lz}$$

The mixtures or cells are then given in the following order
\begin{enumerate}
\item from surface \moc{X-} to surface \moc{X+} ($i=1,\textit{lx}$ for each $j$ and $k$);
\item from surface \moc{Y-} to surface \moc{Y+} ($j=1,\textit{ly}$ for $k$);
\item from surface \moc{Z-} to surface \moc{Z+} ($k=1,\textit{lz}$).
\end{enumerate}

\item with diagonal symmetry (\moc{X-} and \moc{Y+}). 

$$N_{t}=\frac{\textit{lx}\times (\textit{lx}+1)}{2}\times\textit{lz}$$

The mixtures or cells are then given in the following order
\begin{enumerate}
\item from surface \moc{X-} to surface \moc{X+} ($i=j,\textit{lx}$ for each $j$ and $k$);
\item from surface \moc{Y-} to surface \moc{Y+} ($j=1,\textit{ly}$) for each $k$);
\item from surface \moc{Z-} to surface \moc{Z+} ($k=1,\textit{lz}$).
\end{enumerate}
 

\item with diagonal symmetry (\moc{X+} and \moc{Y-}). 

$$N_{t}=\frac{\textit{lx}\times (\textit{lx}+1)}{2}\times\textit{lz}$$

The mixtures or cells are then given in the following order
\begin{enumerate}
\item from surface \moc{X-} to surface \moc{X+} ($i=1,j$ for each $j$ and $k$);
\item from surface \moc{Y-} to surface \moc{Y+} ($j=1,\textit{ly}$ for each $k$);
\item from surface \moc{Z-} to surface \moc{Z+} ($k=1,\textit{lz}$).
\end{enumerate}

\end{itemize}

\item \moc{CARCELX} geometry.

$$N_{t}=(\textit{lr}+1)\times\textit{ly}\times \textit{lz}\times \textit{lx} $$

The mixtures are then given in the following order
\begin{enumerate}
\item radially outward ($l=1,\textit{lr}$) and such that \dusa{imix} is arbitrary (not used) if radial region $l$ does not intersect Cartesian region $(j,k,i)$;
\item $l=\textit{lr+1}$ for the mixture outside the annular regions but inside Cartesian region $(j,k,i)$;
\item from surface \moc{Y-} to surface \moc{Y+} ($j=1,\textit{ly}$ for each $k$ and $i$);
\item from surface \moc{Z-} to surface \moc{Z+} ($k=1,\textit{lz}$ for each $i$);
\item from surface \moc{X-} to surface \moc{X+} ($i=1,\textit{lx}$).
\end{enumerate}

\item \moc{CARCELY} geometry.

$$N_{t}=(\textit{lr}+1)\times\textit{lz}\times \textit{lx}\times \textit{ly}$$

The mixtures are then given in the following order
\begin{enumerate}
\item radially outward ($l=1,\textit{lr}$) and such that \dusa{imix} is arbitrary (not used) if radial region $l$ does not intersect Cartesian region $(k,i,j)$;
\item $l=\textit{lr+1}$ for the mixture outside the annular regions but inside Cartesian region $(k,i,j)$;
\item from surface \moc{Z-} to surface \moc{Z+} ($k=1,\textit{lz}$ for each $i$ and $j$);
\item from surface \moc{X-} to surface \moc{X+} ($i=1,\textit{lx}$ for each $j$);
\item from surface \moc{Y-} to surface \moc{Y+} ($j=1,\textit{ly}$).
\end{enumerate}

\item \moc{CARCELZ} geometries.

$$N_{t}=(\textit{lr}+1)\times\textit{lx}\times \textit{ly}\times \textit{lz}$$

The mixtures are then given in the following order
\begin{enumerate}
\item radially outward ($l=1,\textit{lr}$) and such that \dusa{imix} is arbitrary (not used) if radial region $l$ does not intersect Cartesian region $(i,j,k)$;
\item $l=\textit{lr+1}$ for the mixture outside the annular regions but inside Cartesian region $(i,j,k)$;
\item from surface \moc{X-} to surface \moc{X+} ($i=1,\textit{lx}$ for each $j$ and $k$);
\item from surface \moc{Y-} to surface \moc{Y+} ($j=1,\textit{ly}$ for each $k$).
\item from surface \moc{Z-} to surface \moc{Z+} ($k=1,\textit{lz}$).
\end{enumerate}

\item \moc{HEX} geometry.

$$N_{t}=\textit{lh}$$
The mixtures or cells are then given in the order provided in \Figto{s30}{compl}.

\item \moc{HEXT} geometry.

Three options are possible here:
\begin{itemize}
\item All the triangles in an hexagonal crown have the same mixture. In this case
\begin{align*}N_{t}&=\textit{nhr}\end{align*}
and the real and virtual mixtures are given from each crown starting at the center of the cell.

\item All the triangles in an hexagonal crown in a given sector have the same mixture. In this case
\begin{align*}N_{t}&=6\times \textit{nhr}\end{align*}
and the real and virtual mixtures are given in the following order 
\begin{enumerate}
\item from each crown in sector $j$ starting from the center of the cell;
\item for each sector $j=1,6$.
\end{enumerate}

\item All the triangles contain a different mixture. In this case
\begin{align*}N_{t}&=6\times \textit{nhr}^{2}\end{align*}
and the real and virtual mixtures are given in the following order 
\begin{enumerate}
\item from each triangle $l$ ($l=1,2\times \textit{nhc}-1$) in hexagonal crown $i$ of sector $j$. \Fig{GeoHEXT4} illustrates region and surface ordering in the case where the default value of \dusa{hexmsh} is used and \Fig{GeoHEXT4C} the same information when a different value of \dusa{hexmsh} is provided.
\item from each crown in sector $j$ starting from the center of the cell;
\item for each sector $j=1,6$.
\end{enumerate}
\end{itemize}

\item \moc{HEXCEL} geometries.

$$N_{t}=(\textit{lr}+1)$$

The mixtures are then given in the following order
\begin{enumerate}
\item radially outward ($l=1,\textit{lr}$);
\item $l=\textit{lr+1}$ for the mixture outside the annular regions but inside the hexagonal region.
\end{enumerate}

\item \moc{HEXZ} geometry.

$$N_{t}=\textit{lh}\times \textit{lz}$$

The mixtures or cells are then given in the following order

\begin{enumerate}
\item according to \Figto{s30}{compl} for plane $k$;
\item from surface \moc{Z-} to surface \moc{Z+} ($k=1,\textit{lz}$).
\end{enumerate}

\item \moc{HEXTCEL} geometries.

Three options are possible here:
\begin{itemize}
\item All the triangles in an hexagonal crown have the same mixture. In this case
\begin{align*}N_{t}&=(\textit{lr}+1)\times \textit{nhr}\end{align*}
and the real and virtual mixtures are given in the following order
\begin{enumerate}
\item radially outward ($l=1,\textit{lr}+1$) for each crown ($l=\textit{lr}+1$ is for the part of crown outside the annular regions);
\item from each crown starting from the center of the cell.
\end{enumerate}

\item All the triangles in an hexagonal crown in a given sector have the same mixture. In this case
\begin{align*}N_{t}&=6\times (\textit{lr}+1)\times \textit{nhr}\end{align*}
and the real and virtual mixtures are given in the following order 
\begin{enumerate}
\item radially outward ($l=1,\textit{lr}+1$) for each crown of each sector ($l=\textit{lr}+1$ is for the part of crown outside the annular regions);
\item from each crown in sector $j$ starting from the center of the cell;
\item for each sector $j=1,6$.
\end{enumerate}

\item All the triangles contain a different mixture. In this case
\begin{align*}N_{t}&=6\times (\textit{lr}+1)\times \textit{nhr}^{2}\end{align*}
and the real and virtual mixtures are given in the following order 
\begin{enumerate}
\item radially outward ($l=1,\textit{lr}+1$) for each triangle ($l=\textit{lr}+1$ is for the part of triangle outside the annular regions);
\item from each triangle $l$ ($l=1,2\times \textit{nhc}-1$) in hexagonal crown $i$ of sector $j$. \Fig{GeoHEXT4} illustrates region and surface ordering in the case where the default value of \dusa{hexmsh} is used and \Fig{GeoHEXT4C} the same information when a different value of \dusa{hexmsh} is provided.
\item from each crown in sector $j$ starting from the center of the cell;
\item for each sector $j=1,6$.
\end{enumerate}
\end{itemize}

\item \moc{HEXTZ} geometry.

Three options are again possible here:
\begin{itemize}
\item All the triangles in an hexagonal crown in a plane have the same mixture. In this case
\begin{align*}N_{t}&=\textit{nhr}\times  \textit{lz}\end{align*}
and the real and virtual mixtures are given in the following order
\begin{enumerate}
\item from each crown starting from the center of the cell;
\item from lowest (\moc{Z-}) to highest (\moc{Z+}) plane ($k=1,\textit{lz}$).
\end{enumerate}

\item All the triangles in an hexagonal crown in a given sector in a plane have the same mixture. In this case
\begin{align*}N_{t}&=6\times \textit{nhr}\times \textit{lz}\end{align*}
and the real and virtual mixtures are given in the following order 
\begin{enumerate}
\item from each crown in sector $j$ starting from the center of the cell;
\item for each sector $j=1,6$;
\item from lowest (\moc{Z-}) to highest (\moc{Z+}) plane ($k=1,\textit{lz}$).
\end{enumerate}

\item All the triangles contain a different mixture. In this case
\begin{align*}N_{t}&=6\times \textit{nhr}^{2}\times \textit{lz}\end{align*}
and the real and virtual mixtures are given in the following order 
\begin{enumerate}
\item from each triangle $l$ ($l=1,2\times \textit{nhc}-1$) in hexagonal crown $i$ of sector $j$. \Fig{GeoHEXT4} illustrates region and surface ordering in the case where the default value of \dusa{hexmsh} is used and \Fig{GeoHEXT4C} the same information when a different value of \dusa{hexmsh} is provided.
\item from each crown in sector $j$ starting from the center of the cell;
\item for each sector $j=1,6$;
\item from lowest (\moc{Z-}) to highest (\moc{Z+}) plane ($k=1,\textit{lz}$).
\end{enumerate}
\end{itemize}


\item \moc{HEXCELZ} geometries.

$$N_{t}=(\textit{lr}+1)\times \textit{lz}$$

\item \moc{HEXTCELZ} geometries.

Three options are possible here:
\begin{itemize}
\item All the triangles in an hexagonal crown have the same mixture. In this case
\begin{align*}N_{t}&=(\textit{lr}+1)\times \textit{nhr}\times \textit{lz}\end{align*}
and the real and virtual mixtures are given in the following order
\begin{enumerate}
\item radially outward ($l=1,\textit{lr}+1$) for each crown ($l=\textit{lr}+1$ is for the part of crown outside the annular regions);
\item from each crown starting from the center of the cell;
\item from lowest (\moc{Z-}) to highest (\moc{Z+}) plane ($k=1,\textit{lz}$).
\end{enumerate}

\item All the triangles in an hexagonal crown in a given sector have the same mixture. In this case
\begin{align*}N_{t}&=6\times (\textit{lr}+1)\times \textit{nhr}\times \textit{lz}\end{align*}
and the real and virtual mixtures are given in the following order 
\begin{enumerate}
\item radially outward ($l=1,\textit{lr}+1$) for each crown of each sector ($l=\textit{lr}+1$ is for the part of crown outside the annular regions);
\item from each crown in sector $j$ starting from the center of the cell;
\item for each sector $j=1,6$;
\item from lowest (\moc{Z-}) to highest (\moc{Z+}) plane ($k=1,\textit{lz}$).
\end{enumerate}

\item All the triangles contain a different mixture. In this case
\begin{align*}N_{t}&=6\times (\textit{lr}+1)\times \textit{nhr}^{2}\times \textit{lz}\end{align*}
and the real and virtual mixtures are given in the following order 
\begin{enumerate}
\item radially outward ($l=1,\textit{lr}+1$) for each triangle ($l=\textit{lr}+1$ is for the part of triangle outside the annular regions);
\item from each triangle $l$ ($l=1,2\times \textit{nhc}-1$) in hexagonal crown $i$ of sector $j$. \Fig{GeoHEXT4} illustrates region and surface ordering in the case where the default value of \dusa{hexmsh} is used and \Fig{GeoHEXT4C} the same information when a different value of \dusa{hexmsh} is provided.
\item from each crown in sector $j$ starting from the center of the cell;
\item for each sector $j=1,6$.
\item from lowest (\moc{Z-}) to highest (\moc{Z+}) plane ($k=1,\textit{lz}$).
\end{enumerate}
\end{itemize}

\end{itemize}

The mixtures are then given in the following order
\begin{enumerate}
\item radially outward ($l=1,\textit{lr}$) for plane $k$;
\item $l=\textit{lr+1}$ for the mixture outside the annular regions but inside the hexagonal region on plane $k$;
\item from surface \moc{Z-} to surface \moc{Z+} ($k=1,\textit{lz}$).
\end{enumerate}

\begin{figure}[h!]
\begin{center} 
\epsfxsize=8cm
\centerline{ \epsffile{Goricart.eps}}
\parbox{14cm}{\caption{Description of the various rotations allowed for
Cartesian geometries}\label{fig:oricart}}   
\end{center}  
\end{figure}

\begin{figure}[h!]  
\begin{center} 
\epsfxsize=11cm
\centerline{ \epsffile{Gorihex.eps}}
\parbox{14cm}{\caption{Description of the various rotation allowed for
hexagonal geometries}\label{fig:orihex}}
\end{center}  
\end{figure}

\begin{figure}[h!]  
\begin{center} 
\epsfxsize=7cm
\centerline{ \epsffile{Gcluster.eps}}
\parbox{14cm}{\caption{Typical cluster geometry}\label{fig:cluster}}
\end{center}  
\end{figure}

\clearpage

The inputs associated with this structure have the following meaning:

\begin{ListeDeDescription}{mmmmmm}

\item[\moc{MIX}] keyword to specify the isotopic mixture number or
sub-geometry associated
with each region inside the geometry. When diagonal symmetries are considered,
only the mixture associated with regions inside the symmetrized geometry need to
be specified. When a sub-geometry is located inside symmetrized geometry but
outside the calculation region it must be declared {\sl virtual} (for example,
the corners of a nuclear reactor). 

\item[\dusa{imix}] array of $n_{t}\le N_t$ integers {\sl or} character variables associated
with each region. An integer is a mixture number associated with a region
\dusa{imix}$\le$\dusa{maxmix} (see \Sectand{MACData}{LIBData}). If
\dusa{imix}=0, the corresponding volume is replaced by a void region. If
\dusa{imix} is a character variable, it is replaced by the corresponding
sub-geometry or {\sl generating cell}. These values must be specified in
the following order for most geometries: 

\begin{enumerate}
\item radially from the inside out. 
\item from surface \moc{X-} to surface \moc{X+}
\item from surface \moc{Y-} to surface \moc{Y+}
\item from surface \moc{Z-} to surface \moc{Z+}
\end{enumerate}

In the cases where a \moc{CARCELX} and a \moc{TUBEX} geometry are defined then we will use 

\begin{enumerate}
\item radially from the inside out ($lr+1$ mixtures for \moc{CARCELX} and $lr$ for \moc{TUBEX}). 
\item from surface \moc{Y-} to surface \moc{Y+}
\item from surface \moc{Z-} to surface \moc{Z+}
\item from surface \moc{X-} to surface \moc{X+}
\end{enumerate}

Finally, for a \moc{CARCELY} and \moc{TUBEY} geometry are defined the following order is considered:

\begin{enumerate}
\item radially from the inside out ($lr+1$ mixtures for \moc{CARCELY} and $lr$ for \moc{TUBEY}) 
\item from surface \moc{Z-} to surface \moc{Z+}
\item from surface \moc{X-} to surface \moc{X+}
\item from surface \moc{Y-} to surface \moc{Y+}
\end{enumerate}

In the cases where a sectorized cell geometry is defined, \dusa{imix} must
be defined in each sector, following the order shown in \Figand{rect3}{hexa3}.
Also note that \dusa{imix} is {\sl not affected} by the values of the
mesh-splitting indices \dusa{ispltx}, \dusa{isplty}, \dusa{ispltz}
or \dusa{ispltr}.

\item[\moc{REPEAT}] keyword to specify the previous list of mixtures will be repeated. This is valid only when $N_t/n_t$ 
is an integer. If this keyword is absent and $n_t < N_t$, then the missing mixtures will be replaced 
with void (\dusa{imix}(i) $=0$).

\item[\moc{PLANE}] keyword to attribute mixture numbers to each volume inside a single 2-D plane. This option is 
valid only for 3-D geometries, Cartesian or hexagonal. 

\item[\dusa{iplan}] plane number for which material mixture are input. 

\item[\moc{SAME}] keyword to attribute the same material mixture numbers of the \dusa{iplan1} plane to the \dusa{iplan} plane. In 
hexagonal geometry, it can indicate that the mixture numbers of the current crown of the \dusa{iplan}th 
plane will be identical to those of the same crown of the \dusa{iplan1}th plane. 

\item[\dusa{iplan1}] plane number used as reference to input the current plane or crown(s). 

\item[\dusa{lp}] number of volumes in a plane. In Cartesian geometry, $lp=lx*ly$ and in hexagonal geometry, 
$lp=lh$. 

\item[\moc{CROWN}]  keyword to attribute mixture numbers to each hexagon of a single crown. This option is only 
valid for \moc{COMPLETE} hexagonal geometry definition. Each use of the keyword \moc{CROWN} increases 
the crown number by 1. So it is not required to give its number, but crowns must be defined from 
the center to the peripherical regions of a plane. 

\item[\dusa{lc}] number of hexagons in the current crown. For the \dusa{i}th crown of a compelete hexagonal plane, 
$lc=(i-1)*6$. The first crown is composed of only one hexagon. 

\item[\moc{ALL}] keyword to specify that the \dusa{lc} material mixture number of the current crown have the same value 
\dusa{jmix}. 

\item[\moc{UPTO}] keyword to attribute material mixture numbers of the current crown up to the \dusa{ic} one. 

\item[\dusa{ic}] number of the last crown in \moc{UPTO} option. Its value must be greater than equal to the current 
crown number.

\item[\moc{HMIX}] keyword to specify the virtual isotopic mixture associated with each region inside the geometry. These
virtual mixtures will be produced by homogenization in the {\tt EDI:} module (see \Sect{descedi}).

\item[\moc{CELL}] keyword to specify the location of the sub-geometry called
{\sl generating cells} in a Cartesian or hexagonal geometry.

\item[\dusa{HCELL}] array of sub-geometry {\tt character*12} names which will
be superimposed upon the current Cartesian geometry. The same sub-geometry may
appear in different positions within the global geometry if the material
properties and dimensions are identical. The concept of sub-geometry is useful
for the interface current method in a SYBIL calculation since the collision
probability matrix associated with each sub-geometry is computed independently
of its location in the geometry. In general, the neutron fluxes in identical
sub-geometry located at different locations will be different even if they are
associated with the same collision probability matrix. These sub-geometry names
must be specified in the following order: 

\begin{enumerate}
\item from surface \moc{X-} to surface \moc{X+}
\item from surface \moc{Y-} to surface \moc{Y+}
\item from surface \moc{Z-} to surface \moc{Z+}
\end{enumerate}

\item[\moc{MERGE}] keyword to specify that some sub-geometries or regions must
be merged. 

\item[\dusa{imerge}] array of numbers that associate a global sub-geometry or
region number with each sub-geometry or region. All the sub-geometries  or
regions with the same global number will be attributed the same flux.

\item[\moc{TURN}] keyword to specify that some sub-geometries must be rotated
in space before being located at a specific position.

\item[\dusa{HTURN}] array of  {\tt character*1} keywords to rotate
conveniently each sub-geometry. The letters {\tt A} to {\tt L} are used as
keywords to specify these rotation. For Cartesian geometries, the eight possible
orientations are shown in \Fig{oricart} while for hexagonal geometries
the permitted orientations are shown in \Fig{orihex}. For 3-D cells, the
same letters can be used to describe the rotation in the $X-Y$ plane. However,
an additional $-$ sign can be glued to the 2-D rotation identifier to
indicate reflection of the cell along the $Z$-axis ({\tt -A} to {\tt -L}).

\item[\moc{CLUSTER}] keyword to specify that pin (cylindrical) sub-geometry
will be inserted in the geometry (see \Fig{cluster}). 

\item[\dusa{NAMPIN}] array of cylindrical sub-geometry {\tt character*12} name 
representing a pin. This sub-geometry must be of type \moc{TUBE}, \moc{TUBEX},
\moc{TUBEY} or \moc{TUBEZ}.

\item[\moc{MIX-NAMES}] keyword to specify character names to material mixtures.
By default, the material mixtures are not named.

\item[\dusa{NAMMIX}] array of {\tt character*12} names for the material
mixtures.

\end{ListeDeDescription}

\clearpage

\subsubsection{Double-heterogeneity}\label{sect:descDH}

The structure \dstr{descDH} provides the possibility to define a stochastic mixture of cylindrical or spherical micro-structures that can be distributed inside {\sl composite mixtures} of the current {\sl macro-geometry}. A composite mixture is represented by a {\sl material mixture index} with a value greater than \dusa{maxmix}, the maximum number of real mixtures. Each micro-structure can be composed of many micro-volumes.\cite{BIHET}

\begin{DataStructure}{Structure \dstr{descDH}}
$[$ \moc{BIHET}  $\{$ \moc{TUBE} $|$ \moc{SPHE} $\}$ \dusa{nmistr}
\dusa{nmilg} \\
\hskip 1.0cm (\dusa{ns}(i),i=1,\dusa{nmistr}) \\
\hskip 1.0cm((\dusa{rs}(i,j),j=1,\dusa{ns}(i)+1),i=1,\dusa{nmistr})\\
\hskip 1.0cm(\dusa{milie}(i),i=1,\dusa{nmilg})\\
\hskip 1.0cm(\dusa{mixdil}(i),i=1,\dusa{nmilg})\\
\hskip 1.0cm( (\dusa{fract}(i,j),j=1,\dusa{nmistr}) 
( $[$(\dusa{mixgr}(i,j,k),k=1,\dusa{ns}(j))$]$,j=1,\dusa{nmistr}), i=1,\dusa{nmilg}) $]$
\end{DataStructure}

\noindent where
\begin{ListeDeDescription}{mmmmmmm}

\item[\moc{BIHET}] keyword to specify that the current macro-geometry is containing composite mixtures.

\item[\moc{TUBE}] keyword to specify that the micro-structures are of a
cylindrical geometry;

\item[\moc{SPHE}] keyword to specify that the micro-structures are of a
spherical geometry.

\item[\dusa{nmistr}] maximum number of micro-structure types in the composite mixtures. Each type of
micro-structure is characterized by its dimension and may have distinct
volumetric concentrations in each of the macro-geometry volumes. All the
micro-structures of a given type have the same nuclear properties in a given
macro-volume. The micro-structures of a given type may have different nuclear
properties within different macro-volumes.

\item[\dusa{nmilg}] number of composite mixtures. This is the number of material mixture indices of the macro-geometry with a value $>$\dusa{maxmix}.

\item[\dusa{ns}] array giving the number of sub-regions (tubes or spherical
shells) in the  micro-structures. Each type of micro-structures may contain a
different number of micro-volumes.

\item[\dusa{rs}] array giving the radius of the tubes or spherical shells
making up the micro-structures. For each type of micro structure $i$, we will
have an initial radius of \dusa{rs}$(1,i)=0.0$.

\item[\dusa{milie}] array giving the indices used to defined composite mixtures in the macro-geometry. These composite mixture indices must be $>$\dusa{maxmix}.

\item[\dusa{mixdil}] array giving the mixture indices associated with the diluent in each composite mixtures of the macro-geometry.  These values must be $\le$\dusa{maxmix}.

\item[\dusa{fract}] array of volumetric concentration ($V_{G}/V_{R}$) of
each micro-structures (volume $V_{G}$) in a given region (volume $V_{R}$) of the
macro-geometry.  

\item[\dusa{mixgr}] array giving the mixture index associated with each
region of the micro-structures. Note that \dusa{mixgr} should be specified only
for the regions of the micro-structure which have a concentration
\dusa{fract}$>$0. These values must be $\le$\dusa{maxmix}.

\end{ListeDeDescription}

Examples of geometry definitions can be found in \Sect{ExGEOData}.

\subsubsection{Do-it-yourself geometries}\label{sect:descSIJ}

A {\sl do-it-yourself} geometry is an abstract representation of an assembly of arbitrary unit-cells defined in term of their probability of presence and of their probability to have a particular neighbor. Structure \dstr{descSIJ} is defined as

\begin{DataStructure}{Structure \dstr{descSIJ}}
$[$ \moc{POURCE} (\dusa{pcinl}(i),i=1,\dusa{lp}) $]$\\
$[$ \moc{PROCEL} ((\dusa{pijcel}(i,j),j=1,\dusa{lp}),i=1,\dusa{lp}) $]$
\end{DataStructure}

\noindent where
\begin{ListeDeDescription}{mmmmmmm}

\item[\moc{POURCE}] keyword to specify that a {\sl do-it-yourself} type
geometry is to be defined, that is to say a geometry resembling the multicell
geometry seen in APOLLO-1.\cite{apollo1} This option permits the interactions
between different arbitrarily arranged cells in an infinite lattice to be
treated. The cells are identified by the information
following the keyword \moc{CELL}. The user must ensure that the total number of
regions appearing in all the cells must be less than \dusa{maxreg}.

\item[\dusa{pcinl}] array giving the proportion of each cell type in the
lattice such that:

$$|\sum_{i=1}^{{\it lp}}{\it pcinl}(i)-1.|<10^{-5}$$

\item[\moc{PROCEL}] keyword to specify that in a {\sl do-it-yourself} type
geometry rather than using a statistical arrangement of cells, a pre-calculated
cell distribution is to be considered. If the \moc{POURCE} structure is
given without the \moc{PROCEL} structure, a {\sl statistical} approximation
is used, as defined in Ref.~\citen{apollo1}.

\item[\dusa{pijcel}] array giving the pre-calculated probability for a neutron
leaving a cell of type i to enter a cell of type j without crossing any other
cell. We require: 

$$|S(i){\it pcinl}(i){\it pijcel}(i,j)-S(j){\it pcinl}(j) {\it
pijcel}(j,i)|<10^{-4}$$

\noindent where $S(i)$ and $S(j)$ are the exterior surfaces area of the cells of
type $i$ and $j$ respectively.

\end{ListeDeDescription}

Examples of geometry definitions can be found in \Sect{ExGEOData}.

\eject
