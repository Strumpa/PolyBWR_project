\subsection{The {\tt INFO:} module}\label{sect:INFOData}

The \moc{INFO:} module is mainly used to compute the number densities for
selected isotopes at specific local conditions. The module can also be used to
compute the density $\rho(T,p,x)$ for a mixture containing a fraction $x$ of heavy and $(1-x)$ of light water according at a temperature $T$ and pressure $p$:
  $$
\rho(T,p,x) = {{\ \rho_{H_2O}(T,p)\ \rho_{D_2O}(T,p)}
\over{ x\ \rho_{H_2O}(T,p) +  (1-x) \ \rho_{D_2O}(T,p)}}\ .
  $$
where $\rho_{H_2O}(T,p)$ and $\rho_{D_2O}(T,p)$ will take different forms depending on the option selected.\cite{Kieffer}

\vskip 0.2cm

The calling specifications are:

\begin{DataStructure}{Structure \dstr{INFO:}}
\moc{INFO:} \moc{::} \dstr{descinfo}
\end{DataStructure}

\goodbreak
\noindent where
\begin{ListeDeDescription}{mmmmmmmm}

\item[\dstr{descinfo}] structure containing the input data to this module
(see \Sect{descinfo}).

\end{ListeDeDescription}

\vskip 0.2cm

\subsubsection{Data input for module {\tt INFO:}}\label{sect:descinfo}

\begin{DataStructure}{Structure \dstr{info}}
$[$ \moc{EDIT} \dusa{iprint} $]$ \\
$[$ \moc{LIB:} $\{$ \moc{DRAGON} $|$ \moc{MATXS}  $|$ \moc{MATXS2} $|$
                    \moc{WIMSD4} $|$ \moc{WIMSAECL} $|$ \moc{NDAS} $|$    
                    \moc{APLIB2} $|$ \moc{APLIB1} $\}$ \\
~~~~~~~\moc{FIL:} \dusa{NAMEFIL} $]$ \\
$[$ \moc{TMP:} \dusa{temp} $\{$ \moc{K} $|$ \moc{C} $\}$ $]$ \\
$[$ \moc{PUR:} \dusa{purity} $\{$ \moc{WGT\%} $|$ \moc{ATM\%} $\}$ $]$ \\
$[$ \moc{PRES:} \dusa{pressure} $\{$ \moc{bar} $|$ \moc{Pa} $|$ \moc{kPa} $|$ \moc{MPa} $\}$ $]$ \\
$[$ \moc{CALC} \moc{DENS} $\{$ \moc{WATER} $>>$\dusa{dens}$<<$ $|$ \moc{PWATER} $>>$\dusa{dens}$<<$ $\}$ $]$ \\
$[$ \moc{ENR:} \dusa{enrichment} $\{$ \moc{WGT\%} $|$ \moc{ATM\%} $\}$ $]$ \\
$[[$ \moc{ISO:} \dusa{nbiso} (\dusa{ISONAM}($i$), $i$=1,nbiso)  \\
$\ \ $  $\{$ \moc{GET}   \moc{MASS} ($>>$\dusa{mass}($i$)$<<$, $i$=1,nbiso) $|$ 
           \moc{CALC}  \moc{WGT\%}  $\{$  \\
\hskip 1.5cm \moc{D2O} $>>$\dusa{nh1}$<<$ $>>$\dusa{hd2}$<<$ $>>$\dusa{no16}$<<$
$|$\\
\hskip 1.5cm \moc{H2O} $>>$\dusa{nh1}$<<$ $>>$\dusa{hd2}$<<$ $>>$\dusa{no16}$<<$
$|$\\
\hskip 1.5cm \moc{UO2} $>>$\dusa{nu5}$<<$ $>>$\dusa{hu8}$<<$ $>>$\dusa{no16}$<<$
$|$\\
\hskip 1.5cm \moc{THO2} $>>$\dusa{nth2}$<<$ $>>$\dusa{nu3}$<<$ $>>$\dusa{no16}$<<$
$\}$ $\}$ 
$]]$
\end{DataStructure}

\noindent
where

\begin{ListeDeDescription}{mmmmmmmm}

\item[\moc{EDIT}] keyword used to modify the print level \dusa{iprint}.

\item[\dusa{iprint}] index used to control the printing of the module. The
amount of output produced by this tracking module will vary substantially
depending on the print level specified.

\item[\moc{LIB:}] keyword to specify the type of library from which the
isotopic mass ratio is to be read. 

\item[\moc{DRAGON}] keyword to specify that the isotopic depletion chain or
the microscopic cross sections are in the DRAGLIB format.

\item[\moc{MATXS}] keyword to specify that the microscopic cross sections are
in the MATXS format of NJOY-II and NJOY-89 (no depletion data available for
libraries using this format).

\item[\moc{MATXS2}] keyword to specify that the microscopic cross sections are
in the MATXS format of NJOY-91  (no depletion data available for libraries using
this format).

\item[\moc{WIMSD4}] keyword to specify that the isotopic depletion chain and the
microscopic cross sections are in the WIMSD4 format.

\item[\moc{WIMSAECL}] keyword to specify that the isotopic depletion chain and the
microscopic cross sections are in the WIMS-AECL format.

\item[\moc{NDAS}] keyword to specify that the isotopic depletion chain and the
microscopic cross sections are in the NDAS format, as used in recent versions of WIMS-AECL.

\item[\moc{APLIB1}] keyword to specify that the microscopic cross sections are
in the APOLLO-1 format.

\item[\moc{APLIB2}] keyword to specify that the microscopic cross sections are
in the APOLLO-2 format.

\item[\moc{FIL:}] keyword to specify the name of the file where is  stored the mass
ratio data. 

\item[\dusa{NAMEFIL}] \verb|character*8| name of the library where the mass ratio
are stored.

\item[\moc{TMP:}] keyword to specify the isotopic temperature.

\item[\dusa{temp}] temperature $T$ in \moc{K} or \moc{C}.

\item[\moc{PUR:}] keyword to specify the water purity, that is fraction of heavy
water in a mix of heavy and light water.

\item[\dusa{purity}] percent weight (\moc{WGT\%}) or atomic (\moc{ATM\%}) fraction of heavy
water in a mix of heavy and light water ($100\times x$).

\item[\moc{PRES:}] keyword to specify the pressure.

\item[\dusa{pressure}] pressure $p$ in \moc{bar}, \moc{Pa}, \moc{kPa} or \moc{MPa}.

\item[\moc{ENR:}] keyword to specify the fuel enrichment.

\item[\dusa{enrichment}] fuel enrichment in weight percent (\moc{WGT\%}) or atomic
percent (\moc{ATM\%}).

\item[\moc{ISO:}] keyword to specify an isotope list. This list will be used either
for getting mass values of isotopes or for computing number  densities.

\item[\dusa{nbiso}] number of isotopic names used for a calculation (limited to
\dusa{nbiso}$\leq 3$).

\item[\dusa{ISONAM}] \verb|character*12| name of an isotope.

\item[\moc{GET MASS}] keyword to recover the mass values as written in the library.
It returns the mass value of each isotope in the output parameter \dusa{mass}. 

\item[\moc{CALC}] keyword to ask the module to compute some parametric values. It
returns one value in the output parameter \dusa{dens}. 

\item[\moc{DENS}] compute density of a mixture of light and heavy water.

\item[\moc{WATER}] keyword to recover the water density as a
function of its temperature and purity (independent of pressure $p$). This option requires the setting of
temperature and purity, and it does not affect any given list of isotope names. This module relies on the water density calculator of WIMS-AECL.\cite{WIMS}

\item[\moc{PWATER}] new keyword to recover the water density as a
function of temperature, pressure and purity developed by C. Kieffer.\cite{Kieffer} This option requires the setting of
temperature, pressure and purity, and it does not affect any given list of isotope names. For light water, it uses the \moc{freesteam} routines.\cite{Freesteam} For heavy water, two options are considered.
\begin{enumerate}
\item For $90\text{ C}<T<350\text{ C}$ and $p<22$ MPa, the heavy water routines written by Ji Zhang at AECL and distributed freely by B. Garland from McMaster University are considered.\cite{McMaster}
\item Otherwise, the density is that of obtained from \moc{freesteam} multiplied by a factor of 1.11 which is approximately the ratio of the molecular mass of D2O to H2O.
\end{enumerate}  

\item[\moc{WGT\%} \moc{D2O}] keywords to recover 3 number densities for a compound
mixture of heavy and light water. The isotope list is assumed to contain $^{1}$H,
$^{2}$D and $^{2}$O. Temperature and purity are supposed to be available. It returns
concentration of these isotopes in the output parameters \dusa{nh1}, \dusa{nd2} and
\dusa{no16}.

\item[\moc{WGT\%} \moc{H2O}] is identical to \moc{WGT\%} \moc{D2O}.

\item[\moc{WGT\%} \moc{UO2}] keywords to recover 3 number densities for a compound
mixture of Uranium oxide. The isotope list is assumed to contain $^{235}$U,
$^{238}$U and $^{16}$O. The $^{235}$U enrichment is supposed to be available. Note
that the number densities will sum to 100. It returns concentration of these
isotopes in the output parameters \dusa{nu5}, \dusa{nu8} and \dusa{no16}.

\item[\moc{WGT\%} \moc{THO2}] keywords to recover 3 number densities for a compound
mixture of Thorium/Uranium oxide. The isotope list is assumed to contain
$^{232}$Th,  $^{233}$U  and $^{16}$O. The $^{233}$U enrichment is supposed to be
available. Note that the number densities will sum to 100. It returns concentration
of these isotopes in the output parameters \dusa{nth2}, \dusa{nu3} and \dusa{no16}.

\end{ListeDeDescription}

The \moc{INFO:} module works the following way. For a given isotope list, the mass is
extracted from the library or a calculation process is expected. Once this
calculation is has been performed, it is possible to list other isotopes and ask for
further calculations. Finally note that the number of output parameters, denoted by
$>>$\dusa{param}$<<$, are recovered as CLE-2000 variables in \dstr{descinfo}. The number
of these parameters must be equal to the number of isotopes
names given, plus the water density when a command \moc{CALC} \moc{DENS} \moc{WATER}
is issued.

\eject
