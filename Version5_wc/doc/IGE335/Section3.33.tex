\subsection{The {\tt HEAT:} module}\label{sect:HEATData}

This module is used to compute the energy and charge deposition values from many particles.

\vskip 0.02cm

The calling specifications are:

\begin{DataStructure}{Structure \dstr{HEAT:}}
\dusa{DEPOS}~\moc{:=}~\moc{HEAT:}~$[$~\dusa{DEPOS}~$]~[[$~\dusa{MACRO}~$]]$ \moc{::} \dstr{HEAT\_data} \\
\end{DataStructure}

\noindent where
\begin{ListeDeDescription}{mmmmmmm}

\item[\dusa{DEPOS}] {\tt character*12} name of a {\sc deposition} (type {\tt L\_DEPOSITION}) object containing mixture-ordered energy and charge deposition values, summed over many extended mactolibs. This object can be created by module {\tt HEAT:} or used in modification mode to
accumulate deposition values gathered from successive solutions of the Boltzmann and/or Boltzmann Fokker-Planck transport equations.

\item[\dusa{MACRO}] {\tt character*12} name of an extended {\sc macrolib} (type {\tt L\_MACROLIB}) object containing {\tt FLUX-INTG} and {\tt H-FACTOR} values.
{\tt C-FACTOR} values are also recovered if they are available. There are as many macrolibs on the RHS as particles contributing to the energy and charge deposition.

\item[\dusa{HEAT\_data}] input data structure containing specific data (see \Sect{descHEAT}).

\end{ListeDeDescription}

\subsubsection{Data input for module {\tt HEAT:}}\label{sect:descHEAT}

\vskip -0.5cm

\begin{DataStructure}{Structure \dstr{HEAT\_data}}
$[$~\moc{EDIT} \dusa{iprint}~$]$ \\
$[~\{$ \moc{POWR} \dusa{power} $|$ \moc{SOUR} \dusa{snumb} $|$ \moc{NORM} \dusa{rho}($i$), $i$=1,\dusa{nbmix} $\}~]$ \\
$[~\{$ \moc{BC} $|$ \moc{NBC} $\}~]$ \\
$[~\{$ \moc{PICKE}  {\tt >>} \dusa{esum} {\tt <<} $|$ \moc{PICKC}  {\tt >>} \dusa{csum} {\tt <<} $\}~]$ \\
;
\end{DataStructure}

\noindent where
\begin{ListeDeDescription}{mmmmmm}

\item[\moc{EDIT}] keyword used to set \dusa{iprint}.

\item[\dusa{iprint}] index used to control the printing in module {\tt HEAT:}. =0 for no print; =1 for minimum printing (default value).

\item[\moc{POWR}] keyword used to set \dusa{power}.

\item[\dusa{power}] value of the power in MW used to normalize the flux. By default, the flux is not normalized.

\item[\moc{SOUR}] keyword used to set \dusa{snumb}. Fixed source information (record {\tt FIXE}) must be available in the first extended macrolib \dusa{MACRO}.

\item[\dusa{snumb}] number of source particles used to normalize the flux. By default, the flux is not normalized.

\item[\moc{NORM}] keyword used to obtain the energy deposition per voxels by normalizing the flux by the total number of source particles and the voxel material density. The output units are MeV/g$\times$cm$^{N}$ per particles, where $N$ is the geometry dimension.

\item[\dusa{rho}] densities in g/cm$^{3}$ of the \dusa{nbmix} material defined in calculations.

\item[\moc{BC}]  keyword used to take into account the contribution of the particules falling below the cutoff energy to the charge and energy deposition calculations using method from Morel\cite{morel1996} (by default).

\item[\moc{NBC}] keyword used to disable the contribution of the particules falling below the cutoff energy to the charge and energy deposition calculations.

\item[\moc{PICKE}]  keyword used to recover the total energy deposition value (MeV/cm$^{3}$/s) in a CLE-2000 variable.

\item[\dusa{esum}] \texttt{character*12} CLE-2000 variable name in which the extracted total energy deposition value will be placed.

\item[\moc{PICKC}]  keyword used to recover the total charge deposition value (electrons/cm$^{3}$/s) in a CLE-2000 variable.

\item[\dusa{csum}] \texttt{character*12} CLE-2000 variable name in which the total charge deposition value will be placed.

\end{ListeDeDescription}
\clearpage
