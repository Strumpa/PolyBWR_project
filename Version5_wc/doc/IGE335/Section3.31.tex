\subsection{The {\tt FMAC:} module}\label{sect:FMACData}

This module is used to extract macrocoscopic cross-section data from a FMAC-M {\sc ascii} file.
Transition source information from companion particles are recovered from the FMAC-M file and written in the
output {\sc macrolib}.

\vskip 0.02cm

The calling specifications are:

\begin{DataStructure}{Structure \dstr{FMAC:}}
\dusa{MACRO}~\moc{:=}~\moc{FMAC:}~\dusa{fmac.txt}~\moc{::}~\dstr{FMAC\_data} \\
\end{DataStructure}

\noindent where
\begin{ListeDeDescription}{mmmmmmm}

\item[\dusa{MACRO}] {\tt character*12} name of the output {\sc macrolib} (type {\tt L\_MACROLIB}) object that is created by {\tt FMAC:}.

\item[\dusa{fmac.txt}] {\tt character*12} name of a {\sc ascii} file, open in read-only mode, containing FMAC-M information.

\item[\dusa{FMAC\_data}] input data structure containing specific data (see \Sect{descFMAC}).

\end{ListeDeDescription}

\subsubsection{Data input for module {\tt FMAC:}}\label{sect:descFMAC}

\vskip -0.5cm

\begin{DataStructure}{Structure \dstr{FMAC\_data}}
$[$~\moc{EDIT} \dusa{iprint}~$]$ \\
\moc{PARTICLE} \dusa{htype} \\
\moc{;}
\end{DataStructure}

\noindent where
\begin{ListeDeDescription}{mmmmmmmm}

\item[\moc{EDIT}] keyword used to set \dusa{iprint}.

\item[\dusa{iprint}] index used to control the printing in module {\tt FMAC:}. =0 for no print; =1 for minimum printing (default value).

\item[\moc{PARTICLE}] keyword used to specify the type of particle corresponding to the {\sc macrolib} (secondary state of the transition).

\item[\dusa{htype}] character*1 name of the particle. Usual names are {\tt N}: neutrons, {\tt G}: photons, {\tt B}: electrons,
{\tt C}: positrons and {\tt P}: protons.

\end{ListeDeDescription}

\eject
