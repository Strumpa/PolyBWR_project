\subsection{The {\tt LIB:} module}\label{sect:LIBData}

The general format of the input data for the \moc{LIB:} module is the following:

\vspace{-0.2cm}

\begin{DataStructure}{Structure \dstr{LIB:}}
\dusa{MICLIB} \moc{:=} \moc{LIB:} $[$ \dusa{MICLIB} $]~[~\{$ \dusa{MICRHS} $|$ \dusa{MACRHS} $|$ \dusa{EVORHS} $\}~]$ 
\moc{::} \dstr{desclib}
\end{DataStructure}

\vspace{-0.6cm}

\noindent
where

\begin{ListeDeDescription}{mmmmmmmm}

\item[\dusa{MICLIB}] {\tt character*12} name of the \dds{microlib} that will contain the internal
library. If \dusa{MICLIB} appears on both LHS and RHS, it is updated; otherwise, it is created. 

\item[\dusa{MICRHS}] {\tt character*12} name of a read-only \dds{microlib} data structure used by the
\moc{CATL} or \moc{MAXS} option of \Sect{desclib}.

\item[\dusa{MACRHS}] {\tt character*12} name of a read-only \dds{macrolib} data structure to be included
directly in \dusa{MICLIB} before updating it.

\item[\dusa{EVORHS}] {\tt character*12} name of a read-only \dds{burnup} data structure used by the
\moc{BURN} option of \Sect{desclib}. The number densities for the isotopes in file \dusa{MICLIB}
will be replaced selectively by those found in \dusa{EVORHS}.

\item[\dstr{desclib}] input structure for this module (see \Sect{desclib}).

\end{ListeDeDescription}

\subsubsection{Data input for module {\tt LIB:}}\label{sect:desclib}

In the case where \dusa{MICRHS} is absent or represents a \dds{macrolib}, \dstr{desclib} takes the form:

\begin{DataStructure}{Structure \dstr{desclib}}
$[$ \moc{EDIT} \dusa{iprint} $]$ \\
$[$ \moc{NGRO} \dusa{ngroup} $]$ \\
$[$ \moc{MXIS} \dusa{nmisot} $]$ \\
$[$ \moc{NMIX} \dusa{nmixt} $]$ \\
$[$ \moc{CALENDF} \dusa{ipreci} $]$ \\
$[$ \moc{CTRA} $\{$ \moc{NONE} $|$ \moc{APOL} $|$ \moc{WIMS} $|$ \moc{OLDW} $|$ \moc{LEAK} $\}$ $]$
$[$ \moc{ANIS}  \dusa{naniso} $]$ \\
$[$ \moc{STERN} \dusa{nstern} $]$ \\
$[$ \moc{ADJ} $]~[$ \moc{PROM} $]$ \\ 
$[~\{$ \moc{CDEPCHN} $|$ \moc{RDEPCHN} $\}~]$ \\
$[~\{$ \moc{SKIP} $|$ \moc{INTR} $|$ \moc{SUBG} $|$ \moc{PT} $|$ \moc{PTMC} $|$ \moc{PTSL} $|$ \moc{RSE} $[$ \dusa{svdeps} $]~|$ \moc{NEWL} $\}~]$ $[$
\moc{MACR} $]$\\ 
$[$ \moc{ADED} \dusa{nedit}  ( \dusa{HEDIT}(i), i=1,\dusa{nedit} ) $]$  \\
$[$ \moc{DEPL} $\{$ \moc{LIB:} $\{$ \moc{DRAGON} $|$ \moc{WIMSD4} $|$ \moc{WIMSE} $|$ \moc{WIMSAECL} $|$ \moc{NDAS} $|$ \moc{APLIB3} $\}$ \moc{FIL:} \dusa{NAMEFIL} \\
\hskip 0.6cm $|$ \moc{LIB:} $\{$ \moc{APLIB2} $|$ \moc{APXSM} $\}$ \moc{FIL:} \dusa{NAMEFIL} \dstr{descdeplA2} \\
\hskip 0.6cm $|$ \dusa{ndepl} \dstr{descdepl} $\}$ $]$ \\
$[[$ \moc{MIXS} \moc{LIB:} \\
\hskip 0.6cm  $\{$ \moc{DRAGON} $|$ \moc{MATXS} $|$ \moc{MATXS2} $|$
                  \moc{WIMSD4} $|$ \moc{WIMSE} $|$ \moc{WIMSAECL} $|$ \moc{NDAS} $|$    
                  \moc{APLIB1} $|$ \moc{APLIB2} \\
\hskip 0.85cm  $|$ \moc{APXSM} $|$ \moc{APLIB3} $|$ \moc{MICROLIB} $\}$ \\
\hskip 0.6cm  \moc{FIL:} \dusa{NAMEFIL} $[[$ \dstr{descmix1} $]]$ $]]$ \\
{\tt ;}
\end{DataStructure}

\noindent It is possible to reset an existing  \dds{microlib} (i.e., \dusa{MICLIB} is present
in both the LHS and RHS) and to reprocess all the isotopes from the cross section libraries.
In this case, \dstr{desclib} takes the simplified form:

\begin{DataStructure}{Structure \dstr{desclib}}
$[$ \moc{EDIT} \dusa{iprint} $]$ \\
$\{$ \moc{INTR} $|$ \moc{SUBG} $|$ \moc{PT} $|$ \moc{PTMC} $|$ \moc{PTSL} $|$ \moc{RSE} $[$ \dusa{svdeps} $]~|$ \moc{NEWL} $\}~[$ \moc{MACR} $]$ \\ 
\moc{MIXS} \\
{\tt ;}
\end{DataStructure}

\noindent
If keyword \moc{CATL} is given, \dusa{MICLIB} is catenated with the RHS \dusa{LIBRHS} \dds{microlib} .

\begin{DataStructure}{Structure \dstr{desclib}}
$[$ \moc{EDIT} \dusa{iprint} $]$ \\
$[$ \moc{MXIS} \dusa{nmisot} $]$ \\
$[$ \moc{NMIX} \dusa{nmixt} $]$ \\
$[~\{$ \moc{SKIP} $|$ \moc{MACR} $\}~]$
$[~\{$ \moc{CDEPCHN} $|$ \moc{RDEPCHN} $\}~]$ \\
$[$ \moc{DEPL} $\{$ \moc{LIB:} $\{$ \moc{DRAGON} $|$ \moc{WIMSD4} $|$ \moc{WIMSE} $|$ \moc{WIMSAECL} $|$ \moc{NDAS} $|$ \moc{APLIB3} $\}$ \moc{FIL:} \dusa{NAMEFIL} \\
\hskip 0.6cm $|$ \moc{LIB:} $\{$ \moc{APLIB2} $|$ \moc{APXSM} $\}$ \moc{FIL:} \dusa{NAMEFIL} \dstr{descdeplA2} \\
\hskip 0.6cm $|$ \dusa{ndepl} \dstr{descdepl} $\}$ $]$ \\
\moc{CATL} $[[$ \dstr{descmix2} $]]$ \\
{\tt ;}
\end{DataStructure}

\noindent
Alternatively if keyword \moc{BURN} or \moc{MAXS} is given, \dstr{desclib} takes the form:

\begin{DataStructure}{Structure \dstr{desclib}}
$[$ \moc{EDIT} \dusa{iprint} $]$ \\
$\{$ \moc{BURN} $\{$ \dusa{iburn} $|$ \dusa{tburn} $\}~|$ \moc{MAXS} $\}$
$[[$ \dstr{descmix2} $]]$ \\
{\tt ;}
\end{DataStructure}
\noindent where the RHS data structure is a \dds{burnup} (\dusa{EVORHS}) or a \dds{microlib} (\dusa{LIBRHS}) data structure. \dstr{desclib} options are:

\begin{ListeDeDescription}{mmmmmm}

\item[\moc{EDIT}] keyword used to modify the print level \dusa{iprint}.

\item[\dusa{iprint}] index used to control the printing in this operator. It
must be set to 0 if no printing on the output file is required while values
$>$0 will increase in steps the amount of information transferred to the output
file. If \dusa{iprint}$\ge$10, the depletion chain is printed in the format of
structure \dstr{descdepl}. If \dusa{iprint}$\ge$20, the depletion chain is also
printed in the format of structure \dstr{descdeplA2}.

\item[\moc{MXIS}] keyword used to redefine the maximum number of isotopes per
mixture.  

\item[\dusa{nmisot}] the maximum number of isotopes per
mixture. By default up to 300 different isotopes per mixture are permitted.

\item[\moc{NMIX}] keyword used to define the number of material mixtures. This
data is required if \dusa{MICLIB} is created.

\item[\dusa{nmixt}] the maximum number of mixtures (a mixture
is characterized by a distinct set of macroscopic cross sections). 

\item[\moc{CALENDF}] keyword to set the accuracy of the CALENDF probability
tables.

\item[\dusa{ipreci}] integer set to 1, 2, 3 or 4. The highest the value, the
more accurate are the probability tables. The default value is \dusa{ipreci}=4.

\item[\moc{CTRA}] keyword to specify the type of transport correction that
should be generated and stored on the \dds{microlib}. The transport correction is to be
substracted from the total and isotropic ($P_0$) within-group scattering cross sections. A leakage correction, equal 
to the difference between current-- and flux--weighted total cross sections ($\sigma_{1}-\sigma_{0}$)
is also applied in the \moc{APOL}, \moc{OLDW} and \moc{LEAK} cases. All the operators that
will read this \dds{microlib} will then have access to transport corrected
cross sections. The default is no transport correction.

\item[\moc{NONE}] keyword to specify that no transport correction should be
used in this calculation.

\item[\moc{APOL}] keyword to specify that an APOLLO type transport correction
based on the linearly anisotropic ($P_1$) within-group scattering cross sections is to be set. This correction assumes that
the micro-reversibility principle is valid for all energy groups. This type of
correction uses $P_1$ scattering information present on the library.

\item[\moc{WIMS}] This type of correction uses directly a transport-correction
provided on the library.
Such information is available in WIMSD4, WIMSE and WIMS--AECL libraries. This is
the new recommended option with WIMS-type libraries. {\sl This option has no effect on
libraries that does not contain transport correction information.}

\item[\moc{OLDW}] keyword to specify that a WIMS type transport
correction based on the $P_1$ scattering cross sections is to be
set. This correction
assumes that the micro-reversibility principle is valid only for groups energies
less than 4.0 eV. For the remaining groups a $1/E$ current spectrum is considered
in the evaluation of the transport correction. This type of correction uses
$P_1$ scattering information present on the library.

\item[\moc{LEAK}] A leakage correction is applied to the total and
$P_0$ within-group scattering cross sections. No transport correction is 
applied in this case.

\item[\moc{ANIS}] keyword to specify the maximum level of anisotropy for the
scattering cross sections.

\item[\dusa{naniso}] number of Legendre orders for the representation of the
scattering cross sections. Isotropic scattering is represented by
\dusa{naniso}=1 while  \dusa{naniso}=2 represents linearly anisotropic
scattering. Generally the linearly anisotropic ($P_1$) scattering contributions are
taken into account via the transport correction (see \moc{CTRA} keyword) in the
transport calculation. For $B_{1}$ or $P_{1}$ leakage calculations, the linearly
anisotropic scattering cross sections are taken into account explicitly.  The
default value is \dusa{naniso}=2.

\item[\moc{STERN}] keyword to specify the application of the Sternheimer density correction for charged particles.

\item[\dusa{nstern}] index used to control the Sternheimer correction application. Sternheimer correction applied for both restricted total stopping power
and heat deposition cross section ({\tt H-FACTOR}) is represented by \dusa{nstern} $=1$. A complete desactivation of the Sternheimer correction is obtained
by setting \dusa{nstern} $=0$. By default, the Sternheimer density correction is applied for both quantities. Notes: 1) The Sternheimer density correction should be
applied for both quantities except for specific charged particles cross sections perturbations analysis; 2) The Sternheimer density correction should be
applied on macroscopic cross sections. However, the heat deposition cross section contains a microscopic collisional stopping power which has not been
corrected in ELECTR module of NJOY. This is why the charged particle {\tt H-FACTOR} data $-$ recovered from microscopic libraries produced by ELECTR, but not
those produced by CEPXS-BFP $-$ should be corrected in DRAGON5.

\item[\moc{ADJ}] keyword to specify the production of adjoint macroscopic
cross sections. By default, direct cross sections are produced.

\item[\moc{PROM}] keyword to specify that prompt neutrons are to be considered
for the calculation of the fission spectrum. By default, the contribution due to
delayed neutrons is considered. This option is only compatible with a
\moc{MATXS} or \moc{MATXS2} format library.

\item[\moc{CDEPCHN}] keyword to enable the automatic completion of burnup chains.

\item[\moc{DDEPCHN}] keyword to avoid the automatic completion of burnup chains.

\item[\moc{SKIP}] keyword to recover the user--defined microlib data without processing
any library (i.e., without temperature and/or dilution interpolation).

\item[\moc{INTR}] keyword to perform a temperature and dilution interpolation
of the microscopic cross sections present in the libraries. The bin-type
cross-section data is not processed. This is the default option.

\item[\moc{SUBG}] keyword to activate the calculation of the physical probability
tables using the tempera\-tu\-re-interpolated cross-section data as
input.\cite{subg,nse2004} The bin-type cross-section data is not processed.

\item[\moc{PT}] keyword to activate the calculation of the CALENDF-type
mathematical probability tables ({\sl without} slowing-down correlated weight matrices)
using the bin-type cross-section data as input.\cite{pt} This option is
compatible with the Sanchez-Coste self-shielding method and with the subgroup projection method (SPM).\cite{SPM09}

\item[\moc{PTMC}] this option is similar to the \moc{PT} procedure. Here, the base points of the probability tables corresponding
to fission and scattering cross sections and to components of the transfer scattering matrix are also obtained using the CALENDF approach.

\item[\moc{PTSL}] keyword to activate the calculation of the CALENDF-type
mathematical probability tables and slowing-down correlated weight matrices
using the bin-type cross-section data as input.\cite{nse2004}

\item[\moc{RSE}] keyword to activate the generation of information for the resonance spectrum expansion (RSE) method.\cite{rse2021}

\item[\dusa{svdeps}] rank accuracy $\epsilon_{\rm svd}$ of the singular value decomposition. Singular values $w_i \le \epsilon_{\rm svd}\Delta u_{\rm elem}$ are set to zero.
$\Delta u_{\rm elem}$ is the elementary lethargy width of the Autolib. The default value is \dusa{svdeps}=1.0 $\times 10^{-3}$.

\item[\moc{NEWL}] keyword to activate the calculation of a microlib
containing temperature-interpo\-la\-ted cross-section data. The bin-type
cross-section data is also interpolated. Probability tables are not computed.

\item[\moc{MACR}] keyword to force the calculation of the embedded
macrolib. By default, the embedded macrolib is computed, {\sl except if} one of the
key words \moc{SKIP}, \moc{INTR}, \moc{SUBG}, \moc{PT} or \moc{NEWL} is used.

\item[\moc{ADED}] keyword to specify the input of additional cross sections to
be treated by DRAGON. These cross sections are not needed to solve the transport
equation but are recognized by the \moc{EDI:} and utility operators.

\item[\dusa{nedit}] number of types of additional cross sections.

\item[\dusa{HEDIT}] {\tt character*6} name of an additional
cross-section type. This name also corresponds to vectorial reactions in a
\moc{MATXS} and
\moc{MATXS2} format library. For example:

\moc{NWT0}/\moc{NWT1}=$P_0/P_1$ library weight functions.\\
\moc{NTOT0}/\moc{NTOT1}=$P_0/P_1$ neutron total cross sections.\\
\moc{NELAS}=Neutron elastic scattering cross sections (MT=2).\\
\moc{NINEL}=Neutron inelastic scattering cross sections (MT=4).\\
\moc{NG}=Neutron radiative capture cross sections (MT=102).\\
\moc{NFTOT}=Total fission cross sections (MT=18).\\
\moc{NUDEL}=Number of delayed secondary neutrons (Nu-D / MT=455).\\
\moc{NFSLO}=$\nu*$slow fission cross section.\\
\moc{NHEAT}=Heat production cross section.\\
\moc{CHIS}/\moc{CHID}=Slow/delayed fission spectrum.\\
\moc{NF}/\moc{NNF}/\moc{N2NF}/\moc{N3NF}=$\nu*$partial fission cross sections (MT=19, 20, 21 and 38).\\ 
\moc{N2N}/\moc{N3N}/\moc{N4N}=(n,2n), (n,3n), (n,4n) cross sections (MT=16, 17 and 37).\\
\moc{NP}/\moc{NA}=(n,p) and (n,$\alpha$) transmutation cross sections (MT=103 and 107).

By default, DRAGON will always attempt to recover the additional cross sections
\moc{NG}, \moc{NFTOT}, \moc{NHEAT} and \moc{N2N} which are required for the depletion
calculations. 

\item[\moc{DEPL}] keyword to specify that the isotopic depletion (burnup)
chain is to be read. For a given \moc{LIB:} execution only one isotopic
depletion chain can be read. 

\item[\moc{MIXS}] keyword to specify that the mixture description is to be
read. For a given \moc{LIB:} execution more than one cross-section library can
be read. 

\item[\moc{LIB:}] keyword to specify the type of library from which the
isotopic depletion chain or microscopic cross section is to be read. It is
optional when preceded by the keyword \moc{DEPL} in which case the isotopic
depletion chain is read from the standard input file. 

\item[\moc{DRAGON}] keyword to specify that the isotopic depletion chain or
the microscopic cross sections are in the {\sc draglib} format.

\item[\moc{MATXS}] keyword to specify that the microscopic cross sections are
in the MATXS format of NJOY-II and NJOY-89 (no depletion data available for
libraries using this format).

\item[\moc{MATXS2}] keyword to specify that the microscopic cross sections are
in the MATXS format of NJOY-91 (no depletion data available for libraries using
this format). The MATXS file is a binary sequential file by default. If the name
\dusa{NAMEFIL} has a leading ``{\tt \_}'' character, the MATXS file is expected to be
BCD-formatted, as produced by NJOY.

\item[\moc{WIMSD4}] keyword to specify that the isotopic depletion chain and the
microscopic cross sections are in the WIMSD4 format, as produced by module {\tt wimsr} of NJOY with flag
{\tt iverw} $=4$. This format is supported by the WLUP project.\cite{wlup}

\item[\moc{WIMSE}] keyword to specify that the isotopic depletion chain and the
microscopic cross sections are in the WIMSE format, as produced by module {\tt wimsr} of NJOY with flag
{\tt iverw} $=5$.

\item[\moc{WIMSAECL}] keyword to specify that the isotopic depletion chain and the
microscopic cross sections are in the WIMS-AECL format.

\item[\moc{NDAS}] keyword to specify that the isotopic depletion chain and the
microscopic cross sections are in the NDAS format, as used in recent versions of WIMS-AECL.

\item[\moc{APLIB1}] keyword to specify that the microscopic cross sections are
in the APOLLO-1 format. There are no depletion chains available for libraries using this
format.

\item[\moc{APLIB2}] keyword to specify that the microscopic cross sections are
in the APOLLO-2 direct access format. There are no depletion chains available for libraries
using this format. However, fission yields, radioactive decay constants and
energy released per fission or radiative capture are recovered from the file.
Only versions of the APOLIB-2 libraries subsequent or equal to CEA93-V4 can be
processed. The list of isotopes (standard and self-shielded) available in an APOLIB-2
is printed by setting the print flag to a value \dusa{iprint}$\ge$10.

\item[\moc{APXSM}] keyword to specify that the microscopic cross sections are
in the APOLIB-XSM format, the output format of N2A2 utility. There are no depletion chains available for libraries
using this format. However, fission yields, radioactive decay constants and
energy released per fission or radiative capture are recovered from the file.
The list of isotopes (standard and self-shielded) available in an APOLIB-XSM
is printed by setting the print flag to a value \dusa{iprint}$\ge$10.

\item[\moc{APLIB3}] keyword to specify that the microscopic cross sections are
in the APOLIB-3 format, the output format of the Galilee system. An ENDF/B evaluation is
represented by three HDF5 files:
\begin{description}
\item[\dusa{NAME1}:] HDF5 file containing infinite dilution information
\item[\dusa{NAME2}:] HDF5 file containing resonance self-shielding information
\item[\dusa{NAME3}:] HDF5 file containing depletion chains, branching ratio, fission yields and energy deposition information.
\end{description}
After \moc{DEPL}, the \moc{FIL:} keyword is followed by the concatenation of \dusa{NAME1} and \dusa{NAME3} with a colon character ({\tt :}) between
the two names. After \moc{MIXS}, the \moc{FIL:} keyword is followed by the concatenation of \dusa{NAME1} and \dusa{NAME2} with a colon character ({\tt :}) between
the two names. The list of isotopes (standard and self-shielded) available in an APOLIB-3
is printed by setting the print flag to a value \dusa{iprint}$\ge$10.

\item[\moc{MICROLIB}] keyword to specify that the microscopic cross sections are
in a {\sc microlib}-formatted object, as produced by DRAGON. This format is similar to the {\sc draglib}
format where the isotopes are stored in elements of list {\tt ISOTOPESLIST} instead of been stored
as independent sub-directories.

\item[\moc{FIL:}] keyword to specify the name of the file where is stored the
isotopic depletion data. 

\item[\dusa{NAMEFIL}] {\tt character*64} name of the library
where the isotopic depletion chain or the microscopic cross sections are stored.

Library names in {\sc draglib} format are limited to 12 characters.

An \moc{APLIB3} library name is the concatenation of two names with a colon character ({\tt :}) between them:
\begin{verbatim}
  DEPL LIB: APLIB3 FIL: CLA99CEA93:CLA99CEA93_EVO
  MIXS LIB: APLIB3 FIL: CLA99CEA93:CLA99CEA93_SS
\end{verbatim}

A \moc{NDAS} library is made of two or more files. These file names must be concatenated in a single
\dusa{NAMEFIL} name, using colons as separators. The {\sc ascii} index file is always the first,
followed by optional patch files, and terminated by the main direct-access binary file. The
following sample data line corresponds to a {\sc ndas} library without patch:
\begin{verbatim}
  MIXS LIB: NDAS FIL: E65LIB6.idx:E65LIB6.sdb
\end{verbatim}

\item[\dusa{ndepl}] number of isotopes in the depleting chain.

\item[\dstr{descdepl}] input structure describing the
depletion chain (see \Sect{descdepl}).

\item[\dstr{descdeplA2}] simplified input structure describing the
depletion chain in cases where an APOLIB-2 or APOLIB-XSM file is used (see \Sect{descdepl}).

\item[\moc{CATL}] keyword to perform the following operations:
\vspace{-0.15cm}
\begin{itemize}
\item create a new microlib or recover an existing \dds{microlib} in modification mode,
\item catenate with a RHS \dds{microlib} in read-only mode,
\item create the embedded  \dds{macrolib}.
\end{itemize}

\item[\moc{MAXS}] keyword to specify that the mixture density on \dusa{MICLIB}
are to be modified. If \dusa{MICRHS} is present and \dstr{descmix2} is absent, a
direct one to one correspondence between the isotope on both libraries is
assumed. If \dusa{MICRHS} and \dstr{descmix2} are present, only the
mixture on the library file specified by \dstr{descmix2} are updated using
information from the \dusa{MICRHS}. If \dusa{MICRHS} is absent and
\dstr{descmix2} is present, only the mixture on  \dusa{MICLIB} specified by
\dstr{descmix2} are updated. This option is useful for implementing two-level
computational schemes similar to REL-2005.

\item[\moc{BURN}] keyword to specify that the mixture density on \dusa{MICLIB}
are to be updated using information taken from \dusa{EVORHS}. If \dstr{descmix2}
is absent, a direct one to one correspondence between the isotope on
\dusa{EVORHS} and \dusa{MICLIB}  is assumed. If  \dstr{descmix2} is present, only
the mixture specified by \dstr{descmix2} are updated using information from
\dusa{EVORHS}. This option is useful for performing branching calculations.

\item[\dusa{iburn}] burnup step from the burnup file to use. This step must be
already present on the burnup file.

\item[\dusa{tburn}] burnup time in days from the burnup file to use. This time
step must be already present on the burnup file.

\item[\dstr{descmix1}] input structure describing the
isotopic and physical properties of a given mixture (see \Sect{descmix}).

\item[\dstr{descmix2}] input structure describing perturbations to the
isotopic and physical properties of a given mixture (see \Sect{descmix}).


\end{ListeDeDescription}

Note that it is possible to recompute the embedded macrolib in an existing microlib
named {\tt MICRO} by writing
\begin{verbatim}
MICRO := LIB: MICRO :: MACR MIXS ;
\end{verbatim}

\subsubsection{Depletion data structure}\label{sect:descdepl}

The structure \dstr{descdepl} describes the heredity of the radioactive decay
and the neutron activation chain to be used in the isotopic depletion
calculation.
\begin{DataStructure}{Structure \dstr{descdepl}}
\moc{CHAIN} \\
$[[$ \dusa{NAMDPL} $[$ \dusa{izae} $]$ \\
\hskip 1.0cm $[[~\{$ \moc{DECAY} \dusa{dcr} $|$ \\
\hskip 2.0cm \dusa{reaction} $[$ \dusa{energy} $]~\}~]]$ \\
\hskip 1.0cm $[~\{$ \moc{STABLE} $|$ \\
\hskip 2.0cm \moc{FROM} $[[~\{$ \moc{DECAY} $|$ \dusa{reaction} $\}$
$[[$ \dusa{yield} \dusa{NAMPAR} $]]~]]~\}~]~]]$\\
\moc{ENDCHAIN}
\end{DataStructure}

\vspace{-0.15cm}

\noindent
with:

\begin{ListeDeDescription}{mmmmmm}

\item[\moc{CHAIN}] keyword to specify the beginning of the depletion chain.

\item[\dusa{NAMDPL}] {\tt character*12} name of an isotope (or isomer) of the
depletion chain that appears in the cross-section library.

\item[\dusa{izae}] optional six digit integer representing the isotope. The first two
digits represent the atomic number of the isotope; the next three indicate its
mass number and the last digit indicates the  excitation level of the nucleus (0
for a nucleus in its ground state, 1 for an isomer in its first exited state,
etc.). For example, $^{238}$U in its ground state will be represented by
\dusa{izae}=922380.

\item[\moc{DECAY}] indicates that a decay reaction takes place either for
production of this isotope or its depletion.

\item[\dusa{dcr}] radioactive decay constant (in $10^{-8}$ s$^{-1}$) of the
isotope. By default, \dusa{dcr}=0.0.

\item[\dusa{reaction}] {\tt character*6} identification of a neutron-induced
reaction that takes place either for production of this isotope, its depletion,
or for producing energy. Example of reactions are following:

\begin{ListeDeDescription}{mmmmmmmm}
\item[\moc{NG}] indicates that a radiative capture reaction takes place either
for production of this isotope, its depletion or for producing energy.

\item[\moc{N2N}] indicates that the following reaction is taking place:
$$ n +^{A}X_Z \to 2 n + ^{A-1}X_Z$$

\item[\moc{N3N}] indicates that the following reaction is taking place:
$$ n +^{A}X_Z \to 3 n + ^{A-2}X_Z$$

\item[\moc{N4N}] indicates that the following reaction is taking place:
$$ n +^{A}X_Z \to 4 n + ^{A-3}X_Z$$

\item[\moc{NP}] indicates that the following reaction is taking place:
$$ n +^{A}X_Z \to p + ^AY_{Z-1}$$

\item[\moc{NA}] indicates that the following reaction is taking place:
$$ n +^{A}X_Z \to ^4{\rm He}_2 + ^{A-3}X_{Z-2}$$

\item[\moc{NFTOT}] indicates that a fission is taking place.
\end{ListeDeDescription}

\item[\dusa{energy}] energy (in MeV) recoverable per neutron-induced
reaction of type \dusa{reaction}. If the energy associated to radiative capture
is not explicitely given, it should be added to the energy released per fission. By
default, \dusa{energy}=0.0 MeV.

\item[\moc{STABLE}] non depleting isotope. Such an isotope may produces
energy by neutron-induced reactions (such as radiative capture).

\item[\moc{FROM}] indicates that this isotope is produced from decay or
neutron-induced reactions.

\item[\dusa{yield}] branching ratio or production yield expressed in fraction.

\item[\dusa{NAMPAR}] {\tt character*12} name of the a parent isotope
(or isomer) that appears in the cross-section library.

\item[\moc{ENDCHAIN}] keyword to specify the end of the depletion chain.

\end{ListeDeDescription}

\vskip 0.15cm

If the keyword \moc{APLIB2} or \moc{APXSM} was used in structure \dstr{desclib}, part of the
depletion data is recovered from the APOLIB file: the fission yields, the
radioactive decay constants and the energy released per fission or radiative
capture. Moreover, the following simplified structure is used to provide the
remaining depletion data:

\begin{DataStructure}{Structure \dstr{descdeplA2}}
\moc{CHAIN} \\
$[[$ \dusa{NAMDPL} $[$ \moc{FROM} $[[$ $\{$ \moc{DECAY} $|$ \dusa{reaction} $\}$
\dusa{yield} \dusa{NAMPAR} $]]$ $]$ $]]$\\
\moc{ENDCHAIN}
\end{DataStructure}

\vskip 0.15cm

In this case, the following rules apply:
\begin{itemize}
\item We should provide the names \dusa{NAMDPL} of {\sl all} the depleting
isotopes (i.e. isotopes with a time-dependent number density), including the
pseudo fission products (PFP).
\item The fission father reactions (\moc{NFTOT}) are not given.
\item The stable isotopes are automatically recovered from the
APOLIB file. They are not given in structure \dstr{descdeplA2}.
\item An isotope is considered to be stable if it is not present in
structure \dstr{descdeplA2}, has no father and no daughter,
but can release energy by fission or radiative capture.
\item It is possible to truncate the isotope name \dusa{NAMDPL} at the
underscore. For example, {\tt D2O\_3\_P5} can be simply written {\tt D2O}.
\item Only the radioactive decay constants of the isotopes present in
structure \dstr{descdeplA2} are recovered from the APOLIB file. The
radioactive decay constants of the other isotopes are set to zero.
\end{itemize}

\subsubsection{Mixture description structure}\label{sect:descmix}

The structure \dstr{descmix1} is used to describe the isotopic composition and
the physical properties, such as the temperature and density, of a mixture.

\begin{DataStructure}{Structure \dstr{descmix1}}
\moc{MIX} $[$ \dusa{matnum} $]$ $\{$ \\
\hskip 1.0cm $[$\dusa{temp} $[$ \dusa{denmix} $]~]~~[~\{$ \moc{NOEV} $|$ \moc{EVOL} $\}~]~~[~\{$ \moc{NOGAS}
    $|$ \moc{GAS}$\}~]$\\
\hskip 2.0cm $[[~[$ \dusa{NAMALI} \moc{=} $]$ \dusa{NAMISO} \dusa{dens} $[~\{$ \dusa{dil} 
    $|$ \moc{INF} $\}~]$\\
\hskip 2.0cm $[~[$ \moc{CORR} $]$ \dusa{inrs} $]~[$ \moc{DBYE} \dusa{tempd} $]~[$ \moc{SHIB} \dusa{NAMS} $]$ \\
\hskip 2.0cm $[$ \moc{THER} \dusa{ntfg} \dusa{HINC} $[$ \moc{TCOH} \dusa{HCOH} $]~[$ \moc{RESK} $]~]$ \\
\hskip 2.0cm $[$ \moc{IRSET} $\{$ \dusa{gir} $|~\{$ \moc{PT} $|$ \moc{PTMC} $|$ \moc{PTSL}$\}~\}~\{$
\dusa{nir} $|$ \moc{NONE} $\}~]~~[~\{$ \moc{NOEV} $|$ \moc{EVOL} $|$ \moc{SAT} $\}~]~]]$ \\
\hskip 1.0cm $|$ \\
\hskip 1.0cm \moc{COMB} $[[$ \dusa{mati} \dusa{relvol} $]]~\}$
\end{DataStructure}

\vspace{-0.15cm}

\noindent
where:

\begin{ListeDeDescription}{mmmmmm}

\item[\moc{MIX}] keyword to specify the number identifying the next mixture to
be read.

\item[\dusa{matnum}] mixture identifier. The maximum value that \dusa{matnum}
may have is \dusa{nmixt}. When \dusa{matnum} is absent, the mixtures are
numbered successively starting from 1 if no mixture has yet been specified or
from the last mixture number specified + 1.

\item[\dusa{temp}] absolute temperature (in Kelvin) of the isotopic mixture.
It is optional only when this mixture is to be updated, in which case the old
temperature associated with the mixture is used.

\item[\dusa{denmix}] mixture density in $g \ cm^{-3}$. 

\item[\dusa{NAMALI}] {\tt character*8} alias name for an isotope to be used
locally. When the alias name is absent, the isotope name used locally is
identical to the first 8-character isotope name on the library.

\item[\moc{=}] keyword to specify to which isotope in a library is associated
the previous alias name.

\item[\dusa{NAMISO}] {\tt character*12} name of an isotope present in the
library which is included in this mixture.

\item[\dusa{dens}]  isotopic concentration of the isotope \dusa{NAMISO} in the
current mixture in $10^{24}cm^{-3}$.  When the mixture density  \dusa{denmix}
is specified, the relative weight percentage of each of the isotopes in this
mixture is to be provided.

\item[\dusa{dil}] group independent microscopic dilution cross section (in
barns) of the isotope \dusa{NAMISO} in this mixture. It is possible to
recalculate a group dependent dilution for an isotope by the use of the
\moc{SHI:} or \moc{TONE:} operator (see \Sect{SHIData} and \Sect{TONEData}). In this case, the dilution is only used
as a starting point for the self-shielding iterations and has no effect on the
final result. If the dilution is not given or is larger than $10^{10}$ barns,
an infinite dilution is assumed.

\item[\moc{INF}] keyword to specify that a dilution of $10^{10}$ barns is to
be associated with this isotope. This value represents an infinite dilution (the
isotope is present in trace amounts only). It is possible to
recalculate a group dependent dilution for an isotope by the use of the
\moc{SHI:} operator (see \Sect{SHIData}) or \moc{TONE:} operator (see \Sect{TONEData}). In this case, the dilution is only used
as a starting point for the self-shielding iterations and has no effect on the
final result. If the dilution is not given an infinite dilution is assumed.

\item[\moc{CORR}] keyword to specify that the resonances of an isotope are correlated
with those of other isotopes with the same \dusa{inrs} index. This option is only
available with the {\sl Ribon extended} model\cite{nse2004} or wth the {\sl subgroup
projection method} (SPM)\cite{SPM09}  in energy groups where
this model is set. If this option is selected for
an isotope, it must be set for all isotopes with the same \dusa{inrs} index. By default,
the resonances of distinct isotopes are assumed to be uncorrelated.

\item[\dusa{inrs}] index of the resonant region associated with this isotope.
By default \dusa{inrs}=0 and the isotope is not a candidate for self-shielding.
When \dusa{inrs}$\ne$0, the isotope can be self-shielded where it is assumed that a given
isotope distributed with different concentrations in a number of mixtures and
having the same value of \dusa{inrs} will share the same fine flux. 
Should we wish to self-shield both the clad and the fuel it is important
to assign a different \dusa{inrs} number
to each. If a single type of fuel is located in different mixture in
{\sl onion-peel fashion}, it is necessary to attribute a single \dusa{inrs} value
to this fuel.

\item[\moc{DBYE}] keyword to specify that the absolute temperature of the
isotope is different from that of the isotopic mixture. This option is useful to
define Debye-corrected temperature.

\item[\dusa{tempd}] absolute temperature (in Kelvin) of the isotope. By
default \dusa{tempd}=\dusa{temp}.

\item[\moc{SHIB}] keyword to specify that the name of the isotope containing
the information related to the self-shielding is different from the initial name
of the isotope. This option is not required if a MATXS or a {\sc draglib} file is used.

\item[\dusa{NAMS}] {\tt character*12} name of a record in the library
containing the self-shielding data. This name is required if the dilution is
not infinite or a non zero resonant region is associated with this isotope and \dusa{NAMS}
is different from \dusa{NAMISO}. This record must be contained in the same
library file as record \dusa{NAMISO}.

\item[\moc{THER}] keyword to specify that the thermalization and resonant elastic
scattering kernel effects are to be included with the cross sections when using a
\moc{MATXS} or \moc{MATXS2} format library.

\item[\dusa{HINC}] {\tt character*6} name  of the incoherent thermalization
effects which will be taken into account. The incoherent effects are those that
may be described by the $S(\alpha,\beta)$ scattering law. The value \moc{FREE}
is used to simulate the effects of a gas.

\item[\moc{TCOH}]  keyword to specify that coherent thermalization effects
will be taken into account.

\item[\dusa{HCOH}] {\tt character*6} name of the coherent thermalization
effects which will be taken into account. The coherent effects are the
{\sl vectorial reactions} in the \moc{MATXS} or \moc{MATXS2} format library where
the name is terminated by the `\$' suffix. They are generally available for
graphite, beryllium, beryllium oxide, polyethylene and zirconium hydroxide.

\item[\moc{RESK}]  keyword to specify that resonant elastic scattering kernel effects
will be taken into account.

\item[\dusa{ntfg}]  number of energy groups that will be affected by the
thermalization and resonant elastic scattering kernel effects.

\item[\moc{IRSET}] keyword to specify an intermediate resonance (IR)
approximation or the {\sl Ribon extended} model for some energy groups. By default, an
IR approximation with the value of the Goldstein-Cohen parameter found on the library
is used. If no value is found on the library, a statistical (ST) model\cite{st} is set in
all groups by default. The ``{\tt IRSET PT 1}'' option is set by default if keyword \moc{PT}, \moc{PTMC} or
\moc{PTSL} is selected in structure \dstr{desclib}.

\item[\dusa{gir}]  imposed Goldstein-Cohen IR parameter. A Goldstein-Cohen IR parameter
$0 \le \lambda_g\le 1$ is set in energy group $g$. A value of 1.0 stands for
a statistical (ST) approximation. A value of 0.0 stands for an infinite mass
(IM or WR) approximation.

\item[\moc{PT}] keyword to enable the calculation of CALENDF--type probability tables in some energy groups. The
slowing-down correlated weight matrices are {\sl not} computed. This type of probability tables is consistent
with the Sanchez-Coste self-shielding method and with the subgroup projection method (SPM).\cite{SPM09}

\item[\moc{PTMC}] keyword to enable the calculation of CALENDF--type probability tables, similar to the \moc{PT}
procedure. Here, the base points of the probability tables corresponding
to fission and scattering cross sections and to components of the transfer scattering matrix are also obtained using the CALENDF approach.

\item[\moc{PTSL}] keyword to enable the calculation of CALENDF--type probability tables, consistent
with the Ribon extended model, in some energy groups.

\item[\dusa{nir}]  the intermediate resonance (IR) approximation or the Ribon extended
model is imposed for energy groups with an index equal or greater than \dusa{nir}.
A statistical (ST) model is set in other groups.

\item[\moc{NONE}] keyword to specify that a statistical (ST) model is set in
all groups.

\item[\moc{NOEV}] keyword to force a mixture or a nuclide to be non-depleting (even in
cases where it is potentially depleting). Note that the mixture or nuclide keeps its
capability to produce energy. By default, the depleting isotopes are
automatically regognized as depleting.

\item[\moc{EVOL}] keyword to force a mixture or a nuclide to be depleting. By default, only fission products and
fissile isotopes are depleting.

\item[\moc{NOGAS}] keyword to specify that a mixture has a solid or liquid state (used for stopping power correction).
This is the default option.

\item[\moc{GAS}] keyword to specify that a mixture has a gaseous state (used for stopping power correction).

\item[\moc{SAT}] keyword to force a nuclide to be at saturation. By default, the saturation approximation is
automatically set as a function of the half life and capture cross sections of the isotope.

\item[\moc{COMB}]  keyword to specify that this mixture is reset with a
combination of previously defined mixtures.

\item[\dusa{mati}]  number associated with a previously defined mixture. In
order to insert some void in a mixture use \dusa{mati}=0. If the mixture is not
already defined one assumes that it represents a voided mixture.

\item[\dusa{relvol}] relative volume $V_{i}$ occupied by mixture
\dusa{mati}=$i$ in \dusa{matnum}.  Two cases can be considered, namely that
where the density $\rho_{i}$ of each mixture \dusa{mati} is provided along with
the weight percent for each isotope $J$ ($W_{i}^{j}$) and the case where the
explicit concentration $N_{i}^{j}$ of each isotope in a \dusa{mati} was provided
(it is forbidden to combined two mixtures with different isotopic content
description). In the case where the initial mixtures are defined using densities
$\rho_{i}$, the density ($\rho_k$) and volume ($V_{k}$) of the final mixture
will become:
  $$V_{k}=\sum_{i} V_{i} $$
  $$\rho_{k}=\frac{1}{V_{k}} \sum_{i}\rho_{i}V_{i}$$
and the weight percent will be changed in a consistent way, namely
  $$W_{k,J}=\frac{\rho_{i}V_{i}W_{i,J}}{\rho_{k} V_{k} } $$
When the explicit concentration are given we will use:
  $$N_{k,J}=\frac{V_{i}N_{i,J}}{V_{k} } $$

\vskip 0.08cm

There is a very common usage of keyword \moc{COMB}. In the following example, a new mixture with index 42
is defined in such a way to be identical to an existing mixture with index 25. 
\begin{verbatim}
    MIX 42 COMB 25 1.0
\end{verbatim}

\end{ListeDeDescription}

Note that in the structure \dstr{descmix1} one only needs to describe the
isotopes initially present in each mixture. DRAGON will then automatically
associate with each depleting mixture the additional isotopes required by the
available burnup chain. Moreover, the microscopic cross-section library
associated with these new isotopes will be the same as that of their parent
isotope. For example, suppose that mixture 1 contains isotope {\tt U235} which
is to be read on the DRAGON-formatted library associated with file {\tt
DRAGLIB}. Assume also that the depletion chain, which is written on the 
WIMS--AECL format library associated with file {\tt WIMSLIB}, states that isotope
{\tt U236} (initially absent in the mixture) can be generated form {\tt U235} by
neutron capture. Then, one can either specify explicitly from which library file
the microscopic cross sections associated with isotope {\tt U236} (zero
concentration) are to be read, or omit {\tt U236} from the mixture description
in which case DRAGON will assume that the microscopic cross sections associated
with isotope {\tt U236} are to be read from the same library as the cross
section for isotope {\tt U235}. Note that the isotopes added automatically will
remain at infinite dilution.

\vskip 0.15cm

If the \moc{SHI:} or \moc{TONE:} module is used for performing self-shielding calculation,
the self-shielding data for an isotope takes the form
\begin{verbatim}
    U235     = U235  5.105E-5 1
\end{verbatim}
\noindent where the last index indicates the self-shielding region (1 in this case). 

\vskip 0.15cm

If the {\tt USS:} module implementing the subgroup method is used,
additional self-shielding data is required:
\begin{itemize}
\item Physical probability tables are used (keyword {\tt SUBG}). Consider the following data:
\begin{verbatim}
    U235     = U235  5.105E-5 1 IRSET 0.0 81
\end{verbatim}
The data ``{\tt IRSET 0.0 81}'' indicates that a Goldstein-Cohen parameter
$\lambda_g$ equal
to 0.0 is used for all energy groups with an index equal or greater than 81. A value
of $\lambda_g=1.0$ corresponding to a statistical model is used by default.

\item Mathematical probability tables (with slowing-down correlated weight matrices) are used (keyword {\tt PTSL})
{\sl or} mathematical probability tables with the subgroup projection method (SPM)\cite{SPM09} are used (keyword {\tt PT}
or {\tt PTMC}). Consider the following data:
\begin{verbatim}
    U235     = U235  5.105E-5 1 IRSET PT 5
\end{verbatim}
The Goldstein-Cohen approximation is not used with mathematical (CALENDF) probability tables. The data ``{\tt IRSET PT 5}''
indicates that the CALENDF probability tables are used for energy groups with an index equal
or greater than 5, {\sl with the exception of the energy groups where no Autolib data
is available} and a statistical model (with physical probability tables) is used for energy groups with an index smaller
than 5. A statistical model is also imposed in groups where no Autolib data is available.

\vskip 0.15cm

The following data:
\begin{verbatim}
    U235     = U235  5.105E-5 1 IRSET PT NONE
\end{verbatim}
\noindent is useful to impose the statistical model (with physical probability tables) in all energy groups. This is equivalent of selecting
the {\tt SUBG} keyword in structure \dstr{desclib}.

\vskip 0.15cm

Mathematical (CALENDF) probability tables are used in each energy group where Autolib data is available if the following data is set:
\begin{verbatim}
    U235     = U235  5.105E-5 1 IRSET PT 1
\end{verbatim}
\noindent {\sl This latter definition is equivalent to the default behavior obtained using}
\begin{verbatim}
    U235     = U235  5.105E-5 1
\end{verbatim}
\end{itemize}

\vskip 0.25cm
\goodbreak

The structure \dstr{descmix2} is used to describe the modifications in the isotopic composition of a mixture.

\begin{DataStructure}{Structure \dstr{descmix2}}
\moc{MIX}  \dusa{matnum} $[$ \dusa{matold} $]$ $[$ \dusa{relden} $]$
$[$ \dusa{NAMALI} \dusa{dens} $]~[~\{$ \moc{NOEV} $|$ \moc{EVOL} $\}~]$
\end{DataStructure}

\vspace{-0.15cm}

\noindent
where:

\begin{ListeDeDescription}{mmmmmm}

\item[\moc{MIX}] keyword to specify the number identifying the next mixture to
be updated.

\item[\dusa{matnum}] mixture identifier on \dusa{MICLIB}. 

\item[\dusa{matold}] mixture identifier on \dusa{MICRHS}. By default, \dusa{matold}]$=$\dusa{matnum}.

\item[\dusa{relden}] relative density of updated mixture. The  concentration
of each isotope in the mixture is to be multiplied by this factor whether it 
comes from \dusa{MICLIB}, from \dusa{MICRHS} or is
specified explicitly using \dusa{dens}. 

\item[\dusa{NAMALI}] {\tt character*8} alias name for an isotope on
\dusa{MICLIB} to be modified. 

\item[\dusa{dens}] isotopic concentration of the isotope \dusa{NAMISO} in the
current mixture in $10^{24}cm^{-3}$.  When \dusa{relden} is specified, the
isotopic concentration becomes \dusa{dens}$\times$\dusa{relden}.

\item[\moc{NOEV}] keyword to force a mixture to be non-depleting (even in
cases where it is potentially depleting). Note that the mixture keeps its
capability to produce energy.

\item[\moc{EVOL}] keyword to force a mixture to be depleting. By default, only
mixtures containing fission products and/or fissile isotopes are depleting.

\end{ListeDeDescription}

\vskip 0.2cm

\subsubsection{Cross sections in Dragon}\label{sect:xs}
Multigroup cross sections in Draglibs files are of two types:
\begin{itemize}
\item Vectorial cross sections $\sigma_{x,g}$
\item Matrix cross sections $\sigma_{x,g\leftarrow h}.$
\end{itemize}
\begin{enumerate}
\item Total cross sections $\sigma_g$ are provided in ENDF evaluations as {\tt MT} $=1$. They are redundent with other information in the same evaluation. The vectorial total cross section is defined as
\begin{eqnarray}
\nonumber \sigma_g\negthinspace &=&\negthinspace \sigma_{{\rm e},g}+\sigma_{{\rm in},g}+\sigma_{{\rm (n,2n)},g}+\sigma_{{\rm (n,3n)},g}+\sigma_{{\rm (n,4n)},g}+\sigma_{{\rm f},g}+\sigma_{{\rm p},g}+\sigma_{\gamma,g}
+\sigma_{{\rm d},g}+\sigma_{{\rm t},g}+\sigma_{\alpha,g}\\
&+&\negthinspace \sigma_{2\alpha,g}+\sigma_{{\rm (n,np)},g}+\sigma_{{\rm any},g}
\end{eqnarray}
\noindent where $\sigma_{{\rm e},g}$ and $\sigma_{{\rm in},g}$ are the elastic and inelastic scattering cross sections and where the matrix cross sections are transformed into vectorial cross sections using
\begin{equation}
\sigma_{x,g}=\sum_h \sigma_{x,h\leftarrow g} \ , \ \ {\rm except \ for \ (n,}x{\rm n) \ reactions.}
\end{equation}
\item Inelastic scattering cross sections are sum over {\tt MT} 51 to 91 in the ENDF evaluation:
\begin{equation}
\sigma_{{\rm in},g}=\sum_{{\sl mt}=51}^{91} \sigma_{{\sl mt},g}=\sum_{{\sl mt}=51}^{91} \sum_h \sigma_{{\sl mt},h\leftarrow g} .
\end{equation}
\item (n,$x$n) vectorial cross sections are divided by the secondary neutron multiplicity:
\begin{equation}
\sigma_{{\rm (n,2n)},g}={1\over 2}\sum_h \sigma_{{\rm (n,2n)},h\leftarrow g} \ , \ \ \sigma_{{\rm (n,3n)},g}={1\over 3}\sum_h \sigma_{{\rm (n,3n)},h\leftarrow g} \ , \ \ \sigma_{{\rm (n,4n)},g}={1\over 4}\sum_h \sigma_{{\rm (n,4n)},h\leftarrow g} .
\end{equation}
\item {\tt SCAT} matrix reactions in Dragon are defined as
\begin{eqnarray}
\nonumber \sigma_{{\tt scat},h\leftarrow g} \negthinspace\negthinspace &=& \negthinspace\negthinspace \sigma_{{\rm e},h\leftarrow g}+\sigma_{{\rm (n,2n)},h\leftarrow g}+\sigma_{{\rm (n,3n)},h\leftarrow g}+\sigma_{{\rm (n,4n)},h\leftarrow g}
+\sum_{{\sl mt}=51}^{91} \sigma_{{\sl mt},h\leftarrow g} \\
&+& \negthinspace\negthinspace \sigma_{{\rm any},h\leftarrow g} \, .
\end{eqnarray}
\item Vectorial {\sl neutronic scattering} ({\tt SIGS}) in Dragon is defined as
\begin{equation}
\sigma_{{\tt sigs},g}=\sum_h \sigma_{{\tt scat},h\leftarrow g}
\end{equation}
\noindent so that the {\sl neutronic absorption}, used to compute the $K_\infty$ is
\begin{eqnarray}
\nonumber \sigma_g-\sigma_{{\tt sigs},g}\negthinspace &=&\negthinspace \sigma_{{\rm f},g}+\sigma_{{\rm p},g}+\sigma_{\gamma,g}
+\sigma_{{\rm d},g}+\sigma_{{\rm t},g}+\sigma_{\alpha,g}+\sigma_{2\alpha,g}+\sigma_{{\rm (n,np)},g}\\
&-&\negthinspace \sigma_{{\rm (n,2n)},g}-2\sigma_{{\rm (n,3n)},g}-3\sigma_{{\rm (n,4n)},g}
\label{eq:eq1}
\end{eqnarray}
\noindent where all these terms are available in the Dragon microlib under the following names:\\
\vskip 0.1cm
\begin{tabular}{| l | l | l |}
\hline
Dragon name & $\sigma_x$ & type \\
\hline
{\tt NTOT0} & $\sigma_g$ & total \\
{\tt SIGS00} & $\sigma_{{\tt sigs},g}$ & neutronic scattering \\
{\tt NFTOT} &$\sigma_{{\rm f},g}$ & fission \\
{\tt NP} & $\sigma_{{\rm p},g}$ & (n,p) \\
{\tt NG} & $\sigma_{\gamma,g}$ & (n,$\gamma$) \\
{\tt ND} &$\sigma_{{\rm d},g}$ & (n,d) \\
{\tt NT} &$\sigma_{{\rm t},g}$ & (n,t) \\
{\tt NA} &$\sigma_{\alpha,g}$ & (n,$\alpha$) \\
{\tt N2A} &$\sigma_{2\alpha,g}$ & (n,2$\alpha$) \\
{\tt NNP} &$\sigma_{{\rm (n,np)},g}$ & (n,np) \\
{\tt NX} &$\sigma_{{\rm any},g}$ & (n,anything) \\
{\tt N2N} &$\sigma_{{\rm (n,2n)},g}$ & (n,2n) \\
{\tt N3N} &$\sigma_{{\rm (n,3n)},g}$ & (n,3n) \\
{\tt N4N} &$\sigma_{{\rm (n,4n)},g}$ & (n,4n) \\
\hline
\end{tabular}
\item The {\sl infinite multiplication factor} $K_\infty$ in a Dragon mixture is defined as
\begin{equation}
K_\infty={\sum\limits_g \nu\Sigma_{{\rm f},g}\bar\phi_g \over \sum\limits_g \left(\Sigma_g-\Sigma_{{\tt sigs},g}\right)\bar\phi_g}
\end{equation}
\noindent where $\nu\Sigma_{{\rm f},g} $, $\Sigma_g$ and $\Sigma_{{\tt sigs},g}$ are the macroscopic $\nu$-fission, total and
neutronic scattering cross sections, and $\bar\phi_g$ is the neutron flux.

\end{enumerate}

\eject
