\begin{thebibliography}{99}

\bibitem{PIP2009}
A. H\'EBERT, {\sl Applied Reactor Physics}, Second Edition, Presses Internationales Polytechnique, ISBN 978-2-553-01698-1, 396 p., Montr\'eal, 2016.

\bibitem{cle2000}
R.~ROY, \textsl{The CLE-2000 Tool-Box}, 
Report IGE--163, Institut de g\'enie nucl\'eaire, \'{E}cole Polytechnique de Montr\'eal,
Montr\'{e}al, Qu\'{e}bec (1999).

\bibitem{ganlib5}
A. H\'EBERT and R.~ROY,
``The Ganlib5 kernel guide (64--bit clean version),"
Report IGE-332, \'Ecole Polytechnique de Montr\'eal, January 2013.

\bibitem{bivac}
A. H\'EBERT, ``Application of the Hermite Method for Finite Element Reactor Calculations," {\sl Nucl. Sci. Eng.}, {\bf 91}, 34 (1985).

\bibitem{SVAT1}
A. H\'EBERT, ``Variational Principles and Convergence Acceleration Strategies for the Neutron Diffusion Equation,'' {\sl Nucl. Sci. Eng.}, {\bf 91}, 414 (1985).

\bibitem{SVAT2}
A. H\'EBERT, ``Preconditioning the Power Method for Reactor Calculations,'' {\sl Nucl. Sci. Eng.}, {\bf 94}, 1 (1986).

\bibitem{MCFD}
A. H\'EBERT, ``Development of the Nodal Collocation Method for Solving the Neutron Diffusion Equation,'' {\sl Ann. Nucl. Energy}, {\bf 14}, 527 (1987).

\bibitem{Trivac}
A. H\'EBERT, ``TRIVAC, A Modular Diffusion Code for Fuel Management and Design Applications,'' {\sl Nucl. J. of Canada}, Vol. 1, No. 4, 325 (1987).

\bibitem{mixte-dual}
A. H\'EBERT, ``Application of a Dual Variational Formulation to Finite Element Reactor Calculations,'' {\sl Ann. nucl. Energy}, {\bf 20}, 823 (1993).

\bibitem{nse2005}
A. H\'EBERT, ``The Search for Superconvergence in Spherical Harmonics Approximations,'' {\sl Nucl. Sci. Eng.}, {\bf 154}, 134 (2006).

\bibitem{ane10a}
A. H\'EBERT, ``Mixed-dual implementations of the of the simplified $P_n$ method," {\sl Ann. nucl. Energy}, {\bf 37}, 498 (2010).

\bibitem{iram}
J. BAGLAMA, ``Augmented Block Householder Arnoldi Method,"
{\sl Linear Algebra Appl.}, {\bf 429}, Issue 10, 2315--2334 (2008).

\bibitem{wilkinson}
J. H. WILKINSON, ``The Algebraic Eigenvalue Problem,'' {\sl Clarendon Press}, Oxford (1965).

\bibitem{roy}
R. ROY, Private communication.

\bibitem{monier}
A. MONIER, ``Application of the Collocation Technique to the Spatial Discretization of the Generalized Quasistatic Method for Nuclear Reactors," Ph. D. Thesis, Polytechnique Montr\'eal, Institut de G\'enie \'Energ\'etique (December 1991).

\bibitem{benaboud}
A. BENABOUD, ``R\'esolution de l'\'equation de la diffusion neutronique pour une g\'eom\'etrie hexagonale," Ph. D. Thesis, Polytechnique Montr\'eal, Institut de G\'enie \'Energ\'etique (December 1992).

\bibitem{rts}
A. H\'EBERT, ``A Raviart--Thomas--Schneider solution of the diffusion equation in hexagonal geometry", {\sl Ann. nucl. Energy},
{\bf 35}, 363 (2008).

\bibitem{recipes}
W. H. PRESS, S. A. TEUKOLSKY, W. T. VETTERLING and B. P. FLANNERY, ``Numerical Recipes in FORTRAN," Second Edition, Chapter 16, Cambridge University Press (1992).

\bibitem{cronos}
J. J. LAUTARD, S. LOUBI\`ERE and C. FEDON-MAGNAUD, ``CRONOS, a Computational Modular System for Neutronic Core Calculations," Proc. International Atomic Energy Agency Specialists Mtg. on Advanced Calculational Methods for Power Reactors, Cadarache, France, September 1990.

\bibitem{nestle}
P. J. TURINSKY, R. M. K. AL-CHALABI, P. ENGRAND, H. N. SARSOUR, F. X. FAURE and  W. GUO, ``NESTLE: Few-group neutron diffusion equation solver utilzing the nodal expansion
method for eigenvalue, adjoint, fixed-source steady-state and transient problem,'' Electric Power Research Center, North Carolina State University, Raleigh, NC 27695-7909 (1994).

\bibitem{anm08}
A. H\'EBERT, ``A simplified presentation of the multigroup analytic nodal method
in 2--D Cartesian geometry," {\sl Ann. nucl. Energy}, {\bf 35}, 2142--2149 (2008).

\bibitem{anl}
``Argonne Code Center: Benchmark Problem Book," ANL-7416, Supp. 2, ID11-A2, Argonne National Laboratory (1977).

\bibitem{greenman}
G. GREENMAN, ``A Quasi-Static Flux Synthesis Temporal Integration Scheme for an Analytic Nodal Method," Nuclear Engineer's Thesis, Massachusetts Institute of Technology, Department of Nuclear Engineering (May 1980).

\end{thebibliography}
