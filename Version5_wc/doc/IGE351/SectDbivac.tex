\subsection{The \moc{bivact} dependent records on a
\dir{tracking} directory}\label{sect:bivactrackingdir}

When the \moc{BIVACT:} operator is used ($\mathsf{CDOOR}$={\tt 'BIVAC'}), the following elements in the vector
$\mathcal{S}^{t}_{i}$ will also be defined.

\begin{itemize}
\item $\mathcal{S}^{t}_{6}$: ({\tt ITYPE}) Type of BIVAC geometry:
\begin{displaymath}
\mathcal{S}^{t}_{6} = \left\{
\begin{array}{rl}
 2 & \textrm{Cartesian 1-D geometry} \\
 3 & \textrm{Tube 1-D geometry} \\
 4 & \textrm{Spherical 1-D geometry} \\
 5 & \textrm{Cartesian 2-D geometry} \\
 6 & \textrm{Tube 2-D geometry} \\
 8 & \textrm{Hexagonal 2-D geometry}
\end{array} \right.
\end{displaymath}

\item $\mathcal{S}^{t}_{7}$: ({\tt IHEX}) Type of hexagonal symmetry if $\mathcal{S}^{t}_{6}= 8$:
\begin{displaymath}
\mathcal{S}^{t}_{7} = \left\{
\begin{array}{rl}
 0 & \textrm{non-hexagonal geometry} \\
 1 & \textrm{S30} \\
 2 & \textrm{SA60} \\
 3 & \textrm{SB60} \\
 4 & \textrm{S90} \\
 5 & \textrm{R120} \\
 6 & \textrm{R180} \\
 7 & \textrm{SA180} \\
 8 & \textrm{SB180} \\
 9 & \textrm{COMPLETE}
\end{array} \right.
\end{displaymath}

\item $\mathcal{S}^{t}_{8}$: ({\tt IELEM}) Type of finite elements:
\begin{displaymath}
\mathcal{S}^{t}_{8} = \left\{
\begin{array}{rl}
 <0 & \textrm{Order $-\mathcal{S}^{t}_{8}$ primal finite elements} \\
 >0 & \textrm{Order $\mathcal{S}^{t}_{8}$ dual finite elements. The Thomas-Raviart or Thomas-Raviart-Schneider} \\
    & \textrm{method is used except if $\mathcal{S}^{t}_{9}=4$ in which case a mesh-centered finite difference} \\
    & \textrm{approximation is used}
\end{array} \right.
\end{displaymath}

\item $\mathcal{S}^{t}_{9}$: ({\tt ICOL}) Type of quadrature used to integrate
the mass matrix:
\begin{displaymath}
\mathcal{S}^{t}_{9} = \left\{
\begin{array}{rl}
 1 & \textrm{Analytical integration} \\
 2 & \textrm{Gauss-Lobatto quadrature (finite difference/collocation method)} \\
 3 & \textrm{Gauss-Legendre quadrature (superconvergent approximation)} \\
 4 & \textrm{mesh-centered finite difference approximation in hexagonal geometry}
\end{array} \right.
\end{displaymath}

\item $\mathcal{S}^{t}_{10}$: ({\tt ISPLH}) Type of hexagonal mesh splitting:
\begin{displaymath}
\mathcal{S}^{t}_{10} = \left\{
\begin{array}{rl}
 1 & \textrm{No mesh splitting}; \emph{or} \\
   & \textrm{$3$ lozenges per hexagon with Thomas-Raviart-Schneider approximation} \\
 K & \textrm{$6\times(K-1)\times(K-1)$ triangles per hexagon with finite-difference approximations} \\
   & \textrm{$3\times K \times K$ lozenges per hexagon with Thomas-Raviart-Schneider approximation}
\end{array} \right.
\end{displaymath}

\item $\mathcal{S}^{t}_{11}$: ({\tt LL4}) Order of the group-wise matrices.
Generally equal to
$\mathcal{S}^{t}_{2}$ except in cases where averaged fluxes are appended to the
unknown vector. $\mathcal{S}^{t}_{11}\le\mathcal{S}^{t}_{2}$.

\item $\mathcal{S}^{t}_{12}$: ({\tt LX}) Number of elements along the $X$ axis in Cartesian geometry or number of
hexagons.

\item $\mathcal{S}^{t}_{13}$: ({\tt LY}) Number of elements along the $Y$ axis.

\item $\mathcal{S}^{t}_{14}$: ({\tt NLF}) Number of components in the angular expansion of the flux. Must be a positive
even number. Set to zero for diffusion theory. Set to 2 for $P_1$ method.

\item $\mathcal{S}^{t}_{15}$: ({\tt ISPN}) Type of transport approximation if {\tt NLF}$\ne 0$:
\begin{displaymath}
\mathcal{S}^{t}_{15} = \left\{
\begin{array}{rl}
 0 & \textrm{Complete $P_n$ approximation of order {\tt NLF}$-1$} \\
 1 & \textrm{Simplified $P_n$ approximation of order {\tt NLF}$-1$}
\end{array} \right.
\end{displaymath}

\item $\mathcal{S}^{t}_{16}$: ({\tt ISCAT}) Number of terms in the scattering sources if {\tt NLF}$\ne 0$:
\begin{displaymath}
\mathcal{S}^{t}_{16} = \left\{
\begin{array}{rl}
 1 & \textrm{Isotropic scattering in the laboratory system} \\
 2 & \textrm{Linearly anisotropic scattering in the laboratory system} \\
 $n$ & \textrm{order $n-1$ anisotropic scattering in the laboratory system}
\end{array} \right.
\end{displaymath}
\noindent A negative value of $\mathcal{S}^{t}_{16}$ indicates that $1/3D^{g}$ values are used as $\Sigma_1^{g}$ cross sections.

\item $\mathcal{S}^{t}_{17}$: ({\tt NVD}) Number of base points in the Gauss-Legendre quadrature used to integrate
void boundary conditions if {\tt ICOL} $=3$ and {\tt NLF}$\ne 0$:
\begin{displaymath}
\mathcal{S}^{t}_{17} = \left\{
\begin{array}{rl}
 0 & \textrm{Use a ({\tt NLF}$+1$)--point quadrature consistent with $P_{{\rm NLF}-1}$ theory} \\
 1 & \textrm{Use a {\tt NLF}--point quadrature consistent with $S_{\rm NLF}$ theory} \\
 2 & \textrm{Use an analytical integration consistent with diffusion theory}
\end{array} \right.
\end{displaymath}

\end{itemize}

\goodbreak

The following records will also be present on the main level of a \dir{tracking}
directory.

\begin{DescriptionEnregistrement}{The \moc{bivact} records in
\dir{tracking}}{8.0cm}
\IntEnr
  {NCODE\blank{7}}{$6$}
  {Record containing the types of boundary conditions on each surface. =0 side
   not used; =1 VOID; =2 REFL; =4 TRAN; =5 SYME; =7 ZERO. {\tt NOODE(5)} and
   {\tt NOODE(6)} are not used.} 
\RealEnr
  {ZCODE\blank{7}}{$6$}{$1$}
  {Record containing the albedo value (real number) on each surface. {\tt ZOODE(5)}
   and {\tt ZOODE(6)} are not used.} 
\OptRealEnr
  {SIDE\blank{8}}{$1$}{$\mathcal{S}^{t}_{6}=8$}{cm}
  {Side of a hexagon.} 
\OptRealEnr
  {XX\blank{10}}{$\mathcal{S}^{t}_{1}$}{$\mathcal{S}^{t}_{6}\ne 8$}{cm}
  {Element-ordered $X$-directed mesh spacings after mesh-splitting for type 2
   and 5 geometries. Element-ordered radius after mesh-splitting for type 3
   and 6 geometries.} 
\OptRealEnr
  {YY\blank{10}}{$\mathcal{S}^{t}_{1}$}{$\mathcal{S}^{t}_{6}=5 \ {\rm or} \ 6$}{cm}
  {Element-ordered $Y$-directed mesh spacings after mesh-splitting for type 5
   and 6 geometries.} 
\OptRealEnr
  {DD\blank{10}}{$\mathcal{S}^{t}_{1}$}{$\mathcal{S}^{t}_{6}=3 \ {\rm or} \ 6$}{cm}
  {Element-ordered position used with type 3 and 6 geometries.} 
\IntEnr
  {KN\blank{10}}{$N_{\rm kn}\times\mathcal{S}^{t}_{1}$}
  {Element-ordered unknown list. $N_{\rm kn}$ is the number of unknowns per element.} 
\RealEnr
  {QFR\blank{9}}{$N_{\rm surf}\times\mathcal{S}^{t}_{1}$}{}
  {Element-ordered boundary condition. $N_{\rm surf}=4$ in Cartesian geometry and $=6$ in hexagonal geometry.} 
\IntEnr
  {IQFR\blank{8}}{$N_{\rm surf}\times\mathcal{S}^{t}_{1}$}
  {Element-ordered physical albedo indices. $N_{\rm surf}=4$ in Cartesian geometry and $=6$ in hexagonal geometry.} 
\RealEnr
  {BFR\blank{9}}{$N_{\rm surf}\times\mathcal{S}^{t}_{1}$}{}
  {Element-ordered boundary surface fractions.} 
\IntEnr
  {MU\blank{10}}{$\mathcal{S}^{t}_{11}$}
  {Indices used with compressed diagonal storage mode matrices.} 
\OptIntEnr
  {IPERT\blank{7}}{$\mathcal{S}^{t}_{12}\times (\mathcal{S}^{t}_{10})^2$}{*}
  {Mixture permutation index. This information is provided if and only if $\mathcal{S}^{t}_{6}=8, \ \mathcal{S}^{t}_{8}>0 \ {\rm and} \
  \mathcal{S}^{t}_{9}\le 3$.} 
\DirEnr
  {BIVCOL\blank{6}}
  {Sub-directory containing the unit matrices (mass, stiffness, nodal coupling,
   etc.) for a finite element discretization.}
\end{DescriptionEnregistrement}

\goodbreak

The following records will be present on the \moc{/BIVCOL/} sub-directory:

\begin{DescriptionEnregistrement}{Description of the \moc{/BIVCOL/} sub-directory}{8.0cm}
\RealEnr
  {T\blank{11}}{$L$}{}
  {Cartesian linear product vector. $L=|\mathcal{S}^{t}_{8}|+1$} 
\RealEnr
  {TS\blank{10}}{$L$}{}
  {Cylindrical linear product vector.} 
\RealEnr
  {R\blank{11}}{$L\times L$}{}
  {Cartesian mass matrix.} 
\RealEnr
  {RS\blank{10}}{$L\times L$}{}
  {Cylindrical mass matrix.} 
\RealEnr
  {Q\blank{11}}{$L\times L$}{}
  {Cartesian stiffness matrix.} 
\RealEnr
  {QS\blank{10}}{$L\times L$}{}
  {Cylindrical stiffness matrix.} 
\RealEnr
  {V\blank{11}}{$L\times (L-1)$}{}
  {Nodal coupling matrix.} 
\RealEnr
  {H\blank{11}}{$L\times (L-1)$}{}
  {Piolat transform coupling matrix (used with Thomas-Raviart-Schneider method).} 
\RealEnr
  {E\blank{11}}{$L\times L$}{}
  {Polynomial coefficients.} 
\RealEnr
  {RH\blank{10}}{6$\times$6}{}
  {Hexagonal mass matrix.} 
\RealEnr
  {QH\blank{10}}{6$\times$6}{}
  {Hexagonal stiffness matrix.} 
\RealEnr
  {RT\blank{10}}{3$\times$3}{}
  {Triangular mass matrix.} 
\RealEnr
  {QT\blank{10}}{3$\times$3}{}
  {Triangular stiffness matrix.} 
\end{DescriptionEnregistrement}

\eject
