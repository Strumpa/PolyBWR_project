\section{Contents of a \dir{microlib} directory}\label{sect:microlibdir}

A \dir{microlib} directory contains the set of multigroup microscopic 
cross sections associated with a set of isotopes. It also includes a \dir{macrolib} directory
where the macroscopic cross sections for the mixtures to which are associated these
isotopes are stored (see \Sect{macrolibdir}). Finally it may contains a \dir{depletion} directory
(see \Sect{microlibdirdepletion}) which is required for burnup calculation and a \dir{selfshield}
directory which is generated by the \moc{SHI:} or \moc{USS:} operator (see
\Sect{subgroupdirselfshield}). It is
therefore multi-level, namely, it contains sub-directories. Note that the contents of such a
directory will vary depending on the operator which was used to create or modify it. Here for
convenience we will define the variable
$\mathcal{M}$ to identify the creation operator:
\begin{displaymath}
\mathcal{M} = \left\{
\begin{array}{ll}
1 & \textrm{if the microlib is created or modified by the \moc{LIB:} or \moc{EVO:} operator}\\
2 & \textrm{if the microlib is created or modified by the \moc{EDI:} or
\moc{C2M:} operator}\\
3 & \textrm{if the microlib is modified by the \moc{SHI:} or \moc{USS:} operator}\\
4 & \textrm{if the microlib is part of a {\sc compo} object and is created by the \moc{COMPO:} operator}
\end{array} \right.
\end{displaymath}

In the case where the \moc{LIB:} or \moc{C2M:} operator is used to create the microlib, it appears on the root
level of the data structure. For the other case it is embedded as a
subdirectory of a surrounding data structure.

\subsection{State vector content for the \dir{microlib} data structure}\label{sect:microlibstate}

The dimensioning parameters for the \dir{microlib} data structure, which are stored in
the state vector $\mathcal{S}^{m}$, represent:

\begin{itemize}
\item The maximum number of mixtures $M_{m}=\mathcal{S}^{m}_{1}$
\item The number of isotopes $N_{I}=\mathcal{S}^{m}_{2}$ 
\item The number of groups ${G}=\mathcal{S}^{m}_{3}$ 
\item The order for the scattering anisotropy $L=\mathcal{S}^{m}_{4}$
($L=1$ is an isotropic collision; $L=2$ is a linearly anisotropic collision,
etc.)
\item The transport correction option $I_{tr}=\mathcal{S}^{m}_{5}$ 
\begin{displaymath}
I_{tr} = \left\{
\begin{array}{ll}
0 & \textrm{do not use a transport correction}\\
1 & \textrm{use an APOLLO-type transport correction (micro-reversibility at
all energies)}\\
2 & \textrm{recover a transport correction from the cross-section library}\\
3 & \textrm{use a WIMS-type transport correction (micro-reversibility below
4eV;}\\
  & \textrm{1/E current spectrum elsewhere)}\\
4 & \textrm{use a leakage correction based on {\tt NTOT1} data.}
\end{array} \right.
\end{displaymath}
\item Format of the included \dir{macrolib} $I_{p}=\mathcal{S}^{m}_{6}$
\begin{displaymath}
I_{p} = \left\{
\begin{array}{ll}
0 & \textrm{for the direct macroscopic cross sections} \\
1 & \textrm{for the adjoint macroscopic cross sections}
\end{array} \right.
\end{displaymath}
\item Option for removing delayed neutron effects from the \dir{microlib}
$I_{t}=\mathcal{S}^{m}_{7}$ 
\begin{displaymath}
I_{t} = \left\{
\begin{array}{ll}
1 & \textrm{include the delayed and prompt neutron effect} \\
2 & \textrm{consider only the prompt neutrons. This option is only available
with}\\
& \textrm{MATXS--type libraries.}
\end{array} \right.
\end{displaymath}
\item The number of independent libraries $N_{\mathrm{lib}}=\mathcal{S}^{m}_{8}$ 
\item The number of fast groups without self-shielding $N_{g,f}=\mathcal{S}^{m}_{9}$ 

Represents the number of fast energy groups to be treated without including resonance
effects. It is automatically determined from the cross-section libraries. This value,
which is only used by the self-shielding operator, can be modified using the keyword \moc{GRMAX}.

\item The maximum index of all groups with self-shielding $N_{g,e}=\mathcal{S}^{m}_{10}$.

In the case of a WIMS--type library, it represents the total number of energy groups above
$4.0$ eV. Otherwise, it is automatically determined from the cross-section libraries. This
value, is used by the self-shielding operator and can be modified locally in
this operator using the keyword \moc{GRMIN}.

\item The number of depleting isotopes $N_{d}=\mathcal{S}^{m}_{11}$ 
\item The number of depleting mixtures $N_{d,f}=\mathcal{S}^{m}_{12}$ 
\item The number of additional $\phi$--weighted editing cross sections $N_{e}=\mathcal{S}^{m}_{13}$ 
\item The number of mixtures $N_{m}=\mathcal{S}^{m}_{14}$ 
\item The number of resonant mixtures $N_{r}=\mathcal{S}^{m}_{15}$ 
\item The number of energy-dependent fission spectra $G_{\rm chi}=\mathcal{S}^{m}_{16}$. By default ($G_{\rm chi}=0$),
a unique fission spectrum is used. The theory of multiple fission spectra is presented in Ref.~\citen{mosca}.
\item Option for processing the cross-section libraries $I_{\rm proc}=\mathcal{S}^{m}_{17}$
\begin{displaymath}
I_{\rm proc} = \left\{
\begin{array}{ll}
-1 & \textrm{skip the library processing (i.e., no interpolation).} \\
0 & \textrm{perform an interpolation in temperature and dilution.} \\
1 & \textrm{perform an interpolation in temperature and compute probability} \\
  & \textrm{tables based on the tabulation in dilution.} \\
2 & \textrm{perform an interpolation in temperature and build a new temperature-} \\
  & \textrm{independent cross-section library in DRAGON format.} \\
3 & \textrm{perform an interpolation in temperature and compute CALENDF--type} \\
  & \textrm{mathematical probability tables based on BIN--type cross sections. Do} \\
  & \textrm{not compute the slowing-down correlated weight matrices. Option} \\
  & \textrm{compatible with the subgroup projection method (SPM).} \\
4 & \textrm{perform an interpolation in temperature and compute CALENDF--type} \\
  & \textrm{mathematical probability tables and slowing-down correlated weight} \\
  & \textrm{matrices based on BIN--type cross sections. Option compatible with} \\
  & \textrm{the Ribon extended method.} \\
5 & \textrm{perform an interpolation in temperature and compute CALENDF--type} \\
  & \textrm{mathematical probability tables based on BIN--type cross sections. This} \\
  & \textrm{option is similar to the $I_{\rm proc} =3$ procedure. Here, the base points of the} \\
  & \textrm{probability tables corresponding to fission and scattering cross sections} \\
  & \textrm{and to components of the transfer scattering matrix are also obtained} \\
  & \textrm{using the CALENDF approach.} \\
6 & \textrm{perform an interpolation in temperature and compute RSE--type proba-} \\
  & \textrm{bility tables based on BIN--type cross sections. RSE is the resonance} \\
  & \textrm{spectrum expansion method.} \\
\end{array} \right.
\end{displaymath}
\item Option for computing the macrolib $I_{\rm mac}=\mathcal{S}^{m}_{18}$
\begin{displaymath}
I_{\rm mac} = \left\{
\begin{array}{ll}
0 & \textrm{do not build an embedded macrolib.} \\
1 & \textrm{build an embedded macrolib. Mandatory if the microlib is to be used to} \\
  & \textrm{perform micro-depletion.}
\end{array} \right.
\end{displaymath}
\item The number of precursor groups producing delayed neutrons $N_{\rm del}=\mathcal{S}^{m}_{19}$.
\item The number of fissile isotopes producing fission products with {\tt PYIELD} data $N_{\rm dfi}=\mathcal{S}^{m}_{20}$ (see Table~\ref{tabl:tabiso3})
\item Option for completing the depletion chains with the missing isotopes $I_{\rm cmp}=\mathcal{S}^{m}_{21}$
\begin{displaymath}
I_{\rm cmp} = \left\{
\begin{array}{ll}
0 & \textrm{complete} \\
1 & \textrm{do not complete.}
\end{array} \right.
\end{displaymath}
\item The maximum number of isotopes per mixture $M_{\rm I}=\mathcal{S}^{m}_{22}$.
\item An integer index (1, 2, 3 or 4) used to set the accuracy of the CALENDF probability
tables. The highest the value, the more accurate are the tables. $N_{\rm
ipreci}=\mathcal{S}^{m}_{23}$.
\item Discontinuity factor flag $I_{\rm df}=\mathcal{S}^{m}_{24}$. This information is available in \dir{macrolib} directory (see \Sect{macrolibdir})
\begin{displaymath}
I_{\rm df} = \left\{
\begin{array}{ll}
0 & \textrm{no discontinuity factor information}\\
1 & \textrm{multigroup boundary current information is available}\\
2 & \textrm{boundary flux information (see \Sect{macroADF}) is available}\\
3 & \textrm{discontinuity factor information (see \Sect{macroADF}) is available}\\
4 & \textrm{matrix ($G \times G$) discontinuity factor information (see \Sect{macroADF}) is available.}
\end{array} \right.
\end{displaymath}
\item The maximum Legendre order of the weighting functions $I_{w}=\mathcal{S}^{m}_{25}$ 
\begin{displaymath}
I_{w} = \left\{
\begin{array}{ll}
0 & \textrm{use the flux as weighting function for all cross sections}\\
1 & \textrm{use the fundamental current ${\cal J}$ as weighting function for
scattering cross sections}\\
& \textrm{with order $\ge 1$ and compute both $\phi$-- and
${\cal J}$--weighted total cross sections.}
\end{array} \right.
\end{displaymath}
\item Number of companion particles in coupled sets $I_{\rm part}=\mathcal{S}^{M}_{26}$
\begin{displaymath}
I_{\rm part} = \left\{
\begin{array}{ll}
0 & \textrm{the microlib doesn't include coupled sets}\\
>0 & \textrm{number of companion particles.}
\end{array} \right.
\end{displaymath}
\item Option for performing the Sternheimer density correction for charged particle cases $I_{\rm ster}=\mathcal{S}^{m}_{27}$
\begin{displaymath}
I_{\rm ster} = \left\{
\begin{array}{ll}
0 & \textrm{do not perform the correction.} \\
1 & \textrm{perform Sternheimer correction applied for both restricted total stopping power}\\
& \textrm{and heat deposition cross section.}
\end{array} \right.
\end{displaymath}
\end{itemize}

\goodbreak
\clearpage

\subsection{The main \dir{microlib} directory}\label{sect:microlibdirmain}

The following records and sub-directories will be found on the first level of a \dir{microlib}
directory:
\begin{DescriptionEnregistrement}{Main records and sub-directories in \dir{microlib}}{7.0cm}
\CharEnr
  {SIGNATURE\blank{3}}{$*12$}
  {Signature of the \dir{microlib} data structure ($\mathsf{SIGNA}=${\tt L\_LIBRARY\blank{3}}).}
\IntEnr
  {STATE-VECTOR}{$40$}
  {Vector describing the various parameters associated with this data structure $\mathcal{S}^{m}_{i}$,
  as defined in \Sect{microlibstate}.}
\RealEnr
  {ENERGY\blank{6}}{$G+1$}{eV}
  {Energy groups limits $E_{g}$}
\RealEnr
  {DELTAU\blank{6}}{$G$}{}
  {Lethargy width of each group $U_{g}$}
\OptRealEnr
  {CHI-ENERGY\blank{2}}{$G_{\rm chi}+1$}{$G_{\rm chi}\ne 0$}{eV}
  {$E_{\rm chi}(g)$: Group energy limits defining the energy-dependent fission spectra. By default, a unique fission spectra is used.}
\OptIntEnr
  {CHI-LIMITS\blank{2}}{$G_{\rm chi}+1$}{$G_{\rm chi}\ne 0$}
  {$N_{\rm chi}(g)$: Group limit indices defining the energy-dependent fission spectra. By default, a unique fission spectra is used.}
\DirlEnr
  {ISOTOPESLIST}{$N_{I}$}
  {List of {\sc isotope} directories. Each component of this list follows the \dir{isotope} specification
  presented in Tables~\ref{tabl:tabiso1} to \ref{tabl:tabiso5} and is containing the cross section
  information associated with a specific isotope. The name of these isotopes is specified by 
  $\mathsf{NALIAS}_{i}$ as given in record {\tt ISOTOPESUSED}.}
\CharEnr
  {ISOTOPESUSED}{$(N_{I})*12$}
  {Alias name associated with each isotope $\mathsf{NALIAS}_{i}$. The first eight characters of the name of a macroscopic residual are set to {\tt '*MAC*RES'}.}
\OptCharEnr
  {ISOTOPERNAME}{$(N_{I})*12$}{$\mathcal{M}=1,3$}
  {Reference name associated with each isotope $\mathsf{NISO}_{i}$}
\OptIntEnr
  {ISOTOPESMIX\blank{1}}{$N_{I}$}{$\mathcal{M}\ne 4$}
  {Mixture number associated with each isotope $N_{I}$}
\RealEnr
  {ISOTOPESDENS}{$N_{I}$}{(cm b)$^{-1}\ $}
  {Isotopic density $\rho_{i}$}
\RealEnr
  {ISOTOPESTEMP}{$N_{I}$}{K}
  {Isotope temperature $T_{i}$}
\OptIntEnr
  {ISOTOPESTODO}{$N_{I}$}{$\mathcal{M}=1,3$}
  {=0: automatic detection of depletion for isotope $i$; =1: isotope $i$ is
  forced to be non depleting (keeps its capability to produce energy); =2: isotope $i$ is
  forced to be depleting; =3: isotope $i$ is at saturation.}
\IntEnr
  {ISOTOPESTYPE}{$N_{I}$}
  {Type index associated with each isotope $\mathsf{ITYP}_{i}$. $=1$: the isotope is
  not fissile and not a fission product; $=2$: fissile isotope; $=3$: fission
  product.}
\OptRealEnr
  {ISOTOPESVOL\blank{1}}{$N_{I}$}{$\mathcal{M}=2, 4$}{cm$^{3}$}
  {Volume occupied by isotope $V_{i}$}
\OptCharEnr
  {ILIBRARYTYPE}{$(N_{I})*8$}{$N_{\mathrm{lib}}\ge 1$}
  {Library type associated with each isotope $\mathsf{NLTY}_{i}$}
\OptCharEnr
  {ILIBRARYNAME}{$(N_{\mathrm{lib}})*64$}{$N_{\mathrm{lib}}\ge 1$}
  {Name associated with each cross-section library}
\OptIntEnr
  {ILIBRARYINDX}{$N_{I}$}{$N_{\mathrm{lib}}\ge 1$}
  {Index of the cross-section library associated with each isotope $1 \le \mathsf{LLIB}_{i}\le N_{\mathrm{lib}}$}
\OptCharEnr
  {ISOTOPESCOH\blank{1}}{$(N_{I})*8$}{$N_{\mathrm{lib}}\ge 1$}
  {Name of coherent scattering type at thermal energies $\mathsf{NCOH}_{i}$}
\OptCharEnr
  {ISOTOPESINC\blank{1}}{$(N_{I})*8$}{$N_{\mathrm{lib}}\ge 1$}
  {Name of incoherent scattering type at thermal energies $\mathsf{NINC}_{i}$}
\OptCharEnr
  {ISOTOPESRESK}{$(N_{I})*8$}{$N_{\mathrm{lib}}\ge 1$}
  {Name of resonance elastic scattering kernel (RESK) type at epithermal energies $\mathsf{NRSK}_{i}$}
\OptIntEnr
  {ISOTOPESNTFG}{$N_{I}$}{$N_{\mathrm{lib}}\ge 1$}
  {Number of thermal groups involved in coherent or incoherent scattering $G_{s,i}$}
\OptCharEnr
  {ISOTOPESHIN\blank{1}}{$(N_{I})*12$}{$N_{\mathrm{lib}}\ge 1$}
  {Name of resonant isotope associated with each isotope $\mathsf{NSHI}_{i}$}
\OptIntEnr
  {ISOTOPESSHI\blank{1}}{$N_{I}$}{$N_{\mathrm{lib}}\ge 1$}
  {Resonant mixture associated with each isotope $I_{R,i}$}
\OptRealEnr
  {ISOTOPESDSN\blank{1}}{$G \times N_{I}$}{${\displaystyle N_{\mathrm{lib}}\ge 1 \atop
  \displaystyle I_{\rm proc}=0}$}{b}
  {Standard dilution cross section for isotope $\sigma_{\mathrm{dil},i}$ in each energy group}
\OptRealEnr
  {ISOTOPESDSB\blank{1}}{$G \times N_{I}$}{${\displaystyle N_{\mathrm{lib}}\ge 1 \atop
  \displaystyle I_{\rm proc}=0}$}{b}
  {Livolant-Jeanpierre dilution cross section for isotope $\sigma_{\mathrm{LJ},i}$ in each energy group}
\OptIntEnr
  {ISOTOPESNIR\blank{1}}{$N_{I}$}{$N_{\mathrm{lib}}\ge 1$}
  {Use Goldstein-Cohen factor $\lambda_i$ in groups with index $\ge N^{\rm ir}_i$.
  Use $\lambda=1$ in other groups}
\OptRealEnr
  {ISOTOPESGIR\blank{1}}{$N_{I}$}{$N_{\mathrm{lib}}\ge 1$}{1}
  {Goldstein-Cohen parameter in low-energy resonant groups $\lambda_i$. Set to -998.0 if
  $I_{\rm proc}=3$, to -999.0 if $I_{\rm proc}=4$, to -1000.0 if $I_{\rm proc}=5$ and to -1001.0 if $I_{\rm proc}=6$.}
\OptRealEnr
  {MIXTURESVOL\blank{1}}{$N_{m}$}{$\mathcal{M}=2, 4$}{cm$^{3}$}
  {Volume occupied by each mixture}
\OptRealEnr
  {MIXTURESDENS}{$N_{m}$}{$\mathcal{M}=1$}{g/cm$^{3}$~~}
  {Volumetric mass density of each mixture $\rho_{m}$}
\OptCharEnr
  {ADDXSNAME-P0}{$(N_{e})*8$}{$N_{e}\ge 1$}
  {Names of the additional $\phi$--weighted editing cross sections $\mathsf{ADDXS}_{k}$ stored on \dir{macrolib}}
\OptCharEnr
  {PARTICLE\blank{4}}{$*1$}{$I_{\rm part}\ge 1$} 
  {Character name of the particle associated to the microlib. Usual names for
  particles are {\tt N} (neutrons), {\tt G} (photons), {\tt B} (electrons),
  {\tt C} (positrons) and {\tt P} (protons).}
\OptCharEnr
  {PARTICLE-NAM}{($I_{\rm part}+1$)$*1$}{$I_{\rm part}\ge 1$} 
  {Character name associated to each particle.}
\OptIntEnr
  {PARTICLE-NGR}{$I_{\rm part}+1$}{$I_{\rm part}\ge 1$}
  {Number of energy groups associated to each particle.}
\OptRealEnr
  {PARTICLE-MC2}{$I_{\rm part}+1$}{$I_{\rm part}\ge 1$}{eV}
  {Rest energy associated to each particle.}
\OptRealVar
  {\listedir{penergy}}{$G_i+1$}{$I_{\rm part}\ge 1$}{eV}
  {Set of arrays containing energy groups limits for a companion particle. The character name
  of each sub-directory is the concatenation of the character*1 name of the particle with ``{\tt ENERGY}''.
  For example, {\tt GENERGY} contains the energy mesh of secondary photons ($G_i+1$ values).}
\OptRealEnr
  {TIMESPER\blank{4}}{$2\times 3$}{$\mathcal{M}=2$}{}
  {Array $T_{j,i}$ that contains $T_{j,1}=t$, $T_{j,2}=B$ and $T_{j,3}=w$, the
   lower ($j=1$) and upper bounds ($j=2$) for the reference time in days, burnup
   in MW day T$^{-1}$ and irradiation in Kb$^{-1}$ respectively for which the
   perturbative expansion is valid}
\OptRealEnr
  {K-EFFECTIVE\blank{1}}{$1$}{*}{}
  {Effective multiplication constant $k_{\mathrm{eff}}$}
\OptRealEnr
  {K-INFINITY\blank{2}}{$1$}{*}{}
  {Infinite multiplication constant $k_{\mathrm{inf}}$}
\OptRealEnr
  {B2\blank{2}B1HOM\blank{3}}{$1$}{*}{cm$^{-2}$~~}
  {Homogeneous Buckling $B_{\mathrm{hom}}$}
\OptDirEnr
  {MACROLIB\blank{4}}{$I_{\rm mac} = 1$}
  {Sub-directory containing the \dir{macrolib} associated with this
  library, following the specification presented in \Sect{macrolibdirmain}.}
\OptDirEnr
  {DEPL-CHAIN\blank{2}}{$N_{d} \ge 1$}
  {Sub-directory containing the \dir{depletion} associated with this library, following
  the specification presented in \Sect{microlibdirdepletion}.}
\OptDirEnr
  {SHIBA\blank{7}}{$\mathcal{M}=3$}
  {Sub-directory containing the \dir{selfshield} associated with this
  library, following the specification presented in \Sect{shibadirselfshield}.
  This data is used by the \moc{SHI:} self-shielding module.}
\OptDirEnr
  {SHIBA\_SG\blank{4}}{$\mathcal{M}=3$}
  {Sub-directory containing the \dir{uss-selfshield} associated with this
  library, following the specification presented in \Sect{subgroupdirselfshield}.
  This sub-directory is present in the library builded by \moc{USS:} self-shielding module and used by \moc{USS:}.}
\IntEnr
  {MIXTUREGAS\blank{2}}{$N_{m}$}
  {State of each mixture (used for stopping power correction).}
\OptDirEnr
  {INDEX\blank{7}}{*}
  {Sub-directory containing indexing or table-of-content data for specific library
  files}
\end{DescriptionEnregistrement}

One will find in \Sect{macrolibdir} the description of a 
\dir{macrolib} directory and in 
\Sect{isotopedir} the contents of an \dir{isotope} directory. Note that if $N_{I}=2$ and 
\begin{displaymath}
\mathsf{NALIAS}_{i} = \left\{
\begin{array}{lll}
\texttt{U235    0001} & \textrm{for}& i=1\\
\texttt{Pu239   0003} & \textrm{for}& i=2
\end{array} \right.
\end{displaymath}
then \listedir{isotope} will correspond to the following two directories:

\begin{DescriptionEnregistrement}{Examples of isotopes directory in a \dir{microlib}}{7.5cm}
\DirEnr
  {U235\blank{4}0001}
  {Directory where the microscopic cross sections of \Iso{U}{235} are stored. These are 
   self-shielded cross section already interpolated in temperature. They correspond to the
  properties of mixture $1$}
\DirEnr
  {Pu239\blank{3}0003}
  {Directory where the microscopic cross sections of \Iso{Pu}{239} are stored. These are 
   self-shielded cross section already interpolated in temperature. They correspond to the
  properties of mixture $3$}
\end{DescriptionEnregistrement}

\subsection{State vector content for the depletion sub-directory}\label{sect:chainlibstate}

The dimensioning parameters for the depletion sub-directory, which are stored in
the state vector $\mathcal{S}^{d}$, represent:

\begin{itemize}
\item The number of depleting isotopes $N_{\mathrm{depl}}=\mathcal{S}^{d}_{1}$
\item The number of direct fissile isotopes (i.e., producing fission products) $N_{\mathrm{dfi}}=\mathcal{S}^{d}_{2}$
\item The number of fission fragments $N_{\mathrm{dfp}}=\mathcal{S}^{d}_{3}$. A fission fragment is produced directly by the
fission reaction. A fission product is a fission fragment or a daughter isotope
produced by decay or neutron-induced reaction.
\item The number of heavy isotopes $N_{\mathrm{H}}=\mathcal{S}^{d}_{4}$ 

This number represents the combination of fissile isotopes and the other isotopes produced from
these isotopes by reactions other than fission.

\item The number of fission products $N_{\mathrm{fp}}=\mathcal{S}^{d}_{5}$ 

This number represents the combination of fission fragments and the other
daughter isotopes produced by any reaction (decay or neutron induced).

\item The number of other isotopes $N_{\mathrm{O}}=\mathcal{S}^{d}_{6}$ 

This number represents the other depleting isotopes which are not produced by fission or by reaction
with fission isotopes or fission products but have a depletion chain.

\item The number of stable isotopes $N_{\mathrm{H}}=\mathcal{S}^{d}_{7}$ 

This number represents the non-depleting isotopes producing energy (mainly
by radiative capture). An isotope is considered to be stable if:
\begin{itemize}
\item its radioactive decay constant is zero
\item the isotope has no father and no daughter
\item energy is produced by the isotope.
\end{itemize}

\item The maximum number of depleting reactions, including radioactive decay and
neutron-induced reactions $M_{\mathrm{R}}=\mathcal{S}^{d}_{8}$ 

\item The maximum number of parent isotopes leading to the production of an isotope in the
depletion chain $M_{\mathrm{S}}=\mathcal{S}^{d}_{9}$ 

\item The number of energy-dependent fission yield matrices $N_{\mathrm{ndp}}=\mathcal{S}^{d}_{10}$ 

\end{itemize}

\subsection{The depletion sub-directory \dir{depletion} in
\dir{microlib}}\label{sect:microlibdirdepletion}

The following records and sub-directories will be found on the first level of a
\dir{depletion} directory:

\begin{DescriptionEnregistrement}{Main records and sub-directories in
\dir{depletion}}{7.5cm}
\label{tabl:tabchain}
\IntEnr
  {STATE-VECTOR}{$40$}
  {Vector describing the various parameters associated with this data structure $\mathcal{S}^{d}_{i}$,
  as defined in \Sect{chainlibstate}.}
\CharEnr
  {ISOTOPESDEPL}{$(N_{\mathrm{depl}})*12$}
  {Reference name of the isotopes $\mathsf{NISOD}_{i}$ present in the depletion chain}
\IntEnr
  {CHARGEWEIGHT}{$N_{\mathrm{depl}}$}
  {6-digit (integer number) nuclide identifier with atomic number $Z$ (2
  digits), mass number $A$ (3 digits) and energy state $E$ (0 for ground state, 1
  for first excited level, etc.). This identifier is not defined for pseudo
  fission products.}
\CharEnr
  {DEPLETE-IDEN}{$(M_{\mathrm{R}})*8$}
  {Reference name of the depletion reactions}
\IntEnr
  {DEPLETE-REAC}{$M_{\mathrm{R}}\times N_{\mathrm{depl}}$}
  {List of identifier for the depletion of an isotope $K_{r,i}^{\rm d}$}
\RealEnr
  {DEPLETE-ENER}{$M_{\mathrm{R}}\times N_{\mathrm{depl}}$}{Mev}
  {Energy per reaction associated with each depletion reaction $R_{r,i}^{\rm d}$}
\RealEnr
  {DEPLETE-DECA}{$N_{\mathrm{depl}}$}{$10^{-8}$ s$^{-1}\ $}
  {Radioactive decay constants}
\IntEnr
  {PRODUCE-REAC}{$M_{\mathrm{S}}\times N_{\mathrm{depl}}$}
  {List of identifier for the production of an isotope $K_{s,i}^{\rm p}$}
\RealEnr
  {PRODUCE-RATE}{$M_{\mathrm{S}}\times N_{\mathrm{depl}}$}{1}
  {Branching ratio associated with each production reaction $R_{s,i}^{\rm p}$}
\RealEnr
  {FISSIONYIELD}{$N_{\mathrm{ndp}} \times N_{\mathrm{dfi}}\times N_{\mathrm{dfp}}$}{1}
  {Fission yield for each direct fissile isotope $i$ to each fission fragment $j$ in fission yield
  macrogroup $k$ $Y_{k,{i\to j}}$}
\OptRealEnr
  {ENERGY-YIELD}{$N_{\mathrm{ndp}+1}$}{$N_\mathrm{ndp}\ge 2$}{eV}
  {Energy limits of fission yield macrogroups $E_{k}^{\rm fiss}$}
\end{DescriptionEnregistrement}

An isotope $\mathsf{NISO}_{i}$ defined in \Sect{microlibdirmain} is considered
to be part of the depletion chain only if one can find a value of $1 \le j \le N_{\rm depl}$
such that $\mathsf{NISO}_{i}= \mathsf{NISOD}_{j}$.
Some depleting isotopes may be automatically added to the \dir{microlib} directory.
In this case, the reference name in record {\tt ISOTOPERNAME} is taken equal
to its reference name in {\tt ISOTOPESDEPL} and the alias name in record
{\tt ISOTOPESUSED} is taken equal to the
first 8 characters of its reference name in {\tt ISOTOPESDEPL}, completed by a
4-digit mixture identifier. If the reference name contains an underscore, the
alias name is truncated at the first underscore. For example, an isotope
present in mixture 2 with a reference name equal to {\tt D2O\_3\_P5} is
translated into an alias name equal to {\tt D2O\blank{5}0002}.

\vskip 0.2cm

The contents of the variables $K_{r,i}^{\rm d}$ is used to identify the type of isotope under
consideration. For each isotope $i$, $r$ will take
successively the values $1$ to $M_{\mathrm{D}}$ depending on the type of
reaction $\mathsf{NREAD}_{r}$ one wishes to analyze, namely

\vskip 0.2cm

\begin{tabular}{|l|l|}
\hline
$\mathsf{NREAD}_{1}=${\tt DECAY\blank{3}} & isotope may undergo radioactive decay \\
$\mathsf{NREAD}_{2}=${\tt NFTOT\blank{3}} & isotope may undergo fission or is a
fission fragment \\
& $^{1}_{0}n + ^{A}_{Z}X \to \ ^{A+1-\nu-B}_{Z-Y}U + ^{B}_{Y}V + \nu \ ^{1}_{0}n + \gamma$ \\
$\mathsf{NREAD}_{3}=${\tt NG\blank{6}} & isotope may undergo neutron capture (mt$=$102) \\
& $^{1}_{0}n + ^{A}_{Z}X \to \ ^{A+1}_{Z}X + \gamma$ \\
$\mathsf{NREAD}_{4}=${\tt N2N\blank{5}} & isotope may undergo (n,2n) reaction (mt$=$16) \\
& $^{1}_{0}n + ^{A}_{Z}X \to \ ^{A-1}_{Z}X + 2 \ ^{1}_{0}n  + \gamma$ \\
$\mathsf{NREAD}_{5}=${\tt N3N\blank{5}} & isotope may undergo (n,3n) reaction (mt$=$17) \\
$\mathsf{NREAD}_{6}=${\tt N4N\blank{5}} & isotope may undergo (n,4n) reaction (mt$=$37) \\
$\mathsf{NREAD}_{7}=${\tt NA\blank{6}} & isotope may undergo (n,$\alpha$) reaction (mt$=$107) \\
$\mathsf{NREAD}_{8}=${\tt NP\blank{6}} & isotope may undergo (n,p) reaction (mt$=$103) \\
$\mathsf{NREAD}_{9}=${\tt N2A\blank{5}} & isotope may undergo (n,2$\alpha$) reaction (mt$=$108) \\
$\mathsf{NREAD}_{10}=${\tt NNP\blank{5}} & isotope may undergo (n,np) reaction (mt$=$28) \\
$\mathsf{NREAD}_{11}=${\tt ND\blank{6}} & isotope may undergo (n,d) reaction (mt$=$104)\\
$\mathsf{NREAD}_{12}=${\tt NT\blank{6}} & isotope may undergo (n,t) reaction (mt$=$105) \\
\hline
\end{tabular}

\vskip 0.3cm

\noindent where symbols n, $\alpha$, p, d and t represent neutron, alpha particle, proton, deuteron
and triton, respectively.

\vskip 0.2cm

The contents of the variable $K_{r,i}^{\rm d}$ is used to specify the
properties of reaction $r$ for each isotope $i$ under consideration.
Here $K_{r,i}^{\rm d}$ contains two different types of informations, namely
$d(r)$ and $i(r)$ which are defined as follows:

\begin{equation}
  d(r)=K_{r,i}^{\rm d} \bmod \ 100 \ \ \ \ {\rm and} \ \ \ \ i(r)={K_{r,i}^{\rm d}  \over 100}
\end{equation}

\noindent where

\begin{displaymath}
d(r) = \left\{
\begin{array}{ll}
0 & \textrm{isotope $i$ does not deplete by reaction $\mathsf{NREAD}_{r}$} \\
1 & \textrm{isotope $i$ will deplete by reaction $\mathsf{NREAD}_{r}$} \\
2 & \textrm{isotope $i$ does not deplete by reaction $\mathsf{NREAD}_{r}$ but yields energy production} \\
3 & \textrm{isotope $i$ is fissile without fission yield. Valid only for $r$ such
that $\mathsf{NREAD}_{r}=${\tt NFTOT}} \\
4 & \textrm{isotope $i$ is fissile with fission yield. Valid only for $r$ such
that $\mathsf{NREAD}_{r}=${\tt NFTOT}} \\
5 & \textrm{isotope $i$ is a fission fragment. Valid only for $r$ such
that $\mathsf{NREAD}_{r}=${\tt NFTOT}}
\end{array} \right.
\end{displaymath}

\noindent and $i(r)=0$ unless $4\le d(r)\le 5$. When $d(r)=4$, $i(r)$ represents the fissile
isotope index while for $d(r)=5$, $i(r)$ represents the fission fragment index.
The fractional yield for the production of the fission fragment $i(r')$ from the
fissile isotope $i(r)$ is stored in matrix $Y_{i(r)\to i(r')}$.
The contents of the vector $R_{r,i}^{\rm d}$ is the energy in MeV emitted per
decay or reaction.

\vskip 0.2cm

The contents of the variables $K_{s,i}^{\rm p}$ is used to identify explicitly the parent isotope
which can generate the current isotope $i$. The maximum number of parent reaction for this
depletion chain is $M_{\mathrm{S}}$. $K_{s,i}^{\rm p}$ contains two different types of information,
namely $r(s)$ and $i(s)$ which are defined as follows:

\begin{equation}
  r(s)=K_{s,i}^{\rm p}\bmod 100 \ \ \ \ {\rm and} \ \ \ \ i(s)={{K_{s,i}^{\rm p}}\over{100}}
\end{equation}

\noindent where $r(s)=0$ indicates that the list of parent isotopes is complete while $r(s)>0$
refers to the reaction type $\mathsf{NREAD}_{r(s)}$ and can take the following values:

\begin{displaymath}
r(s) = \left\{
\begin{array}{ll}
1 & \textrm{isotope $i$ produced by radioactive decay}\\
2 & \textrm{isotope $i$ produced by fission (this contribution is kept apart from record} \\
  & \textrm{{\tt 'FISSIONYIELD'})} \\
3 & \textrm{isotope $i$ produced by neutron capture} \\
\ge 4 & \textrm{isotope $i$ produced by $\mathsf{NREAD}_{r(s)}$ reaction}
\end{array} \right.
\end{displaymath}

In the case where $r(s)>0$, $i(s)$ represents the isotope index associated
with the parent isotope and $R_{s,i}^{\rm p}$ represents the branching
ratio in fraction for the production of isotope $\mathsf{NISOD}_{i}$ from a neutron
reaction with the parent isotope $\mathsf{NISOD}_{i(s)}$.

\goodbreak

\subsection{State vector content for the {\sc shiba} self-shielding sub-directory}\label{sect:ssshibastate}

The dimensioning parameters for the self-shielding sub-directory, which are stored in the state vector
$\mathcal{S}^{s}$, represent:

\begin{itemize}
\item The first group for which self-shielding takes place $G_{\mathrm{min}}=\mathcal{S}^{s}_{1}$
       By default $G_{\mathrm{min}}=N_{g,f}+1$

\item The last group for which self-shielding takes place $G_{\mathrm{max}}=\mathcal{S}^{s}_{2}$
       By default $G_{\mathrm{max}}=N_{g,e}$

\item The maximum number of iterations in the self-shielding calculation $M_{r}=\mathcal{S}^{s}_{3}$ 
       
\item Enabling flag for the Livolant-Jeanpierre normalization $I_{\mathrm{lj}}=\mathcal{S}^{s}_{4}$ 
       
\item Enabling flag for the use of Goldstein-Cohen parameters $I_{\mathrm{gc}}=\mathcal{S}^{s}_{5}$ 

\item The transport correction option used in self-shielding $I_{\mathrm{tc}}=\mathcal{S}^{s}_{6}$ 
\begin{displaymath}
I_{\mathrm{tc}} = \left\{
\begin{array}{ll}
0 & \textrm{no transport correction applied in self-shielding calculation} \\
1 & \textrm{use transport corrected cross section in self-shielding calculation}
\end{array} \right.
\end{displaymath}

\item Type of self-shielding model $I_{\mathrm{level}}=\mathcal{S}^{s}_{7}$ 
\begin{displaymath}
I_{\mathrm{level}} = \left\{
\begin{array}{ll}
0 & \textrm{Stamm'ler model without distributed self-shielding effects} \\
1 & \textrm{Stamm'ler model with the Nordheim (PIC) distributed self-shielding model} \\
2 & \textrm{Stamm'ler model with both Nordheim (PIC) distributed self-shielding model} \\
  & \textrm{and Riemann integration method.}
\end{array} \right.
\end{displaymath}

\item The option to indicate whether a specific flux solver or collision probability matrices
are used to perform the self-shielding calculation $I_{\mathrm{flux}}=\mathcal{S}^{s}_{8}$
(see \moc{PIJ} and \moc{ARM} keyword in \moc{SHI:} operator input option)
\begin{displaymath}
I_{\mathrm{flux}} = \left\{
\begin{array}{rl}
 1 & \textrm{use a specific flux solver (the \moc{ARM} keyword was selected)} \\
 2 & \textrm{use collision probability matrices (the \moc{PIJ} keyword was selected)}
\end{array} \right.
\end{displaymath}

\end{itemize}

\subsection{State vector content for the subgroup self-shielding sub-directory}\label{sect:sssubgroupstate}

The dimensioning parameters for the self-shielding sub-directory, which are stored in the state vector
$\mathcal{S}^{s}$, represent:

\begin{itemize}
\item The first group for which self-shielding takes place $G_{\mathrm{min}}=\mathcal{S}^{s}_{1}$
       By default $G_{\mathrm{min}}=N_{g,f}+1$

\item The last group for which self-shielding takes place $G_{\mathrm{max}}=\mathcal{S}^{s}_{2}$
       By default $G_{\mathrm{max}}=N_{g,e}$
       
\item SPH enabling flag $I_{\mathrm{sph}}=\mathcal{S}^{s}_{3}$ 

\begin{displaymath}
I_{\mathrm{sph}} = \left\{
\begin{array}{ll}
0 & \textrm{skip the multigroup equivalence procedure} \\
1 & \textrm{perform a multigroup equivalence procedure (SPH procedure or} \\
  & \textrm{Livolant-Jeanpierre equivalence)}
\end{array} \right.
\end{displaymath}

\item The transport correction option used in self-shielding $I_{\mathrm{tc}}=\mathcal{S}^{s}_{4}$ 
\begin{displaymath}
I_{\mathrm{tc}} = \left\{
\begin{array}{ll}
0 & \textrm{no transport correction applied in self-shielding calculation} \\
1 & \textrm{use transport corrected cross section in self-shielding calculation}
\end{array} \right.
\end{displaymath}

\item The number of iterations in the self-shielding calculation $M_{r}=\mathcal{S}^{s}_{5}$ 

\item The option to indicate whether a specific flux solver or collision probability matrices
are used to perform the self-shielding calculation $I_{\mathrm{flux}}=\mathcal{S}^{s}_{6}$
(see \moc{PIJ} and \moc{ARM} keyword in \moc{USS:} operator input option)
\begin{displaymath}
I_{\mathrm{flux}} = \left\{
\begin{array}{rl}
 1 & \textrm{use a specific flux solver (the \moc{ARM} keyword was selected)} \\
 2 & \textrm{use collision probability matrices (the \moc{PIJ} keyword was selected)} 
\end{array} \right.
\end{displaymath}

\item The $\gamma$ factor enabling flag $I_{\mathrm{\gamma}}=\mathcal{S}^{s}_{7}$. These factors
are used to represent the moderator absorption effect in the Sanchez--Coste self-shielding method.
\begin{displaymath}
I_{\mathrm{\gamma}} = \left\{
\begin{array}{ll}
0 & \textrm{the $\gamma$ factors are set to 1.0} \\
1 & \textrm{the $\gamma$ factors are computed}
\end{array} \right.
\end{displaymath}

\item The simplified self-shielding enabling flag $I_{\mathrm{calc}}=\mathcal{S}^{s}_{8}$ 
\begin{displaymath}
I_{\mathrm{calc}} = \left\{
\begin{array}{ll}
0 & \textrm{perform a delailed self-shielding calculation} \\
1 & \textrm{perform a simplified self-shielding calculation using data recovered from the} \\
 & {\tt -DATA-CALC-} \textrm{ directory}
\end{array} \right.
\end{displaymath}

\item The flag for ignoring the activation of the mutual resonance shielding model $I_{\mathrm{noco}}=\mathcal{S}^{s}_{9}$ 
\begin{displaymath}
I_{\mathrm{noco}} = \left\{
\begin{array}{ll}
0 & \textrm{follow the directives set by {\tt LIB}} \\
1 & \textrm{ignore the directives set by {\tt LIB}}
\end{array} \right.
\end{displaymath}

\item Maximum number of fixed point iterations for the ST scattering source convergence $I_{\mathrm{max}}=\mathcal{S}^{s}_{10}$ 

\item Type of elastic slowing-down kernel in Autosecol $I_{\mathrm{ialt}}=\mathcal{S}^{s}_{11}$ 
\begin{displaymath}
I_{\mathrm{ialt}} = \left\{
\begin{array}{ll}
0 & \textrm{use exact elastic kernel} \\
1 & \textrm{use an approximate kernel for the resonant isotopes}
\end{array} \right.
\end{displaymath}

\item Maximum storage size for the slowing-down kernel values in Autosecol $I_{\mathrm{tra}}=\mathcal{S}^{s}_{12}$ 

\item Normalization flag for the collision probabilities $I_{\mathrm{norm}}=\mathcal{S}^{s}_{13}$ 
\begin{displaymath}
I_{\mathrm{norm}} = \left\{
\begin{array}{ll}
0 & \textrm{no normalization} \\
1 & \textrm{remove any remaining leakage from collision probabilities}
\end{array} \right.
\end{displaymath}

\item Seed integer used by the random number generator $I_{\mathrm{seed}}=\mathcal{S}^{s}_{14}$.

\end{itemize}

\clearpage

\subsection{The {\sc shiba} self-shielding sub-directory \dir{selfshield} in
\dir{microlib}}\label{sect:shibadirselfshield}

\begin{DescriptionEnregistrement}{Main records and sub-directories in \dir{selfshield}}{7.5cm}
\IntEnr
  {STATE-VECTOR}{$40$}
  {Vector describing the various parameters associated with this data structure $\mathcal{S}^{s}_{i}$,
  as defined in \Sect{ssshibastate}.}
\RealEnr
  {EPS-SHIBA\blank{3}}{$1$}{1}
  {Value of the relative convergence criterion for the self-shielding iterations in {\tt SHI:}. }
\end{DescriptionEnregistrement}

\subsection{The subgroup self-shielding sub-directory \dir{uss-selfshield} in
\dir{microlib}}\label{sect:subgroupdirselfshield}

\begin{DescriptionEnregistrement}{Main records and sub-directories in \dir{uss-selfshield}}{7.5cm}
\IntEnr
  {STATE-VECTOR}{$40$}
  {Vector describing the various parameters associated with this data structure $\mathcal{S}^{s}_{i}$,
  as defined in \Sect{sssubgroupstate}.}
\OptDirEnr
  {-DATA-CALC-\blank{1}}{$I_{\mathrm{calc}} = 1$}
  {Name of directory containing the data required by a simplified self-shielding
  calculation. This type of calculation allows the definition of a single
  self-shielded isotope in several resonant mixtures.}
\DirVar
  {\listedir{isodir}}
  {List of sub-directories that contain isotopic subgroup information collected by the {\tt USS:} module.}
\end{DescriptionEnregistrement}

The list of directory \listedir{isodir} named $\mathsf{ISODIR}$ will be composed according to
\begin{quote}
\verb|WRITE(|$\mathsf{ISODIR}$,\verb|'(1HC,I5,1H/,I5)')| $iso$,$nbiso$
\end{quote}
\noindent where $iso$ is the isotope index and $nbiso$ is the total number of isotopes. \listedir{isodir} is defined in Table~\ref{table:isodir}.

\begin{DescriptionEnregistrement}{Main records and sub-directories in \listedir{isodir}}{7.5cm}\label{table:isodir}
\DirVar
  {\listedir{cordir}}
  {List of sub-directories that contain correlated isotopic subgroup information collected by the {\tt USS:} module.}
\end{DescriptionEnregistrement}

The list of directory \listedir{cordir} named $\mathsf{CORDIR}$ will be composed according to
\begin{quote}
\verb|WRITE(|$\mathsf{CORDIR}$,\verb|'(3HCOR,I4,1H/,I4)')| $ires$,$nires$
\end{quote}
\noindent where $ires$ is the correlated isotope index and $nires$ is the total number of correlated isotopes. \listedir{cordir} is defined
in Table~\ref{table:cordir}.

\begin{DescriptionEnregistrement}{Main records and sub-directories in \listedir{cordir}}{7.5cm}\label{table:cordir}
\DirlEnr
  {NWT0-PT\blank{5}}{$G$}
  {List of real arrays. Each component of this list contains subgroup flux information in correlated fuel regions, as computed by {\tt USS:}.
  Each real array has dimension $N_{\rm nbnrs}\times K_g$, where $N_{\rm nbnrs}$ is the number of correlated fuel regions and $K_g$ is the
  number of base points in energy group $g$.}
\OptDirlEnr
  {ASSEMB-PHYS\blank{1}}{$N_{\rm asm}$}{$I_{\mathrm{calc}} = 1$}
  {List of {\sc assemb-phys} directories. Each component of this list contains subgroup assembly information for the subgroup method with
  physical probability tables. The specification of this directory is given in Sect.~\ref{sect:asminfodhdirgroup} or~\ref{sect:asminfodirgroup}
  depending if a double-heterogeneity is present or not. A double-heterogeneity is present if $\mathcal{S}^{t}_{40}=1$
  in the {\sc tracking} object.}
\OptDirlEnr
  {ASSEMB-RIBON}{$N_{\rm asm}$}{$I_{\mathrm{calc}} = 3,4$}
  {List of {\sc assemb-ribon} directories. Each component of this list contains subgroup assembly information for the subgroup projection
  or Ribon extended method. The specification of this directory is given in Sect.~\ref{sect:asminfodhdirgroup} or~\ref{sect:asminfodirgroup}
  depending if a double-heterogeneity is present or not. A double-heterogeneity is present if $\mathcal{S}^{t}_{40}=1$
  in the {\sc tracking} object.}
\OptDirlEnr
  {ASSEMB-RSE\blank{2}}{$N_{\rm asm}$}{$I_{\mathrm{calc}} = 6$}
  {List of {\sc assemb-rse} directories. Each component of this list contains subgroup assembly information for the resonance spectrum
  expansion method. The specification of this directory is given in Sect.~\ref{sect:asminfodhdirgroup} or~\ref{sect:asminfodirgroup}
  depending if a double-heterogeneity is present or not. A double-heterogeneity is present if $\mathcal{S}^{t}_{40}=1$
  in the {\sc tracking} object.}
\end{DescriptionEnregistrement}

\goodbreak

\subsection{Contents of an \dir{isotope} directory}\label{sect:isotopedir}

Each isotope directory always contains a cross section identifier record {\tt SCAT-SAVED\blank{2}}
which must be used to verify if a given cross section type has
been saved for this isotope.

\begin{DescriptionEnregistrement}{Isotopic cross section identifier records}{7.5cm}
\label{tabl:tabiso1}
\OptCharEnr
  {ALIAS\blank{7}}{$*12$}{$\mathcal{M} \ge 0$}
  {Alias character*12 name of a microlib isotope. This record is not provided in {\sc draglib} objects.}
\IntEnr
  {SCAT-SAVED\blank{2}}{$L$}
  {Vector $\kappa^{\rm scat}_{k}$ to identify the various type of
   Legendre-dependent cross sections saved for this isotope}
\RealEnr
  {AWR\blank{9}}{$1$}{nau}
  {Ratio of the isotope mass divided by the neutron mass}
\OptDirEnr
  {PT-TABLE\blank{4}}{$I_{\rm proc}\ge 1$}
  {Sub-directory containing probability table information, following the specification given in \Sect{pt-table}.
  $I_{\rm proc}$ is defined in \Sect{microlibdir}. This sub-directory is present in the microlib builded by the {\tt LIB:} module.}
\end{DescriptionEnregistrement}

Delayed neutron data can be present for some fissile isotopes on the \dir{isotope} directory. If $N_{\rm
del}\ge 1$ precursor groups are used, the following information is available:

\begin{DescriptionEnregistrement}{Delayed neutron reaction
records}{7.5cm}
\label{tabl:tabiso2}
\OptRealVar
  {\{nusid\}}{$G$}{$N_{del}\ge 1$}{b}
  {$\nu\sigma_{{\rm f},\ell}^{{\rm D},g}$: The product of $\sigma_{\rm f}^{g}$, the fission cross section with
   $\nu_{\ell}^{{\rm D},g}$, the averaged number of fission--emitted delayed
   neutron produced in the precursor group $\ell$.}
\OptRealVar
  {\{chid\}}{$G$}{$N_{del}\ge 1$}{1}
  {$\chi^{{\rm D},g}_\ell$: Delayed fission spectrum, normalized to one, for the delayed fission
   neutrons in precursor group $\ell$.}
\OptRealEnr
  {LAMBDA-D\blank{4}}{$N_{\rm del}$}{$N_{\rm del}\ge 1$}{s$^{-1}$}
  {$\lambda^{\rm D}_\ell$: Decay constant associated with the precursor group $\ell$. We must have
   $0 <\lambda^{\rm D}_\ell<\lambda^{\rm D}_{\ell+1}$.}
\end{DescriptionEnregistrement}

The delayed component of the fission yields in each precursor group $\ell$ is given as
$\nu_\ell^{{\rm D},g}$. The quantities $\pi^{{\rm D},g}$ and $\nu_\ell^{{\rm D},g} \ \sigma_{\rm f}^g$ are defined as
$$\pi^{{\rm D},g}={\nu^{{\rm D},g} \ \sigma_{\rm f}^g \over
   \left( \nu^g \sigma_{\rm f}^g \right)^{\rm ss}} \ \ .$$

\noindent and

$$\nu_\ell^{{\rm D},g} \ \sigma_{\rm f}^g=\omega_\ell \ \pi^{{\rm D},g} \
   \left( \nu^g \sigma_{\rm f}^g \right)^{\rm ss}$$

\noindent where the superscript ${\rm ss}$ indicates steady-state values. The
delayed neutron records {\sl \{nusid\}} and {\sl \{chid\}} will be
composed, using the following FORTRAN instructions, as $\mathsf{NUSIGD}$ and $\mathsf{CHID}$:
  \begin{displaymath}
    \mathtt{WRITE(}\mathsf{NUSIGD}\mathtt{,'(A6,I2.2)')} \ \mathtt{'NUSIGF'},ell
  \end{displaymath}
  \begin{displaymath}
    \mathtt{WRITE(}\mathsf{CHID}\mathtt{,'(A3,I2.2)')} \ \mathtt{'CHI'},ell
  \end{displaymath}
for $1\leq ell \leq N_{\rm del}$. For example, in the case where two group cross sections are considered
($N_{\rm del}=2$), the following records would be generated:

\begin{DescriptionEnregistrement}{Example of delayed--neutron records in
\dir{isotope}}{8.0cm}
\OptRealEnr
  {NUSIGF01\blank{4}}{$G$}{$N_{\rm del}\ge 1$}{b}
  {$\nu\sigma_{{\rm f},1}^{{\rm D},g}$: The product of $\sigma_{\rm f}^{g}$, the fission cross section with
   $\nu_1^{{\rm D},g}$, the averaged number of fission--emitted delayed
   neutron produced in the precursor group 1.}
\OptRealEnr
  {NUSIGF02\blank{4}}{$G$}{$N_{\rm del}\ge 2$}{b}
  {$\nu\sigma_{{\rm f},2}^{{\rm D},g}$: The product of $\sigma_{\rm f}^{g}$, the fission cross section with
   $\nu_2^{{\rm D},g}$, the averaged number of fission--emitted delayed
   neutron produced in the precursor group 2.}
\OptRealEnr
  {CHI01\blank{7}}{$G$}{$N_{\rm del}\ge 1$}{1}
  {$\chi^{{\rm D},g}_1$: Delayed fission spectrum,
   normalized to one, for the delayed fission neutrons in
   precursor group 1.}
\OptRealEnr
  {CHI02\blank{7}}{$G$}{$N_{\rm del}\ge 2$}{1}
  {$\chi^{{\rm D},g}_2$: Delayed fission spectrum,
   normalized to one, for the delayed fission neutrons in
   precursor group 2.}
\end{DescriptionEnregistrement}

\vskip 0.2cm

In cases where the /isotope/ directory is produced by the edition module, some
depletion-related information may be available in this directory, in order to facilitate
subsequent data processing. This information is described in
Table~\ref{tabl:tabiso3}.

\begin{DescriptionEnregistrement}{Depletion-related information}{7.5cm}
\label{tabl:tabiso3}
\OptRealEnr
  {MEVG\blank{8}}{$1$}{$N_d \ge 1$}{MeV}
  {Energy in MeV produced by radiative capture. $N_d$ is defined in \Sect{microlibdir}.}
\OptRealEnr
  {MEVF\blank{8}}{$1$}{$N_d \ge 1$}{MeV}
  {Energy in MeV produced by fission.}
\OptRealEnr
  {DECAY\blank{7}}{$1$}{$N_d \ge 1$}{10$^{-8}$ s$^{-1}$}
  {Radioactive decay constant}
\OptRealEnr
  {YIELD\blank{7}}{$G+1$}{$N_d \ge 1$}{1}
  {Fission fragment yield per energy group. The first value is the average yield
  over all the energy spectrum. This record is given only for fission fragments.}
\OptIntEnr
  {PIFI\blank{8}}{$N_{\rm dfi}$}{$N_{\rm dfi} \ge 1$}
  {Position in {\tt ISOTOPESUSED} of the mother fissile isotopes. This record is
  given only for fission fragments.}
\OptRealEnr
  {PYIELD\blank{6}}{$N_{\rm dfi}$}{$N_{\rm dfi} \ge 1$}{1}
  {Fission product yield per fissile isotope. This record is given only for
  fission fragments.}
\end{DescriptionEnregistrement}

\vskip 0.2cm

We will first consider the more usual case where constant vector reactions are
stored on the isotopic directory. A typical example of the microscopic cross
section directory may be:

\begin{DescriptionEnregistrement}{Example of isotopic vector reaction records}{7.0cm}
\label{tabl:tabiso4}
\RealEnr
  {NTOT0\blank{7}}{$G$}{b}
  {The $\phi$--weighted multigroup total cross section $\sigma_0^{g}$}
\RealEnr
  {TRANC\blank{7}}{$G$}{b} 
  {The multigroup transport correction $\sigma_{tc}^{g}$}
\RealEnr
  {NUSIGF\blank{6}}{$G$}{b} 
  {The product of $\sigma_{f}^{g}$, the multigroup fission cross section with
   $\nu^{g}$, the steady-state number of neutron produced per fission,
   $\nu\sigma_{f}^{{\rm ss},g}$}
\RealEnr
  {NFTOT\blank{7}}{$G$}{b}
  {The multigroup fission cross section $\sigma_{f}^{g}$}
\OptRealEnr
  {CHI\blank{9}}{$G$}{$G_{\rm chi}=0$}{}
  {The multigroup energy spectrum of the neutron emitted by fission $\chi^{g}$}
\OptRealEnr
  {CHI--01\blank{5}}{$G$}{$G_{\rm chi}\ge 1$}{}
  {The first energy-dependent multigroup energy spectrum of the neutron emitted by fission $\chi^{g,1}$}
\OptRealEnr
  {CHI--02\blank{5}}{$G$}{$G_{\rm chi}\ge 2$}{}
  {The second energy-dependent multigroup energy spectrum of the neutron emitted by fission $\chi^{g,2}$}
\OptRealEnr
  {CHI--03\blank{5}}{$G$}{$G_{\rm chi}\ge 3$}{}
  {The third energy-dependent multigroup energy spectrum of the neutron emitted by fission $\chi^{g,3}$}
\OptRealEnr
  {CHI--04\blank{5}}{$G$}{$G_{\rm chi}\ge 4$}{}
  {The fourth energy-dependent multigroup energy spectrum of the neutron emitted by fission $\chi^{g,4}$}
\RealEnr
  {NG\blank{10}}{$G$}{b} 
  {The multigroup neutron capture cross section $\sigma_{c}^{g}$}
\RealEnr
  {H-FACTOR\blank{4}}{$G$}{J b}
  {Energy production coefficients $H^{g}$ (product of each microscopic cross section
  times the energy emitted by this reaction).}
\OptRealEnr
  {C-FACTOR\blank{4}}{$G$}{*}{electron b}
  {Charge deposition coefficients $C^{g}$ (product of each microscopic cross section
  times the charge deposed by this reaction). Information provided if {\tt PARTICLE}$=${\tt B}, {\tt C} or {\tt P}.}
\RealEnr
  {N2N\blank{9}}{$G$}{b} 
  {The multigroup cross section
   $\sigma_{(n,2n)}^{g}$ for the reaction 
   $^{A}_{Z}X+n \to ^{A-1}_{Z}X+2n$}
\RealEnr
  {N3N\blank{9}}{$G$}{b} 
  {The multigroup cross section
   $\sigma_{(n,3n)}^{g}$ for the reaction 
   $^{A}_{Z}X+n \to ^{A-2}_{Z}X+3n$}
\RealEnr
  {N4N\blank{9}}{$G$}{b} 
  {The multigroup cross section
   $\sigma_{(n,4n)}^{g}$ for the reaction 
   $^{A}_{Z}X+n \to ^{A-3}_{Z}X+4n$}
\RealEnr
  {NP\blank{10}}{$G$}{b}
  {The multigroup cross section
   $\sigma_{(n,p)}^{g}$ for the reaction 
   $^{A}_{Z}X+n \to ^{A}_{Z-1}X+p$}
\RealEnr
  {NA\blank{10}}{$G$}{b}
  {The multigroup cross section
   $\sigma_{(n,\alpha)}^{g}$ for the reaction 
   $^{A}_{Z}X+n \to ^{A-3}_{Z-2}X+\alpha$ }
\RealEnr
  {NGOLD\blank{7}}{$G$}{} 
  {The multigroup Goldstein-Cohen parameters as recovered from {\tt GIR} array in main \dir{microlib} directory
   $\lambda^{g}$}
\RealEnr
  {NWT0\blank{8}}{$G$}{s$^{-1}$cm$^{-2}$}
  {The multigroup neutron flux spectrum $\phi_{w}^{g}$} 
\RealEnr
  {STRD\blank{8}}{$G$}{b}
  {The multigroup transport cross section 
   homogenized over all directions
   $\sigma_{\rm strd}^{g}$}
\RealEnr
  {STRD-X\blank{6}}{$G$}{b} 
  {The $x-$directed multigroup transport cross
   section $\sigma_{{\rm strd},x}^{g}$}
\RealEnr
  {STRD-Y\blank{6}}{$G$}{b} 
  {The $y-$directed multigroup transport cross
   section $\sigma_{{\rm strd},y}^{g}$}
\RealEnr
  {STRD-Z\blank{6}}{$G$}{b}
  {The $z-$directed multigroup transport cross
   section $\sigma_{{\rm strd},z}^{g}$}
\RealEnr
  {OVERV\blank{7}}{$G$}{cm$^{-1}$s}
  {The average of the inverse neutron velocity \hbox{$<1/v>_{m}^g$}}
\RealEnr
  {NTOT1\blank{7}}{$G$}{b}
  {The ${\cal J}$--weighted multigroup total cross section $\sigma_1^{g}$}
\RealEnr
  {NWT1\blank{8}}{$G$}{s$^{-1}$cm$^{-2}$}
  {The multigroup fundamental current spectrum ${\cal J}_{w}^{g}$} 
\RealEnr
  {NWAT0\blank{7}}{$G$}{1}
  {The multigroup neutron adjoint flux spectrum $\phi_{w}^{*g}$} 
\RealEnr
  {NWAT1\blank{7}}{$G$}{1}
  {The multigroup fundamental adjoint current spectrum ${\cal J}_{w}^{*g}$} 
\end{DescriptionEnregistrement}

\vskip 0.2cm

We can also use this isotopic directory to store time dependent cross sections in the form of a power series expansion:
\begin{equation}
    v_{k}^{g}(t)=\sum_{i=0}^{I} v_{k,i}^{g} t^{i}
\label{eq:TimeSerie}
\end{equation}
where the presence of these various terms is specified using $\kappa_{k}$.
Note that the last three characters of each of the records in Table~\ref{tabl:tabiso4} correspond to the extension $\mathsf{EXT}$=\verb*|'   '| that is
associated with term $i=0$ in the power series expansion for the cross sections (see
\Eq{TimeSerie}). For $i=1, 2$, the extension takes successively the value $\mathsf{EXT}$=\verb*|'LIN'| and $\mathsf{EXT}$=\verb*|'QUA'|.
For example, if one considers the total cross section and assumes that $F_{i}(\kappa_{1})=1$ for $i=0,2$, then this implies the presence
of the following additional records in the \dir{isotope}:

\begin{DescriptionEnregistrement}{Additional total cross section records for $I=2$}{6.0cm}
\RealEnr
  {TOTAL\blank{4}LIN}{$G$}{d$^{-1}$b}
  {array  $v_{1,1}^{g}=\Delta\sigma^{g}$ containing the first order coefficients in the power series expansion for the multigroup total
cross section}
\RealEnr
  {TOTAL\blank{4}QUA}{$G$}{d$^{-2}$b}
  {array  $v_{1,2}^{g}=\Delta^{2}\sigma^{g}$ containing the second order coefficients in the power series expansion for the multigroup
total cross section}
\end{DescriptionEnregistrement}

\vskip 0.2cm

The multigroup scattering cross section matrix, which gives the probability for a
neutron in group $h$ to appear in group $g$ after a collision with this isotope
is represented by the form:
  \begin{displaymath}
    \sigma_{s}^{h\to g}(\vec{\Omega}\to\vec{\Omega}')
      =\sum_{\ell=0}^{L}{{2\ell+1}\over{4\pi}} P_{\ell}(\vec{\Omega}\cdot\vec{\Omega}')
    \sigma_{\ell}^{h\to g}
      =\sum_{\ell=0}^{L}\sum_{m=-\ell}^{\ell}
      Y_{\ell}^{m}(\vec{\Omega})Y_{\ell}^{m}(\vec{\Omega}')\sigma_{\ell}^{h\to g}
  \end{displaymath}
using a spherical harmonic series expansion to order $L-1$. Assuming these 
spherical harmonic are orthonormalized, namely:
  \begin{displaymath}
    \int_{4\pi} d^{2}\Omega \ Y_{\ell}^{m}(\vec{\Omega}) Y_{l'}^{m'}(\vec{\Omega})=
    \delta_{m,m'}\delta_{\ell,\ell'}
  \end{displaymath}
we can define $\sigma_{\ell}^{h\to g}$ in terms of $\sigma_{s}^{h\to
g}(\vec{\Omega}\to\vec{\Omega}')$ using the following integral:
  \begin{displaymath}
    \sigma_{\ell}^{h\to g}
      =\int_{4\pi}d^{2}\Omega \ \sigma_{s}^{h\to g}(\vec{\Omega}\to\vec{\Omega}')
     P_{\ell}(\vec{\Omega}\cdot\vec{\Omega}')
  \end{displaymath}
Note that this definition of $\sigma_{\ell}^{h\to g}$ is not unique and some authors
include the factor $2l+1$ directly in different angular moments of the 
scattering cross section.

\vskip 0.2cm

Here instead of storing on these $G\times G$
matrices $\sigma_{\ell}^{h\to g}$, a vector which contains a compress form for this
matrix will be considered. This choice is justified by the fact that the number
of energy groups which will lead to scattering in a specific group is generally
relatively small compared to the total number of groups in the library and that
these groups are clustered around the final energy group.
Here we will first define two different integer vectors $n_{\ell}^{g}$ and
$h_{\ell}^{g}$ for each order in the scattering cross section and for each final
energy group $g$ which will contain respectively the number of
successive initial energy groups for which the scattering cross section does
not vanish and the maximum energy group number for which scattering to the
final group $g$ does not vanishes. Accordingly, for a scattering cross section
of the form:

\begin{center}
\begin{tabular}{c||cccc}
$\sigma_{0}^{h\to g}$ &$g=1$   & $g=2$   & $g=3$   & $g=4$    \\ \hline\hline
$h=1$                 & $a_{1}$ & $a_{2}$ & 0       & 0        \\
$h=2$                 & 0       & $a_{3}$ & $a_{4}$ & $a_{5}$  \\
$h=3$                 & 0       & $a_{6}$ & $a_{7}$ & 0        \\
$h=4$                 & 0       & $a_{8}$ & 0       & $a_{9}$  \\ \hline\hline
$h_{0}^{g}$           & 1       & 4       & 3       & 4       \\
$n_{0}^{g}$           & 1       & 4       & 2       & 3        \\
\end{tabular}
\end{center}

The compress scattering matrix will then contain the following information:
  \begin{displaymath}
    \sigma_{\ell,c}=\left(\sigma_{\ell}^{h^{1}\to 1},\sigma_{\ell}^{h^{1}-1\to 1},
    \ldots,\sigma_{\ell}^{h^{1}-n_{1}+1\to 1},\sigma_{\ell}^{h^{2}\to
    2},\ldots,\sigma_{\ell}^{h^{G}-n_{G}+1\to G}\right)
  \end{displaymath}
which for the example above leads to
  \begin{displaymath}
    \sigma_{\ell,c}=\left(a_{1},a_{8},a_{6},
                     a_{3},a_{2},a_{7},a_{4},a_{9},0,a_{5}\right)
  \end{displaymath}
As a result $\sigma_{\ell}^{h\to g}$ can be
reconstructed using 
\begin{displaymath}
\sigma_{\ell}^{h\to g} = \left\{
\begin{array}{lll}
0 & \textrm{if} & h > h_{\ell}^{g}\\
0 & \textrm{if} & h < h_{\ell}^{g}-n_{\ell}^{g}+1\\
\sigma_{\ell,c}^{k} & \textrm{otherwise} & k=\sum_{h=1}^{g-1} n_{\ell}^{h} +
h_{\ell}^{g}-h+1
\end{array} \right.
\end{displaymath}

Finally, we will also save the total scattering cross section vector of order
$\ell$ which is defined as 
  \begin{displaymath}
    \sigma_{\ell,s}^{h}=\sum_{g=1}^{G} \sigma_{\ell}^{h\to g}
  \end{displaymath}
In the case where only the order $\ell=0$ moment of scattering cross section is non
vanishing (isotropic scattering) the following records can be found on the
isotopic directory.

\begin{DescriptionEnregistrement}{Optional scattering records}{7.0cm}
\label{tabl:tabiso5}
\RealEnr
 {SIGS00\blank{6}}{$G$}{b}
 {The isotropic component ($\ell=0$) of the multigroup total scattering cross
  section
  $\sigma_{0,s}^{g}$}
\IntEnr
  {IJJS00\blank{6}}{$G$}
  {Highest energy group number for which 
   the isotropic component of the scattering cross section to group $g$ does not
   vanish, $h_{0}^{g}$}
\IntEnr
  {NJJS00\blank{6}}{$G$}
  {Number of energy groups for which 
   the isotropic component of the scattering cross section to group $g$ does not
   vanish, $n_{0}^{g}$}
\RealEnr
  {SCAT00\blank{6}}{$\sum_{g=1}^{G} n_{0}^{g}$}{b}
  {Compressed isotropic component of the scattering matrix
   $\sigma_{0,c}^{k}$}
\OptDirVar
  {\listedir{subiso}}{$I_{\rm part}\ge 1$}
  {Set of sub-directories containing scattering information towards a companion particle. \listedir{subiso}
  is the name of the companion particle (set to {\tt N}, {\tt G}, {\tt B}, {\tt C} or {\tt P}). This information
  is used to construct coupled sets of cross sections.}
\end{DescriptionEnregistrement}

If the scattering cross section is
expanded to order $L>1$ in Legendre polynomials, additional set of scattering
records similar to those described above will be presentin the cross section directory.
The first four characters and last 6 characters in the names of
these records will again be identical to those described above while character 5
and 6 will differ from level to level. For example, the order
$\ell=5$ compressed scattering matrix will be identified by
\texttt{SCAT05\blank{6}} while for order 
$\ell=50$ we will use \texttt{SCAT50\blank{6}}.

\vskip 0.2cm

The {\tt STRD} cross sections are normalized in such a way to permit the
calculation of a diffusion coefficient using the following formula:

\begin{equation}
D^g={\displaystyle 1\over\displaystyle 3 \ \sum_i N_i \ \sigma_{{\rm strd},i}^g}
\end{equation}

\noindent where $N_i$ is the isotopic density of isotope $i$ and $\sigma_{{\rm strd},i}^g$
is the {\tt STRD} cross section of isotope $i$ in energy group $g$. The sum is
performed over {\sl all} isotopes present in the mixture. The {\tt STRD} cross
sections for isotope $i$ are defined as

\begin{eqnarray}
\sigma_{{\rm strd},i}^g&=&{1\over (\mu^g)^2} \ {\left<\phi \right>_g \over 3
\left<(\Sigma_1-\Sigma_{\rm s1}){\cal J}\right>_g} \ (\sigma_{1,i}^g-\sigma_{{\rm
s1},i}^g) \ \
\ {\rm if \ a \ streaming \ model \ is \ used} \\
&=&{1\over (\mu^g)^2} \ {\left<\phi \right>_g^2 \over 3 \left< D \phi \right>_g
\left<(\Sigma_0-\Sigma_{\rm s1})\phi\right>_g} \ (\sigma_{0,i}^g-\sigma_{{\rm
s1},i}^g) \ \
\ {\rm if \ no \ streaming \ model \ used}
\end{eqnarray}
\noindent where

\begin{description}
\item [$\phi^g$] fundamental flux
\item [${\cal J}^g$] fundamental current
\item [$\mu^g$] SPH equivalence factor
\item [$\Sigma_0^g$] $\phi$--weighted macroscopic total cross section of the
mixture
\item [$\Sigma_1^g$] ${\cal J}$--weighted macroscopic total cross section of the
mixture
\item [$\Sigma_{\rm s1}^g$] macroscopic $P_1$ scattering cross section of the
mixture (${\cal J}$--weighted
if a streaming model is used; $\phi$--weighted if no streaming model used)
\item [$D^g$] diffusion coefficient
\item [$\sigma_{0,i}^g$] $\phi$--weighted microscopic total cross section for
isotope $i$
\item [$\sigma_{1,i}^g$] ${\cal J}$--weighted microscopic total cross section for
isotope $i$
\item [$\sigma_{{\rm s1}.i}^g$] microscopic $P_1$ scattering cross section for
isotope $i$ (${\cal J}$--weighted
if a streaming model is used; $\phi$--weighted if no streaming model used)
\end{description}

\vskip 0.2cm

On the other hand the so-called directional cross
section {\tt STRD\blank{1}X}, {\tt STRD\blank{1}Y}
and {\tt STRD\blank{1}Z} are obtained in such a way that

\begin{equation}
D_k^g={\displaystyle 1\over\displaystyle 3 \ \sum_i N_i \ \sigma_{{\rm strd},k,i}^g}
\ ; \ \ \ k=x,\ y \ {\rm or} \ z \ \ \ .
\end{equation}

\vskip 0.2cm

For example, for an isotope with only total and scattering cross sections, we will find the
following records on the cross section directory.

\begin{DescriptionEnregistrement}{Example of cross section records}{7.5cm}
\RealEnr
  {NTOT0\blank{7}}{$G$}{b}
  {The multigroup total cross section $\sigma^{g}$}
\RealEnr
 {SIGS00\blank{6}}{$G$}{b}
 {The isotropic component ($\ell=1$)of the multigroup total scattering cross
  section
  $\sigma_{0,s}^{g}$}
\IntEnr
  {IJJS00\blank{6}}{$G$}
  {Highest energy group number for which 
   the isotropic component of the scattering cross section to group $g$ does not
   vanishes, $h_{0}^{g}$}
\IntEnr
  {IJJS00\blank{3}}{$G$}
  {Highest energy group number for which the first order perturbation in
   the isotropic component of the scattering cross section to group $g$ does not
   vanishes, $h_{0,1}^{g}$}
\IntEnr
  {NJJS00\blank{6}}{$G$}
  {Number of energy groups for which 
   the isotropic component of the scattering cross section to group $g$ does not
   vanishes, $n_{0}^{g}$}
\RealEnr
  {SCAT00\blank{6}}{$\sum_{g=1}^{G} n_{0}^{g}$}{b}
  {Compressed isotropic component of the scattering matrix
   $\sigma_{0,c}^{k}$}
\RealEnr
 {SIGS01\blank{6}}{$G$}{b}
 {The linearly anisotropic component ($\ell=1$) 
  of the multigroup total scattering cross section 
  $\sigma_{1,s}^{g}$}
\IntEnr
  {IJJS01\blank{6}}{$G$}
  {Highest energy group number for which 
   the linearly anisotropic component of the scattering cross section
   to group $g$ does not vanishes,
   $h_{1}^{g}$}
\IntEnr
  {NJJS01\blank{6}}{$G$}
  {Number of energy groups for which 
   the linearly anisotropic component of the scattering cross section 
   to group $g$ does not vanishes,
   $n_{1}^{g}$}
\RealEnr
  {SCAT01\blank{6}}{$\sum_{g=1}^{G} n_{1}^{g}$}{b}
  {Compressed linearly anisotropic component of the scattering matrix
   $\sigma_{1,c}^{k}$}
\end{DescriptionEnregistrement}

Note that most of these cross sections are not required to perform a cell
calculation. In fact, in a typical transport calculation, only
$\sigma^{g}$, $\sigma_{tc}^{g}$, $\nu\sigma_{f}^{g}$, $\chi^{g}$ and the
isotropic and linearly anisotropic scattering matrix are
used. For burnup calculations, depending on the depletion chain prescribed,
the following cross sections may be required:
$\sigma_{f}^{g}$, $\sigma_{c}^{g}$, $\sigma_{(n,2n)}^{g}$, $\sigma_{(n,3n)}^{g}$,
$\sigma_{(n,4n)}^{g}$, $\sigma_{(n,p)}^{g}$, $\sigma_{(n,\alpha)}^{g}$.
Finally, when editing isotopic cross sections, all the cross sections types in
the library will be processed.

\vskip 0.15cm

A final note on the use of the transport correction and the homogenized and
directional transport cross section. In DARGON, the transport correction cross
section is used to correct the total and isotropic scattering cross
section using the relations
\begin{eqnarray*}
\sigma_{c}^{g}       &=& \sigma^{g}         -\sigma_{tc}^{g}\\
\sigma_{c,0}^{g\to g}&=& \sigma_{0}^{g\to g}-\sigma_{tc}^{g}
\end{eqnarray*}

\goodbreak

\subsubsection{The probability table directory {\tt PT-TABLE} in \dir{isotope}}\label{sect:pt-table}

Physical probability tables ($I_{\rm proc}=1$) are obtained from a least-square fit of the
self-shielded cross sections against dilution. Mathematical probability tables ($I_{\rm proc}\ge 3$) are obtained from
Autolib data using the CALENDF formalism.
Resonance spectrum expansion (RSE) information ($I_{\rm proc}=6$) is obtained from Autolib data using a singular value decomposition (SVD) of the
form $\shadowA=\shadowU \shadowW \shadowV^\top$ where
\begin{description}
\item[$\shadowA$:] snapshot flux matrix of size $N_{{\rm ufg},g}\times N_{\rm dil}$ recovered from the Draglib or Apollo2 file,
\item[$\shadowU$:] first orthogonal SVD matrix of size $N_{{\rm ufg},g}\times K_g$,
\item[$\shadowW$:] singular-value diagonal matrix of size $K_g\times K_g$,
\item[$\shadowV$:] second orthogonal SVD matrix of size $N_{\rm dil}\times K_g$
\end{description}
\noindent where $N_{{\rm ufg},g}$ is the number of ultra-fine groups in coarse group $g$, $N_{\rm dil}$ is the number of snapshot ultra-fine group
flux distributions in coarse group $g$ (corresponding to the number of dilutions) and $K_g$ is the SVD rank in coarse group $g$.

\begin{DescriptionEnregistrement}{Probability tables or RSE tables in \dir{isotope}}{7.5cm}
\OptDirlEnr
  {GROUP-PT\blank{4}}{$G$}{$I_{\rm proc}\ne 6$}
  {List of energy-group sub-directories. Each component of the list is a directory containing
  the probability-table information associated with a specific group. See table~\ref{table:pt}.}
\OptDirlEnr
  {GROUP-RSE\blank{3}}{$G$}{$I_{\rm proc}= 6$}
  {List of energy-group sub-directories. Each component of the list is a directory containing
  the resonance spectrum expansion information associated with a specific coarse group. See table~\ref{table:rse}.}
\OptDirlEnr
  {\listedir{isotope2}}{$n_{\rm pos}$}{$I_{\rm proc}= 6$}
  {List of group-dependent $K_g \times L_h$ microscopic scattering double precision matrices correlated to the base points in microscopic total cross sections in group $g$ and to
  the scattering sources from {\sl isotope2} in group $h$. Here, $n_{\rm pos}$ is the total number of scattering double precision matrices taking into account self-scattering and out-of-group scattering.
  $L_h$ is the order of the probability table for {\sl isotope2} in group $h$. Record \listedir{isotope} (with {\sl isotope2} $=$ {\sl isotope} and $L_h=K_h$) is always present if $I_{\rm proc}= 6$.}
\IntEnr
  {NOR\blank{9}}{$G$}
  {Order $K_g$ of the probability table or of the resonance spectrum expansion tables in each energy group $g$.
  If $I_{\rm proc}= 6$, the RSE rank $K_g \le N_{\rm dil}$ where $N_{\rm dil}$ is the number of dilutions.}
\IntEnr
  {NDEL\blank{8}}{$1$}
  {Number of delayed neutron precursor groups for this resonant isotope.}
\OptRealEnr
  {SVD-EPS\blank{5}}{$1$}{$I_{\rm proc}= 6$}{~}
  {Rank accuracy of the SVD.}
\OptRealEnr
  {NJJS00\blank{6}}{$G$}{$I_{\rm proc}= 6$}{~}
  {Bandwidth $n_{{\rm njj},g}$ of records \listedir{isotope2}. $n_{\rm pos}=\sum_g n_{{\rm njj},g}$.}
\end{DescriptionEnregistrement}

\vskip -0.3cm

\begin{DescriptionEnregistrement}{Group-dependent non-RSE directories in \dir{isotope}}{7.5cm}\label{table:pt}
\RealEnr
  {PROB-TABLE\blank{2}}{$12,N_{\rm part}$}{~}
  {Probability tables. $N_{\rm part}$ is the total number of reactions
   represented by probability tables. 12 is the maximum allowed order of a
   probability table.}
\OptRealEnr
  {SIGQT-SIGS\blank{2}}{$K_g$}{$I_{\rm proc}=4$}{b}
  {Probability table in secondary slowing-down cross section.}
\OptRealEnr
  {SIGQT-SLOW\blank{2}}{$K_g,K_g$}{$I_{\rm proc}=4$}{b}
  {Slowing-down correlated weight matrix.}
\OptRealVar
  {\listedir{isotope2}}{$K_g,L_g$}{*}{1}
  {Set of records, each containing the correlated weights
  between the current total xs and the total xs of {\sl isotope2}. $L_g$ is the
  order of the probability table for {\sl isotope2}. (*) This data is optional
  and is provided only if $I_{\rm proc}\ge 3$ and if the mutual self-shielding
  effect is to be taken into account.}
\IntEnr
  {ISM-LIMITS\blank{2}}{$2,L$}
  {Minimum (index 1) and maximum (index 2) secondary group for each Legendre
   order of the scattering matrices}
\end{DescriptionEnregistrement}

\vskip -0.4cm

\begin{DescriptionEnregistrement}{Group-dependent RSE directories in \dir{isotope}}{7.5cm}\label{table:rse}
\DbleEnr
  {RSE-TABLE\blank{3}}{$N_{\rm part}, K_g$}{~}
  {Resonance spectrum expansion (RSE) table $\shadowP$. $N_{\rm part}$ is the total number of flux and reactions
  represented by RSE tables and $K_g$ is the RSE rank (equal to the number of base points). Here, $\shadowP=\shadowS\shadowV \shadowW^{-1}$
  where $\shadowS$ is a $N_{\rm part} \times N_{\rm dil}$ double precision matrix containing dilution-dependent
  homogeneous flux and effective cross sections recovered from the Draglib or Apollo2 file.}
\DbleEnr
  {SIGT\_V\blank{6}}{$K_g$}{~}
  {Double precision vector corresponding to the base points in microscopic total cross sections. These
  values are the eigenvalues of the linear transformation.}
\DbleEnr
  {XI\_V\blank{8}}{$K_g$}{~}
  {Double precision vector $\Xi_{k,g}=\sum_m U_{m,k}$.}
\DbleEnr
  {GAMMA\_V\blank{5}}{$K_g$}{~}
  {Double precision vector $\gamma_{k,g}=\sum_m \Delta u_g^{(m)}\, U_{m,k}$.}
\OptDbleVar
  {\listedir{isotope2}}{$K_g,K_g$}{*}{1}
  {Set of matrices representing the correlation of microscopic total cross sections between {\sl isotope2} and {\sl isotope} in group $g$. (*) This data is optional
  and is provided only if {\sl isotope2} is resonant and if {\sl isotope2} $\neq$ {\sl isotope}.}
\IntEnr
  {ISM-LIMITS\blank{2}}{$2,L$}
  {Minimum (index 1) and maximum (index 2) secondary group for each Legendre
   order of the scattering matrices}
\end{DescriptionEnregistrement}
\eject
