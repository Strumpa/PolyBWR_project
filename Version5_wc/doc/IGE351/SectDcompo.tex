\section{Contents of a \dir{multicompo} directory}\label{sect:multicompodir}

This object, shown in \Fig{multicompo}, is used to collect information gathered
from many DRAGON {\sl elementary calculations} performed under various conditions.
Each elementary calculation is characterized by a tuple of {\sl global} and {\sl local parameters}.
These parameters are of different types, depending on the nature of the
study under consideration: type of assembly, power, temperature in a mixture,
concentration of an isotope, time, burnup or exposure rate in a depletion calculation,
etc. Each step of a depletion calculation represents an elementay calculation.
The {\sc multicompo} object is often presented as a {\sl multi-parameter reactor database}.

\begin{figure}[h!]
\begin{center} 
\epsfxsize=9cm
\centerline{ \epsffile{compo.eps}}
\parbox{15cm}{\caption{Representation of a {\sc multicompo} 
object.}\label{fig:multicompo}}   
\end{center}  
\end{figure}

\vskip 0.2cm

The {\sc multicompo} object contains table-of-content information apart from a list
of {\sl homogenized mixture} directories. Each {\sl homogenized mixture}
directory contain a list of {\sl elementary calculation} directories whose
components are embedded {\sc microlib} objects containing the useful data.
The localization of an elementary calculation
is done using a tuple of global and local parameters. The elementary calculation indices are
stored in a tree with the number of levels equal to the number of global {\sl and} local parameters.
An example of a tree with three parameters is shown in \Fig{ctree}. Each node
of this tree is associated with the index of the corresponding global parameter (positive index) or
local parameter (negative index) and with the
reference to the daughter nodes if they exist. The number if leafs is equal to the number
of nodes for the last (third) parameter and is equal to the number of elementary
calculations stored in the {\sc multicompo} object. The index of each elementary calculation is
therefore an attribute of each leaf.

\begin{figure}[h!]  
\begin{center} 
\epsfxsize=12cm
\centerline{ \epsffile{tree.eps}}
\parbox{14cm}{\caption{Parameter tree in a {\sc multicompo} object}\label{fig:ctree}}   
\end{center}  
\end{figure}

\vskip 0.1cm

In each embedded {\sc microlib} directory, the {\tt COMPO:} module recover
cross sections for a number of {\sl particularized isotopes} and {\sl macroscopic
sets} named {\tt '*MAC*RES'}, a collection of isotopic cross sections weighted by
isotopic number densities. Other information is also recovered: multigroup neutron
fluxes, isotopic number densities, fission spectrum and a set
of {\sl local parameters}. The local parameters are values that characterize each
homogenized mixture: local power, burnup, exposure rate, etc.

\subsection{State vector content for the \dir{multicompo} data structure}\label{sect:multicompostate}

The dimensioning parameters for this data structure, which are stored in the state vector
$\mathcal{S}^{\rm cpo}$, represent:

\begin{itemize}
\item The number of homogenized mixtures $M_{m}=\mathcal{S}^{\rm cpo}_{1}$. $=0$
for an empty {\sc multicompo} object.
\item The number of groups ${G}=\mathcal{S}^{\rm cpo}_{2}$ 
\item The exact number of elementary calculations in the {\sc multicompo} ${N_{\rm cal}}=\mathcal{S}^{\rm cpo}_{3}$ 
\item The maximum number of elementary calculations in the {\sc multicompo} ${N_{\rm max}}=\mathcal{S}^{\rm cpo}_{4}$ 
\item The number of global parameters $N_{\rm glob}=\mathcal{S}^{\rm cpo}_{5}$ 
\item The number of local parameters $N_{\rm loc}=\mathcal{S}^{\rm cpo}_{6}$ 
\item The number of global parameters linked with isotopes $N_{\rm gl\_iso}=\mathcal{S}^{\rm cpo}_{7}$
\item The number of global parameters linked with {\sc microlib} objects $N_{\rm gl\_bib}=\mathcal{S}^{\rm cpo}_{8}$
\item The number of local parameters linked with isotopes $N_{\rm loc\_iso}=\mathcal{S}^{\rm cpo}_{9}$
\item The number of lines of comment $N_{\rm doc}=\mathcal{S}^{\rm cpo}_{10}$
\item Geometry index $L_{\rm geo}=\mathcal{S}^{\rm cpo}_{11}$:
\begin{displaymath}
L_{\rm geo} = \left\{
\begin{array}{ll}
0 & \textrm{the geometries are not available}\\
1 & \textrm{calculation--ordered homogenized geometries are available.}\\
\end{array} \right.
\end{displaymath}
\item Version identificator $\mathcal{S}^{\rm cpo}_{12}$. Currently equal to {\tt 2006}. This value
  will change if the {\sc multicompo} specification is to be modified in the future.
\item The number of user-defined particularized isotopes $\mathcal{S}^{\rm cpo}_{13}$.
\item Group form factor index $N_{\rm gff}=\mathcal{S}^{\rm cpo}_{14}$:
\begin{displaymath}
N_{\rm gff} = \left\{
\begin{array}{ll}
-1 & \textrm{the group form factors will be recovered from an {\sc edition} object}\\
0 & \textrm{the group form factors are not processed}\\
>0 & \textrm{number of group form factors per energy group. The information is stored in}\\
 & \textrm{a {\sc macrolib} subdirectory in the {\sc multicompo}.}\\
\end{array} \right.
\end{displaymath}
\item Physical albedo index $N_{\rm alb}=\mathcal{S}^{\rm cpo}_{15}$:
\begin{displaymath}
N_{\rm alb} = \left\{
\begin{array}{ll}
-1 & \textrm{the physical albedos will be recovered from an {\sc edition} object}\\
0 & \textrm{the physical albedos are not processed}\\
>0 & \textrm{number of physical albedos per energy group. The information is stored in}\\
 & \textrm{a {\sc macrolib} subdirectory in the {\sc multicompo}.}\\
\end{array} \right.
\end{displaymath}
\item Discontinuity factor index $I_{\rm df}=\mathcal{S}^{\rm cpo}_{16}$:
\begin{displaymath}
I_{\rm df} = \left\{
\begin{array}{ll}
-1 & \textrm{discontinuity factor information will be recovered from an {\sc edition} object}\\
0 & \textrm{no discontinuity factor information}\\
1 & \textrm{multigroup boundary current information is available}\\
2 & \textrm{discontinuity factor information (see \Sect{macroADF}) is available.}
\end{array} \right.
\end{displaymath}
\end{itemize}

\subsection{The main \dir{multicompo} directory}\label{sect:cpodirmain}

On its first level, the
following records and sub-directories will be found in the \dir{multicompo} directory:

\begin{DescriptionEnregistrement}{Main records in \dir{multicompo}}{8.0cm}
\CharEnr
  {SIGNATURE\blank{3}}{$*12$}
  {Signature of the data structure ($\mathsf{SIGNA}=${\tt L\_MULTICOMPO}).}
\DirVar
  {\listedir{namdir}}
  {Set of sub-directories, each of them containing an independent {\sc multicompo}
  structure}
\end{DescriptionEnregistrement}

\begin{DescriptionEnregistrement}{Main records and sub-directories in \listedir{namdir}}{8.0cm}
\IntEnr
  {STATE-VECTOR}{$40$}
  {Vector describing the various parameters associated with this data structure $\mathcal{S}^{\rm cpo}_{i}$,
  as defined in \Sect{multicompostate}.}
\OptDirEnr
  {DEPL-CHAIN\blank{2}}{$*$}
  {directory containing the \dir{depletion} associated with directory \dir{namdir}, following
  the specification presented in \Sect{microlibdirdepletion}.}
\OptCharEnr
  {COMMENT\blank{5}}{$(N^{\rm cpo}_{10})*80$}{$\mathcal{S}^{\rm cpo}_{10}> 0$}
  {User--defined comments about the data structure}
\OptCharEnr
  {NOMISP\blank{6}}{$(\mathcal{S}^{\tt cpo}_{13})*8$}{$\mathcal{S}^{\tt cpo}_{13}\ge 1$}
  {Names of the user-defined particularized isotopes.}
\DirEnr
  {GLOBAL\blank{6}}
  {Table--of--content for global parameter information: definition and tabulated values. The specification is presented in \Sect{cpodirarbre}.}
\OptDirEnr
  {LOCAL\blank{7}}{$\mathcal{S}^{\rm cpo}_{6}\ge 1$}
  {Table--of--content for local parameter information. The specification is presented in \Sect{cpodirlocal}.}
\OptDirlEnr
  {MIXTURES\blank{4}}{$\mathcal{S}^{\rm cpo}_1$}{$\mathcal{S}^{\rm cpo}_{1}\ge 1$}
  {List of {\sl homogenized mixture} directories. Each component of this list follows the specification
   presented in \Sect{homo_mix}.}
\OptDirlEnr
  {GEOMETRIES\blank{2}}{$\mathcal{S}^{\rm cpo}_4$}{$\mathcal{S}^{\rm cpo}_{11} = 1$}
  {List of homogenized {\sl geometry} directories. Each component of this list follows the specification
   presented in \Sect{geometrydirmain}.}
\end{DescriptionEnregistrement}

\subsection{The {\tt GLOBAL} sub-directory in \dir{multicompo}}\label{sect:cpodirarbre}

This directory is a table--of--content for the globals parameters. Its
specification follows:

\begin{DescriptionEnregistrement}{Contents of sub-directory {\tt GLOBAL} in
\dir{multicompo}}{7.0cm}
\label{tabl:tabglob}
\CharEnr
  {PARKEY\blank{6}}{$(\mathcal{S}^{\rm cpo}_{5})*12$}
  {User--defined key-words for the global parameters.}
\CharEnr
  {PARTYP\blank{6}}{$(\mathcal{S}^{\rm cpo}_{5})*4$}
  {Character identification for the types of global parameters (eg: {\tt TEMP}, {\tt CONC}, {\tt IRRA},
  etc.). Temperatures are given in Kelvin.}
\CharEnr
  {PARFMT\blank{6}}{$(\mathcal{S}^{\rm cpo}_{5})*8$}
  {Types for the global parameters (eg: {\tt REAL}, {\tt STRING} or {\tt INTEGER}).}
\OptCharEnr
  {PARCHR\blank{6}}{$(\mathcal{S}^{\rm cpo}_{7})*8$}{$\mathcal{S}^{\rm cpo}_{7}\ge 1$}
  {Isotope names linked to type--{\tt CONC} global parameters.}
\IntEnr
  {NVALUE\blank{6}}{$\mathcal{S}^{\rm cpo}_{5}$}
  {Number of specific values for a global parameters.}
\IntEnr
  {PARCAD\blank{6}}{$\mathcal{S}^{\rm cpo}_{5}+1$}
  {Address of the first element in array {\tt PARCHR}.}
\IntEnr
  {PARPAD\blank{6}}{$\mathcal{S}^{\rm cpo}_{5}+1$}
  {Address of the first element in array {\tt PARBIB}.}
\OptIntEnr
  {PARMIL\blank{6}}{$\mathcal{S}^{\rm cpo}_{8}$}{$\mathcal{S}^{\rm cpo}_{8}\ge 1$}
  {Mixture indices linked to type--{\tt TEMP} or --{\tt CONC} global parameters.}
\OptCharEnr
  {PARBIB\blank{6}}{$(\mathcal{S}^{\rm cpo}_{8})*12$}{$\mathcal{S}^{\rm cpo}_{8}\ge 1$}
  {{\sc microlib} names linked to type--{\tt TEMP} or --{\tt CONC} global parameters.}
\OptRealVar
  {\listedir{gvaldir}}{\tt NVALUE(J)}{*}{}
  {Set of real global parameter arrays. Each element of \listedir{gvaldir} contains a
   real array containing the tabulated values of the $J$--th global parameter.}
\OptIntVar
  {\listedir{gvaldir}}{\tt NVALUE(J)}{*}
  {Set of integer global parameter arrays.  Each element of \listedir{gvaldir} contains an
   integer array containing the tabulated values of the $J$--th global parameter.}
\OptCharVar
  {\listedir{gvaldir}}{$({\tt NVALUE(J)})*12$}{*}
  {Set of character$*12$ global parameter arrays. Each element of \listedir{gvaldir} contains a
   character$*12$ array containing the tabulated values of the $J$--th global parameter.}
\end{DescriptionEnregistrement}

Item \listedir{gvaldir} represents a set of $\mathcal{S}^{\rm cpo}_{5}$ real,
integer or character*$12$ records. The name of each
\listedir{gvaldir} record is a {\tt character*12} variable ({\tt text12})
composed using the following FORTRAN instruction:
$$
\mathtt{WRITE(}\mathsf{text12}\mathtt{,'(''pval'',I8.8)')}\: J
$$
where $J$ is the index of the global parameter with $1 \leq J \leq \mathcal{S}^{\rm cpo}_{5}$. The global
parameter values of type {\tt REAL} or {\tt INTEGER} are sorted.

\subsection{The {\tt LOCAL} sub-directory in \dir{multicompo}}\label{sect:cpodirlocal}

This directory is a table--of--content for the locals parameters. Its
specification follows:

\begin{DescriptionEnregistrement}{Contents of sub-directory {\tt LOCAL} in
\dir{multicompo}}{7.0cm}
\label{tabl:tabloc}
\CharEnr
  {PARKEY\blank{6}}{$(\mathcal{S}^{\rm cpo}_{6})*12$}
  {User--defined key-words for the local parameters.}
\CharEnr
  {PARTYP\blank{6}}{$(\mathcal{S}^{\rm cpo}_{6})*4$}
  {Character identification for the types of local parameters (eg: {\tt TEMP}, {\tt IRRA}, {\tt FLUB},
  etc.).}
\OptCharEnr
  {PARCHR\blank{6}}{$(\mathcal{S}^{\rm cpo}_{9})*8$}{$\mathcal{S}^{\rm cpo}_{9}\ge 1$}
  {Isotope names linked to type--{\tt CONC} local parameters.}
\IntEnr
  {PARCAD\blank{6}}{$\mathcal{S}^{\rm cpo}_{6}+1$}
  {Address of the first element in array {\tt PARCHR}.}
\end{DescriptionEnregistrement}

\subsection{The homogenized mixture sub-directory in \dir{multicompo}}\label{sect:homo_mix}

Each component of the list named {\tt 'MIXTURES'} contains the parameter tree and a list of
directories:

\begin{DescriptionEnregistrement}{Component of the homogenized mixture directory}{7.5cm}
\label{tabl:mixture_cpo}
\DirEnr
  {TREE\blank{8}}
  {Parameter tree. The specification is presented in \Sect{cpodirtree}.}
\DirlEnr
  {CALCULATIONS}{$\mathcal{S}^{\rm cpo}_{4}$}
  {List of {\sc microlib} directories. Each component of this list follows the specification
   presented in \Sect{microlibdirmain} with $M_{m}=1$ (1 mixture) and $\mathcal{M}=4$.}
\end{DescriptionEnregistrement}

\subsection{The {\tt TREE} sub-directory in a {\tt MIXTURES} component}\label{sect:cpodirtree}

This directory contains local parameter values and the parameter tree. Its specification follows:

\begin{DescriptionEnregistrement}{Contents of sub-directory {\tt TREE} in
{\tt MIXTURES}}{7.0cm}
\label{tabl:tabctree}
\IntEnr
  {NCALS\blank{7}}{$1$}
  {Number of elementary calculations.}
\IntEnr
  {NVP\blank{9}}{$2$}
  {{\tt NVP(1)}: Exact number of nodes in the parameter tree. {\tt NVP(2)}: Maximum number of nodes in the parameter tree.}
\OptIntEnr
  {NVALUE\blank{6}}{$\mathcal{S}^{\rm cpo}_{6}$}{$\mathcal{S}^{\rm cpo}_{6}\ge 1$}
  {Number of specific values for a local parameters.}
\OptRealVar
  {\listedir{lvaldir}}{\tt NVALUE(J)}{$\mathcal{S}^{\rm cpo}_{6}\ge 1$}{}
  {Set of real local parameter arrays. Each element of \listedir{lvaldir} contains a
   real array containing the tabulated values of the $J$--th local parameter.}
\IntEnr
  {DEBARB\blank{6}}{{\tt NVP(1)}$+1$}
  {- If the node does not correspond to the last parameter:
     index in {\tt DEBARB} of the first daughter of the node.
   - If the node correspond to the last parameter: index in
     {\tt DEBARB} where we recover the index of an elementary
     calculation.}
\IntEnr
  {ARBVAL\blank{6}}{{\tt NVP(1)}}
  {-For a global parameter: index of the parameter in the \listedir{gvaldir}
   record (see Table~\ref{tabl:tabglob}).
   -For a local parameter: index of the parameter in the \listedir{lvaldir} record}
\IntEnr
  {ORIGIN\blank{6}}{$\mathcal{S}^{\rm cpo}_{4}$}
  {Index of the mother elementary calculation. This information is useful to follow the
  historical relation between calculations.}
\end{DescriptionEnregistrement}

Item \listedir{lvaldir} represents a set of $\mathcal{S}^{\rm cpo}_{6}$ real records. The name of each
\listedir{lvaldir} record is a {\tt character*12} variable ({\tt text12})
composed using the following FORTRAN instruction:
$$
\mathtt{WRITE(}\mathsf{text12}\mathtt{,'(''pval'',I8.8)')}\: J
$$
where $J$ is the index of the local parameter with $1 \leq J \leq \mathcal{S}^{\rm cpo}_{6}$. The local
parameter values are sorted.

\vskip 0.2cm

The parameter tree has the same number of stages as global {\sl and} local parameters. The local parameters
always follow the global parameters. For each value of
the $i$--th parameter, the tree indicates the beginning position of the $(i+1)$--th parameter.
The arrays {\tt DEBARB} and {\tt ARBVAL} are set to localize the results of an elementary calculation
identified by a specific parameter tuple.

\vskip 0.2cm

An example of a parameter tree is represented here:
\begin{verbatim}
     dn = value in DEBARB,  (m) = value in ARBVAL

     Root                          *(0)
                                     !
     Param. Nb 1                  d2(1)
                            -------------------
                           !                   !
     Param. Nb 2        d3(1)                4(2)
                       ---------           ---------
                      !         !         !    !    !
     Param. Nb 3   d5(1)      6(3)     d7(1) 8(2) 9(3)   d10

     Calculation Nb:  4         5         1    2    3

     DEBARB:      2  3  5  7 10  4  5  1  2  3
     ARBVAL:      0  1  1  2  1  3  1  2  3
\end{verbatim}

\vskip 0.2cm

The useful dimensions of variables {\tt DEBARB}, {\tt ARBVAL} and {\tt ORIGIN} in
Table~\ref{tabl:tabctree} are respectively equal to {\tt NVP(1)}$+1$, {\tt NVP(1)}
and $\mathcal{S}^{\rm cpo}_{3}$. The allocated sizes may be bigger.

\vskip 0.2cm

Each elementary calculation is fully identified by a parameter tuple (the Fortran array {\tt MUPLET(NPTOT)}), an
integer array of dimension {\tt NPTOT}$=\mathcal{S}^{\rm cpo}_{5}+\mathcal{S}^{\rm cpo}_{6}$. The first
$\mathcal{S}^{\rm cpo}_{5}$ components correspond to global parameter indices; the following
$\mathcal{S}^{\rm cpo}_{6}$ components correspond to local parameter indices. Here, {\tt NVP} represents
the useful size of the {\tt ARBVAL} array. A recursive procedure is required to search the elementary
calculation index {\tt ICAL} corresponding to this parameter tuple {\tt MUPLET}. Note that some components of the tuple
can be set to zero in case where the number of global and local parameters is overdetermined. The recursive function
is called using
\begin{verbatim}
      ICAL=NICAL(1,NVP,NPTOT,DEBARB,ARBVAL,MUPLET)
\end{verbatim}

\noindent and is implemented in Fortran-90 as
\begin{verbatim}
RECURSIVE INTEGER FUNCTION NICAL(II,NVP,NPTOT,DEBARB,ARBVAL,MUPLET) RESULT(ICAL)
   INTEGER II,NVP,NPTOT,DEBARB(NVP+1),ARBVAL(NVP),MUPLET(NPTOT)
   IF(NPTOT==0) THEN
      ICAL=DEBARB(II+1)
      RETURN
   ENDIF
   NBOK=0
   IKEEP=0
   DO I=DEBARB(II),DEBARB(II+1)-1
      IF((MUPLET(1)==0).OR.(MUPLET(1)==ARBVAL(I))) THEN
         JICAL=NICAL(I,NVP,NPTOT-1,DEBARB,ARBVAL,MUPLET(2))
         IF(JICAL > 0) THEN
            IKEEP=JICAL
            NBOK=NBOK+1
         ELSE IF(JICAL==-1) THEN
            NBOK=2
         ENDIF
      ENDIF
   ENDDO
   IF(NBOK > 1) THEN
      ! Many elementary calculation exist for this tuple.
      ICAL=-1
   ELSE IF(NBOK==0) THEN
      ! No elementary calculation exists for this tuple.
      ICAL=0
   ELSE
      ICAL=IKEEP
   ENDIF
END FUNCTION NICAL
\end{verbatim}

\vskip 0.2cm

Similarly, another Fortran program can be used to search the parameter tuple corresponding to
the {\tt ICAL}--th elementary calculation:
\begin{verbatim}
SUBROUTINE COMUPL(NVP,NPTOT,ICAL,NCALS,DEBARB,ARBVAL,MUPLET)
   INTEGER NVP,NPTOT,ICAL,NCALS,DEBARB(NVP+1),ARBVAL(NVP),MUPLET(NPTOT)
   DO I=NVP-NCALS+1,NVP
      IF(DEBARB(I+1)==ICAL) THEN
         I0=I
         EXIT
      ENDIF
   ENDDO
   MUPLET(NPTOT)=ARBVAL(I0)
   DO IPAR=NPTOT-1,1,-1
      DO I=1,NVP-NCALS
         IF(DEBARB(I+1) > I0) THEN
            I0=I
            EXIT
         ENDIF
      ENDDO
      MUPLET(IPAR)=ARBVAL(I0)
   ENDDO
END SUBROUTINE COMUPL
\end{verbatim}

\clearpage
