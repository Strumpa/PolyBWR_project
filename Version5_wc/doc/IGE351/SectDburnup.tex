\section{Contents of a 
\dir{burnup} directory}\label{sect:burnupdir}

This directory contains the main burnup information, namely  the multigroup flux and the
isotopic concentration at each time or burnup step.

\subsection{State vector content for the \dir{burnup} data structure}\label{sect:burnupstate}

The dimensioning parameters for the \dir{burnup} data structure, which are stored in
the state vector $\mathcal{S}^{b}$, represent:

\begin{itemize}
\item The type of solution considered $I_{s}=\mathcal{S}^{b}_{1}$ where
\vskip -0.8cm
\begin{displaymath}
I_{s} = \left\{
\begin{array}{rl}
 1 & \textrm{Fifth-order Cash-Karp method}\\
 2 & \textrm{Forth-order Kaps-Rentrop method}  
\end{array} \right.
\end{displaymath}

\item The type of burnup considered $I_{t}=\mathcal{S}^{b}_{2}$ where
\vskip -0.8cm
\begin{displaymath}
I_{t} = \left\{
\begin{array}{rl}
 0 & \textrm{Out of core or zero flux/power depletion} \\
 1 & \textrm{Constant flux depletion} \\
 2 & \textrm{Constant fuel power depletion} \\
 3 & \textrm{Constant assembly power depletion} 
\end{array} \right.
\end{displaymath}

\item Number of time steps for which burnup properties are present in this directory
$N_{t}=\mathcal{S}^{b}_{3}$
 
\item Total number of isotopes $N_{I}=\mathcal{S}^{b}_{4}$ 

\item Number of depleting mixtures $N^{\rm depl}_{M}=\mathcal{S}^{b}_{5}$

\item Number of depleting reactions $N^{\rm depl}_{R}=\mathcal{S}^{b}_{6}$

\item Number of depleting isotopes $N^{\rm depl}_{I}=\mathcal{S}^{b}_{7}$

\item Number of mixtures $N_m=\mathcal{S}^{b}_{8}$

\item Microscopic reaction rate extrapolation option in solving the burnup equations
$I_{e}=\mathcal{S}^{b}_{9}$ where
\vskip -0.8cm
\begin{displaymath}
I_{e} = \left\{
\begin{array}{rl}
 0 & \textrm{Do not extrapolate} \\
 1 & \textrm{Perform linear extrapolation} \\
\end{array} \right.
\end{displaymath}

\item Constant power normalization option for the burnup calculation
$I_{g}=\mathcal{S}^{b}_{10}$ where
\vskip -0.8cm
\begin{displaymath}
I_{g} = \left\{
\begin{array}{rl}
 0 & \textrm{Compute the burnup using the power released in fuel} \\
 1 & \textrm{Compute the burnup using the power released in the global geometry} \\
\end{array} \right.
\end{displaymath}
This option have an effect only in cases
where some non-depleting mixtures are producing energy.

\item Saturation of initial number densities $I_{s}=\mathcal{S}^{b}_{11}$ where
\vskip -0.8cm
\begin{displaymath}
I_{s} = \left\{
\begin{array}{rl}
 0 & \textrm{Do not store saturated initial number densities in the {\sc burnup}
 object} \\
 1 & \textrm{Store saturated initial number densities} \\
\end{array} \right.
\end{displaymath}
This option have an effect only in cases where some depleting isotopes are
at saturation.

\item Type of saturation model $I_{d}=\mathcal{S}^{b}_{12}$ where
\vskip -0.8cm
\begin{displaymath}
I_{d} = \left\{
\begin{array}{rl}
 0 & \textrm{Do not use Dirac functions in saturated number densities} \\
 1 & \textrm{Use Dirac functions in saturated number densities} \\
\end{array} \right.
\end{displaymath}
This option have an effect only in cases where some depleting isotopes are
at saturation.

\item Perturbation flag for cross sections $I_{p}=\mathcal{S}^{b}_{13}$ where
\vskip -0.8cm
\begin{displaymath}
I_{p} = \left\{
\begin{array}{rl}
 0 & \textrm{Time-dependent cross sections will be used if available} \\
 1 & \textrm{Time-independent cross sections will be used} \\
\end{array} \right.
\end{displaymath}

\item Neutron flux recovery flag $I_{f}=\mathcal{S}^{b}_{14}$ where
\vskip -0.8cm
\begin{displaymath}
I_{f} = \left\{
\begin{array}{rl}
 0 & \textrm{Neutron flux is recovered from a L\_FLUX object} \\
 1 & \textrm{Neutron flux is recovered from the embedded macrolib present in a} \\
 & \textrm{L\_LIBRARY object} \\
\end{array} \right.
\end{displaymath}

\item Fission yield data recovery flag $I_{y}=\mathcal{S}^{b}_{15}$ where
\vskip -0.8cm
\begin{displaymath}
I_{y} = \left\{
\begin{array}{rl}
 0 & \textrm{Fission yield data is recovered from {\tt DEPL-CHAIN} directory (see \Sect{microlibdirdepletion})} \\
 1 & \textrm{Fission yield data is recovered from {\tt PIFI} and {\tt PYIELD} records in /isotope/} \\
 & \textrm{directory (see Table~\ref{tabl:tabiso3})} \\
\end{array} \right.
\end{displaymath}
\end{itemize}

\subsection{The main \dir{burnup} directory}\label{sect:burnupdirmain}

On its first level, the
following records and sub-directories will be found in the \dir{burnup} directory:

\begin{DescriptionEnregistrement}{Main records and sub-directories in \dir{burnup}}{8.0cm}
\CharEnr
  {SIGNATURE\blank{3}}{$*12$}
  {Signature of the \dir{burnup} data structure ($\mathsf{SIGNA}=${\tt L\_BURNUP\blank{4}}).}
\IntEnr
  {STATE-VECTOR}{$40$}
  {Vector describing the various parameters associated with this data structure
  $\mathcal{S}^{b}_{i}$, as defined in \Sect{burnupstate}.}
\RealEnr
  {EVOLUTION-R\blank{1}}{$5$}{}
  {Vector describing the various parameters associated with the burnup calculation options
$R_{i}$}
\CharEnr
  {LINK.LIB\blank{4}}{$*12$}
  {Name of the {\sc microlib} on which the last depletion step was based.}
\RealEnr
  {DEPL-TIMES\blank{2}}{$N_{t}$}{$10^{8}$ s}
  {Vector describing the various time steps at which burnup information has been saved
$T_{i}$}
\RealEnr
  {FUELDEN-INIT}{$3$}{}
  {Vector giving the initial density of heavy element in the fuel $\rho_{f}$ (g
  cm$^{-3}$), the initial mass of heavy element in the fuel $m_{f}$ (g) and the
  initial mass of heavy element in the fuel divided by the global geometry
  volume (g cm$^{-3}$)}
\RealEnr
  {VOLUME-MIX\blank{2}}{$N_m$}{cm$^3$}
  {Vector giving the mixture volumes}
\RealEnr
  {FUELDEN-MIX\blank{1}}{$N_m$}{g}
  {Initial mass of heavy element contained in each mixture}
\RealEnr
  {WEIGHT-MIX\blank{2}}{$N_m$}{g}
  {Initial mass of all the isotopes contained in each mixture}
\IntEnr
  {DEPLETE-MIX\blank{1}}{$N_m \times N^{\rm depl}_{I}$}
  {Matrix giving the index in the {\tt ISOTOPESDENS} record of each depleting
   isotope in each mixture.}
\CharEnr
  {ISOTOPESUSED}{$(N_{I})*12$}
  {Alias name of the isotopes}
\IntEnr
  {ISOTOPESMIX\blank{1}}{$N_{I}$}
  {Mixture number associated with each isotope}
\IntEnr
  {MIXTURESBurn}{$N_m$}
  {Depletion flag array. A component is set to 1 to indicate that a mixture is depleting.}
\IntEnr
  {MIXTURESPowr}{$N_m$}
  {Power flag array. A component is set to 1 to indicate that a mixture is producing power.}
\DirVar
  {\listedir{depldir}}
  {Set of $N_{t}$ sub-directories containing the properties associated with each
   burnup step $T_{i}$}
\end{DescriptionEnregistrement}

The set of directory \listedir{depldir} names $\mathsf{DEPLDIR}$ will be composed according to the
following laws. The first eight character ($\mathsf{DEPLDIR}$\verb*|(1:8)|) will always be given by 
\verb*|DEPL-DAT|. The last four characters  
($\mathsf{DEPLDIR}$\verb*|(9:12)|) represent the time step saved. For the case where
$N_{t}$ time steps were saved we would use the following FORTRAN instructions to create
the last four characters of each of the directory names:
$$
\mathtt{WRITE(}\mathsf{DEPLDIR}\mathtt{(9:12),'(I4.4)')}\: J
$$
for $1\leq J \leq N_{t}$ with the time stamp associated with each directory being given by
$T_{J}$.  For the case where ($N_{t}=2$), two such directory would be generated, namely

\begin{DescriptionEnregistrement}{Example of depletion directories}{8.0cm}
\DirEnr
  {DEPL-DAT0001}{Sub-directories which contain the information associated with
   time step 1}
\DirEnr
  {DEPL-DAT0002}{Sub-directories which contain the information associated with
   time step 2}
\end{DescriptionEnregistrement}

\clearpage

\subsection{The depletion sub-directory \dir{depldir} in
\dir{burnup}}\label{sect:burnupdirdepletion}

Inside each depletion directory the following records and sub-directories will be found:

\begin{DescriptionEnregistrement}{Contents of a depletion sub-directory in \dir{burnup}}{7.0cm}
\RealEnr
  {ISOTOPESDENS}{$N_{I}$}{(cm b)$^{-1}$}
  {Isotopic densities $\rho_{i}$ for each of the isotopes described in the \dir{microlib} directory
   where the order of the isotopes is also specified}
\RealEnr
  {MICRO-RATES\blank{1}}{$N^{\rm dim}$}{$10^{-8}$ s$^{-1}\ $}
  {Values of the microscopic reaction rate of the depleting reactions for each
   depleting isotope and each mixture. The macroscopic reaction rate related to the
   non-depleting isotopes is stored at location $N^{\rm depl}_{I}+1$. The
   $N^{\rm depl}_{R}$ reaction types are stored in the order of the {\tt
   'DEPLETE-IDEN'} array in Table~\ref{tabl:tabchain}, starting with the {\tt 'NFTOT'}
   reaction. The flux-induced power factors are stored in location $N^{\rm depl}_{R}$.
   The decay power (delayed) factors are stored in location $N^{\rm depl}_{R}+1$ Both
   flux-induced and decay power are given in units of $10^{-8}$ MeV/s.
   $N^{\rm dim}=(N^{\rm depl}_{I}+1)
   \times (N^{\rm depl}_{R}+1) \times N_m$}
\RealEnr
  {INT-FLUX\blank{4}}{$N_m$}{cm s$^{-1}$}
  {Integrated flux in each mixture.}
\RealEnr
  {FLUX-NORM\blank{3}}{$1$}{$1$}
  {Flux normalization constant. It is zero for out of core depletion and
   represents the
   normalization of the flux $\phi_{r}^{g}$ that is used to ensure that the cell integrated flux or
   power is that required when fixed flux or power burnup is requested}
\RealEnr
  {ENERG-MIX\blank{3}}{$N_m$}{Joule}
  {Energy realeased during the time step in each mixture}
\OptRealEnr
  {FORM-POWER\blank{2}}{1}{$I_{t}=3$}{1}
  {Ratio of the global power released in the complete geometry divided by the
   power released in fuel.}
\RealEnr
  {BURNUP-IRRAD}{$2$}{}
  {Fuel burnup (MW d T$^{-1}$) and irradiation (Kb$^{-1}$) reached at this time step}
\end{DescriptionEnregistrement}

\eject
