\subsection{The \moc{LZC:} module}\label{sect:lzc}

The \moc{LZC:} module is used for the modeling of liquid zone controllers, which
are normally presented in the CANDU6-type reactor core. The liquid zone controllers
specifications are read from the input data file. Note that this modeling can be made
after the rod-type devices have been previously defined using the \moc{DEVINI:}
module (see \Sect{dev}). In this case, the previously created \dds{device} object
will be updated by the \moc{LZC:} module; it will store the additional and separate
information with respect to the liquid controllers (see \Sect{device}). \\

The liquid zone controller specification includes several device parameters, such as:
the whole device position, water filling level, direction of filling, etc. Note that a liquid
zone controller is normally composed of two parts: one part is empty and the second
part is full-filled. The water level can be adjusted according to the control reactivity
requirements. The controllers positions are referred using \dusa{3-D}-Cartesian
coordinates. Several devices parameters can be modified using the \moc{DSET:}
module (see \Sect{dset}). The liquid controllers can also be divided into the several
user-defined groups so that they can be manipulated simultaneously.\\

\noindent
The \moc{LZC:} module specification is:

\begin{DataStructure}{Structure \moc{LZC:}}
\dusa{DEVICE} \dusa{MATEX} \moc{:=} \moc{LZC:}
$[$ \dusa{DEVICE} $]$ \dusa{MATEX} \moc{::} \dstr{desclzc}
\end{DataStructure}

\noindent where

\begin{ListeDeDescription}{mmmmmmmm}

\item[\dusa{DEVICE}] \texttt{character*12} name of the \dds{device} object.
Note, if the rod-type devices are not present in the reactor core, then \dusa{DEVICE}
object must appear only on the LHS (i.e. in create mode), it will contain the information
only with respect to the liquid zone controllers. However, if the rod-type devices
are present in the reactor core, then they must be specified first (i.e. before the
liquid controllers) using the \moc{DEVINI:} module (see \Sect{dev}). In the last
case, the \dusa{DEVICE} object must also appear on the RHS (i.e. in modification
mode), it will contain the additional and separate information with respect to the
liquid zone controllers.

\item[\dusa{MATEX}] \texttt{character*12} name of the \dds{matex} object
that will be updated by the module. The lzc-devices material mixtures are
appended to the previous material index and the lzc-devices indices are
also modified, accordingly.

\item[\dstr{desclzc}] structure describing the input data to the \moc{LZC:} module.

\end{ListeDeDescription}

\vskip 0.2cm

\subsubsection{Input data to the \moc{LZC:} module}\label{sect:lzcstr}

Note that the input order must be respected.

\begin{DataStructure}{Structure \dstr{desclzc}}
$[$ \moc{EDIT} \dusa{iprint} $]$ \\
\moc{NUM-LZC} \dusa{nlzc} \\
(\dstr{dev-lzc}, i = 1, \dusa{nlzc}) \\
$[$ \moc{CREATE}  \moc{LZC-GR} \dusa{ngrp}  (\dstr{lzc-group}, i = 1, \dusa{ngrp}) $]$ \\
;
\end{DataStructure}

\noindent where
\begin{ListeDeDescription}{mmmmmmmm}

\item[\moc{EDIT}] keyword used to set \dusa{iprint}.

\item[\dusa{iprint}] integer index used to control the printing on screen:
= 0 for no print; = 1 for minimum printing (default value); larger values
produce increasing amounts of output.

\item[\moc{NUM-LZC}] keyword used to specify \dusa{nlzc}.

\item[\dusa{nlzc}] integer total number of liquid zone controllers.
This number must be greater than 0.

\item[\moc{CREATE}] keyword used to create the lzc-groups of devices.
The creation of groups is optional.

\item[\moc{LZC-GR}] keyword used to set \dusa{ngrp}.

\item[\dusa{ngrp}] integer total number of the lzc groups to be created.
This number must be greater than 0.

\item[\dstr{dev-lzc}] structure describing the input data for each individual
liquid controller.
 
\item[\dstr{lzc-group}] structure describing the input data for each group of
liquid controllers.

\end{ListeDeDescription}

\vskip 0.2cm

\subsubsection{Description of dev-lzc input structure}\label{sect:devlzcstr}

Note that the devices positions can not overlap in the reactor core. The input order of data
must be respected.

\begin{DataStructure}{Structure \dstr{dev-lzc}}
\moc{LZC} \dusa{id} \\
\moc{MAXPOS}  ( \dusa{pos}(i) , i = 1, 6 ) \\
\moc{MAX-FULL} \dusa{fmax} \\
\moc{AXIS} $\{$ \moc{X} $|$ \moc{Y} $|$ \moc{Z} $\}$ \\
\moc{LEVEL} \dusa{value} \\
$[$ \moc{RATE} \dusa{rate} $]$ \\
$[$ \moc{TIME} \dusa{time} $]$ \\
\moc{EMPTY-MIX}  ( \dusa{mixE}(n), n = 1, 2 ) \\
\moc{FULL-MIX}  ( \dusa{mixF}(n), n = 1, 2 )
\end{DataStructure}

\noindent where
\begin{ListeDeDescription}{mmmmmmmm}

\item[\moc{LZC}] keyword used to specify the liquid controller
\dusa{id} number.

\item[\dusa{id}] integer identification number of the current liquid controller.
Each controller must be assigned a unique \dusa{id} number, given in an
ascending order ranging from 1 to \dusa{nlzc}.

\item[\moc{MAXPOS}] keyword used to specify the entire position of a
liquid zone controller, including its empty and full parts.

\item[\dusa{pos}] real array containing 3-D Cartesian coordinates of the
liquid zone controller position in the reactor core. These coordinates
must be given in the order: X$-$  X$+$   Y$-$  Y$+$   Z$-$  Z$+$

\item[\moc{MAX-FULL}] keyword used to specify \dusa{fmax}.

\item[\dusa{fmax}] real value of the limiting coordinate along the controller
filling axis, which corresponds to the maximum full-filling level for the current
liquid controller.

\item[\moc{AXIS}] keyword used to specify the controller filling axis.
A liquid controller can be filled along only one (vertical) axis.

\item[\moc{X}] keyword used to specify that a liquid controller is filled along X axis.

\item[\moc{Y}] keyword used to specify that a liquid controller is filled along Y axis.

\item[\moc{Z}] keyword used to specify that a liquid controller is filled along Z axis.

\item[\moc{LEVEL}] keyword used to specify the actual filling level.

\item[\dusa{value}] real positive value of the water level. This value is minimal
(\dusa{value} = 0.0) when the controller is empty, and it is maximal (\dusa{value} = 1.0)
when the controller is full-filled. For the partially filled controller the water level
must be: 0.0 $<$ \dusa{value} $<$ 1.0

\item[\moc{RATE}] keyword used to specify \dusa{rate}.

\item[\dusa{rate}] real positive value of the water filling rate, given in
m$^{3}$/s. This value is needed only for the reactor regulating purpose.

\item[\moc{TIME}] keyword used to specify \dusa{time}.

\item[\dusa{time}] real value of the filling time, given in sec.
This value is needed only for the reactor regulating purpose.

\item[\moc{EMPTY-MIX}] keyword used to specify \dusa{mixE}.

\item[\dusa{mixE}] two integer mixture indices, specified for the empty-part of
liquid controller. The first and the second mixture indices correspond to the
perturbed and the reference cross sections, respectively. These indices will be
used to compute the incremental cross sections in the \moc{NEWMAC:} module.

\item[\moc{FULL-MIX}] keyword used to specify \dusa{mixF}.

\item[\dusa{mixF}] two integer mixture indices, specified for the full-part of liquid
controller. The first and the second mixture indices correspond to the perturbed
and the reference cross sections, respectively. These indices will be used
to compute the incremental cross sections in the \moc{NEWMAC:} module.

\end{ListeDeDescription}
\clearpage

\subsubsection{Description of lzc-group input structure}\label{sect:lzcgroupstr}

The partition of lzc-devices into groups is similar to that of rod-devices.

\begin{DataStructure}{Structure \dstr{lzc-group}}
\moc{GROUP-ID} \dusa{igrp} $\{$ \moc{LZC-ID}
$[[$ \dusa{id} $]]$  $|$  \moc{ALL} $\}$
\end{DataStructure}

\noindent where
\begin{ListeDeDescription}{mmmmmmmm}

\item[\moc{GROUP-ID}] keyword used to set \dusa{igrp} number.

\item[\dusa{igrp}] integer identification number of a group to be created.
Each controllers group must be assigned a unique identification number,
given in ascending order ranging from 1 to \dusa{ngrp}.

\item[\moc{LZC-ID}] keyword used to set the controllers \dusa{id} numbers.

\item[\dusa{id}] integer identification numbers of the liquid controllers which
belong to the same group \dusa{igrp}. A particular controller (or several devices)
may belong to different groups, but it could not be repeated inside the same
group. The total number of liquid controllers in any group must be between
1 and \dusa{nlzc}.

\item[\moc{ALL}] keyword used to specify that all liquid controllers
will belong to the same group \dusa{igrp}.

\end{ListeDeDescription}
\clearpage
