\subsection{The {\tt NAP:} module}\label{sect:NAPData}

The \moc{NAP:} module supplies the main transport-diffusion equivalence options to DRAGON and DONJON. It can be
used to perform the pin power reconstruction.\cite{Chambon2014,Fliscounakis2011} The calling specifications are:

\begin{DataStructure}{Structure \dstr{NAP:}}
\dusa{COMPO} \moc{:=} \moc{NAP:} \dusa{COMPO} \dusa{TRKNAM} \dusa{FLUNAM} \moc{::} \dstr{descnap1} \\
\dusa{MAP} \moc{:=} \moc{NAP:} \dusa{MAP} \dusa{TRKNAM} \dusa{FLUNAM} \dusa{MATEX} \dusa{MACRES} \moc{::} \dstr{descnap2} \\
\dusa{GEONEW} \moc{:=} \moc{NAP:} \dusa{GEOOLD} \dusa{COMPO} \moc{::} \dstr{descnap3} \\
\end{DataStructure}

\noindent
where
\begin{ListeDeDescription}{mmmmmmmm}

\item[\dusa{COMPO}] {\tt character*12} name of the \dds{multicompo} data
structure ({\tt L\_COMPO} signature) where the detailed subregion properties will be stored.

\item[\dusa{TRKNAM}] {\tt character*12} name of the read-only \dds{tracking} data
structure ({\tt L\_TRACK} signature) containing the tracking. 

\item[\dusa{FLUNAM}] {\tt character*12} name of the read-only \dds{fluxunk} data
structure ({\tt L\_FLUX} signature) containing a transport solution.

\item[\dusa{MAP}] {\tt character*12} name of the \dds{map} data
structure ({\tt L\_MAP} signature) containing fuel regions description, global and
local parameter information (burnup, fuel/coolant temperatures, coolant density, etc). A previous call to the \moc{FLPOW:} module is highly recommended prior to the pin-power reconstruction to normalize the flux and compute the assembly power. If not, the pin-power reconstruction  will be normalized using the whole core power instead of a normalization for each assembly.
Keyword \moc{PPR} is expected in \dstr{descnap2}.

\item[\dusa{MATEX}] {\tt character*12} name of the read-only \dds{matex} data
structure ({\tt L\_MATEX} signature). The object corresponds to the heterogeneously splited geometry. Keyword \moc{PPR} is expected in \dstr{descnap2}.

\item[\dusa{MACRES}] {\tt character*12} name of the read-only \dds{macrolib} data
structure ({\tt L\_MACROLIB} signature) containing a cross section for the fuel. The \dds{macrolib} data
structure must have been created with a \dds{multicompo} data structure with pin level properties (transport flux, H-factor, infinite domain diffusion flux). Keyword \moc{PPR} is expected in \dstr{descnap2}.

\item[\dusa{GEONEW}] {\tt character*12} name of the created \dds{geometry} data
structure ({\tt L\_GEOM} signature) containing the detailed core geometry definition at heterogeneous assembly level.

\item[\dusa{GEOOLD}] {\tt character*12} name of the read-only \dds{geometry} data
structure ({\tt L\_GEOM} signature) containing the core geometry definition with homogeneous assembly (only 1 mesh per assembly mandatory).

\item[\dstr{descnap1}] structure containing the input data to this module to compute additional properties for subregions
(see \Sect{descnap1}).

\item[\dstr{descnap2}] structure containing the input data to this module to perform pin power reconstruction
(see \Sect{descnap2}).

\item[\dstr{descnap3}] structure containing the input data to this module to automatically define the core geometry with heterogeneous assembly
(see \Sect{descnap3}).

\end{ListeDeDescription}

\subsubsection{Additional properties calculations}\label{sect:descnap1}

\begin{DataStructure}{Structure \dstr{descnap1}}
$[$ \moc{EDIT} \dusa{iprint} $]$ \\
\moc{PROJECTION}\\
 \moc{STEP} \dusa{namedir} \\
$[$ \moc{IFX} \dusa{ifx} $]$ \\
$\{$ $[[$ \moc{SET} \dusa{pname} \dusa{pvalue} $]]$ $\}$ \\
;
\end{DataStructure}

\noindent where

\begin{ListeDeDescription}{mmmmmmmm}

\item[\moc{EDIT}] keyword used to modify the print level \dusa{iprint}.

\item[\dusa{iprint}] integer index used to control  the printing in module {\tt NAP:}.
=0 for no print; =1 for minimum printing (default value); larger values of \dusa{iprint}
will produce increasing amounts of output.

\item[\moc{PROJECTION}] keyword to specify that additional properties for subregions will be computed and stored in the \dds{multicompo} data-structure \dusa{COMPO}.

\item[\moc{STEP}] keyword to specify \dusa{namedir}. 

\item[\dusa{namedir}] name of the directory containing the homogenized cross sections (homogeneous or heterogeneous). 

\item[\moc{IFX}] keyword to specify \dusa{ifx}. 

\item[\dusa{ifx}] number used to create the name of the flux record representing $\psi_{m,p}^{d,\infty}$. This flux represents the results of calculation in an infinite domain computed in diffusion with cross sections homogenized either homogeneously or heterogeneously. One record is associated with each type of homogenization when the \dds{multicompo} is created. Thus \dds{multicompo} can be "enriched" up to 5 times, each time using a different homogenization of the assembly at the end of the transport calculations. The following format is used for the flux record name (\verb+RECNAME+): \\
\verb+WRITE(RECNAME,'5HFINF_,I3.3') IFX+\\
Thus, for example for \dusa{ifx}=2, \verb+RECNAME+ would be \verb+FINF_002+.

\item[\moc{SET}] keyword to specify assembly calculations at which $\psi_{m,p}^{d,\infty}$ have been calculated. Repeated as many times as there are parameters in the \dds{multicompo}.

\item[\dusa{pname}] name of the parameter in the \dds{multicompo}. 

\item[\dusa{pvalue}] value of the parameter in the \dds{multicompo}. 

\end{ListeDeDescription}

Note that in the case of heterogeneously homogenized assembly, the pin-wise projected diffusion flux is stored in mixture 1.

\subsubsection{Pin power reconstruction}\label{sect:descnap2}

\begin{DataStructure}{Structure \dstr{descnap2}}
$[$ \moc{EDIT} \dusa{iprint} $]$ \\
\moc{PPR}\\
\moc{NZASS} \dusa{nzass}  \\ 
\moc{METH}  %$\{$  \\
     \moc{GPPR} \dusa{ifx} %$\}$
\\
$[$ \moc{POWER} \dusa{pow} $]$ \\
;
\end{DataStructure}

\noindent where

\begin{ListeDeDescription}{mmmmmmmm}

\item[\moc{EDIT}] keyword used to modify the print level \dusa{iprint}.

\item[\dusa{iprint}] index used to control the printing of this module. The
\dusa{iprint} parameter is important for adjusting the amount of data that is
printed by this calculation step:

\begin{itemize}

\item \dusa{iprint}=0 results in no output;

\item \dusa{iprint}=1 ...

\end{itemize}

\item[\moc{PPR}] keyword to perform pin power reconstruction and stored results in the \dds{map} data-structure \dusa{MAP}.

\item[\moc{NZASS}] keyword to specify \dusa{nzass}. 

\item[\dusa{nzass}] number of mesh in Z direction along assemblies. 

\item[\moc{METH}] keyword to select the type of methodology used for pin power reconstruction.

\item[\moc{GPPR}] keyword to select the generalized pin power reconstruction from an heterogeneous assembly definition (several mixtures per assembly). Note that if there is only one mixture (homogeneous assembly) this method is actually the regular pin power reconstruction.

\item[\dusa{ifx}] number used to create the named of the flux record representing $\psi_{m,p}^{d,\infty}$. See Sect. \ref{sect:descnap1} for more details.

\item[\moc{POWER}] keyword used to normalize the flux. If this keyword is not used, the flux directly computed by the \moc{FLUD:} is used to perform the pin-power reconstruction,  then normalization has to be performed either previously by the \moc{FLPOW:} module  or by the user independently. 

\item[\dusa{pow}] power used to normalize the flux (MW). 

\end{ListeDeDescription}

\clearpage

\subsubsection{Heterogeneous assembly geometry definition}\label{sect:descnap3}

\begin{DataStructure}{Structure \dstr{descnap3}}
$[$ \moc{EDIT} \dusa{iprint} $]$ \\
\moc{DIRGEO} \dusa{namedir} $[$ \moc{MACGEO} $]$  \\
\moc{MIXASS} \dusa{nmix} (\dusa{imix}(i), i=1,\dusa{nmix})  \\
$[$ \moc{SLPITX-ASS} (\dusa{ispx}(i), i=1,\dusa{nxass}) $]$ \\
$[$ \moc{SLPITY-ASS} (\dusa{ispy}(i), i=1,\dusa{nyass}) $]$ \\
$[$ \moc{MAX-MIX-GEO} \dusa{nmxgeo} $]$ \\
;
\end{DataStructure}

\noindent where

\begin{ListeDeDescription}{mmmmmmmm}

\item[\moc{EDIT}] keyword used to modify the print level \dusa{iprint}.

\item[\dusa{iprint}] index used to control the printing of this module. The
\dusa{iprint} parameter is important for adjusting the amount of data that is
printed by this calculation step:

\begin{itemize}

\item \dusa{iprint}=0 results in no output;

\item \dusa{iprint}=1 ...

\end{itemize}

\item[\moc{DIRGEO}] keyword to specify \dusa{namedir}. 

\item[\dusa{namedir}] name of the directory containing the heterogeneously or homogeneously homogenized cross sections. 

\item[\moc{MACGEO}] keyword to specify that the macro-geometry stored in the \dusa{GFF} record of the macrolib will be used instead of the macro-geometry of the  heterogeneously or homogeneously homogenized cross sections. Usually this other macro-geometry of the assembly corresponds to the pin-by-pin geometry. Note that another multicompo with all pin-wise properties is needed to be able to use the automatically generated geometry. 

\item[\moc{MIXASS}] keyword to specify mixtures corresponding to assembly in the coarse geometry. 

\item[\dusa{nmix}] number of type of assembly. 

\item[\dusa{imix}] mixture number of the assemblies.

\item[\moc{SLPITX-ASS}] keyword to specify the mesh splitting at assembly level. 

\item[\dusa{ispx}] split along x-direction for each mesh of the heterogeneous assembly. 

\item[\dusa{nxass}] number of mesh of the heterogeneous assembly along x-direction. 

\item[\moc{SLPITY-ASS}] keyword to specify the mesh splitting at assembly level. 

\item[\dusa{ispy}] split along y-direction for each mesh of the heterogeneous assembly. 

\item[\dusa{nyass}] number of mesh of the heterogeneous assembly along y-direction. 

\item[\moc{MAX-MIX-GEO}] keyword to specify the number of mixtures in the original core geometry (i.e. before the core geometry is splited by the \moc{NAP:} module). This keyword is mandatory if there is a reflector in the geometry otherwise the numbers for fuel mixtures will not match between the split core geometry (\dds{geometry}) and the split fuel geometry (\dds{geometry} embedded in \dds{map}).

\item[\dusa{nmxgeo}] number of mixtures in the original core geometry. 

\end{ListeDeDescription}

{\bf Note: The included geometry in the \dusa{COMPO} has to be unfolded, even if the transport calculations are done on a 1/8th assembly. Moreover no split can be defined in the geometry, one mesh ONLY per heterogeneous mixture  is mandatory.} 

\eject
