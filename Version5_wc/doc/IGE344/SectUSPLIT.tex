\subsection{The \moc{USPLIT:} module}\label{sect:link}

\vskip 0.2cm
The \moc{USPLIT:} module is used to create a \moc{MATEX} object that will provide
a link between the reactor geometry and material index. The \dusa{3-D} Cartesian or
\dusa{3-D} Hexagonal reactor geometry, which is previously produced in the \moc{GEO:}
module, is analyzed and the material mixture indices are recomputed in order to provide
a unique mixture number for each material sub-volume. Such renumbering permits a complex reactor
core modeling. A \moc{MATEX} object is also used to store some additional information
that will be required and updated by other DONJON modules (see \Sect{matex}).\\

\noindent
The \moc{USPLIT:} module specification is: 

\begin{DataStructure}{Structure \moc{USPLIT:}}
\dusa{GEOM} \dusa{MATEX} \moc{:=} \moc{USPLIT:}
$\{$ \dusa{GEOM} $|$ \dusa{GEOMOLD} $\}$ \moc{::} \dstr{desclink}
\end{DataStructure}

\noindent where
\begin{ListeDeDescription}{mmmmmmmm}

\item[\dusa{GEOM}] \texttt{character*12} name of a \dds{geometry} object. This object is defined
in creation (appears only on LHS) or modification (appears on both LHS and RHS) mode. An existing geometry previously created in
the \moc{GEO:} module is modified. Only \dusa{3-D} Cartesian or \dusa{3-D} Hexagonal reactor geometries are allowed.

\item[\dusa{MATEX}] \texttt{character*12} name of a \dds{matex} object
to be created by the module.

\item[\dusa{GEOMOLD}] \texttt{character*12} name of a \dds{geometry} object previously created in the \moc{GEO:} module.
This object must be specified in read-only mode (appears only on RHS). It is copied into \dusa{GEOM} at the beginning of \moc{USPLIT:} module.
Only \dusa{3-D} Cartesian or \dusa{3-D} Hexagonal reactor geometries are allowed.

\item[\dstr{desclink}] structure describing the input data to the \moc{USPLIT:} module.

\end{ListeDeDescription}

\vskip 0.2cm

\subsubsection{Input data to the \moc{USPLIT:} module}\label{sect:usplitstr}
\noindent
Note that the fuel-type and reflector-type mixture indices are need to be
specified explicitly and the input order must be respected.
\begin{DataStructure}{Structure \dstr{desclink}}
$[$ \moc{EDIT} \dusa{iprint} $]$ \\
\moc{NGRP}  \dusa{ngrp} \\
\moc{MAXR}  \dusa{maxreg} \\
\moc{NMIX}  \dusa{nmixt} \\
$[$ \moc{NREFL} \dusa{nrefl} \moc{RMIX} ( \dusa{mixr}(i) ,  i = 1, \dusa{nrefl} )  $]$ \\
$[$ \moc{NFUEL} $\{$ \dusa{nfuel} \moc{FMIX} ( \dusa{mixf}(i) ,  i = 1, \dusa{nfuel} )  $|$ \moc{ASBLY} $\}$  $]$ \\
;
\end{DataStructure}

\noindent where
\begin{ListeDeDescription}{mmmmmmmm}

\item[\moc{EDIT}] keyword used to set \dusa{iprint}.

\item[\dusa{iprint}] integer index used to control the printing on
screen: = 0 for no print; = 1 for minimum printing (default value);
larger values produce increasing amounts of output.

\item[\moc{NGRP}] keyword used to specify \dusa{ngrp}.

\item[\dusa{ngrp}] integer total number of energy groups.
This value must be greater than 0.

\item[\moc{MAXR}] keyword used to specify \dusa{maxreg}.

\item[\dusa{maxreg}] integer maximum number of mesh-splitted
regions in the reactor geometry. In \dusa{3-D} Hezagonal geometry, it corresponds to
the total number of prismatic blocks $l_{h}$*$l_{z}$.

\item[\moc{NMIX}] keyword used to extend number of material mixtures in case new fuels are going to be inserted in
the fuel map in upcoming fuel cycles. By default, \dusa{nmixt} is set to the maximum mixture index in RHS geometry \dusa{GEOM} or \dusa{GEOMOLD}.

\item[\dusa{nmixt}] the maximum fuel mixture index in the complete life of the reactor. This number must be greater
than the maximum mixture index in RHS geometry \dusa{GEOM} or \dusa{GEOMOLD}.

\item[\moc{NREFL}] keyword used to specify \dusa{nrefl}.

\item[\dusa{nrefl}] integer total number of reflector types.
A reactor should have at least one reflector material.

\item[\moc{RMIX}] keyword used to specify \dusa{mixr}.

\item[\dusa{mixr}] integer array of the reflector-type mixture indices.
Each reflector type is assigned a distinct mixture number as
previously defined in the \dds{geometry} object.

\item[\moc{NFUEL}] keyword used to specify \dusa{nfuel}.

\item[\dusa{nfuel}] integer total number of fuel types.
A reactor should have at least one fuel type.

\item[\moc{FMIX}] keyword used to specify \dusa{mixf}.

\item[\dusa{mixf}] integer array of the fuel-type mixture indices.
Each fuel type is assigned a distinct mixture number as
previously defined in the \dds{geometry} object.

\item[\moc{ASBLY}] keyword used to compute automatically \dusa{nfuel} and \dusa{mixf}(i). This option is only available when the geometry has been split by the \moc{NAP:} module.

\end{ListeDeDescription}
\clearpage
