\subsection{The \moc{DETINI:} module}\label{sect:tinst}

\vskip 0.2cm
The \moc{DETINI:} module is used to read and store detector information. 
A detector is represented by a 2-D or 3-D Cartesian/Hexagonal geometry.\\

\noindent
The \moc{DETINI:} module specification is:

\begin{DataStructure}{Structure \moc{DETINI:}}
\dusa{DETECT} \moc{:=} \moc{DETINI:}
$[$ \dusa{DETECT} $]$ \moc{::} \dstr{descdet}
\end{DataStructure}

\noindent where

\begin{ListeDeDescription}{mmmmmmmm}

\item[\dusa{DETECT}] \texttt{character*12} name of the \dds{detect} object
that will be created by the module; it will contain the detector informations.
If \dds{detect} appear on RHS, it is updated, otherwise, it is created.

\item[\dstr{descdev}] structure describing the input data to the
\moc{DETINI:} module.

\end{ListeDeDescription}

\vskip 0.2cm
\subsubsection{Input data to the \moc{DETINI:} module}\label{sect:strtinst}

\noindent
Note that the input order must be respected. \\
\begin{DataStructure}{Structure \dstr{descinidet}}
$[$ \moc{EDIT} \dusa{iprt} $]$ $[$ HEXZ $]$
\moc{NGRP} \dusa{ngrp} \\
$[[$ \moc{TYPE} \dusa{NAMTYP} \\
\moc{INFO} \dusa{ndetect} \dusa{nrep}
$\{$ \moc{SPECTRAL} ( \dusa{spec}(i), i=1,\dusa{ngrp} ) $|$
\moc{DEFAULT} $\}$  \\ 
$[$ \moc{INVCONST} ( \dusa{tinv}(i), i=1,\dusa{nrep}$-2$ ) $]$
$[$ \moc{FRACTION} ( \dusa{fract}(i), i=1,\dusa{nrep}$-1$ ) $]$ \\
( \dstr{descdet}, i=1,\dusa{ndetect} ) $]]$ \\
;
\end{DataStructure}

\noindent where
\begin{ListeDeDescription}{mmmmmmmm}

\item[\moc{EDIT}] keyword used to set \dusa{iprt}.

\item[\dusa{iprt}] index used to control the printing in module \moc{
INIDET:}. =1,2 for no print(default value); =3 for printing the contents 
of the output \dds{detect}.

\item[\moc{HEXZ}] keyword to specify that only hexagonal detectors will be 
defined.
If this keyword is absent, Cartesian detectors will be
defined. 

\item[\moc{NGRP}] keyword used to set \dusa{ngrp}.

\item[\dusa{ngrp}] number of energy groups in the calculation. It must be
equal to the number set in the \moc{MACD:} module or by the \dds{compo}
files. 

\item[\moc{TYPE}] keyword to specify the detector type.

\item[\dusa{NAMTYP}] \texttt{character*12} name of the detector type.
To correspond to the actual detector response model encoded,
the type of detector must be in this list: 
\begin{itemize}
\item  \texttt{PLATN\_REGUL}
\item  \texttt{PLATN\_SAU}
\item  \texttt{VANAD\_REGUL} 
\item  \texttt{CHION\_SAU}
\item  \texttt{CHION\_REGUL} 
\end{itemize}
For other type names, only a fixed normalisation can be performed.

\item[\moc{INFO}] keyword to specify the information associated 
with the detector type.

\item[\dusa{ndetect}] number of detectors of the specified type.

\item[\dusa{nrep}] number of detector response components for the specified 
type. It must be greater or equal to 2, corresponding to a response in
fraction and the reference flux value.
 
\item[\moc{SPECTRAL}] keyword to specify the energy spectral of
a detector type.

\item[\dusa{spec}] array containing the energy spectral of a detector type.

\item[\moc{DEFAULT}] keyword to specify the energy spectral will be
initialized as 1.0 for the highest energy group and 0.0 for other groups.

\item[\moc{INVCONST}] keyword to specify the inverse time constants of
the detector type model. This option is only valid for platinum, 
(\dusa{NAMTYP}(1:5) = 'PLATN'),
detector type. 
\item[\dusa{tinv}] array containing the inverse time constants of
the detector model.

\item[\moc{FRACTION}] keyword to specify the fractions corresponding to
each delayed or prompt reponse of the detector type model. This option is 
only valid for platinum, (\dusa{NAMTYP}(1:5) = 'PLATN'), detector type. 

\item[\dusa{frac}] array containing the detector type model fractions.

\item[\dstr{descdet}] structure describing the format used to read
detector information.
 
\end{ListeDeDescription}

\vskip 0.2cm

\subsubsection{Description of the detector data}

Note that the information input order must be respected. 

\begin{DataStructure}{Structure \dstr{descdet}}
\moc{NAME} \dusa{NAMDET} \\
$[$ \moc{NHEX} \dusa{nhex} \moc{HEX} ( \dusa{ihex}(i), i=1,\dusa{nhex} ) $]$ \\
\moc{POSITION} ( \dusa{pos}(i), i=1,6 )\\
\moc{RESP} ( \dusa{rep}(i), i=1,\dusa{nrep} )\\
\moc{ENDN} 
\end{DataStructure}

\noindent where
\begin{ListeDeDescription}{mmmmmmmm}

\item[\moc{NAME}] keyword to specify the detector name.

\item[\dusa{NAMDET}] \texttt{character*12} name of the detector.
The different names in alphabetical order must fit
their usual numbering in the core.(Ex: PLATN01, CHION01C) 

\item[\moc{NHEX}] keyword to set the number of hexagons where the detector
is placed.

\item[\dusa{nhex}] number of hexagons.

\item[\moc{HEX}] keyword to set the hexagon numbers corresponding to
the detector position.

\item[\dusa{ihex}] array containing the hexagon numbers
 where the detector is present, as ordered in the geometry definition. 

\item[\moc{POSITION}] keyword to specify the detector coordinates.

\item[\dusa{pos}] array containing the positions of the specified detector.
The positions must be read as X$-$ X+ Y$-$ Y+ Z$-$ Z+ . For 2-D geometry,
Z coordinates must be 0.0 and a value greater than 1.0. For hexagonal geometry,
only Z coordinates are used in 3-D representation.

\item[\moc{RESP}] keyword to specify the detector initial responses.

\item[\dusa{rep}] array containing the initial responses of the detector.
To use the current detector models in DONJON, responses are given as

\begin{itemize}
\item For vanadium detectors: current response, last response.
\item For platinum detectors: current response, reference flux, last
detector slow 
responses.
\item For ion chamber detectors: current logarithmic response, current
log rate response, 
reference flux.
\end{itemize}
 
\item[\moc{ENDN}] keyword to specify the end of the detector informations.

\end{ListeDeDescription}
\clearpage
