\subsection{The \moc{MACINI:} module}\label{sect:macini}

\vskip 0.2cm
The \moc{MACINI:} module is used to construct an extended \dds{macrolib},
in which the properties are stored per each material region over the whole
mesh-splitted reactor geometry. This \dds{macrolib} is obtained by combining
the material properties which are contained in the two distinct \dds{macrolib}
objects:

\begin{itemize}
\item The first \dds{macrolib} contains the material properties which are
evolution-independent, such as reflector and device properties. It is created
using either \moc{MAC:}, \moc{CRE:}, \moc{NCR:} or \moc{AFM:} module.
\item The second is a fuel-map \dds{macrolib} created using either \moc{CRE:},
\moc{NCR:} or \moc{AFM:} module. It must contain the interpolated fuel properties per
each fuel bundle.
\end{itemize}

The resulting \dds{macrolib} will contain the properties that are stored for each
reactor material and per each mesh-splitted volume. When the devices are not
present in the reactor core, then the resulting \dds{macrolib} can be considered
as a complete reactor \dds{macrolib} and it can be directly used for the numerical
solving. However, when the devices are inserted into the reactor core, the resulting
\dds{macrolib} is not yet complete; it must be subsequently updated with respect
to the device properties, using the \moc{NEWMAC:} module (see \Sect{newmac}).\\

\noindent
The \moc{MACINI:} module specification is:

\begin{DataStructure}{Structure \moc{MACINI:}}
\dusa{MACRO2} \dusa{MATEX} \moc{:=} \moc{MACINI:}
\dusa{MATEX} \dusa{MACRO} $[$ \dusa{MACFL}  $]$ \moc{::}
$[$ \moc{EDIT} \dusa{iprint} $]~[$ \moc{FUEL} $]$ ;
\end{DataStructure}

\noindent where
\begin{ListeDeDescription}{mmmmmmmm}

\item[\dusa{MACRO2}] \texttt{character*12} name of the extended
\dds{macrolib} to be created by the module.

\item[\dusa{MATEX}] \texttt{character*12} name of the \dds{matex}
object containing an extended material index over the reactor geometry.
\dusa{MATEX} must be specified in the modification mode; it will store
the recovered h-factors per each fuel region.

\item[\dusa{MACRO}] \texttt{character*12} name of a \dds{macrolib}, created
using either \moc{MAC:}, \moc{CRE:}, \moc{NCR:} or \moc{AFM:} module, for the evolution-independent
material properties (see structure \dstr{desccre1} or refer to the user guide\cite{dragon}).

\item[\dusa{MACFL}] \texttt{character*12} name of a fuel-map \dds{macrolib},
created using either \moc{CRE:}, \moc{NCR:} or \moc{AFM:} module, for the interpolated fuel
properties (see structure \dstr{desccre2} or refer to the user guide\cite{dragon}).

\item[\moc{EDIT}] keyword used to set \dusa{iprint}.

\item[\dusa{iprint}] integer index used to control the printing on screen: = 0 for
no print; = 1 for minimum printing; larger values produce increasing amounts of
output. The default value is \dusa{iprint} = 1.

\item[\moc{FUEL}] keyword used to indicate that \dusa{MACRO} is a fuel-map \dds{macrolib} in case where only two RHS
objects are defined. By default, \dusa{MACRO} contains evolution-independent cross sections.

\end{ListeDeDescription}
\clearpage
