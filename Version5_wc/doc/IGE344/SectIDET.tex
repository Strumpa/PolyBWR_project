\subsection{The \moc{IDET:} module}\label{sect:idet}

\vskip 0.2cm
The \moc{IDET:} module can perform an evaluation of fission chamber response in a PWR by integrating the fission rate over
the detector positions. This module is limited to Cartesian geometry.

\vskip 0.08cm
\noindent
The \moc{IDET:} module specification is:

\begin{DataStructure}{Structure \moc{IDET:}}\label{table:tidet}
\dusa{IDETEC} \moc{:=} \moc{IDET:} $[$ \dusa{IDETEC} $]$ \dusa{TRKNAM} \dusa{FLUNAM} \dusa{LIBNAM} $[$ \dusa{FMAP} $]$ \\
\moc{::} \dstr{descidet}
\end{DataStructure}

\noindent where

\begin{ListeDeDescription}{mmmmmmmm}

\item[\dusa{IDETEC}] {\tt character*12} name of a \dds{idetect} data structure,
({\tt L\_INTDETEC} signature) that will be created or updated by the \moc{IDET:} module.

\item[\dusa{TRKNAM}] {\tt character*12} name of the read-only \dds{tracking} data
structure ({\tt L\_TRACK} signature) containing the finite-element tracking.

\item[\dusa{FLUNAM}] {\tt character*12} name of the read-only \dds{fluxunk} data
structure ({\tt L\_FLUX} signature) containing the finite-element solution.

\item[\dusa{LIBNAM}] {\tt character*12} name of the read-only \dds{macrolib}
data structure ({\tt L\_LIBRARY} signature) that contains the interpolated microscopic
cross sections.

\item[\dusa{FMAP}] \texttt{character*12} name of the read-only  \dds{fmap} data structure
({\tt L\_MAP} signature) containing renumbered mixture indices. This object is optionnal.

\item[\dstr{descidet}] structure describing the input data to the \moc{IDET:} module.

\end{ListeDeDescription}

\subsubsection{Input data to the \moc{IDET:} module}\label{sect:stridet}

\begin{DataStructure}{Structure \dstr{descidet}}
$[$ \moc{EDIT} \dusa{iprint} $]$ \\
$[~\{$ \moc{NOCCOR} $|$ \moc{CCOR} $\}~]$ \\
$[$ \moc{DETNAME} \dusa{dname} $]~[$ \moc{REANAME} \dusa{rname} $]$ \\
\moc{DETECTOR} \\
\hspace{0.3cm} $[[$ \moc{POSITION} $\{$ \dusa{valx} $|$ \moc{INTEG} \dusa{valx1} \dusa{valx2} $\}~\{$ \dusa{valy} $|$ \moc{INTEG} \dusa{valy1} \dusa{valy2} $\}$ \\
\hspace{2.02cm} $[~\{$ \dusa{valz} $|$ \moc{INTEG} \dusa{valz1} \dusa{valz2} $\}~]~]]$ \\
\moc{ENDD} \\
;
\end{DataStructure}

\noindent where
\begin{ListeDeDescription}{mmmmmmmm}

\item[\moc{EDIT}] keyword used to set \dusa{iprint}.

\item[\dusa{iprint}] integer index used to control the printing on screen:
 = 0 for no print; = 1 for minimum printing (default value); =2 for more printouts.

\item[\moc{NOCCOR}] keyword used to deactivate {\sl corner flux correction} with 2D/3D nodal methods.

\item[\moc{CCOR}] keyword used to activate {\sl corner flux correction} with 2D/3D nodal methods (default option).

\item[\moc{DETNAME}] keyword used to set \dusa{dname}, the alias name of the isotope used as detector. By default, \dusa{dname}$=${\tt U235} is used.

\item[\dusa{dname}] character*12 alias name of the isotope used as detector.

\item[\moc{REANAME}] keyword used to set \dusa{rname}, the name of the nuclear reaction used as detector. By default, \dusa{rname}$=${\tt NFTOT} is used.

\item[\dusa{rname}] character*12 name of the nuclear reaction used as detector.

\item[\moc{POSITION}] keyword defining the position of a single detector.

\item[\moc{INTEG}] keyword indicating that the detector reading will be averaged between two Cartesian positions.

\item[\dusa{valx}] position (real number) of the detector along $X$ axis.

\item[\dusa{valx1}] starting position (real number) of the detector along $X$ axis.

\item[\dusa{valx2}] ending position (real number) of the detector along $X$ axis. We must have \dusa{valx1}$<$\dusa{valx2}.
 
\item[\dusa{valy}] position (real number) of the detector along $Y$ axis.

\item[\dusa{valy1}] starting position (real number) of the detector along $Y$ axis.

\item[\dusa{valy2}] ending position (real number) of the detector along $Y$ axis. We must have \dusa{valy1}$<$\dusa{valy2}.
 
\item[\dusa{valz}] position (real number) of the detector along $Z$ axis. Detector position along $Z$ axis is given only for 3D geometries.

\item[\dusa{valz1}] starting position (real number) of the detector along $Z$ axis.

\item[\dusa{valz2}] ending position (real number) of the detector along $Z$ axis. We must have \dusa{valz1}$<$\dusa{valz2}.

\end{ListeDeDescription}
\clearpage
