\subsection{WIMSD4 microscopic cross-section examples.}\label{sect:ExWWIMSD4} 

The test cases we will consider here use the \moc{LIB:} module to enter
microscopic cross sections taken from a WIMSD4 69 groups library. We will assume
that this library is located in file {\tt iaea}. The test cases are numbered
successively from \tst(TCWU01) to \tst(TCWU31). 


\subsubsection{\tst(TCWU01) -- The Mosteller benchmark.}

\begin{figure}[h!]  
\begin{center} 
\epsfxsize=5cm \centerline{ \epsffile{GTCW01.eps}}
\parbox{14cm}{\caption{Geometry for the  Mosteller
benchmark problem.}\label{fig:TCWU01}}   
\end{center}  
\end{figure}

This benchmark uses both a cartesian 2-D cell with a central annular pin and an
equivalent annular cell.\cite{Mostel} No depletion information is required in
this case since the module \moc{EVO:} will not be executed. A comparison between
various calculation options is provided here. We first consider the annular
geometry with a \moc{SYBILT:} self-shielding and a \moc{SYBILT:} transport
calculation. This is then repeated for the cartesian 2--D cell. Finally, we used
an isotropic (\moc{TISO}) and a specular (\moc{TSPC}) \moc{EXCELT:} tracking
successively for the self-shielding and transport calculations.

\listing{TCWU01.x2m}

\subsubsection{\tst(TCWU02) -- A $17\times 17$ PWR type assembly}

\begin{figure}[h!]  
\begin{center} 
\epsfxsize=12cm \centerline{ \epsffile{GTCW02.eps}}
\parbox{14cm}{\caption{Geometry for test case \tst(TCWU02).}\label{fig:TCWU02}}
\end{center}  
\end{figure}

This test case represents a production calculation of a normal PWR assembly
with cell grouping (\moc{MERGE} and \moc{TURN} options). Its configuration is
shown in \Fig{TCWU02}.

\listing{TCWU02.x2m}

\subsubsection{\tst(TCWU03) -- An hexagonal assembly}

\begin{figure}[h!]  
\begin{center} 
\epsfxsize=11cm \centerline{ \epsffile{GTCW03.eps}}
\parbox{14cm}{\caption{Geometry for test case \tst(TCWU03).}\label{fig:TCWU03}}   
\end{center}   \end{figure}

This test case represents a production calculation of a
typical hexagonal control assembly. Its configuration is presented in
\Fig{TCWU03}.
                                                            
\listing{TCWU03.x2m}

\subsubsection{\tst(TCWU04) -- A Cylindrical cell with burnup.}

\begin{figure}[h!]  
\begin{center} 
\epsfxsize=15cm \centerline{ \epsffile{GTCA04.eps}}
\parbox{14cm}{\caption{Depletion chain of heavy
 isotopes.}\label{fig:TCA04}}    \end{center}   \end{figure}

This test case represents a burnup calculation for the mosteller annular
geometry.

\listing{TCWU04.x2m}

\subsubsection{\tst(TCWU05) -- A CANDU-6 type annular cell with burnup.}

\begin{figure}[h!]  
\begin{center} 
\epsfxsize=9cm \centerline{ \epsffile{GTCW05.eps}}
\parbox{14cm}{\caption{Geometry of the CANDU-6 cell.}\label{fig:TCWU05}}   
\end{center}  
\end{figure}

This test case represents the typical CANDU type cell with an annular moderator
region defined in \Fig{TCWU05}. Both its cross section and depletion data are
taken from the same WIMSD4 file. Depletion calculations are performed for 50 day at
a fixed power.\cite{Mtl93b} The {\sc microlib} is defined by the procedure
{\tt TCWU05Lib.c2m} presented in \Sect{TCWU05Lib}.

\listing{TCWU05.x2m}

\subsubsection{\tst(TCWU06) -- A CANDU-6 type supercell with control rods.}

This test case treats both the CANDU cell with a cartesian moderator region
(similar to the cell described in defined \Fig{TCWU05}) and the
supercell containing a stainless steel rod which can be either in the inserted
or extracted position (see \Fig{TCM04}). Two groups incremental cross sections
corresponding to the rod in the inserted and extracted position with respect to
the original supercell containing only 3--D fuel elements are computed.\cite{Mtl93b}
The {\sc microlib} is defined by the procedure {\tt TCWU05Lib.c2m} presented in \Sect{TCWU05Lib}.

\listing{TCWU06.x2m}


\subsubsection{\tst(TCWU07) -- A CANDU-6 type calculation using various leakage
options.}

This test case treats the CANDU cell with a cartesian moderator region
(similar to the cell described in defined \Fig{TCWU05}) using various leakage
options. The {\sc microlib} is defined by the procedure {\tt TCWU05Lib.c2m} presented in \Sect{TCWU05Lib}.


\listing{TCWU07.x2m}

\subsubsection{\tst(TCWU08) -- Burnup of an homogeneous cell.}

This case illustrate the burnup of an homogeneous cell that spends the first
1000 days in a reactor before being removed. The depletion of the isotopes in
this cell for an additional 1000 days outside of the core is also investiguated.

\listing{TCWU08.x2m}

\subsubsection{\tst(TCWU09) -- Testing boundary conditions.}

This case test different boundary conditions for the Mosteller cell.

\listing{TCWU09.x2m}

\subsubsection{\tst(TCWU10) -- Fixed source problem in multiplicative media.}

This case verify the use of a fixed source inside a cell where fission also
takes place.

\listing{TCWU10.x2m}

\subsubsection{\tst(TCWU11) -- Two group burnup of a CANDU-6 type cell.}

This case is similar to \tst(TCWU05) except that the burnup module
uses DRAGON generated two groups time dependent microscopic cross sections.
The {\sc microlib} is defined by the procedure {\tt TCWU05Lib.c2m} presented in \Sect{TCWU05Lib}.

\listing{TCWU11.x2m}

\subsubsection{\tst(TCWU12) -- Mixture composition.}

This case illustrates the use of the \moc{INFO:} module of DRAGON 
(see \Sect{INFOData}) as well as the new \moc{COMB} option in the module 
\moc{LIB:} (see \Sect{LIBData}).

\listing{TCWU12.x2m}

\subsubsection{\tst(TCWU13) -- Solution by the method of cyclic characteristics}\label{sect:ExTCWU13}

This case illustrates the use of the \moc{MOCC:} module of DRAGON for a solution by the transport equation by the method of cyclic
characteristics. This test case also uses the embedded DRAGON procedure stored in the {\tt TCWU05Lib.c2m} file.

\listing{TCWU13.x2m}

\subsubsection{\tst(TCWU14) -- SPH Homogenisation without tracking}\label{sect:ExTCWU14}

This case illustrates the use of the \moc{SPH} homogenisation procedure in the \moc{EDI:} module of DRAGON when a tracking data
structure is provided as input. This test case also uses the embedded DRAGON procedure stored in the {\tt TCWU05Lib.c2m} file.

\listing{TCWU14.x2m}

\subsubsection{\tst(TCWU15) -- A CANDU--6 type Cartesian cell with burnup}\label{sect:ExTCWU15}

This test case is similar to \tst{TCWU05} except that the cell boundary are Cartesian and the \moc{NXT:} tracking module is used. It uses the embedded DRAGON procedure stored in the {\tt TCWU05Lib.c2m} file.

\listing{TCWU15.x2m}

\subsubsection{\tst(TCWU17) -- A 2-D CANDU--6 supercell with control rods}\label{sect:ExTCWU17}

\begin{figure}[h!]  
\begin{center} 
\parbox{15.0cm}{\epsfxsize=15cm \epsffile{GTCW17.eps}}
\parbox{14cm}{\caption{Geometry of 2-D CANDU--6 supercell with control rods.}\label{fig:TCWU17}}   
\end{center}  
\end{figure}

This test case treats a 2-D CANDU--6 supercell containing fuel clusters and control rods (see \Fig{TCWU17}). The use of the virtual homogenization mixtures defined by \moc{HMIX} is also illustrated. This test case uses the embedded DRAGON procedure
stored in the {\tt TCWU17Lib.c2m} file.
 
\listing{TCWU17.x2m}

\subsubsection{\tst(TCWU17Lib) -- Microlib definition.}\label{sect:TCWU17Lib}

This CLE-2000 procedure is used in data-set {\tt TCWU17} to define the {\sc microlib} isotopic content.

\listing{TCWU17Lib.c2m}

\subsubsection{\tst(TCWU31) -- Compo-based two group burnup of a CANDU-6 type cell.}

This case is similar to \tst(TCWU11) except that the two-group burnup calculation
recover all its information from a {\sc compo} database.
The {\sc microlib} is defined by the procedure {\tt TCWU05Lib.c2m} presented in \Sect{TCWU05Lib}.

\listing{TCWU31.x2m}

\subsubsection{\tst(TCWU05Lib) -- Microlib definition.}\label{sect:TCWU05Lib}

This CLE-2000 procedure is used in previous data-sets to define the {\sc microlib} isotopic content.

\listing{TCWU05Lib.c2m}
\eject
