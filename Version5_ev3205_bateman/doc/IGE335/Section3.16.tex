\subsection{The {\tt COMPO:} module}\label{sect:COMPOData}

This component of the lattice code is dedicated to the constitution of the
reactor database intended to store {\sl all} the nuclear data, produced in
the lattice code, that is useful
in reactor calculations including fuel management and space-time kinetics.
Multigroup lattice calculations are too expensive to be executed dynamically
from the driver of the global reactor calculation. A more feasible
approach is to create a reactor database where a finite number of lattice
calculation results are tabulated against selected {\sl global} and/or {\sl local parameters}
chosen so as to represent expected operating conditions of the reactor.

\vskip 0.1cm

The {\tt COMPO:} module is used to create and construct a {\sc multicompo} object.
This object is generally {\sl persistent} and used to collect information gathered
from many DRAGON {\sl elementary calculations} performed under various conditions.

\vskip 0.1cm

For each elementary calculation, the results are recovered from the output of the
{\tt EDI:} module and stored in a list of {\sl homogenized mixture}
directories. The {\tt EDI:} module is responsible for performing condensation
in energy, homogenization in space of the microscopic cross sections and constitution
of {\sl macroscopic sets} for collecting together many isotopes. All the elementary
calculations gathered in a single {\sc multicompo} object are characterized by the same
number of {\sl homogenized mixtures} and by a specific output energy-group structure.

\vskip 0.1cm

\begin{figure}[h!]
\begin{center} 
\epsfxsize=9cm
\centerline{ \epsffile{compo.eps}}
\parbox{15cm}{\caption{Organization of a {\tt multicompo} 
object.}\label{fig:compo}}   
\end{center}  
\end{figure}

Each elementary calculation is characterized by a tuple of {\sl global} and/or {\sl local parameters}
Global parameters are characteristics of the complete lattice, while local parameters
are characteristics of each homogenized mixture. These parameters are of different types,
depending on the nature of the
study under consideration: type of assembly, power, temperature in a mixture,
concentration of an isotope, time, burnup or exposure rate in a depletion calculation,
etc. Each step of a depletion calculation represents an elementay calculation.
The {\sc multicompo} object is often presented as a {\sl multi-parameter reactor database}.

\vskip 0.1cm

\begin{figure}[h!]  
\begin{center} 
\epsfxsize=9.5cm
\centerline{ \epsffile{tree.eps}}
\parbox{14cm}{\caption{Parameter tree in a {\sc multicompo} object}\label{fig:ctree}}   
\end{center}  
\end{figure}

The {\sc multicompo} object is organized as shown in \Fig{compo}. The root of the object contains
table--of--content information for global and local parameters and two lists of
directories. Each component of the first list ({\tt 'MIXTURES'})
contains the directory {\tt 'TREE'} (the parameter tree) and the list ({\tt 'CALCULATIONS'})
made of {\sc microlib} objects. Each component of the second list ({\tt 'GEOMETRIES'}) contains the homogenized
geometry of an elementary calculation.

\vskip 0.1cm

The localization of an elementary calculation is done using
a tuple of global and/or local parameters. The elementary calculation indices are
stored in a tree with the number of levels equal to the number of global and local parameters.
An example of a tree with three parameters is shown in \Fig{ctree}. Each node
of this tree is associated with the index of the corresponding parameter and with the
reference to the daughter nodes if they exist. The number if leafs is equal to the number
of nodes for the last (third) parameter and is equal to the number of elementary
calculations stored in the {\sc multicompo} object. The index of each elementary calculation is
therefore an attribute of each leaf.

\vskip 0.1cm

In each homogenized mixture component, the {\tt COMPO:} module recover
cross sections for a number of {\sl particularized isotopes} and of a single {\sl macroscopic
set}, a collection of the remaining isotopic cross sections weighted by isotopic number densities.
Other information is also recovered: multigroup neutron
fluxes, isotopic number densities, fission spectrum, delayed neutron data, etc.

\vskip 0.1cm

A different specification of the \moc{COMPO:} function call is used for
creation and construction of the {\sc multicompo} object.
\begin{itemize}
\item The first specification is used to initialize the {\sc multicompo} data structure
and to set the choice of global and local parameters.
\item A modification call to the \moc{COMPO:} function is performed after each
elementary calculation in order to recover output information processed by \moc{EDI:}
(condensed and homogenized cross sections) and \moc{EVO:} (burnup dependant values).
Global and local parameters can optionnally be recovered from \dds{microlib}
objects.
\item Another modification call to the \moc{COMPO:} function is used to
catenate a {\sl read-only} {\sc multicompo} object into a {\sl master} {\sc multicompo} object.
\end{itemize}

The calling specifications are:

\begin{DataStructure}{Structure \dstr{COMPO:}}
$\{$~~\dusa{CPONAM} \moc{:=} \moc{COMPO:} $[$ \dusa{CPONAM} $]$ \moc{::} \dstr{compo\_data1} \\
$|$~~\dusa{CPONAM} \moc{:=} \moc{COMPO:} \dusa{CPONAM}~\dusa{EDINAM}~$[$ \dusa{EDINA2} $]~[$ \dusa{BRNNAM} $]~[$~\dusa{HMIC1}~$[$~\dusa{HMIC2} $]~]$ \\
~~~~~~~~~~~~~~ \moc{::} \dstr{compo\_data2} \\
$|$~~\dusa{CPONAM} \moc{:=} \moc{COMPO:} \dusa{CPONAM} \dusa{CPORHS} \moc{::} \dstr{compo\_data3} \\
$|$~~\moc{COMPO:} \dusa{CPONAM} \moc{::} \dstr{compo\_data4}~~$\}$ \\
\end{DataStructure}

\noindent where
\begin{ListeDeDescription}{mmmmmmm}

\item[\dusa{CPONAM}] {\tt character*12} name of the {\sc lcm} object containing the
{\sl master} {\sc multicompo} data structure.

\item[\dusa{EDINAM}] {\tt character*12} name of the {\sc lcm} object (type {\tt
L\_EDIT}) containing the {\sc edition} data structure corresponding to an elementary
calculation. This {\sc edition} data structure is containing homogenized and
condensed cross-section information. The {\sc edition} data produced by the last call
to the {\tt EDI:} module is used. It is possible to provide a {\sc macrolib} data structure
as replacement for the {\sc edition} data structure. In this case, the \moc{MACRO} keyword
is automatically set.

\item[\dusa{EDINA2}] {\tt character*12} name of an optional {\sc lcm} object (type {\tt
L\_EDIT}) containing the {\sc edition} data structure corresponding to an elementary
calculation. This {\sc edition} data structure is containing {\sl group form factor}
information. The {\sc edition} data produced by the last call to the {\tt EDI:} module
is used.

\item[\dusa{BRNNAM}] {\tt character*12} name of the {\sc lcm} object (type {\tt
L\_BURNUP}) containing the {\sc burnup} data structure.

\item[\dusa{HMIC1}] {\tt character*12} name of a \dds{microlib} (type {\tt
L\_LIBRARY}) containing global parameter information.

\item[\dusa{HMIC2}] {\tt character*12} name of a \dds{microlib} (type {\tt
L\_LIBRARY}) containing global parameter information.

\item[\dusa{CPORHS}] {\tt character*12} name of the {\sl read-only} {\sc multicompo} data structure. This
data structure is concatenated to \dusa{CPONAM} using the \dusa{compo\_data3} data structure,
as presented in \Sect{desccpo3}. \dusa{CPORHS} must be defined with the same number of energy
groups and the same number of homogeneous regions as \dusa{CPONAM}. Moreover, all the
global and local parameters of \dusa{CPORHS} must be defined in \dusa{CPONAM}. \dusa{CPONAM}
may be defined with {\sl global} parameters not defined in \dusa{CPORHS}.

\item[\dusa{compo\_data1}] input data structure containing initialization information (see \Sect{desccpo1}).

\item[\dusa{compo\_data2}] input data structure containing information related to the recovery of an
elementary calculation (see \Sect{desccpo2}).

\item[\dusa{compo\_data3}] input data structure containing information related to the catenation of a
{\sl read-only} {\sc multicompo} (see \Sect{desccpo3}).

\item[\dusa{compo\_data4}] input data structure containing information related to the display of a
{\sl read-only} {\sc multicompo} (see \Sect{desccpo4}).

\end{ListeDeDescription}

\subsubsection{Initialization data input for module {\tt COMPO:}}\label{sect:desccpo1}

\vskip -0.5cm

\begin{DataStructure}{Structure \dstr{compo\_data1}}
$[$~\moc{EDIT} \dusa{iprint}~$]$ \\
$[[~[$ \moc{STEP} \moc{UP} \dusa{NAMDIR} $]$ \\
~~~$[$~\moc{MAXCAL} \dusa{maxcal}~$]$ \\
~~~$[$~\moc{COMM}~$[[$~\dusa{HCOM}~$]]$~\moc{ENDC}~$]$ \\
~~~$[[$~\moc{PARA}~\dusa{PARKEY} \\
~~~~~~\{~\moc{TEMP}~\dusa{HMIC}~\dusa{imix}~$|$~\moc{CONC}~\dusa{HISO1}~\dusa{HMIC}~\dusa{imix}~$|$~\moc{IRRA}~$|$~\moc{FLUB}~$|$ \\
~~~~~~~~~\moc{POWR}~$|$~\moc{MASL}~$|$~\moc{FLUX}~$|$~\moc{TIME}~$|$~\moc{VALU}~\{~\moc{REAL}~$|$~\moc{CHAR}~$|$~\moc{INTE}~\}~\} \\
~~~$]]$ \\
~~~$[[$~\moc{LOCA}~\dusa{PARKEY} \\
~~~~~~\{~\moc{TEMP}~$|$~\moc{CONC}~\dusa{HISO2}~$|$~\moc{IRRA}~$|$~\moc{FLUB}~$|$~\moc{FLUG}~$|$~\moc{POWR}~$|$~\moc{MASL}~$|$~\moc{FLUX}~\} \\
~~~$]]$ \\
$[$~\moc{ISOT}~\dusa{nisp} (\dusa{HISOP}(i),i=1,\dusa{nisp})~$]$ \\
$[$ \moc{GFF} $]~[~\{$ \moc{NOALBP} $|$ \moc{ALBP} $\}~]~[~\{$ \moc{NOJSURF} $|$ \moc{JSURF} $\}~]$ \\
\moc{INIT} $]]$ \\
{\tt ;}
\end{DataStructure}

\noindent where
\begin{ListeDeDescription}{mmmmmmmm}

\item[\moc{EDIT}] keyword used to set \dusa{iprint}.

\item[\dusa{iprint}] index used to control the printing in module {\tt
COMPO:}. =0 for no print; =1 for minimum printing (default value).

\item[\moc{STEP}] keyword used to create the database from a sub-directory named \dusa{NAMDIR}. This capability
make possible the creation of a single object with many independent {\sc multicompo} structures in it. By default,
the database is created on directory {\tt 'default'}.

\item[\moc{UP}] keyword used to move up towards a sub-directory of \dusa{CPONAM}.

\item[\dusa{NAMDIR}] create the {\sc multicompo} structure in the sub-directory named \dusa{NAMDIR}.

\item[\moc{MAXCAL}] keyword used to set \dusa{maxcal}.

\item[\dusa{maxcal}] maximum number of elementary calculations to be stored
in the {\sc multicompo}. \dusa{maxcal}$=10$ by default. This maximum size is
automatically increased when the number of elementary calculations exceeds
the current value of \dusa{maxcal}.

\item[\moc{COMM}] keyword used to input a general comment for the {\sc multicompo}.

\item[\dusa{HCOM}] {\tt character*80} user-defined comment.

\item[\moc{ENDC}] end--of--HCOM keyword.

\item[\moc{PARA}] keyword used to define a single global parameter.

\item[\moc{LOCA}] keyword used to define a single local parameter.

\item[\dusa{PARKEY}] {\tt character*12} user-defined keyword associated to a global
or local parameter.

\item[\dusa{HMIC}] {\tt character*12} name of the \dds{microlib} (type {\tt
L\_LIBRARY}) associated to a global parameter. The corresponding \dds{microlib} will be required on
RHS of the \moc{COMPO:} call described in Sect.~\ref{sect:desccpo2}.

\item[\dusa{imix}] index of the mixture associated to a global parameter. This mixture is
located in \dds{microlib} named \dusa{HMIC}.

\item[\dusa{HISO1}] {\tt character*8} alias name of the isotope associated to a global
parameter. This isotope is located in \dds{microlib} data structure named \dusa{HMIC}.

\item[\dusa{HISO2}] {\tt character*8} alias name of the isotope associated to a local
parameter. This isotope is located in the \dds{microlib} directory of the {\sc edition}
data structure named \dusa{EDINAM}.

\item[\moc{TEMP}] keyword used to define a temperature (in Kelvin) as global or
local parameter.

\item[\moc{CONC}] keyword used to define a number density as global or
local parameter.

\item[\moc{IRRA}] keyword used to define a burnup (in MWday/Tonne) as global
or local parameter.

\item[\moc{FLUB}] keyword used to define a {\sl fuel-only} exposure rate (in n/kb) as global
or local parameter. The exposure rate is recovered from the \dusa{BRNNAM}
LCM object.

\item[\moc{FLUG}] keyword used to define an exposure rate in global homogenized mixtures (in n/kb) as
local parameter. The exposure rate is recovered from the \dusa{BRNNAM}
LCM object.

\item[\moc{POWR}] keyword used to define the power as global or
local parameter.

\item[\moc{MASL}] keyword used to define the mass density of heavy isotopes as
global or local parameter.

\item[\moc{FLUX}] keyword used to define the volume-averaged, energy-integrated flux as
global or local parameter.

\item[\moc{TIME}] keyword used to define the time (in seconds) as global parameter.

\item[\moc{VALU}] keyword used to define a user-defined quantity as global parameter.
This keyword must be followed by the type of parameter.

\item[\moc{REAL}] keyword used to indicate that the user-defined global parameter
is a floating point value.

\item[\moc{CHAR}] keyword used to indicate that the user-defined global parameter
is a {\tt character*12} value.

\item[\moc{INTE}] keyword used to indicate that the user-defined global parameter
is an integer value.

\item[\moc{ISOT}] keyword used to select the set of particularized isotopes. By default, all the
isotopes available in the {\sc edition} data structure \dusa{EDINAM} are selected.

\item[\dusa{nisp}] number of user-defined particularized isotopes.

\item[\dusa{HISOP}] {\tt character*8} names of the user-defined particularized isotopes. These names must be present
in the {\sc edition} data structure \dusa{EDINAM}.

\item[\moc{GFF}] keyword used to enable the recovery of group form factor information from {\sc edition} data structure \dusa{EDINA2}.

\item[\moc{NOALBP}] keyword used to avoid the recovery of physical albedo information from {\sc edition} data structure \dusa{EDINAM}.

\item[\moc{ALBP}] keyword used to enable the recovery of physical albedo information from {\sc edition} data structure \dusa{EDINAM} (default option).

\item[\moc{NOJSURF}] keyword used to avoid the recovery of discontinuity factor and boundary multigroup current information from {\sc edition} data structure \dusa{EDINAM}.

\item[\moc{JSURF}] keyword used to enable the recovery of discontinuity factor and boundary multigroup current information from {\sc edition} data structure \dusa{EDINAM} (default option).

\item[\moc{INIT}] keyword used to create the empty structure in the {\sc multicompo}.

\end{ListeDeDescription}

\clearpage

\subsubsection{Modification data input for module {\tt COMPO:}}\label{sect:desccpo2}

\vskip -0.5cm

\begin{DataStructure}{Structure \dstr{compo\_data2}}
$[$ \moc{EDIT} \dusa{iprint} $]$ \\
$[$ \moc{ALLX} $]$ \\
$[$ \moc{STEP} \moc{UP} $\{$ \dusa{NAMDIR} $|$ \moc{*} $\}~]$ \\
$[$ \moc{ORIG} \dusa{orig} $]$ \\
$[[$ \dusa{PARKEY} \dusa{value} $]]$ \\
$[$ \moc{MACRO} $]~[$ \moc{SET} \dusa{xtr} $\{$ \moc{S} $|$ \moc{DAY} $|$ \moc{YEAR} $\}$ $]$ \\
{\tt ;}
\end{DataStructure}

\noindent where
\begin{ListeDeDescription}{mmmmmmmm}

\item[\moc{EDIT}] keyword used to set \dusa{iprint}.

\item[\dusa{iprint}] index used to control the printing in module {\tt
COMPO:}. =0 for no print; =1 for minimum printing (default value).

\item[\moc{ALLX}] keyword used to register the region number of each isotope before merging. This option is useful if the same
keyword has been specified in \moc{EDI:} before. This allows to perform subsequent depletion calculations, in taking into account
different fuel regions in the diffusion calculation.

\item[\moc{STEP}] keyword used to access the database from a sub-directory named \dusa{NAMDIR} instead of
accessing it from the root of \dusa{CPONAM}.

\item[\moc{UP}] keyword used to move up towards a sub-directory of \dusa{CPONAM}.

\item[\dusa{NAMDIR}] access the {\sc multicompo} structure in the sub-directory named \dusa{NAMDIR}.

\item[\moc{*}] use a sub-directory name identical to the directory in \dusa{EDINAM}
where the edition data is coming from.

\item[\moc{ORIG}] keyword used to define the father node in the parameter tree. By
default, the index of the previous elementary calculation is used.

\item[\dusa{orig}] index of the elementary calculation associated to the father node in the
parameter tree.

\item[\dusa{PARKEY}] {\tt character*12} keyword associated to a user-defined
global parameter.

\item[\dusa{value}] floating-point, integer or {\tt character*12} value of a user-defined
global parameter.

\item[\moc{MACRO}] keyword used to recover cross-section information from the
macrolib directory in \dusa{EDINAM}. By default, the cross-section information
is recovered from the microlib in \dusa{EDINAM}.

\item[\moc{SET}] keyword used to recover the flux normalization factor already
stored on \dusa{BRNNAM} from a sub-directory corresponding to a specific time.

\item[\dusa{xtr}] time associated with the current flux calculation. The
name of the sub-directory where this information is stored will be given by
`{\tt DEPL-DAT}'//{\tt CNN} where {\tt CNN} is a  {\tt character*4} variable
defined by  {\tt WRITE(CNN,'(I4.4)') INN} where {\tt INN} is an index associated
with the time \dusa{xtr}.

\item[\moc{S}] keyword to specify that the time is given in seconds.

\item[\moc{DAY}] keyword to specify that the time is given in days.

\item[\moc{YEAR}] keyword to specify that the time is given in years.

\end{ListeDeDescription}

\subsubsection{Modification (catenate) data input for module {\tt COMPO:}}\label{sect:desccpo3}

\vskip -0.5cm

\begin{DataStructure}{Structure \dstr{compo\_data3}}
$[$ \moc{EDIT} \dusa{iprint} $]$ \\
$[$ \moc{STEP} \moc{UP} \dusa{NAMDIR} $]$ \\
$[$ \moc{ORIG} \dusa{orig} $]$ \\
$[[$ \dusa{PARKEY} \dusa{value} $]]$ \\
$[$ \moc{WARNING-ONLY} $]$ \\
{\tt ;}
\end{DataStructure}

\noindent where
\begin{ListeDeDescription}{mmmmmmmm}

\item[\moc{EDIT}] keyword used to set \dusa{iprint}.

\item[\dusa{iprint}] index used to control the printing in module {\tt
COMPO:}. =0 for no print; =1 for minimum printing (default value).

\item[\moc{STEP}] keyword used to access the database from a sub-directory named \dusa{NAMDIR} instead of
accessing it from the root of \dusa{CPONAM}.

\item[\moc{UP}] keyword used to move up towards a sub-directory of \dusa{CPONAM}.

\item[\dusa{NAMDIR}] access the {\sc multicompo} structure in the sub-directory named \dusa{NAMDIR}.

\item[\moc{ORIG}] keyword used to define the father node in the parameter tree. By
default, the index of the previous elementary calculation is used.

\item[\dusa{orig}] index of the elementary calculation associated to the father node in the
parameter tree.

\item[\dusa{PARKEY}] {\tt character*12} keyword associated to a
global parameter that is specific to \dusa{CPONAM} (not defined in \dusa{CPORHS}).

\item[\dusa{value}] floating-point, integer or {\tt character*12} value of a
global parameter that is specific to \dusa{CPONAM}.

\item[\moc{WARNING-ONLY}] This option is useful if an elementary calculation in \dusa{CPORHS} 
is already present in \dusa{CPONAM}. If this keyword is set, a warning is send and the \dusa{CPONAM} values
are kept, otherwise the run is aborted (default).

\end{ListeDeDescription}

\clearpage

\subsubsection{Display data input for module {\tt COMPO:}}\label{sect:desccpo4}

\vskip -0.5cm

\begin{DataStructure}{Structure \dstr{compo\_data4}}
$[$ \moc{EDIT} \dusa{iprint} $]$ \\
$[$ \moc{STEP} \moc{UP} \dusa{NAMDIR} $]$ \\
 \moc{DB-STRUC} \\
{\tt ;}
\end{DataStructure}

\noindent where
\begin{ListeDeDescription}{mmmmmmmm}

\item[\moc{EDIT}] keyword used to set \dusa{iprint}.

\item[\dusa{iprint}] index used to control the printing in module {\tt
COMPO:}. $<$2 for MUPLET display only (default value) and parameters values are presented at the end, $\ge$2 for the parameter value display for each calculation.

\item[\moc{STEP}] keyword used to access the database from a sub-directory named \dusa{NAMDIR} instead of
accessing it from the root of \dusa{CPONAM}.

\item[\moc{UP}] keyword used to move up towards a sub-directory of \dusa{CPONAM}.

\item[\dusa{NAMDIR}] access the {\sc multicompo} structure in the sub-directory named \dusa{NAMDIR}.

\item[\moc{DB-STRUC}] {\tt character*12} keyword used to display the content of the \dusa{CPONAM} object for the \dusa{NAMDIR} directory.

\end{ListeDeDescription}

\clearpage
