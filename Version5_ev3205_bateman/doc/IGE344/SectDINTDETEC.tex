\subsection{Contents of a \dir{intdetec} data structure}\label{sect:intdetecdir}

\vskip 0.2cm
The \dir{intdetec} data structure is used to store integrated detector positions and 
responses in a PWR. This object has a signature {\tt L\_INTDETEC}; it is created using
the \moc{IDET:} module.

\subsubsection{The state-vector content}\label{sect:devicestate}
\noindent
The dimensioning parameters $\mathcal{S}_i$, which are stored in the state
vector for this data structure, represent:

\begin{itemize}

\item  The first dimension of matrices {\tt COORD1}, {\tt COORD2} and {\tt COORD3}, representing
the maximum number of interpolation points for a single detector. $N_{\rm max} = \mathcal{S}_1$.

\item  The total number of detectors. $N_{\rm det} = \mathcal{S}_2$. 

\end{itemize}

\subsubsection{The main \dir{intdetec} directory}\label{sect:intdetecdir}
\noindent
The following records will be found on the first level of \dir{intdetec} directory:

\begin{DescriptionEnregistrement}{Records in \dir{intdetec} data structure}{7cm} 

\CharEnr
{SIGNATURE\blank{3}}{$*12$}
{Signature of the \dir{intdetec} data structure ($\mathsf{SIGNA}=${\tt L\_INTDETEC\blank{2}}).}

\IntEnr
{STATE-VECTOR}{$40$}
{Vector describing the various parameters associated with this data structure $\mathcal{S}_i$}

\IntEnr
{NINX\blank{8}}{$\mathcal{S}_2$}
{Number of interpolation point for the detector reading along $X$ axis.}

\IntEnr
{NINY\blank{8}}{$\mathcal{S}_2$}
{Number of interpolation point for the detector reading along $Y$ axis.}

\IntEnr
{NINZ\blank{8}}{$\mathcal{S}_2$}
{Number of interpolation point for the detector reading along $Z$ axis.}

\RealEnr
{COORD1\blank{6}}{$\mathcal{S}_1,\mathcal{S}_2$}{cm}
{Values of the interpolation coordinates along $X$ axis.}

\RealEnr
{COORD2\blank{6}}{$\mathcal{S}_1,\mathcal{S}_2$}{cm}
{Values of the interpolation coordinates along $Y$ axis.}

\RealEnr
{COORD3\blank{6}}{$\mathcal{S}_1,\mathcal{S}_2$}{cm}
{Values of the interpolation coordinates along $Z$ axis.}

\CharEnr
{DETNAM\blank{6}}{$*12$}
{Alias name of the isotope used as detector.}

\CharEnr
{REANAM\blank{6}}{$*12$}
{Name of the nuclear reaction used as detector.}

\RealEnr
{RESPON\blank{6}}{$\mathcal{S}_2$}{}
{The integrated responses of the detectors.}

\end{DescriptionEnregistrement}
\clearpage
