\subsection{Contents of \dir{thm} data structure}\label{sect:thmdir}

This data structure contains the thermal-hydraulics information required in a multi-physics calculation

\subsubsection{The main \dir{thm} directory}\label{sect:thmdirmain}

The following records and sub-directories will be found in the first level of a \dir{thm} directory:

\begin{DescriptionEnregistrement}{Main records and sub-directories in \dir{thm}}{8.0cm}
\CharEnr
  {SIGNATURE\blank{3}}{$*12$}
  {parameter $\mathsf{SIGNA}$ containing the signature of the data structure}
\IntEnr
  {STATE-VECTOR}{$40$}
  {array $\mathcal{S}^{th}_{i}$ containing various integer parameters that are required to describe this data structure}
\RealEnr
  {REAL-PARAM\blank{2}}{$40$}{}
  {array $\mathcal{R}^{th}_{i}$ containing various floating-point parameters that are required to describe this data structure}
\RealEnr
  {MESHZ\blank{7}}{$\mathcal{S}^{th}_{1}$}{m}
  {initial axial meshes as recovered from the fuelmap}
\RealEnr
  {REF-RAD\blank{5}}{$(\mathcal{S}^{th}_{7}-1)\times\mathcal{S}^{th}_{1}$}{m}
  {initial radial meshes as recovered from the first call to {\tt THM:} module}
\RealEnr
  {NB-FUEL\blank{5}}{$\mathcal{S}^{th}_{2}$}{}
  {number of active fuel rods in a single assembly or number of active fuel pins in the cluster}
\RealEnr
  {NB-TUBE\blank{5}}{$\mathcal{S}^{th}_{2}$}{}
  {number of active guide tubes in a single assembly}
\RealEnr
  {FRACT-PU\blank{4}}{$\mathcal{S}^{th}_{2}$}{}
  {plutonium mass enrichment}
\OptRealEnr
  {KCONDF\blank{6}}{$\mathcal{S}^{th}_{16}+3$}{$\mathcal{S}^{th}_{12}\ne 0$}{}
  {coefficients of the user-defined correlation for the fuel thermal conductivity}
\OptCharEnr
  {UCONDF\blank{6}}{$12$}{$\mathcal{S}^{th}_{12}\ne 0$}
  {string variable set to {\tt KELVIN} or to {\tt CELSIUS}}
\OptRealEnr
  {KCONDC\blank{6}}{$\mathcal{S}^{th}_{17}+3$}{$\mathcal{S}^{th}_{13}\ne 0$}{}
  {coefficients of the user-defined correlation for the clad thermal conductivity}
\OptCharEnr
  {UCONDC\blank{6}}{$12$}{$\mathcal{S}^{th}_{13}\ne 0$}
  {string variable set to {\tt KELVIN} or to {\tt CELSIUS}}
\RealEnr
  {ERROR-T-FUEL}{1}{K}
  {absolute error in fuel temperature}
\RealEnr
  {ERROR-D-COOL}{1}{g/cc}
  {absolute error in coolant density}
\RealEnr
  {ERROR-T-COOL}{1}{K}
  {absolute error in coolant temperature}
\RealEnr
  {MIN-T-FUEL\blank{2}}{1}{K}
  {minimum fuel temperature}
\RealEnr
  {MIN-D-COOL\blank{2}}{1}{g/cc}
  {minimum coolant density}
\RealEnr
  {MIN-T-COOL\blank{2}}{1}{K}
  {minimum coolant temperature}
\RealEnr
  {MAX-T-FUEL\blank{2}}{1}{K}
  {maximum fuel temperature}
\RealEnr
  {MAX-D-COOL\blank{2}}{1}{g/cc}
  {maximum coolant density}
\RealEnr
  {MAX-T-COOL\blank{2}}{1}{K}
  {maximum coolant temperature}
 \DirEnr
  {HISTORY-DATA}
  {sub-directory containing the historical values taken by the thermal-hydraulics parameters (mass flux, density, pressure, enthalpy, temperature)
  in the coolant and in the fuel rod for the whole geometry}
\OptRealEnr
  {RAD-PROF\_R\blank{2}}{$\mathcal{S}^{th}_{18}$}{$\mathcal{S}^{th}_{18}\ne 0$}{m}
  {abscissas of the user-defined radial form factor table}
\OptRealEnr
  {RAD-PROF\_F\blank{2}}{$\mathcal{S}^{th}_{18}$}{$\mathcal{S}^{th}_{18}\ne 0$}{ }
  {form-factor values of the user-defined radial form factor table}
\OptRealEnr
  {TIME-SR1\blank{2}}{$\mathcal{S}^{th}_{19}$}{$\mathcal{S}^{th}_{19}\ne 0$}{s}
  {tabulation abscissa in time}
\OptRealEnr
  {POWER-SR1\blank{2}}{$\mathcal{S}^{th}_{19}$}{$\mathcal{S}^{th}_{19}\ne 0$}{ }
  {tabulation power factor corresponding to each tabulation abscissa in time}
\end{DescriptionEnregistrement}

The signature for this data structure is $\mathsf{SIGNA}$=\verb*|L_THM|. The array $\mathcal{S}^{h}_{i}$
contains the following information: 

\begin{itemize}
\item $\mathcal{S}^{th}_{1}$ contains the number of axial meshes $N_{\rm z}$.
\item $\mathcal{S}^{th}_{2}$ contains the number of channels in the radial plane $N_{\rm ch}$.
\item $\mathcal{S}^{th}_{3}$ contains the maximum number of iterations for computing the
conduction integral.
\item $\mathcal{S}^{th}_{4}$ contains the maximum number of iterations for computing the
center pellet temperature.
\item $\mathcal{S}^{th}_{5}$ contains the maximum number of iterations for computing the
coolant parameters (velocity, pressure, enthapy, density) in case of a transient calculation.
\item $\mathcal{S}^{th}_{6}$ contains the number of discretisation points in fuel.
\item $\mathcal{S}^{th}_{7}$ contains the number of total discretisation points in the whole fuel rod (fuel+cladding) $N_{\rm dtot}$.
\item $\mathcal{S}^{th}_{8}$ contains the type of calculation performed by the \moc{THM:} module:

\begin{displaymath} \mathcal{S}^{th}_{8} = \left\{
\begin{array}{rl}
 0 & \textrm{steady-state calculation} \\
 1 & \textrm{transient calculation.} \\
\end{array} \right.
\end{displaymath}

\item $\mathcal{S}^{th}_{9}$ contains the current time index.
\item $\mathcal{S}^{th}_{10}$ flag to set the gap correlation:

\begin{displaymath} \mathcal{S}^{th}_{10} = \left\{
\begin{array}{rl}
 0 & \textrm{built-in correlation is used} \\
 1 & \textrm{set the heat exchange coefficient of the gap as a user-defined constant.} \\
\end{array} \right.
\end{displaymath}

\item $\mathcal{S}^{th}_{11}$ flag to set the heat transfer coefficient between the clad and fluid:

\begin{displaymath} \mathcal{S}^{th}_{11} = \left\{
\begin{array}{rl}
 0 & \textrm{built-in correlation is used} \\
 1 & \textrm{set the heat exchange coefficient between the clad and fluid as a user-defined constant.} \\
\end{array} \right.
\end{displaymath}

\item $\mathcal{S}^{th}_{12}$ flag to set the fuel thermal conductivity:

\begin{displaymath} \mathcal{S}^{th}_{12} = \left\{
\begin{array}{rl}
 0 & \textrm{built-in correlation is used} \\
 1 & \textrm{set the fuel thermal conductivity as a function of a simple user-defined correlation.} \\
\end{array} \right.
\end{displaymath}

\item $\mathcal{S}^{th}_{13}$ flag to set the clad thermal conductivity:

\begin{displaymath} \mathcal{S}^{th}_{13} = \left\{
\begin{array}{rl}
 0 & \textrm{built-in correlation is used} \\
 1 & \textrm{set the clad thermal conductivity as a function of a simple user-defined correlation.} \\
\end{array} \right.
\end{displaymath}

\item $\mathcal{S}^{th}_{14}$ type of approximation used during the fuel conductivity evaluation:

\begin{displaymath} \mathcal{S}^{th}_{14} = \left\{
\begin{array}{rl}
 0 & \textrm{use a rectangle quadrature approximation} \\
 1 & \textrm{use an average approximation.} \\
\end{array} \right.
\end{displaymath}

\item $\mathcal{S}^{th}_{15}$ type of subcooling model:

\begin{displaymath} \mathcal{S}^{th}_{15} = \left\{
\begin{array}{rl}
 0 & \textrm{use the Bowring correlation} \\
 1 & \textrm{use the Saha-Zuber correlation.} \\
\end{array} \right.
\end{displaymath}

\item $\mathcal{S}^{th}_{16}$ contains the number of terms in the user-defined correlation for the fuel thermal conductivity (if $\mathcal{S}^{th}_{12}=1$).
\item $\mathcal{S}^{th}_{17}$ contains the number of terms in the user-defined correlation for the clad thermal conductivity (if $\mathcal{S}^{th}_{13}=1$).
\item $\mathcal{S}^{th}_{18}$ type of radial form factor for the power:

\begin{displaymath} \mathcal{S}^{th}_{18} = \left\{
\begin{array}{rl}
 0 & \textrm{flat radial form factor} \\
 N_{\rm rad} & \textrm{number of point in the radial form factor table.} \\
\end{array} \right.
\end{displaymath}

\item $\mathcal{S}^{th}_{19}$ number of points in the user-defined time-power table.

\item $\mathcal{S}^{th}_{20}$ type of fluid:
\begin{displaymath} \mathcal{S}^{th}_{20} = \left\{
\begin{array}{rl}
 0 & \textrm{light water (H$_2$O)} \\
 1 & \textrm{heavy water (D$_2$O).} \\
\end{array} \right.
\end{displaymath}

\end{itemize}

The array $\mathcal{R}^{th}_{i}$ contains the following information: 

\begin{itemize}
\item $\mathcal{R}^{th}_{1}$ contains the current time step in s. 
\item $\mathcal{R}^{th}_{2}$ contains the fraction of reactor power released in fuel.
\item $\mathcal{R}^{th}_{3}$ contains the critical heat flux in W/m$^2$.
\item $\mathcal{R}^{th}_{4}$ contains the inlet coolant velocity in m/s.
\item $\mathcal{R}^{th}_{5}$ contains the outlet coolant pressure in Pa.
\item $\mathcal{R}^{th}_{6}$ contains the inlet coolant temperature in K.
\item $\mathcal{R}^{th}_{7}$ contains the fuel porosity.
\item $\mathcal{R}^{th}_{8}$ contains the fuel pellet radius
\item $\mathcal{R}^{th}_{9}$ contains the internal clad rod radius in m.
\item $\mathcal{R}^{th}_{10}$ contains the external clad rod radius in m.
\item $\mathcal{R}^{th}_{11}$ contains the guide tube radius in m.
\item $\mathcal{R}^{th}_{12}$ contains the hexagonal side in m. Used only for cluster geometries.
\item $\mathcal{R}^{th}_{13}$ contains the temperature maximum absolute error (in K) allowed in the solution of the conduction equations.
\item $\mathcal{R}^{th}_{14}$ contains the maximum relative error allowed in the matrix resolution of the conservation equations of the coolant.
\item $\mathcal{R}^{th}_{15}$ contains the relaxation parameter for the multiphysics parameters (temperature of fuel and coolant  and density of coolant).
\item $\mathcal{R}^{th}_{16}$ contains the time in s.
\item $\mathcal{R}^{th}_{17}$ contains the heat transfer coefficient of the gap (if $\mathcal{S}^{th}_{10}=1$).
\item $\mathcal{R}^{th}_{18}$ contains the heat transfer coefficient between the clad and fluid (if $\mathcal{S}^{th}_{11}=1$).
\item $\mathcal{R}^{th}_{19}$ contains the surface temperature weighting factor of effective fuel temperature for the Rowlands approximation.
\item $\mathcal{R}^{th}_{20}$ reactor power, as defined after the {\tt POWER-LAW} keyword.
\item $\mathcal{R}^{th}_{21}$ maximum of variable variations in local parameters (used for time step adjustment strategy).
\item $\mathcal{R}^{th}_{22}$ contains the rugosity of the fuel rod in m, used in M\"uller-Steinhagen correlation for coolant friction.
\item $\mathcal{R}^{th}_{23}$ contains the angle in radians of the fuel channel with respect of the vertical axis.
\end{itemize}

\subsubsection{The \moc{HISTORY-DATA} sub-directory}\label{sect:thmdirhistorydata}

In the \moc{HISTORY-DATA} directory, the following sub-directories will be found:
\begin{DescriptionEnregistrement}{Sub-directories in \moc{HISTORY-DATA} directory}{7.0cm} \label{tabl:tabhistorydatadir}
 \DirEnr
 {TIMESTEP0000}
 {sub-directory containing all the values of the thermal-hydraulics parameters computed by the \moc{THM:} module {\sl in steady-state conditions}.}
 \DirEnr
 {TIMESTEP{\sl numt}}
 {sub-directories containing all the values of the thermal-hydraulics parameters computed by the \moc{THM:} module in transient conditions at a given time index {\sl numt}. {\sl numt} can take values between 1 and 9999 in I4.4 format.}
\end{DescriptionEnregistrement}

\noindent In the \moc{TIMESTEP0000} and in each of the \moc{TIMESTEP}{\sl numt} sub-directories, the following records will be found:
\begin{DescriptionEnregistrement}{Records in \moc{TIMESTEP} directories}{7.0cm} \label{tabl:tabtimestepdir}
 \RealEnr
 {TIME\blank{8}}{$1$}{$s$}
 {time}
 \DirlEnr
 {\moc{CHANNEL}\blank{5}}{$N_{\rm ch}$}
 {list of $N_{\rm ch}$ sub-directories containing all the values of the thermal-hydraulics parameters computed by the \moc{THM:} module and sorted channel by channel.}
\end{DescriptionEnregistrement}

\noindent In each of the \moc{CHANNEL} sub-directories, the following records will be found:
\begin{DescriptionEnregistrement}{Records in each \moc{CHANNEL} directory}{7.0cm} \label{tabl:tabchanneldir}
 \RealEnr
 {VINLET\blank{6}}{$1$}{$m.s^{-1}$}
 {inlet velocity}
 \RealEnr
 {TINLET\blank{6}}{$1$}{$K$}
 {inlet temperature}
 \RealEnr
 {PINLET\blank{6}}{$1$}{$Pa$}
 {inlet pressure}
 \RealEnr
 {VELOCITIES\blank{2}}{$N_{\rm z}$}{$m.s^{-1}$}
 {velocity in each of the $N_{\rm z}$ bundles of the channel numbered {\sl numc}}
 \RealEnr
 {PRESSURE\blank{4}}{$N_{\rm z}$}{$Pa$}
 {pressure in each bundle of the channel}
 \RealEnr
 {ENTHALPY\blank{4}}{$N_{\rm z}$}{$J.kg^{-1}$}
 {enthalpy in each bundle of the channel}
 \RealEnr
 {DENSITY\blank{5}}{$N_{\rm z}$}{$kg.m^{-3}$}
 {density in each bundle of the channel}
 \RealEnr
 {LIQUID-DENS\blank{1}}{$N_{\rm z}$}{$kg.m^{-3}$}
 {density of liquid phase in each bundle of the channel}
 \RealEnr
 {TEMPERATURES}{$N_{\rm z}, N_{\rm dtot}$}{$K$}
 {distribution of the temperature in the fuel-pin for each bundle of the channel}
 \RealEnr
 {CENTER-TEMPS}{$N_{\rm z}$}{$K$}
 {center fuel pellet temperature in each bundle of the channel}
 \RealEnr
 {RADII}{$N_{\rm z}, N_{\rm dtot}-1$}{$m$}
 {fuel and clad radii}
\end{DescriptionEnregistrement}

\clearpage
