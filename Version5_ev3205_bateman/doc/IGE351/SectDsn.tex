\subsection{The \moc{snt} dependent records on a \dir{tracking} directory}\label{sect:sntrackingdir}

When the \moc{SNT:} operator is used ($\mathsf{CDOOR}$={\tt 'SN'}), the following elements in the vector
$\mathcal{S}^{t}_{i}$ will also be defined.

\begin{itemize}
\item $\mathcal{S}^{t}_{6}$: ({\tt ITYPE}) Type of SN geometry:
\begin{displaymath}
\mathcal{S}^{t}_{6} = \left\{
\begin{array}{rl}
 2 & \textrm{Cartesian 1-D geometry} \\
 3 & \textrm{Tube 1-D geometry} \\
 4 & \textrm{Spherical 1-D geometry} \\
 5 & \textrm{Cartesian 2-D geometry} \\
 6 & \textrm{Tube 2-D geometry (R-Z geometry)} \\
 7 & \textrm{Cartesian 3-D geometry} \\
 8 & \textrm{Hexagonal 2-D geometry} \\
 9 & \textrm{Hexagonal 3-D geometry}
\end{array} \right.
\end{displaymath}

\item $\mathcal{S}^{t}_{7}$: ({\tt NSCT}) Number of spherical harmonics components used to expand the flux and the sources.

%Update to tabular format?
\item $\mathcal{S}^{t}_{8}$: ({\tt IELEM}) Measure of order of the spatial approximation. The Legendre polynomials (for both HODD \emph{and} DG (see $\mathcal{S}^{t}_{10}$)) used are of order 0 (constant), 1 (linear), 2 (parabolic) or $>$3 (higher-orders), corresponding to {\tt IELEM} values of:
\begin{displaymath}
\mathcal{S}^{t}_{8} = \left\{
\begin{array}{rl}
 1 & \textrm{Constant- default for HODD} \\
 2 & \textrm{Linear - default for DG} \\
 3 & \textrm{Parabolic} \\
 >4 & \textrm{Higer-orders}
\end{array} \right.
\end{displaymath}

\item $\mathcal{S}^{t}_{9}$: ({\tt NDIM}) Number of geometric dimensions (1, 2 or 3).

\item $\mathcal{S}^{t}_{10}$: ({\tt ISCHM}) Method of spatial discretisation:
\begin{displaymath}
\mathcal{S}^{t}_{10} = \left\{
\begin{array}{rl}
 1 & \textrm{High-Order Diamond Differencing method (HODD) -- default option if unspecified} \\
 2 & \textrm{Discontinuous Galerkin finite element method (DG) -- available if $\mathcal{S}^{t}_{6} = 2, 5,$ or $7$}\\
 3 & \textrm{Adaptive Weighted difference (AWD) -- available if $\mathcal{S}^{t}_{6} = 2, 5,$ or $7$}
\end{array} \right.
\end{displaymath}

\item $\mathcal{S}^{t}_{11}$: ({\tt LL4}) Number of mesh-centered flux components in one energy group.
Generally equal to
$\mathcal{S}^{t}_{2}$ except in cases where surfacic fluxes are appended to the
unknown vector. $\mathcal{S}^{t}_{11}\le\mathcal{S}^{t}_{2}$.

\item $\mathcal{S}^{t}_{12}$: ({\tt LX}) Number of elements along the $X$ axis.

\item $\mathcal{S}^{t}_{13}$: ({\tt LY}) Number of elements along the $Y$ axis.

\item $\mathcal{S}^{t}_{14}$: ({\tt LZ}) Number of elements along the $Z$ axis.

\item $\mathcal{S}^{t}_{15}$: ({\tt NLF}) Order of the $S_N$ approximation (even number $\ge 2$).

\item $\mathcal{S}^{t}_{16}$: ({\tt ISCAT}) Number of terms in the scattering sources:
\begin{displaymath}
\mathcal{S}^{t}_{16} = \left\{
\begin{array}{rl}
 1 & \textrm{Isotropic scattering in the laboratory system} \\
 2 & \textrm{Linearly anisotropic scattering in the laboratory system} \\
 n & \textrm{order $n-1$ anisotropic scattering in the laboratory system}
\end{array} \right.
\end{displaymath}

\item $\mathcal{S}^{t}_{17}$: ({\tt IQUAD}) Type of angular quadrature:
\begin{displaymath}
\mathcal{S}^{t}_{17} = \left\{
\begin{array}{rl}
 1 & \textrm{Level symmetric, Lathrop and Carlson type} \\
 2 & \textrm{Level symmetric, optimized $\mu_1$ values} \\
 3 & \textrm{Level symmetric, compatible with code SNOW} \\
 4 & \textrm{Legendre-Chebyshev quadrature} \\
 5 & \textrm{symmetric Legendre-Chebyshev quadrature} \\
 6 & \textrm{quadrupole range (QR) quadrature} \\
 10 & \textrm{Gauss-Legendre and Gauss-Chebyshev product quadrature}
\end{array} \right.
\end{displaymath}

\item $\mathcal{S}^{t}_{18}$: ({\tt IFIX}) Flag for negative flux fixup:
\begin{displaymath}
\mathcal{S}^{t}_{18} = \left\{
\begin{array}{rl}
 0 & \textrm{Non enabled} \\
 1 & \textrm{Enabled}
\end{array} \right.
\end{displaymath}

\item $\mathcal{S}^{t}_{19}$: ({\tt IDSA}) Flag for synthetic diffusion acceleration:
\begin{displaymath}
\mathcal{S}^{t}_{19} = \left\{
\begin{array}{rl}
 0 & \textrm{Non enabled} \\
 1 & \textrm{Enabled}
\end{array} \right.
\end{displaymath}

\item $\mathcal{S}^{t}_{20}$: ({\tt NSTART}) Type of acceleration for the scattering iterations:
\begin{displaymath}
\mathcal{S}^{t}_{20} = \left\{
\begin{array}{rl}
 0 & \textrm{GMRES non enabled; use a one-parameter Livolant acceleration} \\
 >0 & \textrm{Restarts the GMRES method every {\tt NSTART} iterations}
\end{array} \right.
\end{displaymath}

\item $\mathcal{S}^{t}_{21}$: ({\tt NSDSA}) The synthetic acceleration is applied on every other $\mathcal{S}^{t}_{21}$ number of inner flux iterations.

\item $\mathcal{S}^{t}_{22}$: ({\tt MAXI}) Maximum number of inner iterations (resp. maximum number of GMRES(m) iterations if $\mathcal{S}^{t}_{20}>0$).

\item $\mathcal{S}^{t}_{23}$: ({\tt ILIVOL}) Flag for enabling/disabling Livolant acceleration method.
\begin{displaymath}
\mathcal{S}^{t}_{23} = \left\{
\begin{array}{rl}
 0 & \textrm{Non enabled} \\
 1 & \textrm{Enabled}
\end{array} \right.
\end{displaymath}

\item $\mathcal{S}^{t}_{24}$: ({\tt icl1}) number of free iterations in the Livolant method.
\item $\mathcal{S}^{t}_{25}$: ({\tt icl2}) number of accelerated iterations in the Livolant method.
\item $\mathcal{S}^{t}_{26}$: ({\tt ISPLH}) Type of hexagonal mesh splitting if $\mathcal{S}^{t}_{6}\ge 8$:
\begin{displaymath}
\mathcal{S}^{t}_{26} = \left\{
\begin{array}{rl}
 1 & \textrm{$3$ lozenges per hexagon} \\
 K & \textrm{$3\times K \times K$ lozenges per hexagon}
\end{array} \right.
\end{displaymath}

\item $\mathcal{S}^{t}_{27}$: (INSB) Flux vectorization option where
\begin{displaymath}
\mathcal{S}^{t}_{27} = \left\{
\begin{array}{rl}
 0 & \textrm{Scalar algorithm. The multigroup flux is computed as a sequence of one-group}\\
 & \textrm{solutions using Gauss-Seidel iterations.} \\
 1 & \textrm{Vectorial algorithm. The multigroup flux is computed in parallel for a set of energy}\\
 & \textrm{groups.}
\end{array} \right.
\end{displaymath}

\item $\mathcal{S}^{t}_{28}$: (NOMP) Type of OpenMP multithreading strategy in 2D and 3D geometries where
\begin{displaymath}
\mathcal{S}^{t}_{28} = \left\{
\begin{array}{rl}
 0 & \textrm{Standard energy group and discrete angle nested loops}\\
 M & \textrm{Domino type nested loops with $M\times M$ or $M \times M \times M$ macrocells}
\end{array} \right.
\end{displaymath}

\item $\mathcal{S}^{t}_{29}$: (IGAV) Type of condition at axial axis for cylindrical and spherical 1D geometries where
\begin{displaymath}
\mathcal{S}^{t}_{29} = \left\{
\begin{array}{rl}
 1 & \textrm{Specular reflection}\\
 2 & \textrm{Zero-weight reflection}\\
 3 & \textrm{Averaged reflection}\\
\end{array} \right.
\end{displaymath}

\item $\mathcal{S}^{t}_{30}$: ({\tt LSHOOT}) Flag for enabling/disabling the shooting method in 1D.
\begin{displaymath}
\mathcal{S}^{t}_{30} = \left\{
\begin{array}{rl}
 0 & \textrm{Non enabled} \\
 1 & \textrm{Enabled}
\end{array} \right.
\end{displaymath}

\item $\mathcal{S}^{t}_{31}$: ({\tt IBFP}) Type of equation solved by the discrete ordinates method.
\begin{displaymath}
\mathcal{S}^{t}_{31} = \left\{
\begin{array}{rl}
 0 & \textrm{Boltzmann transport equation} \\
 1 & \textrm{Boltzmann Fokker-Planck equation with Galarkin energy propagation factors} \\
 2 & \textrm{Boltzmann Fokker-Planck equation with Przybylski and Ligou energy propagation} \\
  & \textrm{factors}
\end{array} \right.
\end{displaymath}

\item $\mathcal{S}^{t}_{32}$: (NMPI) Type of MPI parallelisation strategy in 2D and 3D geometries using WYVERN where
\begin{displaymath}
\mathcal{S}^{t}_{28} = \left\{
\begin{array}{rl}
 1 & \textrm{parallelisation over number of angular directions per octant/dodecant.}\\
 M & \textrm{parallelisation over both angles and macrocells with $M\times M$ or $M \times M \times M$ macrocells in 2D and 3D Cartesian geometry, or $M$ macrocells along the $z$-axis in hexagonal 3D geometry.}
\end{array} \right.
\end{displaymath}

\item $\mathcal{S}^{t}_{33}$: ({\tt ISOLVSA}) Type of solver to be used for the synthetic acceleration. Note that TRIVAC generally works better and is faster with hexagonal geometries for the matrix assemblies. Also, for 3D geometries, TRIVAC \emph{has} to be chosen.
\begin{displaymath}
\mathcal{S}^{t}_{33} = \left\{
\begin{array}{rl}
 1 & \textrm{BIVAC}\\
 2 & \textrm{TRIVAC}
\end{array} \right.
\end{displaymath}

\item $\mathcal{S}^{t}_{34}$: ({\tt NFOU}) Number of frequencies to be investigated in 1D Fourier analysis along the range $[0, \frac{2\pi}{L})$ where $L$ is the length of the slab.

\item $\mathcal{S}^{t}_{35}$: ({\tt EELEM}) Measure of order of the energy approximation for the continuous slowing-down term of the Boltzmann Fokker-Planck equation. The Legendre polynomials (for both HODD \emph{and} DG (see $\mathcal{S}^{t}_{36}$)) used are of order 0 (constant), 1 (linear), 2 (parabolic) or $>$3 (higher-orders), corresponding to {\tt EELEM} values of:
\begin{displaymath}
\mathcal{S}^{t}_{35} = \left\{
\begin{array}{rl}
 1 & \textrm{Constant- default for HODD} \\
 2 & \textrm{Linear - default for DG} \\
 3 & \textrm{Parabolic} \\
 >4 & \textrm{Higer-orders}
\end{array} \right.
\end{displaymath}

\item $\mathcal{S}^{t}_{36}$: ({\tt ESCHM}) Method of energy discretisation for the continuous slowing-down term of the Boltzmann Fokker-Planck equation:
\begin{displaymath}
\mathcal{S}^{t}_{36} = \left\{
\begin{array}{rl}
 1 & \textrm{High-Order Diamond Differencing method (HODD) -- default option if unspecified} \\
 2 & \textrm{Discontinuous Galerkin finite element method (DG) -- available if $\mathcal{S}^{t}_{6} = 2, 5,$ or $7$}\\
 3 & \textrm{Adaptive Weighted difference (AWD) -- available if $\mathcal{S}^{t}_{6} = 2, 5,$ or $7$}
\end{array} \right.
\end{displaymath}

\end{itemize}

The following records will also be present on the main level of a \dir{tracking}
directory.
\begin{DescriptionEnregistrement}{The \moc{snt} records in
\dir{tracking}}{8.0cm}
\IntEnr
  {NCODE\blank{7}}{$6$}
  {Record containing the types of boundary conditions on each surface. =0 side
   not used; =1 VOID; =2 REFL; =4 TRAN. {\tt NOODE(5)} and {\tt NOODE(6)} are not used.} 
\RealEnr
  {ZCODE\blank{7}}{$6$}{$1$}
  {Record containing the albedo value (real number) on each surface. {\tt ZOODE(5)}
   and {\tt ZOODE(6)} are not used.} 
\IntEnr
  {KEYFLX\$ANIS\blank{1}}{$\mathcal{S}^t_1,\mathcal{S}^{t}_{8}**\mathcal{S}^{t}_{9},\mathcal{S}^{t}_{7}$}
  {Location in unknown vector of averaged regional flux moments.} 
\OptDirEnr
  {DSA\blank{9}}{$\mathcal{S}^{t}_{19}= 1$}
  {Sub-directory containing the data related to the diffusion synthetic acceleration using BIVAC (2D) or TRIVAC (3D).
  The specification of this directory is given in \Sect{bivactrackingdir} or in \Sect{trivatrackingdir} } 
\RealEnr
  {EPSI\blank{8}}{$1$}{$1$}
  {Record containing the convergence criterion on inner iterations.}

\end{DescriptionEnregistrement}

If $\mathcal{S}^{t}_{6}=2$ (Cartesian 1-D geometry), the following records will also be present on the main level of a \dir{tracking}
directory.

\begin{DescriptionEnregistrement}{The \moc{snt} records in
\dir{tracking} (Cartesian 1-D geometry)}{8.0cm}
\RealEnr
  {U\blank{11}}{$\mathcal{S}^{t}_{15}$}{$1$}
  {Base points of the angular Gauss-Legendre quadrature.} 
\RealEnr
  {W\blank{11}}{$\mathcal{S}^{t}_{15}$}{$1$}
  {Weights of the angular Gauss-Legendre quadrature.} 
\RealEnr
  {PL\blank{10}}{$\mathcal{S}^{t}_{16},\mathcal{S}^{t}_{15}$}{$1$}
  {Discrete values of the Legendre polynomials on the quadrature base points.} 
\RealEnr
  {WX\blank{10}}{$\mathcal{S}^{t}_{8}+1$}{$1$}
  {Weights of the incoming and moments fluxes in the spatial closure relations for the streaming term.}
\OptRealEnr
  {WE\blank{10}}{$\mathcal{S}^{t}_{35}+1$}{$\mathcal{S}^{t}_{31}\neq0$}{$1$}
  {Weights of the incoming and moments fluxes in the energy closure relations for the continuous slowing-down term of the Boltzmann Fokker-Planck equation.}
\RealEnr
  {CST\blank{9}}{max\{$\mathcal{S}^{t}_{8},\mathcal{S}^{t}_{35}$\}}{$1$}
  {Normalized Legendre polynomials (defined over -1/2 to 1/2) value at boundaries.}
\end{DescriptionEnregistrement}
\clearpage

If $\mathcal{S}^{t}_{6}=3$ (Tube 1-D geometry), the following records will also be present on the main level of a \dir{tracking}
directory. The number of discrete directions in two octants $N_{\rm angl}$ and the number of spherical harmonics components of the flux $N_{\rm pn}$
are given in term of the $S_N$ order $N=\mathcal{S}^{t}_{15}$ as
$$
N_{\rm angl}=\cases{{\displaystyle 1\over \displaystyle 2}N\left(1+{\displaystyle N\over \displaystyle 2}\right),
&if $\mathcal{S}^{t}_{17}<10$;\cr {\displaystyle 1\over \displaystyle 2}N^2, &otherwise. \cr}
$$
$$
N_{\rm pn}={\mathcal{S}^{t}_{16}\over 2}\left( 1+{\mathcal{S}^{t}_{16}\over 2}\right)+
{1\over 2} (1+\mathcal{S}^{t}_{16})\, (\mathcal{S}^{t}_{16}{\rm mod} \, 2)
$$

\begin{DescriptionEnregistrement}{The \moc{snt} records in
\dir{tracking} (tube 1-D geometry)}{8.0cm}
\IntEnr
  {JOP\blank{9}}{$N/2$}
  {Number of base points in each $\xi$ level.} 
\RealEnr
  {U\blank{11}}{$N/2$}{$1$}
  {Base points (levels) of the angular quadrature in $\xi$ (positive values).} 
\RealEnr
  {UPQ\blank{9}}{$N_{\rm angl}$}{$1$}
  {Direction cosines of the angular two-octants spherical harmonics quadrature in $\mu$.} 
\RealEnr
  {WPQ\blank{9}}{$N_{\rm angl}$}{$1$}
  {Weights of the angular two-octants spherical harmonics quadrature.} 
\RealEnr
  {ALPHA\blank{7}}{$N_{\rm angl}$}{$1$}
  {Angular redistribution parameters.} 
\RealEnr
  {PLZ\blank{9}}{$N_{\rm pn},N/2$}{$1$}
  {Discrete values of the real spherical harmonics on the zero-weight base points.} 
\RealEnr
  {PL\blank{10}}{$N_{\rm pn},N_{\rm angl}$}{$1$}
  {Discrete values of the real spherical harmonics on the quadrature base points.} 
\RealEnr
  {SURF\blank{8}}{$\mathcal{S}^{t}_{12}+1$}{$1$}
  {Surfaces.} 
\end{DescriptionEnregistrement}

If $\mathcal{S}^{t}_{6}=4$ (Spherical 1-D geometry), the following records will also be present on the main level of a \dir{tracking}
directory.

\begin{DescriptionEnregistrement}{The \moc{snt} records in
\dir{tracking} (spherical 1-D geometry)}{8.0cm}
\RealEnr
  {U\blank{11}}{$\mathcal{S}^{t}_{15}$}{$1$}
  {Base points of the angular Gauss-Legendre quadrature.} 
\RealEnr
  {W\blank{11}}{$\mathcal{S}^{t}_{15}$}{$1$}
  {Weights of the angular Gauss-Legendre quadrature.} 
\RealEnr
  {ALPHA\blank{7}}{$\mathcal{S}^{t}_{15}$}{$1$}
  {Angular redistribution parameters.} 
\RealEnr
  {PLZ\blank{9}}{$\mathcal{S}^{t}_{16}$}{$1$}
  {Discrete values of the Legendre polynomials on the zero-weight base points at $\mu=-1$.} 
\RealEnr
  {PL\blank{10}}{$\mathcal{S}^{t}_{16},\mathcal{S}^{t}_{15}$}{$1$}
  {Discrete values of the Legendre polynomials on the quadrature base points.} 
\RealEnr
  {SURF\blank{8}}{$\mathcal{S}^{t}_{12}+1$}{$1$}
  {Surfaces.} 
\RealEnr
  {XXX\blank{9}}{$\mathcal{S}^{t}_{12}+1$}{$1$}
  {Mesh-edge radii.} 
\end{DescriptionEnregistrement}

If $\mathcal{S}^{t}_{6}=5$ (Cartesian 2-D geometry) or $\mathcal{S}^{t}_{6}=6$ (R-Z geometry), the following records will also be present on the main level of a \dir{tracking}
directory. The number of discrete directions in four octants (including zero-weight points) $N_{\rm angl}$ and the number of spherical harmonics components of the flux $N_{\rm pn}$
are given in term of the $S_N$ order $N=\mathcal{S}^{t}_{15}$ as
$$
N_{\rm angl}={1\over 2}(N+4)N
$$
$$
N_{\rm pn}={\mathcal{S}^{t}_{16}\over 2}\left( 1+\mathcal{S}^{t}_{16}\right)
$$

\begin{DescriptionEnregistrement}{The \moc{snt} records in
\dir{tracking} (Cartesian 2-D and R-Z geometries)}{8.0cm}
\RealEnr
  {DU\blank{10}}{$N_{\rm angl}$}{$1$}
  {Direction cosines of the angular four-octants spherical harmonics quadrature in $\mu$.} 
\RealEnr
  {DE\blank{10}}{$N_{\rm angl}$}{$1$}
  {Direction cosines of the angular four-octants spherical harmonics quadrature in $\eta$.} 
\RealEnr
  {W\blank{11}}{$N_{\rm angl}$}{$1$}
  {Weights of the angular four-octants spherical harmonics quadrature.} 
\IntEnr
  {MRM\blank{9}}{$N_{\rm angl}$}
  {Quadrature offsets.} 
\IntEnr
  {MRMY\blank{8}}{$N_{\rm angl}$}
  {Quadrature offsets.} 
\RealEnr
  {DB\blank{10}}{$\mathcal{S}^{t}_{12},N_{\rm angl}$}{$1$}
  {Diamond-scheme parameter.} 
\RealEnr
  {DA\blank{10}}{$\mathcal{S}^{t}_{12},\mathcal{S}^{t}_{13},N_{\rm angl}$}{$1$}
  {Diamond-scheme parameter.} 
\OptRealEnr
  {DAL\blank{9}}{$\mathcal{S}^{t}_{12},\mathcal{S}^{t}_{13},N_{\rm angl}$}{$\mathcal{S}^{t}_{6}=6$}{$1$}
  {Angular redistribution parameters.} 
\RealEnr
  {PL\blank{10}}{$N_{\rm pn},N_{\rm angl}$}{$1$}
  {Discrete values of the real spherical harmonics on the quadrature base points.} 
\RealEnr
  {WX\blank{10}}{$\mathcal{S}^{t}_{8}+1$}{$1$}
  {Weights of the incoming and moments fluxes in the spatial closure relations for the streaming term.}
\OptRealEnr
  {WE\blank{10}}{$\mathcal{S}^{t}_{35}+1$}{$\mathcal{S}^{t}_{31}\neq0$}{$1$}
  {Weights of the incoming and moments fluxes in the energy closure relations for the continuous slowing-down term of the Boltzmann Fokker-Planck equation.}
\RealEnr
  {CST\blank{9}}{max\{$\mathcal{S}^{t}_{8},\mathcal{S}^{t}_{35}$\}}{$1$}
  {Normalized Legendre polynomials (defined over -1/2 to 1/2) value at boundaries.}
\end{DescriptionEnregistrement}

If $\mathcal{S}^{t}_{6}=7$ (Cartesian 3-D geometry), the following records will also be present on the main level of a \dir{tracking}
directory. The number of discrete directions in height octants  $N_{\rm angl}$ and the number of spherical harmonics components of the flux $N_{\rm pn}$
are given in term of the $S_N$ order $N=\mathcal{S}^{t}_{15}$ as
$$
N_{\rm angl}=(N+2)N
$$
$$
N_{\rm pn}=\left( 1+\mathcal{S}^{t}_{16}\right)^{2}
$$

\begin{DescriptionEnregistrement}{The \moc{snt} records in
\dir{tracking} (Cartesian 3-D geometry)}{8.0cm}
\RealEnr
  {DU\blank{10}}{$N_{\rm angl}$}{$1$}
  {Direction cosines of the angular height-octants spherical harmonics quadrature in $\mu$.} 
\RealEnr
  {DE\blank{10}}{$N_{\rm angl}$}{$1$}
  {Direction cosines of the angular height-octants spherical harmonics quadrature in $\eta$.} 
\RealEnr
  {DZ\blank{10}}{$N_{\rm angl}$}{$1$}
  {Direction cosines of the angular height-octants spherical harmonics quadrature in $\xi$.} 
\RealEnr
  {W\blank{11}}{$N_{\rm angl}$}{$1$}
  {Weights of the angular height-octants spherical harmonics quadrature.} 
\IntEnr
  {MRMX\blank{9}}{$N_{\rm angl}$}
  {Quadrature offsets.} 
\IntEnr
  {MRMY\blank{8}}{$N_{\rm angl}$}
  {Quadrature offsets.} 
\IntEnr
  {MRMZ\blank{8}}{$N_{\rm angl}$}
  {Quadrature offsets.} 
\RealEnr
  {DC\blank{10}}{$\mathcal{S}^{t}_{12},\mathcal{S}^{t}_{13},N_{\rm angl}$}{$1$}
  {Diamond-scheme parameter.} 
\RealEnr
  {DB\blank{10}}{$\mathcal{S}^{t}_{12},\mathcal{S}^{t}_{14},N_{\rm angl}$}{$1$}
  {Diamond-scheme parameter.} 
\RealEnr
  {DA\blank{10}}{$\mathcal{S}^{t}_{13},\mathcal{S}^{t}_{14},N_{\rm angl}$}{$1$}
  {Diamond-scheme parameter.} 
\RealEnr
  {PL\blank{10}}{$N_{\rm pn},N_{\rm angl}$}{$1$}
  {Discrete values of the real spherical harmonics on the quadrature base points.} 
\RealEnr
  {WX\blank{10}}{$\mathcal{S}^{t}_{8}+1$}{$1$}
  {Weights of the incoming and moments fluxes in the spatial closure relations for the streaming term.}
\OptRealEnr
  {WE\blank{10}}{$\mathcal{S}^{t}_{35}+1$}{$\mathcal{S}^{t}_{31}\neq0$}{$1$}
  {Weights of the incoming and moments fluxes in the energy closure relations for the continuous slowing-down term of the Boltzmann Fokker-Planck equation.}
\RealEnr
  {CST\blank{9}}{max\{$\mathcal{S}^{t}_{8},\mathcal{S}^{t}_{35}$\}}{$1$}
  {Normalized Legendre polynomials (defined over -1/2 to 1/2) value at boundaries.}
\end{DescriptionEnregistrement}

\eject
