\subsection{The \moc{excelt} dependent records on a
\dir{tracking} directory}\label{sect:excelltrackingdir}

When the \moc{EXCELT:} modules is used ($\mathsf{CDOOR}$={\tt 'EXCELL'}), the following elements in the vector
$\mathcal{S}^{t}_{i}$ will also be defined.

\begin{itemize}
\item $\mathcal{S}^{t}_{6}$ is the number of Legendre orders used for the flux expansions, where
\begin{displaymath}
\mathcal{S}^{t}_{6} = \left\{
\begin{array}{rl}
 1 & \textrm{isotropic} \\
 2 & \textrm{linearly anisotropic.} \\
\end{array} \right.
\end{displaymath}

\item $\mathcal{S}^{t}_{7}$ is the specific EXCELL tracking procedure considered where
\begin{displaymath}
\mathcal{S}^{t}_{7} = \left\{
\begin{array}{rl}
 1 & \textrm{Cartesian 2D/3D assembly using {\tt EXCELT:}} \\
 2 & \textrm{hexagonal 2D/3D assembly using {\tt EXCELT:}} \\
 3 & \textrm{2D cluster geometry using {\tt EXCELT:}} \\
 4 & \textrm{2D and 3D Cartesian assemblies with clusters using {\tt NXT:}.} \\
\end{array} \right.
\end{displaymath}

\item $\mathcal{S}^{t}_{8}$ is the track normalization flag where
\begin{displaymath}
\mathcal{S}^{t}_{8} = \left\{
\begin{array}{rl}
 -1 & \textrm{direction dependent track normalization to exact volumes;} \\
 0 & \textrm{global track normalization to exact volumes;} \\
 1 & \textrm{no normalization.} \\
\end{array} \right.
\end{displaymath}

\item $\mathcal{S}^{t}_{9}$ is the tracking type where
\begin{displaymath}
\mathcal{S}^{t}_{9} = \left\{
\begin{array}{rl}
 0 & \textrm{means that a standard tracking procedure was considered (\moc{TISO});} \\
 1 & \textrm{means that a cyclic tracking procedure was considered  (\moc{TSPC}).} \\
\end{array} \right.
\end{displaymath}

\item $\mathcal{S}^{t}_{10}$ is the type of boundary conditions where
\begin{displaymath}
\mathcal{S}^{t}_{10} = \left\{
\begin{array}{rl}
 0 & \textrm{means that isotropic (white) boundary conditions will be considered (\moc{PISO});} \\
 1 & \textrm{means that mirror-like (specular) boundary conditions will be considered (\moc{PSPC}).} \\
\end{array} \right.
\end{displaymath}

Note that mirror-like boundary conditions ($\mathcal{S}^{t}_{10} = 1$) can be used only if
a cyclic tracking procedure was considered ($\mathcal{S}^{t}_{9} = 1$). 

\item $\mathcal{S}^{t}_{11}=N_\Omega$ is the order of the azimuthal (2-D) or solid (3-D) angular quadrature. For 2-D geometry, the order 
of the azimuthal quadrature represents:
\begin{itemize}
\item the number of equal sectors (trapezoidal quadrature) in the $[0,\pi]$ range when the {\tt EXCELT:} module is
used for Cartesian assemblies; 
\item the number of equal sectors (trapezoidal quadrature) in the $[0,2\pi/3]$ range when the {\tt EXCELT:} module 
is used for hexagonal geometries;
\item the number of equal sectors (trapezoidal quadrature) in the $[0,\max (\mathcal{S}^{t}_{12},2) \, \pi]$ range when the {\tt EXCELT:}
module is used for cluster geometries; 
\item the number of trapezoidal sectors in the $[0,\pi/2]$ range when the {\tt NXT:} module is used. 
\end{itemize}

For 3-D geometry, the order of the solid angle quadrature is: 
\begin{itemize}
\item the order $n$ of the $EQ_n$ quadrature in a quadrant ($0\le \varphi\le \pi/2$ and $0\le \theta \le \pi/2$) when the
{\tt EXCELT:} module is used for Cartesian assemblies for $N_{\rm dir}=n(n+2)/8$ direction in each 
quadrant; 
\item the number of equal sectors (trapezoidal quadrature) in the $[0,2\pi/3]$ range when the {\tt EXCELT:} module 
is used for hexagonal geometries; 
\item not used for the {\tt EXCELT:} module in cluster geometries; 
\item the order $n$ of the $EQ_n$ , $LC_n$ or $LT_n$ quadrature in a quadrant with $0\le \varphi\le \pi/2$ and $0\le \theta \le \pi/2$.
When the {\tt NXT:} modules is used, the number $N_{\rm dir}$ of directions for the azimuthal (2-D) or solid (3-D) angle quadrature
results in $N_{\rm dir}=n(n+2)/8$ directions for the $EQ_n$ quadrature, 
$N_{\rm dir}=3n(n+2)/8$ for the $LC_n$ quadrature and $N_{\rm dir}=3n^2/2$ for the $LT_n$ quadrature.
\end{itemize}

\item $\mathcal{S}^{t}_{12}$ is the angular symmetry factor;

\item $\mathcal{S}^{t}_{13}$ is the polar quadrature type:
\begin{displaymath}
\mathcal{S}^{t}_{13} = \left\{
\begin{array}{rl}
 0 & \textrm{Gauss-Legendre} \\
 1 & \textrm{CACTUS type 1} \\
 2 & \textrm{CACTUS type 2} \\
 3 & \textrm{McDaniel} \\
 4 & \textrm{McDaniel with $P_1$ constraint} \\
 5 & \textrm{Gauss optimized.}
\end{array} \right.
\end{displaymath}

\item $\mathcal{S}^{t}_{14}$ is the order of the polar quadrature.

\item $\mathcal{S}^{t}_{15}$ is the azimuthal (2-D) or solid (3-D) angle quadrature type where
\begin{displaymath}
\mathcal{S}^{t}_{15} = \left\{
\begin{array}{rl}
 1 & \textrm{for a $EQ_n$ (3-D) or trapezoidal (2-D) quadrature;} \\
 2 & \textrm{for a Gauss quadrature (2-D hexagonal geometries);} \\
 3 & \textrm{for a median angle quadrature;} \\
 4 & \textrm{for a $LC_n$ 3-D quadrature;} \\
 5 & \textrm{for a $LT_n$ 3-D quadrature;} \\
 6 & \textrm{for a $\mu_1$--optimized level-symmetric 3-D quadrature;} \\
 7 & \textrm{for a quadrupole range (QR) 3-D quadrature.} \\
\end{array} \right.
\end{displaymath}

\item $\mathcal{S}^{t}_{16}$ is the number of geometric dimensions (1, 2 or 3).

\item $\mathcal{S}^{t}_{17}$ is the number of tracking points on a line.

\item $\mathcal{S}^{t}_{18}$ is the maximum length of a track.

\item $\mathcal{S}^{t}_{19}$ is the total number of tracks generated.

\item $\mathcal{S}^{t}_{20}$ is the total number of track directions processed.

\item $\mathcal{S}^{t}_{21}$ is the line format option for \moc{TLM:} module, where
\begin{displaymath}
\mathcal{S}^{t}_{21} = \left\{
\begin{array}{rl}
 0 & \textrm{short format} \\
 1 & \textrm{complete format.} \\
\end{array} \right.
\end{displaymath}

\item $\mathcal{S}^{t}_{22}$ is the vectorization option for computing collision probability matrices where
\begin{displaymath}
\mathcal{S}^{t}_{22} = \left\{
\begin{array}{rl}
 0 & \textrm{scalar algorithm. The tracking is readed for each energy group.} \\
 1 & \textrm{vectorial algorithm. The tracking is readed once and the collision
 probability} \\
   & \textrm{matrices are computed in many energy groups.} \\
 2 & \textrm{vectorial algorithm of type \moc{EXCELL:}. The tracking is computed in \moc{DOORVP}} \\
   & \textrm{and the collision probability matrices are computed in many energy groups.} \\
 3 & \textrm{vectorial algorithm of type \moc{NXT:}. The tracking is computed in \moc{DOORVP}} \\
   & \textrm{and the collision probability matrices are computed in many energy groups.} \\
\end{array} \right.
\end{displaymath}

\item $\mathcal{S}^{t}_{23}$ is the tracking flag, where
\begin{displaymath}
\mathcal{S}^{t}_{23} = \left\{
\begin{array}{rl}
 -1 & \textrm{the LCM object information is used for Monte-Carlo calculations. No} \\
   & \textrm{tracking file produced.} \\
 0 & \textrm{this option de-activates the tracking file production if \moc{TISO} or \moc{TSPC} are} \\
   & \textrm{specified.} \\
 1 & \textrm{this option produces the tracking file.} \\
\end{array} \right.
\end{displaymath}

\item $\mathcal{S}^{t}_{39}$ is the prismatic tracking activation flag along $z$ axis for 3D geometry (0/3: off/on-$z$ axis).

\end{itemize}

The following records will also be present on the main level of a \dir{tracking}
directory.

\begin{DescriptionEnregistrement}{The \moc{EXCELT:} records in \dir{tracking}}{8.0cm}
\RealEnr
  {EXCELTRACKOP}{$40$}{}
  {array ${\cal R}_i$ containing additional EXCELL or NXT tracking parameters.} 
\IntEnr
  {ICODE\blank{7}}{$6$}
  {array ${\cal I}_{\beta,k}$ containing the surface albedo index (geometric surface albedo
  $\beta_{{\rm g},k}$ are used if ${\cal I}_{\beta,k}< 0$ while physical 
  surface albedo $\beta_{{\rm p},k}$ are used if ${\cal I}_{\beta,k}> 0$).} 
\RealEnr
  {ALBEDO\blank{6}}{$6$}{}
  {array $\beta_{{\rm g},k}$ containing the geometric surface albedo (used only if ${\cal I}_{\beta,k}\ge 0$).}
\OptDirEnr
  {EXCELL\blank{6}}{$\mathcal{S}^{t}_{7}< 4$}
  {directory containing additional {\tt EXCELT:}  records for the cases where $\mathcal{S}^{t}_{7}=1$ or $\mathcal{S}^{t}_{7}=3$.}
\OptDirEnr
  {NXTRecords\blank{2}}{$\mathcal{S}^{t}_{7}=4$}
  {directory containing additional {\tt NXT:} records.}
\OptDirEnr
  {PROJECTION\blank{2}}{$\mathcal{S}^{t}_{39}>0$}
  {directory containing the analysis of the projection of a 3D prismatic geometry.}
\end{DescriptionEnregistrement}

The record ${\cal R}_i$ contains the following information:
\begin{itemize}
\item ${\cal R}_1$ is the maximum error allowed on the exponential function. 
\item ${\cal R}_2$ is the user requested tracking density in cm$^{-1}$ and in cm$^{-2}$ respectively for 2D and 3D calculations. 
\item ${\cal R}_3$ is the maximum distance in cm between an integration line and a surface. 
\item ${\cal R}_4$ is the computed tracking density in cm$^{-1}$ and in cm$^{-2}$ respectively for 2D and 3D calculations (used 
only if $\mathcal{S}^{t}_{7}=4$).
\item ${\cal R}_5$ is the computed line spacing in cm (used only if $\mathcal{S}^{t}_{7}=4$). 
\item ${\cal R}_6$ is the weight of the spatial quadrature (used only if $\mathcal{S}^{t}_{7}=4$). 
\item ${\cal R}_7$ is the minimal radius of the circle (2-D) or sphere (3-D) containing the geometry (used only if $\mathcal{S}^{t}_{7}=4$). 
\item ${\cal R}_8$ is the $x$ position of the center of the minimal circle (2-D) or sphere (3-D) containing the geometry (used 
only if $\mathcal{S}^{t}_{7}=4$). 
\item ${\cal R}_9$ is the $y$ position of the center of the minimal circle (2-D) or sphere (3-D) containing the geometry (used 
only if $\mathcal{S}^{t}_{7}=4$). 
\item ${\cal R}_{10}$ is the $z$ position of the center of the minimal circle (2-D) or sphere (3-D) containing the geometry (used 
only if $\mathcal{S}^{t}_{7}=4$). 
\item ${\cal R}_{40}$ user requested tracking density in cm$^{-1}$ for inline contruction of 3D tracks when a prismatic tracking is considered (only used if $\mathcal{S}^{t}_{39}>0$).
\end{itemize}

The \namedir{NXTRecords} directory contains the information required to track the geometry using the \moc{NXT:} module module once it has been analyzed. The contents of this directory is presented in Table~\ref{tabl:NXTRecords}.

\begin{DescriptionEnregistrement}{Global geometry records in \namedir{NXTRecords}}{8.0cm}\label{tabl:NXTRecords}
\IntEnr
  {G00000001DIM}{$40$}
  {array $N^{\text{GG}}_{i}$ containing the dimensioning information required to rebuilt the assembly} 
\IntEnr
  {G00000001CUF}{$2,N^{\text{GG}}_{5}$}
  {array $D^{\text{GG}}_{i,j}$ containing the assembly description of the geometry in terms of cells and rotations. The first element ($i=1$) identifies the cell number while the second element identifies the cell rotation} 
\IntEnr
  {G00000001CIS}{$4,N^{\text{GG}}_{4}$}
  {array $S^{\text{GG}}_{i,j}$ containing the cell intrinsic symmetry properties. A value of $1$ indicates that a center cell reflexion symmetry is present while a value of $0$ indicates that the symmetry is not considered (see below for a more complete description of this array)} 
\IntEnr
  {G00000001CFE}{$0:8,N^{\text{GG}}_{4}$}
  {array $F^{\text{GG}}_{i,j}$ containing the assembly external surface identification index (see below for a more complete description of this array)}
\DbleEnr
  {G00000001SMX}{$0:N^{\text{GG}}_{13}$}{cm}
  {array $x^{\text{GG}}$ containing the $x$-directed mesh for the cell assembly in a Cartesian or Cylindrical geometry and the $x$ position of the cell center for an hexagonal assembly (see below for more explanations)}
\DbleEnr
  {G00000001SMY}{$0:N^{\text{GG}}_{14}$}{cm}
  {array $y^{\text{GG}}$ containing the $y$-directed mesh for the cell assembly in a Cartesian or Cylindrical geometry and the $y$ position of the cell center for an hexagonal assembly (see below for more explanations)}
\OptDbleEnr
  {G00000001SMZ}{$0:N^{\text{GG}}_{15}$}{$N^{\text{GG}}_{1}=3$}{cm}
  {array $z^{\text{GG}}$ containing the $z$-directed mesh for the cell assembly (see below for more explanations)}
\OptDbleEnr
  {G00000001SMR}{$0:1$}{$N^{\text{GG}}_{2}=1$}{cm}
  {the radius $r^{\text{GG}}$ of the outer assembly boundary (see below for more explanations)}
\IntEnr
  {KEYMRG\blank{6}}{$-N^{\text{GG}}_{22}:N^{\text{GG}}_{23}$}
  {array $\text{MRG}_{i}$ containing the merged surface and region number associated with each individual surfaces and regions in this geometry}
\IntEnr
  {MATALB\blank{6}}{$-N^{\text{GG}}_{22}:N^{\text{GG}}_{23}$}
  {array containing the albedo number associated with each surface and the physical mixture number associated with each region in this geometry}
\IntEnr
  {HOMMATALB\blank{3}}{$-N^{\text{GG}}_{22}:N^{\text{GG}}_{23}$}
  {array containing the albedo number associated with each surface and the virtual (homogenization) mixture number associated with each region in this geometry}
\DbleEnr
  {SAreaRvolume}{$-N^{\text{GG}}_{22}:N^{\text{GG}}_{23}$}{}
  {array containing the area ($S_{\alpha}$ in cm for 2-D and cm$^{2}$ for 3-D problems) and volume ($V_{i}$ cm$^{2}$ for 2-D and
cm$^{3}$ for 3-D problems) associated with each surface and region in this geometry}
\end{DescriptionEnregistrement}

The dimensioning vector for the global geometry contains the following information:
\begin{itemize}
\item $N^{\text{GG}}_{1}$ number of dimensions for the problem.
\item $N^{\text{GG}}_{2}$ type of boundary, defined as follows:
  \begin{itemize}
  \item $i=0$: Cartesian geometry;
  \item $i=1$: cylindrical geometry;
  \item $i=2$: isocel geometry with specular reflection;
  \item $i=3$: hexagonal geometry with translation;
  \item $i=4$: isocel geometry with RA60 rotation and translation;
  \item $i=5$: lozenge geometry with R120 rotation and translation.
  \end{itemize}
\item $N^{\text{GG}}_{3}$ first direction to process in the analysis. For cylinder, this is the direction of the first axis of the plane normal to the cylinder axis. For Cartesian and hexagonal geometries a value of 1 ($x$-axis) is selected by default. 
\item $N^{\text{GG}}_{4}$ number of cells in the original geometry (before unfolding).
\item $N^{\text{GG}}_{5}$ number of cells in the geometry after the original geometry is unfolded according to the symmetries.
\item $N^{\text{GG}}_{6}$ diagonal symmetry flag. A value of $0$ indicates that this symmetry is not used. A value of $-1$ indicates that the symmetry is used for the $x_{-}=y_{+}$ plane and a value of $1$ that the symmetry is used for the $x_{+}=y_{}$ plane.
\item $N^{\text{GG}}_{7}$ flag to identify symmetries with respect to the $x$-axis ($x_{-}$ or $x_{+}$). A value of $0$ indicates that no symmetry is present, $N^{\text{GG}}_{7}=\pm 1$ is for a \moc{SYME} symmetry at the $x_{\pm}$ plane, $N^{\text{GG}}_{7}=\pm 2$ represents a \moc{SSYM} symmetry at the $x_{\pm}$ plane and $N^{\text{GG}}_{7}= 3$ implies a translation symmetry is the $x$ direction ($x_{-}=x_{+}$).
\item $N^{\text{GG}}_{8}$ flag to identify symmetries with respect to the $y$-axis ($y_{-}$ or $y_{+}$). A value of $0$ indicates that no symmetry is present, $N^{\text{GG}}_{7}=\pm 1$ is for a \moc{SYME} symmetry at the $y_{\pm}$ plane, $N^{\text{GG}}_{7}=\pm 2$ represents a \moc{SSYM} symmetry at the $y_{\pm}$ plane and $N^{\text{GG}}_{7}= 3$ implies a translation symmetry is the $y$ direction ($y_{-}=y_{+}$).
\item $N^{\text{GG}}_{9}$ flag to identify symmetries with respect to the $z$-axis ($z_{-}$ or $z_{+}$). A value of $0$ indicates that no symmetry is present, $N^{\text{GG}}_{7}=\pm 1$ is for a \moc{SYME} symmetry at the $z_{\pm}$ plane, $N^{\text{GG}}_{7}=\pm 2$ represents a \moc{SSYM} symmetry at the $z_{\pm}$ plane and $N^{\text{GG}}_{7}= 3$ implies a translation symmetry is the $z$ direction ($z_{-}=z_{+}$).
\item $N^{\text{GG}}_{10}$ number of $x$ mesh subdivisions or hexagons in the original geometry.
\item $N^{\text{GG}}_{11}$ number of $y$ mesh subdivisions or hexagons in the original geometry.
\item $N^{\text{GG}}_{12}$ number of $z$ mesh subdivisions in the original geometry.
\item $N^{\text{GG}}_{13}$ number of $x$ mesh subdivisions or hexagons in the unfolded geometry.
\item $N^{\text{GG}}_{14}$ number of $y$ mesh subdivisions or hexagons in the unfolded geometry.
\item $N^{\text{GG}}_{15}$ number of $z$ mesh subdivisions in the unfolded geometry.
\item $N^{\text{GG}}_{16}$ maximum number cells required to represent this geometry.
\item $N^{\text{GG}}_{17}$ maximum number of region for this geometry.
\item $N^{\text{GG}}_{18}$ total number of clusters in this geometry.
\item $N^{\text{GG}}_{19}$ maximum number of pins in this geometry.
\item $N^{\text{GG}}_{20}$ maximum dimensions of any mesh array for a cell in this geometry.
\item $N^{\text{GG}}_{21}$ maximum dimensions of any mesh array for a pin in this geometry. 
\item $N^{\text{GG}}_{22}$ number of external surfaces for this geometry.
\item $N^{\text{GG}}_{23}$ number of regions for this geometry.
\item $N^{\text{GG}}_{24}$ maximum number of external surfaces in a sub-geometry included in this geometry.
\item $N^{\text{GG}}_{25}$ maximum number of regions in a sub-geometry included in this geometry.
\item $N^{\text{GG}}_{26}$ MERGE flag defined as follows:
  \begin{itemize}
  \item $i=0$: no merge;
  \item $i=1$: {\tt MERGE MIX} applied to regions.
  \end{itemize}
\end{itemize}

The indexing of array $S^{\text{GG}}_{i,j}$ for the axis of symmetry is as follows
\begin{enumerate}
 \item Cartesian  assemblies: 
  \begin{itemize}
  \item $i=1$ refers to a reflexion of the geometry on a plane normal the $x$-axis; 
  \item $i=2$ refers to a reflexion of the geometry on a plane normal the $y$-axis; 
  \item $i=3$ refers to a reflexion of the geometry on the plane $x=y$; 
  \item $i=4$ refers to a reflexion of the geometry on a plane normal the $z$-axis. 
  \end{itemize}
\item Hexagonal assemblies (symmetries not yet programmed).
  \begin{itemize}
  \item $i=1$ refers to a reflexion of the geometry on a plane normal the $u$-axis; 
  \item $i=2$ refers to a reflexion of the geometry on a plane normal the $v$-axis; 
  \item $i=3$ refers to a reflexion of the geometry on the plane $w$; 
  \item $i=4$ refers to a reflexion of the geometry on a plane normal the $z$ axis. 
  \end{itemize}
\end{enumerate}

\begin{figure}[htbp]  
\begin{center} 
\parbox{6cm}{\epsfxsize=6cm \epsffile{HexAssmbR.eps}}\hspace{1.0cm} \parbox{6cm}{\epsfxsize=6cm \epsffile{HexFaces.eps}}
\caption{Example of an assembly of hexagons (left) and external faces identification for an hexagon}\label{fig:HexAssmbR}   
\end{center}  
\end{figure}

The indexing of array $F^{\text{GG}}_{i,j}$ for external surface identification is as follows. First $F^{\text{GG}}_{0,j}$ represents the number of times the cell appears in the geometry after it has been unfolded. For $i>0$, $F^{\text{GG}}_{i,j}$ can take the following values
$$
F^{\text{GG}}_{i,j} =\cases{ 1 & \text{surface associated with direction $i$ of cell $j$ is an external boundary of the assembly}\cr
                                   0 & \text{surface associated with direction $i$ of cell $j$ is not an external boundary of the assembly}\cr
                      }
$$
with the following planes associated with different values of $i$:
\begin{enumerate}
 \item Cartesian assemblies: 
  \begin{itemize}
  \item $i=1$ surfaces on the $x_{-}$ plane for cell $j$; 
  \item $i=2$ surfaces on the $x_{+}$ plane for cell $j$;
  \item $i=3$ surfaces on the $y_{-}$ plane for cell $j$;
  \item $i=4$ surfaces on the $y_{+}$ plane for cell $j$;
  \item $i=5$ surfaces on the $z_{-}$ plane for cell $j$; 
  \item $i=6$ surfaces on the $z_{+}$ plane for cell $j$.
  \end{itemize}
%\item Cylindrical assemblies (not yet programmed).
%  \begin{itemize}
%  \item $i=5$ identify the surfaces on the $z_{-}$ plane for cell $j$; 
%  \item $i=6$ identify the surfaces on the $z_{+}$ plane for cell $j$.
%  \item $i=8$ identify the surfaces on the $r_{-}$ plane for cell $j$; 
%  \end{itemize}
\item Hexagonal assemblies (see \Fig{HexAssmbR}):
  \begin{itemize}
  \item $i=1$ surfaces on the $u_{-}$ plane for cell $j$; 
  \item $i=2$ surfaces on the $u_{+}$ plane for cell $j$;
  \item $i=3$ surfaces on the $v_{-}$ plane for cell $j$;
  \item $i=4$ surfaces on the $v_{+}$ plane for cell $j$;
  \item $i=5$ surfaces on the $z_{-}$ plane for cell $j$; 
  \item $i=6$ surfaces on the $z_{+}$ plane for cell $j$;
  \item $i=7$ surfaces on the $w_{-}$ plane for cell $j$;
  \item $i=8$ surfaces on the $w_{+}$ plane for cell $j$.
  \end{itemize}
\end{enumerate}

The arrays $x^{\text{GG}}$, $y^{\text{GG}}$, $z^{\text{GG}}$ and $r^{\text{GG}}$ contain the following information:
\begin{enumerate}
\item Cartesian assemblies: 
  \begin{itemize}
  \item $x^{\text{GG}}_{i-1}$ and $x^{\text{GG}}_{i}$ are the lower and upper $x$ limits of mesh element $i$ ($i=1,n^{x}$);
  \item $y^{\text{GG}}_{j-1}$ and $y^{\text{GG}}_{j}$ are the lower and upper $y$ limits of mesh element $j$ ($j=1,n^{y}$);
  \item $z^{\text{GG}}_{k-1}$ and $z^{\text{GG}}_{k}$ are the lower and upper $z$ limits of mesh element $k$ ($k=1,n^{z}$).
  \end{itemize}
%\item Cylindrical assemblies: not yet programmed.
\item Hexagonal assemblies (see \Fig{HexAssmbR}): 
  \begin{itemize}
  \item $x^{\text{GG}}_{0}=h$ is the width of one face of the hexagon and $x^{\text{GG}}_{i}$ is the position in $x$ of the center of cell $i$ in the assembly;
  \item $y^{\text{GG}}_{0}=h$ is the width of one face of the hexagon and $y^{\text{GG}}_{j}$ is the position in $y$ of the center of cell $j$ in the assembly;
  \item $z^{\text{GG}}_{k-1}$ and $z^{\text{GG}}_{k}$ are the lower and upper $z$ limits of mesh element $k$ ($k=1,n^{z}$).
  \end{itemize}
\end{enumerate}

As we noted above, the global geometry is always an assembly containing cells. For each cell $i$ in this assembly, several records will be generated in the \namedir{NXTRecords} directory. These records are identified using a FORTRAN \verb|CHARACTER*12| variable as follows 
\begin{quote}
\begin{verbatim}
INTEGER      I
CHARACTER*12 NAMREC
CHARACTER*3  NREC
WRITE(NAMREC,'(A1,I8.8,A3)') 'C',I,NREC
\end{verbatim}
\end{quote}
where the variable \verb|NREC| can take the following values:
\begin{itemize}
\item DIM for dimensioning information;
\item SMR for the radial mesh description;
\item SMX for the $x$-directed mesh description;
\item SMY for the $y$-directed mesh description;
\item SMZ for the $z$-directed mesh description;
\item MIX for physical mixture description;
\item HOM for virtual mixture description;
\item VSE for areas and volumes results;
\item VSI for local surfaces and regions identification;
\item RID for final region numbering;
\item SID for final surface numbering
\item PNT for pin contents description;
\item PIN for pins location.
\end{itemize}

In Table~\ref{tabl:NXTCell}, a description of the  additional \namedir{NXTRecords} records associated with cell $i=1$ can be found.

\begin{DescriptionEnregistrement}{Cell $i=1$ records in \namedir{NXTRecords}}{8.0cm}\label{tabl:NXTCell}
\IntEnr
  {C00000001DIM}{$40$}
  {array $N^{\text{GC}}_{j}$ containing the dimensioning information required to rebuilt the cell} 
\DbleEnr
  {C00000001SMR}{$N^{\text{GC}}_{2}$}{cm}
  {array $r^{\text{GC}}_{j}$ containing the cell radial mesh description}
\DbleEnr
  {C00000001SMX}{$N^{\text{GC}}_{3}$}{cm}
  {array $x^{\text{GC}}_{j}$ containing the cell $x$-directed mesh description}
\DbleEnr
  {C00000001SMY}{$N^{\text{GC}}_{4}$}{cm}
  {array $y^{\text{GC}}_{j}$ containing the cell $y$-directed mesh description}
\DbleEnr
  {C00000001SMZ}{$N^{\text{GC}}_{5}$}{cm}
  {array $z^{\text{GC}}_{j}$ containing the cell $z$-directed mesh description}
\IntEnr
  {C00000001MIX}{$N^{\text{GC}}_{6}$}
  {array $M^{\text{GC}}_{j}$ containing the cell physical mixture for each region}
\IntEnr
  {C00000001HOM}{$N^{\text{GC}}_{6}$}
  {array $H^{\text{GC}}_{j}$ containing the cell virtual mixture for each region}
\DbleEnr
  {C00000001VSE}{$-N^{\text{GC}}_{9}:N^{\text{GC}}_{8}$}{}
  {array $\text{SV}^{\text{GC}}_{j}$ containing surface area $j$ ($\text{SV}^{\text{GC}}_{-j}=S^{\text{GC}}_{j}$ in cm for 2-D and cm$^{2}$ for 3-D problems) and
regional volumes $j$ ($\text{SV}^{\text{GC}}_{j}=V^{\text{GC}}_{j}$ in cm$^{2}$ for 2-D and cm$^{3}$ for 3-D problems)}
\IntEnr
  {C00000001VSI}{$4,-N^{\text{GC}}_{9}:N^{\text{GC}}_{8}$}
  {array $\text{VSI}^{\text{GC}}_{k,j}$ containing the location of a surface ($j<0$) and a region ($j>0$)}
\IntEnr
  {C00000001RID}{$N^{\text{GC}}_{8}$}
  {index array $\text{RID}^{\text{GC}}_{j}$ associating local and global region numbering}
\IntEnr
  {C00000001SID}{$N^{\text{GC}}_{9}$}
  {index array $\text{SID}^{\text{GC}}_{j,i}$ associating local and global outer surface numbering}
\IntEnr
  {C00000001PNT}{$3,N^{\text{GC}}_{18}$}
  {array $\text{PC}^{\text{GC}}_{k,j}$ containing the cell pin contents}
\DbleEnr
  {C00000001PIN}{$-1:4,N^{\text{GC}}_{18}$}{}
  {array $\text{p}^{\text{GC}}_{k,j}$ containing the location of the pins in cell}
\end{DescriptionEnregistrement}
Note that the record names above are built using the following FORTRAN instructions:
  \begin{quote}
    \verb|WRITE(NAMREC,'(A1,I8.8,A3)') 'C',|$i$\verb|,NAMEXT|
  \end{quote}

The cell dimensioning array $N^{\text{GC}}_{i}$ for cell $i$ contains the following information:
\begin{itemize}
\item $N^{\text{GC}}_{1}$ cell geometry type (see the definition of $\mathcal{S}^{G}_{1}$ in \Sect{geometrydirmain});
\item $N^{\text{GC}}_{2}$ dimensions of the radial mesh array; 
\item $N^{\text{GC}}_{3}$ dimensions of the $x$-directed mesh array;  
\item $N^{\text{GC}}_{4}$ dimensions of the $y$-directed mesh array;  
\item $N^{\text{GC}}_{5}$ dimensions of the $z$-directed mesh array;  
\item $N^{\text{GC}}_{6}$ dimensions of the mixture record;
\item $N^{\text{GC}}_{7}$ geometry level (1 for cell);
\item $N^{\text{GC}}_{8}$ number of regions in the cell before symmetry considerations;
\item $N^{\text{GC}}_{9}$ number of surfaces in the cell before symmetry considerations;
\item $N^{\text{GC}}_{10}$ number of regions in the cell after symmetry considerations;
\item $N^{\text{GC}}_{11}$ number of surfaces in the cell after symmetry considerations;
\item $N^{\text{GC}}_{12}$ first global region number for cell;
\item $N^{\text{GC}}_{13}$ last global region number for cell;
\item $N^{\text{GC}}_{14}$ first global surface number for cell;
\item $N^{\text{GC}}_{15}$ last global surface number for cell;
\item $N^{\text{GC}}_{16}$ number of pin cluster geometries in cell;
\item $N^{\text{GC}}_{17}$ first pin cluster geometry associated with cell;
\item $N^{\text{GC}}_{18}$ total number of pins in cell;
\item $N^{\text{GC}}_{19}$ number of times this cell is used in the global cell.
\end{itemize}
while the remaining elements are not used. 

The array $x^{\text{GC}}_{j}$ contains the following information:
\begin{itemize}
\item $x^{\text{GC}}_{-1}$ contains the displacement of the center of the cylindrical region with respect to the center of the Cartesian mesh in direction $x$. This
center is located at:
$$
x_{c}=\frac{x^{\text{Gc}}_{n^{x}}+x^{\text{GC}}_{0}}{2}
$$
where we have used $n^{x}=N^{\text{GC}}_{3}$.
\item $x^{\text{GC}}_{j-1}$ and $x^{\text{GC}}_{j}$ are the lower and upper $x$ limits of mesh element $j$ ($j=1,n^{x}$).
\end{itemize}
The array $y^{\text{GC}}_{j}$ contains the following information:
\begin{itemize}
\item $y^{\text{GC}}_{-1}$ contains the displacement of the center of the cylindrical region with respect to the center of the Cartesian mesh in direction $y$. This
center is located at:
$$
y_{c}=\frac{y^{\text{GC}}_{n^{y}}+y^{\text{GC}}_{0}}{2}
$$
where we have used $n^{y}=N^{\text{GC}}_{4}$.
\item $y^{\text{GC}}_{j-1}$ and $y^{\text{GC}}_{j}$ are the lower and upper $y$ limits of mesh element $j$ ($j=1,n^{y}$).
\end{itemize}
The array $z^{\text{GC}}_{j}$ contains the following information:
\begin{itemize}
\item $z^{\text{GC}}_{-1}$ contains the displacement of the center of the cylindrical region with respect to the center of the Cartesian mesh in direction $z$. This
center is located at:
$$
z_{c}=\frac{z^{\text{GC}}_{n^{z}}+z^{\text{GC}}_{0}}{2}
$$
where we have used $n^{z}=N^{\text{GC}}_{5}$.
\item $z^{\text{GC}}_{j-1}$ and $z^{\text{GC}}_{j}$ are the lower and upper $z$ limits of mesh element $j$ ($j=1,n^{z}$).
\end{itemize}
  The array $r^{\text{GC}}_{j}$ contains the following information:
\begin{itemize}
\item $r^{\text{GC}}_{-1}=r^{\text{GC}}_{0}=0$. 
\item $r^{\text{GC}}_{j-1}\le r\le r^{\text{GC}}_{j}$ describes the position in $r$ of mesh element $j$ ($j=1,N^{\text{GC}}_{2}$). 
\end{itemize}
The array $p^{\text{GC}}_{j}$ contains the following information:
\begin{itemize}
\item $p^{\text{GC}}_{-1}$ is the angular position of $z$-, $x$- or $y$-directed pin with respect to the $x$, $y$ or $z$ axis. 
\item $p^{\text{GC}}_{0}$ is the radial position of $z$-, $x$- or $y$-directed pin with respect to the $x-y$, $y-z$ or $z-x$ center of the cell where the pin is located. 
\item $p^{\text{GC}}_{1}$ is the height of a $x$-directed pin.
\item $p^{\text{GC}}_{2}$ is the height of a $y$-directed pin.
\item $p^{\text{GC}}_{3}$ is the height of a $z$-directed pin.
\item $p^{\text{GC}}_{4}$ is the outer radius of the pin.
\end{itemize}

In Table~\ref{tabl:NXTCluster}, a description of the additional \namedir{NXTRecords} records associated with pin $i=1$ can be found. These records are identified using a procedure similar to that used for cell records, namely
\begin{quote}
\begin{verbatim}
INTEGER      I
CHARACTER*12 NAMREC
CHARACTER*3  NREC
WRITE(NAMREC,'(A1,I8.8,A3)') 'P',I,NREC
\end{verbatim}
\end{quote}
where the variable \verb|NREC| can take the same values as for cell records, except for \moc{NREC}=\moc{PNT} and \moc{NREC}=\moc{PIN} which are now forbidden.

\begin{DescriptionEnregistrement}{Pin $i=1$ records in \namedir{NXTRecords}}{8.0cm}\label{tabl:NXTCluster}
\IntEnr
  {P00000001DIM}{$40$}
  {array $N^{\text{GP}}_{j}$ containing the dimensioning information required to rebuilt the pin} 
\DbleEnr
  {P00000001SMR}{$N^{\text{GP}}_{2}$}{cm}
  {array $r^{\text{GP}}_{j}$ containing the pin radial mesh description}
\DbleEnr
  {P00000001SMX}{$N^{\text{GP}}_{3}$}{cm}
  {array $x^{\text{GP}}_{j}$ containing the pin $x$-directed mesh description}
\DbleEnr
  {P00000001SMY}{$N^{\text{GP}}_{4}$}{cm}
  {array $y^{\text{GP}}_{j}$ containing the pin $y$-directed mesh description}
\DbleEnr
  {P00000001SMZ}{$N^{\text{GP}}_{5}$}{cm}
  {array $z^{\text{GP}}_{j}$ containing the pin $z$-directed mesh description}
\IntEnr
  {P00000001MIX}{$N^{\text{GP}}_{6}$}
  {array $M^{\text{GP}}_{j}$ containing the pin physical mixture for each region}
\IntEnr
  {P00000001HOM}{$N^{\text{GP}}_{6}$}
  {array $H^{\text{GP}}_{j}$ containing the pin virtual mixture for each region}
\DbleEnr
  {P00000001VSE}{$-N^{\text{GP}}_{9}:N^{\text{GP}}_{8}$}{}
  {array $\text{SV}^{\text{GP}}_{j}$ containing surface area $j$ ($\text{SV}^{\text{GP}}_{-j}=S^{\text{GP}}_{j}$ in cm for 2-D and cm$^{2}$ for 3-D problems) and
regional volumes $j$ ($\text{SV}^{\text{GP}}_{j}=V^{\text{GP}}_{j}$ in cm$^{2}$ for 2-D and cm$^{3}$ for 3-D problems)}
\IntEnr
  {P00000001VSI}{$4,-N^{\text{GP}}_{9}:N^{\text{GP}}_{8}$}
  {array $\text{VSI}^{\text{GP}}_{k,j}$ containing the location of a surface ($j<0$) and a region ($j>0$)}
\IntEnr
  {P00000001RID}{$N^{\text{GP}}_{8}$}
  {index array $\text{RID}^{\text{GP}}_{j}$ associating local and global region numbering}
\IntEnr
  {P00000001SID}{$N^{\text{GP}}_{9}$}
  {index array $\text{SID}^{\text{GP}}_{j,i}$ associating local and global outer surface numbering}
\end{DescriptionEnregistrement}

The pin dimensioning array $N^{\text{GP}}$ contains the following information:
\begin{itemize}
\item $N^{\text{GP}}_{1,}$ pin geometry type (see the definition of $\mathcal{S}^{G}_{1}$ in \Sect{geometrydirmain});
\item $N^{\text{GP}}_{2,}$ dimensions of the radial mesh array;
\item $N^{\text{GP}}_{3,}$ dimensions of the $x$-directed mesh array;  
\item $N^{\text{GP}}_{4,}$ dimensions of the $y$-directed mesh array;  
\item $N^{\text{GP}}_{5,}$ dimensions of the $z$-directed mesh array;  
\item $N^{\text{GP}}_{6,}$ dimensions of the mixture record;
\item $N^{\text{GP}}_{7,}$ geometry level (2 for pins);
\item $N^{\text{GP}}_{8,}$ number of regions in the pin before symmetry considerations;
\item $N^{\text{GP}}_{9,}$ number of surfaces in the pin before symmetry considerations;
\item $N^{\text{GP}}_{10}$ number of regions in the pin after symmetry considerations;
\item $N^{\text{GP}}_{11}$ number of surfaces in the pin after symmetry considerations;
\item $N^{\text{GP}}_{12}$ first global region number for pins in cluster;
\item $N^{\text{GP}}_{13}$ last global region number for pins in cluster;
\item $N^{\text{GP}}_{14}$ first global surface number for pins in cluster;
\item $N^{\text{GP}}_{15}$ last global surface number for pins in cluster;
\item $N^{\text{GP}}_{16}$ first pin cluster geometry for pins in cluster.
\item $N^{\text{GP}}_{17}$ total number of pins in cluster.
\end{itemize}
while the remaining elements are not used. 
The array $x^{\text{GP}}_{j}$ contains the following information:
\begin{itemize}
\item $x^{\text{GP}}_{-1}$ contains the displacement of the center of the cylindrical region with respect to the center of the Cartesian mesh in direction $x$. This
center is located at:
$$
x_{c}=\frac{x^{\text{GP}}_{n^{x}}+x^{\text{GP}}_{0}}{2}
$$
where we have used $n^{x}=N^{\text{GP}}_{3}$.
\item $x^{\text{GP}}_{j-1}$ and $x^{\text{GP}}_{j}$ re the lower and upper $x$ limits of mesh element $j$ ($j=1,n^{x}$).
\end{itemize}
The array $y^{\text{GP}}_{j}$ contains the following information:
\begin{itemize}
\item $y^{\text{GP}}_{-1}$ contains the displacement of the center of the cylindrical region with respect to the center of the Cartesian mesh in direction $y$. This
center is located at:
$$
y_{c}=\frac{y^{\text{GP}}_{n^{y}}+y^{\text{GP}}_{0}}{2}
$$
where we have used $n^{y}=N^{\text{GP}}_{4}$.
\item $y^{\text{GP}}_{j-1}$ and $y^{\text{GP}}_{j}$ are the lower and upper $y$ limits of mesh element $j$ ($j=1,n^{y}$).
\end{itemize}
The array $z^{\text{GP}}_{j}$ contains the following information:
\begin{itemize}
\item $z^{\text{GP}}_{-1}$ contains the displacement of the center of the cylindrical region with respect to the center of the Cartesian mesh in direction $z$. This
center is located at:
$$
z_{c}=\frac{z^{\text{GP}}_{n^{z}}+z^{\text{GP}}_{0}}{2}
$$
where we have used $n^{z}=N^{\text{GP}}_{5}$.
\item $z^{\text{GP}}_{j-1}$ and $z^{\text{GP}}_{j}$ are the lower and upper $z$ limits of mesh element $j$ ($j=1,n^{z}$).
\end{itemize}
  The array $r^{\text{GP}}_{j}$ contains the following information:
\begin{itemize}
\item $r^{\text{GP}}_{-1}=r^{\text{GP}}_{0}=0$. 
\item $r^{\text{GP}}_{j-1}\le r\le r^{\text{GP}}_{j}$ describes the position in $r$ of mesh element $j$ with $j=1,N^{\text{GP}}_{2}$. 
\end{itemize}

Finally the \namedir{NXTRecords} directory also contains records associated with global identification of the surfaces and volumes as illustrated in Table~\ref{tabl:NXTGlobal}.

\begin{DescriptionEnregistrement}{Global geometry records in \namedir{NXTRecords}}{8.0cm}\label{tabl:NXTGlobal}
\DbleEnr
  {TrackingDnsA}{$\mathcal{S}^{t}_{20}$}{cm}
  {array $D_{i}$ containing the spatial spacing for each track direction}
\DbleEnr
  {TrackingDirc}{$N^{\text{GG}}_{1},\mathcal{S}^{t}_{20}$}{}
  {array $\alpha_{j,i}$ containing the director cosine for axis $j$ for each track direction}
\DbleEnr
  {TrackingOrig}{$N^{\text{GG}}_{1},N_{p},\mathcal{S}^{t}_{20}$}{cm}
  {array $L_{k,j,i}$ containing the origin in space ($k=1,N^{\text{GG}}_{1}$) and the direction of the normal plan for each plane $j$ and track direction $i$}
\DbleEnr
  {TrackingWgtD}{$\mathcal{S}^{t}_{20}$}{}
  {array $W_{i}$ containing the integration weight for each track direction}
\IntEnr
  {NumMerge\blank{4}}{$N^{\text{GG}}_{23}$}
  {merging indices used for normalizing the tracks}
\DbleEnr
  {VolMerge\blank{4}}{$N^{\text{GG}}_{23}$}{}
  {volumes used for normalizing the tracks}
\DbleEnr
  {VTNormalize\blank{1}}{$N^{\text{GG}}_{23}$}{}
  {array $R_{i}$ containing the ratio of the analytical and numerical volume for each region}
\OptDbleEnr
  {VTNormalizeD}{$N^{\text{GG}}_{23},\mathcal{S}^{t}_{20}$}{$\mathcal{S}^{t}_{8}=-1$}{}
  {array $R_{i}$ containing the ratio of the analytical and numerical volume for region $i$ for each track direction}
\IntEnr
  {KEYMRG\blank{6}}{$-N^{\text{GG}}_{22}:N^{\text{GG}}_{23}$}
  {array $\text{MRG}_{i}$ containing the global merging index}
\IntEnr
  {MATALB\blank{6}}{$-N^{\text{GG}}_{22}:N^{\text{GG}}_{23}$}
  {array containing the albedo number associated with each surface and the physical mixture number associated with each region}
\IntEnr
  {HOMMATALB\blank{3}}{$-N^{\text{GG}}_{22}:N^{\text{GG}}_{23}$}
  {array containing the albedo number associated with each surface and the virtual mixture number associated with each region}
\DbleEnr
  {SAreaRvolume}{$-N^{\text{GG}}_{22}:N^{\text{GG}}_{23}$}{}
  {array containing the area ($S_{\alpha}$ in cm for 2-D and cm$^{2}$ for 3-D problems) and volumes ($V_{i}$ cm$^{2}$ for 2-D and
cm$^{3}$ for 3-D problems) of each external surface and region in the geometry}
\end{DescriptionEnregistrement}

\clearpage
