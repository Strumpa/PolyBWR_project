\section{EXAMPLES OF INPUT DATA FILES}

\subsection{IAEA-2D benchmark}

The IAEA-2D benchmark is defined in Refs.~\citen{bivac,anl} and its geometry is represented in Fig.~\fig(iaea2d). Here, it is solved using a parabolic variational collocation method without mesh splitting of the elements:

\begin{figure}[htbp]
\begin{center} 
\epsfxsize=8.0cm
\centerline{ \epsffile{iaea2d.eps}}
\parbox{14cm}{\caption{Description of the IAEA-2D benchmark.}\label{fig:iaea2d}}  \end{center} 
\end{figure}

\begin{verbatim}
LINKED_LIST IAEA MACRO TRACK SYSTEM FLUX EDIT ;
MODULE GEO: MAC: TRIVAT: TRIVAA: FLUD: OUT: END: ;
*
IAEA := GEO: :: CAR2D 9 9
           EDIT 2
           X- DIAG X+ VOID
           Y- SYME Y+ DIAG
           MIX  3 2 2 2 3 2 2 1 4
                  2 2 2 2 2 2 1 4
                    2 2 2 2 1 1 4
                      2 2 2 1 4 4
                        3 1 1 4 0
                          1 4 4 0
                            4 0 0
                              0 0
                                0
           MESHX 0.0 20.0 40.0 60.0 80.0 100.0 120.0 140.0 160.0 180.0
           ;
MACRO := MAC: ::
 EDIT 2 NGRO 2 NMIX 4
 READ INPUT
 MIX     1
      DIFF  1.500E+00  4.0000E-01
     TOTAL  3.012E-02  8.0032E-02
    NUSIGF  0.000E+00  1.3500E-01
  H-FACTOR  0.000E+00  1.3500E-01
      SCAT  1 1 0.0 2 2 0.0 0.2E-01
 MIX     2
      DIFF  1.500E+00  4.0000E-01
     TOTAL  3.012E-02  8.5032E-02
    NUSIGF  0.000E+00  1.3500E-01
  H-FACTOR  0.000E+00  1.3500E-01
      SCAT  1 1 0.0 2 2 0.0 0.2E-01
 MIX     3
      DIFF  1.500E+00  4.00000E-01
     TOTAL  3.012E-02  1.30032E-01
    NUSIGF  0.000E+00  1.35000E-01
  H-FACTOR  0.000E+00  1.35000E-01
      SCAT  1 1 0.0 2 2 0.0 0.2E-01
 MIX     4
      DIFF  2.000E+00  3.0000E-01
     TOTAL  4.016E-02  1.0024E-02
      SCAT  1 1 0.0 2 2 0.0 0.4E-01
 ;
TRACK := TRIVAT: IAEA ::
      TITLE 'IAEA-2D BENCHMARK'
      MAXR 81 PRIM 2 ;
SYSTEM := TRIVAA: MACRO TRACK :: ;
FLUX := FLUD: SYSTEM ::
      EDIT 2 ;
EDIT := OUT: FLUX ::
       EDIT 2 INTG
       1  2  3  4  5  6  7  8  0
          9 10 11 12 13 14 15  0
            16 17 18 19 20 21  0
               22 23 24 25  0  0
                  26 27 28  0  0
                     29  0  0  0
                         0  0  0
                            0  0
                               0
       ;
END: ;
\end{verbatim}

\subsection{Biblis-2D benchmark}

The rods-withdrawn configuration of the Biblis-2D benchmark is defined in Ref.~\citen{bivac} and its geometry is represented in Fig.~\fig(biblis). Here, it is solved using a parabolic variational collocation method without mesh splitting of the elements:

\begin{figure}[htbp]
\begin{center} 
\epsfxsize=8.0cm
\centerline{ \epsffile{biblis.eps}}
\parbox{14cm}{\caption{Description of the Biblis-2D benchmark, rods-withdrawn configuration.}\label{fig:biblis}}  \end{center} 
\end{figure}

\begin{verbatim}
LINKED_LIST BIBLIS MACRO TRACK SYSTEM FLUX EDIT ;
MODULE GEO: MAC: TRIVAT: TRIVAA: FLUD: OUT: END: ;
*
BIBLIS := GEO: :: CAR2D 9 9
           EDIT 2
           X- DIAG X+ VOID
           Y- SYME Y+ DIAG
       MIX 1 8 2 6 1 7 1 4 3
             1 8 2 8 1 1 4 3
               1 8 2 7 1 4 3
                 2 8 1 8 4 3
                   2 5 4 3 3
                     4 4 3 0
                       3 3 0
                         0 0
                           0
       MESHX 0.0 23.1226 46.2452 69.3678 92.4904 115.613 138.7356
             161.8582 184.9808 208.1034
       ;
MACRO := MAC: ::
 EDIT 2 NGRO 2 NMIX 8
 READ INPUT
 MIX     1
      DIFF  1.436000E+00  3.635000E-01
     TOTAL  2.725820E-02  7.505800E-02
    NUSIGF  5.870800E-03  9.606700E-02
  H-FACTOR  2.376800E-03  3.889400E-02
      SCAT  1 1 0.0 2 2 0.0  1.775400E-02
 MIX     2
      DIFF  1.436600E+00  3.636000E-01
     TOTAL  2.729950E-02  7.843600E-02
    NUSIGF  6.190800E-03  1.035800E-01
  H-FACTOR  2.506400E-03  4.193500E-02
      SCAT  1 1 0.0 2 2 0.0  1.762100E-02
 MIX     3
      DIFF  1.320000E+00  2.772000E-01
     TOTAL  2.576220E-02  7.159600E-02
      SCAT  1 1 0.0 2 2 0.0  2.310600E-02
 MIX     4
      DIFF  1.438900E+00  3.638000E-01
     TOTAL  2.746400E-02  9.140800E-02
    NUSIGF  7.452700E-03  1.323600E-01
  H-FACTOR  3.017300E-03  5.358700E-02
      SCAT  1 1 0.0 2 2 0.0  1.710100E-02
 MIX     5
      DIFF  1.438100E+00  3.665000E-01
     TOTAL  2.729300E-02  8.482800E-02
    NUSIGF  6.190800E-03  1.035800E-01
  H-FACTOR  2.506400E-03  4.193500E-02
      SCAT  1 1 0.0 2 2 0.0  1.729000E-02
 MIX     6
      DIFF  1.438500E+00  3.665000E-01
     TOTAL  2.732400E-02  8.731400E-02
    NUSIGF  6.428500E-03  1.091100E-01
  H-FACTOR  2.602600E-03  4.417400E-02
      SCAT  1 1 0.0 2 2 0.0  1.719200E-02
 MIX     7
      DIFF  1.438900E+00  3.679000E-01
     TOTAL  2.729000E-02  8.802400E-02
    NUSIGF  6.190800E-03  1.035800E-01
  H-FACTOR  2.506400E-03  4.193500E-02
      SCAT  1 1 0.0 2 2 0.0  1.712500E-02
 MIX     8
      DIFF  1.439300E+00  3.680000E-01
     TOTAL  2.732100E-02  9.051000E-02
    NUSIGF  6.428500E-03  1.091100E-01
  H-FACTOR  2.602600E-03  4.417400E-02
      SCAT  1 1 0.0 2 2 0.0  1.702700E-02
       ;
TRACK := TRIVAT: BIBLIS ::
      TITLE 'BIBLIS BENCHMARK'
      EDIT 5 MAXR 81 PRIM 2 ;
SYSTEM := TRIVAA: MACRO TRACK ::
      EDIT 5 ;
FLUX := FLUD: SYSTEM ::
      EDIT 2 ;
EDIT := OUT: FLUX ::
       EDIT 2 INTG
       1  2  3  4  5  6  7  8  0
          9 10 11 12 13 14 15  0
            16 17 18 19 20 21  0
               22 23 24 25 26  0
                  27 28 29  0  0
                     30 31  0  0
                         0  0  0
                            0  0
                               0
       ;
END: ;
\end{verbatim}
\eject

\subsection{IAEA-3D benchmark}

The IAEA-3D benchmark is defined in Ref.~\citen{anl} and its geometry is represented in Fig.~\fig(iaea3d). Here, it is solved using a cubic mixed-dual method with mesh splitting of the second axial plane:

\begin{figure}[htbp]
\begin{center} 
\epsfxsize=15cm
\centerline{ \epsffile{iaea3d.eps}}
\parbox{14cm}{\caption{Description of the IAEA-3D benchmark.}\label{fig:iaea3d}}  \end{center} 
\end{figure}

\begin{verbatim}
LINKED_LIST IAEA3D MACRO TRACK SYSTEM FLUX EDIT ;
MODULE GEO: MAC: TRIVAT: TRIVAA: FLUD: OUT: END: ;
*
IAEA3D := GEO: :: CAR3D 9 9 4
          EDIT 2
          X- DIAG  X+ VOID 
          Y- SYME  Y+ DIAG 
          Z- VOID  Z+ VOID 
          MESHX 0.0 20.0 40.0 60.0 80.0 100.0 120.0 140.0 160.0 180.0 
          MESHZ 0.0 20.0 280.0 360.0 380.0 
          SPLITZ 1 2 1 1
          (* PLANE NB 1 *) 
          MIX 4 4 4 4 4 4 4 4 4 
                4 4 4 4 4 4 4 4 
                  4 4 4 4 4 4 4 
                    4 4 4 4 4 4 
                      4 4 4 4 0 
                        4 4 4 0 
                          4 0 0 
                            0 0 
                              0 
              (* PLANE NB 2 *) 
              3 2 2 2 3 2 2 1 4 
                2 2 2 2 2 2 1 4 
                  2 2 2 2 1 1 4 
                    2 2 2 1 4 4 
                      3 1 1 4 0 
                        1 4 4 0 
                          4 0 0 
                            0 0 
                              0 
              (* PLANE NB 3 *) 
              3 2 2 2 3 2 2 1 4 
                2 2 2 2 2 2 1 4 
                  3 2 2 2 1 1 4 
                    2 2 2 1 4 4 
                      3 1 1 4 0 
                        1 4 4 0 
                          4 0 0 
                            0 0 
                              0 
              (* PLANE NB 4 *) 
              5 4 4 4 5 4 4 4 4 
                4 4 4 4 4 4 4 4 
                  5 4 4 4 4 4 4 
                    4 4 4 4 4 4 
                      5 4 4 4 0 
                        4 4 4 0 
                          4 0 0 
                            0 0 
                              0 
           ;
MACRO := MAC: ::
 EDIT 2 NGRO 2 NMIX 5
 READ INPUT
 MIX     1
      DIFF  1.500E+00  4.0000E-01
     TOTAL  3.000E-02  8.0000E-02
    NUSIGF  0.000E+00  1.3500E-01
  H-FACTOR  0.000E+00  1.3500E-01
      SCAT  1 1 0.0 2 2 0.0 0.2E-01
 MIX     2
      DIFF  1.500E+00  4.0000E-01
     TOTAL  3.000E-02  8.5000E-02
    NUSIGF  0.000E+00  1.3500E-01
  H-FACTOR  0.000E+00  1.3500E-01
      SCAT  1 1 0.0 2 2 0.0 0.2E-01
 MIX     3
      DIFF  1.500E+00  4.00000E-01
     TOTAL  3.000E-02  1.30000E-01
    NUSIGF  0.000E+00  1.35000E-01
  H-FACTOR  0.000E+00  1.35000E-01
      SCAT  1 1 0.0 2 2 0.0 0.2E-01
 MIX     4
      DIFF  2.000E+00  3.0000E-01
     TOTAL  4.000E-02  1.0000E-02
      SCAT  1 1 0.0 2 2 0.0 0.4E-01
 MIX     5
      DIFF  2.000E+00  3.0000E-01
     TOTAL  4.000E-02  5.5000E-02
      SCAT  1 1 0.0 2 2 0.0 0.4E-01
 ;
TRACK := TRIVAT: IAEA3D ::
      TITLE 'TEST IAEA 3D'
      EDIT 5 MAXR 405 DUAL 3 1 ;
SYSTEM := TRIVAA: MACRO TRACK ::
      EDIT 5 ;
FLUX := FLUD: SYSTEM ::
      EDIT 2 ;
EDIT := OUT: FLUX ::
       EDIT 2 INTG
       (* PLANE NB 1 *) 
       0  0  0  0  0  0  0  0  0 
          0  0  0  0  0  0  0  0 
             0  0  0  0  0  0  0 
                0  0  0  0  0  0 
                   0  0  0  0  0
                      0  0  0  0
                         0  0  0
                            0  0
                               0
       (* PLANE NB 2 *) 
       1  2  3  4  5  6  7  8  0 
          9 10 11 12 13 14 15  0 
            16 17 18 19 20 21  0 
               22 23 24 25  0  0 
                  26 27 28  0  0
                     29  0  0  0
                         0  0  0
                            0  0
                               0
       (* PLANE NB 3 *) 
       30 31 32 33 34 35 36 37  0 
          38 39 40 41 42 43 44  0 
             45 46 47 48 49 50  0 
                51 52 53 54  0  0 
                   55 56 57  0  0
                      58  0  0  0
                          0  0  0
                             0  0
                                0
       (* PLANE NB 4 *) 
       0  0  0  0  0  0  0  0  0 
          0  0  0  0  0  0  0  0 
             0  0  0  0  0  0  0 
                0  0  0  0  0  0 
                   0  0  0  0  0
                      0  0  0  0
                         0  0  0
                            0  0
                               0
       ;
END: ;
\end{verbatim}

\subsection{S30 hexagonal benchmark in 2-D}

The S30 hexagonal benchmark in 2-D is defined in Ref.~\citen{benaboud}. Its geometry is represented in Fig.~\fig(hexS30). Here, it is solved using a mesh centered finite difference method without mesh splitting of the hexagonal elements:

\begin{figure}[htbp]
\begin{center} 
\epsfxsize=6cm
\centerline{ \epsffile{hexS30.eps}}
\parbox{14cm}{\caption{Description of the S30 hexagonal benchmark.}\label{fig:hexS30}}  \end{center} 
\end{figure}

\begin{verbatim}
LINKED_LIST HEX MACRO TRACK SYSTEM FLUX EDIT ;
MODULE GEO: MAC: TRIVAT: TRIVAA: FLUD: OUT: END: ;
*
HEX := GEO: :: HEX   6
       EDIT 2
       HBC S30 ZERO
       SIDE 13.044
       SPLITH 0
       MIX
       1
       2
       2  2
       3  3
       ;
MACRO := MAC: ::
 EDIT 2 NGRO 2 NMIX 3
 READ INPUT
 MIX     1
      DIFF  1.5E+00  4.00E-01
     TOTAL  3.0E-02  1.30E-01
    NUSIGF  0.0E+00  1.35E-01
  H-FACTOR  0.0E+00  1.35E-01
      SCAT  1 1 0.0 2 2 0.0  0.2E-01
 MIX     2
      DIFF  1.5E+00  4.00E-01
     TOTAL  3.0E-02  8.50E-02
    NUSIGF  0.0E+00  1.35E-01
  H-FACTOR  0.0E+00  1.35E-01
      SCAT  1 1 0.0 2 2 0.0  0.2E-01
 MIX     3
      DIFF  2.0E+00  3.0E-01
     TOTAL  4.0E-02  1.0E-02
      SCAT  1 1 0.0 2 2 0.0  0.4E-01
 ;
TRACK := TRIVAT: HEX ::
      TITLE 'S30 HEXAGONAL BENCHMARK IN 2-D.'
      EDIT 5 MAXR 50 MCFD (* IELEM= *) 1 ;
SYSTEM := TRIVAA: MACRO TRACK ::
      EDIT 5 ;
FLUX := FLUD: SYSTEM ::
      EDIT 2 ;
EDIT := OUT: FLUX ::
       EDIT 2 INTG IN ;
END: ;
\end{verbatim}

\subsection{LMW benchmark in 2-D}

The LMW benchmark in 2-D is a space-time kinetics problem introduced by Greenman\cite{greenman} and used by Monier\cite{monier}.
Its geometry is represented in Fig.~\fig(lmw). Here, it is solved using a parabolic nodal collocation method with $2\times 2$ mesh splitting
of each element. A reactivity transient is induced by the rapid withdrawal of the control rod in material mixture 6. The control rod is
removed in 26.7 s, causing a negative ramp variation in total cross section.

\begin{figure}[htbp]
\begin{center} 
\epsfxsize=7cm
\centerline{ \epsffile{lmw.eps}}
\parbox{14cm}{\caption{Description of the LMW benchmark in 2-D.}\label{fig:lmw}}  \end{center} 
\end{figure}

\goodbreak
\begin{verbatim}
*----
*  TEST CASE LMW 2D
*
*  REF: G. Greenman, "A Quasi-Static Flux Synthesis Temporal Integration
*       Scheme for an Analytic Nodal Method," Nuclear Engineer's Thesis,
*       Massachusetts Institute of Technology, Department of Nuclear
*       Engineering (May 1980).
*
*----
*  Define STRUCTURES and MODULES used
*----
LINKED_LIST LMW TRACK MACRO1 SYSTEM1 MACRO2 SYSTEM2 FLUX KINET ;
MODULE GEO: MAC: TRIVAT: TRIVAA: FLUD: INIKIN: KINSOL: GREP: DELETE:
       END: ;
REAL fnorm sigt1 sigt2 ;
REAL TIME := 0.0 ;
PROCEDURE assertS assertS2 ;
*
LMW := GEO: :: CAR2D 6 6
      X- REFL X+ ZERO
      Y- REFL Y+ ZERO
      MIX 1 1 1 2 3 4
          1 1 1 1 3 4
          1 1 5 1 3 4
          6 1 1 3 3 4
          3 3 3 3 4 4
          4 4 4 4 4 0
      MESHX  0.0 10. 30. 50. 70. 90. 110.
      MESHY  0.0 10. 30. 50. 70. 90. 110.
      SPLITX 2 2 2 2 2 2
      SPLITY 2 2 2 2 2 2
      ;
MACRO1 := MAC: ::
 EDIT 0 NGRO 2 NMIX 6
 READ INPUT
 MIX     1
      DIFF  1.423910E+00  3.563060E-01
     TOTAL  2.795756E-02  8.766216E-02
    NUSIGF  6.477691E-03  1.127328E-01
  H-FACTOR  2.591070E-03  4.509310E-02
      SCAT  1 1 0.0 2 2 0.0  0.175555E-01
      OVERV 0.800E-07  4.000E-06
 MIX     2
      DIFF  1.423910E+00  3.563060E-01
     TOTAL  2.850756E-02  9.146219E-02
    NUSIGF  6.477691E-03  1.127328E-01
  H-FACTOR  2.591070E-03  4.509310E-02
      SCAT  1 1 0.0 2 2 0.0  0.175555E-01
      OVERV 0.800E-07  4.000E-06
 MIX     3
      DIFF  1.425610E+00  3.505740E-01
     TOTAL  2.817031E-02  9.925634E-02
    NUSIGF  7.503282E-03  1.378004E-01
  H-FACTOR  3.001310E-03  5.512106E-02
      SCAT  1 1 0.0 2 2 0.0  0.171777E-01
      OVERV 0.800E-07  4.000E-06
 MIX     4
      DIFF  1.634220E+00  2.640020E-01
     TOTAL  3.025750E-02  4.936351E-02
      SCAT  1 1 0.0 2 2 0.0  0.275969E-01
      OVERV 0.800E-07  4.000E-06
 MIX     5
      DIFF  1.423910E+00  3.563060E-01
     TOTAL  2.795756E-02  8.766216E-02
    NUSIGF  6.477691E-03  1.127328E-01
  H-FACTOR  2.591070E-03  4.509310E-02
      SCAT  1 1 0.0 2 2 0.0  0.175555E-01
      OVERV 0.800E-07  4.000E-06
 MIX     6
      DIFF  1.423910E+00  3.563060E-01
     TOTAL  2.850756E-02  9.146217E-02
    NUSIGF  6.477691E-03  1.127328E-01
  H-FACTOR  2.591070E-03  4.509310E-02
      SCAT  1 1 0.0 2 2 0.0  0.175555E-01
      OVERV 0.800E-07  4.000E-06
 ;
TRACK := TRIVAT: LMW ::
      TITLE 'LMW 2-D BENCHMARK'
      EDIT 1 MAXR 144 MCFD 2 ;
SYSTEM1 := TRIVAA: MACRO1 TRACK ::
      EDIT 1 UNIT ;
FLUX := FLUD: SYSTEM1 TRACK ::
      EDIT 1 EXTE 5.0E-7 ;
assertS FLUX :: 'K-EFFECTIVE' 1 1.014803 ;
*----
*  Crank-Nicholson space-time kinetics
*----
EVALUATE TIME := 0.0 ;
KINET := INIKIN: MACRO1 TRACK SYSTEM1 FLUX :: EDIT 1
      NDEL 6
      BETA   0.000247 0.0013845 0.001222 0.0026455 0.000832 0.000169
      LAMBDA 0.0127 0.0317 0.115 0.311 1.40 3.87    
      CHID   1.0 1.0 1.0 1.0 1.0 1.0
             0.0 0.0 0.0 0.0 0.0 0.0
      NORM POWER-INI 1.0E4 ;
EVALUATE sigt1 := 2.850756E-02 ;
EVALUATE sigt2 := 9.146217E-02  ;
WHILE TIME 26.7 <= DO
  EVALUATE sigt1 := sigt1 5.5E-4 0.1 26.7 / * - ;
  EVALUATE sigt2 := sigt2 3.8E-3 0.1 26.7 / * - ;
  MACRO2 := MAC: MACRO1 ::
      EDIT 0
      READ INPUT
      MIX     6
         TOTAL  <<sigt1>>  <<sigt2>>
      ;
  SYSTEM2 := TRIVAA: MACRO2 TRACK ::
      EDIT 1 UNIT ;
  KINET := KINSOL: KINET MACRO2 TRACK SYSTEM2 MACRO1 SYSTEM1 ::
      EDIT 5  DELTA 0.1 
      SCHEME  FLUX CRANK PREC CRANK EXTE 1.0E-6 ;
  GREP: KINET :: GETVAL 'TOTAL-TIME' 1 >>TIME<< ;
  ECHO "TIME=" TIME "S" "sigt=" sigt1 sigt2 ;
  IF TIME 1.0 - ABS 1.0E-3 < THEN
    assertS2 KINET :: 'CTRL-FLUX' 1 1.986270E+02 ;
    assertS2 KINET :: 'CTRL-PREC' 1 1.095509E-01 ;
    assertS2 KINET :: 'E-POW'     1 1.008753E+04 ;
  ELSEIF TIME 5.0 - ABS 1.0E-3 < THEN
    assertS2 KINET :: 'CTRL-FLUX' 1 2.090369E+02 ;
    assertS2 KINET :: 'CTRL-PREC' 1 1.097266E-01 ;
    assertS2 KINET :: 'E-POW'     1 1.063990E+04 ;
  ELSEIF TIME 10.0 - ABS 1.0E-3 < THEN
    assertS2 KINET :: 'CTRL-FLUX' 1 2.305455E+02 ;
    assertS2 KINET :: 'CTRL-PREC' 1 1.104699E-01 ;
    assertS2 KINET :: 'E-POW'     1 1.176902E+04 ;
  ELSEIF TIME 15.0 - ABS 1.0E-3 < THEN
    assertS2 KINET :: 'CTRL-FLUX' 1 2.641221E+02 ;
    assertS2 KINET :: 'CTRL-PREC' 1 1.121002E-01 ;
    assertS2 KINET :: 'E-POW'     1 1.352433E+04 ;
  ELSEIF TIME 20.0 - ABS 1.0E-3 < THEN
    assertS2 KINET :: 'CTRL-FLUX' 1 3.157370E+02 ;
    assertS2 KINET :: 'CTRL-PREC' 1 1.150681E-01 ;
    assertS2 KINET :: 'E-POW'     1 1.621938E+04 ;
  ELSEIF TIME 25.0 - ABS 1.0E-3 < THEN
    assertS2 KINET :: 'CTRL-FLUX' 1 3.971426E+02 ;
    assertS2 KINET :: 'CTRL-PREC' 1 1.200883E-01 ;
    assertS2 KINET :: 'E-POW'     1 2.047011E+04 ;
  ELSEIF TIME 26.7 - ABS 1.0E-3 < THEN
    assertS2 KINET :: 'CTRL-FLUX' 1 4.351272E+02 ;
    assertS2 KINET :: 'CTRL-PREC' 1 1.224600E-01 ;
    assertS2 KINET :: 'E-POW'     1 2.245449E+04 ;
  ENDIF ;
  MACRO1 SYSTEM1 := DELETE: MACRO1 SYSTEM1 ;
  MACRO1 := MACRO2 ;
  SYSTEM1 := SYSTEM2 ;
  MACRO2 SYSTEM2 := DELETE: MACRO2 SYSTEM2 ;
ENDWHILE ;
ECHO "test lmw2D completed" ;
\end{verbatim}
