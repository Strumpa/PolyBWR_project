\subsection{Contents of \dir{d2p\_info} data structure}\label{sect:d2pdir}

A \dir{d2p\_info} data structure is used to store cross sections and 
Saphyb information such as keff, kinf, assembly discontinuity factors. This object has a signature {\tt L\_INFO}; it is created using the \moc{D2P:} module.

\subsubsection{The state-vector content}\label{sect:infostate}

\noindent
The dimensioning parameters $\mathcal{S}_i$, which are stored in the state
vector for this data structure, represent:

\begin{itemize}

\item The number of energy groups $N_{\rm gr} = \mathcal{S}_1$

\item The number of pkey for the generation of the PMAXS file $N_{\rm pk} = \mathcal{S}_2$

\item The number of cross sections recovered using the \moc{SCR:} module $N_{\rm xs} = \mathcal{S}_3$

\item The number of points with respect to burnup $N_{\rm bu}$ = $\mathcal{S}_4$

\item The type of branching calculation requested for the generation of PMAXS tree $I_{\rm grid} = \mathcal{S}_5$

\begin{displaymath} I_{\rm grid} = \left\{
\begin{array}{rl}
 0 & \textrm{default meshing} \\
 1 & \textrm{according to the saphyb content} \\
 2 & \textrm{user defined with global otpion} \\
 3 & \textrm{user defined with add otpion} \\
 4 & \textrm{user defined with add new otpion} \\
\end{array} \right.
\end{displaymath}

\item The number of control rod composition $N_{\rm comp}$ = $\mathcal{S}_6$
\item The number of delay neutron groups $N_{\rm del}$ = $\mathcal{S}_7$
\item The number of columns in assembly $N_{\rm cols}$ = $\mathcal{S}_8$
\item The number of rows in assembly $N_{\rm rows}$ = $\mathcal{S}_9$
\item The computed part of asembly assembly $I_{\rm part}$ = $\mathcal{S}_{10}$
\begin{displaymath} I_{\rm part} = \left\{
\begin{array}{rl}
 0 & \textrm{whole} \\
 1 & \textrm{half} \\
 2 & \textrm{quarter} \\
 3 & \textrm{eighth} \\ 
\end{array} \right.
\end{displaymath}

\item The number of sides in assembly. $N_{\rm surf} = \mathcal{S}_{11}$

\item The number of corners in assembly $N_{\rm corn} = \mathcal{S}_{12}$

\item The number of assembly discontinuity factors per energy groups $N_{\rm adf} = \mathcal{S}_{13}$

\item The number of ADF-type boundary flux edits. $N_{\rm type} = \mathcal{S}_{14}$

\item The number of corner discontinuity factors per energy groups. $N_{\rm cdf} = \mathcal{S}_{15}$

\item The number of group-wise form functions per energy groups in compued part of assembly. $N_{\rm gff} = \mathcal{S}_{16}$

\item The number of pin on each side of the assembly. $N_{\rm pin} = \mathcal{S}_{17}$

\item The type of cross section library containing the cross sections. $I_{\rm typ} = \mathcal{S}_{18}$

\begin{displaymath} I_{\rm typ} = \left\{
\begin{array}{rl}
 0 & \textrm{saphyb} \\
 1 & \textrm{multicompo} \\
\end{array} \right.
\end{displaymath}

\item The temperature unit for moderator and fuel. $I_{\rm temp} = \mathcal{S}_{19}$

\begin{displaymath} I_{\rm temp} = \left\{
\begin{array}{rl}
 0 & \textrm{celsius} \\
 1 & \textrm{kelvin } \\
\end{array} \right.
\end{displaymath}

\item The number of additional parameters contained in the saphyb (other than state  variables). $N_{\rm oth} = \mathcal{S}_{20}$

\item The treatment of Assembly Discontinuity Factors. $I_{\rm merge} = \mathcal{S}_{21}$
\begin{displaymath} I_{\rm merge} = \left\{
\begin{array}{rl}
 0 & \textrm{ADF are explicitly stored in the PMAXS} \\
 1 & \textrm{ADF are inserted in the cross sections} \\
\end{array} \right.
\end{displaymath}


\item The correction on Pm 147 fission yield. $I_{\rm cor} = \mathcal{S}_{22}$

\begin{displaymath} I_{\rm cor} = \left\{
\begin{array}{rl}
 0 & \textrm{No correction of Pm 147 fission yield} \\
 1 & \textrm{Correction of Pm 147 fission yield} \\
\end{array} \right.
\end{displaymath}
\end{itemize}
\clearpage
\noindent
The following records and sub-directories will be found on the first level of \dir{d2p\_info}
directory:

\begin{DescriptionEnregistrement}{Records and sub-directories
 in \dir{d2p\_info} data structure}{7.0cm} \label{tabl:tabinfo}

\CharEnr
 {SIGNATURE\blank{3}}{$*12$}
 {Signature of the \dir{d2p\_info} data structure ($\mathsf{SIGNA}=${\tt L\_INFO\blank{6}}).}

  \IntEnr
  {STATE-VECTOR}{$40$}
  { Vector describing the various parameters associated with this data structure}

\IntEnr
  {BARR\_INFO\blank{3}}{$N_{\rm comp}$}
  { Meaning of control rod in the saphyb object. $N_{comp}$ is the number of composition for control rods (including the unrodded case)}
  

  
\OptDirEnr
 {SAPHYB\_INFO\blank{1}}{$iph\geq 1$}
 {Information related to the saphyb content.}  
 
 \OptDirEnr
 {HELIOS\_HEAD\blank{1}}{$iph\geq 1$}
 {Information related to the header of output \moc{.DRA} file}  
 
 \OptDirEnr
 {GENPMAXS\_INP}{$iph\geq 2$}
 {Information related to the GenPMAXS input file.}  
 
 \OptDirEnr
 {BRANCH\_INFO\blank{1}}{$iph\geq 3$}
 {Information related to the current branch calculation.}  
 
 \OptDirlEnr
 {TH\_DATA\blank{5}}{$N_{\rm bu}$}{$iph\geq 2$}
 {Information related to the invariant T/H data block. Each componant of the list is a directory containing TH data for a single burnup point.}  

\end{DescriptionEnregistrement}

%\subsubsection*{The subdirectory \moc{SAPHYB$\_$INFO}}


\begin{DescriptionEnregistrement}{Records and sub-directories
 in \dir{SAPHYB\_INFO} }{7.0cm} \label{tabl:tabsap}
 
\DirlEnr
{PKEY\_INFO\blank{3}}{6}
{Each component of the list is a directory containing information for all possible state variables.}  

\CharEnr
 {OTHPK\blank{7}}{$(N_{\rm oth})*12$}  {Name of other parameters .}

\IntEnr
 {OTHTYP\blank{6}}{$(N_{\rm oth})$} {Type of other parameters (Integer=1, Real=2, String=3.}
 
\CharEnr
 {OTHVAC\blank{6}}{$(N_{\rm oth})*12$} {Value of other parameters of type characater.}

\RealEnr
 {OTHVAR\blank{6}}{$(N_{\rm oth})$} { } {Value of other parameters of type real.}

 \IntEnr
 {NTOT\blank{6}}{1}
 {Total number of parameters in the \moc{SAP} (or \moc{MCO}) object. Exept \dusa{'FLUE'} and \dusa{'TIME'} }
  
\CharEnr
 {STATE\_VAR\blank{3}}{$(N_{\rm pk})*12$} {Name of state variables (\moc{PKEY}).}

\CharEnr
 {ISOTOPES\blank{4}}{$(8)*12$} {Reference name of isotopes in the saphyb object for depletion chain, recovery of fission yield and number densities. 1) Xe135 2) I135 3) Sm 149 4) Pm149 5)Pm148 6) Pm148m 7) Nd147 8)Pm147 .}
  
\RealEnr
 {\textit{/svname/}}{$N_{k}$}{ }
 {Values for each state variable specified by \moc{PKEY$_{k}$}, $N_{k}$ is the number of value taken for the k$^{th}$ PKEY. }  
 
\IntEnr
 {PKIDX\blank{7}}{$N_{\rm pk}$}
 {Correspondance of indices between SAP/MCO and GenPMAXS state variables}

\OptCharEnr 
 {ADF\_TYPE\blank{4}}{$*3$} {ladf}  {Type of ADF recovered (\moc{DRA}, \moc{GET} or \moc{SEL}).}

\OptCharEnr
 {HADF\blank{8}}{$(N_{\rm adf})*8$} {ladf} {Name of the ADF-type boundary flux edit to be recovered from Multicompo }    

\OptCharEnr
 {CDF\_TYPE\blank{4}}{$*3$} {lcdf} {Type of CDF recovered (only \moc{DRA}).}

\OptCharEnr
 {HCDF\blank{8}}{$(N_{\rm adf})*8$} {lcdf} {Name of the CDF-type boundary flux edit to be recovered from Multicompo.}    
 
 \OptCharEnr
 {GFF\_TYPE\blank{4}}{$*3$} {lgff} {Type of GFF recovered (only \moc{DRA}).}

 \OptCharEnr
 {YLD\_OPT\blank{5}}{$*3$} {lyld} {Type of fission yield to be recovered (\moc{FIX},\moc{MAN} or \moc{DEF}).}

 \OptRealEnr
 {YLD\_FIX\blank{5}}{$3$} {lyld} {} {User defined values for fission yields (1:I, 2:XE, 3:PM)}

  \OptRealEnr
 {YLD\_LOC\blank{5}}{$6$} {lyld} {} {Values for state parameter on which fission yields are calculated.}

 \end{DescriptionEnregistrement}
 
Each component of the list \moc{PKEY$\_$INFO} is a directory containing for all possible state variables ( 1=\moc{BARR}, 2=\moc{DMOD}, 3=\moc{CBOR}, 4=\moc{TCOM}, 5=\moc{TMOD}, 6=\moc{BURN}). Inside each groupwise directory, the following records will be found:

 \begin{DescriptionEnregistrement}{Records in \dir{PKEY\_INFO} }{7.0cm} \label{tabl:tabPK}
 
 \IntEnr
 {LFLAG\blank{7}}{1}
 {Indicates if the corresponding state variable can be found in the \moc{SAP} (or \moc{MCO}) object.}
  
  \OptCharEnr
 {NAME\blank{8}} {*12} {lflag} 
 {Name of the state parameter in the \moc{SAP} (or \moc{MCO}) object. For \moc{BARR} and \moc{BURN} state variables, this record exists even if \moc{LFLAG} is $\textit{false}$. }  
 
   
 \end{DescriptionEnregistrement} 
 


\begin{DescriptionEnregistrement}{Records and sub-directories
 in \dir{HELIOS\_HEAD} }{7.0cm} \label{tabl:tabhel}

\RealEnr
 {FILE\_CONT\_1\blank{1}}{$2$}{ }
 {Set of data for \moc{FILE$\_$CONT$\_$1} block in \moc{DRA} file : Heavy metal Density(\moc{HM$\_$Dens}), Bypass Density(\moc{ByPass}). } 
  
\RealEnr
 {FILE\_CONT\_2\blank{1}}{$8$}{ }
 {Set of data for \moc{FILE$\_$CONT$\_$2} block in \moc{DRA} file : Lower Energy Limits of Neutron Groups. } 
  
\RealEnr
 {FILE\_CONT\_3\blank{1}}{$7$}{ }
 {Set of data for \moc{FILE$\_$CONT$\_$3} block in \moc{DRA} file : Volume of coolant (\moc{VCool}), moderator (\moc{VModr}), control rods (\moc{VCnRd}), fuel(\moc{VFuel}), cladding (\moc{VClad}), channels (\moc{VChan}), water (\moc{VWatR}). }
 
 \RealEnr
 {FILE\_CONT\_4\blank{1}}{$3$}{ }
 {Set of data for \moc{FILE$\_$CONT$\_$4} block in \moc{DRA} file : Cell Pitch and X,Y Position of First Cell. }
 
  \RealEnr
 {XS\_CONT\blank{6}}{$1$}{ }
 {Set of data for \moc{XS$\_$CONT} block in \moc{DRA} file :  (\moc{VFCM}). }
    
 \end{DescriptionEnregistrement}
 
 \begin{DescriptionEnregistrement}{Records and sub-directories
 in \dir{GENPMAXS\_INP} }{7.0cm} \label{tabl:tabgen}
\IntEnr
 {FLAG\blank{8}}{$1$}
 {Indicates the end of a branch calculation.} 
 
\CharEnr
 {JOB\_TIT\blank{5}}{$*16$}
 {Name of final PMAXS file. } 
  
\CharEnr
 {FILE\_NAME\blank{3}}{$*12$}
 {Name of output \moc{DRA} file } 
  
\CharEnr
 {DERIVATIVE\blank{2}}{$*4$}
 {Set of data for \moc{FILE$\_$CONT$\_$3} block in \moc{DRA} file : Volume of coolant (\moc{VCool}), moderator (\moc{VModr}), control rods (\moc{VCnRd}), fuel(\moc{VFuel}), cladding (\moc{VClad}), channels (\moc{VChan}), water (\moc{VWatR}). }
 
 \RealEnr
 {VERSION\blank{5}}{$1$}{ }
 {Version of PARCS used. }
 
  \CharEnr
 {COMMENT\blank{5}}{$*40$}
 {Free comment. }
    
\CharEnr
 {JOB\_OPT\blank{5}}{$(14)*1$}
 {Options for GenPMAXS running (see \cite{GENPMAXS}). }
 
 \RealEnr
 {DAT\_SRC\blank{5}}{$5$}{ }
 {Data source information (see \cite{GENPMAXS}). }
  
 \end{DescriptionEnregistrement}
 
Each component of the list \moc{TH$\_$DATA} is a directory containing TH data for a single burnup point. Inside each groupwise directory, the following records associated will be found:
 
 \begin{DescriptionEnregistrement}{Records 
 in sub-directory \dir{TH\_DATA} }{7.0cm} \label{tabl:tabTH}
 
 \OptRealEnr
 {CHI\blank{9}}{$N_{\rm gr}$} {$lchi$} { }
 {The steady-state energy spectrum of the neutron emitted by fission $\chi$. }
 
  \OptRealEnr
 {OVERV\blank{7}}{$N_{\rm gr}$}{$linv$} {s.cm$^{-1}$}
 {The average of the inverse neutron velocity. }
 
   \OptRealEnr
 {LAMBDA\blank{6}}{$N_{\rm del}$}{$lamb$} {s$^{-1}$}
 {Decay constant of delayed neutron. }
 
   \OptRealEnr
 {BETA\blank{8}}{$N_{\rm del}$}{$lbet$} {}
 {Effective delayed neutron fraction. }
  
   \OptRealEnr
 {YLDPm\blank{7}}{$1$}{$lyld$} { }
 { Effective yield value of Pm-149 }
 
    \OptRealEnr
 {YLDXe\blank{7}}{$1$}{$lyld$} { }
 { Effective yield value of Xe-135 }
 
     \OptRealEnr
 {YLDI\blank{8}}{$1$}{$lyld$} { }
 { Effective yield value of I-135 }

  
 \end{DescriptionEnregistrement} 
 
 
 \begin{DescriptionEnregistrement}{Records and sub-directories
 in sub-directory \dir{BRANCH\_INFO} }{7.0cm} \label{tabl:tabBR}
 
 \IntEnr
 {BRANCH\_IT\blank{3}}{$1$}
 {Index of the current branch type calculation.}
 
 \CharEnr
 {BRANCH\blank{6}}{$*4$}
 {Current BRANCH type calculation. } 
  
 \IntEnr
 {BRANCH\_NB\blank{3}}{$1$}
 {Number of branches to be created in PMAXS files.}  
 
 \RealEnr
 {REF\_STATE\blank{3}}{$N_{\rm pk}-1$}{ }
 {Values of state variables for reference state, burnup excepted. }
 
 \RealEnr
 {HST\_STATE\blank{3}}{$N_{\rm pk}-1$}{ }
 {Values of state variables for history state, burnup excepted. }
 
 \IntEnr
 {STATE\_INDEX\blank{1}}{$N_{\rm pk}$}
 {Current index value for each  state variables.}
 
 \IntEnr
 {BRANCH\_INDEX}{$N_{\rm pk}$}
 {Index of current  branch calculation.}
 
 \RealEnr 
 {STATE\blank{7}}{$N_{\rm pk}$}{ }
 {State variable values for the current branch  calculation. } 
 
 \IntEnr
 {REWIND\blank{6}}{$1$}
 {Indicates the end of a branch calculation (\moc{REWIND}=1).}
 
 \IntEnr
 {STOP\blank{8}}{$1$}
 {Indicates the end of calculation (\moc{STOP}=1).}
  
  
  \DirlEnr
 {CROSS\_SECT\blank{2}}{$N_{\rm bu}$}
 {Each component of the list is a directory containing cross sections for all  burnup point of a specific branch(cross sections are overwrited after each branch calcualtion).}  
 
  \OptDirlEnr
 {DIVERS\blank{6}}{$N_{\rm bu}$} {$I_{grid} \leq 1$}
 {Each component of the list is a directory containing the \moc{DIVERS} data block recovered from the saphyb object(inforamtions are overwrited after each branch calcualtion).}  
   
 \end{DescriptionEnregistrement} 
Each component of the list \moc{CROSS$\_$SECT} is a directory containing cross sections for all  burnup point of a specific branch. Cross sections for a single burnup points are filled seqentially during the branching calculation.  Inside each groupwise directory, the following sub-directories will be found:

 \begin{DescriptionEnregistrement}{Sub-directories
 in \dir{CROSS\_SECT} }{7.0cm} \label{tabl:tabXS}
  
  \DirEnr
 {MACROLIB\_XS\blank{1}}
 {Directory for macroscopic cross sections for a given calculated branch  (cross sections are overwrited after each branch calculation).}  
 
  \OptDirlEnr
 {MICROLIB\_XS\blank{1}}{} {$lxes$} 
 {Directory for microscopic cross sections for a given calculated branch (cross sections are overwrited after each branch calculation).}  
   
 \end{DescriptionEnregistrement} 
 
 
 \begin{DescriptionEnregistrement}{Records 
 in the sub-directory \dir{MACROLIB\_XS} }{7.0cm} \label{tabl:tabMA}
 
 \RealEnr
 {XTR\blank{9}}{$N_{\rm gr}$}{cm$^{-1}$}
 {The transport cross section $\Sigma^g_f$:
 $\Sigma^g_t = 1 / ( 3 \cdot D_{g} ) $. }
 
 \RealEnr
 {ABSORPTION\blank{2}}{$N_{\rm gr}$}{cm$^{-1}$}
 {The absorption cross section $\Sigma^g_a$: $\Sigma^g_a = \Sigma^g_{tot} - \sum_{g'} \Sigma^{g \rightarrow g'}_s $. }
 
 \RealEnr
 {KAPPA\_FI\blank{4}}{$N_{\rm gr}$}{cm$^{-1}$}
 {The product of $\Sigma^g_f$, the fission cross section with  the energy emitted by this reaction $\kappa$ :  $\kappa\Sigma^g_f$. }

 \RealEnr
 {X\_NU\_FI\blank{5}}{$N_{\rm gr}$}{cm$^{-1}$}
 {The product of $\Sigma^g_f$, the fission cross section with  the averaged number of fission–emitted delayed
neutron $\nu$ :  $\nu\Sigma^g_f$. }
 
 \RealEnr
 {SCAT\blank{8}}{$N_{\rm gr}*N_{\rm gr}$}{cm$^{-1}$}
 {Scattering cross section elements  from group g to g' : $\Sigma^{g \rightarrow g'}_s$. }
 
 \OptRealEnr
 {DET\blank{9}}{$N_{\rm gr}$} {$ldet$} { } 
 {The detector response parameter, it is product of cross section and local flux ratio. }
 
 \OptRealEnr
 {SFI\blank{9}}{$N_{\rm gr}$} {$lxes$} {cm$^{-1}$} 
 {The fission cross section : $\Sigma^g_f$. } 
 
 \OptRealEnr
 {ADF\blank{9}}{$N_{\rm adf},N_{\rm gr}$} {$ladf$} { }  
 { Assembly Discontinuity Factors. } 
 
 \OptRealEnr
 {CDF\blank{9}}{$N_{\rm cdf},N_{\rm gr}$} {$lcdf$} { }  
 { Corner Discontinuity Factors. } 
 
  \OptRealEnr
 {GFF\blank{9}}{$N_{\rm pin} * N_{\rm pin },N_{\rm gr}$} {$lgff$} { }  
 { Group-wise Form Function. } 
  
 \end{DescriptionEnregistrement} 
 
  \begin{DescriptionEnregistrement}{Records 
 in the sub-directory \dir{MICROLIB\_XS} }{7.0cm} \label{tabl:tabMI}
 
 \OptRealEnr
 {XENG\blank{8}}{$N_{\rm gr}$} {} { cm$^{-2}$} 
 {The microscopic absorption cross section of Xenon: $\sigma^g_{a,Xe} = \sigma^g_{tot,Xe} - \sum_{g'} \Sigma^{g \rightarrow g'}_{s, Xe} $. }
 
 \OptRealEnr
 {SMNG\blank{8}}{$N_{\rm gr}$} {} {cm$^{-2}$ } 
 {The microscopic absorption cross section of Samarium: $\sigma^g_{a,Sm} = \sigma^g_{tot,Sm} - \sum_{g'} \Sigma^{g \rightarrow g'}_{s, Sm} $.}
 
  \OptRealEnr
 {XEND\blank{8}}{$N_{\rm gr}$} {} {(b.cm)$^{-1}$} 
 {The Xenon number density: $n_{Xe}$.}
 
  \OptRealEnr
 {SMND\blank{8}}{$N_{\rm gr}$} {} {(b.cm)$^{-1}$} 
 {The Samarium number density: $n_{Sm}$.}
  
 \end{DescriptionEnregistrement} 

Each component of the list \moc{DIVERS} is a directory containing informations for all  burnup points of a specific branch recovered from the \moc{DIVERS} block of the saphyb object. Inside each groupwise directory, the following records will be found:

  \begin{DescriptionEnregistrement}{Records 
 in the sub-directory \dir{DIVERS} }{7.0cm} \label{tabl:tabDI}
 
 \RealEnr
 {B2\blank{10}}{$1$} { cm$^{-2}$} 
 {Critical buckling. }
 
 \RealEnr
 {KEFF\blank{8}}{$1$} { } 
 {Effective multiplication factor. }
 
 \RealEnr
 {KINF\blank{8}}{$1$} { } 
 {Infinite multiplication factor. }
 
 \end{DescriptionEnregistrement} 
\clearpage
