\subsection{The \moc{PKINI:} module}\label{sect:pkini}

\vskip 0.2cm
The \moc{PKINI:} module is used to initialize the point kinetics parameters, to define the delayed neutron information
and to set the global feedback parameters. The point kinetics equations are solved for a {\sl time stage} using module
\moc{PKINS:} (See Sect.~\ref{sect:pkins}) as a function of a fixed set of global parameters recovered from the \dds{map} object \dusa{MAPFL}.
Modules \moc{PKINI:} and \moc{PKINS:} are intended to be used with thermal-hydraulics module \moc{THM:} (See Sect.~\ref{sect:thm})
to simulate a single reactor channel.

\vskip 0.08cm

\noindent
The \moc{PKINI:} module specification is:

\begin{DataStructure}{Structure \moc{PKINI:}}
\dusa{MAPFL} \moc{:=} \moc{PKINI:} \dusa{MAPFL} \moc{::} \dstr{descpkini}
\end{DataStructure}

\noindent where

\begin{ListeDeDescription}{mmmmmmmm}

\item[\dusa{MAPFL}] \texttt{character*12} name of the \dds{map} 
object containing fuel regions description and global parameter informations.

\item[\dstr{descpkini}] structure describing the input data to the \moc{PKINI:} module. 

\end{ListeDeDescription}

\vskip 0.2cm

\subsubsection{Input data to the \moc{PKINI:} module}\label{sect:pkinistr}

\begin{DataStructure}{Structure \dstr{descpkini}}
$[$ \moc{EDIT} \dusa{iprint} $]$ \\
\moc{POWER} \dusa{power} \\
$[$ \moc{EPSILON} \dusa{epsilon} $]$ \\
\moc{TIME} \dusa{t0} \dusa{dt} \\
\moc{LAMBDA} \dusa{lambda} \\
\moc{NGROUP} \dusa{ngroup} \\
\moc{BETAI} (\dusa{beta}(i),i=1,\dusa{ngroup}) \moc{LAMBDAI} (\dusa{dlambda}(i),i=1,\dusa{ngroup}) \\
\moc{ALPHA} $[[$ \dusa{PNAME} $\{$ \moc{DIRECT} $|$ \moc{DERIV} $|$ \moc{SQDERIV} $\}$ \\
~~~~$[~\{$ \moc{LINEAR} $|$ \moc{CUBIC} $\}~]$ \dusa{nalpha} (\dusa{x}(i) \dusa{y}(i),i=1,\dusa{nalpha}) $]]$ \moc{ENDA} \\
$[$ \moc{PTIME} $[[$ \dusa{PNAME} $[~\{$ \moc{LINEAR} $|$ \moc{CUBIC} $\}~]~\{$\dusa{ntime} (\dusa{t}(i) \dusa{x}(i),i=1,\dusa{ntime}) $|$ \\
~~~~\moc{T-DELT} $[$ \dusa{nxy} $]$ \dusa{t1} \dusa{t2} \moc{P-VALV} \dusa{gamma} \dusa{p1} \dusa{p2} \dusa{tb1}
\dusa{bval1} $[$ \moc{RESET} \dusa{p3} \dusa{tb2} \dusa{bval2} $]~\}~]]$ \moc{ENDP} $]$ \\
;
\end{DataStructure}

\noindent where
\begin{ListeDeDescription}{mmmmmmmm}

\item[\moc{EDIT}] keyword used to set \dusa{iprint}.

\item[\dusa{iprint}] integer index used to control the printing on screen:
= 0 for no print; = 1 for minimum printing; larger values produce
increasing amounts of output.

\item[\moc{POWER}] keyword used to set the initial reactor power.

\item[\dusa{power}] reactor power in MW.

\item[\moc{EPSILON}] keyword used to set the epsilon of the Runge-Kutta solver used to adjust the internal time step. The default value is $\epsilon=1.0\times 10^{-2}$.

\item[\dusa{epsilon}] user-selected epsilon.

\item[\moc{TIME}] keyword used to set the initial time and stage duration. It is possible to readjust the stage duration in module \moc{PKINS:}.

\item[\dusa{t0}] initial time (s).

\item[\dusa{dt}] stage duration (s).

\item[\moc{LAMBDA}] keyword used to set the prompt neutron generation time.

\item[\dusa{lambda}] prompt neutron generation time (s).

\item[\moc{NGROUP}] keyword used to set the number of delayed precursor groups.

\item[\dusa{ngroup}] number of delayed precursor groups.

\item[\moc{BETAI}] keyword used to set the delayed neutron fraction vector.

\item[\dusa{betai}] value of the delayed neutron fraction in a delayed precursor group.

\item[\moc{LAMBDAI}] keyword used to set the delayed neutron time constant vector.

\item[\dusa{lambdai}] value of the delayed neutron time constant (s) in a delayed precursor group.

\item[\moc{ALPHA}] keyword used to set the feedback parameters.

\item[\moc{PNAME}] character*12 name of a feedback parameter. A feedback parameter should be a global parameter defined in the fuelmap.

\item[\moc{DIRECT}] keyword indicating that the reactivity is a direct function of the global parameter.

\item[\moc{DERIV}] keyword indicating that the reactivity is a function of the variation of the local parameter with respect to its initial value.

\item[\moc{SQDERIV}] keyword indicating that the reactivity is a function of the variation of the square root of the local parameter with respect to its initial value.
This option is generally used to represent the Doppler effect due to the fuel temperature.

\item[\moc{LINEAR}] keyword indicating that interpolation of the reactivity effect uses linear Lagrange polynomials (default option).

\item[\moc{CUBIC}] keyword indicating that interpolation of the reactivity effect uses the Ceschino method
with cubic Hermite polynomials, as presented in Ref.~\citen{Intech2011}.

\item[\dusa{nalpha}] number of values in the table describing reactivity effects for feedback parameter \moc{PNAME}.

\item[\dusa{x}] value of the global parameter.

\item[\dusa{y}] corresponding reactivity coefficient.

\item[\moc{PTIME}] keyword used to set the time laws for some feedback parameters.

\item[\dusa{ntime}] number of values in the time law for feedback parameter \moc{PNAME}.

\item[\dusa{t}] time (s)

\item[\dusa{x}] corresponding value of the global parameter.

\item[\moc{T-DELT}] keyword used to set an analytical time law between two times.

\item[\dusa{nxy}] number of points used to construct the discrete time law. The default value is \dusa{nxy} $=1001$.

\item[\dusa{t1}] initial time (s) for the analytical time law domain.

\item[\dusa{t2}] final time (s) for the analytical time law domain.

\item[\moc{P-VALV}] keyword used to define a time law corresponding to the depressurization of a gas reservoir. The time-vatiation of the pressure $P(t)$ in a
reservoir is given by a relation of the form
\begin{equation}
P(t)=\begin{cases} P_{\rm max} & {\rm if} \ t \le {\rm \dusa{tb1}}\\
\max{\left(P_{\rm min},{\displaystyle P_{\rm max}\over\displaystyle (1+Bt)^\alpha}\right)} & {\rm otherwise}\end{cases}
\label{eq:pkin1}
\end{equation}

\noindent where $P_{\rm max}$ is the pressure of the reservoir at time $t\le$ \dusa{tb1}, before depressurization and $P_{\rm min}$ is the final pressure after
depressurization. $B$ (s$^{-1}$) is the time constant of the exhaust pipe and $\alpha$ is given by relation
\begin{equation}
\alpha={2\gamma\over \gamma-1}
\label{eq:pkin2}
\end{equation}

\noindent where $\gamma$ is a constant related to the gas capacitance ($=1.66$ for Helium).

\item[\dusa{gamma}] value of $\gamma$ parameter in Eq.~(\ref{eq:pkin2}).

\item[\dusa{p1}] value of pressure $P_{\rm max}$ in Eq.~(\ref{eq:pkin1}).

\item[\dusa{p2}] value of pressure $P_{\rm min}$ in Eq.~(\ref{eq:pkin1}).

\item[\dusa{tb1}] time (s) when the exhaust pipe is open with value \dusa{bval1}. We must have \dusa{t1}$\le$\dusa{tb1}$<$\dusa{t2}.

\item[\dusa{bval1}] value of $B$ parameter in Eq.~(\ref{eq:pkin1}) after time \dusa{tb1}.

\item[\moc{RESET}] optional keyword used to reset the opening of the exhaust pipe after a fixed period of time.

\item[\dusa{p3}] value of pressure $P_{\rm min}$ in Eq.~(\ref{eq:pkin1}) after reset.

\item[\dusa{tb2}] time (s) when the exhaust pipe is open with value \dusa{bval2}. We must have \dusa{tb1}$<$\dusa{tb2}$<$\dusa{t2}.

\item[\dusa{bval2}] value of $B$ parameter in Eq.~(\ref{eq:pkin1}) after time \dusa{tb2}.

\end{ListeDeDescription}
\clearpage
