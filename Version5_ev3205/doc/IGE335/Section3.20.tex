\subsection{The {\tt CHAB:} module}\label{sect:CHABData}

This component of the lattice code is dedicated to the modification of cross section
information in a {\sc microlib}.

\vskip 0.02cm

The calling specifications are:

\begin{DataStructure}{Structure \dstr{CHAB:}}
$\{$~\dusa{MICRO1}~$|$~\dusa{DRAGLIB1}~$\}$~\moc{:=}~\moc{CHAB:}~$\{$~\dusa{MICRO1}~$|$~\dusa{MICRO2}~$|$~\dusa{DRAGLIB2}~$\}$~\moc{::}~\dstr{CHAB\_data} \\
\end{DataStructure}

\noindent where
\begin{ListeDeDescription}{mmmmmmm}

\item[\dusa{MICRO1}] {\tt character*12} name of a {\sc microlib} (type {\tt L\_LIBRARY}) object that is created or modified by {\tt CHAB:}.

\item[\dusa{DRAGLIB1}] {\tt character*12} name of a {\sc draglib} (type {\tt L\_DRAGLIB}) object that is created by {\tt CHAB:}.

\item[\dusa{MICRO2}] {\tt character*12} name of a {\sc microlib} (type {\tt L\_LIBRARY}) object open in read-only mode.

\item[\dusa{DRAGLIB2}] {\tt character*12} name of a {\sc draglib} (type {\tt L\_DRAGLIB}) object open in read-only mode.

\item[\dusa{CHAB\_data}] input data structure containing specific data (see \Sect{descCHAB}).

\end{ListeDeDescription}

\subsubsection{Data input for module {\tt CHAB:}}\label{sect:descCHAB}

\vskip -0.5cm

\begin{DataStructure}{Structure \dstr{CHAB\_data}}
$[$~\moc{EDIT} \dusa{iprint}~$]$ \\
$[[$~\moc{MODI} \dusa{TYPSEC} \dusa{igm} \moc{TO} \dusa{igp} $\{$
\moc{VALE}~$[[$~\dusa{val}~$]]~|$~\moc{CONS}~\dusa{value}~$|$~\moc{PLUS}~\dusa{value}~$|$~\moc{MULT}~\dusa{value} $\}$ \dusa{HISOT}~$]]$
{\tt ;}
\end{DataStructure}

\noindent where
\begin{ListeDeDescription}{mmmmmmmm}

\item[\moc{EDIT}] keyword used to set \dusa{iprint}.

\item[\dusa{iprint}] index used to control the printing in module {\tt CHAB:}. =0 for no print; =1 for minimum printing (default value).

\item[\moc{MODI}] keyword used to define a modification of a nuclear reaction belonging to a given isotope.

\item[\dusa{TYPSEC}] {\tt character*8} name of an existing nuclear reaction chosen among the following values:
\begin{description}
\item[{\tt 'NTOT0'}] Total cross section.
\item[{\tt 'NG'}] Radiative capture cross section. The total ({\tt 'NTOT0'}) cross section is modified accordingly.
\item[{\tt 'NA'}] $(n,\alpha)$ cross section. The total ({\tt 'NTOT0'}) cross section is modified accordingly.
\item[{\tt 'NP'}] $(n,p)$ cross section. The total ({\tt 'NTOT0'}) cross section is modified accordingly.
\item[{\tt 'ND'}] $(n,d)$ cross section. The total ({\tt 'NTOT0'}) cross section is modified accordingly.
\item[{\tt 'NT'}] $(n,t)$ cross section. The total ({\tt 'NTOT0'}) cross section is modified accordingly.
\item[{\tt 'CAPT'}] Capture cross sections. Each present reaction of capture (\textbf{NG}, \textbf{NA}, \textbf{NP}, \textbf{ND}, \textbf{NT}) are taken into account. The total ({\tt 'NTOT0'}) cross section is modified accordingly. Only the keyword \textbf{MULT}, indicating a multiplication of the all cross sections, is available.
\item[{\tt 'NELAS'}] Elastic scattering cross section. The scattering ({\tt 'SIGS00'} and {\tt 'SCAT00'}) and total
({\tt 'NTOT0'}) cross sections are modified accordingly.
\item[{\tt 'NINEL'}] Inelastic scattering cross section. The scattering ({\tt 'SIGS00'} and {\tt 'SCAT00'}) and total
({\tt 'NTOT0'}) cross sections are modified accordingly.
\item[{\tt 'N2N'}] ($n$,$2n$) cross section. The scattering ({\tt 'SIGS00'} and {\tt 'SCAT00'}) and total
({\tt 'NTOT0'}) cross sections are modified accordingly.
\item[{\tt 'N3N'}] ($n$,$3n$) cross section. The scattering ({\tt 'SIGS00'} and {\tt 'SCAT00'}) and total
({\tt 'NTOT0'}) cross sections are modified accordingly.
\item[{\tt 'N4N'}] ($n$,$4n$) cross section. The scattering ({\tt 'SIGS00'} and {\tt 'SCAT00'}) and total
({\tt 'NTOT0'}) cross sections are modified accordingly.
\item[{\tt 'SIGS00'}, {\tt 'SIGS01'}, etc.] Scattering cross section. The total ({\tt 'NTOT0'}) cross section is modified accordingly.
\item[{\tt 'SCAT00'}, {\tt 'SCAT01'}, etc.] Differential scattering cross section. The total ({\tt 'NTOT0'}) cross section is modified accordingly.
\item[{\tt 'NUSIGF'}] $\nu$ times the fission cross section. The fission ({\tt 'NFTOT'}) and total ({\tt 'NTOT0'}) cross sections are modified accordingly.
\item[{\tt 'NFTOT'}] Fission cross section. The $\nu$ times fission ({\tt 'NUSIGF'}) and total ({\tt 'NTOT0'}) cross sections are modified accordingly.
\item[{\tt 'NU'}] Number of neutrons emitted per fission.The $\nu$ times fission ({\tt 'NUSIGF'}) cross section is modified accordingly.
\item[{\tt 'CHI'}] Fission spectrum. The resulting spectrum is normalized.
\end{description}

\item[\dusa{igm}] lower energy group index of the energy domain where the modification is taking place.

\item[\dusa{igp}] upper energy group index of the energy domain where the modification is taking place.

\item[\moc{VALE}] keyword indicating a replacement of all values in the above energy domain by different values.

\item[\dusa{val}] group--dependent real variable used as replacement value. We expect \dusa{igp}$-$\dusa{igm}$+$1 components.

\item[\moc{CONS}] keyword indicating a replacement of all values in the above energy domain by \dusa{value}.

\item[\moc{PLUS}] keyword indicating that \dusa{value} is added to all values in the above energy domain.

\item[\moc{MULT}] keyword indicating a multiplication of all values in the above energy domain by \dusa{value}.

\item[\dusa{value}] real variable used to modify the nuclear reaction.

\item[\dusa{HISOT}] {\tt character*8} or {\tt character*12} name of the isotope to modify. If \dusa{HISOT} is a {\tt character*8} value,
all {\tt character*12} isotope names prefixed by \dusa{HISOT} are modified.

\end{ListeDeDescription}

\eject
