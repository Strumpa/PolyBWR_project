\section{Contents of a \dir{edition} directory}\label{sect:editiondir}

This directory contains the main editing results. For the purpose of illustration
we will assume that the \moc{EDI:} module is executed using the following data:

\vskip -0.1cm
\begin{verbatim}
EDITING := EDI: FLUX LIBRARY VOLMAT ::
  MERG COMP COND 27 69 ALL SAVE ON EDITCELL2G ;
\end{verbatim}
\vskip -0.1cm

\noindent
where \moc{EDITING} is the final \dds{edition} data structure. The data structures \moc{FLUX},
\moc{LIBRARY} and \moc{VOLMAT} are respectively of type \dds{fluxunk}, \dds{microlib} and
\dds{tracking}. Assuming that the initial number of regions in \moc{VOLMAT} is $N$ and the number of
groups in \moc{LIBRARY} is $G=69$, then the final information that will be stored in
the \dds{editing} data structure will represent a two group ($G_{c}=2$) one mixture
$N_{h}$ \dir{microlib}. 

\subsection{State vector content for the \dir{edition} data structure}\label{sect:editionstate}

The dimensioning parameters for this data structure, which are stored in the state vector
$\mathcal{S}^{\rm edi}_{i}$, represent:

\begin{itemize}
\item The number of homogeneous mixtures saved $N_{H}=\mathcal{S}^{\rm edi}_{1}$ for the last editing step
\item The number of condensed groups considered $M_{G}=\mathcal{S}^{\rm edi}_{2}$  for the last editing step
\item Editing flag to indicate the presence of 4 factor editing $I_{4f}=\mathcal{S}^{\rm edi}_{3}$ for
the last editing step
\item Editing flag to indicate that the up-scattering contributions have all been transferred to
      the diagonal part of the scattering matrix $I_{U}=\mathcal{S}^{\rm edi}_{4}$ for the last editing step
\item The number of mixture activated $N_{A}=\mathcal{S}^{\rm edi}_{5}$ for the last editing step 
\item Editing flag to indicate the types of statistics generated by \moc{EDI:}
      $I_{S}=\mathcal{S}^{\rm edi}_{6}$ for the last editing step
\item Editing flag to indicate which boundary flux editions are used in \moc{EDI:}. These editions are required for computing
assembly discontinuity factors (ADF) or to perform some types of {\sl superhomog\'en\'eisation} (SPH) calculations.
      $I_{\rm adf}=\mathcal{S}^{\rm edi}_{7}$ for the last editing step
\begin{displaymath}
I_{\rm adf} = \left\{
\begin{array}{ll}
0 & \textrm{no boundary flux editions;} \\
1 & \textrm{use boundary currents obtained using the \moc{ALBS} keyword in DRAGON;} \\
2 & \textrm{recover boundary fluxes from informations located in the {\tt REF:ADF} directory;} \\
-2 & \textrm{compute assembly discontinuity factors (ADF) from informations located in} \\
  & \textrm{the {\tt REF:ADF} directory;} \\
3 & \textrm{use boundary currents obtained from the current iteration method in Eurydice;} \\
4 & \textrm{recover boundary fluxes or discontinuity factors from the {\tt MACROLIB/ADF}} \\
  & \textrm{directory.} \\
\end{array} \right.
\end{displaymath}
\item Editing flag to indicate the type of tracking to be performed on a macro-geometry built by module {\tt EDI:}.
      $I_{\rm cell}=\mathcal{S}^{\rm edi}_{8}$ for the last editing step
\begin{displaymath}
I_{\rm cell} = \left\{
\begin{array}{ll}
1 & \textrm{the macro-geometry is tracked by module {\tt SYBILT:} or {\tt EXCELT:};} \\
2 & \textrm{the macro-geometry is tracked by module {\tt NXT:};} \\
3 & \textrm{the macro-geometry is tracked by another module.} \\
\end{array} \right.
\end{displaymath}
\item The number of extracted isotopes in the output microlib $I_{m}=\mathcal{S}^{\rm edi}_{9}$ for the last editing step
\item The print level considered $I_{p}=\mathcal{S}^{\rm edi}_{10}$ for the last editing step
\item Editing flag to indicate the types of cross section saved in \moc{EDI:}
      $I_{x}=\mathcal{S}^{\rm edi}_{11}$ for the last editing step
\item The type of weighting used for $P_1$ cross section information $I_{\rm w}=\mathcal{S}^{\rm edi}_{12}$ for the
last editing step ($=0$: flux weighting; $=1$ current weighting)
\item The maximum number of isotopes per mixture $M_{I}=\mathcal{S}^{\rm edi}_{13}$ 
\item The maximum number of condensed groups in all editing $M_{g}=\mathcal{S}^{\rm edi}_{14}$ 
\item The maximum number of homogeneous mixtures in all editing $M_{h}=\mathcal{S}^{\rm edi}_{15}$ 
\item The total number of ISOTXS files generated $M_{F}=\mathcal{S}^{\rm edi}_{16}$ 
\item The maximum number of regions before homogenization $M_{\rm max}=\mathcal{S}^{\rm edi}_{17}$ 
\item Editing flag $=1$ for H-factor edition; $=0$ otherwise $I_{H-fac}=\mathcal{S}^{\rm edi}_{18}$ 
\item Number of delayed neutron precursor groups $N_{\rm del}=\mathcal{S}^{\rm edi}_{19}$
\item Geometry index $L_{\rm geo}=\mathcal{S}^{\rm edi}_{20}$
\begin{displaymath}
L_{\rm geo} = \left\{
\begin{array}{ll}
0 & \textrm{the macro geometry is not available}\\
1 & \textrm{the macro-geometry of the last editing is available}\\
\end{array} \right.
\end{displaymath}
\item Type of weighting for homogenization or/and condensation of cross-section information $I_{\rm adj}=\mathcal{S}^{\rm edi}_{21}$
\begin{displaymath}
I_{\rm adj} = \left\{
\begin{array}{ll}
0 & \textrm{use direct flux;} \\
1 & \textrm{use adjoint flux.} \\
\end{array} \right.
\end{displaymath}
\item Type of current used for $P_1$ weighting if $I_{\rm w}\ne 0$. $I_{\rm curr}=\mathcal{S}^{\rm edi}_{22}$
\begin{displaymath}
I_{\rm curr} = \left\{
\begin{array}{ll}
1 & \textrm{use a current obtained from an heterogeneous leakage model;} \\
2 & \textrm{use the Todorova flux;} \\
4 & \textrm{use spherical harmonics weighting.} \\
\end{array} \right.
\end{displaymath}
\item Number of reactions saved on output microlib $N_{\rm reac}=\mathcal{S}^{\rm edi}_{23}$
\begin{displaymath}
N_{\rm reac} = \left\{
\begin{array}{ll}
0 & \textrm{all available reactions are saved;} \\
>0 & \textrm{only reactions listed in {\tt REF:HVOUT} array are saved.} \\
\end{array} \right.
\end{displaymath}
\item Edition flag for the integrated net currents along each axis $I_{\rm intcur}=\mathcal{S}^{\rm edi}_{24}$
\begin{displaymath}
I_{\rm intcur} = \left\{
\begin{array}{ll}
0 & \textrm{not set;} \\
1 & \textrm{integrated current edition.} \\
\end{array} \right.
\end{displaymath}
\item Type of condensation of the diffusion coefficients $I_{\rm golver}=\mathcal{S}^{\rm edi}_{25}$
\begin{displaymath}
I_{\rm golver} = \left\{
\begin{array}{ll}
0 & \textrm{use weighting of leakage coefficients;} \\
1 & \textrm{use weighting of transport cross sections with the Golfier-Vergain correction} \\
 & \textrm{factors.} \\
\end{array} \right.
\end{displaymath}

\end{itemize}

\subsection{The main \dir{edition} directory}\label{sect:editiondirmain}

On its first level, the
following records and sub-directories will be found in the \dir{edition} directory:

\begin{DescriptionEnregistrement}{Main records and sub-directories in \dir{edition}}{7.0cm}
\CharEnr
  {SIGNATURE\blank{3}}{$*12$}
  {Signature of the data structure ($\mathsf{SIGNA}=${\tt L\_EDIT\blank{6}}).}
\IntEnr
  {STATE-VECTOR}{$40$}
  {Vector describing the various parameters associated with this data structure $\mathcal{S}^{\rm edi}_{i}$,
  as defined in \Sect{editionstate}.}
\CharEnr
  {TITLE\blank{7}}{$*72$}
  {Title of the last editing performed ($\mathsf{TITLE}$) }
\OptCharEnr
  {LAST-EDIT\blank{3}}{$*12$}{$|\mathcal{S}^{\rm edi}_{11}|\ge 2$}
  {Name of the last editing sub-directory saved ($\mathsf{LAST}$)}
\CharEnr
  {LINK.GEOM\blank{3}}{$*12$}
  {Name of the {\sc geometry} on which the last edition was based.}
\IntEnr
  {REF:IMERGE\blank{2}}{$\mathcal{S}^{\rm edi}_{17}$}
  {Merged region number associated with each of the original region number $M_{r}$}
\RealEnr
  {REF:VOLUME\blank{2}}{$\mathcal{S}^{\rm edi}_{17}$}{cm$^{3}$}
  {Volume associated with each of the original region number $V_{r}$}
\IntEnr
  {REF:MATCOD\blank{2}}{$\mathcal{S}^{\rm edi}_{17}$}
  {Mixture number associated with each of the original region number $M_{\rm mix}$}
\IntEnr
  {REF:IGCOND\blank{2}}{$\mathcal{S}^{\rm edi}_{2}$}
  {Reference group limits associated with the merged groups $C_{g}$}
\OptDirEnr
  {REF:ADF\blank{5}}{$\mathcal{S}^{\rm edi}_{7} = 2$}
  {ADF--related input data as presented in \Sect{editionADF}.}
\OptCharEnr
  {REF:HVOUT\blank{3}}{$(\mathcal{S}^{\rm edi}_{23})*8$}{$\mathcal{S}^{\rm edi}_{23} > 0$}
  {Names of the reactions saved in the output microlib.}
\OptCharEnr
  {CARISO\blank{6}}{$(\mathcal{S}^{\rm edi}_{9})*12$}{$\mathcal{S}^{\rm edi}_{9}\ge 1$}
  {Name of extracted isotopes saved during the last editing ($\mathsf{NAMI}$)}
\OptIntEnr
  {IACTI\blank{7}}{$\mathcal{S}^{\rm edi}_{5}$}{$\mathcal{S}^{\rm edi}_{5}\ge 1$}
  {Original mixture numbers for which activation data was generated ($A_{m}$)}
\OptDirEnr
  {MACRO-GEOM\blank{2}}{$\mathcal{S}^{\rm edi}_{20} = 1$}
  {Macro--{\sl geometry} directory. This geometry may be used to complete the {\sc compo} database, for performing
  a geometry equivalence ({\sl equigeom}) and/or as the macro--geometry in SPH calculations. This directory follows
  the specification presented in \Sect{geometrydirmain}.}
\OptCharEnr
  {LINK.MACGEOM}{$*12$}{$\mathcal{S}^{\rm edi}_{20} = 1$}
  {Name of the macro--{\sl geometry} on which the last edition was based.}
\DirVar
  {\listedir{micdir}}
  {Set of sub-directories containing the editing information. This directory follows
  the specification presented in \Sect{microlibdirmain}.}
\end{DescriptionEnregistrement}

The set of directory \listedir{micdir} names $\mathsf{EDIDIR}$ will be composed according to the
following rules. In the case where the set of keywords \moc{SAVE}
\moc{ON} are used followed by a directory name as above, the contents of
$\mathsf{EDIDIR}$ will be identical the name of the specified directory
(e.~g., \verb*|EDITCELL2G  |). If the \moc{SAVE} option is used without specifying a specific
directory, then the first eight characters of $\mathsf{EDIDIR}$ ($\mathsf{EDIDIR}$\verb*|(1:8)|)
will be given as
\verb*|REF-CASE| while the last four character ($\mathsf{EDIDIR}$\verb*|(9:12)|) will be a
unique character variable representing the successive directory saved. This character variable will
be created as follows:
\vskip -0.1cm
$$
\mathtt{WRITE(}\mathsf{EDIDIR}\mathtt{(9:12),'(I4.4)')}\: J
$$
\vskip -0.1cm
where $1\leq J $ represents the $J^{\textrm{th}}$ execution of the \moc{EDI:} module. In the case
above, we would have a single editing directory of the form: 

\begin{DescriptionEnregistrement}{Example of an editing directory}{8.0cm}
\DirEnr
  {EDITCELL2G\blank{2}}{Two groups \dir{microlib} sub-directory}
\end{DescriptionEnregistrement}

\goodbreak

\subsection{The \moc{/REF:ADF/} sub-directory in \dir{edition}}\label{sect:editionADF}

Sub-directory containing input data for ADF-type boundary flux edition.

\begin{DescriptionEnregistrement}{Records in the \moc{/REF:ADF/} sub-directory}{7.5cm}
\IntEnr
  {NTYPE\blank{7}}{$1$}
  {Number of ADF-type boundary flux edits.}
\IntEnr
  {NADF\blank{8}}{\tt NTYPE}
  {$N^{\rm adf}_i$: number of regions included in each ADF-type boundary flux edit.}
\CharEnr
  {HADF\blank{8}}{({\tt NTYPE})$*8$}
  {Name of each ADF-type boundary flux edit. Standard names are: $=$ \moc{FD\_C}:
corner flux edition; $=$ \moc{FD\_B}: surface (assembly gap) flux edition; $=$ \moc{FD\_H}:
row flux edition. These are the first row of surrounding cells in the assembly.}
\IntVar
  {\listedir{type}}{$N^{\rm adf}_i$}
  {Set of integer arrays containing the editing information. Indices of the regions of the reference geometry belonging to the
  flux edition. Name {\sl type} is a component of {\tt HADF} array.}
\end{DescriptionEnregistrement}

\clearpage
