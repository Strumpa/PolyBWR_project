\subsection{The {\tt SPH:} module}\label{sect:SPHData}

The {\sl superhomog\'en\'eisation} (SPH) equivalence technique is based on the calculation of a set of {\sl equivalence factors}
$\{\mu_{m,k}, m \in C_m \ {\rm and} \ k \in M_k\}$, where $C_m$ and $M_k$ is a macro region and a coarse energy group of a full-core or macro calculation (see Sect. 4.4 of Ref.~\citen{PIP2009}). These equivalence factors are computed in such a way that a macro calculation made over $C_m$ and $M_k$ with a simplified transport operator leads to the same leakage and reaction rates as a reference calculation performed without homogenization and with a fine group discretization.

\vskip 0.08cm

The SPH correction is applied differently, depending on the type of macro-calculation:
\begin{itemize}

\item In the case where the macro-calculation is done with the diffusion theory, neutron balance is satisfied if the SPH correction is applied as
follows:
\begin{equation}
\nabla\cdot\bff(J)_g(\bff(r))+\mu_g\Sigma_g(\bff(r)) {\phi_g(\bff(r))\over \mu_g}={\chi_g\over k_{\rm eff}}\sum_{h=1}^G \mu_h \nu\Sigma_{{\rm f},h}(\bff(r)) {\phi_h(\bff(r))\over \mu_h}
+\sum_{h=1}^G \mu_h \Sigma_{{\rm s0},g\leftarrow h}(\bff(r)){\phi_h(\bff(r))\over \mu_h}
\label{eq:sph1}
\end{equation}


\noindent and
\begin{equation}
\bff(J)_g(\bff(r))=-\mu_g D_g(\bff(r)){\nabla\phi_g(\bff(r))\over \mu_g} .
\label{eq:sph2}
\end{equation}

In conclusion:
\begin{itemize}
\item Diffusion coefficients and all $P_0$ cross sections (including the total cross section {\tt NTOT0}) must be multiplied by $\mu_g$.
\item Scattering matrix terms $\Sigma_{{\rm s0},g\leftarrow h}(\bff(r))$ must be multiplied by $\mu_h$.
\item Fluxes (such as {\tt NWT0} and {\tt FLUX-INTG}) must be divided by $\mu_g$.
\end{itemize}

\item In the case where the macro-calculation is done with the simplified $P_n$ method, the neutron balance is satisfied if the SPH correction is applied on even parity equations as
follows:\cite{sphedf2}
\begin{equation}
\mu_g\Sigma_{0,g}(\bff(r)) {\phi_{0,g}(\bff(r))\over \mu_g}+\nabla\cdot\bff(\phi)_{1,g}(\bff(r))={\chi_g\over k_{\rm eff}}\sum_{h=1}^G \mu_h \nu\Sigma_{{\rm f},h}(\bff(r)) {\phi_{0,h}(\bff(r))\over \mu_h}
+\sum_{h=1}^G \mu_h \Sigma_{{\rm s0},g\leftarrow h}(\bff(r)){\phi_{0,h}(\bff(r))\over \mu_h}
\label{eq:sph3}
\end{equation}
\begin{equation}
{2\ell\over 4\ell+1}\nabla\cdot\bff(\phi)_{2\ell-1,g}(\bff(r))+\mu_g\Sigma_{0,g}(\bff(r)) {\phi_{2\ell,g}(\bff(r))\over \mu_g}+{2\ell+1\over 4\ell+1}\nabla\cdot\bff(\phi)_{2\ell+1,g}(\bff(r))=\sum_{h=1}^G \mu_h \Sigma_{{\rm s2\ell},g\leftarrow h}(\bff(r)){\phi_{2\ell,h}(\bff(r))\over \mu_h}
\label{eq:sph4}
\end{equation}

\noindent and on odd-parity equations as follows:
\begin{equation}
{2\ell+1\over 4\ell+3}\nabla{\phi_{2\ell,g}(\bff(r))\over \mu_g}+{\Sigma_{1,g}(\bff(r))\over \mu_g}\bff(\phi)_{2\ell+1,g}(\bff(r))+{2\ell+2\over 4\ell+3}\nabla{\phi_{2\ell+2,g}(\bff(r))\over \mu_g}=\sum_{h=1}^G {\Sigma_{{\rm s2\ell+1},g\leftarrow h}(\bff(r))\over \mu_g}\phi_{2\ell+1,h}(\bff(r))
\label{eq:sph5}
\end{equation}
\noindent where $\ell\ge 1.$
\vskip 0.08cm

In conclusion:
\begin{itemize}
\item All $P_0$ cross sections (including the total cross section {\tt NTOT0} in the even-parity equations) must be multiplied by $\mu_g$.
\item The total cross section {\tt NTOT1} in the odd-parity equations must be divided by $\mu_g$.
\item Scattering matrix terms $\Sigma_{{\rm s\ell},g\leftarrow h}(\bff(r))$ with $\ell$ even must be multiplied by $\mu_h$.
\item Scattering matrix terms $\Sigma_{{\rm s\ell},g\leftarrow h}(\bff(r))$ with $\ell$ odd must be divided by $\mu_g$.
\item Even parity fluxes (such as {\tt NWT0} and {\tt FLUX-INTG}) must be divided by $\mu_g$.
\item Odd parity fluxes (such as {\tt NWT1} and {\tt FLUX-INTG-P1}) are not modified.
\end{itemize}

\item In the case where the macro-calculation is done in transport theory, but not with a $P_n$--type method, the macroscopic
total cross section {\sl is not modified}, and the even-odd corrections consistent with the simplified $P_n$ method are reported to the
macroscopic within-group scattering cross sections. They are now corrected as\cite{cns2015}
\begin{equation}
\tilde\Sigma_{{\rm s2\ell},g\leftarrow g}(\bff(r))=\mu_g\Sigma_{{\rm s2\ell},g\leftarrow g}(\bff(r))+(1-\mu_g)\, \Sigma_{0,g}(\bff(r))
\label{eq:sph6}
\end{equation}

\noindent and
\begin{equation}
\tilde\Sigma_{{\rm s2\ell+1},g\leftarrow g}(\bff(r))={\Sigma_{{\rm s2\ell+1},g\leftarrow g}(\bff(r))\over\mu_g}+\left(1-{1\over \mu_g}\right)\Sigma_{1,g}(\bff(r))
\label{eq:sph7}
\end{equation}
\noindent where $\ell\ge 0.$

\vskip 0.08cm

Other cross sections and scattering matrix terms are corrected the same way as for the simplified $P_n$ method.

\end{itemize}

\subsubsection{Data input for module {\tt SPH:}}

The \moc{SPH:} module perform a SPH equivalence calculation using
information recovered in a macrolib and apply SPH factors to the corresponding \dds{edition} ({\tt L\_EDIT}),
\dds{microlib} ({\tt L\_LIBRARY}), \dds{macrolib} ({\tt L\_MACROLIB}) or \dds{saphyb} ({\tt L\_SAPHYB}) object. This module is also useful
to extract a corrected or non-corrected \dds{microlib} or \dds{macrolib} from the first RHS object. The calling specification is:

\begin{DataStructure}{Structure \dstr{SPH:}}
$\{$ \dusa{EDINEW} $|$ \dusa{LIBNEW} $|$ \dusa{MACNEW} $|$ \dusa{SAPNEW} $|$ \dusa{CPONEW} $|$ \dusa{EDINAM} $|$ \dusa{LIBNAM} $|$ \dusa{MACNAM} \\
~~~~~$|$ \dusa{SAPNAM} $|$ \dusa{CPONAM} $|$ \dusa{APXNAM} $\}$ \\
~~~~~\moc{:=} \moc{SPH:} $\{$ \dusa{EDINAM} $|$ \dusa{LIBNAM}
$|$ \dusa{MACNAM} $|$ \dusa{SAPNAM} $|$ \dusa{CPONAM} $|$ \dusa{APXNAM} $\}$ \\
~~~~~$\{~[$ \dusa{TRKNAM} $[$ \dusa{TRKFIL} $]~]~|$ \dusa{OPTIM} $\}~[$ \dusa{FLUNAM} $]$ \\
~~~~~\moc{::} \dstr{descsph}
\end{DataStructure}

\noindent
where
\begin{ListeDeDescription}{mmmmmmmm}

\item[\dusa{EDINEW}] {\tt character*12} name of the new \dds{edition} data
structure containing SPH-corrected information (see \Sect{EDIData}). In this
case, an existing \dds{edition} data structure must appear on the RHS.

\item[\dusa{LIBNEW}] {\tt character*12} name of the new \dds{microlib} data
structure containing SPH-corrected information (see \Sect{LIBData}). In this
case, an existing \dds{edition}, \dds{microlib} or \dds{multicompo} data structure
must appear on the RHS.

\item[\dusa{MACNEW}] {\tt character*12} name of the new \dds{macrolib} data
structure containing SPH-corrected information (see \Sect{MACData}).

\item[\dusa{SAPNEW}] {\tt character*12} name of the new \dds{saphyb} data
structure containing SPH information (see \Sect{SAPHYBData}). In this
case, data structure \dusa{SAPNAM} must appear on the RHS.

\item[\dusa{CPONEW}] {\tt character*12} name of the new \dds{multicompo} data
structure containing SPH-corrected information (see \Sect{COMPOData}). In this
case, data structure \dusa{CPONAM} must appear on the RHS.

\item[\dusa{EDINAM}] {\tt character*12} name of the existing \dds{edition} data
structure where the edition information is recovered (see \Sect{EDIData}).

\item[\dusa{LIBNAM}] {\tt character*12} name of the existing \dds{microlib} data
structure where the edition information is recovered (see \Sect{LIBData}).

\item[\dusa{MACNAM}] {\tt character*12} name of the existing \dds{macrolib} data
structure where the edition information is recovered (see \Sect{MACData}).

\item[\dusa{SAPNAM}] {\tt character*12} name of the existing \dds{saphyb} data
structure where the edition information is recovered (see \Sect{SAPHYBData}).

\item[\dusa{CPONAM}] {\tt character*12} name of the existing \dds{multicompo} data
structure where the edition information is recovered (see \Sect{COMPOData}).

\item[\dusa{APXNAM}] {\tt character*12} name of the existing \dds{apex} or \dds{mpo} file where the
edition information is recovered.

\item[\dusa{TRKNAM}] {\tt character*12} name of the existing \dds{tracking} data
structure containing the tracking of the macro-geometry (see \Sect{TRKData}). This object
is compulsory only if a macro-calculation is to be performed by module {\tt SPH:}.

\item[\dusa{TRKFIL}] {\tt character*12} name of the existing sequential binary tracking
file used to store the tracks lengths of the macro-geometry. This file is given if and only if it was
required in the previous tracking module call (see \Sect{TRKData}).

\item[\dusa{OPTIM}] {\tt character*12} name of a {\tt L\_OPTIMIZE} object. The
SPH factors are set equal to the control-variable data recovered from \dusa{OPTIM} if keyword \moc{SPOP} is set.

\item[\dusa{FLUNAM}] {\tt character*12} name of an initialization flux used to start SPH iterations (see \Sect{FLUData}). By
default, a flat estimate of the flux is used.

\item[\dstr{descsph}] structure containing the input data to this module
(see \Sect{descsph}).

\end{ListeDeDescription}

Note: Saphyb files generated by APOLLO2 don't have a signature. If such a Saphyb is given as input
to module {\tt SPH:}, a signature must be included before using it. The following instruction
can do the job:
\begin{verbatim}
Saphyb := UTL: Saphyb :: CREA SIGNATURE 3 = 'L_SA' 'PHYB' ' ' ;
\end{verbatim}

\subsubsection{Specification for the type of equivalence calculation}\label{sect:descsph}

This structure is used to specify the type of equivalence calculation where the
flux and the condensed and/or homogenized cross sections are corrected by SPH
factors, in such a way as to respect a specified transport-transport or
transport-diffusion equivalence criteria.\cite{ALSB1,ALSB2,ALSB3} This
structure is defined as:

\begin{DataStructure}{Structure \dstr{descsph}}
$[$ \moc{EDIT} \dusa{iprint} $]$ \\
$[[$ \moc{STEP} $\{$ \moc{UP} \dusa{NOMDIR} $|$ \moc{AT} \dusa{index} $\}~]]$ \\
$[~\{$ \moc{IDEM} $|$ \moc{MACRO} $|$ \moc{MICRO} $\}~]$ \\
$[~\{$ \moc{OFF} $|$ \moc{SPRD} $[$ \dusa{nmerge} \dusa{ngcond} (\dusa{sph}($i$), $i$=1, \dusa{nmerge}$\times$\dusa{ngcond}) $]~|$ \moc{SPOP} $|$ \moc{HOMO} $|$ \moc{ALBS} $\}~]$  \\
$[~\{$ \moc{PN} $|$ \moc{SN} $[$ \moc{BELL} $]~\}~]$ \\
$[~\{$ \moc{STD} $|$ \moc{SELE\_ALB} $|$ \moc{SELE\_FD} $|$ \moc{SELE\_MWG} $|$ \moc{SELE\_EDF} $|$ \moc{ASYM} \dusa{mixs} $|$ \moc{STD\_FISS} $\}~]~~[$ \moc{ARM} $]$ \\
$[$ \moc{ITER}  $[$ \dusa{maxout} $]$  $[$ \dusa{epsout} $]~]$ \\
$[$ \moc{MAXNB} \dusa{maxnb} $]$ \\
$[$ \moc{GRMIN}  \dusa{lgrmin} $]~[$ \moc{GRMAX}  \dusa{lgrmax} $]$ \\
$[~\{$ \moc{EQUI} \dusa{TEXT4} $[$ \moc{LOCNAM} \dusa{TEXT80} $]~|~$ \moc{EQUI} \dusa{TEXT80} $\}~]$ \\
$[$~\moc{UPS}~$]~[$~\moc{LEAK}~$[$~\dusa{b2}~$]~]$ \\
\end{DataStructure}

\noindent where

\begin{ListeDeDescription}{mmmmmmmm}

\item[\moc{EDIT}] keyword used to modify the print level \dusa{iprint}.

\item[\dusa{iprint}] index used to control the printing of this module. The
\dusa{iprint} parameter is important for adjusting the amount of data that is
printed by this calculation step.

\item[\moc{STEP}] keyword used to set a specific elementary calculation from the first RHS.

\item[\moc{UP}] keyword used to select an elementary calculation located in a subdirectory of \dusa{EDINAM} or \dusa{CPONAM}. By default,
\begin{itemize}
\item the sub-directory name stored in record {\tt 'LAST-EDIT'} is selected if \dusa{EDINAM} is defined at RHS.
\item the sub-directory {\tt 'default'} is selected if \dusa{CPONAM} is defined at RHS.
\end{itemize}

\item[\dusa{NOMDIR}] name of an existing sub-directory of \dusa{EDINAM} or \dusa{CPONAM}. Can also be used to step up in the {\tt output\_n}
group of a \dds{mpo} file.

\item[\moc{AT}] keyword used to select the \dusa{index}--th elementary calculation in \dusa{SAPNAM}, \dusa{CPONAM} or \dusa{APXNAM}.

\item[\dusa{index}] index of the elementary calculation. Can also be used to step at the {\tt statept\_}$($\dusa{index} $-1)$
group of a \dds{mpo} file.

\item[\moc{IDEM}] keyword to force the production of a LHS object of the same type as the RHS (default option).

\item[\moc{MACRO}] keyword to force the production of a macrolib at LHS.

\item[\moc{MICRO}] keyword to force the production of a microlib at LHS.

\item[\moc{OFF}] keyword to specify the SPH factors are all set to 1.0,
meaning no correction. This keyword is useful to get rid of a SPH correction which have been set previously. By
default, the \moc{PN} or \moc{SN} option is activated.

\item[\moc{SPRD}] keyword to specify that the SPH factors are read from input  (if \dusa{nmerge}, \dusa{ngcond} and \dusa{sph}
are set) or recovered from a RHS object (otherwise).

\item[\moc{SPOP}] keyword to specify that the SPH factors are recovered from a RHS object of type \dds{optimize} ({\tt L\_OPTIMIZE}).

\item[\dusa{nmerge}] number of regions.

\item[\dusa{ngcond}] number of energy groups.

\item[\dusa{sph}($i$)] initial value of each SPH factor in each mixture (inner loop) and each group (outer loop).

\item[\moc{HOMO}] keyword to specify that the SPH factors are uniform over the complete
macro-geometry. This option is generally used with a complete homogenization of the
reference geometry, obtained using option \moc{MERG} \moc{COMP}. In this case the
neutron flux  (transport or diffusion) will be
uniform, which allows the SPH factors to be obtained (one per macro-group) using
a non-iterative strategy. For a given macro-group the SPH factor will be equal
to the ratio between the average flux of the region and the surface flux if the
\moc{SELE} option is used otherwise the SPH factor are all set equal to 1.0 (no
correction). The \moc{SELE} option allows an SPH factor equal to the inverse of
the discontinuity factor to be calculated.

\item[\moc{ALBS}] keyword to specify that the albedo of the geometry are to be
taken into account in the complete homogenization process. Thus the \moc{MERG}
and \moc{COMP} options must be specified. The SPH factors are obtained using a
transport-transport equivalence based on a calculation using the collision
probabilities. This option requires a geometry with \moc{VOID} (see
\Sect{descBC}) external boundary conditions to be closed using \moc{ALBS} in
modules \moc{ASM:} (see \Sect{descasm}).\cite{ALSB2} 

\item[\moc{PN}] keyword to activate a calculation of heterogeneous SPH factors based on a converging series of
macro-calculations with the correction strategy of Eqs.~(\ref{eq:sph1}) to~(\ref{eq:sph5}). This is the default option
if the macro-calculation is of diffusion, PN or SPN type. A normalization condition must be set if the macro-geometry
has no boundary leakage ({\sl fundamental mode} condition). If boundary leakage is present, no normalization condition
is used but the SPH iterations are difficult to converge in this case.

\item[\moc{SN}] keyword to activate a calculation of heterogeneous SPH factors based on a converging series of
macro-calculations with the correction strategy of Eqs.~(\ref{eq:sph6}) and~(\ref{eq:sph7}). This is the default option
if the macro-calculation is of PIJ, IC, SN or MOC type. A normalization condition must be set if the macro-geometry
has no boundary leakage ({\sl fundamental mode} condition). If boundary leakage is present, no normalization condition
is used but the SPH iterations are difficult to converge in this case.

\item[\moc{BELL}] keyword to activate the Bell procedure to accelerate the convergence of the SPH factors. This feature is currently
available with macro-calculations of PIJ type.\cite{madrid2}

\item[\moc{STD}] keyword to specify the use of flux-volume normalization for the SPH factors (default option). In each macro-group, the macro-fluxes
in macro regions $i$ are normalized using
$$
\tilde\phi_i=\phi_i\,{\bar\phi_{\rm ref}\over \bar\phi_{\rm mc}}
$$
\noindent where $\bar\phi_{\rm ref}$ is the averaged volumic flux of the reference calculation and $\bar\phi_{\rm mc}$ is the averaged volumic flux of the macro-calculation. Using this definition, the averaged SPH factor is equal to one.

\item[\moc{SELE\_ALB}] keyword to specify the use of Selengut normalization for the SPH factors. It is necessary to know the averaged surfacic flux of the reference calculation. Two possibilities exist:
\begin{itemize}
\item We use collision probabilities. We define the reference geometry with \moc{VOID}
external boundary conditions (see \Sect{descBC}) and to close the region for the collision probability calculations using the \moc{ALBS} option (see \Sect{descasm}).
\item We perform a flux calculation with the current iteration method in Eurydice. This option is only available if a \moc{SYBILT:} tracking is used and if
keyword \moc{ARM} is set in module \moc{ASM:} (see \Sect{descasm}).
\end{itemize}

\item[\moc{SELE\_FD}] keyword to specify the use of Selengut normalization for the SPH factors. It is necessary to know the averaged surfacic flux of the reference calculation. This value can be obtained by defining
a small region near boundary in the reference geometry and by using the \moc{ADF FD\_B} data structure in \Sect{descedi}.

\noindent In each macro-group, the macro-fluxes in macro regions $i$ are normalized using
$$
\tilde\phi_i=\phi_i\,{\phi_{\rm ref}^{\rm gap}\over \bar\phi_{\rm mc}}
$$
\noindent where $\phi_{\rm ref}^{\rm gap}$ is the averaged surfacic flux of the reference calculation. Using this definition, the averaged SPH factor is equal to
$$
\bar\mu={\bar\phi_{\rm ref}\over \phi_{\rm ref}^{\rm gap} } \ \ .
$$

\item[\moc{SELE\_MWG}] keyword to specify the use of Selengut {\sl macro calculation water gap} normalization for the SPH factors.\cite{Chambon2014} It is necessary to know the averaged surfacic flux of the reference and that of the {\sl macro} calculations. This reference value can be obtained by defining
a small region near boundary in the reference geometry and by using the \moc{ADF FD\_B} data structure in \Sect{descedi}.

\noindent In each macro-group, the macro-fluxes in macro regions $i$ are normalized using
$$
\tilde\phi_i=\phi_i\,{\phi_{\rm ref}^{\rm gap}\over \phi^{\rm surf}_{\rm mc}}
$$
\noindent where $\phi_{\rm ref}^{\rm gap}$ is the averaged surfacic flux of the reference calculation and $\phi^{\rm surf}_{\rm mc}$ is the averaged surfacic flux of the macro calculation. Using this definition, the averaged SPH factor is equal to
$$
\bar\mu={\bar\phi_{\rm ref} \,\phi^{\rm surf}_{\rm mc}\over \bar\phi_{\rm mc} \,\phi_{\rm ref}^{\rm gap} } \ \ .
$$

\item[\moc{SELE\_EDF}] keyword to specify the use of generalized Selengut normalization for the SPH factors.\cite{sphedf} It is necessary to know the averaged surfacic flux and the
averaged volumic flux in a row of cells of the reference calculation. The surfacic flux is obtained as with the \moc{SELE} option. The  value of the volumic flux in a row of
cells is computed using index information from the \moc{ADF FD\_H} data structure in \Sect{descedi}.

\noindent In each macro-group, the macro-fluxes in macro regions $i$ are normalized using
$$
\tilde\phi_i=\phi_i\,{\bar\phi_{\rm ref} \, \phi_{\rm ref}^{\rm gap}\over \bar\phi_{\rm mc} \, \phi_{\rm ref}^{\rm row} }
$$
\noindent where $\phi_{\rm ref}^{\rm gap}$ is the averaged surfacic flux of the reference calculation and $\phi_{\rm ref}^{\rm row}$ is the averaged volumic flux in a row of cells of the reference calculation. Using this definition, the averaged SPH factor is equal to
$$
\bar\mu={\phi_{\rm ref}^{\rm row}\over \phi_{\rm ref}^{\rm gap} } \ \ .
$$

\item[\moc{ASYM}] keyword to specify the use of asymptotic normalization of the
SPH factors. The SPH factors in homogenized mixture \dusa{mixs} are set to one
in all macro-energy groups.

\item[\dusa{mixs}] index of the homogenized mixture where asymptotic normalization
is performed.

\item[\moc{STD\_FISS}] keyword to specify the use of flux-volume normalization in all {\sl fissile zones} for the SPH factors. This option is useful for representing assemblies
containing reflector zones.

\item[\moc{ARM}] keyword to activate a solution technique other than the collision probability method. Used with the Eurydice
solution technique within \moc{SYBILT:} to activate the current iteration method.

\item[\moc{ITER}] keyword to specify the main convergence parameters used to control SPH iterations.

\item[\dusa{maxout}] user-defined maximum number of SPH iterations (default value: $200$).

\item[\dusa{epsout}] user-defined convergence criterion (default value: $1.0 \times 10^{-4}$).

\item[\moc{MAXNB}] keyword to specify an auxiliary convergence parameter used to control SPH iterations.

\item[\dusa{maxnb}] acceptable number of SPH iterations with an increase in convergence error before
aborting (default value: $10$).

\item[\moc{GRMIN}] keyword to specify the minimum group index considered
during the equivalence process.

\item[\dusa{lgrmin}] first group number considered during the
equivalence process. By default, \dusa{lgrmin} $=1$.

\item[\moc{GRMAX}]  keyword to specify the maximum group index considered
during the equivalence process.

\item[\dusa{lgrmax}] last group index considered during the equivalence
process. By default, \dusa{lgrmax} is set to the last group
index in the RHS macrolib.

\item[\moc{EQUI}] keyword used to select an existing set of SPH factors in \dusa{SAPNAM} or to store
a new set of SPH factors in \dusa{SAPNEW} or \dusa{SAPNAM}. Also used as \moc{EQUI} \dusa{TEXT80} to select an
existing set of SPH factors in \dusa{APXNAM} or to store a new set of SPH factors in \dusa{APXNAM}.

\item[\dusa{TEXT4}] character*4 user-defined keyword of a set of SPH factors. This keyword is related to variable
\dusa{parkey}, as defined in Sect.~\ref{sect:descsap1} for a local variable.

\item[\moc{LOCNAM}] keyword used to define a character*80 name for the set of SPH factors, if this set is created. By
default, \dusa{TEXT80} is taken equal to \dusa{TEXT4}.

\item[\dusa{TEXT80}] character*80 user-defined name associated to keyword \dusa{TEXT4}. This name is related to
variable \dusa{parnam}, as defined in Sect.~\ref{sect:descsap1} for a local variable. Also used to identify a set of
SPH factors in an \dds{apex} or \dds{mpo} file.

\item[\moc{UPS}] keyword to specify that the macrolib and/or microlib cross sections recovered from a Saphyb, APEX or MPO file are
corrected so as to eliminate up-scattering. This option is useful for reactor analysis codes which cannot
take into account such cross sections.

\item[\moc{LEAK}] keyword used to introduce leakage in the embedded {\sc macrolib}. The buckling is recovered from
a RHS \dds{multicompo}, \dds{saphyb}, \dds{apex} or \dds{mpo} database. Otherwise, the buckling is recovered from optional variable \dusa{b2}.
This option should only be used for non-regression tests.

\item[\dusa{b2}] the imposed buckling corresponding to the leakage.

\end{ListeDeDescription}
\eject
