\subsection{The \moc{TONE:} module}\label{sect:TONEData}

The \moc{TONE:} module perform self-shielding calculations in DRAGON, using
the Tone's method.\cite{tone}  This approach is based on an heterogeneous-homogeneous equivalence principle. In this case, an {\sl equivalent dilution parameter} $\sigma_{{\rm e},g}$ is computed for each resonant isotope, in each resonant region and
each resonant energy group $g$. This dilution parameter is used to interpolate pretabulated effective cross sections for the infinite homogeneous medium, previously obtained with the {\sl flux calculator} of the {\tt GROUPR} module in code NJOY.\cite{njoy2010}
Each resonant isotope, identified as such by the \dusa{inrs}
parameter defined in \Sect{LIBData}, is to be recalculated. The general format of
the data for this module is:

\begin{DataStructure}{Structure \dstr{TONE:}}
\dusa{MICLIB} \moc{:=} \moc{TONE:} $\{$ \dusa{MICLIB} $|$ \dusa{OLDLIB} $\}$ 
\dusa{TRKNAM} $[$ \dusa{TRKFIL} $]$ \moc{::} \dstr{desctone}
\end{DataStructure}

\noindent where

\begin{ListeDeDescription}{mmmmmmmm}

\item[\dusa{MICLIB}] {\tt character*12} name of the \dds{microlib} that will
contain the microscopic and macroscopic cross sections updated by the
self-shielding module. If
\dusa{MICLIB} appears on both LHS and RHS, it is updated; otherwise, the
internal library \dusa{OLDLIB} is copied into
\dusa{MICLIB} and \dusa{MICLIB} is updated.

\item[\dusa{OLDLIB}] {\tt character*12} name of a read-only \dds{microlib} 
that is copied into \dusa{MICLIB}.

\item[\dusa{TRKNAM}] {\tt character*12} name of the required \dds{tracking}
data structure.

\item[\dusa{TRKFIL}] {\tt character*12} name of the sequential binary tracking
file used to store the tracks lengths. This file is given if and only if it was
required in the previous tracking module call (see \Sect{TRKData}).

\item[\dstr{desctone}] structure describing the self-shielding options.

\end{ListeDeDescription}

Each time the \moc{TONE:} module is called, a sub-directory is updated in the
\dds{microlib} data structure to hold the last values defined in the
\dstr{desctone} structure. The next time this module is called,
these values will be used as floating defaults.

\subsubsection{Data input for module \moc{TONE:}}\label{sect:desctone}

\begin{DataStructure}{Structure \dstr{desctone}}
$[$ \moc{EDIT} \dusa{iprint} $]$ \\
$[$ \moc{GRMIN}  \dusa{lgrmin} $]~~[$ \moc{GRMAX}  \dusa{lgrmax} $]$ \\
$[$ \moc{MXIT} \dusa{imxit} $]~~[$ \moc{EPS}  \dusa{valeps} $]$  \\
$[~\{$ \moc{SPH} $|$ \moc{NOSP} $\}~]$
$[~\{$ \moc{TRAN} $|$ \moc{NOTR} $\}~]$
$[~\{$ \moc{PIJ} $|$ \moc{ARM} $\}~]$ \\
{\tt ;}
\end{DataStructure}

\noindent 
where

\begin{ListeDeDescription}{mmmmmmmm}

\item[\moc{EDIT}] keyword used to modify the print level \dusa{iprint}.

\item[\dusa{iprint}] index used to control the printing of this module. The
amount of output produced by this tracking module will vary substantially
depending on the print level specified. 

\item[\moc{GRMIN}] keyword to specify the minimum group number considered
during the self-shielding process.

\item[\dusa{lgrmin}] first group number considered during the
self-shielding process. By default, \dusa{lgrmin} is set to the first group
number containing self-shielding data in the library.

\item[\moc{GRMAX}]  keyword to specify the maximum group number considered
during the self-shielding process.

\item[\dusa{lgrmax}] last group number considered during the self-shielding
process. By default, \dusa{lgrmax} is set is set to the last group
number containing self-shielding data in the library.

\item[\moc{MXIT}]  keyword to specify the maximum number of iterations during
the self-shielding process.

\item[\dusa{imxit}] the maximum number of iterations. The default is
\dusa{imxit}=20.

\item[\moc{EPS}] keyword to specify the convergence criterion for the
self-shielding iteration.

\item[\dusa{valeps}] the convergence criterion for the self-shielding iteration.
By default, \dusa{valeps}=$1.0\times 10^{-4}$.

\item[\moc{SPH}] keyword to activate the SPH equivalence scheme which
modifies the self-shielded averaged neutron fluxes in
heterogeneous geometries. This is the default option.

\item[\moc{NOSP}] keyword to deactivate the SPH equivalence scheme which
modifies the self-shielded averaged neutron fluxes in
heterogeneous geometries.

\item[\moc{TRAN}] keyword to activate the transport correction option for
self-shielding calculations (see \moc{CTRA} in \Sectand{MACData}{LIBData}). This is the default option.

\item[\moc{NOTR}] keyword to deactivate the transport correction option for
self-shielding calculations (see \moc{CTRA} in \Sectand{MACData}{LIBData}).

\item[\moc{PIJ}] keyword to specify the use of complete collision
probabilities in the self-shielding calculations of \moc{TONE:}.
This is the default option for \moc{EXCELT:} and \moc{SYBILT:} trackings.
This option is not available for \moc{MCCGT:} trackings.

\item[\moc{ARM}] keyword to specify the use of iterative flux techniques
in the self-shielding calculations of \moc{TONE:}.
This is the default option for \moc{MCCGT:} trackings.

\end{ListeDeDescription}
\eject
