\subsection{The \moc{CPO:} module}\label{sect:CPOData}

The \moc{CPO:} module is used to generate the reactor cross-section database in Version3 format to be used in a full core
calculation using DONJON. This type of database is only parametrized in burnup
(or irradiation). The calling specifications are:

\begin{DataStructure}{Structure \dstr{CPO:}}
\dusa{CPONAM} \moc{:=} \moc{CPO:}  $[$ \dusa{CPONAM} $]$ \dusa{EDINAM} 
$[$ \dusa{BRNNAM} $]$ \moc{::} \dstr{desccpo}
\end{DataStructure}

\noindent
 where

\begin{ListeDeDescription}{mmmmmmmm}

\item[\dusa{CPONAM}] \verb|character*12| name of the \dds{cpo} data structure containing the reactor
database. Additional contributions can be included in the reactor cross-section database if \dusa{CPONAM}
appears on the RHS.

\item[\dusa{EDINAM}] \verb|character*12| name of the read-only \dds{edition} data structure.

\item[\dusa{BRNNAM}] \verb|character*12| name of the read-only \dds{burnup} data structure containing the
depletion history. This information is given only if the reactor database is to contain burnup dependent data.

\item[\dstr{desccpo}] structure containing the input data to this module (see \Sect{desccpo}).
\end{ListeDeDescription}


\subsubsection{Data input for module \moc{CPO:}}\label{sect:desccpo}

\begin{DataStructure}{Structure \dstr{desccpo}}
$[$ \moc{EDIT} \dusa{iprint} $]$ \\
$[$ \moc{B2} $]~~[$ \moc{NOTR} $]$ \\
$\{$ \moc{STEP} \dusa{NOMDIR} $|$ \moc{BURNUP} \dusa{PREFIX} $\}$ \\
$[$ $[$ \moc{EXTRACT}  $\{$ \moc{ALL} $|$ \dusa{NEWNAME} (\dusa{OLDNAME}($i$), $i$=1,niext) $\}$ $]$ $]$ \\ 
$[$ \moc{NAME} \dusa{NDIR} $]$ \\
$[~\{$ \moc{GLOB} $|$ \moc{LOCA} $\}~]$
\end{DataStructure}

\noindent
 where

\begin{ListeDeDescription}{mmmmmmmm}
\item[\moc{EDIT}] keyword used to modify the print level \dusa{iprint}.

\item[\dusa{iprint}] index used to control the printing of this module. The amount of output produced by this
tracking module will vary substantially depending on the print level specified.

\item[\moc{B2}] keyword to specify that the buckling correction ($dB^{2}$) is to be applied to the cross
section to be stored on the reactor database. By default (fixed default), such a correction is not taken into
account.

\item[\moc{NOTR}] keyword to specify that the cross section to be stored on the reactor database are not to
be transport corrected. By default (fixed default), transport corrected cross section are considered when
the \moc{CTRA} option is activated in \moc{MAC:} or \moc{LIB:} (see \Sectand{MACData}{LIBData}).

\item[\moc{STEP}] keyword to specify that a specific cross section directory stored in \dusa{EDINAM} via the
\moc{SAVE} option in the \moc{EDI:} module is to be transferred to \dusa{CPONAM}.

\item[\dusa{NOMDIR}] \verb|character*12| name of the specific cross section directory to be treated.

\item[\moc{BURNUP}] keyword to specify that a chain of cross section directory stored in \dusa{EDINAM} via
the \moc{SAVE} option in the \moc{EDI:} module will be transferred to \dusa{CPONAM}.

\item[\dusa{PREFIX}] \texttt{character*8} prefix name of the cross section directory to be treated. DRAGON
will transfer into the reactor database all the directories with full name \verb|NAMDIR| created using

\begin{quote}
\verb|WRITE(NAMDIR,'(A8,I4)')| \textit{PREFIX},\verb|nb|
\end{quote}
 where \verb|nb| is an integer greater than 0 indicating the depletion step index. 

\item[\moc{EXTRACT}] keyword to specify that the contribution of some isotopes to the macroscopic cross
sections associated with each homogenized mixture should be extracted before being stored on the reactor
database. The microscopic cross sections and concentrations associated with these isotopes should also be
generated and stored on the reactor database.  

\item[\moc{ALL}] keyword to specify that all the isotopes processed using the \moc{MICR} option of the
\moc{EDI:} module should be extracted from the macroscopic cross sections associated with each homogenized
mixture.

\item[\dusa{NEWNAME}] \verb|character*12| name under which a given set of extracted isotope will be stored
on the reactor database.

\item[\dusa{OLDNAME}] array of \verb|character*8| name of isotopes to be extracted from the macroscopic
cross section associated with each homogenized mixture.

\item[\moc{NAME}] keyword to specify the prefix for the name of the sub-directory where the information
corresponding to a single homogenized region will be stored. The fixed default is
\dusa{NDIR}=\verb*|'COMPO~~~'|.

\item[\dusa{NDIR}] \verb|character*8| prefix for the name of the sub-directory. The complete name is
constructed by the concatenation of \dusa{NDIR} with a four digit integer value. 

\item[\moc{GLOB}] keyword to specify that global parameters are used to index the database (default option). A global parameter is
defined over the complete calculation domain.

\item[\moc{LOCA}] keyword to specify that local parameters are used to index the database. A local parameter is
defined over each homogenization mixture.

\end{ListeDeDescription}

\eject

