\subsection{MATXS7A microscopic cross-section examples}\label{sect:ExMATXS}

The test cases we will consider here use the \moc{LIB:} module to specify that
the cross sections will be taken from a MATXS7A 69 groups microscopic
cross-sections library. We will assume that this library is located in file
\moc{MATXS7A}.

\subsubsection{\tst(TCXA01) -- The Mosteller benchmark.} 
              
The typical input data required to analyze this
benchmark\cite{Mostel} with DRAGON is of the form:

\listing{TCXA01.x2m}

\vskip 0.3cm

The input deck begins with declarations for the linked lists and the interface
files and the various modules used for this DRAGON execution. Any word not declared is considered as
a keyword.

The {\tt LIB:} module is used to interpolate the microscopic cross sections
in absolute temperature and dilution and to produce group-ordered macroscopic
cross sections. We use the MATXS format 69 groups microscopic cross
section library named {\tt 'MATXS7A'}.\cite{MATXS7A}.
Each mixture at a given absolute temperature (in Kelvin) is defined in terms
of MATXS isotope names ({\tt U235, U238, O16,} etc.). In this case, the
number density (in $10^{24}$ particules per cubic centimeter) for each isotope is
provided. Resonant region indices and the type of thermal scattering
approximation used with the 42 thermal groups (free gas or H$_2$O molecular
model) is also specified. Only MATXS type libraries require the thermalization
model to be set.


The {\tt GEO:} module is used to define the geometry. Here two types of geometry are considered,
\moc{MOSTELA} a 1--D annular geometry and \moc{MOSTELC} a 2--D Cartesian geometry. These geometries
are defined before knowing the type of discretization or numerical treatment that will follow.
For \moc{MOSTELA} the first line indicates that the geometry has circular boundaries and that it
contains three concentric annular subregions. The boundary conditions (reflection), the annular
radii and the mixture index corresponding to each region of the cell are
given successively. For \moc{MOSTELC} the first line indicates that this geometry has 2--D
Cartesian boundaries containing three subregions, two of which are annular. The boundary conditions
(reflection on each side), the annular radii, the external side widths and the mixture index
corresponding to each region of the cell are given successively.

Four cases are then considered. First we will analyse the annular geometry using the \moc{SYBILT:} module for flux
calculation. The \moc{DISCR} and dds{tracking} structures are thereby
generated. The {\tt SHI:} module uses microscopic cross section data contained in the
\moc{LIBRARY} and tracking information contained in {\tt 'DISCR'} and {\tt 'TRACKS'} in order to
compute the actual dilution of each resonant isotope ({\tt U235} and {\tt U238}) and to perform a
new interpolation in the MATXS file. Dilutions are only computed for the energy groups with resonance data present on the library; the other groups are assumed to stay at infinite dilution.

For the second case we will analyse the Cartesian geometry using the again the
\moc{SYBILT:} tracking module for self shielding calculations and the \moc{SYBILT:} module for
flux calculation. The \moc{DISCR} and \dds{tracking} structures are thereby generated.

Four cases are then considered. First we will analyse the annular geometry using the {\tt SYBILT:}
tracking module allows the geometry named {\tt 'MOSTEL'} to be discretized by the full CP tracking
algorithm. A new tracking file (sequential binary) is created and named {\tt 'TRACKS'}, together
with a
\dds{tracking}l structure named {\tt 'DISCR'}. A periodic tracking (with 12
angles and 20.0 tracks per cm) is considered here.

The {\tt ASM:} module uses macroscopic cross section data contained in the
embedded \dds{macrolib} of {\tt 'LIBRARY'} and tracking information contained
in {\tt 'DISCR'} and {\tt 'TRACKS'} in order to compute the reduced and
scattering modified collision probability matrices for each of the 69 energy
groups. We have not used the important capability of DRAGON to use a different
tracking to perform self-shielding and flux calculations.


The {\tt FLU:} module uses macroscopic cross section data contained in {\tt
'LIBRARY'} (recovered from the dependency tree) and CPs contained in {\tt
'CP'} in order to compute the neutron flux for each of the 69 energy groups. The
transport equation is solved for the effective multiplication factor
without buckling or leakage model.


Next, the {\tt EDI:} module performs spatial homogenization (the cross sections
are smeared over the complete cell) and coarse energy group condensation. The
first coarse energy group contains the micro-groups 1 to 27; the second coarse
energy group contains the remaining micro-groups. 

\eject
