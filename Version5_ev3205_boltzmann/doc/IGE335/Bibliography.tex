\begin{thebibliography}{99}

\bibitem{PIP2009}
A. H\'ebert, {\sl Applied Reactor Physics}, Second Edition, Presses Internationales Polytechnique, ISBN 978-2-553-01698-1, 396 p., Montr\'eal, 2016.

\bibitem{Dragon1}
G. Marleau and A. H\'ebert, ``A New Driver for Collision Probability Transport
Codes", {\sl Int. Top. Mtg. on Advances in Nuclear Engineering
Computation and Radiation Shielding}, Santa Fe, New Mexico, April 9--13 (1989).

\bibitem{Dragon2}
G.~Marleau, R.~Roy and A.~H\'{e}bert,
``DRAGON: A Collision Probability Transport Code for Cell and Supercell
Calculations'',  Report IGE--157,  \'{E}cole Polytechnique de Montr\'{e}al
(1993).

\bibitem{Dragon3}
G.~Marleau, A.~H\'{e}bert and R.~Roy, ``New Computational Methods Used in
the Lattice Code DRAGON'',  {\sl Top Mtg. on Advances in Reactor Physics},
Charleston, SC, March 8-11 1992; 

\bibitem{Dragon4}
A.~H\'{e}bert, G.~Marleau and R.~Roy, ``Application of the Lattice Code
DRAGON to CANDU Analysis'',  {\sl Trans. Am. Nucl. Soc.}, {\bf 72}, 335 (1995);

\bibitem{ganlib5}
A. H\'ebert and R.~Roy,
``The Ganlib5 kernel guide (64--bit clean version),"
Report IGE-332, \'Ecole Polytechnique de Montr\'eal, January 2013.

\bibitem{cle2000}
R.~Roy, \textsl{The CLE-2000 Tool-Box}, 
Report IGE--163, Institut de g\'enie nucl\'eaire, \'{E}cole Polytechnique de Montr\'eal,
Montr\'{e}al, Qu\'{e}bec (1999).

\bibitem{Apollo}
A.~Hoffman et al., ``APOLLO: Code Multigroupe de r\'esolution de l'\'equation du
transport pour les neutrons  thermiques et rapides'', CEA-N-1610, Commisariat
\`a l'\'Energie Atomique, France (1973).

\bibitem{Apollo2}
S. Loubi\`ere, R. Sanchez, M. Coste, A. H\'ebert, Z. Stankovski, C. Van Der Gucht and I. Zmijarevic, ``APOLLO2,
Twelve Years Later", {\sl Int. Conf. on Mathematics and Computation,
Reactor Physics and Environmental Analysis in Nuclear Applications}, Madrid, Spain, September 27--30, 1999.

\bibitem{Apollo3}
I. Zmijarevic, N. Huot, F. Auffret and P. Bellier, ``Description of the APOLLO3 Multi-parameter Output Library for the
version AP3-2.0,'' DEN/DANS/DM2S/SERMA/LTSD/RT/17-6237/A, Commissariat \`a l'\'energie atomique et aux \'energies alternatives (2017).

\bibitem{ndas}
P. J. Laughton, ``NJOYPREP and WILMAPREP: UNIX-Based Tools for WIMS-
AECL Cross-Section Library Production," Atomic Energy of Canada,
Report COG-92-414 (Rev. 0), June 1993.

\bibitem{subg}
A.~H\'ebert, ``A Comparison of Three Techniques for Computing Probability
Tables", {\sl Int. Conf. on the Physics of Nuclear Science and Technology},
Long Island, New York, October 5 -- 8, 1998.

\bibitem{pt}
A.~H\'ebert and M.~Coste, ``Computing Moment-Based Probability Tables for
Self-Shielding Calculations in Lattice Codes," {\sl Nucl. Sci. Eng.}, {\bf 142},
245 - 257 (2002).

\bibitem{nse2004}
A.~H\'ebert, ``The Ribon Extended Self-Shielding Model," {\sl Nucl. Sci. Eng.}, {\bf 151}, 1-24  (2005).

\bibitem{SPM09}
A. H\'ebert, ``Development of the Subgroup Projection Method for Resonance Self-Shielding Calculations," {\sl Nucl. Sci. Eng.} {\bf 162}, 56-75 (2009).

\bibitem{rse2021}
R. Kondo, T. Endo, A. Yamamoto, S. Takeda, H. Koike, K. Yamaji and D. Sato, ``A New Resonance Calculation Method Using Energy
Expansion Based on a Reduced Order Model," {\sl Nucl. Sci. Eng.} {\bf 195}, 694-716 (2021).

\bibitem{wlup}
B. Dodd, S. Basu, S. Paranjpe and A. Trkov, ``WIMS-D Library Update," International Atomic Energy Agency
Report STI/PUB/1264, Vienna, Austria (2007).

\bibitem{st}
M.~Coste {\sl et al}, ``New Improvements in the Self-Shielding Formalism of the APOLLO2
Code", {\sl Joint Int. Conf. on
Mathematical Methods and Supercomputing in Nuclear Applications}, Karlsruhe,
Germany, April 19 -- 23, 1993.

\bibitem{SPH}
A.~H\'ebert, ``D\'eveloppement de la m\'ethode SPH: Homog\'en\'eisation de
cellules dans un r\'eseau non uniforme  et calcul des param\`etres de
r\'eflecteur'', CEA-N-2209, Commissariat \`a l'\'Energie Atomique, France (1981).

\bibitem{roy}
R.~Roy, D.~Rozon, A.~H\'ebert and G.~Hotte, `` Treatment of Circular Boundary
Conditions in Neutron Diffusion Calculations", Third Int. Conf. on Simulation
Methods in Nuclear Engineering, Montr\'eal, Canada, April 18 -- 20,
1990.

\bibitem{DragonPIJI}
R.~Roy, A.~H\'ebert and G.~Marleau, ``A Transport Method for Treating
Three-Dimensional Lattices of Heterogeneous Cells", {\sl Nucl.~Sci.~Eng.}, {\bf
101}, 217 (1989).

\bibitem{Mtl93a}
 R.~Roy, G.~Marleau, J.~Tajmouati and D.~Rozon, ``Modelling of CANDU Reactivity
Control Devices with the Lattice Code DRAGON'', {\sl Ann.~nucl.~Energy}, {\bf
21}, 115 (1994).

\bibitem{mccg}
I.~R.~Suslov, ``Solution of Transport Equation in 2-- and 3--Dimensional
Irregular Geometry by the Method of Characteristics", {\sl Joint Int. Conf. on
Mathematical Methods and Supercomputing in Nuclear Applications}, Karlsruhe,
Germany, April 19 -- 23, 1993.

\bibitem{suslov2}
I. R. Suslov, ``An Algebraic Collapsing Acceleration Method for Acceleration of the Inner (Scattering) Iterations in Long Characteristics Transport Theory",
{\sl Int. Conf. on Supercomputing in Nuclear Applications}, Paris, France, September 22 -- 24, 2003.

\bibitem{chicago2}
R. Le Tellier and A. H\'ebert, ``Application of the DSA Preconditioned GMRES Formalism to the Method of Characteristics -- First Results",
{\sl Int. Mtg. on the Physics of Fuel Cycles and Advanced Nuclear Systems:
Global Developments. PHYSOR-2004}, Chicago, Illinois, April 25 -- 29, 2004.

\bibitem{DragonPIJS1}
R.~Roy, ``Anisotropic Scattering for Integral Transport Codes. Part 1. Slab
Assemblies", {\sl Ann.~nucl.~Energy}, {\bf 17}, 379 (1990).

\bibitem{DragonPIJS2}
R.~Roy, ``Anisotropic Scattering for Integral Transport Codes. Part 2. Cyclic
Tracking and its Application to $XY$ Lattices" {\sl Ann.~nucl.~Energy}, {\bf
18}, 511 (1991).

\bibitem{DragonPIJS3}
R.~Roy, G.~Marleau, A.~H\'ebert and D. Rozon, ``A Cyclic Tracking Procedure for Collision
Probability Calculations in 2-D Lattices'', {\sl Int. Topical Meeting on
Advances in Mathematica, Computation and Reactor Physics}, Pittsburgh, PA, April
28 -- May 2, 1991. 

\bibitem{Mtl93b} 
G.~Marleau and R.~Roy, ``Use of Specular Boundary Conditions for CANDU Cell
Analysis'',  {\sl Fourth Int. Conf. on Simulation Methods in Nuclear
Engineering}, Montr\'eal, June 2-4, 1993.

\bibitem{cdd}
H.~Khalil, ``Effectiveness of a Consistently Formulated Diffusion Synthetic
Acceleration Differencing approach", {\sl Nucl.~Sci.~Eng.}, {\bf 98}, 226 (1988).

\bibitem{domino}
S. Moustafa, I. Dutka-Malen, L. Plagne, A. Pon\c cot and P. Ramet, ``Shared memory parallelism for 3D Cartesian discrete ordinates solver," {\sl Ann.~nucl.~Energy}, {\bf 82}, 179 (2015).

\bibitem{ligou}
K. Przybylski and J. Ligou, ``Numerical Analysis of the Boltzmann Equation Including Fokker-Planck Terms," {\sl Nucl.~Sci.~Eng.}, {\bf 81}, 92 (1982).

\bibitem{gmres}
Y. Saad and M. H. Schultz, ``GMRES: A Generalized Minimal RESidual Algorithm For Solving Nonsymmetric Linear Systems", {\sl SIAM J. Sci. Stat. Comput.}, {\bf 7}, 856-869 (1986).

\bibitem{quadrupole}
E. M. Baker, ``Quadruple range quadrature verification and extension," Los Alamos National Laboratory,
Report LA--UR--07--8050, September 2006.

\bibitem{PIM}
G. Marleau and A. H\'ebert, ``Solving the Multigroup Transport Equation Using the
Power Iteration Method'',  {\sl 1985 Simulation Symposium on Reactor Dynamics and
Plant Control}, Kingston,  Ontario, April 22-23, 1985.

\bibitem{Buck}
G. Marleau and A. H\'ebert, ``Introduction of an Improved Critical Buckling
Search in WIMS'', {\sl 1986 Simulation Symposium on Reactor Dynamics and Plant
Control}, Hamilton, Ontario, April 21- 22, 1986.

\bibitem{MATXS}
R.E.~Macfarlane, ``TRANSX-CTR: A code for Interfacing MATXS Cross-Section
Libraries to Nuclear  Transport Codes for Fusion Systems Analysis'', LA-9863-MS,
Los Alamos Scientific Laboratory, New  Mexico (1984).

\bibitem{WIMS}
J.V.~Donnelly, ``WIMS-CRNL, A User's Manual for the Chalk River Version of
WIMS'', AECL-8955,  Atomic Energy of Canada Limited (1986).

\bibitem{WIMS-D}
J.R.~Askew et al., ``A General Description of the Lattice Code WIMS'', {\sl J. of British Nucl. Energy Soc.}, {\bf 5}, 564 (1966).

\bibitem{TRANSX2}
R.E.~Macfarlane, ``TRANSX-2: A Code for Interfacing MATXS Cross-Section
Libraries to Nuclear Transport Codes'', LA-12312-MS, Los Alamos Scientific
Laboratory, New  Mexico (1992).

\bibitem{DragonDataStructures}
A. H\'ebert, G.~Marleau and R.~Roy,
``A Description of the DRAGON and TRIVAC Version4 Data Structures,"
Report IGE-295, \'Ecole Polytechnique de Montr\'eal, August 2006.

\bibitem{ige260}
G. Marleau, ``New Geometries Processing in DRAGON: The NXT: Module," Technical Report IGE-260, \'Ecole 
Polytechnique de Montr\'eal (2006).

\bibitem{eqn}
B. G. Carlson, ``Tables of Equal Weight Quadrature $EQ_{n}$ Over the Unit Sphere," Technical Report LA-4734, 
Los Alamos Scientific Laboratory (1971).

\bibitem{pntn}
G. Longoni, and A. Haghighat, ``Development of New Quadrature Sets with the ``Ordinate Splitting" Technique,"
M\&C-2001, American Nuclear Society Topical Meeting in Mathematics and Computations, Salt 
Lake City, Utah (2001), (Proceedings available on CD-Rom). 

\bibitem{sms}
R. Sanchez, L. Mao, and S. Santandrea, ``Treatment of Boundary Conditions in Trajectory-Based Deterministic
Transport Methods," {\sl Nucl. Sci. Eng.}, {\bf 140}, 23--50 (2002). 

\bibitem{RoyMoc1}
R.~Roy, 
``The Cyclic Characteristics Method'', 
{\sl IInt. Conf. on the Physics of Nuclear Science and Technology}, 
Long Island, New York, October 5--8, 1998,

\bibitem{RoyMoc2}
R.~Roy, 
``The Cyclic Characteristics Method with Anisotropic Scattering'', 
\textsl{M\& C'99 Mathematics and Computation, Reactor Physics and Environmental Analysis
in Nuclear Applications}, Madrid, Spain, September  27--30, 1999,

\bibitem{LCMD}
A. Leonard and C. T. McDaniel, ``Optimal Polar Angles and Weights for the
Characteristics Method," \textsl{Trans. Am. Nucl. Soc.}, \textbf{73}, 172 (1995).

\bibitem{LeTellierpa}
R. Le Tellier and A. H\'ebert, ``Anisotropy and Particle Conservation for Trajectory--Based 
Deterministic Methods," {\sl Nucl. Sci. Eng.}, {\bf 158}, 28--39 (2008).

\bibitem{CACTUS}
M.J.~Halsall, 
\textsl{CACTUS, A Characteristics Solution to the Neutron Transport 
Equation in Complicated Geometries}, 
Report AEEW-R 1291, Atomic Energy Establishment,
Winfrith (1980).

\bibitem{sapo}
R. Sanchez, ``Renormalized Treatment of the Double Heterogeneity with
the Method of Characteristics,"
{\sl Int. Mtg. on the Physics of Fuel Cycles and Advanced Nuclear Systems:
Global Developments. PHYSOR-2004}, Chicago, Illinois, April 25 -- 29, 2004.

\bibitem{BIHET}
A.~H\'ebert, ``A Collision Probability Analysis of the Double-Heterogeneity
Problem", {\sl Nucl.~Sci.~Eng.}, {\bf 115}, 177 (1993).

\bibitem{She2017}
D.~She, Z.~Liu and L.~Shi, ``An Equivalent Homogenization Method for Treating the Stochastic Media", 
{\sl Nucl.~Sci.~Eng.}, {\bf 185}, 351 (2017).

\bibitem{apollo1}
A.~Hoffmann, F.~Jeanpierre, A.~Kavenoky, M. Livolant and H. Lorain,
``APOLLO: Code Multigroupe de R\'esolution de l'\'Equation du Transport
pour les Neutrons Thermiques et Rapides'', Note CEA-N-1610, Commissariat
\`a l'\'Energie Atomique, Saclay, France (1973).

\bibitem{ALCOL}
A.~Kavenoky, ``Calcul et utilisation des probabilit\'es de premi\`ere collision
pour les milieux h\'et\'erog\`enes \`a  une dimension'', CEA-N-1077,
Commissariat \`a l'\'Energie Atomique, France (1969).

\bibitem{SANCHEZ}
R.~Sanchez, ``Quelques sch\'emas approximatifs dans la r\'esolution
par la m\'ethode des probabilit\'es de collision de l'\'equation
int\'egrale du transport \`a deux dimensions'', CEA-N-2165,
Commissariat \`a l'\'Energie Atomique, France (1980).

\bibitem{BIVAC}
A.~H\'ebert, ``Application of a Dual Variational Formulation to Finite Element
Reactor Calculations", {\sl Ann.~nucl.~Energy}, {\bf 20}, 823 (1993).

\bibitem{TRIVAC}
A.~H\'ebert, ``TRIVAC, A Modular Diffusion Code for Fuel Management and Design
Applications", {\sl Nucl. J. of Canada}, Vol. 1, No. 4, 325 (1987).

\bibitem{nse2005}
A. H\'ebert, ``The Search for Superconvergence in Spherical Harmonics Approximations,'' {\sl Nucl. Sci. Eng.}, {\bf 154}, 134 (2006).

\bibitem{ane10a}
A. H\'ebert, ``Mixed-dual implementations of the of the simplified $P_n$ method," {\sl Ann. nucl. Energy}, {\bf 37}, 498 (2010).

\bibitem{SHIBA}
A.~H\'ebert and G.~Marleau, ``Generalization of the Stamm'ler Method for the
Self-Shielding of Resonant Isotopes in Arbitrary Geometries," {\sl
Nucl.~Sci.~Eng.}, {\bf 108}, 230 (1991).

\bibitem{njoy2010}
R. E. MacFarlane and A. C. Kahler, ``Methods for Processing ENDF/B-VII with NJOY," {\sl Nuclear Data Sheets}, {\bf 111}, 2739 (2010).

\bibitem{tone}
T. Tone, ``A Numerical Study of Heterogeneity Effects in Fast Reactor Critical Assemblies," {\sl J. Nucl. Sci. Technol.}, {\bf 12[8]}, 467 (1975).

\bibitem{toronto04}
A.~H\'ebert, ``Revisiting the Stamm'ler Self-Shielding Method," paper presented at the
\textsl{25th CNS Annual Conference}, June 6--9, Toronto, 2004.

\bibitem{hasan}
H.~Saygin and  A.~H\'ebert, ``A New Self-Shielding Method Based on a Detailed
Cross-Section Representation in the Resolved Energy Domain,"
{\sl Nucl.~Sci.~Eng.}, {\bf 122}, 276 (1996).

\bibitem{coste}
M.~Coste, ``Absorption r\'esonnante des noyaux lourds dans les
r\'eseaux h\'et\'erog\`enes -- I-Formalisme du module
d'autoprotection d'APOLLO2,"
CEA-N-2746, Commissariat \`a l'\'Energie Atomique, France (1994).

\bibitem{RENOR}
R.~Roy and G.~Marleau, ``Normalization Techniques for Collision Probability
Matrices'', {\sl PHYSOR-90}, Marseille, France, April 23--27, 1990.

\bibitem{Helios}
E.A.~Vliiarino, R.J.J.~Stammler, A.A.~Ferri and J.J.~Casal, ``HELIOS: Angularly
Dependent Collision Probabilities'', {\sl
Nucl.~Sci.~Eng.}, {\bf 112}, 16-31 (1992).

\bibitem{ecco}
M. J.~Grimstone, J. D.~Tullett and G.~Rimpault, ``Accurate Treatments of
Fast Reactor Fuel Assembly Heterogeneity with the ECCO Cell Code'',
{\sl Proc. Int. Conf. on the Physics of Reactors: Operation, Design and
Computation -- PHYSOR 90}, Marseille, France, p. IX:24, April 23-27 (1990).

\bibitem{rimpault}
G.~Rimpault, ``Algorithmic Features of the ECCO Cell Code for Treating
Heterogeneous Fast Reactor Subassemblies", {\sl Int. Conf. on
Mathematics and Computations, Reactor Physics, and Environmental Analyses},
Portland, Oregon, April 30 -- May 4, 1995.

\bibitem{PIJK0}
P.~Benoist, J.~Mondot and I.~Petrovic, `Calculational and Experimental
Investigations of Void Effect -- A Simple Theoretical Model for Space-Dependent
Leakage Treatment of Heterogeneous Assemblies'', {\sl Nucl.~Sci.~Eng.}, {\bf
118}, 197 (1994).

\bibitem{PIJK}
I.~Petrovic, P.~Benoist and G.~Marleau, ``A Quasi-Isotropic Reflecting Boundary
Condition for the Heterogeneous Leakage Model Tibere'', {\sl Nucl.~Sci.~Eng.},
{\bf 122}, 151 (1996)

\bibitem{MATXS7A}
``MATXS7A - 69 Neutron Group Cross Section Library in MATXS'', DLC-117, RSIC
Data Library  Collection, Oak Ridge National Laboratory (1985).

\bibitem{WIMKAL}
J.-D. Kim, J.T. Lee, C.-S. Gil and H.R. Kim, ``Generation and Benchmarking of a 69--group Cross Section Library for Thermal Reactor Applications'', {\sl J. of the Korean Nucl. Soc.}, {\bf 21} 245 (1989).

\bibitem{JEF}
P.~Vontobel and S.~Pelloni, ``New JEF/EFF Based MATXS-Formatted Nuclear Data
Libraries", {\sl Nucl.~Sci.~Eng.}, {\bf 101}, 298 (1989).

\bibitem{ALSB1}
A.~H\'ebert, ``A Consistent Technique for the Pin-by-Pin Homogenization of a
Pressurized Water Reactor Assembly", {\sl Nucl.~Sci.~Eng.}, {\bf 113}, 227
(1993).

\bibitem{ALSB2}
A.~H\'ebert and G.~Mathonni\`ere, ``Development of a Third-Generation {\sl
Superhomog\'en\'eisation} Method for the Homogenization of a Pressurized Water
Reactor Assembly", {\sl Nucl.~Sci.~Eng.}, {\bf 115}, 129 (1993).

\bibitem{ALSB3}
A.~H\'ebert , ``Development of a Second Generation SPH Technique for the
Pin-by-Pin Homogenization of a Pressurized Water Reactor Assembly in Hexagonal
Geometry'', {\sl Trans. Am. Nucl. Soc.}, {\bf 71}, 253 (1994).

\bibitem{madrid2}
P. Blanc-Tranchant, A. Santamarina, G. Willermoz and A. H\'ebert, ``Definition and Validation of a 2-D Transport
Scheme for PWR Control Rod Clusters", paper presented at the {\sl Int. Conf. on Mathematics and Computation,
Reactor Physics and Environmental Analysis in Nuclear Applications}, Madrid, Spain, September 27--30, 1999.

\bibitem{Chambon2014}
R. Chambon, ``Specifications and User Guide for {\tt NAP:} module in DRAGON/DONJON VERSION5
(Pin Power Reconstruction module)," Report IGE-345,
\'Ecole Polytechnique de Montr\'eal,
 Institut de G\'enie Nucl\'eaire (2014).

\bibitem{sphedf}
T. Courau, M. Cometto, E. Girardi, D. Couyras and N. Schwartz, ``Elements of Validation of Pin-by-Pin Calculations with the Future EDF Calculation Scheme Based on APOLLO2 and COCAGNE Codes," Proceedings of ICAPP 08, Anaheim, CA USA, June 8--12, 2008.

\bibitem{recipie}
W.~H.~Press, B.~P.~Flannery, S.~A.~Teukolsky and W.~T.~Vetterling, ``Numerical
Recipes, Second Edition (FORTRAN Version)'', Cambridge University Press,
Cambridge (1994).

\bibitem{MRG1}
G.~Marleau, 
``Fine Mesh 3--D Collision Probability Calculations Using the Lattice Code DRAGON'',
\textsl{Int. Conf. on the Physics of Nuclear Science and Technology}, 
Long Island, New York, October 5--8, 1998,

\bibitem{MRG2}
G.~Marleau, ``New Geometric Capabilities of DRAGON'',
\textsl{Nineteenth Annual Conf. of the Canadian Nuclear Society}, 
Toronto, Ontario, October 18--21, 1998,

\bibitem{Kodeli2001a}
I. Kodeli, ``Multidimensional Deterministic Nuclear Data Sensitivity and Uncertainty Code System: Method and Application,"
{\sl Nucl.~Sci.~Eng.}, {\bf 138}, 45--66 (2001).

\bibitem{Bidaud2009a}
A. Bidaud, G. Marleau, and E. Nablat, ``Nuclear Data Uncertainty Analysis using the coupling of DRAGON with SUSD3D," M\&C 2009 , Saratoga Springs, NY (2009).

\bibitem{nestle}
P. J. Turinsky, R. M. K. Al-Chalabi, P. Engrand, H. N. Sarsour, F. X. Faure and  W. Guo, ``NESTLE: Few-group neutron diffusion equation solver utilzing the nodal expansion
method for eigenvalue, adjoint, fixed-source steady-state and transient problem,'' Electric Power Research Center, North Carolina State University, Raleigh, NC 27695-7909 (1994).

\bibitem{LLB}
S. Marguet, {\sl The Physics of Nuclear Reactors}, Springer, 2018.

\bibitem{Koebke}
K. Koebke, H. Haase, L. Hetzelt, and H.-J. Winter, ``Application and Verification of the Simplified Equivalence Theory
for Burnup States," {\sl Nucl. Sci. Eng.}, {\bf 92}, 56--65 (1986).

\bibitem{Frohlicher}
K. Fr\"{o}hlicher, V. Salino and A. H\'ebert, ``Investigating fission distribution behavior under various
homogenization techniques for asymmetrical fuel assemblies and different reflector equivalence methods,"
{\sl Ann. nucl. Energy}, {\bf 157}, 1--12 (2021).

\bibitem{PSPLOT}
K.E.~Kohler, 
\textsl{PostScript for Technical Drawings PSPLOT: A FORTRAN-Callable PostScript Plotting
Library User's Manual}, 
Technical Report Nova Southeastern University, Oceanographic
Center, 8000 North Ocean Drive, Dania, Florida; One can get a feel for the flavor of
PSPLOT at \url{http://www.nova.edu/ocean/} while access to the full psplot library is via
anonymous ftp: whitetip.ocean.nova.edu in the directory psplot. 

\bibitem{GHOSTVIEW}
T.O.~Theisen, 
\textsl{Ghostview: An X11 user interface for Ghostscript}, 
This program is free software under the term of the GNU general public licence as
published by the Free Software Fundation.

\bibitem{matlab}
MATLAB, {\sl The Language of Technical Computing}, {\tt www.mathworks.com} (2006). 

\bibitem{Plamondon2006}
C. Plamondon, {\sl V\'erification des lignes d'int\'egration et illustration des g\'eom\'etries DRAGON}, Technical Report 
IGE-290, \'Ecole Polytechnique de Montr\'eal (2006). 

\bibitem{Intech2011}
A. H\'ebert, {\sl Revisiting the Ceschino Interpolation Method}, in {\sl MATLAB -- A Ubiquitous Tool for the Practical
Engineer}, Clara M. Ionescu (Ed.), InTech Open Access Publisher, ISBN 978-953-307-907-3, Croatia, 2011.

\bibitem{chambon}
R. Chambon, {\sl Optimisation de la gestion du combustible dans les r\'eacteurs
CANDU refroidis \`a l'eau l\'eg\`ere}, Ph. D. Thesis, \'Ecole Polytechnique de Montr\'eal (2006). 

\bibitem{Mostel}
R.D.~Mosteller, L.D.~Eisenhart, R.C.~Little, W.J.~Eich and J.~Chao, ``Benchmark
Calculations for the Doppler Coefficient of Reactivity", {\sl Nucl.~Sci.~Eng.},
{\bf 107}, 265 (1991).

\bibitem{Stankovski}
\v{Z}.~Stankovski, ``Refinement of the Substructure Method for Integral
Transport Calculations", {\sl Nucl.~Sci.~Eng.}, {\bf 92}, 255 (1986).

\bibitem{Akroyd}
R.T.~Akroyd and N.S.~Riyait, ``Iteration and Extrapolation Methods for the
Approximate Solution of the Even-Parity Transport Equation for systems with
voids", {\sl Ann.~nucl.~Energy}, {\bf 16}, 1 (1989).

\bibitem{sissaoui}
M. T. Sissaoui, G. Marleau and D. Rozon, ``CANDU Reactor Simulations Using the
Feedback Model with Actinide Burnup History," {\sl Nucl. Technology}, {\bf 125},
197 (1999).

\bibitem{Kieffer}
C. Kieffer, ``Impl\'{e}mentation dans le code DRAGON d'un module de calcul de la masse volumique de l'eau l\'{e}g\`{e}re et lourde en fonction de la temp\'{e}rature et de la pression.," Technical Report IGE-315, \'Ecole Polytechnique de Montr\'eal (2009).

\bibitem{Freesteam}
J. Pye, "Freesteam", (2013),
  \url{http://freesteam.sourceforge.net/}.

\bibitem{McMaster}
B. Garland, "Heavy Water Properties at McMaster University", (2015), \url{http://www.nuceng.ca/d2o/d2ohome.htm}.

\bibitem{Marleau2001}
G. Marleau, ``DRAGON Theory Manual Part 1: Collision Probability Calculations," Technical Report IGE-236 
Rev. 1, \'Ecole Polytechnique de Montr\'eal (2001).

\bibitem{sphedf2}
P. Gu\'erin, T. Courau, D. Couyras and E. Girardi, ``\'Equivalence et correction de transport dans COCAGNE," Compte-Rendu CR-I23/2010/042, SINETICS, \'Electricit\'e de France, January 2011.

\bibitem{cns2015}
A.~H\'ebert, ``A Reformulation of the Transport-Transport SPH Equivalence Technique," paper presented at the
\textsl{7th International Conference on Modelling and Simulation in Nuclear Science and Engineering (7ICMSNSE)}, Ottawa, Canada, October 18--21, 2015.

\bibitem{autosecol}
M. Grandotto-Biettoli, ``AUTOSECOL, un calcul automatique de l'auto-protection
des r\'esonances des isotopes lourds," Note CEA-N-1961, Commissariat \`a
l'\'Energie Atomique, 1977. 

\bibitem{vdg}
M. Coste, H. Tellier, P. Ribon, C. Raepsaet and C. Van Der Gucht, ``New improvements in the self-shielding
formalism of the APOLLO2 code,"
{\sl Int. Top. Mtg. on mathematical methods and supercomputing in nuclear applications. M\&C$+$SNA},
Karlsruhe, Germany, April 19 -- 23, 1993.

\bibitem{salome}
A. Ribes and C. Caremoli, ``Salom\'e platform component model for numerical simulation,'' {\sl COMPSAC 07: Proceeding of the 31st Annual International Computer Software and Applications Conference}, pages 553-564, Washington, DC, USA, 2007, IEEE Computer Society.
See the website at \url{http://www.salome-platform.org}.

\bibitem{tdt}
R. Sanchez, ``TDT, an advanced Integral Transport Method," {\sl Annual Meeting of the American Nuclear Society}, San-Diego, California, June 20-24, 1993.

\bibitem{lyioussi}
N. Lyoussi-Charrat, ``Calcul du transport neutronique dans le code APOLLO2 par la m\'ethode des probabilit\'es de
collision dans une g\'eom\'etrie cart\'esienne g\'en\'erale," Th\`ese de doctorat, Universit\'e de Clermont-Ferrand 2, France,
Mars 1994.

\bibitem{salt}
X. Warin, ``Notice th\'eorique de la m\'ethode des caract\'eristiques 2D et du g\'en\'erateur de trajectoires SALT," Report IGE-329, \'Ecole Polytechnique de Montr\'eal, Mars 2002.

\bibitem{ane15b}
A. H\'ebert, ``DRAGON5 and DONJON5, the contribution of \'Ecole Polytechnique de Montr\'eal to the SALOME platform," {\sl Ann. nucl. Energy}, {\bf 87}, 12--20 (2016).

\bibitem{alamos}
D. Tomatis, F. Bidault, A. Bruneton and Z. Stankovski, ``Overview of SERMA's Graphical  User Interfaces for Lattice Transport Calculations,'' {\sl Energies}, vol. 15, no. 4, 2022.

\bibitem{Harrisson2011a}
 G.~Harrisson, and G.~Marleau, ``Modeling of a {3-D} {SCWR} unit cell,''
  {\sl 32nd Annual conference of the CNS}, Niagara Falls, ON (2011).

\bibitem{Laville}
C. Laville, {\sl \'Etude de diff\'erentes m\'ethodes de calculs de coefficients de
sensibilit\'es du keff aux donn\'ees nucl\'eaires}, Master Thesis, \'Ecole Polytechnique de Montr\'eal (2011).

\bibitem{SCALE}
``SCALE : A Modular Code System for Performing Standardized Computer Analyses for Licensing Evaluation",
Oak Ridge National Laboratory, ORNL/TM-2005/39, Version 6.1 (2011).

\bibitem{clio}
S. Perruchot-Triboulet and R. Sanchez, ``D\'ecomposition par m\'ethodes perturbatives de r\'eactivit\'e de deux syst\`emes,'' Note CEA-N-2817, Commissariat \`a l'\'Energie Atomique, France, F\'evrier 1997.

\bibitem{todorova}
G. Todorova, H. Nishi and J. Ishibashi, ``Method for Condensation of the Macroscopic Transport Cross-Sections for Criticality Analyses of FBR MONJU by the Code NSHEX", {\sl J. of Nucl. Sci. and Tech.},
{\bf 41}, No. 12, 1237 (2004).

\bibitem{condPn}
J.-F. Vidal, P. Archier, B. Faure, V. Jouault, J.-M. Palau, V. Pascal, G. Rimpault, F. Auffret, L. Graziano, E. Masiello {\sl et al.}, ``Apollo3
homogenization techniques for transport core calculations -- application to the Astrid CFV core," {\sl Nucl.~Eng.~and~Technology}, {\bf 49}, 1379 (2017).

\bibitem{serpent}
J. Lepp\"anen, ``Serpent, a Continuous-energy Monte Carlo Reactor Physics Burnup Calculation Code,'' VTT Technical Research Centre, Finland, March 2013.

\bibitem{hdf5}
The HDF Group, \url{https://www.hdfgroup.org}.

\bibitem{morel1996}
J.E. Morel, Leonard J. Lorence, Jr., Ronald P. Kensek, John A. Halbleib and D. P. Sloan, ``A Hybrid Multigroup/Continuous-Energy Monte Carlo Method for
Solving the Boltzmann-Fokker-Planck Equation,'' {\sl Nucl.~Eng.~and~Technology}, {\bf 124:3}, 369-389 (1996).

\bibitem{cygwin}
The home of the Cygwin project, \url{http://www.cygwin.com/}.

\bibitem{wsl}
Windows Subsystem for Linux, \url{https://learn.microsoft.com/en-us/windows/wsl/}.

\end{thebibliography}
