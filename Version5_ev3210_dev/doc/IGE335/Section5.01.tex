\subsection{Scattering cross sections}\label{sect:ExXSData}

In DRAGON, the angular dependence of the
scattering cross section is expressed in a Legendre series expansion of
the form:
  $$
\Sigma_{s}(\Omega\cdot\Omega')=\Sigma_{s}(\mu)=
\sum_{l=0}^{L}\left({{(2l+1)}\over{4\pi}}\right)\Sigma_{s,l}P_{l}(\mu).
  $$
Since the Legendre polynomials satisfy the following
orthogonality conditions:
  $$
\int_{-1}^{1} d\mu P_{l}(\mu)P_{m}(\mu) =
\left({{2\delta_{l,m}}\over{(2l+1)}}\right),
  $$
we will have
  $$
\Sigma_{s,l}=\int_{-1}^{1}d\mu\int_{0}^{2\pi}d\varphi\Sigma_{s}(\mu)P_{l}(\mu)=
2\pi \int_{-1}^{1}d\mu\Sigma_{s}(\mu)P_{l}(\mu).
  $$

Let us now consider the following three-group (\dusa{ngroup}=3) isotropic and
linearly anisotropic scattering cross sections ($L$=\dusa{naniso}=2) given by:

\begin{center}
\begin{tabular}{|llccc|}\hline\hline
$l$ & $g$ &  $\Sigma_{s,l}^{g\to 1}$ (\xsunit)
          &  $\Sigma_{s,l}^{g\to 2}$ (\xsunit)
          &  $\Sigma_{s,l}^{g\to 3}$ (\xsunit) \\ \hline
    &  1  & 0.90 & 0.80 & 0.00 \\
0   &  2  & 0.00 & 0.70 & 0.60 \\
    &  3  & 0.00 & 0.30 & 0.40 \\ \hline
    &  1  & 0.09 & 0.05 & 0.08 \\
1   &  2  & 0.00 & 0.07 & 0.06\\
    &  3  & 0.03 & 0.00 & 0.04 \\ \hline\hline
\end{tabular}
\end{center}

\noindent
In DRAGON this scattering cross section must be entered as

\begin{verbatim}
SCAT  (* L=0 *) 1 1 (* 3->1 *)      (* 2->1 *)      (* 1->1 *) 0.90
                3 3 (* 3->2 *) 0.30 (* 2->2 *) 0.70 (* 1->2 *) 0.80 
                2 3 (* 3->3 *) 0.40 (* 2->3 *) 0.60 (* 1->3 *)
SCAT  (* L=1 *) 3 3 (* 3->1 *) 0.03 (* 2->1 *) 0.00 (* 1->1 *) 0.09
                2 2 (* 3->2 *)      (* 2->2 *) 0.07 (* 1->2 *) 0.05 
                3 3 (* 3->3 *) 0.04 (* 2->3 *) 0.06 (* 1->3 *) 0.08 
\end{verbatim}

