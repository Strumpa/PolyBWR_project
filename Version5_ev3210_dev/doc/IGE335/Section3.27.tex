\subsection{The {\tt SENS:} module}\label{sect:SENSData}

This module is used to perform an explicit sensitivity analysis of keff to nuclear data represented by the cross sections.\cite{Laville}
The calculations are performed using adjoint-based first-order-linear perturbation theory and require the adjoint flux (see \Sect{FLUData}).
The sensitivity coefficients are stored in a \textit{SDF} text file that is compatible with the \moc{JAVAPENO} module of SCALE\cite{SCALE}
(this compatibility is achieved via a slight modification of the \textit{rdragon} execution script).
An example of modification is presented in the file \moc{sens.save} from the \textit{non regression testcase} \moc{sens.x2m}.

\vskip 0.02cm

The calling specifications are:

\begin{DataStructure}{Structure \dstr{SENS:}}
\dusa{SENS.sdf}~\moc{:=}~\moc{SENS:}~\dusa{FLUNAM}~\dusa{ADJ$\_$FLUNAM}~\dusa{TRKNAM}~\dusa{MACRO}~\moc{::}~\dstr{SENS\_data} \\
\end{DataStructure}

\noindent where
\begin{ListeDeDescription}{mmmmmmm}

\item[\dusa{SENS.sdf}] {\tt character*12} name of a {\sc SDF} file object that is created by {\tt SENS:}.

\item[\dusa{FLUNAM}] {\tt character*12} name of the required {\sc flux} (type {\tt L\_FLUX}) object open in read-only mode.

\item[\dusa{ADJ$\_$FLUNAM}] {\tt character*12} name of the required {\sc adjoint flux} (type {\tt L\_FLUX}) object open in read-only mode.

\item[\dusa{TRKNAM}] {\tt character*12} name of the required {\sc tracking} (type {\tt L\_TRACK}) object open in read-only mode.

\item[\dusa{MACRO}] {\tt character*12} name of the required {\sc macrolib} (type {\tt L\_MACROLIB}) object open in read-only mode.

\item[\dusa{SENS\_data}] input data structure containing specific data (see \Sect{descSENS}).

\end{ListeDeDescription}

\subsubsection{Data input for module {\tt SENS:}}\label{sect:descSENS}

\vskip -0.5cm

\begin{DataStructure}{Structure \dstr{SENS\_data}}
$[$~\moc{EDIT} \dusa{iprint}~$]$ \\
$[$~\moc{ANIS} \dusa{nanis}~$]$ \\
\moc{;}
\end{DataStructure}

\noindent where
\begin{ListeDeDescription}{mmmmmmmm}

\item[\moc{EDIT}] keyword used to set \dusa{iprint}.

\item[\dusa{iprint}] index used to control the printing in module {\tt SENS:}. =0 for no print; =1 for minimum printing (default value).

\item[\moc{ANIS}] keyword used to specify the level \dusa{naniso} of anisotropy permitted in the calculation.

\item[\dusa{nanis}] number of Legendre orders for the representation of the scattering cross sections and the anisotropy of the flux. The default value is \dusa{nanis}=1 corresponding to the use of isotropic scattering cross sections and integrated flux. The number of Legendre orders used for the sensitivity calculations is the lowest between \dusa{nanis} and the level of anisotropy available in the \dusa{MACRO} data.

\end{ListeDeDescription}

\eject
