\subsection{The {\tt FLU:} module}\label{sect:FLUData}

The \moc{FLU:} module is used to solve the linear system of multigroup collision
probability or response matrix equations in DRAGON. Different types of solution are
available, such as fixed source problem, fixed source eigenvalue problem (GPT type) or
different types of eigenvalue problems. The calling specifications are:

\begin{DataStructure}{Structure \dstr{FLU:}}
\dusa{FLUNAM} \moc{:=} \moc{FLU:} $[~\{$ \dusa{FLUNAM} $|$ \dusa{FLUDSA} $\}~]$ \dusa{PIJNAM} 
\dusa{LIBNAM}  \dusa{TRKNAM} $[$ \dusa{TRKFIL} $]$ \\
$~~~~[~\{$ \dusa{TRKFLP} \dusa{TRKGPT} $|$ \dusa{SOUNAM} $\}~]$ \moc{::} \dstr{descflu}
\end{DataStructure}

\noindent where
\begin{ListeDeDescription}{mmmmmmmm}

\item[\dusa{FLUNAM}] {\tt character*12} name of the \dds{fluxunk} data structure
containing the solution ({\tt L\_FLUX} signature). If \dusa{FLUNAM} appears on
the RHS, the solution previously stored in \dusa{FLUNAM} (flux and buckling) is used to initialize
the new iterative process; otherwise, a uniform unknown vector and a zero buckling
are used.

\item[\dusa{FLUDSA}] {\tt character*12} name of the \dds{fluxunk} data structure
containing an initial approximation of the solution ({\tt L\_FLUX} signature). This solution
corresponds to a DSA-type simplified
calculation compatible with \dusa{FLUNAM}. This option is only available with a \moc{SNT:} tracking.

\item[\dusa{PIJNAM}] {\tt character*12} name of the \dds{asmpij} data
structure containing the group-dependent system
matrices ({\tt L\_PIJ} signature, see \Sect{ASMData}).

\item[\dusa{LIBNAM}] {\tt character*12} name of the \dds{macrolib} or \dds{microlib} data structure that contains the
macroscopic cross sections ({\tt L\_MACROLIB} or {\tt L\_LIBRARY} signature, see \Sectand{MACData}{LIBData}).
Module {\tt FLU:} is performing a {\sl direct} or {\sl adjoint} calculation, depending if the adjoint flag
is set to {\tt .false.} or {\tt .true.} in the {\tt STATE-VECTOR} record of the \dds{macrolib}.

\item[\dusa{TRKNAM}] {\tt character*12} name of the \dds{tracking} data
structure containing the tracking ({\tt L\_TRACK} signature, see \Sect{TRKData}).

\item[\dusa{TRKFIL}] {\tt character*12} name of the sequential binary tracking
file used to store the tracks lengths. This file is given if and only if it was
required in the previous tracking module call (see \Sect{TRKData}).

\item[\dusa{TRKFLP}] {\tt character*12} name of the \dds{fluxunk} data structure containing the
unperturbed flux used to decontaminate the GPT solution ({\tt L\_FLUX} signature). This object is
mandatory if and only if ``{\tt TYPE P}" is selected.

\item[\dusa{TRKGPT}] {\tt character*12} name of the \dds{source} data structure
containing the GPT fixed sources ({\tt L\_SOURCE} signature). This object is
mandatory if and only if ``{\tt TYPE P}" is selected.

\item[\dusa{SOUNAM}] {\tt character*12} name of the \dds{source} data structure
containing the fixed sources ({\tt L\_SOURCE} signature) used for a ``{\tt TYPE S}" calculation.
By default, piecewise-constant fixed sources available in the \dds{macrolib} (or \dds{microlib}) \dusa{LIBNAM}
are used.

\item[\dstr{descflu}] structure containing the input data to this module (see
\Sect{descflu}).

\end{ListeDeDescription}

\clearpage

\subsubsection{Data input for module {\tt FLU:}}\label{sect:descflu}

\begin{DataStructure}{Structure \dstr{descflu}}
$[$ \moc{EDIT} \dusa{iprint} $]$ \\
$[$ \moc{INIT} $\{$ \moc{OFF} $|$ \moc{ON} $|$ \moc{DSA} $\}~]$ \\
\moc{TYPE} $\{$ \moc{N} $|$ \moc{S} $|$ \moc{F} $|$ \moc{P} $|$ \moc{K} $[$ \dstr{descleak} $]$ $|$
$\{$\moc{B} $|$ \moc{L} $\}$ \dstr{descleak} $\}$ $]$  \\
$[$ \moc{EXTE} $[$ \dusa{maxout} $]~~[$ \dusa{epsout} $]~]$ \\
$[$ \moc{THER} $[$ \dusa{maxthr} $]~~[$ \dusa{epsthr} $]~]~~[$ \moc{REBA} $[$ \moc{OFF} $]~]$ \\
$[$ \moc{UNKT} $[$ \dusa{epsunk} $]~]$ \\
$[$ \moc{ACCE} \dusa{nlibre} \dusa{naccel}  $]$ \\
{\tt ;}
\end{DataStructure}

\goodbreak
\noindent where
\begin{ListeDeDescription}{mmmmmmm}

\item[\moc{EDIT}] keyword used to modify the print level \dusa{iprint}.

\item[\dusa{iprint}] index used to control the printing of this operator. The
amount of output produced by this operator will vary substantially
depending on the print level specified. 

\item[\moc{OFF}] keyword to specify that the neutron flux
is to be initialized with a flat distribution (default option).

\item[\moc{ON}] keyword to specify that the initial neutron flux distribution
is to be recovered from \dusa{FLUNAM} if present in the RHS arguments.

\item[\moc{DSA}] keyword to specify that the initial neutron flux distribution
is to be recovered from the DSA compatible data structure \dusa{FLUDSA} if present in the RHS arguments.
This option is only available with a \moc{SNT:} tracking.

\item[\moc{TYPE}] keyword to specify the type of solution used in the flux
operator.

\item[\moc{N}] keyword to specify that no flux calculation is to be performed.
This option is usually activated when one simply wishes to initialize the
neutron flux distribution and to store this information in \dusa{FLUNAM}.

\item[\moc{S}]  keyword to specify that a fixed source problem is to be
treated. Such problem can also include fission source contributions.

\item[\moc{F}] keyword to specify that a 1D Fourier analysis calculation in $S_n$ is to be treated. This is similar to a fixed source problem, but the calculation stopped early to compute an L2 error norm in the flux. This yields a numerical estimate of the eigenvalue for the scattering source equation.

\item[\moc{P}]  keyword to specify that a fixed source eigenvalue problem (GPT type) is to be
treated. Such problem includes fission source contributions in addition of GPT sources.

\item[\moc{K}] keyword to specify that a fission source eigenvalue problem is
to be treated. The eigenvalue is then the effective multiplication factor with a
fixed buckling. In this case, the fixed sources, if any is present on the
\dds{macrolib} or \dds{microlib} data structure, are not used.  

\item[\moc{B}] keyword to specify that a fission source eigenvalue problem is
to be treated. The eigenvalue in this case is the critical buckling with a fixed
effective multiplication factor. The buckling eigenvalue has meaning only in the
case of a cell without leakages (see the structure \dstr{descBC} in
\Sect{descBC}). It is also possible to use an open geometry with
\moc{VOID} boundary conditions  provided it is closed by the \moc{ASM:} module
(see \Sect{descasm}) using the keywords \moc{NORM} or \moc{ALSB}.

\item[\moc{L}] keyword to specify that a non-multiplicative medium eigenvalue
problem is to be treated. The eigenvalue in this case is the critical buckling
with vanishing fission cross sections. The buckling eigenvalue has meaning only in the
case of a cell without leakages (see the structure \dstr{descBC} in
\Sect{descBC}). It is also possible to use an open geometry with
\moc{VOID} boundary conditions  provided it is closed by the \moc{ASM:} module
(see \Sect{descasm}) using the keywords \moc{NORM} or \moc{ALSB}.

\item[\dstr{descleak}] structure describing the general leakage parameters
options (see \Sect{descleak}). This information is mandatory for producing the
diffusion coefficients.

\item[\moc{EXTE}] keyword to specify that the control parameters for the
external iteration are to be modified. 

\item[\dusa{maxout}] maximum number of external iterations. The fixed default
value for a case with no leakage model is \dusa{maxout}=$2\times n_{f}-1$ where
$n_{f}$ is the number of regions containing fuel. The fixed default value for a
case with a leakage model is \dusa{maxout}=$10\times n_{f}-1$.

\item[\dusa{epsout}] convergence criterion for the external iterations. The
fixed default value is \dusa{epsout}=$5.0\times 10^{-5}$.

\item[\moc{THER}] keyword to specify that the control parameters for the
thermal iterations are to be modified.

\item[\dusa{maxthr}] maximum number of thermal iterations. The fixed default
value is \dusa{maxthr}=2$\times$\dusa{ngroup}-1 (using scattering modified CP)
or \dusa{maxthr}=4$\times$\dusa{ngroup}-1 (using standard CP).

\item[\dusa{epsthr}] convergence criterion for the thermal iterations. The
fixed default value is \dusa{epsthr}=$5.0\times 10^{-5}$.

\item[\moc{UNKT}] keyword to specify the flux error tolerance in
the outer iteration.

\item[\dusa{epsunk}] convergence criterion for flux components in the outer
iteration. The fixed default value is \dusa{epsunk}=\dusa{epsthr}.

\item[\moc{REBA}] keyword used to specify that the flux rebalancing option is
to be turned on or off in the thermal iteration. By default (floating default)
the flux rebalancing option is initially activated. This keyword is required to
toggle between the on and off position of the flux rebalancing option. 

\item[\moc{OFF}] keyword used to deactivate the flux rebalancing option. When
this keyword is absent the flux rebalancing option is reactivated.

\item[\moc{ACCE}] keyword used to modify the variational acceleration
parameters. This option is active by default (floating default) with
\dusa{nlibre}=3 free iterations followed by \dusa{naccel}=3 accelerated
iterations. 

\item[\dusa{nlibre}] number of free iterations per cycle of
\dusa{nlibre}+\dusa{naccel} iterations. 

\item[\dusa{naccel}] number of accelerated iterations per cycle of
\dusa{nlibre}+\dusa{naccel} iterations. Variational acceleration may be
deactivated by using \dusa{naccel}=0.

\end{ListeDeDescription}
\clearpage

\subsubsection{Leakage model specification structure}\label{sect:descleak}

Without leakage model, the multigroup flux $\vec\phi_g$ of the collision
probability method is obtained from equation

\begin{equation}
\vec\phi_g={\bf W}_g \vec Q^{*}_g
\label{eq:eq3.64}
\end{equation}

\noindent where ${\bf W}_g$ is the scattering reduced collision probability matrix
and $ Q^{*}_g$ is the fission and out-of-group scattering source. This equation is
modified by the leakage model. The leakage models \moc{PNLR}, \moc{PNL}, \moc{SIGS}
(default model) and \moc{ECCO} can also be used with solutions techniques other than
the collision probability method.

\vskip 0.2cm

A leakage model can be set in {\sl fundamental mode condition} if all boundary conditions are
conservative (such as \moc{REFL}, \moc{SYME}, \moc{SSYM}, \moc{DIAG}, \moc{ALBE 1.0}). In this
case, the \dstr{descleak} structure allows the following information to be specified:

\begin{DataStructure}{Structure \dstr{descleak} in fundamental mode condition}
$\{$ \moc{LKRD} $|$ \moc{RHS} $|$ \moc{P0} $|$ \moc{P1} $|$ \moc{P0TR} $|$ \moc{B0} $|$ \moc{B1} $|$ \moc{B0TR} $\}$ \\
$\{$ \moc{PNLR} $|$ \moc{PNL} $|$ \moc{SIGS} $|$ \moc{ALBS} $|$ \moc{ECCO} $|$ \moc{HETE}
$[$ $\{$ \moc{G} $|$ \moc{R} $|$ \moc{Z} $|$ \moc{X} $|$ \moc{Y} $\}$ $]$ $\}$ \\
$[$ $\{$ \moc{BUCK} $\{$ \dusa{valb2} $|$ $[$ \moc{G} \dusa{valb2} $]$  
$[$ \moc{R} \dusa{valbr2} $]$ $[$ \moc{Z} \dusa{valbz2} $]$ 
$[$ \moc{X} \dusa{valbx2} $]$
$[$ \moc{Y} \dusa{valby2} $]$ $\}$
$|$ \moc{KEFF} \dusa{valk} $|$ \moc{IDEM} $\}$
$]$  \end{DataStructure}

If a boundary condition is non-conservative (such as \moc{VOID}), it is nevertheless possible to set a
simplified leakage model based on the Todorova approximation with option \moc{SIGS}. Such a model is
usefull to represent leakage in the axial direction. The Todorova leakage model available in module
\moc{FLU:} is an homogeneous model assuming uniform leakage across the complete domain in each
energy group. An heterogeneous model with more leakage zones is also available.

\vskip 0.08cm

To activate the Todorova approximation, the \dstr{descleak} structure takes the form:

\begin{DataStructure}{Structure \dstr{descleak} in non-fundamental mode condition}
$\{$ \moc{LKRD} $|$ \moc{RHS} $|$ \moc{P0} $|$ \moc{P1} $|$ \moc{P0TR} $\}$ \\
$[$ \moc{MERG} (\dusa{imergl}(ii),ii=1,nbmix) $]$ \\
$[$ \moc{BUCK} \dusa{valb2} $]$
\end{DataStructure}

\begin{ListeDeDescription}{mmmmmmm}

\item[\moc{LKRD}] keyword used to specify that the leakage coefficients are
recovered from data structure named \dusa{LIBNAM}. The \moc{LKRD} option is not
available with the \moc{ECCO} and \moc{HETE} leakage models.

\item[\moc{RHS}] keyword used to specify that the leakage coefficients are
recovered from RHS flux data structure named \dusa{FLUNAM}. The \moc{RHS} option is not
available with the \moc{ECCO} and \moc{HETE} leakage models. If the flux calculation is
an adjoint calculation, the energy group ordering of the leakage coefficients is permuted.

\item[\moc{P0}] keyword used to specify that the leakage coefficients are
calculated using a $P_0$ model.

\item[\moc{P1}] keyword used to specify that the leakage coefficients are
calculated using a $P_1$ model. 

\item[\moc{P0TR}] keyword used to specify that the leakage coefficients are
calculated using a $P_0$ model with transport correction.

\item[\moc{B0}] keyword used to specify that the leakage coefficients are
calculated using a $B_0$ model. This is the default value when a buckling
calculation is required (\moc{B}).

\item[\moc{B1}] keyword used to specify that the leakage coefficients are
calculated using a $B_1$ model.

\item[\moc{B0TR}] keyword used to specify that the leakage coefficients are
calculated using a $B_0$ model with transport correction.

\item[\moc{PNLR}] keyword used to specify that the elements of the scattering
modified collision probability matrix
are multiplied by the adequate non-leakage homogeneous buckling dependent
factor.\cite{ALSB1}. The non-leakage
factor $P_{{\rm NLR},g}$ is defined as

\begin{equation}
P_{{\rm NLR},g}={\bar\Sigma_g-\bar\Sigma_{{\rm s0},g \gets g}\over{\bar\Sigma_g-\bar\Sigma_{{\rm s0},g \gets g}+d_g(B) \ B^2}}
\end{equation}

\noindent where transport-corrected total
cross sections are used to compute the ${\bf W}_g$ matrix. $\bar\Sigma_{{\rm s0},g \gets g}$ is the average
transport-corrected macroscopic within-group scattering cross section in group $g$,
homogenized over the lattice and transport corrected. \eq(eq3.64) is then replaced by
 
\begin{equation}
\vec\phi_g=P_{{\rm NLR},g} {\bf W}_g \vec Q^*_g \ \ \ .
\label{eq:eq5.36}
\end{equation}

\item[\moc{PNL}] keyword used to specify that the elements of the collision
probability matrix are multiplied by the adequate non-leakage homogeneous buckling
dependent factor.\cite{ALSB1}. The non-leakage factor $P_{{\rm NL},g}$ is defined as

\begin{equation}
P_{{\rm NL},g}={\bar\Sigma_g\over{\bar\Sigma_g+d_g(B) \ B^2}}
\end{equation}

\noindent where $\bar\Sigma_g$ is the average transport-corrected macroscopic total cross section
in group $g$, homogenized over the lattice and transport corrected. \eq(eq3.64) is then replaced by

\begin{equation}
\vec\phi_g={\bf W}_g \left[ P_{{\rm NL},g} \vec Q^*_g -(1-P_{{\rm NL},g}) {\bf \Sigma}_{{\rm s0},g\gets g} \ \vec\phi_g \right]
\label{eq:eq5.33b}
\end{equation}

\noindent where ${\bf \Sigma}_{{\rm s0},g\gets g}={\rm diag} \{ \Sigma_{{\rm s0},i,g \gets g}\> ;\> \forall i \}$
and the total cross sections used to compute the ${\bf W}_g$ matrix are also
transport-corrected.

\vskip 0.02cm

\noindent It is important to note that that the \moc{PNLR} option reduces to the \moc{PNL} option in
cases where no scattering reduction is performed. Scattering reduction can be avoided in module
\moc{ASM:} by setting {\tt PIJ SKIP} (See \Sect{descasm}).

\item[\moc{SIGS}] keyword used to specify that an homogeneous buckling
correction is to be applied on the diffusion cross section ($\Sigma_{s} -
dB^{2}$). \eq(eq3.64) is then replaced by

\begin{equation}
\vec\phi_g={\bf W}_g\left[ \vec Q^{*}_g-d_g(B) \ B^2 \ \vec\phi_g\right]
\label{eq:eq5.33}
\end{equation}

\noindent where transport-corrected total
cross sections are used to compute the ${\bf W}_g$ matrix. This is the so called
{\sl DIFFON method} used in the APOLLO-family of thermal lattice codes. The \moc{SIGS} option is
the default option when a buckling calculation is required (\moc{TYPE B} or \moc{TYPE L}) or a
fission source eigenvalue problem (\moc{TYPE K}) with imposed buckling is considered.

\item[\moc{ALBS}] keyword used to specify that an homogeneous buckling
contribution is introduced by a group dependent correction of the
albedo.\cite{ALSB2} This leakage model is restricted to the collision probability
method. It is then necessary to define the geometry with an
external boundary condition of type \moc{VOID} (see \Sect{descBC}) and to close
the region in module \moc{ASM:} using the \moc{ALBS} option (see
\Sect{descasm}). \eq(eq3.64) is then replaced by

\begin{equation}
\vec\phi_g={\bf W}_g \ \vec Q^{*}_g-\left[ {\bf I}+{\bf W}_g{\bf \Sigma}_{{\rm s0},g\gets g}\right] d_g(B) \ B^2 
\ \gamma \ {\bf P}_{{\rm iS},g}
\label{eq:eq5.34}
\end{equation}

\noindent where ${\bf P}_{{\rm iS},g}=\{P_{{\rm iS},g} \ ; \ i=1,I \}$ is the array of escape
probabilities in the open geometry and where

\begin{equation}
\gamma={\sum\limits_j V_j \phi_{j,g} \over \sum\limits_j V_j \phi_{j,g} P_{{\rm jS},g}} \ \ \ .
\label{eq:eq5.35}
\end{equation}

\item[\moc{ECCO}] keyword used to perform an ECCO--type leakage
calculation taking into account isotropic streaming effects. This method
introduces an heterogeneous buckling contribution as a group dependent correction
to the source term.\cite{ecco,rimpault} It is then necessary to set the keyword \moc{ECCO}
in module \moc{ASM:} (see \Sect{descasm}). In the $P_1$ non--consistent case,
\eq(eq3.64) is then replaced by

\vskip -0.3cm

\begin{eqnarray}
\vec\varphi_g&=& {\bf W}_g \left(\vec Q^{*}_g - B^2 \ {i\vec{\cal J}_g\over B}\right)
\label{eq:eq5.37flux} \\
{i\vec{\cal J}_g\over B} &=& {\bf X}_g \left[{1 \over 3}
\ \vec\varphi_g + \sum_{h\not= g} {\bf \Sigma}_{{\rm s1},g \gets h} \
{i\vec{\cal J}_h\over B} \right]
\label{eq:eq5.37cour}
\end{eqnarray}

\noindent where $i\vec{\cal J}_{j,g}/B$ is the multigroup fundamental current, ${\bf \Sigma}_{{\rm s1},g \gets h}={\rm diag}\{ \Sigma_{{\rm s1},i,g \gets h}\> ;\> \forall i \}$ and where

\begin{equation}
{\bf X}_g=[{\bf I}-{\bf p}_g \ {\bf\Sigma}_{{\rm s}1,g\gets g}]^{-1} {\bf p}_g \ \ \ .
\label{eq:eq5.37ter}
\end{equation}

\item[\moc{HETE}] keyword used to perform a TIB\`ERE--type leakage
calculation taking into account anisotropic streaming effects. This method
introduces an heterogeneous buckling contribution as a group dependent correction
to the source term.\cite{PIJK0,PIJK} The heterogeneous buckling contribution is
introduced in the $B_n$ model using directional collision probabilities (PIJK method).
It is then necessary to set the keyword
\moc{PIJK} in module \moc{ASM:} (see \Sect{descasm}).

\item[\moc{G}] keyword used to specify that the buckling search will assume
all directional buckling to be identical (floating default option).

\item[\moc{R}] keyword used to specify that a radial buckling search will be
considered assuming an imposed $z$-direction buckling.

\item[\moc{Z}] keyword used to specify that a $z$-direction buckling search
will be considered  assuming an imposed $x$-direction and $y$-direction
buckling.

\item[\moc{X}] keyword used to specify that a $x$-direction buckling search
will be considered  assuming an imposed $y$-direction and $z$-direction
buckling.

\item[\moc{Y}] keyword used to specify that a $y$-direction buckling search
will be considered  assuming an imposed $x$-direction and $z$-direction
buckling.

\item[\moc{BUCK}] keyword used to specify the initial (for a buckling
eigenvalue problem) or fixed (for a effective multiplication constant eigenvalue
problem) buckling. 

\item[\moc{G}] keyword used to specify that the buckling in the $x$-direction,
$y$-direction and $z$-direction are to be initialized to \dusa{valb2}/3
(floating default).

\item[\moc{R}] keyword used to specify that the buckling in the $x$-direction,
and $y$-direction are to be initialized to \dusa{valbr2}/2.

\item[\moc{Z}] keyword used to specify that the buckling in the $z$-direction,
is to be initialized to \dusa{valbz2}.

\item[\moc{X}] keyword used to specify that the buckling in the $x$-direction,
is to be initialized to \dusa{valbx2}.

\item[\moc{Y}] keyword used to specify that the buckling in the $y$-direction,
is to be initialized to \dusa{valby2}.

\item[\dusa{valb2}] value of the fixed or initial total buckling in $cm^{-2}$.
The floating default value is
$${\it valb2}={\it valbx2}+{\it valby2}+{\it valbz2}.$$

\item[\dusa{valbr2}] value of the fixed or initial radial buckling in
$cm^{-2}$. The floating default value is
$${\it valbr2}={\it valbx2}+{\it valby2}.$$

\item[\dusa{valbz2}] value of the fixed or initial $z$-direction buckling in
$cm^{-2}$. The floating default value is \dusa{valbz2}=0.0 $cm^{-2}$. If
\dusa{valb2} is specified then \dusa{valbz2}=\dusa{valb2}/3.

\item[\dusa{valbx2}] value of the fixed or initial $z$-direction buckling in
$cm^{-2}$. The floating default value is \dusa{valbx2}=0.0 $cm^{-2}$. If
\dusa{valb2} is specified then \dusa{valbx2}=\dusa{valb2}/3. If \dusa{valbr2} is
specified then \dusa{valbx2}=\dusa{valbr2}/2.

\item[\dusa{valby2}] value of the fixed or initial $z$-direction buckling in
$cm^{-2}$. The floating default value is \dusa{valby2}=0.0 $cm^{-2}$. If
\dusa{valb2} is specified then \dusa{valby2}=\dusa{valb2}/3. If \dusa{valbr2} is
specified then \dusa{valby2}=\dusa{valbr2}/2.

\item[\moc{KEFF}] keyword used to specify the fixed (for a buckling eigenvalue
problem) effective multiplication constant. 

\item[\dusa{valk}] value of the fixed effective multiplication constant. The
fixed default value is \dusa{valk}=1.0.

\item[\moc{IDEM}] keyword used to specify that the initial (for a buckling
eigenvalue problem) or fixed (for a effective multiplication constant eigenvalue
problem) buckling is to be read from the data structure \dusa{LIBNAM}. 

\item[\moc{MERG}] keyword to specify leakage zones. A leakage zone is a set of material
mixtures where the diffusion coefficient is forced to be uniform in each energy group.
Using an heterogeneous leakage model is incompatible with options \moc{PNLR} and \moc{PNL}.
By default, an homogeneous leakage model with option \moc{SIGS} is imposed where the leakage
coefficients are all equal to the same value in each energy group.  

\item[\dusa{imergl}] array of homogenized leakage zone indices to which are
associated the material mixtures. \dusa{nbmix} is the total number of material mixtures.

\end{ListeDeDescription}
\eject
