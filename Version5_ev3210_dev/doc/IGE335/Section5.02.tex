\subsection{Geometries}\label{sect:ExGEOData}

In order to illustrate the use of the various geometries presented in
\Sect{GEOData}, lets us consider a few examples that can be treated
by DRAGON.

\begin{itemize}

\item 1--D Slab geometry (see \Fig{plaque}):

\begin{figure}[h!] 
\begin{center} 
\epsfxsize=11cm \centerline{ \epsffile{Gplaque.eps}}
\parbox{14cm}{\caption{Slab geometry with mesh-splitting}
\label{fig:plaque}}    
\end{center}  
\end{figure}

This geometry can be analyzed using a \moc{SYBILT:} tracking
modules:

\begin{verbatim}
PLATE := GEO: :: CAR1D 6
    X- VOID X+ ALBE 1.2
    MESHX  0.0 0.1 0.3 0.5 0.6 0.8 1.0
    SPLITX     2   2   2   1   2   1
    MIX        1   2   3   4   5   6 ;
\end{verbatim}
                                                                
\item 2--D Cartesian geometry containing micro-structures (see
figure \Fig{grains}):

\begin{figure}[h!] 
\begin{center} 
\epsfxsize=10cm \centerline{ \epsffile{Ggrains.eps}}
\parbox{14cm}{\caption{Two-dimensional Cartesian assembly containing
micro-structures} \label{fig:grains}}    
\end{center}  
\end{figure}

This geometry can be analyzed only using \moc{SYBILT:}
tracking modules:

\begin{verbatim}
CARNSG := GEO: :: CAR2D 3 3 
  X- DIAG X+ REFL Y- SYME Y+ DIAG
  MIX   C1  C1  C2
            C3  C2
                C3 
  BIHET SPHE (*NG=*) 2 (* NMILG= *) 2 (* SPHERICAL MICRO-STRUCTURE *)
  (* NS= *) 3 3
  (* M-S-1 *) 0.0 0.1 0.2 0.3 (* M-S 2 *) 0.0 0.2 0.4 0.5
  (* COMPOSITE MIXTURES *)  4 5
  (* MIXTURES SURROUNDING M-S *) 1 1
  (* COMPOSITE MIXTURE 4 FRACT *) 0.4 0.0
  (* REAL MIXTURE CONTENT M-S-1 *) 3 1 3
  (* COMPOSITE MIXTURE 5 FRACT *) 0.2 0.1
  (* REAL MIXTURE CONTENT M-S-1 *) 1 2 1
  (* REAL MIXTURE CONTENT M-S-2 *) 2 3 1
  ::: C1 := GEO: CAR2D 1 1 (* HOMOGENEOUS CELL WITH M-S *)
    MESHX 0.0 1.45 MESHY 0.0 1.45 MIX 4 ;
  ::: C2 := GEO: C1 (* HOMOGENEOUS CELL WITHOUT M-S *)
    MIX 1 ;
  ::: C3 := GEO: CARCEL 2 (* CELL WITH M-S TUBE *)
    MESHX 0.0 1.45 MESHY 0.0 1.45
    RADIUS 0.0 0.6 0.7
    MIX 5 2 1 ;
;
\end{verbatim}


\item Cylindrical and Cartesian cluster geometry (see \Fig{grappe}):

\begin{figure}[h!] 
\begin{center} 
\epsfxsize=10cm \centerline{ \epsffile{Ggrappe.eps}}
\parbox{14cm}{\caption{Cylindrical cluster geometry}\label{fig:grappe}}    
\end{center}  
\end{figure}

The first two geometry, namely {\tt ANNPIN} and {\tt CARPIN} can be analyzed
using a \moc{EXCELT:} tracking modules since the pins in
the clusters are all located between annular region. For the last two geometries,
{\tt ANNSPIN} and {\tt CARSPIN}, which are based on {\tt ANNPIN} and {\tt
CARPIN} respectively, they only be treated by the
\moc{EXCELT:} tracking modules since the pins in the clusters intersect the
annular regions defined by the \moc{SPLITR} option. This later option which was
selected to ensure a uniform thickness of 0.25 cm for each the annular region in
the final geometries.

\begin{verbatim}
ANNPIN := GEO: :: TUBE 3
  R+ REFL  RADIUS 0.0 0.75 2.75 4.75
  MIX 2 1 3 
  CLUSTER C1 C2 
  ::: C1 := GEO: TUBE 2
    MIX 2 4 RADIUS 0.0 0.3 0.6
    NPIN 4  RPIN 1.75  APIN 0.523599 ;
  ::: C2 := GEO: C1
    NPIN 2  RPIN 3.75  APIN 1.570796 ;
;
CARPIN := GEO: :: CARCEL 3
  X- REFL X+ REFL Y- REFL  Y+ REFL
  MESHX 0.0 10.0 MESHY -5.0 5.0
  RADIUS 0.0 0.75 2.75 4.75
  MIX 2 1 3 3
  CLUSTER C1 C2 
  ::: C1 := GEO: TUBE 2
    MIX 2 4 RADIUS 0.0 0.3 0.6
    NPIN 4  RPIN 1.75  APIN 0.523599 ;
  ::: C2 := GEO: C1
    NPIN 2  RPIN 3.75  APIN 1.570796 ;
;
ANNSPIN := GEO: ANNPIN :: 
  SPLITR   3 8 8 ;
CARSPIN := GEO: CARPIN :: 
  SPLITR   3 8 8 ;
\end{verbatim}

Note that even if \moc{MESHX} and \moc{MESHY} differ in {\tt CARPIN}, the
annular regions and pins will still be localized with respect to the center of
the cell located at $(x,y)=(5.0,0.0)$ cm.


\item 2--D hexagonal geometry (see \Fig{hexcel}):

\begin{figure}[h!] 
\begin{center} 
\epsfxsize=10cm \centerline{ \epsffile{Ghexcel.eps}}
\parbox{14cm}{\caption{Two-dimensional hexagonal geometry}
\label{fig:hexcel}}    
\end{center}  
\end{figure}

This geometry can be analyzed using the \moc{SYBILT:} and
\moc{EXCELT:} tracking modules:

\begin{verbatim}
HEXAGON := GEO: :: HEX 12
  HBC S30 ALBE 1.6
  SIDE 1.3
  MIX 1 1 1 2 2 2 3 3 3 4 5 6 
;
\end{verbatim}

\item 3--D Cartesian supercell (see \Fig{supercel}):

\begin{figure}[h!] 
\begin{center} 
\epsfxsize=10cm \centerline{ \epsffile{Gsupercel.eps}}
\parbox{14cm}{\caption{Three-dimensional Cartesian
super-cell}\label{fig:supercel}}     \end{center}  
\end{figure}

This geometry can only be analyzed using the 
\moc{EXCELT:} tracking modules:

\begin{verbatim}
SUPERCELL := GEO: :: CAR3D 4 4 3
  X- REFL   X+ REFL
  Y- REFL   Y+ REFL
  Z- REFL   Z+ REFL
  MIX  A1 C1 D1 A3   A2 C2 D2 D2   A2 C2 C2 C2   A2 C2 C2 C2
       C3 C3 D3 A4   C4 C4 D4 D4   C4 C4 C4 C4   C4 C4 C4 C4
       C3 C3 D3 A4   C4 C4 D4 D4   C4 C4 C4 C4   C4 C4 C4 C4 
  ::: C1 := GEO: CAR3D 1 1 1
    MESHX 0.0 1.0 MESHY 0.0 1.5 MESHZ 0.0 2.0
    MIX   1 ;
  ::: C2 := GEO: C1 MESHY 0.0 1.0 ;
  ::: C3 := GEO: C1 MESHZ 0.0 1.0 ;
  ::: C4 := GEO: C2 MESHZ 0.0 1.0 ;
  ::: D1 := GEO: C1 MIX 2  ;
  ::: D2 := GEO: C2 MIX 2  ;
  ::: D3 := GEO: C3 MIX 2  ;
  ::: D4 := GEO: C4 MIX 2  ;
  ::: A1 := GEO: CARCELY 2 1
    MESHX 0.0 1.0 MESHY 0.0 1.5 MESHZ 0.0 2.0
    RADIUS 0.0 0.4 0.45
    MIX        3   4   1 ;
  ::: A2 := GEO: A1 MESHY 0.0 1.0 ;
  ::: A3 := GEO: CARCELZ 2 1
    MESHX 0.0 1.0 MESHY 0.0 1.5 MESHZ 0.0 2.0
    RADIUS 0.0 0.3 0.35
    MIX        5   6   1 ;
  ::: A4 := GEO: A3 MESHZ 0.0 1.0 ;
;
\end{verbatim}


\item Multicell geometry in a 2--D hexagonal lattice (see \Fig{multihex}).

\begin{figure}[h!] 
\begin{center} 
\epsfxsize=14cm \centerline{ \epsffile{Gmultihex.eps}}
\parbox{14cm}{\caption{Hexagonal multicell lattice geometry}
\label{fig:multihex}}    
\end{center}  
\end{figure}

Here we are considering an infinite lattice having two types of cells such
that

\begin{align*}
\begin{pmatrix}\text{pource}(1) \\ \text{pource}(2) \end{pmatrix}&=
\begin{pmatrix}1/3 \\ 2/3 \\ \end{pmatrix}\ \ \ \ \text{ and} \ \ \ \ 
\begin{pmatrix} \text{procel}(1,1) & \text{procel}(1,2) \\ \text{procel}(2,1) & \text{procel}(2,2) \\ \end{pmatrix}=
\begin{pmatrix}0 & 1 \\ 1/2 & 1/2 \\ \end{pmatrix}
\end{align*}

This lattice, can be represented either in a {\sl do-it-yourself} type geometry
({\tt HEXDIY}) or directly ({\tt HEXDIR}):

\begin{verbatim}
HEXDIY := GEO: :: GROUP 2
  POURCE 0.3333333 0.66666667
  PROCEL 0.0       1.0
         0.5       0.5
  MIX    C1 C2
  ::: C1 := GEO: TUBE 1
    RADIUS 0.0 1.1822093 MIX 1 ;
  ::: C2 := GEO: C1 MIX 2 ;
;
HEXDIR := GEO: :: HEX 2
  HBC S30 SYME SIDE 1.3 MIX 1 2 ;
\end{verbatim}

The first lattice can only be analyzed using the \moc{SYBILT:} tracking module,
while the second lattice can be analyzed using all the tracking
modules of DRAGON. 

\end{itemize}

\eject
