\subsection{The \moc{T:} module}\label{sect:TData}

A \dds{macrolib}  object can be defined directly using module \moc{MAC:} (see \Sect{MACData})
or as part of a \dds{microlib} object using module \moc{LIB:} (see \Sect{LIBData}). It is possible to
transpose a \dds{macrolib}  using the module \moc{T:}. Transposition consists in
\begin{itemize}
\item renumbering the energy groups from thermal to fast
\item transposing the transfer matrices (\moc{SCAT}) so that the primary and secondary energy group indices are permuted
\item storing \moc{NUSIGF} information in \moc{CHI} and storing \moc{CHI} infomation in \moc{NUSIGF}.
\end{itemize}

A transposed \dds{macrolib}  object permits to make adjoint flux calculations.

\vskip 0.08cm

The general format of the data for the \moc{T:} module is the following:

\begin{DataStructure}{Structure \dstr{T:}} 
\dusa{MACLIB1} \moc{:=} \moc{T:} $\{$ \dusa{MACLIB2} $|$ \dusa{LIBRARY} $\}$ \moc{;}
\end{DataStructure}

\noindent where
\begin{ListeDeDescription}{mmmmmmmm}

\item[\dusa{MACLIB1}] {\tt character*12} name of a the transposed \dds{macrolib}

\item[\dusa{MACLIB2}] {\tt character*12} name of a the original \dds{macrolib}

\item[\dusa{LIBRARY}] {\tt character*12} name of a the original \dds{microlib} containing an embedded \dds{macrolib}.

\end{ListeDeDescription}
\eject
