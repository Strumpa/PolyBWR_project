\subsection{The \moc{PSP:} module}\label{sect:PSPData}

\begin{GuideUpdate}{3.04}

The \moc{PSP:} module  is used to generate a graphical file in a PostScript
ASCII format for  a DRAGON 2--D geometry which can be analyzed using the \moc{EXCELT:} tracking module
(see \Sectand{TRKData}{descexcel}). The module \moc{PSP:} is based on the PSPLOT FORTRAN library from Nova
Southeastern University.\cite{PSPLOT} Since only a few routines PSPLOT routines were required 
and because additional PostScript routine not present in the original package were needed, the
routines have been  completely readapted to DRAGON. These routines are no longer machine dependent
and are now Y2K safe. The PostScript files generated by DRAGON can be viewed by any PostScript
viewer, such as Ghostview\cite{GHOSTVIEW} or sent to a printer compatible with this language. In
DRAGON the
\moc{PSP:} module is activated using the following list of commands:

\begin{DataStructure}{Structure \dstr{PSP:}}
\dusa{PSGEO} \moc{:=} \moc{PSP:} $[$ \dusa{PSGEO} $]~\{$ \dusa{GEONAM} $|$ \dusa{TRKNAM} $\}$ $[$ \dusa{FLUNAM} $]$
\moc{::} \dstr{descpsp}
\end{DataStructure}

\noindent 
where

\begin{ListeDeDescription}{mmmmmmmm}  

\item[\dusa{PSGEO}] \texttt{character*12} name of the file
that will contain the graphical description in a POSTSCRIPT format. This file must have a sequential
ASCII format.

\item[\dusa{GEONAM}] \texttt{character*12} name of a read-only \dds{geometry} (see \Sect{GEOData}).

\item[\dusa{TRKNAM}] \texttt{character*12} name of an EXCELL type read-only \dds{tracking} (see \Sect{TRKData}).

\item[\dusa{FLUNAM}] \texttt{character*12} name of an optional read-only \dds{fluxunk} (see \Sect{FLUData}). It is
required only if a flux mapping plot is requested.

\item[\dstr{descpsp}] structure containing the input data to this module (see \Sect{descpsp}).

\end{ListeDeDescription}


\subsubsection{Data input for module \moc{PSP:}}\label{sect:descpsp}

\begin{DataStructure}{Structure \dstr{descpsp}}
$[$ \moc{EDIT} \dusa{iprint} $]$ \\
$[$ \moc{FILL} $\{$ \moc{NONE} $|$ \moc{GRAY} $|$ \moc{RGB} $|$ \moc{CMYK} $|$ \moc{HSB} $\}$ 
$[$ \moc{NOCONTOUR} $]$ $]$\\
$[$ \moc{TYPE} $\{$ \moc{REGION}  $|$ \moc{MIXTURE} $|$ \moc{FLUX} $|$ \moc{MGFLUX} $\}$ $]$ 
\end{DataStructure}

\noindent
where

\begin{ListeDeDescription}{mmmmmmmm}   

\item[\moc{EDIT}] keyword used to modify the print level \dusa{iprint}.

\item[\dusa{iprint}] index used to control the printing in this module.
It must be set to 0 if no printing on the output file is required. 

\item[\moc{FILL}] keyword to specify the drawing options.

\item[\moc{NONE}] keyword to specify that only region contour are to be
drawn.                                                                            

\item[\moc{GRAY}] keyword to specify that the regions will be filled with various levels of
gray.                                                                            

\item[\moc{RGB}] keyword to specify that the regions will be filled with various colors taken using
the RGB color scheme.                                                                            

\item[\moc{CMYK}] keyword to specify that the regions will be filled with various colors taken
using the CMYK color scheme. 

\item[\moc{HSB}] keyword to specify that the regions will be filled with various colors taken
using the HSB color scheme. This is the default option.                                                      

\item[\moc{NOCONTOUR}] keyword to specify that the contour lines delimiting each region will not be drawn.

\item[\moc{TYPE}] keyword to specify the type of graphics generated.

\item[\moc{REGION}] keyword to specify that different colors or gray levels will be associated with each
region. This is the default option.


\item[\moc{MIXTURE}] keyword to specify that different colors or gray levels will be associated with each
mixture.

\item[\moc{FLUX}] keyword to specify that the group integrated flux is to be drawn.

\item[\moc{MGFLUX}] keyword to specify that the group flux is to be drawn.

\end{ListeDeDescription}
   
\end{GuideUpdate}
\eject
