\section{Contents of a \dir{CPO} directory}\label{sect:cpodir}

This directory contains a burnup dependent hierarchical reactor database. For the purpose of illustration we
will assume that the \moc{CPO:} module is executed using the following data:
\begin{quote}
\begin{verbatim}
  CpoResults := CPO: EdiResults EvoResults ::
  BURNUP REF-CASE
  EXTRACT ALL
  NAME MIXTH ;
\end{verbatim}
\end{quote}
 where \moc{EdiResults} is a \dds{edition} data structure that contains 2 homogeneous mixtures,
evaluated and saved at 2 time steps, \moc{EvoResults} is a \dds{burnup} data structure containing information
for the successive burnup calculations used to generate \moc{EdiResults} and finally \moc{CpoResults} is the
final \dds{cpo} data structure that contains the resulting reactor database.  

\subsection{The main directory}\label{sect:cpodirmain}

The following records and sub-directories will be found in the \dir{CPO} directory:

\begin{DescriptionEnregistrement}{Main records and sub-directories in \dir{CPO}}{8.0cm}
\CharEnr
  {SIGNATURE\blank{3}}{$*12$}
   {parameter $\mathsf{SIGNA}$ containing the signature of the data structure}
\IntEnr
  {STATE-VECTOR}{$40$}
  {array $\mathcal{S}^{c}_{i}$ containing various parameters that are required to describe this data
structure}
\DirVar
  {\listedir{MIXDIR}}
  {list of sub-directories that contain homogeneous mixture information}
\end{DescriptionEnregistrement}

The signature for this data structure is $\mathsf{SIGNA}$=\verb*|L_COMPO     |. The array
$\mathcal{S}^{c}_{i}$ contains the following information:

\begin{itemize}
\item $\mathcal{S}^{c}_{1}=N_{H}$ contains the total number of homogeneous mixtures saved. 
\item $\mathcal{S}^{c}_{2}=M_{G}$ contains the maximum number of groups considered. 
\item $\mathcal{S}^{c}_{3}=M_{I}$ contains the maximum number of isotopes.
\item $\mathcal{S}^{c}_{4}=M_{L}$ contains the maximum order for the scattering anisotropy.
\item $\mathcal{S}^{c}_{5}=M_{B}$ contains the maximum number of burnup steps per mixtures. 
\end{itemize}

The list of directory \listedir{MIXDIR} named $\mathsf{MIXDIR}$ will be composed according to the following
laws. The first eight character ($\mathsf{MIXDIR}$\verb*|(1:8)|) will be identical to the first 8 character
of the user data following the keyword \moc{NAME} in the \moc{CPO:} module (here \verb*|MIXTH   |. If the
keyword \moc{NAME} is not used then $\mathsf{MIXDIR}$\verb*|(1:8)| takes the value \verb*|COMPO   |. The last
four characters   ($\mathsf{MIXDIR}$\verb*|(9:12)|) represent the various homogeneous mixture number saved on
the \dds{edition} data structure. For the case where $N_{H}$ such mixtures were produces the following
FORTRAN instructions are used to create the last four character of each of the directory names:
\begin{quote}
\verb|WRITE(|$\mathsf{MIXDIR}$\verb|(9:12),'(I4)') | $J$
\end{quote} 
  for $1\leq J \leq N_{H}$. For the example given above ($N_{H}=2$), two such directories will be
generated, namely

\begin{DescriptionEnregistrement}{Example of homogeneous mixture directories}{8.0cm}
\DirEnr
  {MIXTH\blank{6}1}{is the sub-directory that contains the information associated with
   homogeneous mixture 1}
\DirEnr
  {MIXTH\blank{6}2}{is the sub-directory that contains the information associated with
   homogeneous mixture 2}
\end{DescriptionEnregistrement}

\subsection{The mixture sub-directory}\label{sect:cpodirmixture}

Each mixture directory contains the following records and sub-directories will be found:

\begin{DescriptionEnregistrement}{Contents of a mixture sub-directory in \dir{CPO}}{7.0cm}
\CharEnr
  {TITLE\blank{7}}{$*72$}
  {parameter $\mathsf{T}$ containing the title of the run which produced this mixture}
\IntEnr
  {PARAM\blank{7}}{$4$}
  {array $\mathcal{P}^{\text{cpo}}_{i}$ containing the various parameters associated with this mixture}
\RealEnr
  {VOLUME\blank{6}}{$1$}{cm$^{3}$}
  {parameter $V_{i}$ containing the volume of the region associated this homogeneous mixture in
   the \dds{edition} data structure}
\RealEnr
  {ENERGY\blank{6}}{$G+1$}{eV}
  {array $E_{g}$ containing the energy groups limits}
\RealEnr
  {BURNUP\blank{6}}{$\mathcal{P}^{\text{cpo}}_{4}$}{MW d T$^{-1}$}
  {array $B_{k}$ containing the burnup steps}
\RealEnr
  {N/KB\blank{8}}{$\mathcal{P}^{\text{cpo}}_{4}$}{Kb$^{-1}$}
  {array $w_{k}$ containing the fuel irradiation for the different burnup steps}
\CharEnr
  {ISOTOPESNAME}{$(\mathcal{P}^{\text{cpo}}_{2})*12$}
  {array  $\mathsf{ISO}_{i}$ containing the name of the various isotopes saved for this mixture}
\DirVar
  {\listedir{BRNDIR}}
  {list of sub-directories that contain the burnup dependent cross sections associated
   with this homogeneous mixture}
\end{DescriptionEnregistrement}

The following information is stored in $\mathcal{P}^{\text{cpo}}$:

\begin{itemize}
\item $\mathcal{P}^{\text{cpo}}_{1}=G$ contains the number of groups for this homogeneous mixture.
\item $\mathcal{P}^{\text{cpo}}_{2}=N_{I}$ contains the number of isotopes in this mixture.
\item $\mathcal{P}^{\text{cpo}}_{3}=N_{L}$ contains the order of the scattering anisotropy for this mixture.
\item $\mathcal{P}^{\text{cpo}}_{4}=N_{B}$ contains the number of burnup steps for this mixture.
\end{itemize}

The list of directory \listedir{BRNDIR} names $\mathsf{BRNDIR}$ will be composed according to the following
FORTRAN instructions:
\begin{quote}
\verb|WRITE(|$\mathsf{BRNDIR}$\verb|,'(A8,I4)') 'BURN   ',| $J$
\end{quote}
  for $1\leq J \leq N_{B}$. For the example given above ($N_{B}=2$), two such directories will be
generated, namely

\begin{DescriptionEnregistrement}{Example of homogeneous mixture directories}{8.0cm}
\DirEnr
  {BURN\blank{7}1}{is the sub-directory that contains the information associated with
   burnup step 1}
\DirEnr
  {BURN\blank{7}2}{is the sub-directory associated with burnup step 2}
\end{DescriptionEnregistrement}

\subsection{The burnup sub-directory}\label{sect:cpodirburnup}

A burnup sub-directory contains the following records and sub-directories:

\begin{DescriptionEnregistrement}{Contents of a burnup sub-directory in \dir{CPO}}{7.0cm}
\RealEnr
  {ISOTOPESDENS}{$N_{I}$}{(cm b)$^{-1}$}
  {array $\rho_{i}$ containing the density of each isotopes}
\RealEnr
  {ISOTOPES-EFJ}{$N_{I}$}{J}
  {array $H_{i}$ containing the energy produced per fission for each isotope}
\RealEnr
  {FLUX-INTG\blank{3}}{$G$}{cm s$^{-1}$}  
  {array $\Phi_{m}^{g}$ containing the integrated flux}
\RealEnr
  {OVERV\blank{7}}{$G$}{cm$^{-1}$s}  
  {array $1/v_{m}^{g}$ containing the inverse of the average neutron velocity}
\RealEnr
  {FLUXDISAFACT}{$G$}{}
  {array $F_{g}$ containing the ratio of the flux in the fuel to the flux in the cell}
\DirVar
  {\listedir{ISOTOPE}}
  {list of $N_{I}$ sub-directories that contain the isotopic microscopic cross section 
   for this burnup step}
\end{DescriptionEnregistrement}

The list of directory names is specified by $\mathsf{ISODIR}=\mathsf{ISO}_{i}$ for $i=1$ to
$N_{I}$. The first isotope $\mathsf{ISODIR}$ is named \verb*|MACR        | and represents an equivalent
macroscopic isotope  with a density of $1.0$ (cm b)$^{-1}$. The content of the isotopic multigroup cross
section directory is described in \Sect{isotopedir}. 

