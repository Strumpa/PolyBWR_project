\section{Contents of a \dir{kinet} directory}\label{sect:kinetdir}

The {\tt L\_KINET} specification is used to store the data related to the space-time
neutron kinetics calculations. This directory also contains the main calculations results corresponding
to the current time step of a transient.

\subsection{State vector content for the \dir{kinet} data structure}\label{sect:kinetstate}

The dimensioning parameters for this data structure, which are stored in the state vector
$\mathcal{S}^{k}_{i}$, represent:

\begin{itemize}

\item The current time-step index $N_{tr}=\mathcal{S}^{k}_{1}$

\item The number of delayed-neutron precursor groups $N_{dg}=\mathcal{S}^{k}_{2}$

\item The number of energy groups $N_{gr}=\mathcal{S}^{k}_{3}$

\item The type of geometry $I_{geo} = \mathcal{S}^{k}_{4}$

\item The total number of finite elements $N_{el}=\mathcal{S}^{k}_{5}$

\item The total number of unknowns per energy group $N_{un}=\mathcal{S}^{k}_{6}$

\item The number of flux unknowns per energy group $N_{uf}=\mathcal{S}^{k}_{7}$

\item The number of precursors unknowns per delayed group $N_{up}=\mathcal{S}^{k}_{8}$

\item The number of fissile isotopes $N_{fiss}=\mathcal{S}^{k}_{9}$

\item The type of system matrices $N_{sys}=\mathcal{S}^{k}_{10}$

\item Number of free iteration per variational acceleration cycle $N_{f}=\mathcal{S}^{k}_{11}$
 
\item Number of accelerated iteration per variational acceleration cycle $N_{a}=\mathcal{S}^{k}_{12}$ 

\item Type of normalization for the flux $I_{\rm norm}=\mathcal{S}^{k}_{13}$ where
\begin{displaymath}
I_{\rm norm} = \left\{
\begin{array}{rl}
 0 & \textrm{No normalization} \\
 1 & \textrm{Imposed factor} \\
 2 & \textrm{Maximum flux normalization} \\
 3 & \textrm{Initial power normalization}
\end{array} \right.
\end{displaymath}

\item Maximum number of thermal (up-scattering) iterations $M_{\rm in}=\mathcal{S}^{k}_{14}$

\item Maximum number of outer iterations $M_{\rm out}=\mathcal{S}^{k}_{15}$

\item Initial number of ADI iterations in Trivac $M_{\rm adi}=\mathcal{S}^{k}_{16}$

\item Temporal integration scheme for fluxes $I_{\rm ifl}=\mathcal{S}^{k}_{17}$ where
\begin{displaymath}
I_{\rm ifl} = \left\{
\begin{array}{rl}
 1 & \textrm{Implicit scheme ($\Theta_{\rm f}=1$)} \\
 2 & \textrm{Crank-Nicholson scheme ($\Theta_{\rm f}=0.5$)} \\
 3 & \textrm{General theta method}
\end{array} \right.
\end{displaymath}

\item Temporal integration scheme for precursors $I_{\rm ipr}=\mathcal{S}^{k}_{18}$ where
\begin{displaymath}
I_{\rm ipr} = \left\{
\begin{array}{rl}
 1 & \textrm{Implicit scheme ($\Theta_{\rm p}=1$)} \\
 2 & \textrm{Crank-Nicholson scheme ($\Theta_{\rm p}=0.5$)} \\
 3 & \textrm{General theta method} \\
 4 & \textrm{Analytical integration method for precursors}
\end{array} \right.
\end{displaymath}

\item Exponential transformation flag $I_{\rm iexp}=\mathcal{S}^{k}_{19}$ where
\begin{displaymath}
I_{\rm iexp} = \left\{
\begin{array}{rl}
 0 & \textrm{not used} \\
 1 & \textrm{used}
\end{array} \right.
\end{displaymath}

\item Adjoint kinetics calculation flag $I_{\rm adj}=\mathcal{S}^{k}_{20}$ where
\begin{displaymath}
I_{\rm adj} = \left\{
\begin{array}{rl}
 0 & \textrm{direct (forward) calculation} \\
 1 & \textrm{adjoint (backward) calculation}
\end{array} \right.
\end{displaymath}

\end{itemize}
\goodbreak

\subsection{The main \dir{kinet} directory}\label{sect:kinetdirmain}

The following records and sub-directories will be found in the \dir{kinet} directory:

\begin{DescriptionEnregistrement}{Main records and sub-directories in \dir{kinet}}{8.0cm}

\CharEnr
  {SIGNATURE\blank{3}}{$*12$}
  {Signature of the data structure ($\mathsf{SIGNA}=${\tt L\_KINET\blank{5}})}
\IntEnr
  {STATE-VECTOR}{$40$}
  {Vector describing the various parameters associated with this data structure $\mathcal{S}^{k}_{i}$,
  as defined in \Sect{kinetstate}.}
\RealEnr
  {EPS-CONVERGE}{$4$}{}
  {Convergence parameters $\Delta_i^\epsilon$}
\CharEnr
  {TRACK-TYPE\blank{2}}{$*12$}
  {Type of tracking considered ($\mathsf{CDOOR}$). Allowed values are:
  {\tt 'BIVAC'} and {\tt 'TRIVAC'}}
\IntEnr
  {E-IDLPC\blank{5}}{$N_{el}$}
  {Position of averaged precursor concentrations in vector {\tt E-PREC}.}
\RealEnr
  {DELTA-T\blank{5}}{$1$}{s}
  {Current time increment.}
\RealEnr
  {TOTAL-TIME\blank{2}}{$1$}{s}
  {Total elapsed time from the beginning of a transient.}
\RealEnr
  {BETA-D\blank{6}}{$N_{dg}$}{}
  {Delayed-neutron fraction for each delayed-neutron precursor group.}
\RealEnr
  {LAMBDA-D\blank{4}}{$N_{dg}$}{s$^{-1}$}
  {Radioactive decay constants of each delayed-neutron precursor group.}
\RealEnr
  {CHI-D\blank{7}}{$N_{dg},N_{gr}$}{}
  {Multigroup delayed-neutron fission spectrum in each precursor group.}
\RealEnr
  {E-VECTOR\blank{4}}{$N_{uf},N_{gr}$}{}
  {Kinetics solution for fluxes at current time step.}
\RealEnr
  {E-PREC\blank{6}}{$N_{up},N_{dg}$}{}
  {Kinetics solution for precursor concentrations at current time step.}
\RealEnr
  {E-KEFF\blank{6}}{$1$}{}
  {Steady-state value of the initial $k_{\rm eff}$.}
\RealEnr
  {CTRL-FLUX\blank{3}}{$1$}{}
  {Maximum value of flux used for the controlling purpose.}
\RealEnr
  {CTRL-PREC\blank{3}}{$N_{up}\times N_{fiss}$}{}
  {Precursor concentrations at location of maximum flux.}
\IntEnr
  {CTRL-IDL\blank{4}}{$1$}
  {Position of a maximum value within the flux vector.}
\IntEnr
  {CTRL-IGR\blank{4}}{$1$}
  {Energy group number corresponding to a maximum flux value.}
\OptRealEnr
  {POWER-INI\blank{3}}{$1$}{$I_{\rm norm}=3$}{MW}
  {Initial power.}
\OptRealEnr
  {E-POW\blank{7}}{$1$}{$I_{\rm norm}=3$}{MW}
  {Actual power.}
\OptRealEnr
  {OMEGA\blank{7}}{$N_{mix},N_{gr}$}{$I_{\rm iexp}=1$}{s$^{-1}$}
  {Exponential transformation factor. $N_{mix}$ is the number of material mixtures}
\end{DescriptionEnregistrement}

The convergence parameters $\Delta_i^\epsilon$ represent:
\begin{itemize}
\item $\Delta_1^\epsilon$ is the thermal (up-scattering) iteration flux convergence parameter
\item $\Delta_2^\epsilon$ is the outer iteration flux convergence parameter
\item $\Theta_{\rm f}$ is the value of theta-parameter for fluxes
\item $\Theta_{\rm p}$ is the value of theta-parameter for precursors
\end{itemize}

\eject
