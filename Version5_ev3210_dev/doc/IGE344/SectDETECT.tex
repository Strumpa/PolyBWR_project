\subsection{The \moc{DETECT:} module}\label{sect:flpow}

\vskip 0.2cm
The \moc{DETECT:} module is used to compute the mean flux at each detector site
and the response of each detector.

\noindent
The \moc{DETECT:} module specifications are:

\begin{DataStructure}{Structure \moc{DETECT:}}
\dusa{DETEC} \moc{:=} \moc{DETECT:} \dusa{DETEC} \dusa{FLUX}  \dusa{TRACK} 
\dusa{GEOM} \moc{::}
 \dstr{descdetect} \moc{;}
\end{DataStructure}

\noindent where

\begin{ListeDeDescription}{mmmmmmmm}

\item[\dusa{DETEC}] \texttt{character*12} name of the \dds{detect}
containing the detector positions and responses. 

\item[\dusa{FLUX}] \texttt{character*12} name of the \dds{flux}  
containing the flux solution computed by
the \moc{FLUD:} or \moc{FLPOW:} modules. To obtain a correct result, the best is to
use a normalized flux, coming from the \moc{FLPOW:} module. In this case, the fluxes
are normalized to the reactor power.

\item[\dusa{TRACK}] \texttt{character*12} name of the \dds{track}
containing the TRIVAC tracking.

\item[\dusa{GEOM}] \texttt{character*12} name of the \dds{geometry} 
containing the mesh-splitting geometry created by the
\moc{USPLIT:} or \moc{GEO:} modules.

\item[\dstr{descdetect}] structure containing the data to module 
\moc{DETECT:}.

\end{ListeDeDescription}

\vskip 0.2cm

\subsubsection{Input data to the \moc{DETECT:} module}

\noindent
Note that the fuel-lattice power distribution can be printed only on the screen.\\

\begin{DataStructure}{Structure \dstr{descdetect}}

$[$ \moc{EDIT} \dusa{iprt} $]$ 
\moc{TIME} \dusa{dt} 
\moc{REF} \dusa{kc}  \\
$[$ \moc{NORM} \dusa{vnorm} $]$  \\
$[$ SIMEX $\{$ SPLINE $|$ PARAB $\}$ $]$ \\
;
\end{DataStructure}

\noindent where
\begin{ListeDeDescription}{mmmmmmmm}

\item[\moc{EDIT}] key word used to set \dusa{iprt}.

\item[\dusa{iprt}] index used to control the printing in module \moc{
DETECT:}. =0 for no print; =1 for minimum printing(default value); 
=4 for printing each detector name; =5 for finite element numbers 
and total number of finite elements for each detector. 

\item[\moc{TIME}] key word used to set \dusa{dt}.

\item[\dusa{dt}] time step between two calls to the \moc{DETECT:} module. 

\item[\moc{REF}] key word used to set \dusa{kc}.

\item[\dusa{kc}] index used to control the type of calculation,
 =0 for reference calculation; =1 normal calculation. The reference responses are
used to obtain detector current responses in full power fractions.

\item[\moc{NORM}] key word used to set \dusa{vnorm}.

\item[\dusa{vnorm}] value used to normalized responses of all the detectors
present in \dds{detect}.

\item[\moc{SIMEX}] key word used to specify that a polynomial interpolation of 
detector fluxes according to HQSIMEX method. This interpolation will be 
applied only for vanadium detectors, under \dusa{NAMTYP} of value 
\texttt{VANAD\_REGUL}.

\item[\moc{SPLINE}] key word to specify that the flux at detector site
will be computed with a spline method. 

\item[\moc{PARAB}] key word to specify that the flux at detector site
will be computed with a parabolic method. 

\end{ListeDeDescription}
\clearpage
