\subsection{The {\tt M2T:} module}\label{sect:M2TData}

This component of the lattice code is dedicated to the generation of an {\sc ascii} file
with the Apotrim specification using {\sc macrolib} data. Such a file is useful to transfer multigroup
and macroscopic cross-section data toward a Moret calculation.

\vskip 0.02cm

The calling specifications are:

\begin{DataStructure}{Structure \dstr{M2T:}}
\dusa{APTRIM}~\moc{:=}~\moc{M2T:}~$[$~\dusa{APTRIM}~$]$~\dusa{MLIB}~\moc{::}~\dstr{M2T\_data} \\
\end{DataStructure}

\noindent where
\begin{ListeDeDescription}{mmmmmmm}

\item[\dusa{APTRIM}] {\tt character*12} name of an {\sc ascii} file with the Apotrim specification. If \dusa{APTRIM} appears on the RHS, new information is appended to the existing Apotrim file.

\item[\dusa{MLIB}] {\tt character*12} name of a {\sc macrolib} (type {\tt L\_MACROLIB}) object.

\item[\dusa{M2T\_data}] input data structure containing specific data (see \Sect{descM2T}).

\end{ListeDeDescription}

\subsubsection{Data input for module {\tt M2T:}}\label{sect:descM2T}

\vskip -0.5cm

\begin{DataStructure}{Structure \dstr{M2T\_data}}
$[$~\moc{EDIT} \dusa{iprint}~$]$ \\
$[$~\moc{PN} \dusa{nl}~$]~[$~\moc{TRAN}~$]~[$~\moc{NOMA}~$]$ \\
$[[$~\moc{MIX} \dusa{hmix}~$[$~\moc{FROM}~\dusa{imixold}~$]~[$~\moc{BURN} \dusa{bup}~$]~[$~\moc{TEMP} \dusa{tval}~$]$~\moc{ENDMIX}~$]]$\\
{\tt ;}
\end{DataStructure}

\noindent where
\begin{ListeDeDescription}{mmmmmmmm}

\item[\moc{EDIT}] keyword used to set \dusa{iprint}.

\item[\dusa{iprint}] index used to control the printing in module {\tt
M2T:}. =0 for no print; =1 for minimum printing (default value).

\item[\moc{PN}] keyword used to set the Legendre order of the scattering transfers written on the Apotrim file.

\item[\dusa{nl}] Legendre order. By default, \dusa{nl} $=0$ corresponding to an isotropic collision in LAB.

\item[\moc{TRAN}] keyword used to set a transport correction on cross sections written on the Apotrim file.

\item[\moc{NOMA}] keyword used to avoid writing the energy mesh on the Apotrim file. This
option is useful to catenate additional mixture information on an existing Apotrim file. By
default, the energy mesh is written on the Apotrim file.

\item[\moc{MIX}] keyword used to set \dusa{hmix}.

\item[\dusa{hmix}] {\tt character*20} name of the mixture to be written on the Apotrim file.

\item[\moc{BURN}] keyword used to set the burnup of a mixture.

\item[\dusa{bup}] burnup of a mixture. By default, \dusa{bup} $=0.0$.

\item[\moc{TEMP}] keyword used to set the temperature of a mixture.

\item[\dusa{tval}] temperature of a mixture in Celsius. By default, \dusa{tval} $=0.0 \ ^\circ{\rm C}$.

\item[\moc{FROM}] keyword used to set the index of the mixture in the {\sc macrolib} object.

\item[\dusa{imixold}] index of the mixture that is recovered in the {\sc macrolib} object. By default, \dusa{imixold}$=1$.

\item[\moc{ENDMIX}] end of specification keyword for the material mixture.

\end{ListeDeDescription}

Here is an example of the creation of an Apotrim file named {\tt APOTR} with a Hansen-Roach energy mesh created
from a XMAS 172-group flux calculation. The Apotrim file is created from three
LCM objects {\tt FLUX}, {\tt LIBRARY2} and {\tt TRACK} containing the flux, the XMAS-formatted microlib
and the tracking.

\begin{verbatim}
LINKED_LIST TRACK LIBRARY2 FLUX MAC2 EDIT ;
SEQ_ASCII APOTR ;
...
EDIT := EDI: LIBRARY2 TRACK FLUX :: EDIT 3
*     Hansen-Roach energy mesh follows
      COND 12 17 21 27 33 42 50 60 66 76 84 95 123 140 155 172
      MERGE MIX 1 1 1 1 1 1 2 3 3
      SAVE ON 'EDITCDAT   1' ;
MAC2 := EDIT :: STEP UP 'EDITCDAT   1' STEP UP 'MACROLIB' ;
APOTR := M2T: MAC2 :: EDIT 3 TRAN MIX FUEL FROM 1 ENDMIX
                                  MIX CLAD FROM 2 ENDMIX
                                  MIX COOLANT FROM 3 ENDMIX ;
\end{verbatim}
\eject
