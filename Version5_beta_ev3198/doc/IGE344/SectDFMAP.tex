\subsection{Contents of \dir{fmap} data structure}\label{sect:resinidat}

\vskip 0.2cm
A \dir{fmap} data structure is used to store fuel assembly (or bundle) map and 
fuel information such as powers, average fluxes, control zones, burnup
or refueling scheme. The fuel bundle location are given in an
embedded sub-directory which contains the records as a \dir{geometry}
data structure. This object has a signature {\tt L\_MAP};
it is created using the \moc{RESINI:} module.

\subsubsection{The state-vector content}\label{sect:fmapstate}

\noindent
The dimensioning parameters $\mathcal{S}_i$, which are stored in the state
vector for this data structure, represent:

\begin{itemize}

\item The number of fuel bundles per channel $N_{\rm b} = \mathcal{S}_1$

\item The number of fuel channels $N_{\rm ch} = \mathcal{S}_2$

\item The number of combustion zones $N_{\rm comb} = \mathcal{S}_3$

\item The number of energy groups $N_{\rm gr} = \mathcal{S}_4$

\item The type of interpolation with respect to burnup $I_{\rm btyp}$ = $\mathcal{S}_5$

\begin{displaymath} I_{\rm btyp} = \left\{
\begin{array}{rl}
 0 & \textrm{interpolation type is not provided} \\
 1 & \textrm{according to the time-average model} \\
 2 & \textrm{according to the instantaneous model} \\
\end{array} \right.
\end{displaymath}

\item The number of bundle shift. $N_{\rm sht} = \mathcal{S}_{6}$

\item The number of fuel types $N_{\rm fuel} = \mathcal{S}_7$

\item The number of recorded parameters $N_{\rm parm} = \mathcal{S}_8$

\item The total number of fuel bundles $N_{\rm tot} = \mathcal{S}_9$

\item The number of voided reactor channels $N_{\rm void} = \mathcal{S}_{10}$

\item The option with respect to the core-voiding pattern $I_{\rm void}$ = $\mathcal{S}_{11}$

\begin{displaymath} I_{\rm void} = \left\{
\begin{array}{rl}
 0 & \textrm{voiding pattern not provided} \\
 1 & \textrm{full-core voiding pattern} \\
 2 & \textrm{half-core voiding pattern} \\
 3 & \textrm{quarter-core voiding pattern} \\
 4 & \textrm{checkerboard-full voiding pattern} \\
 5 & \textrm{checkerboard-half voiding pattern} \\
 6 & \textrm{checkerboard-quarter voiding pattern} \\
 7 & \textrm{user-defined voiding pattern} \\
\end{array} \right.
\end{displaymath}

\item The type of the geometry $F_{\rm t}$ = $\mathcal{S}_{12}$

\begin{displaymath} F_{\rm t} = \left\{
\begin{array}{rl}
 7 & \textrm{Cartesian \dusa{3-D} geometry} \\
 9 & \textrm{Hexagonal \dusa{3-D} geometry} \\
\end{array} \right.
\end{displaymath}

\item The naval-coordinate layout used by the {\tt SIM:} module $I_{\rm sim}$ = $\mathcal{S}_{13}$.

\vskip 0.08cm

The number of assemblies along $X$ and $Y$ axis are given using
$$
L_{\rm x}={I_{\rm sim}\over 100} \ \ \ {\rm and} \ \ \ L_{\rm y}={\rm mod}(I_{\rm sim},100)
$$

\item The total number of assemblies $N_{\rm ass} = \mathcal{S}_{14}$

\item The number of assemblies along $X$ direction in Cartesian geometry.  The number of assemblies in the radial plane in hexagonal geometry. $N_{\rm xa} = \mathcal{S}_{15}$

\item The number of assemblies along $Y$ direction in Cartesian geometry $N_{\rm ya} = \mathcal{S}_{16}$

\item The number of plane of the mesh along $Z$ direction where assemblies are located $N_{\rm z,ass} = \mathcal{S}_{17}$

\item The number of particularized isotopes to store in \{hcycle\} sub-directories $N_{\rm is} = \mathcal{S}_{18}$

\item The number of \{hcycle\} sub-directories $N_{\rm cy} = \mathcal{S}_{19}$

\end{itemize}

\subsubsection{The main \dir{fmap} directory}\label{sect:fmapdir}

\noindent
The following records and sub-directories will be found on the first level of \dir{fmap}
directory:

\begin{DescriptionEnregistrement}{Records and sub-directories
 in \dir{fmap} data structure}{7.0cm} \label{tabl:tabfmap}

\CharEnr
 {SIGNATURE\blank{3}}{$*12$}
 {Signature of the \dir{fmap} data structure ($\mathsf{SIGNA}=${\tt L\_MAP\blank{7}}).}

\IntEnr
  {STATE-VECTOR}{$40$}
  {Vector describing the various parameters associated with this data structure
  $\mathcal{S}_i$}

\IntEnr
 {FLMIX\blank{7}}{$N_{\rm ch}, N_{\rm b}$}
 {Fuel type indices per bundle or assembly subdivisions for each reactor channel.}

\OptIntEnr
 {FLMIX-INI\blank{3}}{$N_{\rm ch}, N_{\rm b}$}{$I_{\rm sim}\ne 0$}
 {Fuel type indices per bundle or assembly subdivisions for each reactor channel, as defined by user
 in {\tt RESINI:} module.}

\OptCharEnr
 {S-ZONE\blank{6}}{$(N_{\rm ch})*4$}{$I_{\rm sim}\ne 0$}
 {identification name corresponding to the basic naval-coordinate position of an assembly, as defined by user
 in {\tt RESINI:} module..}

\IntEnr
 {BMIX\blank{8}}{$N_x$, $N_y$, $N_z$}
 {Renumbered mixture indices per each fuel region over the fuel-map
  geometry; for the non-fuel regions these indices are set to 0.}

\OptCharEnr
 {XNAME\blank{7}}{$(N_x)*4$}{$F_{\rm t}=7$}
 {Channel identification names with respect to their horizontal position in Cartesian geometry.}

\OptCharEnr
 {YNAME\blank{7}}{$(N_y)*4$}{$F_{\rm t}=7$}
 {Channel identification names with respect to their vertical position in Cartesian geometry.}

\OptCharEnr
 {HNAME\blank{7}}{$(N_x)*8$}{$F_{\rm t}=9$}
 {Channel identification names with respect to their radial position in hexagonal geometry.}

\OptCharEnr
 {AXNAME\blank{6}}{$(N_{\rm xa})*4$}{$F_{\rm t}=7$}
 {Name of the assembly on X-direction (4 character name per assembly) in Cartesian geometry}

\OptCharEnr
 {AYNAME\blank{6}}{$(N_{\rm ya})*4$}{$F_{\rm t}=7$}
 {Name of the assembly on Y-direction (4 character name per assembly) in Cartesian geometry}

\OptDirlEnr
 {ASSEMBLY\blank{4}}{$N_{\rm ass}$}{}
 {List of {\sl assembly} directories. Each component of this list follows the specification
  presented in \Sect{fmapdirass}.}

\OptIntEnr
 {B-ZONE\blank{6}}{$N_{\rm ch}$}{$N_{\rm comb}\geq1$}
 {Combustion-zone indices per channel.}

\RealEnr
 {BURN-AVG\blank{4}}{$N_{\rm comb}$}{MW d t$^{-1}$}
 {Average exit burnups per combustion zone.}

\OptRealEnr
 {BURN-INST\blank{3}}{$N_{\rm ch}, N_{\rm b}$}{$I_{\rm btyp}=2$}{MW d t$^{-1}$}
 {Instantaneous burnups per bundle or assembly subdivisions for each channel.}

\OptRealEnr
 {BURN-BEG\blank{4}}{$N_{\rm ch}, N_{\rm b}$}{$I_{\rm btyp}=1$}{MW d t$^{-1}$}
 {Low burnup integration limits according to the time-average model.}

\OptRealEnr
 {BURN-END\blank{4}}{$N_{\rm ch}, N_{\rm b}$}{$I_{\rm btyp}=1$}{MW d t$^{-1}$}
 {Upper burnup integration limits according to the time-average model.}

\OptRealEnr
 {BUND-PW\blank{5}}{$N_{\rm ch}, N_{\rm b}$}{*}{kW}
 {Bundle-powers set in \moc{RESINI:} module or recovered from \moc{L\_POWER} object.}

\OptRealEnr
 {BUND-PW-INI\blank{1}}{$N_{\rm ch}, N_{\rm b}$}{*}{kW}
 {Beginning-of-transient bundle-powers recovered from \moc{L\_POWER} object.}

\OptRealEnr
 {FLUX-AV\blank{5}}{$N_{\rm ch}, N_{\rm b}, N_{\rm gr}$}{*}{cm$^{-2}$ s$^{-1}$}
 {The normalized average fluxes recorded per each fuel bundle and for
  each energy group, recovered from \moc{L\_POWER} object.}

\OptRealEnr
 {REACTOR-PW\blank{2}}{$1$}{*}{MW}
 {Full reactor power set in \moc{RESINI:} module or recovered from \moc{L\_POWER} object.}

\OptRealEnr
 {AXIAL-FPW\blank{3}}{$N_{\rm b}$}{*}{1}
 {Axial power form factor set in \moc{RESINI:} module.}

\RealEnr
 {B-EXIT\blank{6}}{$1$}{MW d t$^{-1}$}
 {Core-average discharge burnup.}

\IntEnr
 {REF-SHIFT\blank{3}}{$N_{\rm comb}$}
 {Bundle-shifts per combustion zone. A bundle-shift corresponds to the 
  number of displaced fuel bundles during the refueling operation.}

\IntEnr
 {REF-VECTOR\blank{2}}{$N_{\rm comb}, N_{\rm b}$}
 {Refueling pattern vector per combustion zone.}

\IntEnr
 {REF-SCHEME\blank{2}}{$N_{\rm ch}$}
 {Refueling scheme of each channel; it corresponds to the positive
  or negative bundle-shift number according to the flow direction. }

\RealEnr
 {REF-RATE\blank{4}}{$N_{\rm ch}$}{kg d$^{-1}$}
 {Channel refueling rates.}

\OptRealEnr
 {REF-CHAN\blank{4}}{$N_{\rm ch}$}{}{d}
 {Time values at which channels are refueled inside a refueling time 
 period.}

\OptRealEnr
 {DEPL-TIME\blank{3}}{$1$}{}{d}
 {Refueling time period in days.}

\OptRealVar
 {\{pshift\}}{$N_{\rm ch}, N_{\rm b}$}{$N_{\rm sht}\ge 1$}{$kW$}
 {The power of the bundles shifted the $i$-th time.}
            
\OptRealVar
 {\{bshift\}}{$N_{\rm ch}, N_{\rm b}$}{$N_{\rm sht} \ge 1$}{$MW d T^{-1}$}
 {The burnup of the bundles shifted the $i$-th time.}
            
\OptIntVar
 {\{ishift\}}{$N_{\rm ch},N_{\rm b}$}{$N_{\rm sht} \ge 1$}
 {The number of  shifts per bundle during refueling.}

\OptRealEnr
 {AX-SHAPE\blank{4}}{$N_{\rm ch}, N_{\rm b}$}{$I_{\rm btyp}=1$}{}
 {Normalized axial power-shape values over the fuel bundles. Equal to
 fuel-bundle powers divided by channel powers.}

\OptRealEnr
 {EPS-AX\blank{6}}{$1$}{$I_{\rm btyp}=1$}{}
 {Convergence factor for the axial power-shape calculation; it is
  defined as a relative error between the two successives calculations.}

\OptRealEnr
 {FQ\blank{10}}{$1$}{}{}
 {Hot spot factor: power of the hottest 3D Cartesian pin spot 
  normalized to the mean power of a 3D Cartesian pin spot.
  $$
  FQ={P_{\rm \overset{max}{x,y,z}}(x,y,z)\over P_{\rm \overset{moy}{x,y,z}}(x,y,z)}
  $$
  where P(x,y,z) is the power of an axial part of a pin located at the coordinates 
  (x,y,z).
  }

\OptRealEnr
 {FXY\blank{9}}{$1$}{}{}
 {Radial hot spot factor: power of the hottest pin normalized to the mean power of a pin.
  $$
  Fxy={P_{\rm \overset{max}{x,y}}(x,y)\over P_{\rm \overset{moy}{x,y}}(x,y)}
  $$
  with $$P(x,y)=\displaystyle \int_{0}^{zmax} 
  P(x,y,z) \, \mathrm{d}z$$
  }

\OptRealEnr
 {FXYZ\blank{8}}{$N_{\rm z,ass}$}{}{}
 {For each axial mesh, power of the hottest 2D Cartesian pin spot normalized to the mean 
  power of a 3D Cartesian pin spot.
  $$
  Fxy(z)={P_{\rm \overset{max}{x,y}}(x,y,z)\over P_{\rm \overset{moy}{x,y,z}}(x,y,z)}
  $$
  }

\OptRealEnr
 {FXYASS\blank{6}}{$N_{\rm ass}$}{}{}
 {For each assembly, power of the hottest pin normalized to the mean power of a pin.
  $$
  Fxy(iass)={P_{\rm \overset{max}{x,y}}(x,y,iass)\over P_{\rm \overset{moy}{x,y}}(x,y)}
  $$
  with $$P(x,y,iass)=\displaystyle \int_{0}^{zmax} 
  P(x,y,z,iass) \, \mathrm{d}z$$
  }
       
\OptCharEnr
  {HFOLLOW\blank{5}}{$(N_{\rm is})*8$}{$N_{\rm is}>0$}
  {Name of the particularized isotopes to store in \{hcycle\} directory.}
   
\DirEnr
 {GEOMAP\blank{6}}
 {Sub-directory containing the embedded \dusa{3D}-Cartesian \dir{geometry} of the fuel lattice.}

\DirlEnr
  {FUEL\blank{8}}{$N_{\rm fuel}$}
  {List of fuel--type sub-directories. Each component of the list is a directory containing
   the information relative to a single fuel type.}
   
\OptDirEnr
 {ROD-INFO\blank{4}}{*}
 {Sub-directory containing the information corresponding to the local rod insertion for PWR.
  This sub-directory follows the specification presented in \Sect{dirrodinfo}.}
          
\OptDirlEnr
 {\{hcycle\}}{$N_{\rm burn}$}{$N_{\rm cy}> 0$}
 {Sub-directory containing information related to a fuel cycle in a PWR. $N_{\rm burn}$ is the number of burnup steps used during
 the simulation of the cycle. These burnup steps may not be of increasing values. If module {\tt TINST:} was used to burn fuel, the
 name of the \dusa{hcycle} directory is ``{\tt \_TINST}''.}

\OptCharEnr
 {CYCLE-NAMES\blank{1}}{$(N_{\rm cy})*12$}{$N_{\rm cy}>0$}
 {Names of fuel cycle sub-directories \{hcycle\}.}

\OptDirlEnr
  {PARAM\blank{7}}{$N_{\rm parm}$}{$N_{\rm parm}>0$}
  {List of parameter--type sub-directories corresponding to actual time. Each component of the list is a directory
   containing the information relative to a single parameter (see Sect.~\ref{sect:dirparam}). The total number of sub-directories
   corresponds to the total number of recorded parameters $N_{\rm parm}$ (excluding  burnups).}

\end{DescriptionEnregistrement}

\noindent The contents of the \moc{GEOMAP} sub-directory correspond to the typical
contents of the \dir{geometry} data structure.
The dimensioning parameters $N_x$, $N_y$, and $N_z$ represent the number
of volumes along the corresponding axis in the fuel-map geometry.\\

%\eject
The shifting information records \{pshift\}, \{bshift\} and \{ishift\}
will be composed using the following FORTRAN instructions, respectively, as 
  \begin{displaymath}
    \mathtt{WRITE(}\mathsf{pshift}\mathtt{,'(A6,I2)')} \ 
   \mathtt{'PSHIFT'},ell
  \end{displaymath}
  \begin{displaymath}
    \mathtt{WRITE(}\mathsf{bshift}\mathtt{,'(A6,I2)')} \ 
   \mathtt{'BSHIFT'},ell
  \end{displaymath}
  \begin{displaymath}
    \mathtt{WRITE(}\mathsf{ishift}\mathtt{,'(A6,I2)')} \ 
   \mathtt{'ISHIFT'},ell
  \end{displaymath}
for $1\leq ell \leq N_{\rm sht}$. \\

Each time a bundle is shifted and stay in the reactor, its burnup and power will be saved
in the records \{bshift\} and \{pshift\}. For example, \{bshift i\} and \{pshift i\} will
contain all the burnups and powers of bundles that have been shifted $i$-th time.

\subsubsection{The \moc{FUEL} sub-directories}\label{sect:dirfuel}

Each \moc{FUEL} sub-directory contains the information corresponding
to a single fuel type. Inside each sub-directory, the following records
will be found:

\begin{DescriptionEnregistrement}{Records in \moc{FUEL} sub-directories}
{7.0cm} \label{tabl:tabfuel}

\IntEnr
 {MIX\blank{9}}{$1$} {Fuel-type mixture number.}

\IntEnr
 {TOT\blank{9}}{$1$} {Total number of fuel bundles for this fuel type.}

\IntEnr
 {MIX-VOID\blank{4}}{$1$} {Voided-cell mixture number for this fuel type.}

\RealEnr
{WEIGHT\blank{6}}{$1$}{kg}
{Fuel weight in a bundle for this fuel type.}

\RealEnr
{ENRICH\blank{6}}{$1$}{wt\%}
{Fuel enrichment for this fuel type.}

\RealEnr
{POISON\blank{6}}{$1$}{}
{Poison load for this fuel type.}

\end{DescriptionEnregistrement}

\subsubsection{The \{hcycle\} sub-directories}\label{sect:dirhcycle}

Each \{hcycle\} sub-directory contains the information corresponding
to a single PWR fuel cycle. Inside each sub-directory, the following records
will be found:

\begin{DescriptionEnregistrement}{Records in \{hcycle\} sub-directories}
{7.0cm} \label{tabl:tabhcycle}

\CharEnr
 {ALIAS\blank{7}}{$(*12$}
 {Name of the cycle sub-directory.}

\RealEnr
 {TIME\blank{8}}{$1$}{d}
 {Depletion time corresponding to instantaneous burnup values.}

\OptDirlEnr
  {PARAM\blank{7}}{$N_{\rm parm}$}{$N_{\rm parm}>0$}
  {List of parameter--type sub-directories corresponding to refuelling time. Each component of the list is a directory
   containing the information relative to a single parameter (see Sect.~\ref{sect:dirparam}). The total number of sub-directories
   corresponds to the total number of recorded parameters $N_{\rm parm}$ (excluding  burnups).}

\RealEnr
 {BURNAVG\blank{5}}{$1$}{MW d t$^{-1}$}
 {Average burnup of the assembly.}

\CharEnr
 {NAME\blank{8}}{$(N_{\rm ch})*12$}
 {Names of each assembly or of each quart-of assembly during a refuelling cycle. All
 quart-of-assembly belonging to the same assembly have the same name.}

\CharEnr
 {CYCLE\blank{7}}{$(L_{\rm x}\times L_{\rm y})*4$}
 {Shuffling matrix for refuelling as provided by the plant operator. The name "$|$"
 is reserved for empty locations.}

\IntEnr
 {FLMIX\blank{7}}{$N_{\rm ch}, N_{\rm b}$}
 {Fuel type indices per assembly subdivisions for each reactor channel.}

\RealEnr
 {BURN-INST\blank{3}}{$N_{\rm ch}, N_{\rm b}$}{MW d t$^{-1}$}
 {Instantaneous burnups per assembly subdivisions for each channel.}

\RealEnr
 {POWER-BUND\blank{2}}{$N_{\rm ch}, N_{\rm b}$}{kW}
 {Powers per assembly subdivisions for each channel.}

\RealEnr
 {K-EFFECTIVE\blank{1}}{$1$}{}
 {Effective multiplication factor $k_{\mathrm{eff}}$.}

\OptRealEnr
 {FOLLOW\blank{6}}{$N_{\rm ch}, N_{\rm b}, N_{\rm is}$}{$N_{\rm is}>0$}{(cm b)$^{-1}$}
 {Number densities of the particularized isotopes.}

\end{DescriptionEnregistrement}

\clearpage

\subsubsection{The \moc{PARAM} sub-directories}\label{sect:dirparam}

Each \moc{PARAM} sub-directory contains the information corresponding
to a single local or global parameter (excluding  burnups). Inside a such sub-directory,
the following records will be found:

\begin{DescriptionEnregistrement}{Records in \moc{PARAM} sub-directories}
{7.0cm} \label{tabl:tabparam}

\CharEnr
 {P-NAME\blank{6}}{$*12$} {Unique identification name of this parameter. This
 name is user-defined; however, it is recommended to use the following pre-defined
 values:

\begin{tabular}{|c|l|}
\hline
{\tt C-BORE} & Boron concentration \\
{\tt T-FUEL} & Averaged fuel temperature \\
{\tt T-SURF} & Surfacic fuel temperature \\
{\tt T-COOL} & Averaged coolant temperature \\
{\tt D-COOL} & Averaged coolant density \\
\hline
\multicolumn{2}{|l|}{CANDU-only parameters:} \\
\hline
{\tt T-MODE} & Averaged moderator temperature\\
{\tt D-MODE} & Averaged moderator density \\
\hline
\end{tabular}
 }

\CharEnr
 {PARKEY\blank{6}}{$*12$} {Corresponding name of this parameter as recorded
  in a multi-parameter Compo file.}

\IntEnr
 {P-TYPE\blank{6}}{$1$} {Number associated to the type of recorded parameter:
 $ptype=1$ for global parameter; $ptype=2$ for local parameter.}

\OptRealEnr
 {P-VALUE\blank{5}}{$1$}{$ptype=1$}{}{Recorded single value for global parameter. Temperatures are given in K and densities are given in g/cc.}

\OptRealEnr
  {}{$N_{\rm ch}, N_{\rm b}$}{$ptype=2$}{}{Recorded values for local parameter per each fuel
  bundle for every channel. Temperatures are given in K and densities are given in g/cc.}

\end{DescriptionEnregistrement}

\subsubsection{The \moc{ROD-INFO} sub-directory}\label{sect:dirrodinfo}

The \moc{ROD-INFO} sub-directory contains the information corresponding
to the local rod insertion for PWR. This sub-directory is created after the first calling to the \moc{ROD:} module. Inside this sub-directory,
the following records will be found:

\begin{DescriptionEnregistrement}{Records in \moc{ROD-INFO} sub-directories}
{7.0cm} \label{tabl:tabrodinfo}

\RealEnr
 {ROD-INIT\blank{4}}{$1$}{} {Default value for the rod field. This value corresponds to no rod inserted.}

\IntEnr
 {INS-MAX\blank{5}}{$1$} {Number associated to the maximum of rod insertion steps possible.}

\RealEnr
 {STEP-CM\blank{5}}{$1$}{} {Length of one rod insertion step (in cm).}

\IntEnr
 {REFL-BOTTOM\blank{1}}{$1$} {Number of bottom reflective axial-planes.}

\IntEnr
 {NB-GROUP\blank{4}}{$1$} {Number of rod groups.}

\IntEnr
 {MAX-MIX\blank{5}}{$1$} {Maximum number of rod zones for all the rods defined.}

\OptCharEnr
 {ROD-NAME\blank{4}}{$(N_{\rm gr})*3$}{}
 {Name of each rod group defined by user in {\tt ROD:} module.}

\IntEnr
 {ROD-INSERT\blank{2}}{$N_{gr}$} {Insertion of each rod group defined by user in {\tt ROD:} module.}

\RealEnr
 {ROD-RIN\blank{5}}{$N_{gr}, N_{mmix}$}{} {Rod identification numbers (RIN) for each rod group defined by user in {\tt ROD:} module.}

\IntEnr
 {ROD-NBZONE\blank{2}}{$N_{gr}$} {Number of RIN zones for each rod group defined by user in {\tt ROD:} module.}

\RealEnr
 {ROD-HEIGHT\blank{2}}{$N_{gr}, N_{mmix}$}{}
 {Height between the bottom of the rod and the beginning of the RIN zone for each rod group and each RIN zone defined by user in {\tt ROD:} module.}

\OptCharEnr
 {ROD-MAP\blank{5}}{$(N_{\rm ch})*3$}{}
 {Identification name corresponding to the rod map as defined by user
 in {\tt ROD:} module.}

\end{DescriptionEnregistrement}

\subsubsection{The ASSEMBLY sub-directory}\label{sect:fmapdirass}

\begin{DescriptionEnregistrement}{Assembly type sub-directory}{7.0cm}
\CharEnr
  {LABEL\blank{7}}{$*8$}
  {Label of the assembly. It consists a 8 character name composed of the AYNAME and AXNAME}

\RealEnr
  {PIN-POWER\blank{3}}{$N_{pin}^2*N_{\rm z,ass}$}{}
  {array containing the pin power for each pin on each mesh plane of the assembly.}

\RealEnr
  {ASS-POWER\blank{3}}{1}{}
  {total assembly power.}

\RealEnr
  {SIG\_F*PHI\blank{3}}{$N_{pin}^2*N_{\rm z,ass}$}{}
  {array containing the integral of $\Sigma_f$  times the flux for each pin on each mesh plane of the assembly.}

\RealEnr
  {FLUX\blank{8}}{$N_{pin}^2*N_{\rm z,ass}*N_g$}{}
  {array containing the flux of each group for each pin on each mesh plane of the assembly.}

\end{DescriptionEnregistrement}

\clearpage
