\clearpage
$ $
\vskip 2.0cm

\begin{center}

SUMMARY

\end{center}

TRIVAC is a computer code intended to compute the neutron flux in a 
fractional or in a full core representation of a nuclear reactor. Interested readers
can obtain fundamental informations about full-core calculations in Chapter~5
of Ref.~\citen{PIP2009}. The multigroup
and multidimensional form of the diffusion equation or simplified $P_n$ equation
is first discretized to
produce a consistent matrix system. This matrix system is subsequently solved
using iterative techniques (inverse or preconditioned power method with
ADI preconditioning) and sparse matrix algebra techniques
(triangular factorization). The actual implementation of TRIVAC allows the
discretization of 1-D geometries (slab and cylindrical), 2-D geometries
(Cartesian, cylindrical and hexagonal) and 3-D geometries (Cartesian and
hexagonal). Many discretization techniques are available, including mesh corner
or mesh centered finite difference methods, collocation techniques of various
order and finite element methods based on a primal or dual functional
formulation. TRIVAC also permits  the equations of the generalized
perturbation theory (GPT) to be solved as fixed source eigenvalue problems.
Finally, several implicit numerical schemes are available for the solving of space-time
neutron kinetics problems.

\vskip 0.15cm

The execution of TRIVAC is controlled by CLE-2000.\cite{cle2000} It is
modular and can be interfaced easily with other production codes.
