\section{Contents of a 
\dir{fluxunk} directory}\label{sect:fluxunkdir}

This directory contains the main flux calculations results, including the multigroup flux, the
eigenvalue for the problem and the diffusion coefficients when computed. The following types of
equations can be solved:
\begin{enumerate}
\item Fixed source problem
\begin{equation}
\bf{A} \ \vec\Phi = \vec S
\label{eq:flux1}
\end{equation}
\noindent where $\bf{A}$ is the coefficient matrix, $\vec S$ is the source vector and
$\vec\Phi$ is the unknown vector.

\item Direct eigenvalue problem
\begin{equation}
\bf{A} \ \vec\Phi_\alpha + {1 \over K_{{\rm eff},\alpha}} \ \bf{B} \ \vec\Phi_\alpha = \vec 0
\label{eq:flux2}
\end{equation}
\noindent where $\bf{B}$ is the second coefficient matrix and where (${1 \over K_{{\rm eff},\alpha}}$
,$\vec\Phi_\alpha$) is the eigensolution corresponding to the $\alpha$--th eigenvalue
or harmonic mode. Generally, only the eigensolution corresponding to the maximum value of $K_{{\rm eff},\alpha}$ is found (the fundamental mode).

\item Adjoint eigenvalue problem
\begin{equation}
\bf{A}^\top \ \vec\Phi_\alpha^* + {1 \over K_{{\rm eff},\alpha}} \ \bf{B}^\top \ \vec\Phi_\alpha^* = \vec 0
\label{eq:flux3}
\end{equation}
\noindent where matrices $\bf{A}$ and $\bf{B}$ are transposed.

\item Fixed source direct eigenvalue equation (direct GPT)
\begin{equation}
\bf{A} \ \vec\Gamma_\alpha + {1 \over K_{{\rm eff},\alpha}} \ \bf{B} \ \vec\Gamma_\alpha = \vec S
\ \ \ \ {\rm where} \ \ \ \ \left<\Phi_\alpha^*, \ \vec S \right>=0
\label{eq:flux4}
\end{equation}
\noindent where the direct source vector $\vec S$ is orthogonal to the adjoint flux.

\item Fixed source adjoint eigenvalue equation (adjoint GPT)
\begin{equation}
\bf{A}^\top \ \vec\Gamma_\alpha^* + {1 \over K_{{\rm eff},\alpha}} \ \bf{B}^\top \ \vec\Gamma_\alpha^* = \vec S^*
\ \ \ \ {\rm where} \ \ \ \ \left<\Phi_\alpha, \ \vec S^* \right>=0
\label{eq:flux5}
\end{equation}
\noindent where the adjoint source vector $\vec S^*$ is orthogonal to the direct flux.

\end{enumerate}

\subsection{State vector content for the \dir{fluxunk} data structure}\label{sect:fluxunkstate}

The dimensioning parameters for this data structure, which are stored in the state vector
$\mathcal{S}^{f}_{i}$, represent:

\begin{itemize}

\item The number of energy groups $N_{G}=\mathcal{S}^{f}_{1}$

\item The number of unknowns per energy group $N_{U}=\mathcal{S}^{f}_{2}$

\item The type of equation considered $ I_{e} = \mathcal{S}^{f}_{3} = \alpha_1 + 10 \ \alpha_2 + 100 \ \alpha_3 + 1000 \ \alpha_4 $ where
\vskip -0.45cm

\begin{eqnarray}
\nonumber \alpha_1 &=& 0/1\textrm{:} \ \ \textrm{Fixed source (\Eq{flux1}) or \keff{} (\Eq{flux2}) direct eigenvalue equation} \\
\nonumber &~&\textrm{absent/present} \\
\nonumber \alpha_2 &=& 0/1\textrm{:} \ \ \textrm{Adjoint eigenvalue equation (\Eq{flux3}) absent/present} \\
\nonumber \alpha_3 &=& 0/1\textrm{:} \ \ \textrm{Direct fixed source eigenvalue equation -- or GPT equation (\Eq{flux4})} \\
\nonumber &~&\textrm{absent/present} \\
\nonumber \alpha_4 &=& 0/1\textrm{:} \ \ \textrm{Adjoint fixed source eigenvalue equation -- or GPT equation (\Eq{flux5})} \\
\nonumber &~&\textrm{absent/present}
\end{eqnarray}

\item The number of harmonics considered $N_{h}=\mathcal{S}^{f}_{4}$ where

\begin{displaymath}
N_{h} = \left\{
\begin{array}{rl}
 0 & \textrm{the harmonic calculation is not enabled} \\
 \ge 1 & \textrm{the harmonic calculation is enabled. $N_{h}$ is the number of harmonics.} \\
\end{array} \right.
\end{displaymath}

\item The number of specific GPT equations considered $N_{\rm gpt}=\mathcal{S}^{f}_{5}$ where

\begin{displaymath}
N_{\rm gpt} = \left\{
\begin{array}{rl}
 0 & \textrm{the GPT calculation is not enabled} \\
 \ge 1 & \textrm{the GPT calculation is enabled. $N_{\rm gpt}$ is the number of specific GPT} \\
  & \textrm{equations.} \\
\end{array} \right.
\end{displaymath}

\item The type of $B_n$ solution considered $I_{s}=\mathcal{S}^{f}_{6}$ where

\begin{displaymath}
I_{s} = \left\{
\begin{array}{rl}
-2 & \textrm{1D Fourier analysis, fixed source problem, no eigenvalue}\\
-1 & \textrm{No flux calculation, fluxes taken from input file}\\
 0 & \textrm{Fixed source problem, no eigenvalue} \\
 1 & \textrm{fixed source eigenvalue problem (GPT type) with fission} \\
 2 & \textrm{\keff{} eigenvalue problem with fission and without leakage} \\
 3 & \textrm{\keff{} eigenvalue problem with fission and leakage } \\
 4 & \textrm{Buckling eigenvalue problem with fission and leakage} \\
 5 & \textrm{Buckling eigenvalue problem without fission but with leakage} 
\end{array} \right.
\end{displaymath}

\item The type of leakage model $I_{l}=\mathcal{S}^{f}_{7}$ where

\begin{displaymath}
I_{l} = \left\{
\begin{array}{rl}
 0 & \textrm{No leakage model} \\
 1 & \textrm{Homogeneous \moc{PNLR} calculation} \\
 2 & \textrm{Homogeneous \moc{PNL} calculation} \\
 3 & \textrm{Homogeneous \moc{SIGS} calculation} \\
 4 & \textrm{Homogeneous \moc{ALSB} calculation} \\
 5 & \textrm{Leakage with isotropic streaming effects} \\
16 & \textrm{Leakage with anisotropic streaming effects -- imposed buckling} \\
26 & \textrm{Leakage with anisotropic streaming effects -- X-Buckling search} \\
36 & \textrm{Leakage with anisotropic streaming effects -- Y-Buckling search} \\
46 & \textrm{Leakage with anisotropic streaming effects -- Z-Buckling search} \\
56 & \textrm{Leakage with anisotropic streaming effects -- radial Buckling search} \\
66 & \textrm{Leakage with anisotropic streaming effects -- total Buckling search} \\
\end{array} \right.
\end{displaymath}

\item Number of free iteration per variational acceleration cycle $N_{f}=\mathcal{S}^{f}_{8}$
 
\item Number of accelerated iteration per variational acceleration cycle $N_{a}=\mathcal{S}^{f}_{9}$ 

\item Thermal rebalancing option $I_{r}=\mathcal{S}^{f}_{10}$ where

\begin{displaymath}
I_{r} = \left\{
\begin{array}{rl}
 0 & \textrm{No thermal iteration rebalancing} \\
 1 & \textrm{Thermal iteration rebalancing activated} \\
\end{array} \right.
\end{displaymath}

\item Maximum number of thermal (up-scattering) iterations $M_{\rm in}=\mathcal{S}^{f}_{11}$

\item Maximum number of outer iterations $M_{\rm out}=\mathcal{S}^{f}_{12}$

\item Initial number of ADI iterations in Trivac $M_{\rm adi}=\mathcal{S}^{f}_{13}$

\item Block size of the Arnoldi Hessenberg matrix with the implicit restarted Arnoldi method (IRAM) ($=0$ if the symmetrical variational acceleration technique (SVAT) is used) $N_{\rm blsz}=\mathcal{S}^{f}_{14}$

\item Number of iterations before restarting with the GMRES(m) acceleration method for solving the ADI-preconditionned linear systems in Trivac ($=0$ if $M_{\rm adi}$ free iterations are used) $N_{\rm gmr1}=\mathcal{S}^{f}_{15}$

\item Number of iterations before restarting with the GMRES(m) acceleration method for solving a multigroup fixed-source problem ($=0$ if the variational acceleration technique is used) $N_{\rm gmr2}=\mathcal{S}^{f}_{16}$

\item Number of material mixtures $N_m=\mathcal{S}^{f}_{17}$

\item Number of leakage zones $N_{\rm leak}=\mathcal{S}^{f}_{18}$. Set to zero if no leakage zones are defined.

\end{itemize}

\subsection{The main \dir{fluxunk} directory}\label{sect:fluxunkdirmain}

On its first level, the
following records and sub-directories will be found in the \dir{fluxunk} directory:

\begin{DescriptionEnregistrement}{Main records and sub-directories in \dir{fluxunk}}{8.0cm}
\CharEnr
  {SIGNATURE\blank{3}}{$*12$}
  {Signature of the data structure ($\mathsf{SIGNA}=${\tt L\_FLUX\blank{6}})}
\IntEnr
  {STATE-VECTOR}{$40$}
  {Vector describing the various parameters associated with this data structure $\mathcal{S}^{f}_{i}$,
  as defined in \Sect{fluxunkstate}.}
\CharEnr
  {OPTION\blank{6}}{$*4$}
  {Type of leakage coefficients ({\tt 'LKRD'}: recover leakage coefficients in Macrolib; {\tt 'RHS'}: recover
  leakage coefficients in RHS flux object; {\tt 'B0'}: $B_0$; {\tt 'P0'}: $P_0$; {\tt 'B1'}: $B_1$; {\tt 'P1'}:
   $P_1$; {\tt 'B0TR'}: $B_0$ with transport correction; {\tt 'P0TR'}: $P_0$ with transport correction).}
\RealEnr
  {EPS-CONVERGE}{$5$}{}
  {Convergence parameters $\Delta_i^\epsilon$}
\IntEnr
  {KEYFLX\blank{6}}{$\mathcal{S}^{t}_{1}$}
  {Location in unknown vector of averaged regional flux $I_{r}$}
\OptRealEnr
  {K-EFFECTIVE\blank{1}}{$1$}{$\mathcal{S}^{f}_{6}\ge 1$}{}
  {Computed or imposed effective multiplication factor for direct eigenvalue problem,
  corresponding to the fundamental mode}
\OptRealEnr
  {AK-EFFECTIVE}{$1$}{${\mathcal{S}^{f}_{3}\over 10} \bmod 10 = 1$}{}
  {Computed effective multiplication factor for adjoint eigenvalue problem,
  corresponding to the fundamental mode.
  The theoretical value is equal
  to {\tt 'K-EFFECTIVE'} but difference may occurs for numerical reasons.}
\OptRealEnr
  {K-INFINITY\blank{2}}{$1$}{$\mathcal{S}^{f}_{6}\ge 2$}{}
  {Computed infinite multiplication constant for eigenvalue problem,
  corresponding to the fundamental mode}
\OptRealEnr
  {B2\blank{2}B1HOM\blank{3}}{$1$}{$\mathcal{S}^{f}_{6}\ge 1$}{cm$^{-2}$}
  {Homogeneous buckling $B^{2}$,
  corresponding to the fundamental mode}
\OptRealEnr
  {SPEC-RADIUS\blank{1}}{$1$}{$\mathcal{S}^{f}_{6}= -2$}{cm}
  {Spectral radius}
\OptRealEnr
  {DIFFHET\blank{5}}{$N_{\rm leak}\times G$}{$\mathcal{S}^{f}_{18}\ge 1$}{cm}
  {Multigroup leakage coefficients in each leakage zone and energy group $D_l^g$}
\OptIntEnr
  {IMERGE-LEAK\blank{1}}{$N_m$}{$\mathcal{S}^{f}_{18}\ge 1$}
  {Leakage zone index assigned to each material mixture $L_m^g$}
\OptRealEnr
  {B2\blank{2}HETE\blank{4}}{$3$}{$\mathcal{S}^{f}_{7} \ge 6$}{cm$^{-2}$}
  {Directional buckling components $B^{2}_{i}$,
  corresponding to the fundamental mode}
\OptRealEnr
  {GAMMA\blank{7}}{$G$}{$\mathcal{S}^{f}_{7}\ge 5$}{}
  {Gamma factors used with $B_n$--type streaming models.}
\DirlEnr
  {FLUX\blank{8}}{$\mathcal{S}^{f}_{1}$}
  {List of real arrays. Each component of this list is a real array of dimension $\mathcal{S}^{f}_{2}$
  containing the solution of a fixed source (\Eq{flux1}) or of a direct eigenvalue (\Eq{flux2}) equation,
  corresponding to the fundamental mode.}
\OptDirlEnr
  {AFLUX\blank{7}}{$\mathcal{S}^{f}_{1}$}{${\mathcal{S}^{f}_{3}\over 10} \bmod 10 = 1$}
  {List of real arrays. Each component of this list is a real array of dimension $\mathcal{S}^{f}_{2}$
  containing the solution of an adjoint eigenvalue (\Eq{flux3}) equation,
  corresponding to the fundamental mode.}
\OptDirlEnr
  {MODE\blank{8}}{$\mathcal{S}^{f}_{4}$}{$\mathcal{S}^{f}_{4}\ge 1$}
  {List of {\sl harmonic mode} sub-directories. Each component of this list follows
  the specification presented in \Sect{mode_spec}.}
\OptDirlEnr
  {DFLUX\blank{7}}{$\mathcal{S}^{f}_{5}$}{$\mathcal{S}^{f}_{3}=100$}
  {List of direct (explicit) GPT sub-directories. Each component of this list is a multigroup list of
  dimension $\mathcal{S}^{f}_{1}$. Each component of the multigroup list is a real array of dimension
  $\mathcal{S}^{f}_{2}$ containing the solution of a fixed source direct eigenvalue equation similar to \Eq{flux4}.}
\OptDirlEnr
  {ADFLUX\blank{6}}{$\mathcal{S}^{f}_{5}$}{$\mathcal{S}^{f}_{3}=1000$}
  {List of adjoint (implicit) GPT sub-directories. Each component of this list is a multigroup list of
  dimension $\mathcal{S}^{f}_{1}$. Each component of the multigroup list is a real array of dimension
  $\mathcal{S}^{f}_{2}$ containing the solution of a fixed source adjoint eigenvalue equation similar to \Eq{flux5}.}
\OptDirlEnr
  {DRIFT\blank{7}}{$\mathcal{S}^{f}_{1}$}{$\mathcal{S}^{t}_{12}=6$}
  {Drift coefficients used in nodal correction iterations. Each component of the multigroup list is a real array of dimension
  $6\times \mathcal{S}^{t}_{1}$.}
\end{DescriptionEnregistrement}

The convergence parameters $\Delta_i^\epsilon$ represent:
\begin{itemize}
\item $\Delta_1^\epsilon$ is the thermal (up-scattering) iteration flux convergence parameter
\item $\Delta_2^\epsilon$ is the outer iteration eigenvalue convergence parameter
\item $\Delta_3^\epsilon$ is the outer iteration flux convergence parameter
\item $\Delta_4^\epsilon$ is the GMRES convergence parameter used at inner iteration
\item $\Delta_5^\epsilon$ is the relaxation factor of the flux used in multiphysics applications. $\Delta_5^\epsilon=1$ is equivalent to no
relaxation.
\end{itemize}
\goodbreak

\subsection{The harmonic mode sub-directories in \dir{fluxunk}}\label{sect:mode_spec}

Each component of the list named {\tt 'MODE'} contains the information relative to a specific
harmonic mode.

\begin{DescriptionEnregistrement}{Component of the harmonic mode directory}{7.5cm}
\RealEnr
  {K-EFFECTIVE\blank{1}}{$1$}{}
  {Computed effective multiplication factor for eigenvalue problem,
  corresponding to the $\alpha$--th mode}
\DirlEnr
  {FLUX\blank{8}}{$\mathcal{S}^{f}_{1}$}
  {List of real arrays. Each component of this list is a real array of dimension $\mathcal{S}^{f}_{2}$
  containing the solution of the $\alpha$--th mode of a direct eigenvalue (\Eq{flux2}) equation.}
\OptDirlEnr
  {AFLUX\blank{7}}{$\mathcal{S}^{f}_{1}$}{${\mathcal{S}^{f}_{3}\over 10} \bmod 10 = 1$}
  {List of real arrays. Each component of this list is a real array of dimension $\mathcal{S}^{f}_{2}$
  containing the solution of the $\alpha$--th mode of an adjoint eigenvalue (\Eq{flux3}) equation.}
\end{DescriptionEnregistrement}

\eject
