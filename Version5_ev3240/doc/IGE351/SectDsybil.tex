\subsection{The \moc{sybilt} dependent records and sub-directories on a
\dir{tracking} directory}\label{sect:sybiltrackingdir}

When the \moc{SYBILT:} operator is used ($\mathsf{CDOOR}$={\tt 'SYBIL'}), the following elements in the vector
$\mathcal{S}^{t}_{i}$ will also be defined.

\begin{itemize}
\item The main SYBIL model $\mathcal{S}^{t}_{6}$

\begin{displaymath}
\mathcal{S}^{t}_{6} = \left\{
\begin{array}{rl}
 2 & \textrm{Pure geometry} \\
 3 & \textrm{Do-it-yourself geometry} \\
 4 & \textrm{2-D assembly geometry} \\
\end{array} \right.
\end{displaymath}

\item Minimum space required to store tracks for assembly geometry $\mathcal{S}^{t}_{7}$ 

\item Minimum space required to store interface currents for assembly geometry $\mathcal{S}^{t}_{8}$ 

\item Number of additional unknowns holding the interface currents
$\mathcal{S}^{t}_{9}$. These unknowns are used if and only if a current--based
inner iterative method is set (with option \moc{ARM}).

\end{itemize}

The following sub-directories will also be present on the main level of a \dir{tracking}
directory. 

\begin{DescriptionEnregistrement}{The \moc{sybilt} records and sub-directories in
\dir{tracking}}{8.0cm}\label{table:puregeom}
\RealEnr
  {EPSJ\blank{8}}{$1$}{$1$}
  {Stopping criterion for flux-current iterations of the interface current method}
\OptDirEnr
  {PURE-GEOM\blank{3}}{$\mathcal{S}^{t}_{6}=2$}
  {Sub-directory containing the data related to a pure geometry} 
\OptDirEnr
  {DOITYOURSELF}{$\mathcal{S}^{t}_{6}=3$}
  {Sub-directory containing the data related to a do-it-yourself geometry} 
\OptDirEnr
  {EURYDICE\blank{4}}{$\mathcal{S}^{t}_{6}=4$}
  {Sub-directory containing the data related to an assembly geometry} 
\end{DescriptionEnregistrement}

\vskip -0.4cm
\noindent
where the sub-directories in Table~\ref{table:puregeom} are described in the following subsections.

\subsubsection{The \moc{/PURE-GEOM/} sub-directory in \moc{sybilt}}\label{sect:puregeomtrackingdir}

\begin{DescriptionEnregistrement}{The contents of the \moc{sybilt}
\moc{/PURE-GEOM/} sub-directory}{8.0cm}
\IntEnr
  {PARAM\blank{7}}{$6$}
  {Record containing the parameters for a SYBIL tracking on a pure geometry $\mathcal{P}_{i}$}
\IntEnr
  {NCODE\blank{7}}{$6$}
  {Record containing the types of boundary conditions on each surface $N_{\beta,j}$}
\RealEnr
  {ZCODE\blank{7}}{$6$}{$1$}
  {Record containing the albedo value on each surface}
\OptRealEnr
  {XXX\blank{9}}{$\mathcal{P}_{4}+1$}{$\mathcal{P}_{4}\ge 1$}{cm}
  {$x-$directed mesh coordinates after mesh-splitting for type
   2, 5 and 7 geometries. Region-ordered radius after mesh-splitting for type 3 and 6
   geometries}
\OptRealEnr
  {YYY\blank{9}}{$\mathcal{P}_{5}+1$}{$\mathcal{P}_{5}\ge 1$}{cm}
  {$y-$directed mesh coordinates after mesh-splitting for type 5, 6 and 7 geometries}
\OptRealEnr
  {ZZZ\blank{9}}{$\mathcal{P}_{6}+1$}{$\mathcal{P}_{6}\ge 1$}{cm}
  {$z-$directed mesh coordinates after mesh-splitting for type 7 and 9 geometries}
\OptRealEnr
  {SIDE\blank{8}}{$1$}{$\mathcal{P}_{1}\ge 8$}{cm}
  {Side of a hexagon for type 8 and 9 geometries}
\end{DescriptionEnregistrement}

\vskip -0.2cm
\noindent
with the dimension parameter $\mathcal{P}_{i}$, representing:

\begin{itemize}
\item The type of geometry $\mathcal{P}_{1}$
\begin{displaymath}
\mathcal{P}_{1} = \left\{
\begin{array}{rl}
 2 & \textrm{Cartesian 1-D geometry} \\ 
 3 & \textrm{Tube 1-D geometry}  \\
 4 & \textrm{Spherical 1-D geometry}  \\
 5 & \textrm{Cartesian 2-D geometry}  \\
 6 & \textrm{Tube 2-D geometry}  \\
 7 & \textrm{Cartesian 3-D geometry}  \\
 8 & \textrm{Hexagonal 2-D geometry}   \\
 9 & \textrm{Hexagonal 3-D geometry}  \\
\end{array} \right.
\end{displaymath}

\item The type of hexagonal symmetry $\beta_{h}=\mathcal{P}_{2}$
\begin{displaymath}
\beta_{h} = \left\{
\begin{array}{rl}
 1 & \textrm{S30} \\
 2 & \textrm{SA60} \\
 3 & \textrm{SB60} \\
 4 & \textrm{S90} \\
 5 & \textrm{R120} \\
 6 & \textrm{R180} \\
 7 & \textrm{SA180} \\
 8 & \textrm{SB180} \\
 9 & \textrm{COMPLETE} \\
\end{array} \right.
\end{displaymath}

\item The quadrature parameter $\mathcal{P}_{3}$

\item The number of $x-$directed or radial mesh intervals in the geometry $\mathcal{P}_{4}$

\item The number of $y-$directed mesh intervals in the geometry $\mathcal{P}_{5}$

\item The number of $z-$directed mesh intervals in the geometry $\mathcal{P}_{6}$

\end{itemize}

The type of boundary conditions used will be defined in the following way
\begin{displaymath}
N_{\beta,j} = \left\{
\begin{array}{rl}
 0 & \textrm{Not used} \\
 1 & \textrm{Void boundary condition} \\
 2 & \textrm{Reflection boundary condition} \\
 3 & \textrm{Diagonal reflection boundary condition} \\
 4 & \textrm{Translation boundary condition condition} \\
 5 & \textrm{Symmetric reflection boundary condition} \\
 6 & \textrm{Albedo boundary condition} \\
\end{array} \right.
\end{displaymath}

\subsubsection{The \moc{/DOITYOURSELF/} sub-directory in \moc{sybilt}}\label{sect:doittrackingdir}

\vskip -0.9cm

\begin{DescriptionEnregistrement}{The contents of the \moc{sybilt}
\moc{/DOITYOURSELF/} sub-directory}{8.0cm}
\IntEnr
  {PARAM\blank{7}}{$3$}
  {Record containing the parameters for a SYBIL tracking on a do-it-yourself geometry
   $\mathcal{P}_{i}$} 
\IntEnr
  {NMC\blank{9}}{$M+1$}
  {Offset of the first region in each cell}
\RealEnr
  {RAYRE\blank{7}}{$N_r+M$}{cm}
  {Radius of the tubes in each cell}
\RealEnr
  {PROCEL\blank{6}}{$M,M$}{}
  {Geometric matrix}
\RealEnr
  {POURCE\blank{6}}{$M$}{}
  {Weight assigned to each cell}
\RealEnr
  {SURFA\blank{7}}{$M$}{cm$^{2}$}
  {Surface of each cell }
\end{DescriptionEnregistrement}
\noindent
with the dimension parameter $\mathcal{P}_{i}$, representing:

\begin{itemize}
\item The number of cells $\mathcal{P}_{1}=M$

\item The quadrature parameter $\mathcal{P}_{2}$

\item The statistical option $\mathcal{P}_{3}$
\begin{displaymath}
\mathcal{P}_{3} = \left\{
\begin{array}{rl}
 0 & \textrm{the statistical approximation is not used. Record {\tt 'PROCEL'} is used.} \\
 1 & \textrm{use the statistical approximation. Record {\tt 'PROCEL'} is not used.}
\end{array} \right.
\end{displaymath}

\end{itemize}

\clearpage

\subsubsection{The \moc{/EURYDICE/} sub-directory in \moc{sybilt}}\label{sect:eurydicetrackingdir}

\vskip -0.9cm

\begin{DescriptionEnregistrement}{The contents of the \moc{sybilt}
\moc{/EURYDICE/} sub-directory}{8.0cm}
\IntEnr
  {PARAM\blank{7}}{$16$}
{Record containing the parameters for a SYBIL tracking on an assembly geometry
   $\mathcal{P}_{i}$}
\RealEnr
  {XX\blank{10}}{$\mathcal{P}_{6}$}{cm}
  {$x-$thickness of the generating cells}
\RealEnr
  {YY\blank{10}}{$\mathcal{P}_{6}$}{cm}
  {$y-$thickness of the generating cells}
\IntEnr
  {LSECT\blank{7}}{$\mathcal{P}_{6}$}
  {Type of sectorization for each each generating cell. Equal to zero for
   non-sectorized cells. Allowed values are defined as $F_{\mathrm{sec}}$ in \Sect{geometrydirmain}}
\IntEnr
  {NMC\blank{9}}{$\mathcal{P}_{6}+1$}
  {Offset of the first region index in each generating cell}
\IntEnr
  {NMCR\blank{8}}{$\mathcal{P}_{6}+1$}
  {Offset of the first radius index in each generating cell. Equal to
   {\tt NMC}, unless the cell is sectorized.}
\RealEnr
  {RAYRE\blank{7}}{$M_r$}{cm}
  {Radius of the tubes in each generating cell. $M_r=${\tt NMCR(}$\mathcal{P}_{6}+1${\tt )}}
\IntEnr
  {MAIL\blank{8}}{$2,\mathcal{P}_{6}$}
  {Offsets of the first tracking information in each generating cell. {\tt
   MAIL(1,:)} contains offsets for the integer array {\tt ZMAILI}; {\tt
   MAIL(2,:)} contains offsets for the real array {\tt ZMAILR}.}
\IntEnr
  {ZMAILI\blank{6}}{$\mathcal{P}_{15}$}
  {The integer tracking information}
\RealEnr
  {ZMAILR\blank{6}}{$\mathcal{P}_{16}$}{cm}
  {The tracking lengths}
\IntEnr
  {IFR\blank{9}}{$\mathcal{P}_{4},\mathcal{P}_{14}$}
  {Index numbers of incoming currents}
\RealEnr
  {ALB\blank{9}}{$\mathcal{P}_{4},\mathcal{P}_{14}$}{}
  {Albedo or transmission factors corresponding to incoming currents}
\IntEnr
  {INUM\blank{8}}{$\mathcal{P}_{4}$}
  {Index  number of the merge cell associated to each cell of the assembly}
\IntEnr
  {MIX\blank{9}}{$\mathcal{P}_{5},\mathcal{P}_{14}$}
  {Index  numbers of outgoing currents}
\RealEnr
  {DVX\blank{9}}{$\mathcal{P}_{5},\mathcal{P}_{14}$}{}
  {Weights corresponding to outgoing currents}
\IntEnr
  {IGEN\blank{8}}{$\mathcal{P}_{5}$}
  {Index number of the generating cell associated to each merged cell}
\end{DescriptionEnregistrement}

\noindent
with the dimension parameter $\mathcal{P}_{i}$, representing:

\begin{itemize}

\item The type of hexagonal symmetry $\mathcal{P}_{1}$
\begin{displaymath}
\mathcal{P}_{1} = \left\{
\begin{array}{rl}
 0 & \textrm{Cartesian assembly} \\
 1 & \textrm{S30} \\
 2 & \textrm{SA60} \\
 3 & \textrm{SB60} \\
 4 & \textrm{S90} \\
 5 & \textrm{R120} \\
 6 & \textrm{R180} \\
 7 & \textrm{SA180} \\
 8 & \textrm{SB180} \\
 9 & \textrm{COMPLETE} \\
\end{array} \right.
\end{displaymath}

\item The type of multicell approximation $\mathcal{P}_{2}$
\begin{displaymath}
\mathcal{P}_{2} = \left\{
\begin{array}{ll}
1 & \textrm{Roth approximation}\\
2 & \textrm{Roth$\times 4$ or Roth$\times 6$ approximation}\\
3 & \textrm{DP-0 approximation}\\
4 & \textrm{DP-1 approximation} \end{array} \right.
\end{displaymath}

\item The type of cylinderization $\mathcal{P}_{3}$
\begin{displaymath}
\mathcal{P}_{3} = \left\{
\begin{array}{ll}
1 & \textrm{Askew cylinderization}\\
2 & \textrm{Wigner cylinderization}\\
3 & \textrm{Sanchez cylinderization} \end{array} \right.
\end{displaymath}

\item The total number of cells $\mathcal{P}_{4}$

\item The number of merged cells $\mathcal{P}_{5}$

\item The number of generating cells $\mathcal{P}_{6}$

\item The number of distinct interface currents $\mathcal{P}_{7}$

\item The number of angles for 2-D quadrature $\mathcal{P}_{8}$

\item The number of segments for 2-D quadrature $\mathcal{P}_{9}$

\item The number of segments for homogeneous 2-D cells $\mathcal{P}_{10}$

\item The number of segments for 1-D cells $\mathcal{P}_{11}$

\item The track normalization option $\mathcal{P}_{12}$
\begin{displaymath}
\mathcal{P}_{12} = \left\{
\begin{array}{rl}
 0 & \textrm{Normalize the tracks} \\
 1 & \textrm{Do not normalize the tracks} \\
\end{array} \right.
\end{displaymath}

\item The type of quadrature in angle and space $\mathcal{P}_{13}$
\begin{displaymath}
\mathcal{P}_{13} = \left\{
\begin{array}{rl}
 0 & \textrm{Gauss quadrature} \\
 1 & \textrm{Equal weight quadrature} \\
\end{array} \right.
\end{displaymath}

\item The number of outgoing interface currents per cell $\mathcal{P}_{14}$

\item The number of integer array elements in the tracking arrays $\mathcal{P}_{15}$

\item The number of real array elements in the tracking arrays $\mathcal{P}_{16}$

\end{itemize}

\eject
