\subsubsection{The {\tt SYBILT:} tracking module}\label{sect:SYBILData}

The geometries that can be treated by the module \moc{SYBILT:} are

\begin{enumerate}

\item The homogeneous geometry \moc{HOMOGE}.

\item The one-dimensional geometries \moc{SPHERE}, \moc{TUBE} and
\moc{CAR1D}.\cite{ALCOL}

\item The two-dimensional geometries \moc{CAR2D} and \moc{HEX} including
respectively \moc{CARCEL} and \moc{HEXCEL} sub-geometries as well as 
\moc{VIRTUAL}
sub-geometries. 

\item $S_{ij}$--type two-dimensional non-standard geometries.\cite{Apollo}

\item The double heterogeneity option.\cite{BIHET}

\end{enumerate}

The calling specification for this module is:

\begin{DataStructure}{Structure \dstr{SYBILT:}}
\dusa{TRKNAM}
\moc{:=} \moc{SYBILT:} $[$ \dusa{TRKNAM} $]$
\dusa{GEONAM} \moc{::} \dstr{desctrack} \dstr{descsybil}
\end{DataStructure}

\noindent  where
\begin{ListeDeDescription}{mmmmmmm}

\item[\dusa{TRKNAM}] {\tt character*12} name of the \dds{tracking} data
structure that will contain region volume and surface area vectors in
addition to region identification pointers and other tracking information.
If \dusa{TRKNAM} also appears on the RHS, the previous tracking 
parameters will be applied by default on the current geometry.

\item[\dusa{GEONAM}] {\tt character*12} name of the \dds{geometry} data
structure.

\item[\dstr{desctrack}] structure describing the general tracking data (see
\Sect{TRKData})

\item[\dstr{descsybil}] structure describing the transport tracking data
specific to \moc{SYBILT:}.

\end{ListeDeDescription}

\vskip 0.15cm

The \moc{SYBILT:} specific tracking data in \dstr{descsybil} is defined as

\begin{DataStructure}{Structure \dstr{descsybil}}
$[$ \moc{MAXJ} \dusa{maxcur} $]$  $[$ \moc{MAXZ} \dusa{maxint} $]$ \\
$[$ \moc{HALT} $]$ \\
$[$ \moc{QUA1} \dusa{iqua1} $]$ $[$ \moc{QUA2} \dusa{iqua2}
\dusa{nsegment} $]$ $[$ $\{$ \moc{EQW} $|$ \moc{GAUS} $\}$ $]$ \\
$[$ $\{$ \moc{ROTH} $|$ \moc{ROT+} $|$ \moc{DP00} $|$ \moc{DP01} $\}$ $]$ \\
$[$ $\{$ \moc{WIGN} $|$ \moc{ASKE} $|$ \moc{SANC} $\}$ $]$ $[$ \moc{LIGN} $]$
$[$ \moc{RECT} $]$ \\
$[$ \moc{EPSJ} \dusa{epsj} $]$ \\
$[~[$ \moc{QUAB} \dusa{iquab} $]~[~\{$ \moc{SAPO} $|$ \moc{HEBE} $|$ \moc{SLSI} $[$ \dusa{frtm} $]~\}~]~]$ \\
{\tt ;}
\end{DataStructure}

\noindent where

\begin{ListeDeDescription}{mmmmmmm}

\item[\moc{MAXJ}] keyword to specify the maximum number of interface currents
surrounding the blocks in the calculations. 

\item[\dusa{maxcur}] the maximum number of interface currents surrounding the
blocks. The default value is \dusa{maxcur}=max(18,4$\times$\dusa{maxreg}) for the
\moc{SYBILT:} module.

\item[\moc{MAXZ}] keyword to specify the maximum amount of memory required to
store the integration lines. An insufficiently large value can lead to an
execution failure (core dump).

\item[\dusa{maxint}] the maximum amount of memory required to store the
integration lines. The default value is \dusa{maxint}=10000.

\item[\moc{HALT}] keyword to specify that the program is to be stopped at the
end of the geometry calculations. This option permits the geometry inputs to be
checked, the number of blocks and interface currents to be calculated, and a
conservative estimate of the memory required for storing the tracks to be made
for mixed geometries.

\item[\moc{QUA1}] keyword to specify the one-dimensional integration
parameters.

\item[\dusa{iqua1}] number of basis points for the angular integration of the
blocks in a one-dimensional geometry. This parameter is not used for
\moc{CAR1D} geometries. If a Gauss-Legendre or Gauss-Jacobi quadrature is used,
the values of \dusa{iqua1} allowed are: 1 to 20, 24, 28, 32 or 64. The default
value is \dusa{iqua1}=5. 

\item[\moc{QUA2}] keyword to specify the two-dimensional integration
parameters.

\item[\dusa{iqua2}] number of basis points for the angular integration of the
blocks in a two-dimensional geometry appearing during assembly  
calculations. If a Gauss-Legendre or Gauss-Jacobi formula is used the values
allowed for \dusa{iqua2} are: 1 to 20, 24, 28, 32 or 64. The default value is
\dusa{iqua2}=3 and represents the number of angles in ($0,\pi/4$) for
Cartesian geometries and  ($0,\pi/6$) for hexagonal geometries. 

\item[\dusa{nsegment}] number of basis points for the spatial integration of
the blocks in a two-dimensional geometry appearing during assembly 
calculations. The values of \dusa{nsegment} allowed are: 1 to 10. The default
value is \dusa{nsegment}=3.

\item[\moc{EQW}] keyword to specify the use of equal-weight quadrature.

\item[\moc{GAUS}] keyword to specify the use of the Gauss-Legendre or the
Gauss-Jacobi quadrature. This is the default option.

\item[\moc{ROTH}] keyword to specify that the isotropic ($DP_{0}$) components
of the inter-cell current is used with the incoming current being averaged over
all the faces surrounding a cell. The global collision matrix is calculated in a
annular model. Only used when 2--d assembly of cells are considered.

\item[\moc{ROT+}] keyword to specify that the isotropic ($DP_{0}$) components
of the inter-cell current is used. The global collision matrix is calculated in
a annular model. Only used when 2--d assembly of cells are considered.

\item[\moc{DP00}] keyword to specify that the isotropic ($DP_{0}$) components
of the inter-cell current is used. The global collision matrix are computed
explicitly. Only used when 2--d assembly of cells are considered.

\item[\moc{DP01}] keyword to specify that the linearly anisotropic ($DP_{1}$)
components of the inter-cell current are used. This hypothesis implies 12
currents per cell in a cartesian geometry and 18 currents per cell for an
hexagonal geometry. Linearly anisotropic reflection is used. Only used when 2--d
assembly of cells are considered.

\item[\moc{WIGN}] keyword to specify the use of a {\sl Wigner} cylinderization
which preserves the volume of the external crown. This applies only in cases
where the external surface is annular using the \moc{ROTH} or \moc{ROT+}
options. Only used when 2--d assembly of cells are considered. Note that an
assembly of rectangular cells having unequal volumes cannot use a {\sl Wigner}
cylinderization.  

\item[\moc{ASKE}] keyword to specify the use of an {\sl Askew} cylinderization
which preserves both the external surface of the cells and the material balance
of the external crown (by a modification of its concentration). This applies
only in cases where the external surface is annular using the \moc{ROTH} or
\moc{ROT+} options. Only used when 2--d assembly of cells are considered. Note
that an assembly of rectangular cells having unequal volumes can use an
{\sl Askew} cylinderization.  

\item[\moc{SANC}] keyword to specify the use of a {\sl Sanchez} cylinderization.
This model uses a {\sl Wigner} cylinderization for computing the collision $P_{ij}$
and leakage $P_{iS}$ probabilities. However, the reciprocity and conservation
relations used to compute the incoming $P_{Sj}$ and transmission $P_{SS}$
probabilities are defined in the rectangular cell (with the exact
surface).\cite{SANCHEZ} 
This applies where the external surface is annular using the \moc{ROTH} or
\moc{ROT+} options. Only used when 2--d assembly of cells are considered. Note
that an assembly of rectangular cells having unequal volumes can use a
{\sl Sanchez} cylinderization. This is the default option.

\item[\moc{LIGN}] keyword to specify that all the integration lines are to be
printed. This option should only be used when absolutely necessary because it
generates a rather large amount of output. Only used when 2--d assembly of cells
are considered.

\item[\moc{RECT}] keyword to specify that square cells are to be treated as if
they were rectangular cells, with the inherent loss in performance that this
entails. This option is of purely academic interest.

\item[\moc{EPSJ}] keyword to specify the stopping criterion for the flux-current iterations of the
interface current method in case the {\tt ARM} keyword is set in the {\tt ASM:} module or in
a resonance self-shielding module ({\tt SHI:}, {\tt USS:}, etc.).

\item[\dusa{epsj}] the stopping criterion value. The default value is \dusa{epsj} $= 0.5 \times 10^{-5}$.

\item[\moc{QUAB}] keyword to specify the number of basis point for the
numerical integration of each micro-structure in cases involving double
heterogeneity (Bihet).

\item[\dusa{iquab}] the number of basis point for the numerical integration of
the collision probabilities in the micro-volumes using the  Gauss-Jacobi
formula. The values permitted are: 1 to 20, 24, 28, 32 or  64. The default value
is \dusa{iquab}=5. If \dusa{iquab} is negative, its absolute value will be used in the She-Liu-Shi approach to determine the
split level in the tracking used to compute the probability collisions.

\item[\moc{SAPO}] use the Sanchez-Pomraning double-heterogeneity model.\cite{sapo}

\item[\moc{HEBE}] use the Hebert double-heterogeneity model (default option).\cite{BIHET}

\item[\moc{SLSI}] use the She-Liu-Shi double-heterogeneity model without shadow effect.\cite{She2017}

\item[\dusa{frtm}] the minimum microstructure volume fraction used to compute the size of the equivalent cylinder in She-Liu-Shi approach. The default value is \dusa{frtm} $=0.05$.

\end{ListeDeDescription}
\eject
