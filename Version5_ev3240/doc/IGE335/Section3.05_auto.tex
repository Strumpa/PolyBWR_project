\subsection{The {\tt AUTO:} module}\label{sect:AUTOData}

The Autosecol self-shielding module in DRAGON, called {\tt AUTO:}, allows the
correction of the microscopic cross sections to take into account the
self-shielding effects related to the resonant isotopes.\cite{autosecol}

\vskip 0.08cm

{\sl Autolib data} is a fine-group representation of microscopic cross-section data for the resonant isotopes available in a
{\sl Draglib} or {\sl APOLIB-2} cross-section library. Each fine group in the Autolib has a lethargy width which is an integer multiple of an
{\sl elementary lethargy width}. Elastic slowing-down scattering is assumed for the resonant isotopes.

Integrating the Livolant-Jeanpierre equation over a fine group $g$, the Autosecol equation is written
\begin{equation}
\bff(\Omega)\cdot\bff(\nabla)\varphi_g(\bff(r),\bff(\Omega))\,+\,\Sigma_g(\bff(r))\,\varphi_g(\bff(r),\bff(\Omega))\,=\,{1\over 4\pi} \left[ \Sigma_{{\rm s},g}^+(\bff(r)) \, + \,\sum_h \Sigma_{{\rm s},j,g \leftarrow h}^{*} \, \varphi_h(\bff(r)) {\Delta u_h\over \Delta u_g} \right]
\label{eq:auto1}
\end{equation}

\noindent where the group integrated fine structure function is written
\begin{equation}
\varphi_g(\bff(r))={1\over \Delta u_g}\int_{u_{g-1}}^{u_g} du\, \varphi(\bff(r),u)
\label{eq:auto2}
\end{equation}

\noindent and where the $+$ and $*$ subscripts identify non-resonant and resonant isotopes respectively.

\vskip 0.08cm

The {\sl Autosecol method} consists to solve the Livolant-Jeanpierre equation over the Autolib energy mesh using a solution
technique of the Boltzmann transport equation available in DRAGON.\cite{PIP2009} The Autosecol method
is an accurate self-shielding technique relying on the fine-group solution of an heterogeneous transport equation. This approach may require
substantial CPU resources in actual production cases.

\vskip 0.08cm

Resonant isotopes are identified as such by the \dusa{inrs} parameter, as defined in
\Sect{LIBData}. The Autosecol self-shielding module is based on the following models:

\begin{itemize}
\item The Livolant-Jeanpierre flux factorization and approximations are used to
uncouple the self-shielding treatment from the main flux calculation;
\item The resonant cross sections are represented using {\sl Autolib data} 
recovered by the \moc{LIB:} module.
\item Probability tables are used in the unresolved energy domain to randomly
sample cross-section data into the Autolib fine mesh. The keyword \moc{SUBG} {\sl must} be
set in module {\tt LIB:}.
\item The resonant fine structure values $\varphi_g(\bff(r))$ are obtained as a solution
of the Autosecol Eq.~(\ref{eq:auto1}) over the Autolib fine mesh;
\item The flux can be solved using collision probabilities, or using {\sl any}
flux solution technique for which a tracking module is available;
\item All resonant isotopes with the same \dusa{inrs} index (see Sect.~\ref{sect:descmix})
are computed simultanously;
\item The distributed self-shielded effect is automatically taken into account
if different mixture indices are assigned to different regions inside the
resonant part of the cell. The rim effect can be computed by dividing the fuel
into "onion rings" and by assigning different mixture indices to them. 
\item A SPH (superhomog\'en\'eisation) equivalence is performed to correct the
self-shielded cross sections from the non-linear effects related to the
heterogeneity of the geometry.
\end{itemize}

\vskip 0.2cm

The general format of the data for this module is:

\begin{DataStructure}{Structure \dstr{AUTO:}}
\dusa{MICLIB} \moc{:=} \moc{AUTO:} \dusa{MICLIB\_SG} $[$ \dusa{MICLIB} $]$
\dusa{TRKNAM} $[$ \dusa{TRKFIL} $]$ \moc{::} \dstr{descauto}
\end{DataStructure}

\noindent where

\begin{ListeDeDescription}{mmmmmmmm}

\item[\dusa{MICLIB}] {\tt character*12} name of the \dds{microlib} that will
contain the microscopic and macroscopic cross sections updated by the
self-shielding module. If
\dusa{MICLIB} appears on both LHS and RHS, it is updated; otherwise,
\dusa{MICLIB} is created.

\item[\dusa{MICLIB\_SG}] {\tt character*12} name of the \dds{microlib} builded
by module \moc{LIB:} and containing probability table information for the unresolved
domain.

\item[\dusa{TRKNAM}] {\tt character*12} name of the required \dds{tracking}
data structure.

\item[\dusa{TRKFIL}] {\tt character*12} name of the sequential binary tracking
file used to store the tracks lengths. This file is given if and only if it was
required in the previous tracking module call (see \Sect{TRKData}).

\item[\dstr{descauto}] structure describing the self-shielding options.

\end{ListeDeDescription}

\subsubsection{Data input for module {\tt AUTO:}}\label{sect:descauto}

\begin{DataStructure}{Structure \dstr{descauto}}
$[$ \moc{EDIT} \dusa{iprint} $]$ \\
$[$ \moc{GRMIN}  \dusa{lgrmin} $]~~[$ \moc{GRMAX}  \dusa{lgrmax} $]$~~
$[$ \moc{PASS} \dusa{ipass} $]~~[~\{$ \moc{SPH} $|$ \moc{NOSP} $\}~]$~~$[$ $\{$ \moc{TRAN} $|$ \moc{NOTR} $\}$ $]$ \\ 
$[$ $\{$ \moc{PIJ} $|$ \moc{ARM} $\}$ $]$ \\
$[[$ \moc{DILU} \dusa{isot\_d} \dusa{dilut} $]]$ \\
$[$ \moc{KERN} \dusa{ialter} $]~~[$ \moc{MAXT} \dusa{maxtra} $]$ \\
$[$~\moc{SEED} \dusa{iseed}~$]$ \\
$[$ \moc{CALC} \\
~~~~$[[$ \moc{REGI} \dusa{suffix} $[[$ \dusa{isot} $\{$ \moc{ALL} $|$
(\dusa{imix}(i),i=1,\dusa{nmix}) $\}$ $]]$ \\
~~~~$]]$ \\
\moc{ENDC} $]$ \\
{\tt ;}
\end{DataStructure}

\noindent where

\begin{ListeDeDescription}{mmmmmmmm}

\item[\moc{EDIT}] keyword used to modify the print level \dusa{iprint}.

\item[\dusa{iprint}] index used to control the printing of this module. The
amount of output produced by this tracking module will vary substantially
depending on the print level specified. 

\item[\moc{GRMIN}] keyword to specify the minimum group number considered
during the self-shielding process.

\item[\dusa{lgrmin}] first group number considered during the
self-shielding process. By default, \dusa{lgrmin} is set to the first group
number containing self-shielding data in the library.

\item[\moc{GRMAX}]  keyword to specify the maximum group number considered
during the self-shielding process.

\item[\dusa{lgrmax}] last group number considered during the self-shielding
process. By default, \dusa{lgrmax} is set is set to the last group
number containing self-shielding data in the library.

\item[\moc{PASS}]  keyword to specify the number of outer iterations during
the self-shielding process. If all \dusa{inrs} indices are set to one in module \moc{LIB:},
these iterations are not required.

\item[\dusa{ipass}] the number of iterations. The default is \dusa{ipass} $=1$ if
\dusa{MICLIB} is created.

\item[\moc{SPH}] keyword to activate the SPH equivalence scheme which
modifies the self-shielded averaged neutron fluxes in
heterogeneous geometries (default option).

\item[\moc{NOSP}] keyword to deactivate the SPH equivalence scheme which
modifies the self-shielded averaged neutron fluxes in heterogeneous geometries.

\item[\moc{TRAN}] keyword to activate the transport correction option for
self-shielding calculations (see \moc{CTRA} in \Sectand{MACData}{LIBData}). This
is the default option.

\item[\moc{NOTR}] keyword to deactivate the transport correction option for
self-shielding calculations (see \moc{CTRA} in \Sectand{MACData}{LIBData}).

\item[\moc{PIJ}] keyword to specify the use of complete collision
probabilities in the subgroup and SPH equivalence calculations of {\tt AUTO:}.
This is the default option for \moc{EXCELT:} and \moc{SYBILT:} trackings.
This option is not available for \moc{MCCGT:} trackings.

\item[\moc{ARM}] keyword to specify the use of iterative flux techniques
in the subgroup and SPH equivalence calculations of {\tt AUTO:}.
This is the default option for \moc{MCCGT:} trackings.

\item[\moc{DILU}]  keyword to input an additional microscopic dilution value for a specific isotope. By default, no dilution
source other than $\Sigma_{{\rm s},g}^+(\bff(r))$ is used.

\item[\dusa{isot\_d}] {\tt character*8} alias name of the specific isotope.

\item[\dusa{dilut}] dilution value in barn.

\item[\moc{KERN}]  keyword to input the type of elastic slowing-down kernel.

\item[\dusa{ialter}] integer value indicating the type:
$$
\textsl{ialter} = \left\{
\begin{array}{ll}
0 & \textrm{use exact elastic kernel} \\
1 & \textrm{use an approximate kernel for the resonant isotopes.}
\end{array} \right.
$$

\item[\moc{MAXT}]  keyword to input a maximum storage size for the slowing-down kernel values.

\item[\dusa{maxtra}] integer value indicating the storage size. The default value is \dusa{maxtra} $=$ 10000.

\item[\moc{SEED}] keyword used to set the initial seed integer for the random number generator used in
the unresolved energy domain. By default, the seed integer is set from the processor clock.

\item[\dusa{iseed}] initial seed integer.

\item[\moc{CALC}] keyword to activate the simplified self-shielding
approximation in which a single self-shielded isotope is shared by many
resonant mixtures.

\item[\moc{REGI}] keyword to specify a set of isotopes and mixtures that
will be self-shielded together. All the self-shielded isotopes in this group
will share the same 4--digit suffix.

\item[\dusa{suffix}] {\tt character*4} suffix for the isotope names in this
group

\item[\dusa{isot}] {\tt character*8} alias name of a self-shielded isotope in this
group

\item[\moc{ALL}] keyword to specify that a unique self-shielded isotope will be
made for the complete domain

\item[\dusa{imix}] list of mixture indices that will share the same self-shielded
isotope

\item[\dusa{nmix}] number of mixtures that will share the same self-shielded
isotope

\item[\moc{ENDC}] end of \moc{CALC} data keyword

\end{ListeDeDescription}

\vskip 0.15cm

Here is an example of the data structure corresponding to a production case where
only $^{238}$U is assumed to show distributed self-shielding effects:

\begin{verbatim}
LIBRARY2 := AUTO: LIBRARY TRACK ::
     CALC REGI W1 PU239 ALL
          REGI W1 PU241 ALL
          REGI W1 PU240 ALL
          REGI W1 PU242 ALL
          REGI W1 U235 ALL
          REGI W1 U236 ALL
          REGI W1 PU238 ALL
          REGI W1 U234 ALL
          REGI W1 AM241 ALL
          REGI W1 NP237 ALL
          REGI W1 ZRNAT ALL
          REGI W1 U238 <<COMB0101>> <<COMB0201>> <<COMB0301>>
                       <<COMB0401>> <<COMB0501>>
          REGI W2 U238 <<COMB0102>> <<COMB0202>> <<COMB0302>>
                       <<COMB0402>> <<COMB0502>>
          REGI W3 U238 <<COMB0103>> <<COMB0203>> <<COMB0303>>
                       <<COMB0403>> <<COMB0503>>
          REGI W4 U238 <<COMB0104>> <<COMB0204>> <<COMB0304>>
                       <<COMB0404>> <<COMB0504>>
          REGI W5 U238 <<COMB0105>> <<COMB0205>> <<COMB0305>>
                       <<COMB0405>> <<COMB0505>>
          REGI W6 U238 <<COMB0106>> <<COMB0206>> <<COMB0306>>
                       <<COMB0406>> <<COMB0506>>
     ENDC ;
\end{verbatim}

\vskip 0.15cm

In this case, $^{238}$U is self-shielded within six distributed regions (labeled
{\tt W1} to {\tt W6}) and each of these regions are merging volumes belonging
to five different fuel rods. The mixture indices of the 30 resonant volumes belonging
to the fuel are CLE-2000 variables labeled {\tt <<COMB0101>>} to {\tt <<COMB0506>>}.

\eject
