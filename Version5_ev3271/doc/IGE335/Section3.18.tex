\subsection{The \moc{TLM:} module}\label{sect:TLMData}

The \moc{TLM:} module has been designed to generate a Matlab \moc{m-file} (in an \moc{ASCII} format) that contains the instructions for
plotting the tracking lines generated by the \moc{NXT:} module or by the \moc{SALT:} module (\moc{LONG} option).\cite{Plamondon2006}
The \moc{TLM:} module is activated using the following list of commands:

\begin{DataStructure}{Structure \dstr{TLM:}}
\dusa{MFILE} \moc{:=} \moc{TLM:} \dusa{MFILE}  \dusa{TRKNAM} \dusa{TRKFIL} 
\moc{::} \dstr{desctlm}
\end{DataStructure}

\noindent
 where

\begin{ListeDeDescription}{mmmmmmmm}

\item[\dusa{MFILE}] \verb|character*12| name of the \moc{ASCII} Matlab \moc{m-file} data structure that
will contain the instructions for plotting the tracking lines.

\item[\dusa{TRKNAM}] \verb|character*12| name of the \dds{tracking} data structure that
will contain region volume and surface area vectors in addition to region
identification pointers and other tracking information.

\item[\dusa{TRKFIL}] \verb|character*12| name of the sequential binary tracking file 
used to store the tracks lengths.\cite{Marleau2001}  

\item[\dstr{desctlm}] structure describing the type of graphics generated (see \Sect{desctlm}).

\end{ListeDeDescription}

\subsubsection{Data input for module \moc{TLM:}}\label{sect:desctlm}

\begin{DataStructure}{Structure \dstr{desctlm}}
$[$ \moc{EDIT} \dusa{iprint} $]$ \\
$[$ \moc{MIXTURE} $]$ \\
$[$ \moc{NTPO} \dusa{nplots} $]$ \\
( $\{$ \\
\hspace{0.4cm}   \moc{POINTS} $[$ \moc{NoPause} $]$ $|$ \\
\hspace{0.4cm}   \moc{DIRECTIONS} $[$ \moc{NoPause} $]$ \moc{DIR} \dusa{idir} $[$ \moc{PLAN} \dusa{iplan} 
$\{$ \moc{U} \dusa{iuv} $|$ \moc{V} \dusa{iuv} $\}$ $]$ $|$ \\
\hspace{0.4cm}   \moc{PLANP} $[$ \moc{NoPause} $]$ \moc{DIR} \dusa{idir} \moc{DIST} \dusa{dist} $[$ \moc{PLAN} \dusa{iplan} $]$ $|$\\
\hspace{0.4cm}   \moc{PLANA} $[$ \moc{NoPause} $]$ \moc{A} \dusa{a} \moc{B} \dusa{b} $[$ \moc{C} \dusa{c} $]$ \moc{D} \dusa{d}\\
\hspace{0.4cm} $\}$ , \dusa{iplot}=$1$, \dusa{nplots} )
\end{DataStructure}

\noindent
 where

\begin{ListeDeDescription}{mmmmmmmm}   

\item[\moc{EDIT}] keyword used to modify the print level \dusa{iprint}.

\item[\dusa{iprint}] index used to control the printing in this module. It must be set to 0 if no printing on the output
file is required. 

\item[\moc{MIXTURE}] keyword to set drawing colors as a function of mixtures. By default, colors are set according to region indices.

\item[\moc{NTPO}] keyword to specify the number of figures to draw.

\item[\dusa{nplots}] integer value for the number of figures to draw.

\item[\moc{POINTS}] keyword to specify that the figure will illustrate the intersection points between the lines and the external faces of the geometry.

\item[\moc{DIRECTIONS}] keyword to specify that the figure will illustrate the lines crossing each region as well as the intersection points between the lines
and the external faces of the geometry.

\item[\moc{PLANP}] keyword to specify that the figure will illustrate the points crossing a plane normal to the line direction.

\item[\moc{PLANA}] keyword to specify that the figure will illustrate the points crossing an arbitrary surface in 3-D or line in 2-D. The equation for the
surface in 3-D is~:
$$
\textit{a} X + \textit{b} Y + \textit{c} Z =\textit{d} 
$$
while the equation for the line in 2-D is~:
$$
\textit{a} X + \textit{b} Y =\textit{d} 
$$

\item[\moc{NoPause}] keyword to specify that all the lines the lines must be drawn without Matlab pause. By default, there is a pause after all the points
associated with an external surface and all the lines associated with a region are drawn.

\item[\moc{DIR}] keyword to specify line direction to draw.

\item[\dusa{idir}] integer value to identify the track direction to draw. In the case where \dusa{idir}=0, all the directions will be drawn. A value of  
\dusa{idir}=0 for 2-D geometry is generally acceptable. However, for 3-D geometry the number of lines generated is such that the figure becomes a mess and it
is generally more convenient to draw the lines direction per direction.

\item[\moc{PLAN}] keyword to specify which of the three planes normal to the specified direction in 3-D will be considered for drawing. This plane is defined
by the axes $U-V$. Used only for 3-D geometries.

\item[\dusa{iplan} ] integer value to identify which of the three planes normal to the specified direction in 3-D will be considered for drawing. the only
values permitted are 0, 1, 2 or 3. When a value of 0 is specified (default) all three planes will be drawn. Used only for 3-D geometries.

\item[\moc{U}] keyword to specify that the all the lines in the $V$ axis associated with a position on the $U$ axis will be drawn. Used only for 3-D geometries.

\item[\moc{V}] keyword to specify that the all the lines in the $U$ axis associated with a position on the $V$ axis will be drawn. Used only for 3-D geometries.

\item[\dusa{iuv}] integer value to identify the position on the $U$ or $V$ axis to be drawn. Used only for 3-D geometries.

\item[\moc{DIST}] keyword to specify the distance between the plane normal to the line direction and the origin.

\item[\dusa{dist} ] real or double precision value for the distance of the plane from the origin.

\item[\moc{A}] keyword to specify the value of \dusa{a} for an arbitrary plane or line.

\item[\dusa{a} ] real or double precision value \dusa{a}.

\item[\moc{B}] keyword to specify the value of \dusa{b} for an arbitrary plane or line.

\item[\dusa{b} ] real or double precision value \dusa{b}.

\item[\moc{C}] keyword to specify the value of \dusa{c} for an arbitrary plane.

\item[\dusa{b} ] real or double precision value \dusa{c}.

\item[\moc{D}] keyword to specify the value of \dusa{d} for an arbitrary plane or line.

\item[\dusa{d} ] real or double precision value \dusa{d}.
\end{ListeDeDescription}

\eject
