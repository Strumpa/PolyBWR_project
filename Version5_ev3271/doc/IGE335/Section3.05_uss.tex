\subsection{The {\tt USS:} module}\label{sect:USSData}

The universal self-shielding module in DRAGON, called {\tt USS:}, allows the
correction of the microscopic cross sections to take into account the
self-shielding effects related to the resonant isotopes. These isotopes are
identified as such by the \dusa{inrs}
parameter, as defined in \Sect{LIBData}. The universal
self-shielding module is based on the following models:

\begin{itemize}
\item The Livolant-Jeanpierre flux factorization and approximations are used to
uncouple the self-shielding treatment from the main flux calculation;
\item The resonant cross sections are represented using probability
tables computed in the \moc{LIB:} module (the keyword \moc{SUBG} or \moc{PTSL} {\sl must} be
used). Two approaches can be used to compute the probability tables:
\begin{enumerate}
\item Physical probability tables can be computed using a RMS approach similar
to the one used in Wims-7 and Helios.\cite{subg} In this case, the slowing-down operator of
each resonant isotope is represented as a pure ST\cite{st}, ST/IR or ST/WR approximation;
\item Mathematical probability tables\cite{pt} and slowing-down correlated weight matrices
can be computed in selected energy groups using the {\sl Ribon extended} approach.\cite{nse2004} In this case,
an elastic slowing-down model is used and a mutual self-shielding model is
available.
\end{enumerate}
\item The resonant fluxes are computed for each band of the probability tables
using a subgroup method if \moc{SUBG}, \moc{PT}, \moc{PTMC} or \moc{PTSL} keyword is set in module \moc{LIB:};
\item The resonance spectrum expansion (RSE) method is used if \moc{RSE} keyword is set in module \moc{LIB:};
\item The flux can be solved using collision probabilities, or using {\sl any}
flux solution technique for which a tracking module is available;
\item The resonant isotopes are computed one-a-time, starting from the isotopes
with the lower values of index \dusa{inrs}, as defined in \Sect{LIBData}; If
many isotopes have the same value of \dusa{inrs}, the isotope with the greatest
number of resonant nuclides is self-shielded first. One or many outer iterations
can be performed;
\item The distributed self-shielded effect is automatically taken into account
if different mixture indices are assigned to different regions inside the
resonant part of the cell. The rim effect can be computed by dividing the fuel
into "onion rings" and by assigning different mixture indices to them. 
\item A SPH (superhomog\'en\'eisation) equivalence is performed to correct the
self-shielded cross sections from the non-linear effects related to the
heterogeneity of the geometry.
\end{itemize}

\vskip 0.2cm

The general format of the data for this module is:

\begin{DataStructure}{Structure \dstr{USS:}}
\dusa{MICLIB} \moc{:=} \moc{USS:} \dusa{MICLIB\_SG} $[$ \dusa{MICLIB} $]$
\dusa{TRKNAM} $[$ \dusa{TRKFIL} $]$ \moc{::} \dstr{descuss}
\end{DataStructure}

\noindent
where

\begin{ListeDeDescription}{mmmmmmmm}

\item[\dusa{MICLIB}] {\tt character*12} name of the \dds{microlib} that will
contain the microscopic and macroscopic cross sections updated by the
self-shielding module. If
\dusa{MICLIB} appears on both LHS and RHS, it is updated; otherwise,
\dusa{MICLIB} is created.

\item[\dusa{MICLIB\_SG}] {\tt character*12} name of the \dds{microlib} builded
by module \moc{LIB:} and containing probability table information (the keyword \moc{SUBG} {\sl must} be
used in module {\tt LIB:}).

\item[\dusa{TRKNAM}] {\tt character*12} name of the required \dds{tracking}
data structure.

\item[\dusa{TRKFIL}] {\tt character*12} name of the sequential binary tracking
file used to store the tracks lengths. This file is given if and only if it was
required in the previous tracking module call (see \Sect{TRKData}).

\item[\dstr{descuss}] structure describing the self-shielding options.

\end{ListeDeDescription}

Each time the \moc{USS:} module is called, a sub-directory is updated in the
\dds{microlib} data structure to hold the last values defined in the
\dstr{descuss} structure. The next time this module is called,
these values will be used as floating defaults.

\subsubsection{Data input for module {\tt USS:}}\label{sect:descuss}

\begin{DataStructure}{Structure \dstr{descuss}}
$[$ \moc{EDIT} \dusa{iprint} $]$ \\
$[$ \moc{GRMIN}  \dusa{lgrmin} $]~~[$ \moc{GRMAX}  \dusa{lgrmax} $]$~~
$[$ \moc{PASS} \dusa{ipass} $]~~[$ \moc{NOCO} $]~~[$ \moc{NOSP} $]$~~$[$ $\{$ \moc{TRAN} $|$
\moc{NOTR} $\}$ $]$ \\ 
$[$ $\{$ \moc{PIJ} $|$ \moc{ARM} $\}$ $]$ \\
$[$ \moc{MAXST} \dusa{imax} $]~[$ \moc{FLAT} $]$ \\
$[$ \moc{CALC} \\
~~~~$[[$ \moc{REGI} \dusa{suffix} $[[$ \dusa{isot} $\{$ \moc{ALL} $|$
(\dusa{imix}(i),i=1,\dusa{nmix}) $\}$ $]]$ \\
~~~~$]]$ \\
\moc{ENDC} $]$ \\
{\tt ;}
\end{DataStructure}

\noindent where

\begin{ListeDeDescription}{mmmmmmmm}

\item[\moc{EDIT}] keyword used to modify the print level \dusa{iprint}.

\item[\dusa{iprint}] index used to control the printing of this module. The
amount of output produced by this tracking module will vary substantially
depending on the print level specified. 

\item[\moc{GRMIN}] keyword to specify the minimum group number considered
during the self-shielding process.

\item[\dusa{lgrmin}] first group number considered during the
self-shielding process. By default, \dusa{lgrmin} is set to the first group
number containing self-shielding data in the library.

\item[\moc{GRMAX}]  keyword to specify the maximum group number considered
during the self-shielding process.

\item[\dusa{lgrmax}] last group number considered during the self-shielding
process. By default, \dusa{lgrmax} is set is set to the last group
number containing self-shielding data in the library.

\item[\moc{PASS}]  keyword to specify the number of outer iterations during
the self-shielding process.

\item[\dusa{ipass}] the number of iterations. The default is
\dusa{ipass} $=2$ if \dusa{MICLIB} is created.

\item[\moc{NOCO}]  keyword to ignore the directives set by {\tt LIB} concerning
the mutual resonance shielding model. This keyword has the effect to replace the
mutual resonance shielding model in the subgroup projection method (SPM) by a full
correlation approximation similar
to the technique used in the ECCO code. This keyword can be used to avoid the message
\begin{verbatim}
USSIST: UNABLE TO FIND CORRELATED ISOTOPE ************.
\end{verbatim}
\noindent that appears with the SPM if the correlated weights matrices are missing in
the microlib.

\item[\moc{NOSP}] keyword to deactivate the SPH equivalence scheme which
modifies the self-shielded averaged neutron fluxes in
heterogeneous geometries. The default option is to perform SPH equivalence.

\item[\moc{TRAN}] keyword to activate the transport correction option for
self-shielding calculations (see \moc{CTRA} in \Sectand{MACData}{LIBData}). This
is the default option.

\item[\moc{NOTR}] keyword to deactivate the transport correction option for
self-shielding calculations (see \moc{CTRA} in \Sectand{MACData}{LIBData}).

\item[\moc{PIJ}] keyword to specify the use of complete collision
probabilities in the subgroup and SPH equivalence calculations of {\tt USS:}.
This is the default option for \moc{EXCELT:} and \moc{SYBILT:} trackings.
This option is not available for \moc{MCCGT:} trackings.

\item[\moc{ARM}] keyword to specify the use of iterative flux techniques
in the subgroup and SPH equivalence calculations of {\tt USS:}.
This is the default option for \moc{MCCGT:} trackings.

\item[\moc{MAXST}] keyword to set the maximum number of fixed point iterations
for the ST scattering source convergence.

\item[\dusa{imax}] the maximum number of ST iterations. The default is
\dusa{imax} $=50$ ($=20$ with the RSE method). A non-iterative response matrix approach is available with
the subgroup projection method (SPM) by setting \dusa{imax} $=0$.

\item[\moc{FLAT}] keyword to force the flat-flux initialization of subgroup fluxes if \dusa{MICLIB}
is open in modification mode.

\item[\moc{CALC}] keyword to activate the simplified self-shielding
approximation in which a single self-shielded isotope is shared by many
resonant mixtures.

\item[\moc{REGI}] keyword to specify a set of isotopes and mixtures that
will be self-shielded together. All the self-shielded isotopes in this group
will share the same 4--digit suffix.

\item[\dusa{suffix}] {\tt character*4} suffix for the isotope names in this
group

\item[\dusa{isot}] {\tt character*8} alias name of a self-shielded isotope in this
group

\item[\moc{ALL}] keyword to specify that a unique self-shielded isotope will be
made for the complete domain

\item[\dusa{imix}] list of mixture indices that will share the same self-shielded
isotope

\item[\dusa{nmix}] number of mixtures that will share the same self-shielded
isotope

\item[\moc{ENDC}] end of \moc{CALC} data keyword

\end{ListeDeDescription}

\vskip 0.15cm

Here is an example of the data structure corresponding to a production case where
only $^{238}$U is assumed to show distributed self-shielding effects:

\begin{verbatim}
LIBRARY2 := USS: LIBRARY TRACK ::
     CALC REGI W1 PU239 ALL
          REGI W1 PU241 ALL
          REGI W1 PU240 ALL
          REGI W1 PU242 ALL
          REGI W1 U235 ALL
          REGI W1 U236 ALL
          REGI W1 PU238 ALL
          REGI W1 U234 ALL
          REGI W1 AM241 ALL
          REGI W1 NP237 ALL
          REGI W1 ZRNAT ALL
          REGI W1 U238 <<COMB0101>> <<COMB0201>> <<COMB0301>>
                       <<COMB0401>> <<COMB0501>>
          REGI W2 U238 <<COMB0102>> <<COMB0202>> <<COMB0302>>
                       <<COMB0402>> <<COMB0502>>
          REGI W3 U238 <<COMB0103>> <<COMB0203>> <<COMB0303>>
                       <<COMB0403>> <<COMB0503>>
          REGI W4 U238 <<COMB0104>> <<COMB0204>> <<COMB0304>>
                       <<COMB0404>> <<COMB0504>>
          REGI W5 U238 <<COMB0105>> <<COMB0205>> <<COMB0305>>
                       <<COMB0405>> <<COMB0505>>
          REGI W6 U238 <<COMB0106>> <<COMB0206>> <<COMB0306>>
                       <<COMB0406>> <<COMB0506>>
     ENDC ;
\end{verbatim}

\vskip 0.15cm

In this case, $^{238}$U is self-shielded within six distributed regions (labeled
{\tt W1} to {\tt W6}) and each of these regions are merging volumes belonging
to five different fuel rods. The mixture indices of the 30 resonant volumes belonging
to the fuel are CLE-2000 variables labeled {\tt <<COMB0101>>} to {\tt <<COMB0506>>}.

\eject

