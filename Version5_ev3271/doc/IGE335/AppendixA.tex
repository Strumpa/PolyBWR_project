\section{THE GAN GENERALIZED DRIVER}

A scientific application can be built around the GAN generalized driver by
linking it with application-dependent modules. Such a
scientific application will share the following specifications:

\begin{enumerate}

\item  The GAN generalized driver can handle a custom data type called a {\sl LCM object}
and implemented as an {\sl associative table} or {\sl heterogeneous list}. A associative
table is a data structure similar to the example shown in \Fig{LkList}. An heterogeneous list
is an alternative structure where the component are identified by integer values instead of names. Each
data type mapped to a LCM object is dynamically
allocated using the computer's memory management algorithm and is accessed with
a pointer. LCM objects are the {\sl only} memory-resident data type used to
transfer information between modules. However, interface files can also be used to transfer information
between modules in cases where we want to reduce the memory resource
requirements. A LCM object can therefore be declared as {\tt LINKED\_LIST} to make it
memory-resident or as {\tt XSM\_FILE} to make it persistent. Sequential files (either
binary or ASCII) can also be used.

\begin{figure}[h!]
\begin{center} 
\epsfxsize=7cm \centerline{ \epsffile{GLkList.eps}}
\parbox{14.0cm}{\caption{An example of an associative table.}\label{fig:LkList}}   
\end{center}  
\end{figure}

\item  Building a scientific application requires the definition of the LCM
objects and interface files and the programming of application-dependent modules
to manage these LCM objects.

\item  A driver was written to support the LCM objects and to read
macro-language instructions. The modules are callable from this driver, but  the
possibility of having ``embedded modules", i.e.  modules called directly from a
subroutine written in any of these four languages has also been introduced.

\item  Utility modules are available to backup the LCM object on an XSM
file and to permit code restart.

\end{enumerate}

The modules must be declared in the calling procedure using directives of the form:

\begin{DataStructure}{Structure \dstr{descmodule}}
\moc{MODULE} $[[$ \dusa{name} $]]$ \moc{;} \\
\end{DataStructure}
\noindent with

\begin{ListeDeDescription}{mmmmmmm}

\item[\dusa{name}] character*12 symbolic name of a module used in the procedure.

\end{ListeDeDescription}

\vskip 0.4cm
\goodbreak

The LCM objects or files must be declared in the calling procedure using directives of the form:

\begin{DataStructure}{Structure \dstr{descobject}}
$[[~\{$ \moc{LINKED\_LIST} $|$ \moc{XSM\_FILE} $|$ \moc{SEQ\_ASCII} $|$ \moc{SEQ\_BINARY} $|$ \moc{HDF5\_FILE} $\}~[[$ \dusa{name} $]]$ \moc{;} $]]$ \\
$[[~\{$ \moc{XSM\_FILE} $|$ \moc{SEQ\_ASCII} $|$ \moc{SEQ\_BINARY} $|$ \moc{HDF5\_FILE} $\}$ \dusa{name} \moc{::} \moc{FILE} \dusa{path} \moc{;} $]]$
\end{DataStructure}
\noindent with

\begin{ListeDeDescription}{mmmmmmm}

\item[\dusa{name}] character*12 symbolic name of a LCM object (memory-resident or XSM file) or of a sequential file used in the procedure.

\item[\moc{FILE}] keyword used to set a file path.

\item[\dusa{path}] character*72 path name of a XSM or sequential file used in the procedure. The \moc{FILE} directive is useful to select or
create a file anywhere in the directory structure of the computer. It is also useful to tag a created file and avoid its deletion at end of
execution.

\end{ListeDeDescription}

\vskip 0.4cm

With this user interface, the input to a module named {\tt MOD:} with two
embedded modules {\tt EMB1:} and {\tt EMB2:} will always be of the form:

\vskip 0.4cm

\noindent \begin{tabular}{|c|}
\hline \\
{\sl (list of output LCM objects or files)} := {\tt MOD:} {\sl (list of input LCM objects or files){\tt~::~}(data input)}
\\ 
 {\tt~~~~~~~~~::: EMB1:} {\sl (data input for EMB1:)} {\tt~;} {\tt~~}
\\ 
 {\tt~~~~~~~~~::: EMB2:} {\sl (data input for EMB2:)} {\tt~;} {\tt~;}
\\ \\ \hline
\end{tabular}

\vskip 0.4cm
Note that the main use of embedded modules is to define gigogne geometries in module {\tt GEO:}.

\vskip 0.4cm

The following user's directives are always followed by an application built
around the generalized driver:

\begin{itemize}
\item An LCM object is resident in core memory if
declared as {\tt LINKED\_LIST} in the input data or mapped in a direct
access file (of {\sc xsm} type) if declared as {\tt XSM\_FILE} in the
input data. 
\item All the information declared as {\tt LINKED\_LIST} is
destroyed at the end of a run. All other information is located on files which
are kept at the end of the run, unless explicitely destroyed by a {\tt
DELETE:} command. 
\item Consider the following example in which the operator {\tt MOD1:} is called with the following command: 
\begin{verbatim}
DATA1 DATA2 := MOD1: DATA4 DATA2 ; 
\end{verbatim} 
Here, {\tt DATA1} is opened in {\bf create} mode because it appears only on the left-hand side (LHS) of the
command. {\tt DATA2} is opened in {\bf modification} mode because it appears on
both sides of the command. Finally, {\tt DATA4} is opened in {\bf read-only}
mode because it appears only on the right-hand side (RHS) of the command.
\item The calling sentence to an operator should always end by a ``;". A comment
can follow on the same input data record but a carriage return should be
performed before other significant data can be read by {\tt REDGET}.
\item The possibility of user-defined procedures is also offered. These procedures
give the user the possibility to ``program" an application using the capabilities of
the generalized driver and to use it as a new operator in the main data stream or in a
calling procedure.

\end{itemize}
