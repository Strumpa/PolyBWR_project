\subsection{Contents of \dir{power} data structure}\label{sect:power}

\vskip 0.2cm
A \dir{power} data structure is used to store the information related to
the powers and fluxes over the reactor core. This object has a signature
{\tt L\_POWER}; it is created using the \moc{FLPOW:} module. The reactor
fluxes and powers are recorded using several data formats.

\subsubsection{The state-vector content}\label{sect:powerstate}

\noindent
The dimensioning parameters $\mathcal{S}_i$, which are stored in the state
vector for this data structure, represent:

\begin{itemize}

\item The number of energy groups $N_{gr} = \mathcal{S}_1$

\item The total number of mesh-splitted volumes $N_{el} = \mathcal{S}_2$

\item The number of mesh-splitted volumes along x-axis $L_x = \mathcal{S}_3$

\item The number of mesh-splitted volumes along y-axis $L_y = \mathcal{S}_4$

\item The number of mesh-splitted volumes along z-axis $L_z = \mathcal{S}_5$

\item The number of reactor channels $N_{ch} = \mathcal{S}_6$

\item The number of bundles per channel $N_b = \mathcal{S}_7$

\end{itemize}

\subsubsection{The \dir{power} directory}\label{sect:powerdir}

\noindent
The following records will be found on the \dir{power} directory:

\begin{DescriptionEnregistrement}{Records in \dir{power} data structure}{7.0cm}
\CharEnr
 {SIGNATURE\blank{3}}{$*12$}
 {Signature of the \dir{power} data structure ($\mathsf{SIGNA}=${\tt L\_POWER\blank{5}}).}
\IntEnr
 {STATE-VECTOR}{$40$}
 {Vector describing the various parameters associated with this data structure $\mathcal{S}_i$}
\DbleEnr
 {PTOT\blank{8}}{1}{$MW$}
 {The total reactor power.}
\DbleEnr
 {VTOT\blank{8}}{1}{$cm^3$}
 {The total reactor volume.}
\DbleEnr
 {NORM\blank{8}}{1}{}
 {The flux normalization factor.}
\IntEnr
 {FLMIX\blank{7}}{$N_{ch}, N_b$}
 {Fuel mixture indices per fuel bundle.}
\RealEnr
 {FLUX\blank{8}}{$N_{el}, N_{gr}$}{cm$^{-2}$ s$^{-1}$}
 {The normalized fluxes over the whole reactor geometry,
  recorded per each mesh-splitted volume and per each energy
  group. The flux values over the virtual regions are set to 0.}
\RealEnr
 {VOLU-BUND\blank{3}}{$N_{ch}, N_b$}{cm$^{2}$}
 {The volume of each fuel bundle.}
\RealEnr
 {FLUX-BUND\blank{3}}{$N_{ch}, N_b, N_{gr}$}{cm$^{-2}$ s$^{-1}$}
 {The normalized average fluxes recorded per each fuel bundle and per
  each energy group.}
\RealEnr
 {FLUX-DISTR\blank{2}}{$L_x, L_y, L_z, N_{gr}$}{cm$^{-2}$ s$^{-1}$}
 {The normalized flux distribution over the whole reactor geometry,
  recorded per each X-Y-Z planes and per each energy group.}
\RealEnr
 {FLUX-RATIO\blank{2}}{$L_x, L_y, L_z, N_{gr}-1$}{}
 {The fluxes ratios with respect to the thermal energy-group fluxes.}
\RealEnr
 {POWER-BUND\blank{2}}{$N_{ch}, N_b$}{$kW$}
 {The bundle powers.}
\RealEnr
 {POWER-CHAN\blank{2}}{$N_{ch}$}{$kW$}
 {The channel powers.}
\RealEnr
 {POWER-DISTR\blank{1}}{$L_x, L_y, L_z$}{$W$}
 {The power distribution over the reactor core, recorded per each
  X-Y-Z planes. The power values over the non-fuel regions are set to 0.}
\RealEnr
 {PMAX-CHAN\blank{3}}{$1$}{$kW$}
 {The maximum channel power.}
\RealEnr
 {PMAX-BUND\blank{3}}{$1$}{$kW$}
 {The maximum bundle power.}
\RealEnr
 {FORM-CHAN\blank{3}}{$1$}{}
 {The radial power-form factor, defined as maximum-to-average
  channel power in core.}
\RealEnr
 {FORM-BUND\blank{3}}{$1$}{}
 {The overall power-form factor, defined as maximum-to-average
  bundle power in core.}
\RealEnr
 {K-EFFECTIVE\blank{1}}{$1$}{}
 {The effective multiplication factor, recovered from the
  \dir{flux} data structure.}
\end{DescriptionEnregistrement}

\vskip 0.2cm
\noindent
All stored fluxes are normalized either to the given total reactor power
or using the previously recorded normalization factor. The recorded
values of the maximum channel and bundle powers, the channel and
bundle power-form factors, and the effective multiplication factor, can
be used as power and criticity constraints for the optimization and fuel
management purposes.
\clearpage
